\chapter{Débuts}\cacheNumber{1}\needspace{5\baselineskip}\cacheName{\href{http://coord.info/GC6NNBD}{LABASTIDE-VILLEFRANCHE ~ La cité engloutie...} — \href{http://coord.info/GC6NNBD\Number{}631842785}{1}}\cacheData{{2016/08/25 laminak, Traditional Cache (2/1.5)}}\begin{cacheText}😀  Ma première cache trouvée ....merci\end{cacheText}

\cacheNumber{2}\needspace{5\baselineskip}\cacheName{\href{http://coord.info/GC5FHVW}{lavoir d'Escos} — \href{http://coord.info/GC5FHVW\Number{}631846478}{2}}\cacheData{{2016/08/25 bixlou, Traditional Cache (1.5/1.5)}}\begin{cacheText}Très beau lavoir, bel endroit pour une géocache. Merci\end{cacheText}

\cacheNumber{3}\needspace{5\baselineskip}\cacheName{\href{http://coord.info/GC28R4Y}{Chateau ducal de Gramont} — \href{http://coord.info/GC28R4Y\Number{}632581776}{3}}\cacheData{{2016/08/28 Peyo64, Traditional Cache (1.5/1.5)}}\begin{cacheText}J ai visité le château 2 jours avant de découvrir le geocaching!!!!!  j y suis retournée sous un soleil de plomb (36°) et mes recherches sont restées infructueuses.La réussite vient avec la persévérance ....ca y est je l ai trouvée  en ce dimanche nuageux!!! Merci\end{cacheText}

\cacheNumber{4}\needspace{5\baselineskip}\cacheName{\href{http://coord.info/GC5BKAD}{le pont d Auterrive} — \href{http://coord.info/GC5BKAD\Number{}632602654}{4}}\cacheData{{2016/08/28 bixlou, Traditional Cache (1.5/1.5)}}\begin{cacheText}Tres bonne cache: j ai mis un peu de temps!!!Pas facile avec la circulation de rester discrète .....Prévoir un stylo pour signer\end{cacheText}

\cacheNumber{5}\needspace{5\baselineskip}\cacheName{\href{http://coord.info/GC5EB4W}{L'ile de la Glère} — \href{http://coord.info/GC5EB4W\Number{}633183572}{5}}\cacheData{{2016/08/30 chantou 64, Traditional Cache (1.5/1.5)}}\begin{cacheText}😀par cette belle journée me voici parti me promener sur l île .Bel endroit et bonne cache merci\end{cacheText}

\cacheNumber{6}\needspace{5\baselineskip}\cacheName{\href{http://coord.info/GC64Y5V}{lavoir perdu} — \href{http://coord.info/GC64Y5V\Number{}633244578}{6}}\cacheData{{2016/08/30 lauki3940, Traditional Cache (1.5/1.5)}}\begin{cacheText}Trouvée les 2 caches très rapidement :la 1 ère n ayant pas de logbook j ai continué mes recherches et enfin....mplc l endroit est très joli malgré la route à proximité. N oubliez pas le stylo il n'y en a pas\end{cacheText}

\cacheNumber{7}\needspace{5\baselineskip}\cacheName{\href{http://coord.info/GC32TPD}{Sorde l'Abbaye} — \href{http://coord.info/GC32TPD\Number{}634453479}{7}}\cacheData{{2016/09/04 iPadius LXIV, Traditional Cache (1/1.5)}}\begin{cacheText}Sans trop de difficultés, très beau site à visiter.merci.\end{cacheText}

\cacheNumber{8}\needspace{5\baselineskip}\cacheName{\href{http://coord.info/GC3CG79}{Un lavoir pour nanalandes} — \href{http://coord.info/GC3CG79\Number{}634454379}{8}}\cacheData{{2016/09/04 lokateo64, Traditional Cache (1.5/1.5)}}\begin{cacheText}Très beau lavoir, cache sympa merci.\end{cacheText}

\cacheNumber{9}\needspace{5\baselineskip}\cacheName{\href{http://coord.info/GC2Q8GA}{Panorama de la Butte de Miremont} — \href{http://coord.info/GC2Q8GA\Number{}634750631}{9}}\cacheData{{2016/09/05 lokateo64, Traditional Cache (1.5/1.5)}}\begin{cacheText}Trouvée rapidement merci pour la cache: très beau panorama!!!!\end{cacheText}

\cacheNumber{10}\needspace{5\baselineskip}\cacheName{\href{http://coord.info/GC4XQWC}{Houeillès l'eglise} — \href{http://coord.info/GC4XQWC\Number{}656829019}{10}}\cacheData{{2016/12/25 kar@melos40, Traditional Cache (1.5/1.5)}}\begin{cacheText}Très belle eglise.....mplc\end{cacheText}

\cacheNumber{11}\needspace{5\baselineskip}\cacheName{\href{http://coord.info/GC5W5VF}{le lavoir d'orthevielle 2} — \href{http://coord.info/GC5W5VF\Number{}665673391}{11}}\cacheData{{2017/02/19 mizaga, Traditional Cache (1.5/1.5)}}\begin{cacheText}Après avoir cherché un bon moment je l ai enfin trouvée.....mplc j ai découvert ce charmant lavoir que je ne connaissais pas\end{cacheText}

\cacheNumber{12}\needspace{5\baselineskip}\cacheName{\href{http://coord.info/GC1FPRA}{Petit port des Landes} — \href{http://coord.info/GC1FPRA\Number{}665684877}{12}}\cacheData{{2017/02/19 gadiax, Traditional Cache (1.5/1.5)}}\begin{cacheText}Cette cache a été trouvée en 2 temps 3 mouvements: plusieurs objets, je n ai rien pris. Mplc endroit superbe que je ne connaissais pas\end{cacheText}

\cacheNumber{13}\needspace{5\baselineskip}\cacheName{\href{http://coord.info/GC4FWXC}{Parcours de découverte \Number{} 01} — \href{http://coord.info/GC4FWXC\Number{}666848685}{13}}\cacheData{{2017/02/24 MaryeLoup, Traditional Cache (1.5/1.5)}}\begin{cacheText}Jolie promenade.... j ai cherché j ai trouvé MPLC\end{cacheText}

\cacheNumber{14}\needspace{5\baselineskip}\cacheName{\href{http://coord.info/GC4FWXY}{Parcours de découverte \Number{} 03} — \href{http://coord.info/GC4FWXY\Number{}666848708}{14}}\cacheData{{2017/02/24 MaryeLoup, Traditional Cache (1.5/1.5)}}\begin{cacheText}Une autre de dénichée sur ce joli parcours MPLC\end{cacheText}

\cacheNumber{15}\needspace{5\baselineskip}\cacheName{\href{http://coord.info/GC47ZX7}{la chapelle de sunarthe} — \href{http://coord.info/GC47ZX7\Number{}667790430}{15}}\cacheData{{2017/03/02 chantou 64, Traditional Cache (1.5/1.5)}}\begin{cacheText}Mplc: je suis de la région mais je n ai jamais pris le temps de m'arrêter. Vraiment très joli. Cache facile..... les photos aident bien. Le logbook est plein.... il faut penser à le changer\end{cacheText}

\cacheNumber{16}\needspace{5\baselineskip}\cacheName{\href{http://coord.info/GC28R65}{L'abbaye de Sauvelade} — \href{http://coord.info/GC28R65\Number{}668071602}{16}}\cacheData{{2017/03/04 Peyo64, Traditional Cache (1.5/1.5)}}\begin{cacheText}Rdv 14h30.....il est 14h. Quelle cache à proximité ? L abbaye de Sauvelade.....en route!!!La cache est dégotée en 1 minute elle est à la vue de tous. J essais de La camoufler. Endroit magnifique perdu dans la campagne. MPLC\end{cacheText}

\cacheNumber{17}\needspace{5\baselineskip}\cacheName{\href{http://coord.info/GC69ZBD}{La Fontaine - Lahonce} — \href{http://coord.info/GC69ZBD\Number{}669202858}{17}}\cacheData{{2017/03/10 gilles64, Traditional Cache (1.5/1.5)}}\begin{cacheText}Mplc merveilleux endroit logbook bien protégé\end{cacheText}

\cacheNumber{18}\needspace{5\baselineskip}\cacheName{\href{http://coord.info/GC2J381}{Le lavoir de Audaux} — \href{http://coord.info/GC2J381\Number{}669376613}{18}}\cacheData{{2017/03/11 lokateo64, Traditional Cache (1.5/1.5)}}\begin{cacheText}Encore un beau lavoir....mplc\end{cacheText}

\cacheNumber{19}\needspace{5\baselineskip}\cacheName{\href{http://coord.info/GC2J37X}{Le lavoir de Meritein} — \href{http://coord.info/GC2J37X\Number{}669382143}{19}}\cacheData{{2017/03/11 lokateo64, Traditional Cache (1.5/1.5)}}\begin{cacheText}Bel endroit que je n aurais jamais connu sans geocaching mplc\end{cacheText}

\cacheNumber{20}\needspace{5\baselineskip}\cacheName{\href{http://coord.info/GC71BP4}{Dönerstag 2017 : un kebab = un souvenir} — \href{http://coord.info/GC71BP4\Number{}677141406}{20}}\cacheData{{2017/04/13 DorisBear, Event Cache (1/1)}}\begin{cacheText}Mon premier Évent.....formidable.Merci à Dorisbear pour l'organisation de la soirée et merci aux participants pour leur accueil chaleureux et leurs nombreuses informations. La cache nocturne à travers les bois était super . Vivement le prochain rassemblement\end{cacheText}

\cacheNumber{21}\needspace{5\baselineskip}\cacheName{\href{http://coord.info/GC4R8VY}{Hiriburu : Les Frontons Affrontés} — \href{http://coord.info/GC4R8VY\Number{}677354092}{21}}\cacheData{{2017/04/14 Gamboy, Traditional Cache (1.5/1.5)}}\begin{cacheText}Celle la Je l ai trouvée sans trop de difficulté mpl\end{cacheText}

\cacheNumber{22}\needspace{5\baselineskip}\cacheName{\href{http://coord.info/GC4YQ71}{Hiriburu : Tu lis pas... Tu ris !} — \href{http://coord.info/GC4YQ71\Number{}677355930}{22}}\cacheData{{2017/04/14 Gamboy, Traditional Cache (1.5/1.5)}}\begin{cacheText}Cache débusquée au lendemain de notre rencontre. Trop contente. En cherchant et réfléchissant à l' indice j'ai trouvé.mplc les gamboy\end{cacheText}

\cacheNumber{23}\needspace{5\baselineskip}\cacheName{\href{http://coord.info/GC53QJX}{Alphabet du 64 - C pour Charnègues} — \href{http://coord.info/GC53QJX\Number{}677364206}{23}}\cacheData{{2017/04/14 Charnègues, Traditional Cache (1.5/1.5)}}\begin{cacheText}Direction Bayonne....premier arrêt pour chercher la cache . Trouvée sans difficulté particulière. Merci\end{cacheText}

\cacheNumber{24}\needspace{5\baselineskip}\cacheName{\href{http://coord.info/GC546C9}{Repos à Urt aux bords de l'Adour} — \href{http://coord.info/GC546C9\Number{}677359181}{24}}\cacheData{{2017/04/14 MaryeLoup, Traditional Cache (1/1)}}\begin{cacheText}Direction Bayonne pour aller chercher les internes.....pourquoi ne pas faire quelques caches? Arrêt le long de l Adour pour chercher la cache débusquée en 2 temps 3 mouvements.Quel plaisir de découvrir les noms des géocacheurs qui m ont précédé et sur lesquels  je peux mettre un visage grâce à l évent d hier. Merci ,l endroit est très sympa.\end{cacheText}

\cacheNumber{25}\needspace{5\baselineskip}\cacheName{\href{http://coord.info/GC4BCPX}{le chateau} — \href{http://coord.info/GC4BCPX\Number{}677825477}{25}}\cacheData{{2017/04/15 chantou 64, Traditional Cache (1/1.5)}}\begin{cacheText}Petit détour par la principauté de Laas        Pour augmenter mon petit palmarès. Cache trouvée sans difficulté mplc. En route pour de nouvelles caches.... je connais le château (et son parc qui accueille des concerts )\end{cacheText}

\cacheNumber{26}\needspace{5\baselineskip}\cacheName{\href{http://coord.info/GC2J335}{Le lavoir de Lay Lamidou} — \href{http://coord.info/GC2J335\Number{}677836775}{26}}\cacheData{{2017/04/15 lokateo64, Traditional Cache (1.5/1.5)}}\begin{cacheText}Superbe endroit paisiblement le lavoir est ancien, décrépit...dommage !!!D'un coup d'œil la cache est repérée....mais il me manque quelques centimètres...Un peu de contorsions.... ça y est je l'ai!!!!\end{cacheText}

\cacheNumber{27}\needspace{5\baselineskip}\cacheName{\href{http://coord.info/GC2J333}{Le Camp de GURS} — \href{http://coord.info/GC2J333\Number{}677844353}{27}}\cacheData{{2017/04/15 lokateo64, Traditional Cache (1.5/1.5)}}\begin{cacheText}Un endroit qui se prête au recueillement et au devoir de mémoire Elle m'a donné du fil à retordre ( boussole iPhone pas très efficace ou moi qui était troublée par ce lieu?) mais j ´ai enfin mis la main dessus. Il y a beaucoup de souches dans la zone!!!!!Mplc\end{cacheText}

\cacheNumber{28}\needspace{5\baselineskip}\cacheName{\href{http://coord.info/GC5N47H}{Sacrés pêcheurs} — \href{http://coord.info/GC5N47H\Number{}677846057}{28}}\cacheData{{2017/04/15 chantou 64, Traditional Cache (1/1.5)}}\begin{cacheText}Avant dernière cache avant le retour: beaucoup de passage. Pas très à l'aise pour chercher mais je ne tarde pas pour la débusquer. L'intérieur de la boite est humide. Sculpture grandiose mplc\end{cacheText}

\cacheNumber{29}\needspace{5\baselineskip}\cacheName{\href{http://coord.info/GC2J33A}{Le lavoir de Jasses} — \href{http://coord.info/GC2J33A\Number{}677831425}{29}}\cacheData{{2017/04/15 lokateo64, Traditional Cache (1.5/1.5)}}\begin{cacheText}Encore un très joli lavoir ,rénové ,au sein d 'un beau village béarnais. La cache ,repérée depuis la route,est toujours en place.N n'hésitez pas à continuer le chemin vous apercevrez un superbe moulin. Mplc\end{cacheText}

\cacheNumber{30}\needspace{5\baselineskip}\cacheName{\href{http://coord.info/GC72TJ8}{Le Cri du Kebab [NC]} — \href{http://coord.info/GC72TJ8\Number{}677904319}{30}}\cacheData{{2017/04/15 DorisBear, Unknown Cache (2/2)}}\begin{cacheText}{FTF} collectif

Découvert avec la team Kebab du Dönerstag 2017 Un kebab=Un souvenir du 13/04//17\end{cacheText}

\cacheNumber{31}\needspace{5\baselineskip}\cacheName{\href{http://coord.info/GC6KBHB}{la chapelle de soyarza} — \href{http://coord.info/GC6KBHB\Number{}678292105}{31}}\cacheData{{2017/04/16 chantou 64, Traditional Cache (1/2)}}\begin{cacheText}Après une longue marche de 2 km nous arrivons enfin à La Chapelle qui accueille multiple pèlerins et promeneurs du dimanche. Le monde est petit ...je connais des randonneurs !!! Après avoir discuté un long moment...je peux enfin mener mes recherches. La cache est vite localisée, signée et remise en place au nez et à la barbe des moldus. Je peux admirer ce magnifique point de vu . Mplc\end{cacheText}

\cacheNumber{32}\needspace{5\baselineskip}\cacheName{\href{http://coord.info/GC65EBH}{Chapelle St Nicolas d'Harambels} — \href{http://coord.info/GC65EBH\Number{}678294093}{32}}\cacheData{{2017/04/16 gilles64, Traditional Cache (1.5/1.5)}}\begin{cacheText}Après avoir fait le tour de La chapelle et attendu que le groupe de moldus s'éloigne, je débusque vite fait la cache grâce à l'indice. Mplc\end{cacheText}

\cacheNumber{33}\needspace{5\baselineskip}\cacheName{\href{http://coord.info/GC6JRYX}{Le royaume de petit fébus 1} — \href{http://coord.info/GC6JRYX\Number{}678302622}{33}}\cacheData{{2017/04/16 yvain64, Traditional Cache (1.5/2.5)}}\begin{cacheText}Pâques : certains cherchent les œufs,mon fils et moi ce sont les caches !!! Nous n avons fait que la moitié du circuit mais nous nous sommes régalés : endroit très agréable et caches très originales. Merci pour ce petit trésor.\end{cacheText}

\cacheNumber{34}\needspace{5\baselineskip}\cacheName{\href{http://coord.info/GC6JTQK}{Le royaume de petit fébus 2} — \href{http://coord.info/GC6JTQK\Number{}678305013}{34}}\cacheData{{2017/04/16 yvain64, Traditional Cache (1.5/2.5)}}\begin{cacheText}Cette cache est vite trouvée : Merci l'indice.Mplc\end{cacheText}

\cacheNumber{35}\needspace{5\baselineskip}\cacheName{\href{http://coord.info/GC6JWRN}{Le royaume de petit fébus 3} — \href{http://coord.info/GC6JWRN\Number{}678306862}{35}}\cacheData{{2017/04/16 yvain64, Traditional Cache (1.5/2.5)}}\begin{cacheText}Celle là nous a donné du fil à retordre : GPS en difficulté mais enfin on la trouve... en route pour la suite\end{cacheText}

\cacheNumber{36}\needspace{5\baselineskip}\cacheName{\href{http://coord.info/GC6JWT1}{Le royaume de petit fébus 4} — \href{http://coord.info/GC6JWT1\Number{}678307810}{36}}\cacheData{{2017/04/16 yvain64, Traditional Cache (1.5/2.5)}}\begin{cacheText}Merci Tarzan....\end{cacheText}

\cacheNumber{37}\needspace{5\baselineskip}\cacheName{\href{http://coord.info/GC6JWTM}{Le royaume de petit fébus 5} — \href{http://coord.info/GC6JWTM\Number{}678350308}{37}}\cacheData{{2017/04/16 yvain64, Traditional Cache (1.5/2.5)}}\begin{cacheText}Trouvé sans difficulté, l'indice est parlant\end{cacheText}

\cacheNumber{38}\needspace{5\baselineskip}\cacheName{\href{http://coord.info/GC6JWTZ}{Le royaume de petit fébus 6} — \href{http://coord.info/GC6JWTZ\Number{}678352435}{38}}\cacheData{{2017/04/16 yvain64, Traditional Cache (1.5/2.5)}}\begin{cacheText}Elle est trop mignonne....\end{cacheText}

\cacheNumber{39}\needspace{5\baselineskip}\cacheName{\href{http://coord.info/GC6JWVG}{Le royaume de petit fébus 7} — \href{http://coord.info/GC6JWVG\Number{}678353029}{39}}\cacheData{{2017/04/16 yvain64, Traditional Cache (1.5/2.5)}}\begin{cacheText}Encore une belle surprise\end{cacheText}

\cacheNumber{40}\needspace{5\baselineskip}\cacheName{\href{http://coord.info/GC2Q8JH}{Lavoir,Puit et Fontaine GARAY} — \href{http://coord.info/GC2Q8JH\Number{}678486059}{40}}\cacheData{{2017/04/17 lokateo64, Traditional Cache (1.5/1.5)}}\begin{cacheText}Superbe fontaine à découvrir. La cache est vite trouvée.Mplc je ne serais jamais venu s'il n'y avait pas eu le geocaching\end{cacheText}

\cacheNumber{41}\needspace{5\baselineskip}\cacheName{\href{http://coord.info/GC2Q8MZ}{Le Lavoir de Bergouey} — \href{http://coord.info/GC2Q8MZ\Number{}678486563}{41}}\cacheData{{2017/04/17 lokateo64, Traditional Cache (1.5/1.5)}}\begin{cacheText}Joli petit lavoir depuis lequel on entend le grognement des cochons noirs. La cache est vite trouvée grâce à l'indice. Merci\end{cacheText}

\cacheNumber{42}\needspace{5\baselineskip}\cacheName{\href{http://coord.info/GC6QPEW}{Fronton de Sorde} — \href{http://coord.info/GC6QPEW\Number{}679234634}{42}}\cacheData{{2017/04/19 Ju\And{}Marion, Traditional Cache (1.5/1.5)}}\begin{cacheText}La cache trouvée sans difficulté : l'indice aide bien. Mplc\end{cacheText}

\cacheNumber{43}\needspace{5\baselineskip}\cacheName{\href{http://coord.info/GC6624Q}{Un pont plus loin / Just another bridge} — \href{http://coord.info/GC6624Q\Number{}679235416}{43}}\cacheData{{2017/04/19 DorisBear, Traditional Cache (2/2)}}\begin{cacheText}Nous l'avons trouvé au bout d'un certain temps. Cache très originale : lors du Dönerstag les géocacheurs ´´avertis'´m'avaient dit que Dorisbear plaçait des caches sophistiquées: je confirme !!!! Mplc\end{cacheText}

\cacheNumber{44}\needspace{5\baselineskip}\cacheName{\href{http://coord.info/GC6624Z}{Le Pont de la Coudette} — \href{http://coord.info/GC6624Z\Number{}679235740}{44}}\cacheData{{2017/04/19 DorisBear, Traditional Cache (2.5/2)}}\begin{cacheText}Sur la lancée, on s'arrête pour une nouvelle cache. Elle nous a donné du fil à retordre mais on a lis la main dessus. Très originale....mplc\end{cacheText}

\cacheNumber{45}\needspace{5\baselineskip}\cacheName{\href{http://coord.info/GC6JC4X}{E.B.5. Oktoberfest} — \href{http://coord.info/GC6JC4X\Number{}679244068}{45}}\cacheData{{2017/04/19 DorisBear, Traditional Cache (2/2.5)}}\begin{cacheText}On arrive en fin de matinée et on commence dans le désordre les caches mais on prend soin de noter les indices. Celle la est vite trouvée mplc\end{cacheText}

\cacheNumber{46}\needspace{5\baselineskip}\cacheName{\href{http://coord.info/GC6JC2D}{E.B.4. La cuisine / The food} — \href{http://coord.info/GC6JC2D\Number{}679244346}{46}}\cacheData{{2017/04/19 DorisBear, Traditional Cache (2/2)}}\begin{cacheText}Trouvée mais mon stylo ne marche plus.....je reviendrai signer le logbook. Indice noté et photo faite en guise de preuve. Mplc\end{cacheText}

\cacheNumber{47}\needspace{5\baselineskip}\cacheName{\href{http://coord.info/GC6JC0P}{E.B.3. La bière / The beer} — \href{http://coord.info/GC6JC0P\Number{}679245190}{47}}\cacheData{{2017/04/19 DorisBear, Traditional Cache (1.5/2)}}\begin{cacheText}Très facile mais pas de stylo!!!!! Je relève l' indice et prend la photo. Mplc\end{cacheText}

\cacheNumber{48}\needspace{5\baselineskip}\cacheName{\href{http://coord.info/GC5BKDN}{Le pain de sucre} — \href{http://coord.info/GC5BKDN\Number{}679811457}{48}}\cacheData{{2017/04/21 bixlou, Traditional Cache (1.5/2.5)}}\begin{cacheText}Très belle promenade en ce bel après-midi ensoleillé. Les daims sont très mignons... Cache trouvée sans problème mplc\end{cacheText}

\cacheNumber{49}\needspace{5\baselineskip}\cacheName{\href{http://coord.info/GC2WPZ1}{L'église de l'Hopital d'Orion} — \href{http://coord.info/GC2WPZ1\Number{}679812366}{49}}\cacheData{{2017/04/21 lokateo64, Traditional Cache (1.5/1.5)}}\begin{cacheText}La cache est en place mais un peu humide. Jolie halte sur le chemin de Saint Jacques de Compostelle mplc\end{cacheText}

\cacheNumber{50}\needspace{5\baselineskip}\cacheName{\href{http://coord.info/GC5K04Z}{Heugas : le lavoir cloturé / the fenced wash house} — \href{http://coord.info/GC5K04Z\Number{}679952385}{50}}\cacheData{{2017/04/22 dorisbear, Traditional Cache (2/1.5)}}\begin{cacheText}En route pour le CITO, je m'arrête faire cette petite cache près du lavoir ( je pense aussi que tout Les lavoirs devraient avoir une cache,ce sont des lieux extraordinaires).Assez vite trouvée merci l indice !!!!Mplc\end{cacheText}

\cacheNumber{51}\needspace{5\baselineskip}\cacheName{\href{http://coord.info/GC5D0GB}{[BSD] \Number{}051} — \href{http://coord.info/GC5D0GB\Number{}679967463}{51}}\cacheData{{2017/04/22 GéoLandesTour, Traditional Cache (1.5/1.5)}}\begin{cacheText}Trouvée malgré un GPS qui indique n' importe quoi!!!!! Log mouillé je ne peux pas signer. La photo sert de preuve\end{cacheText}

\cacheNumber{52}\needspace{5\baselineskip}\cacheName{\href{http://coord.info/GC5D0FH}{[BSD] \Number{}050} — \href{http://coord.info/GC5D0FH\Number{}679970706}{52}}\cacheData{{2017/04/22 GéoLandesTour, Traditional Cache (1.5/1.5)}}\begin{cacheText}Mplc circuit très agréable\end{cacheText}

\cacheNumber{53}\needspace{5\baselineskip}\cacheName{\href{http://coord.info/GC72DTP}{CITO 2017 Grand nettoyage de printemps} — \href{http://coord.info/GC72DTP\Number{}680171651}{53}}\cacheData{{2017/04/22 DorisBear \And{} mizaga, Cache In Trash Out Event (1/1)}}\begin{cacheText}Excellente après midi passée avec une équipe de joyeux drilles. Pas mal de déchets ramassés et beaucoup de caches trouvées. Après le traditionnel \Quoted{chou vert} direction l'Auberge In où nous avons partagé un bon repas en discutant geocaching. Merci Dorisbear et Mizaga\end{cacheText}

\cacheNumber{54}\needspace{5\baselineskip}\cacheName{\href{http://coord.info/GC72NNH}{1. Étangs de Gravières - Plus une goutte} — \href{http://coord.info/GC72NNH\Number{}680172642}{54}}\cacheData{{2017/04/22 DorisBear, Traditional Cache (1.5/1.5)}}\begin{cacheText}Trouvée avec La CITO team. Mplc\end{cacheText}

\cacheNumber{55}\needspace{5\baselineskip}\cacheName{\href{http://coord.info/GC72NP5}{2. Étangs de Gravières - Si taudis ...} — \href{http://coord.info/GC72NP5\Number{}680172850}{55}}\cacheData{{2017/04/22 DorisBear, Traditional Cache (2/1.5)}}\begin{cacheText}Trouvé avec la Cito team\end{cacheText}

\cacheNumber{56}\needspace{5\baselineskip}\cacheName{\href{http://coord.info/GC72NPE}{3. Étangs de Gravières - ... pas de lumière} — \href{http://coord.info/GC72NPE\Number{}680173072}{56}}\cacheData{{2017/04/22 DorisBear, Traditional Cache (2/1.5)}}\begin{cacheText}Trouvée avec la Cito team\end{cacheText}

\cacheNumber{57}\needspace{5\baselineskip}\cacheName{\href{http://coord.info/GC72NQ0}{4. Étangs de Gravières - Qui suis-je ?} — \href{http://coord.info/GC72NQ0\Number{}680173293}{57}}\cacheData{{2017/04/22 DorisBear, Traditional Cache (2/1.5)}}\begin{cacheText}Trouvée avec la Cito team\end{cacheText}

\cacheNumber{58}\needspace{5\baselineskip}\cacheName{\href{http://coord.info/GC72NRE}{5. Étangs de Gravières - home cinema} — \href{http://coord.info/GC72NRE\Number{}680173573}{58}}\cacheData{{2017/04/22 DorisBear, Traditional Cache (2/1.5)}}\begin{cacheText}Trouvée avec la Cito team\end{cacheText}

\cacheNumber{59}\needspace{5\baselineskip}\cacheName{\href{http://coord.info/GC72NVD}{6. Étangs de Gravières - Bonus - Coffre-fort} — \href{http://coord.info/GC72NVD\Number{}680174033}{59}}\cacheData{{2017/04/22 DorisBear, Unknown Cache (2/1.5)}}\begin{cacheText}Trouvée avec la Cito team\end{cacheText}

\cacheNumber{60}\needspace{5\baselineskip}\cacheName{\href{http://coord.info/GC5D0GT}{[BSD] \Number{}053} — \href{http://coord.info/GC5D0GT\Number{}680174973}{60}}\cacheData{{2017/04/22 GéoLandesTour, Traditional Cache (2.5/1.5)}}\begin{cacheText}Trouvée avec Isa Asia lors du Cito 2017 Grand Nettoyage de Printemps.\end{cacheText}

\cacheNumber{61}\needspace{5\baselineskip}\cacheName{\href{http://coord.info/GC5KQ2A}{Pont de chemin de fer / Railway bridge} — \href{http://coord.info/GC5KQ2A\Number{}683152908}{61}}\cacheData{{2017/05/03 dorisbear, Traditional Cache (1.5/1.5)}}\begin{cacheText}Afin d'obtenir le Big Blue Switch Souvenir me voilà parti sous la pluie pour débusquer une cache: celle du pont est vite trouvée . Mplc\end{cacheText}

\cacheNumber{62}\needspace{5\baselineskip}\cacheName{\href{http://coord.info/GC5KQ7E}{Bec du Gave : point de vue / viewpoint} — \href{http://coord.info/GC5KQ7E\Number{}683154428}{62}}\cacheData{{2017/05/03 dorisbear, Traditional Cache (1.5/1.5)}}\begin{cacheText}La tentation est trop forte: envie de trouver une autre cache!!!!!A peine descendu de voiture elle nous attend , attachée, à la vue de tous. Surprenant ...Le Bec du gave est superbe: à découvrir absolument. Mplc\end{cacheText}

\cacheNumber{63}\needspace{5\baselineskip}\cacheName{\href{http://coord.info/GC5J59G}{Autour de Quintaou} — \href{http://coord.info/GC5J59G\Number{}683726232}{63}}\cacheData{{2017/05/05 Gamboy, Traditional Cache (2/1.5)}}\begin{cacheText}Impossible de résister à l'appel de cette cache.L'indice m'a bien aidé... merci GAMBOY\end{cacheText}

\cacheNumber{64}\needspace{5\baselineskip}\cacheName{\href{http://coord.info/GC62D5J}{Le miracle de Bayonne} — \href{http://coord.info/GC62D5J\Number{}683739431}{64}}\cacheData{{2017/05/05 Gaingain, Traditional Cache (1.5/2)}}\begin{cacheText}Bel endroit, mplc\end{cacheText}

\cacheNumber{65}\needspace{5\baselineskip}\cacheName{\href{http://coord.info/GC5RQEN}{les points cardinaux} — \href{http://coord.info/GC5RQEN\Number{}684510966}{65}}\cacheData{{2017/05/08 sc65, Traditional Cache (1.5/1.5)}}\begin{cacheText}Venue voir les huitièmes de finale crabos Aviron Bayonnais/Stade Toulousain je m'échappe avant le début du match : cache sympathique ( l'indice aide bien) trouvée facilement.MPLC\end{cacheText}

\cacheNumber{66}\needspace{5\baselineskip}\cacheName{\href{http://coord.info/GC5JCE9}{LALOUBERE-Hippodrome 1} — \href{http://coord.info/GC5JCE9\Number{}684514949}{66}}\cacheData{{2017/05/08 65THIO05, Traditional Cache (1.5/1.5)}}\begin{cacheText}Quelques minutes avant le coup d'envoi du match Crabos Aviron Bayonnais/ Stade Toulousain .... j'ai le temps de dénicher cette belle boite pleine de petits objets un peu collant. MPLC\end{cacheText}

\cacheNumber{67}\needspace{5\baselineskip}\cacheName{\href{http://coord.info/GC3FZK6}{GR8\Number{}38} — \href{http://coord.info/GC3FZK6\Number{}684807305}{67}}\cacheData{{2017/05/09 Peyo64, Traditional Cache (2/1.5)}}\begin{cacheText}Je passais par là : je n'ai pas pu faire autrement que m'arrêter.... cache vite trouvée . Mplc\end{cacheText}

\cacheNumber{68}\needspace{5\baselineskip}\cacheName{\href{http://coord.info/GC6FHZF}{Miradour} — \href{http://coord.info/GC6FHZF\Number{}684808010}{68}}\cacheData{{2017/05/09 Gamboy, Traditional Cache (1.5/1.5)}}\begin{cacheText}Très beau panorama, l'indice aide bien. Mplc\end{cacheText}

\cacheNumber{69}\needspace{5\baselineskip}\cacheName{\href{http://coord.info/GC4HVJ5}{Alphabet du 64 - B pour Bayonne} — \href{http://coord.info/GC4HVJ5\Number{}684809183}{69}}\cacheData{{2017/05/09 Gaingain, Traditional Cache (1.5/1.5)}}\begin{cacheText}Cache vite trouvée grâce à la photo.Un objet voyageur est signalé dans la cache: il doit y avoir une erreur!!!! Comment peut il rentrer?Mplc\end{cacheText}

\cacheNumber{70}\needspace{5\baselineskip}\cacheName{\href{http://coord.info/GC36N8C}{Piste de la Nive 2) La Tour de Sault} — \href{http://coord.info/GC36N8C\Number{}684809754}{70}}\cacheData{{2017/05/09 Gamboy, Traditional Cache (1.5/1.5)}}\begin{cacheText}Jolie fontaine....cache vite trouvée , belle boite aux multiples trésors.Mplc\end{cacheText}

\cacheNumber{71}\needspace{5\baselineskip}\cacheName{\href{http://coord.info/GC5J514}{Au Marché de Quintaou} — \href{http://coord.info/GC5J514\Number{}684872952}{71}}\cacheData{{2017/05/09 Gamboy, Traditional Cache (1.5/1)}}\begin{cacheText}Je suis déjà venue et le léopard m'a donné du fil à retordre !!!!!J'ai finis par le dénicher mais le vendredi après midi la place est très fréquentée: impossible de loguer discrètement. Ce lundi soir la place est déserte ...je peux donc signer le logbook en toute tranquillité. Mplc\end{cacheText}

\cacheNumber{72}\needspace{5\baselineskip}\cacheName{\href{http://coord.info/GC6Q7PC}{\Number{}01 les passerelles de l’Aphatarena} — \href{http://coord.info/GC6Q7PC\Number{}685054750}{72}}\cacheData{{2017/05/10 matchouteam.com  cow040, Traditional Cache (2/2)}}\begin{cacheText}Petite journée de repos : l'envie de faire quelques caches se fait sentir !!!je dois être sur Bidache en fin d'après-midi...je choisis donc les caches du bois de Mixe. Magnifique endroit que je ne connaissais pas. La 1 ère cache me donne du fil à retordre : le portable et le GPS me mènent au même endroit !Rien.... j'élargis la zone et passe tout les arbres en revu...Rien!!!!3/4 d'heure que je cherche ...je renonce et en repartant je tombe nez à nez avec la cache ouffff . Très belle cache intégrée dans le paysage merci\end{cacheText}

\cacheNumber{73}\needspace{5\baselineskip}\cacheName{\href{http://coord.info/GC6Q7KJ}{\Number{}02 les passerelles de l’Aphatarena} — \href{http://coord.info/GC6Q7KJ\Number{}685055599}{73}}\cacheData{{2017/05/10 matchouteam.com  cow040, Traditional Cache (3/3)}}\begin{cacheText}Trouvée en deux temps trois mouvements grâce à l'indice. Très jolie cache mais le logbook est trempé et donc impossible de signer. La photo fera office de preuve. Merci\end{cacheText}

\cacheNumber{74}\needspace{5\baselineskip}\cacheName{\href{http://coord.info/GC6QC5V}{\Number{}05 les passerelles de l’Aphatarena} — \href{http://coord.info/GC6QC5V\Number{}685058281}{74}}\cacheData{{2017/05/10 matchouteam.com  cow040, Traditional Cache (2/2)}}\begin{cacheText}La passerelle et l' endroit sont charmants.La cache est très belle : du grand art. Je l'ai trouvée rapidement grâce à l'indice.... merci\end{cacheText}

\cacheNumber{75}\needspace{5\baselineskip}\cacheName{\href{http://coord.info/GC4FWY5}{Parcours de découverte \Number{} 04} — \href{http://coord.info/GC4FWY5\Number{}685652027}{75}}\cacheData{{2017/05/13 MaryeLoup, Traditional Cache (3/2)}}\begin{cacheText}Frissons garantis dans ce sombre tunnel. ..Cache très sympa vite dénichée. Mplc\end{cacheText}

\cacheNumber{76}\needspace{5\baselineskip}\cacheName{\href{http://coord.info/GC2NTMM}{Aire de Jeux des Arènes} — \href{http://coord.info/GC2NTMM\Number{}686011585}{76}}\cacheData{{2017/05/14 lokateo64, Traditional Cache (1.5/1.5)}}\begin{cacheText}Par ce bel après midi ensoleillé, en attendant la rencontre Biarritz / Agen je m échappe pour loguer cette petite cache fort sympathique. Le logbook est humide.... MPLC\end{cacheText}

\cacheNumber{77}\needspace{5\baselineskip}\cacheName{\href{http://coord.info/GC5CK0Q}{Alphabet Landais ... H pour Hagetmau: crypte} — \href{http://coord.info/GC5CK0Q\Number{}686094507}{77}}\cacheData{{2017/05/15 MLF40, Traditional Cache (1.5/2)}}\begin{cacheText}Le lieu est magnifique et très paisible.Sans cette cache je n'aurais certainement jamais pris le temps de m'arrêter.L'indice aide bien.  MPLC\end{cacheText}

\cacheNumber{78}\needspace{5\baselineskip}\cacheName{\href{http://coord.info/GC6JWW5}{Le royaume de petit fébus 8} — \href{http://coord.info/GC6JWW5\Number{}686565531}{78}}\cacheData{{2017/05/17 yvain64, Traditional Cache (1.5/2.5)}}\begin{cacheText}Sans hésitation, la cache est trouvée mais moment de panique devant elle.... comment ouvrir ? Ou est la clé ? Finalement le log book est signé. Mplc\end{cacheText}

\cacheNumber{79}\needspace{5\baselineskip}\cacheName{\href{http://coord.info/GC6JWWZ}{Le royaume de petit fébus 9} — \href{http://coord.info/GC6JWWZ\Number{}686565956}{79}}\cacheData{{2017/05/17 yvain64, Traditional Cache (1.5/2.5)}}\begin{cacheText}Cache vite trouvée... un joli champignon. MPLC\end{cacheText}

\cacheNumber{80}\needspace{5\baselineskip}\cacheName{\href{http://coord.info/GC6JWX7}{Le royaume de petit fébus 10} — \href{http://coord.info/GC6JWX7\Number{}686566254}{80}}\cacheData{{2017/05/17 yvain64, Traditional Cache (1.5/3)}}\begin{cacheText}L'indice aide bien , cache assez facile. MPLC\end{cacheText}

\cacheNumber{81}\needspace{5\baselineskip}\cacheName{\href{http://coord.info/GC6JX02}{Le royaume de petit fébus 11} — \href{http://coord.info/GC6JX02\Number{}686566733}{81}}\cacheData{{2017/05/17 yvain64, Traditional Cache (1.5/2.5)}}\begin{cacheText}Malgré un GPS qui nous fait tourner en rond la cache est trouvée et l' indice relevé MPLC\end{cacheText}

\cacheNumber{82}\needspace{5\baselineskip}\cacheName{\href{http://coord.info/GC72QE0}{Une cache pour Petit Fébus} — \href{http://coord.info/GC72QE0\Number{}686567327}{82}}\cacheData{{2017/05/17 domino50, Traditional Cache (1.5/1.5)}}\begin{cacheText}C' est la cache qui m'a donné le plus de fil à retordre. En persévérant la cache a été débusquée. MPLC\end{cacheText}

\cacheNumber{83}\needspace{5\baselineskip}\cacheName{\href{http://coord.info/GC6K5YA}{Le royaume de petit fébus 12} — \href{http://coord.info/GC6K5YA\Number{}686567696}{83}}\cacheData{{2017/05/17 yvain64, Traditional Cache (1.5/2.5)}}\begin{cacheText}Joli endroit... la cache ne présente pas de difficulté. MPLC\end{cacheText}

\cacheNumber{84}\needspace{5\baselineskip}\cacheName{\href{http://coord.info/GC6K5YX}{Le royaume de petit fébus 13} — \href{http://coord.info/GC6K5YX\Number{}686567948}{84}}\cacheData{{2017/05/17 yvain64, Traditional Cache (1.5/2.5)}}\begin{cacheText}Merci l'indice ...la cache est vite repérée.MPLC\end{cacheText}

\cacheNumber{85}\needspace{5\baselineskip}\cacheName{\href{http://coord.info/GC6K5Z8}{Le royaume de petit fébus 14} — \href{http://coord.info/GC6K5Z8\Number{}686568264}{85}}\cacheData{{2017/05/17 yvain64, Traditional Cache (1.5/1.5)}}\begin{cacheText}Eléagnus???vive internet...la cache nous attend.MPLC\end{cacheText}

\cacheNumber{86}\needspace{5\baselineskip}\cacheName{\href{http://coord.info/GC6K5ZF}{Le royaume de petit fébus 15} — \href{http://coord.info/GC6K5ZF\Number{}686568618}{86}}\cacheData{{2017/05/17 yvain64, Traditional Cache (1.5/1.5)}}\begin{cacheText}Cache vite repérée : l'indice est noté et la cache remise à sa place. MPLC\end{cacheText}

\cacheNumber{87}\needspace{5\baselineskip}\cacheName{\href{http://coord.info/GC6K5ZX}{Le royaume de petit fébus 16} — \href{http://coord.info/GC6K5ZX\Number{}686569202}{87}}\cacheData{{2017/05/17 yvain64, Traditional Cache (1.5/2)}}\begin{cacheText}Sans problème la cache est débusquée.Un joli champignon au pied d'un arbre.MPLC\end{cacheText}

\cacheNumber{88}\needspace{5\baselineskip}\cacheName{\href{http://coord.info/GC6Q3T6}{BONUS le royaume de petit Fébus} — \href{http://coord.info/GC6Q3T6\Number{}686572115}{88}}\cacheData{{2017/05/17 yvain64, Unknown Cache (1.5/2.5)}}\begin{cacheText}Après avoir fait cette superbe balade,  relevé tout les indices et fait les calculs ...la cache mystère se dévoile enfin!!! Que de trésors!!!c'est ma première Mysterie et suis très heureuse . Merci Yvain64 pour tout ce fabuleux travail.\end{cacheText}

\cacheNumber{89}\needspace{5\baselineskip}\cacheName{\href{http://coord.info/GC5J4EF}{Villa Beatrix Enea} — \href{http://coord.info/GC5J4EF\Number{}687028341}{89}}\cacheData{{2017/05/19 Gamboy, Traditional Cache (1.5/1.5)}}\begin{cacheText}Ayant un peu de temps devant moi avant la fin des cours je décide de dénicher quelques caches. Villa en travaux mais la cache reste accessible. Merci aux divers indices et au GAMBOY pour le travail réalisé.\end{cacheText}

\cacheNumber{90}\needspace{5\baselineskip}\cacheName{\href{http://coord.info/GC6FQZ0}{Villas Brigandière, Closerie et El Patio - Anglet} — \href{http://coord.info/GC6FQZ0\Number{}687028973}{90}}\cacheData{{2017/05/19 gilles64, Traditional Cache (1.5/1.5)}}\begin{cacheText}Très belles villas. La cache est vite trouvée et loguée à l'abris des moldus .Merci Gilles64 pour ce bel endroit\end{cacheText}

\cacheNumber{91}\needspace{5\baselineskip}\cacheName{\href{http://coord.info/GC40E2G}{Lavoir Louillot} — \href{http://coord.info/GC40E2G\Number{}687042848}{91}}\cacheData{{2017/05/20 gilles64, Traditional Cache (1.5/1.5)}}\begin{cacheText}Encore un lavoir démoli , quel dommage !!!! Merci Gilles de nous le présenter. La cache est vite trouvée grâce à l'indice....un peu d'escalade et hop je signe\end{cacheText}

\cacheNumber{92}\needspace{5\baselineskip}\cacheName{\href{http://coord.info/GC6EM42}{Villa Baroja - Anglet} — \href{http://coord.info/GC6EM42\Number{}687043355}{92}}\cacheData{{2017/05/20 gilles64, Traditional Cache (1.5/1.5)}}\begin{cacheText}Le château est en travaux....difficile de l'admirer avec les barrières pleines. Tant pis.... La cache est vite trouvée : le bout de ruban utilisé par GAMBOY dépasse un peu...MPLC\end{cacheText}

\cacheNumber{93}\needspace{5\baselineskip}\cacheName{\href{http://coord.info/GC6FRK1}{Villa Sophia - Anglet} — \href{http://coord.info/GC6FRK1\Number{}687043781}{93}}\cacheData{{2017/05/20 gilles64, Traditional Cache (1.5/1.5)}}\begin{cacheText}Arrivée devant la villa, j'hésite et pense ne pas la trouver. À force de chercher je finis par la déloger.... trop contente MPLC\end{cacheText}

\cacheNumber{94}\needspace{5\baselineskip}\cacheName{\href{http://coord.info/GC40JGE}{Moulin à vent d'Anglet ! (Tour de Lannes)} — \href{http://coord.info/GC40JGE\Number{}687044993}{94}}\cacheData{{2017/05/20 gilles64, Traditional Cache (1.5/2.5)}}\begin{cacheText}Quelle belle surprise ce beau moulin à vent!!!!!Je ne savais pas qu'il existait encore des vestiges au pays basque.Merci Gilles64 pour cette très belle cache qui a été vite découverte grâce à l'indice\end{cacheText}

\cacheNumber{95}\needspace{5\baselineskip}\cacheName{\href{http://coord.info/GC6K608}{Le royaume de petit fébus 17} — \href{http://coord.info/GC6K608\Number{}687757046}{95}}\cacheData{{2017/05/22 yvain64, Traditional Cache (1.5/2)}}\begin{cacheText}Dernière cache et dernier indice.... bien camouflé MPLC\end{cacheText}

\cacheNumber{96}\needspace{5\baselineskip}\cacheName{\href{http://coord.info/GC72FNY}{[GTAQ 25] 03 – Montfort Sud I. - A bâbord toute!} — \href{http://coord.info/GC72FNY\Number{}688694290}{96}}\cacheData{{2017/05/25 DorisBear, Traditional Cache (1.5/3)}}\begin{cacheText}GPS capricieux mais la cache est débusquée.Log humide.merci\end{cacheText}

\cacheNumber{97}\needspace{5\baselineskip}\cacheName{\href{http://coord.info/GC6YKXP}{GTAQ Saison 3 - Episode 1 : L'Adour} — \href{http://coord.info/GC6YKXP\Number{}688690433}{97}}\cacheData{{2017/05/26 GTAQ, Event Cache (1/1)}}\begin{cacheText}Court mais excellent moment passé en compagnie de tous ces passionnés. Merci aux organisateurs\end{cacheText}

\cacheNumber{98}\needspace{5\baselineskip}\cacheName{\href{http://coord.info/GC6620R}{[GTAQ 25] 01 – Montfort Sud I. – Salut Cloclo} — \href{http://coord.info/GC6620R\Number{}688691422}{98}}\cacheData{{2017/05/26 DorisBear, Traditional Cache (1.5/1.5)}}\begin{cacheText}La première de la journée, pas trop de difficulté...MPLC\end{cacheText}

\cacheNumber{99}\needspace{5\baselineskip}\cacheName{\href{http://coord.info/GC72FPJ}{[GTAQ 25] 04 – Montfort Sud I. - Charlie III} — \href{http://coord.info/GC72FPJ\Number{}688697386}{99}}\cacheData{{2017/05/26 DorisBear, Traditional Cache (2/1.5)}}\begin{cacheText}Avec quelques difficultés...Merci à une équipe de geocacheurs qui passait par la et m'a aidé.Belle cache merci\end{cacheText}

\cacheNumber{100}\needspace{5\baselineskip}\cacheName{\href{http://coord.info/GC72FTD}{[GTAQ 25] 07 – Montfort Sud I. – Au sec} — \href{http://coord.info/GC72FTD\Number{}688719285}{100}}\cacheData{{2017/05/26 DorisBear, Traditional Cache (2/2)}}\begin{cacheText}Les traces laissées par mes prédécesseurs m'ont beaucoup aidé La cache n'est pas facile.MPLC\end{cacheText}

\cacheNumber{101}\needspace{5\baselineskip}\cacheName{\href{http://coord.info/GC72GQM}{[GTAQ 25] 08 – Montfort Sud I. - Comestible ?} — \href{http://coord.info/GC72GQM\Number{}688719970}{101}}\cacheData{{2017/05/26 DorisBear, Traditional Cache (1.5/1.5)}}\begin{cacheText}Ce n'est pourtant pas la saison....MPLC\end{cacheText}

\cacheNumber{102}\needspace{5\baselineskip}\cacheName{\href{http://coord.info/GC72FRJ}{[GTAQ 25] 06 – Montfort Sud I. - Sous le pont} — \href{http://coord.info/GC72FRJ\Number{}688721593}{102}}\cacheData{{2017/05/26 DorisBear, Traditional Cache (2/1.5)}}\begin{cacheText}J'ai cherché un moment puis j'ai trouvé....plein de petits trésors MPLC\end{cacheText}

\cacheNumber{103}\needspace{5\baselineskip}\cacheName{\href{http://coord.info/GC72FR2}{[GTAQ 25] 05 – Montfort Sud I. - Bon œil bon pied} — \href{http://coord.info/GC72FR2\Number{}688723929}{103}}\cacheData{{2017/05/26 DorisBear, Traditional Cache (2/1.5)}}\begin{cacheText}Cache élaborée et bien camouflée. Quel travail!!!! MPLC\end{cacheText}

\cacheNumber{104}\needspace{5\baselineskip}\cacheName{\href{http://coord.info/GC72GRQ}{[GTAQ 25] 14 – Montfort Sud I. - La coupure} — \href{http://coord.info/GC72GRQ\Number{}689102763}{104}}\cacheData{{2017/05/27 DorisBear, Traditional Cache (1.5/1.5)}}\begin{cacheText}Super\end{cacheText}

\cacheNumber{105}\needspace{5\baselineskip}\cacheName{\href{http://coord.info/GC72GRH}{[GTAQ 25] 13 – Montfort Sud I. - Drôle de noisette} — \href{http://coord.info/GC72GRH\Number{}689103930}{105}}\cacheData{{2017/05/27 DorisBear, Traditional Cache (2/2)}}\begin{cacheText}Mplc\end{cacheText}

\cacheNumber{106}\needspace{5\baselineskip}\cacheName{\href{http://coord.info/GC72GR6}{[GTAQ 25] 11 – Montfort Sud I. - Sans clé} — \href{http://coord.info/GC72GR6\Number{}689106275}{106}}\cacheData{{2017/05/27 DorisBear, Traditional Cache (2/2)}}\begin{cacheText}Tres Belle cache mplc\end{cacheText}

\cacheNumber{107}\needspace{5\baselineskip}\cacheName{\href{http://coord.info/GC72GR1}{[GTAQ 25] 10 – Montfort Sud I. - Cendrillon} — \href{http://coord.info/GC72GR1\Number{}689107274}{107}}\cacheData{{2017/05/27 DorisBear, Traditional Cache (2/2)}}\begin{cacheText}Cache originale mplc\end{cacheText}

\cacheNumber{108}\needspace{5\baselineskip}\cacheName{\href{http://coord.info/GC72GQX}{[GTAQ 25] 09 – Montfort Sud I. - Occupé} — \href{http://coord.info/GC72GQX\Number{}689108161}{108}}\cacheData{{2017/05/27 DorisBear, Traditional Cache (1.5/2)}}\begin{cacheText}Quelle Belle cache merci\end{cacheText}

\cacheNumber{109}\needspace{5\baselineskip}\cacheName{\href{http://coord.info/GC6C43D}{[GTAQ 25] 02 – Montfort Sud I. – Billard} — \href{http://coord.info/GC6C43D\Number{}689111618}{109}}\cacheData{{2017/05/27 DorisBear, Traditional Cache (1.5/1.5)}}\begin{cacheText}Très belle cache il m'a fallu venir 2 fois pour la trouver Merci\end{cacheText}

\cacheNumber{110}\needspace{5\baselineskip}\cacheName{\href{http://coord.info/GC72GRC}{[GTAQ 25] 12 – Montfort Sud I. - C'est la fuite} — \href{http://coord.info/GC72GRC\Number{}689115708}{110}}\cacheData{{2017/05/27 DorisBear, Traditional Cache (1.5/1.5)}}\begin{cacheText}Excellente cache mplc\end{cacheText}

\cacheNumber{111}\needspace{5\baselineskip}\cacheName{\href{http://coord.info/GC72K17}{[GTAQ 26] 21 – Montfort Sud II. – Érosion} — \href{http://coord.info/GC72K17\Number{}689117798}{111}}\cacheData{{2017/05/27 DorisBear, Traditional Cache (2/2.5)}}\begin{cacheText}Plein de trésors dans la cache merci\end{cacheText}

\cacheNumber{112}\needspace{5\baselineskip}\cacheName{\href{http://coord.info/GC72GW9}{[GTAQ 25] 22 – Montfort Sud I. -Mission Impossible} — \href{http://coord.info/GC72GW9\Number{}689371162}{112}}\cacheData{{2017/05/27 DorisBear, Traditional Cache (3/1.5)}}\begin{cacheText}Mplc !cache très recherchée que j'ai trouvé très rapidement (une fois n'est pas coutume). Par contre le logbook est plein. Il faut le changer rapidement. Je l'ai signé en travers.... encore merci pour votre travail les Dorisbear\end{cacheText}

\cacheNumber{113}\needspace{5\baselineskip}\cacheName{\href{http://coord.info/GC72K2C}{[GTAQ 26] 27 – Montfort Sud II. – Bonus} — \href{http://coord.info/GC72K2C\Number{}689243694}{113}}\cacheData{{2017/05/27 DorisBear, Unknown Cache (2/2)}}\begin{cacheText}Après avoir relevé tous les indices et passé le Check on file en compagnie de Meïssane\And{}Dana. La cache est bien la dans la jolie \Quoted{toupigne}. Merci Dorisbear pour ce beau travail\end{cacheText}

\cacheNumber{114}\needspace{5\baselineskip}\cacheName{\href{http://coord.info/GC72K0V}{[GTAQ 26] 19 – Montfort Sud II. – Ça gave !} — \href{http://coord.info/GC72K0V\Number{}689128441}{114}}\cacheData{{2017/05/27 DorisBear, Traditional Cache (1.5/2)}}\begin{cacheText}Sans difficulté mplc\end{cacheText}

\cacheNumber{115}\needspace{5\baselineskip}\cacheName{\href{http://coord.info/GC72K0N}{[GTAQ 26] 18 – Montfort Sud II. – Joyeux Noël !} — \href{http://coord.info/GC72K0N\Number{}689129952}{115}}\cacheData{{2017/05/27 DorisBear, Traditional Cache (1.5/1.5)}}\begin{cacheText}Plutôt facile mplc\end{cacheText}

\cacheNumber{116}\needspace{5\baselineskip}\cacheName{\href{http://coord.info/GC72K0J}{[GTAQ 26] 17 – Montfort Sud II. – La ruine} — \href{http://coord.info/GC72K0J\Number{}689134940}{116}}\cacheData{{2017/05/27 DorisBear, Traditional Cache (2/1.5)}}\begin{cacheText}Tres Belle cache trouvée en compagnie\end{cacheText}

\cacheNumber{117}\needspace{5\baselineskip}\cacheName{\href{http://coord.info/GC72K0B}{[GTAQ 26] 16 – Montfort Sud II. – Tondu} — \href{http://coord.info/GC72K0B\Number{}689136202}{117}}\cacheData{{2017/05/27 DorisBear, Traditional Cache (2/1.5)}}\begin{cacheText}Bien tondu mplc\end{cacheText}

\cacheNumber{118}\needspace{5\baselineskip}\cacheName{\href{http://coord.info/GC72K09}{[GTAQ 26] 15 – Montfort Sud II. – Oubliettes} — \href{http://coord.info/GC72K09\Number{}689137562}{118}}\cacheData{{2017/05/27 DorisBear, Traditional Cache (1.5/1.5)}}\begin{cacheText}Encore une tres belle réalisation mplc\end{cacheText}

\cacheNumber{119}\needspace{5\baselineskip}\cacheName{\href{http://coord.info/GC72K00}{[GTAQ 26] 14 – Montfort Sud II. – Épiphyte} — \href{http://coord.info/GC72K00\Number{}689175542}{119}}\cacheData{{2017/05/27 DorisBear, Traditional Cache (2/2)}}\begin{cacheText}Parfaitement intégrée 2 émeutes passage mplc\end{cacheText}

\cacheNumber{120}\needspace{5\baselineskip}\cacheName{\href{http://coord.info/GC72JZP}{[GTAQ 26] 12 – Montfort Sud II. – Sans bruit} — \href{http://coord.info/GC72JZP\Number{}689190103}{120}}\cacheData{{2017/05/27 DorisBear, Letterbox Hybrid (1.5/2)}}\begin{cacheText}Superbe mplc\end{cacheText}

\cacheNumber{121}\needspace{5\baselineskip}\cacheName{\href{http://coord.info/GC72JZV}{[GTAQ 26] 13 – Montfort Sud II. – Pénurie} — \href{http://coord.info/GC72JZV\Number{}689194498}{121}}\cacheData{{2017/05/27 DorisBear, Traditional Cache (2.5/1.5)}}\begin{cacheText}Superbe cache mais pas de fée clochette..\end{cacheText}

\cacheNumber{122}\needspace{5\baselineskip}\cacheName{\href{http://coord.info/GC72JZJ}{[GTAQ 26] 11 – Montfort Sud II. –Le bout du tunnel} — \href{http://coord.info/GC72JZJ\Number{}689195435}{122}}\cacheData{{2017/05/27 DorisBear, Traditional Cache (1.5/2)}}\begin{cacheText}Log bien au bout du tuyau mplc\end{cacheText}

\cacheNumber{123}\needspace{5\baselineskip}\cacheName{\href{http://coord.info/GC72JZ8}{[GTAQ 26] 10 – Montfort Sud II. – Leurre} — \href{http://coord.info/GC72JZ8\Number{}689198227}{123}}\cacheData{{2017/05/27 DorisBear, Traditional Cache (2.5/1.5)}}\begin{cacheText}Quel leurre.... superbe cache merci\end{cacheText}

\cacheNumber{124}\needspace{5\baselineskip}\cacheName{\href{http://coord.info/GC72JZ4}{[GTAQ 26] 09 – Montfort Sud II. – Mesurez !} — \href{http://coord.info/GC72JZ4\Number{}689200948}{124}}\cacheData{{2017/05/27 DorisBear, Traditional Cache (2/1.5)}}\begin{cacheText}Mplc\end{cacheText}

\cacheNumber{125}\needspace{5\baselineskip}\cacheName{\href{http://coord.info/GC72JYJ}{[GTAQ 26] 08 – Montfort Sud II. – Justice} — \href{http://coord.info/GC72JYJ\Number{}689201880}{125}}\cacheData{{2017/05/27 DorisBear, Traditional Cache (1.5/2)}}\begin{cacheText}Super\end{cacheText}

\cacheNumber{126}\needspace{5\baselineskip}\cacheName{\href{http://coord.info/GC72JY8}{[GTAQ 26] 07 – Montfort Sud II. – Piège à mouches} — \href{http://coord.info/GC72JY8\Number{}689207174}{126}}\cacheData{{2017/05/28 DorisBear, Traditional Cache (2/1.5)}}\begin{cacheText}Super\end{cacheText}

\cacheNumber{127}\needspace{5\baselineskip}\cacheName{\href{http://coord.info/GC72JY3}{[GTAQ 26] 06 – Montfort Sud II. – Désaltéré} — \href{http://coord.info/GC72JY3\Number{}689209539}{127}}\cacheData{{2017/05/28 DorisBear, Traditional Cache (1.5/1.5)}}\begin{cacheText}Avec cette chaleur ça fait du bien mplc\end{cacheText}

\cacheNumber{128}\needspace{5\baselineskip}\cacheName{\href{http://coord.info/GC72JXV}{[GTAQ 26] 05 – Montfort Sud II. – L'estocade} — \href{http://coord.info/GC72JXV\Number{}689211861}{128}}\cacheData{{2017/05/28 DorisBear, Traditional Cache (2/2)}}\begin{cacheText}Mplc\end{cacheText}

\cacheNumber{129}\needspace{5\baselineskip}\cacheName{\href{http://coord.info/GC72H43}{[GTAQ 26] 04 – Montfort Sud II. – Tête de bois} — \href{http://coord.info/GC72H43\Number{}689214416}{129}}\cacheData{{2017/05/28 DorisBear, Traditional Cache (2/2)}}\begin{cacheText}Logbook signé Belle cache merci\end{cacheText}

\cacheNumber{130}\needspace{5\baselineskip}\cacheName{\href{http://coord.info/GC72H3X}{[GTAQ 26] 03 – Montfort Sud II. – Sol carrelé} — \href{http://coord.info/GC72H3X\Number{}689215635}{130}}\cacheData{{2017/05/28 DorisBear, Traditional Cache (1.5/1.5)}}\begin{cacheText}Mplc\end{cacheText}

\cacheNumber{131}\needspace{5\baselineskip}\cacheName{\href{http://coord.info/GC72H2Y}{[GTAQ 26] 02 – Montfort Sud II. – Oh !!} — \href{http://coord.info/GC72H2Y\Number{}689218142}{131}}\cacheData{{2017/05/28 DorisBear, Traditional Cache (2/2)}}\begin{cacheText}Haut mplc\end{cacheText}

\cacheNumber{132}\needspace{5\baselineskip}\cacheName{\href{http://coord.info/GC72H18}{[GTAQ 26] 01 – Montfort Sud II. – Zeus} — \href{http://coord.info/GC72H18\Number{}689220226}{132}}\cacheData{{2017/05/28 DorisBear, Traditional Cache (1.5/1.5)}}\begin{cacheText}Déplacée à cause des guêpes\end{cacheText}

\cacheNumber{133}\needspace{5\baselineskip}\cacheName{\href{http://coord.info/GC72K1C}{[GTAQ 26] 22 – Montfort Sud II. – Quel choix !} — \href{http://coord.info/GC72K1C\Number{}689243493}{133}}\cacheData{{2017/05/28 DorisBear, Traditional Cache (1.5/1.5)}}\begin{cacheText}Une Belle araignée... j'ai peur!!!! mplc\end{cacheText}

\cacheNumber{134}\needspace{5\baselineskip}\cacheName{\href{http://coord.info/GC72GRZ}{[GTAQ 25] 15 – Montfort Sud I. - Y a quelqu'un ?} — \href{http://coord.info/GC72GRZ\Number{}689253648}{134}}\cacheData{{2017/05/28 DorisBear, Traditional Cache (1.5/1.5)}}\begin{cacheText}La cache est la ....merci\end{cacheText}

\cacheNumber{135}\needspace{5\baselineskip}\cacheName{\href{http://coord.info/GC6YKXW}{GTAQ Saison 3 - Episode 3 : La Gascogne} — \href{http://coord.info/GC6YKXW\Number{}689279809}{135}}\cacheData{{2017/05/28 GTAQ, Event Cache (1/1)}}\begin{cacheText}Excellente journée Merci aux organisateurs\end{cacheText}

\cacheNumber{136}\needspace{5\baselineskip}\cacheName{\href{http://coord.info/GC72K10}{[GTAQ 26] 20 – Montfort Sud II. – Pingu} — \href{http://coord.info/GC72K10\Number{}689370292}{136}}\cacheData{{2017/05/28 DorisBear, Traditional Cache (1.5/1.5)}}\begin{cacheText}Au second passage grâce à l'aide de super geocacheurs la cache est débusquée mplc\end{cacheText}

\cacheNumber{137}\needspace{5\baselineskip}\cacheName{\href{http://coord.info/GC72K26}{[GTAQ 26] 26 – Montfort Sud II. – Gloire} — \href{http://coord.info/GC72K26\Number{}689370404}{137}}\cacheData{{2017/05/28 DorisBear, Traditional Cache (2/1.5)}}\begin{cacheText}Mplc\end{cacheText}

\cacheNumber{138}\needspace{5\baselineskip}\cacheName{\href{http://coord.info/GC72K20}{[GTAQ 26] 25 – Montfort Sud II. – Fish 'n' Chips} — \href{http://coord.info/GC72K20\Number{}689370486}{138}}\cacheData{{2017/05/28 DorisBear, Traditional Cache (2/1.5)}}\begin{cacheText}Mplc\end{cacheText}

\cacheNumber{139}\needspace{5\baselineskip}\cacheName{\href{http://coord.info/GC72K1J}{[GTAQ 26] 23 – Montfort Sud II. – Arghhh !!} — \href{http://coord.info/GC72K1J\Number{}689370555}{139}}\cacheData{{2017/05/28 DorisBear, Traditional Cache (2/1.5)}}\begin{cacheText}Mplc\end{cacheText}

\cacheNumber{140}\needspace{5\baselineskip}\cacheName{\href{http://coord.info/GC72K1V}{[GTAQ 26] 24 – Montfort Sud II. – Boummm !!} — \href{http://coord.info/GC72K1V\Number{}689370634}{140}}\cacheData{{2017/05/28 DorisBear, Traditional Cache (1.5/1.5)}}\begin{cacheText}Mplc\end{cacheText}

\cacheNumber{141}\needspace{5\baselineskip}\cacheName{\href{http://coord.info/GC72GV3}{[GTAQ 25] 16 – Montfort Sud I. - Bêêêê !} — \href{http://coord.info/GC72GV3\Number{}689370717}{141}}\cacheData{{2017/05/28 DorisBear, Traditional Cache (1.5/1.5)}}\begin{cacheText}Mplc\end{cacheText}

\cacheNumber{142}\needspace{5\baselineskip}\cacheName{\href{http://coord.info/GC72GVB}{[GTAQ 25] 17 – Montfort Sud I. - Et alors ?} — \href{http://coord.info/GC72GVB\Number{}689370779}{142}}\cacheData{{2017/05/28 DorisBear, Traditional Cache (1.5/1.5)}}\begin{cacheText}Mplc\end{cacheText}

\cacheNumber{143}\needspace{5\baselineskip}\cacheName{\href{http://coord.info/GC72GVM}{[GTAQ 25] 18 – Montfort Sud I. - Ent} — \href{http://coord.info/GC72GVM\Number{}689370873}{143}}\cacheData{{2017/05/28 DorisBear, Traditional Cache (2/2.5)}}\begin{cacheText}Mplc\end{cacheText}

\cacheNumber{144}\needspace{5\baselineskip}\cacheName{\href{http://coord.info/GC72GVT}{[GTAQ 25] 19 – Montfort Sud I. - Tous aux abris !} — \href{http://coord.info/GC72GVT\Number{}689370960}{144}}\cacheData{{2017/05/28 DorisBear, Traditional Cache (2/1.5)}}\begin{cacheText}Mplc\end{cacheText}

\cacheNumber{145}\needspace{5\baselineskip}\cacheName{\href{http://coord.info/GC72GVZ}{[GTAQ 25] 20 – Montfort Sud I. - A la vôtre !} — \href{http://coord.info/GC72GVZ\Number{}689371070}{145}}\cacheData{{2017/05/28 DorisBear, Traditional Cache (2.5/1.5)}}\begin{cacheText}Mplc\end{cacheText}

\cacheNumber{146}\needspace{5\baselineskip}\cacheName{\href{http://coord.info/GC72GW4}{[GTAQ 25] 21 – Montfort Sud I. - Allo, t'es où ?} — \href{http://coord.info/GC72GW4\Number{}689371249}{146}}\cacheData{{2017/05/28 DorisBear, Traditional Cache (2/1.5)}}\begin{cacheText}Mplc\end{cacheText}

\cacheNumber{147}\needspace{5\baselineskip}\cacheName{\href{http://coord.info/GC72GWE}{[GTAQ 25] 23 – Montfort Sud I. - Vélocipède} — \href{http://coord.info/GC72GWE\Number{}689371364}{147}}\cacheData{{2017/05/28 DorisBear, Traditional Cache (2/1.5)}}\begin{cacheText}Mplc\end{cacheText}

\cacheNumber{148}\needspace{5\baselineskip}\cacheName{\href{http://coord.info/GC72GWX}{[GTAQ 25] 25 – Montfort Sud I. - On avait soif!} — \href{http://coord.info/GC72GWX\Number{}689371500}{148}}\cacheData{{2017/05/28 DorisBear, Traditional Cache (1.5/1.5)}}\begin{cacheText}Mplc\end{cacheText}

\cacheNumber{149}\needspace{5\baselineskip}\cacheName{\href{http://coord.info/GC72GWM}{[GTAQ 25] 24 – Montfort Sud I. - Le Rucher} — \href{http://coord.info/GC72GWM\Number{}689371571}{149}}\cacheData{{2017/05/28 DorisBear, Letterbox Hybrid (1.5/1.5)}}\begin{cacheText}Mplc\end{cacheText}

\cacheNumber{150}\needspace{5\baselineskip}\cacheName{\href{http://coord.info/GC72GX2}{[GTAQ 25] 26 – Montfort Sud I. - Moi aussi !} — \href{http://coord.info/GC72GX2\Number{}689371670}{150}}\cacheData{{2017/05/28 DorisBear, Traditional Cache (2/1.5)}}\begin{cacheText}Mplc\end{cacheText}

\cacheNumber{151}\needspace{5\baselineskip}\cacheName{\href{http://coord.info/GC72GX6}{[GTAQ 25] 27 – Montfort Sud I. - Pause calcul} — \href{http://coord.info/GC72GX6\Number{}689371935}{151}}\cacheData{{2017/05/28 DorisBear, Traditional Cache (2/1.5)}}\begin{cacheText}Mplc\end{cacheText}

\cacheNumber{152}\needspace{5\baselineskip}\cacheName{\href{http://coord.info/GC7370Y}{[GTAQ 25] \Number{}01 - Hinx - Les arènes} — \href{http://coord.info/GC7370Y\Number{}689456343}{152}}\cacheData{{2017/05/28 Sod@'s, Traditional Cache (2/1)}}\begin{cacheText}Mplc\end{cacheText}

\cacheNumber{153}\needspace{5\baselineskip}\cacheName{\href{http://coord.info/GC72J00}{[GTAQ 25] \Number{}02 - Hinx - Le monument} — \href{http://coord.info/GC72J00\Number{}689457388}{153}}\cacheData{{2017/05/28 Sod@'s, Traditional Cache (1.5/1)}}\begin{cacheText}Ok\end{cacheText}

\cacheNumber{154}\needspace{5\baselineskip}\cacheName{\href{http://coord.info/GC72JCV}{[GTAQ 25] \Number{}03 - Hinx - La haie} — \href{http://coord.info/GC72JCV\Number{}689460786}{154}}\cacheData{{2017/05/28 Sod@'s, Traditional Cache (2/2)}}\begin{cacheText}Superbe\end{cacheText}

\cacheNumber{155}\needspace{5\baselineskip}\cacheName{\href{http://coord.info/GC5YFKM}{La vie à la ferme} — \href{http://coord.info/GC5YFKM\Number{}689462108}{155}}\cacheData{{2017/05/28 senninhaN.R., Traditional Cache (1/1.5)}}\begin{cacheText}Tresors\end{cacheText}

\cacheNumber{156}\needspace{5\baselineskip}\cacheName{\href{http://coord.info/GC72JDD}{[GTAQ 25] \Number{}05 - Hinx - WIFI} — \href{http://coord.info/GC72JDD\Number{}689472541}{156}}\cacheData{{2017/05/28 Sod@'s, Traditional Cache (2/1.5)}}\begin{cacheText}Super\end{cacheText}

\cacheNumber{157}\needspace{5\baselineskip}\cacheName{\href{http://coord.info/GC72JDN}{[GTAQ 25] \Number{}06 - Hinx - \Quoted{El ponte}} — \href{http://coord.info/GC72JDN\Number{}689473370}{157}}\cacheData{{2017/05/28 Sod@'s, Traditional Cache (3/1.5)}}\begin{cacheText}Mplc\end{cacheText}

\cacheNumber{158}\needspace{5\baselineskip}\cacheName{\href{http://coord.info/GC7372J}{[GTAQ 25] \Number{}07 - Hinx - Shaun le mouton} — \href{http://coord.info/GC7372J\Number{}689476162}{158}}\cacheData{{2017/05/28 Sod@'s, Traditional Cache (2.5/1.5)}}\begin{cacheText}Belle cache intégrée\end{cacheText}

\cacheNumber{159}\needspace{5\baselineskip}\cacheName{\href{http://coord.info/GC7373B}{[GTAQ 25] \Number{}08 - Hinx - The riders} — \href{http://coord.info/GC7373B\Number{}689477487}{159}}\cacheData{{2017/05/28 Sod@'s, Traditional Cache (1.5/2.5)}}\begin{cacheText}Super\end{cacheText}

\cacheNumber{160}\needspace{5\baselineskip}\cacheName{\href{http://coord.info/GC7374M}{[GTAQ 25] \Number{}09 - Hinx - Geocacheur = randonneur} — \href{http://coord.info/GC7374M\Number{}689478564}{160}}\cacheData{{2017/05/28 Sod@'s, Traditional Cache (2/1.5)}}\begin{cacheText}Super\end{cacheText}

\cacheNumber{161}\needspace{5\baselineskip}\cacheName{\href{http://coord.info/GC72GXD}{[GTAQ 25] 28 – Montfort Sud I. - Bonus} — \href{http://coord.info/GC72GXD\Number{}689446521}{161}}\cacheData{{2017/05/28 DorisBear, Unknown Cache (2/2.5)}}\begin{cacheText}Ayant terminé tard la boucle et étant fatiguée je reviens ce matin faire la bonus....magnifique. L'endroit est superbe et mérite d'être connu. C'est un super camouflage pour cette bonus.mplc\end{cacheText}

\cacheNumber{162}\needspace{5\baselineskip}\cacheName{\href{http://coord.info/GC74PTT}{[GTAQ 25] \Number{}11 - Hinx - La grange} — \href{http://coord.info/GC74PTT\Number{}689490378}{162}}\cacheData{{2017/05/28 Sod@'s, Traditional Cache (1.5/1.5)}}\begin{cacheText}Super\end{cacheText}

\cacheNumber{163}\needspace{5\baselineskip}\cacheName{\href{http://coord.info/GC74PVB}{[GTAQ 25] \Number{}12 - Hinx - ZZZzzzzzz dit l'abeille} — \href{http://coord.info/GC74PVB\Number{}689497591}{163}}\cacheData{{2017/05/28 Sod@'s, Traditional Cache (3/3)}}\begin{cacheText}Avec difficulté mais jolie cache merci\end{cacheText}

\cacheNumber{164}\needspace{5\baselineskip}\cacheName{\href{http://coord.info/GC74PVM}{[GTAQ 25] \Number{}13 - Hinx - Bientôt fini} — \href{http://coord.info/GC74PVM\Number{}689498760}{164}}\cacheData{{2017/05/28 Sod@'s, Traditional Cache (1.5/2)}}\begin{cacheText}Super\end{cacheText}

\cacheNumber{165}\needspace{5\baselineskip}\cacheName{\href{http://coord.info/GC74PVW}{[GTAQ 25] \Number{}14 - Hinx - Hall des Sports} — \href{http://coord.info/GC74PVW\Number{}689501233}{165}}\cacheData{{2017/05/28 Sod@'s, Traditional Cache (2.5/1.5)}}\begin{cacheText}Heureusement on cherche à l'ombre\end{cacheText}

\cacheNumber{166}\needspace{5\baselineskip}\cacheName{\href{http://coord.info/GC753FG}{[GTAQ 25] \Number{}15 - Hinx - L'école} — \href{http://coord.info/GC753FG\Number{}689503598}{166}}\cacheData{{2017/05/28 Sod@'s, Traditional Cache (2/1.5)}}\begin{cacheText}Sans difficulté mplc\end{cacheText}

\cacheNumber{167}\needspace{5\baselineskip}\cacheName{\href{http://coord.info/GC4RR2X}{BONJOUR HINX} — \href{http://coord.info/GC4RR2X\Number{}689521477}{167}}\cacheData{{2017/05/28 senninhaN.R., Traditional Cache (1.5/1.5)}}\begin{cacheText}Cache trouvé lors Du GTAQ mplc\end{cacheText}

\cacheNumber{168}\needspace{5\baselineskip}\cacheName{\href{http://coord.info/GC71XE9}{[GTAQ26]19-Sort en Chalosse-La nature} — \href{http://coord.info/GC71XE9\Number{}689530312}{168}}\cacheData{{2017/05/28 Opmb40, Traditional Cache (1.5/2)}}\begin{cacheText}Belle cache : lierre envahissant !!!!\end{cacheText}

\cacheNumber{169}\needspace{5\baselineskip}\cacheName{\href{http://coord.info/GC71ZG8}{GTAQ3-26\Number{}07 Clermont-Ozourt\Number{}Courroyes} — \href{http://coord.info/GC71ZG8\Number{}689543626}{169}}\cacheData{{2017/05/28 crispol40, Traditional Cache (1.5/1.5)}}\begin{cacheText}Mplc\end{cacheText}

\cacheNumber{170}\needspace{5\baselineskip}\cacheName{\href{http://coord.info/GC71ZFJ}{GTAQ3-26\Number{}06 Clermont-Ozourt\Number{}Bazin} — \href{http://coord.info/GC71ZFJ\Number{}689545359}{170}}\cacheData{{2017/05/28 crispol40, Traditional Cache (1.5/2)}}\begin{cacheText}Elle est prise au milieu des tentacules.... mplc\end{cacheText}

\cacheNumber{171}\needspace{5\baselineskip}\cacheName{\href{http://coord.info/GC71ZFA}{GTAQ3-26\Number{}05 Clermont-Ozourt\Number{}Bergeron 2} — \href{http://coord.info/GC71ZFA\Number{}689546321}{171}}\cacheData{{2017/05/28 crispol40, Traditional Cache (1.5/1.5)}}\begin{cacheText}Mplc\end{cacheText}

\cacheNumber{172}\needspace{5\baselineskip}\cacheName{\href{http://coord.info/GC71ZEZ}{GTAQ3-26\Number{}04\Number{}Clermont-Ozourt\Number{}Bergeron 1} — \href{http://coord.info/GC71ZEZ\Number{}689547885}{172}}\cacheData{{2017/05/28 crispol40, Traditional Cache (1.5/1.5)}}\begin{cacheText}Mplc\end{cacheText}

\cacheNumber{173}\needspace{5\baselineskip}\cacheName{\href{http://coord.info/GC71ZED}{GTAQ3-26\Number{}03\Number{}Clermont-Ozourt\Number{}potiche} — \href{http://coord.info/GC71ZED\Number{}689549021}{173}}\cacheData{{2017/05/28 crispol40, Traditional Cache (1.5/1.5)}}\begin{cacheText}Mplc\end{cacheText}

\cacheNumber{174}\needspace{5\baselineskip}\cacheName{\href{http://coord.info/GC726CN}{GTAQ3-26\Number{}43 Clermont-Ozourt\Number{}chateau} — \href{http://coord.info/GC726CN\Number{}689554411}{174}}\cacheData{{2017/05/28 crispol40, Traditional Cache (1.5/2)}}\begin{cacheText}Merci l'indice!!!!\end{cacheText}

\cacheNumber{175}\needspace{5\baselineskip}\cacheName{\href{http://coord.info/GC726CJ}{GTAQ3-26\Number{}42 Clermont-Ozourt\Number{}Beiou} — \href{http://coord.info/GC726CJ\Number{}689558822}{175}}\cacheData{{2017/05/28 crispol40, Traditional Cache (1.5/1.5)}}\begin{cacheText}Merci l'indice et Mplc\end{cacheText}

\cacheNumber{176}\needspace{5\baselineskip}\cacheName{\href{http://coord.info/GC726CG}{GTAQ3-26\Number{}41 Clermont-Ozourt\Number{}Saint Cricq} — \href{http://coord.info/GC726CG\Number{}689561809}{176}}\cacheData{{2017/05/28 crispol40, Traditional Cache (1.5/2)}}\begin{cacheText}TTF mplc\end{cacheText}

\cacheNumber{177}\needspace{5\baselineskip}\cacheName{\href{http://coord.info/GC726C9}{GTAQ3-26\Number{}40 Clermont-Ozourt\Number{}Casselon} — \href{http://coord.info/GC726C9\Number{}689563523}{177}}\cacheData{{2017/05/28 crispol40, Traditional Cache (1.5/1.5)}}\begin{cacheText}Merci l'indice\end{cacheText}

\cacheNumber{178}\needspace{5\baselineskip}\cacheName{\href{http://coord.info/GC6YKXX}{GTAQ Saison 3 - Episode 4 : La Côte d’Argent} — \href{http://coord.info/GC6YKXX\Number{}689592731}{178}}\cacheData{{2017/05/28 GTAQ, Event Cache (1/1)}}\begin{cacheText}Superbe GTAQ: merci aux organisateurs et aux poseurs pour tout le travail accompli. Que de

belles rencontres .....\end{cacheText}

\cacheNumber{179}\needspace{5\baselineskip}\cacheName{\href{http://coord.info/GC6G50V}{Le Château d'Andurein} — \href{http://coord.info/GC6G50V\Number{}690277823}{179}}\cacheData{{2017/05/30 Cestasgirl, Traditional Cache (1.5/1.5)}}\begin{cacheText}Trouvée sans difficulté, très beau château...mplc\end{cacheText}

\cacheNumber{180}\needspace{5\baselineskip}\cacheName{\href{http://coord.info/GC3FYEW}{GR8\Number{}01} — \href{http://coord.info/GC3FYEW\Number{}690906379}{180}}\cacheData{{2017/06/02 Peyo64, Traditional Cache (2/1.5)}}\begin{cacheText}Par ce bel après-midi, je décide de faire quelques caches du GR8 : celle-là est vite trouvée ( indice et photo très parlant). Merci pour la cache\end{cacheText}

\cacheNumber{181}\needspace{5\baselineskip}\cacheName{\href{http://coord.info/GC3FYGR}{GR8\Number{}02} — \href{http://coord.info/GC3FYGR\Number{}690934450}{181}}\cacheData{{2017/06/02 Peyo64, Traditional Cache (1.5/1.5)}}\begin{cacheText}Joli chemin de halage fréquenté par de nombreux promeneurs et leur chien. La cache est débusquée et signée discrètement grâce à l'indice. Merci\end{cacheText}

\cacheNumber{182}\needspace{5\baselineskip}\cacheName{\href{http://coord.info/GC3FYHY}{GR8\Number{}03} — \href{http://coord.info/GC3FYHY\Number{}690937016}{182}}\cacheData{{2017/06/02 Peyo64, Traditional Cache (2/1.5)}}\begin{cacheText}La cache était bien camouflée: il m'a fallu me battre avec les orties pour y accèder mais j'y suis arrivée. L'indice est prélevé...Merci\end{cacheText}

\cacheNumber{183}\needspace{5\baselineskip}\cacheName{\href{http://coord.info/GC3FYJE}{GR8\Number{}04} — \href{http://coord.info/GC3FYJE\Number{}690937970}{183}}\cacheData{{2017/06/02 Peyo64, Traditional Cache (1.5/1.5)}}\begin{cacheText}Cette cache est facile( Merci l'indice) et vite loguée car des promeneurs arrivent....\end{cacheText}

\cacheNumber{184}\needspace{5\baselineskip}\cacheName{\href{http://coord.info/GC3FYK0}{GR8\Number{}05} — \href{http://coord.info/GC3FYK0\Number{}690938971}{184}}\cacheData{{2017/06/02 Peyo64, Traditional Cache (1.5/1.5)}}\begin{cacheText}Des que j'ai vu le trou j'ai su que la cache était la .Merci\end{cacheText}

\cacheNumber{185}\needspace{5\baselineskip}\cacheName{\href{http://coord.info/GC3FYKJ}{GR8\Number{}06} — \href{http://coord.info/GC3FYKJ\Number{}690903522}{185}}\cacheData{{2017/06/02 Peyo64, Traditional Cache (1.5/1.5)}}\begin{cacheText}Dernière cache de la journée, trouvée sans difficulté grâce à l'indice. Merci Peyo pour ce joli circuit\end{cacheText}

\cacheNumber{186}\needspace{5\baselineskip}\cacheName{\href{http://coord.info/GC66257}{Le port de Guiche / The river port of Guiche} — \href{http://coord.info/GC66257\Number{}690905297}{186}}\cacheData{{2017/06/02 DorisBear, Traditional Cache (2/1.5)}}\begin{cacheText}Par cet après-midi couvert je décide de partir faire quelques caches( nostalgie du GTAQ).Apres avoir visité un coin et touché la cache que je n'ai pas vu ...je m'éloigne . Rien.... je reviens sur mes pas et enfin je la trouve. Une cache 100\Percent{} Dorisbear merci\end{cacheText}

\cacheNumber{187}\needspace{5\baselineskip}\cacheName{\href{http://coord.info/GC3FYKZ}{GR8\Number{}07} — \href{http://coord.info/GC3FYKZ\Number{}691789318}{187}}\cacheData{{2017/06/05 Peyo64, Traditional Cache (2/2)}}\begin{cacheText}La cache est vite trouvée !C'est en repartant que le train décide de passer: l'endroit si paisible devient extrêmement bruyant. Mplc\end{cacheText}

\cacheNumber{188}\needspace{5\baselineskip}\cacheName{\href{http://coord.info/GC3FYMK}{GR8\Number{}08} — \href{http://coord.info/GC3FYMK\Number{}691786846}{188}}\cacheData{{2017/06/05 Peyo64, Traditional Cache (1.5/1.5)}}\begin{cacheText}Grâce à l'indice , je repère l'endroit.

Un promeneur (et son chien )arrive à ce moment-là et me demande ce que j'ai perdu......  heuu non je n'ai rien perdu j'étudie la végétation en milieu humide.....Mplc\end{cacheText}

\cacheNumber{189}\needspace{5\baselineskip}\cacheName{\href{http://coord.info/GC3FYND}{GR8\Number{}09} — \href{http://coord.info/GC3FYND\Number{}691783744}{189}}\cacheData{{2017/06/05 Peyo64, Traditional Cache (1.5/1.5)}}\begin{cacheText}Une de plus , loguée entre 2 groupes de promeneurs.mplc\end{cacheText}

\cacheNumber{190}\needspace{5\baselineskip}\cacheName{\href{http://coord.info/GC3FYRG}{GR8\Number{}10} — \href{http://coord.info/GC3FYRG\Number{}691782142}{190}}\cacheData{{2017/06/05 Peyo64, Traditional Cache (2/2)}}\begin{cacheText}Suite de la promenade et ouiiiiii cache loguée grâce à l'indice. Mplc\end{cacheText}

\cacheNumber{191}\needspace{5\baselineskip}\cacheName{\href{http://coord.info/GC3FYTE}{GR8\Number{}12} — \href{http://coord.info/GC3FYTE\Number{}691799901}{191}}\cacheData{{2017/06/05 Peyo64, Traditional Cache (1.5/1.5)}}\begin{cacheText}Personne à l'horizon, cache vite trouvée. Mplc\end{cacheText}

\cacheNumber{192}\needspace{5\baselineskip}\cacheName{\href{http://coord.info/GC3FYTW}{GR8\Number{}13} — \href{http://coord.info/GC3FYTW\Number{}691805668}{192}}\cacheData{{2017/06/05 Peyo64, Traditional Cache (1.5/1.5)}}\begin{cacheText}À la vue du commentaire précédent,Je ne pensais pas la trouver mais elle est la!!!!Comme quoi il faut toujours chercher. Merci\end{cacheText}

\cacheNumber{193}\needspace{5\baselineskip}\cacheName{\href{http://coord.info/GC3FYVD}{GR8\Number{}14} — \href{http://coord.info/GC3FYVD\Number{}691807637}{193}}\cacheData{{2017/06/05 Peyo64, Traditional Cache (1.5/1.5)}}\begin{cacheText}Le parcours est toujours aussi agréable et la cache est en place. Merci\end{cacheText}

\cacheNumber{194}\needspace{5\baselineskip}\cacheName{\href{http://coord.info/GC3FYW8}{GR8\Number{}15} — \href{http://coord.info/GC3FYW8\Number{}691814224}{194}}\cacheData{{2017/06/05 Peyo64, Traditional Cache (1.5/1.5)}}\begin{cacheText}Endroit paisible mais un peu boueux. La cache m'attend ...mplc\end{cacheText}

\cacheNumber{195}\needspace{5\baselineskip}\cacheName{\href{http://coord.info/GC40R6V}{Eglise d'Aussurucq} — \href{http://coord.info/GC40R6V\Number{}692556672}{195}}\cacheData{{2017/06/08 gilles64, Traditional Cache (1.5/1.5)}}\begin{cacheText}Très jolie église et très joli château dans ce joli petit village Basque. Cache trouvée grâce à l'indice. Merci\end{cacheText}

\cacheNumber{196}\needspace{5\baselineskip}\cacheName{\href{http://coord.info/GC40RCD}{Tumulus Potxo} — \href{http://coord.info/GC40RCD\Number{}692556668}{196}}\cacheData{{2017/06/08 gilles64, Traditional Cache (1.5/1.5)}}\begin{cacheText}Cache trouvée facilement. Merci\end{cacheText}

\cacheNumber{197}\needspace{5\baselineskip}\cacheName{\href{http://coord.info/GC40RFK}{Dolmen d'Ite 1} — \href{http://coord.info/GC40RFK\Number{}692556669}{197}}\cacheData{{2017/06/08 gilles64, Traditional Cache (1.5/1.5)}}\begin{cacheText}Après l'église d'Aussurucq nous voilà parti faire les 2 caches Dolmen d'Ite. Je découvre l'endroit avec grand plaisir .Je ne connaissais absolument pas. La cache est débusquée grâce à l'indice. Merci\end{cacheText}

\cacheNumber{198}\needspace{5\baselineskip}\cacheName{\href{http://coord.info/GC4TQ4A}{Le Bonhomme d'Inchassendaga} — \href{http://coord.info/GC4TQ4A\Number{}692562963}{198}}\cacheData{{2017/06/08 gilles64, Traditional Cache (2/1.5)}}\begin{cacheText}Très joli endroit que je ne connaissais pas. La cache est vite débusquée grâce à l'indice. Merci\end{cacheText}

\cacheNumber{199}\needspace{5\baselineskip}\cacheName{\href{http://coord.info/GC5BA2Y}{Croix de Galzetaburu} — \href{http://coord.info/GC5BA2Y\Number{}692568828}{199}}\cacheData{{2017/06/08 gilles64, Traditional Cache (2/1.5)}}\begin{cacheText}À mon second passage, trouvée sans difficulté (comment n'ai je pas trouvé la première fois???). Merci 😊 Gilles\end{cacheText}

\cacheNumber{200}\needspace{5\baselineskip}\cacheName{\href{http://coord.info/GC5BAKZ}{Les pentes d'Ahusquy} — \href{http://coord.info/GC5BAKZ\Number{}692556675}{200}}\cacheData{{2017/06/08 gilles64, Traditional Cache (1.5/1.5)}}\begin{cacheText}Magnifique, l'endroit est grandiose. La cache est vite trouvée. Merci pour la découverte de ce lieu\end{cacheText}

\cacheNumber{201}\needspace{5\baselineskip}\cacheName{\href{http://coord.info/GC76Q3Z}{Église de Saint Étienne} — \href{http://coord.info/GC76Q3Z\Number{}692556674}{201}}\cacheData{{2017/06/08 Lorelei64, Traditional Cache (2.5/2)}}\begin{cacheText}[{FTF}]

Mon premier FTF sur ma 200 ème dans cette jolie église. Cache trouvée et logée malgré des voisins assez curieux, Merci pour cette jolie cache.\end{cacheText}

\cacheNumber{202}\needspace{5\baselineskip}\cacheName{\href{http://coord.info/GC5BJGE}{BB04 Ballade du Berger - Ahusquy} — \href{http://coord.info/GC5BJGE\Number{}692556666}{202}}\cacheData{{2017/06/09 gilles64, Traditional Cache (1.5/2)}}\begin{cacheText}Point de vu extraordinaire, cache entourée de vaches et brebis. On entend que des cloches ...mplc\end{cacheText}

\cacheNumber{203}\needspace{5\baselineskip}\cacheName{\href{http://coord.info/GC5B8H3}{Dolmen d'Ite 2} — \href{http://coord.info/GC5B8H3\Number{}692556670}{203}}\cacheData{{2017/06/09 gilles64, Traditional Cache (2/1.5)}}\begin{cacheText}Mplc\end{cacheText}

\cacheNumber{204}\needspace{5\baselineskip}\cacheName{\href{http://coord.info/GC3G3RK}{GR8\Number{}95} — \href{http://coord.info/GC3G3RK\Number{}693036700}{204}}\cacheData{{2017/06/10 Peyo64, Traditional Cache (1.5/1.5)}}\begin{cacheText}Après avoir déposé le petit dernier chez ses amis je file faire quelques caches en ce début de soirée. Il m'a fallu attendre un peu que les quelques moldus passent leur chemin pour débusquer la belle !!!! L'indice et la photo m'ont bien aidé. Merci\end{cacheText}

\cacheNumber{205}\needspace{5\baselineskip}\cacheName{\href{http://coord.info/GC3G3T1}{GR8\Number{}96} — \href{http://coord.info/GC3G3T1\Number{}693038721}{205}}\cacheData{{2017/06/10 Peyo64, Traditional Cache (1.5/1.5)}}\begin{cacheText}La cache suivante est vite trouvée d'autant plus que la rue est déserte. Merci\end{cacheText}

\cacheNumber{206}\needspace{5\baselineskip}\cacheName{\href{http://coord.info/GC3G3TC}{GR8\Number{}97} — \href{http://coord.info/GC3G3TC\Number{}693041010}{206}}\cacheData{{2017/06/10 Peyo64, Traditional Cache (1.5/1.5)}}\begin{cacheText}Il s'agit de ma 3 ème cache et visiblement la dernière cache de la série : aucun indice pour la bonus sur celle la non plus! Le PZ est rapidement localisé mais la cache met du temps à se montrer. Encore faut-il chercher sous le bon angle!!!!!! Mplc\end{cacheText}

\cacheNumber{207}\needspace{5\baselineskip}\cacheName{\href{http://coord.info/GC3G81G}{GR8\Number{}98} — \href{http://coord.info/GC3G81G\Number{}693042660}{207}}\cacheData{{2017/06/10 Peyo64, Traditional Cache (2/1.5)}}\begin{cacheText}Sur la lancée j'attaque les caches du 03 GR8: celle ci est identique à celle du 01 01 GR8. Mplc\end{cacheText}

\cacheNumber{208}\needspace{5\baselineskip}\cacheName{\href{http://coord.info/GC3G81V}{GR8\Number{}99} — \href{http://coord.info/GC3G81V\Number{}693060706}{208}}\cacheData{{2017/06/10 Peyo64, Traditional Cache (2/1.5)}}\begin{cacheText}Un peu de marche et j'arrive sur le lieu de la 2 ème cache de la série. L'endroit est superbe et les arbres splendides. L'indice et la photo m'aident à débusquer la cache. Merci\end{cacheText}

\cacheNumber{209}\needspace{5\baselineskip}\cacheName{\href{http://coord.info/GC3G824}{GR8\Number{}100} — \href{http://coord.info/GC3G824\Number{}693057790}{209}}\cacheData{{2017/06/10 Peyo64, Traditional Cache (2/1.5)}}\begin{cacheText}C'est un endroit très paisible dans lequel je trouve un peu de fraîcheur par ce début de soirée. Plusieurs souches mais je trouve assez vite la cache. Merci\end{cacheText}

\cacheNumber{210}\needspace{5\baselineskip}\cacheName{\href{http://coord.info/GC3G82N}{GR8\Number{}101} — \href{http://coord.info/GC3G82N\Number{}693048087}{210}}\cacheData{{2017/06/10 Peyo64, Traditional Cache (2/1.5)}}\begin{cacheText}Endroit moins calme. La cache est vite trouvée et l'indice relevé. Merci\end{cacheText}

\cacheNumber{211}\needspace{5\baselineskip}\cacheName{\href{http://coord.info/GC5H7EV}{L'Eglise de Départ...} — \href{http://coord.info/GC5H7EV\Number{}692974751}{211}}\cacheData{{2017/06/10 plume64, Traditional Cache (1.5/1.5)}}\begin{cacheText}En route pour la 2 eme édition de la fête du lac, je ne peux pas m'empêcher de m'arrêter pour chercher cette cachee. Après avoir lu les commentaires précédents je finis par mettre la main dessus. Merci Plume64\end{cacheText}

\cacheNumber{212}\needspace{5\baselineskip}\cacheName{\href{http://coord.info/GC69YH1}{Esplanade Grancher - Cambo-les-Bains} — \href{http://coord.info/GC69YH1\Number{}693034633}{212}}\cacheData{{2017/06/10 gilles64, Traditional Cache (2/1.5)}}\begin{cacheText}Après avoir déposé le petit dernier chez ses amis pour la soirée je file faire quelques caches avant de rentrer. J'arrive sur le lieu et suis accueillie par une dizaine de chats!!!En ce début de soirée à la température très élevée on ne sent que l'odeur du jasmin ...c'est délicieux . La cache était bien dissimulée mais j'ai finis par la trouvée. Merci pour ce bon moment\end{cacheText}

\cacheNumber{213}\needspace{5\baselineskip}\cacheName{\href{http://coord.info/GC3G82Z}{GR8\Number{}102} — \href{http://coord.info/GC3G82Z\Number{}693052533}{213}}\cacheData{{2017/06/11 Peyo64, Traditional Cache (2/1.5)}}\begin{cacheText}Mplc\end{cacheText}

\cacheNumber{214}\needspace{5\baselineskip}\cacheName{\href{http://coord.info/GC3G3RD}{GR8\Number{}94} — \href{http://coord.info/GC3G3RD\Number{}693164740}{214}}\cacheData{{2017/06/11 Peyo64, Traditional Cache (1.5/1.5)}}\begin{cacheText}Mplc\end{cacheText}

\cacheNumber{215}\needspace{5\baselineskip}\cacheName{\href{http://coord.info/GC3G3R6}{GR8\Number{}93} — \href{http://coord.info/GC3G3R6\Number{}693165851}{215}}\cacheData{{2017/06/11 Peyo64, Traditional Cache (2/1.5)}}\begin{cacheText}Mplc\end{cacheText}

\cacheNumber{216}\needspace{5\baselineskip}\cacheName{\href{http://coord.info/GC3G3R1}{GR8\Number{}92} — \href{http://coord.info/GC3G3R1\Number{}693166489}{216}}\cacheData{{2017/06/11 Peyo64, Traditional Cache (1.5/1.5)}}\begin{cacheText}Mplc\end{cacheText}

\cacheNumber{217}\needspace{5\baselineskip}\cacheName{\href{http://coord.info/GC3G3QQ}{GR8\Number{}91} — \href{http://coord.info/GC3G3QQ\Number{}693167851}{217}}\cacheData{{2017/06/11 Peyo64, Traditional Cache (2/2)}}\begin{cacheText}Ciste\end{cacheText}

\cacheNumber{218}\needspace{5\baselineskip}\cacheName{\href{http://coord.info/GC3G3QC}{GR8\Number{}90} — \href{http://coord.info/GC3G3QC\Number{}693168620}{218}}\cacheData{{2017/06/11 Peyo64, Traditional Cache (2/2)}}\begin{cacheText}Mplc\end{cacheText}

\cacheNumber{219}\needspace{5\baselineskip}\cacheName{\href{http://coord.info/GC3G3Q3}{GR8\Number{}89} — \href{http://coord.info/GC3G3Q3\Number{}693169377}{219}}\cacheData{{2017/06/11 Peyo64, Traditional Cache (1.5/1.5)}}\begin{cacheText}Mplc\end{cacheText}

\cacheNumber{220}\needspace{5\baselineskip}\cacheName{\href{http://coord.info/GC3G3PV}{GR8\Number{}88} — \href{http://coord.info/GC3G3PV\Number{}693170011}{220}}\cacheData{{2017/06/11 Peyo64, Traditional Cache (2/1.5)}}\begin{cacheText}Mplc\end{cacheText}

\cacheNumber{221}\needspace{5\baselineskip}\cacheName{\href{http://coord.info/GC3G3PC}{GR8\Number{}87} — \href{http://coord.info/GC3G3PC\Number{}693171583}{221}}\cacheData{{2017/06/11 Peyo64, Traditional Cache (2/1.5)}}\begin{cacheText}Difficile mplc\end{cacheText}

\cacheNumber{222}\needspace{5\baselineskip}\cacheName{\href{http://coord.info/GC3G3P4}{GR8\Number{}86} — \href{http://coord.info/GC3G3P4\Number{}693172414}{222}}\cacheData{{2017/06/11 Peyo64, Traditional Cache (1.5/1.5)}}\begin{cacheText}Beaucoup de moldus en ce jour d'élections :\end{cacheText}

\cacheNumber{223}\needspace{5\baselineskip}\cacheName{\href{http://coord.info/GC3G3NY}{GR8\Number{}85} — \href{http://coord.info/GC3G3NY\Number{}693173462}{223}}\cacheData{{2017/06/11 Peyo64, Traditional Cache (1.5/1.5)}}\begin{cacheText}Mplc\end{cacheText}

\cacheNumber{224}\needspace{5\baselineskip}\cacheName{\href{http://coord.info/GC3G3NP}{GR8\Number{}84} — \href{http://coord.info/GC3G3NP\Number{}693178415}{224}}\cacheData{{2017/06/11 Peyo64, Traditional Cache (2/2)}}\begin{cacheText}Mplc\end{cacheText}

\cacheNumber{225}\needspace{5\baselineskip}\cacheName{\href{http://coord.info/GC3G3MC}{GR8\Number{}83} — \href{http://coord.info/GC3G3MC\Number{}693179638}{225}}\cacheData{{2017/06/11 Peyo64, Traditional Cache (1.5/2)}}\begin{cacheText}Mplc\end{cacheText}

\cacheNumber{226}\needspace{5\baselineskip}\cacheName{\href{http://coord.info/GC3G3M8}{GR8\Number{}82} — \href{http://coord.info/GC3G3M8\Number{}693180634}{226}}\cacheData{{2017/06/11 Peyo64, Traditional Cache (2/1.5)}}\begin{cacheText}Mplc\end{cacheText}

\cacheNumber{227}\needspace{5\baselineskip}\cacheName{\href{http://coord.info/GC3G3KY}{GR8\Number{}81} — \href{http://coord.info/GC3G3KY\Number{}693181323}{227}}\cacheData{{2017/06/11 Peyo64, Traditional Cache (2/1.5)}}\begin{cacheText}Mplc\end{cacheText}

\cacheNumber{228}\needspace{5\baselineskip}\cacheName{\href{http://coord.info/GC3G3K9}{GR8\Number{}78} — \href{http://coord.info/GC3G3K9\Number{}693185588}{228}}\cacheData{{2017/06/11 Peyo64, Traditional Cache (1.5/1.5)}}\begin{cacheText}Logbook neuf\end{cacheText}

\cacheNumber{229}\needspace{5\baselineskip}\cacheName{\href{http://coord.info/GC3G2YB}{GR8\Number{}71} — \href{http://coord.info/GC3G2YB\Number{}693202498}{229}}\cacheData{{2017/06/11 Peyo64, Traditional Cache (1.5/1.5)}}\begin{cacheText}Dernière cache du circuit pour aujourd'hui: le GPS ne semble pas très fiable. Je me fie à la photo et essaie de localiser l'endroit :au milieu des broussailles et des orties !!!!! Après avoir déblayé le terrain je trouve enfin le pied!!!

La cache est la!!! Merci 😊\end{cacheText}

\cacheNumber{230}\needspace{5\baselineskip}\cacheName{\href{http://coord.info/GC3G2YP}{GR8\Number{}72} — \href{http://coord.info/GC3G2YP\Number{}693199173}{230}}\cacheData{{2017/06/11 Peyo64, Traditional Cache (1.5/1.5)}}\begin{cacheText}La souche est bien la et la cache aussi! Logbook tout neuf....quelle chance!!! Merci\end{cacheText}

\cacheNumber{231}\needspace{5\baselineskip}\cacheName{\href{http://coord.info/GC3G2YV}{GR8\Number{}73} — \href{http://coord.info/GC3G2YV\Number{}693197377}{231}}\cacheData{{2017/06/11 Peyo64, Traditional Cache (1.5/1.5)}}\begin{cacheText}Arrivée sur les lieux le GPS s'affole et la souche présente ne contient pas de cache!!!!! En regardant de plus près la photo j'aperçois la bonne douche mais sur le terrain il n'y a que des herbes 🌿 hautes..... après avoir inspecter l'espace je découvre la bonne souche et la cache à moitié enterrée.....oufffff. Merci Peyo\end{cacheText}

\cacheNumber{232}\needspace{5\baselineskip}\cacheName{\href{http://coord.info/GC3G2YY}{GR8\Number{}74} — \href{http://coord.info/GC3G2YY\Number{}693194860}{232}}\cacheData{{2017/06/11 Peyo64, Traditional Cache (2/2)}}\begin{cacheText}Mplc Elle a été trouvée sans difficulté\end{cacheText}

\cacheNumber{233}\needspace{5\baselineskip}\cacheName{\href{http://coord.info/GC3G2ZB}{GR8\Number{}75} — \href{http://coord.info/GC3G2ZB\Number{}693193952}{233}}\cacheData{{2017/06/11 Peyo64, Traditional Cache (2.5/2)}}\begin{cacheText}Quelle Belle fontaine !!!! Le GPS s'affole : un coup à droite un coup à gauche. Après avoir inspecté les poutres et lu les commentaires je pense que la cache a disparu.... he bien non .... elle est bien la !!!! Merci pour cette jolie cache.\end{cacheText}

\cacheNumber{234}\needspace{5\baselineskip}\cacheName{\href{http://coord.info/GC3G3JM}{GR8\Number{}76} — \href{http://coord.info/GC3G3JM\Number{}693189738}{234}}\cacheData{{2017/06/11 Peyo64, Traditional Cache (2/2)}}\begin{cacheText}Je ne connaissais absolument pas La Chapelle Saint Sauveur. L'endroit est désert et le GPS nous guide vers les sanitaires. Avec l'indice nous trouvons la cache .....en hauteur !!!! Et comment l'atteindre ???? Logbook signé et direction La Chapelle. Merci pour la découverte de ce lieu.\end{cacheText}

\cacheNumber{235}\needspace{5\baselineskip}\cacheName{\href{http://coord.info/GC3G3K0}{GR8\Number{}77} — \href{http://coord.info/GC3G3K0\Number{}693187698}{235}}\cacheData{{2017/06/11 Peyo64, Traditional Cache (1.5/1.5)}}\begin{cacheText}Encore une fois, le GPS s'affole et on s'aide de la photo. Pas évident !!!! La végétation n'est pas la même et à beaucoup poussé mais on finit par mettre la main dessus. Merci\end{cacheText}

\cacheNumber{236}\needspace{5\baselineskip}\cacheName{\href{http://coord.info/GC3G3KD}{GR8\Number{}79} — \href{http://coord.info/GC3G3KD\Number{}693184526}{236}}\cacheData{{2017/06/11 Peyo64, Traditional Cache (1.5/1.5)}}\begin{cacheText}La cache est loguée en deux temps deux mouvements grâce à l'indice et la photo.Mplc\end{cacheText}

\cacheNumber{237}\needspace{5\baselineskip}\cacheName{\href{http://coord.info/GC3G845}{GR8\Number{}103} — \href{http://coord.info/GC3G845\Number{}693240046}{237}}\cacheData{{2017/06/11 Peyo64, Traditional Cache (2/1.5)}}\begin{cacheText}Joli petit chemin à l'ombre.... on apprécie !!!! La cache est débusquée grâce à l'indice Merci\end{cacheText}

\cacheNumber{238}\needspace{5\baselineskip}\cacheName{\href{http://coord.info/GC3G84Z}{GR8\Number{}104} — \href{http://coord.info/GC3G84Z\Number{}693237648}{238}}\cacheData{{2017/06/11 Peyo64, Traditional Cache (1.5/2)}}\begin{cacheText}Mplc\end{cacheText}

\cacheNumber{239}\needspace{5\baselineskip}\cacheName{\href{http://coord.info/GC69WFQ}{Villa Arnaga - Cambo les Bains} — \href{http://coord.info/GC69WFQ\Number{}693244423}{239}}\cacheData{{2017/06/11 gilles64, Traditional Cache (1.5/1.5)}}\begin{cacheText}Mplc\end{cacheText}

\cacheNumber{240}\needspace{5\baselineskip}\cacheName{\href{http://coord.info/GC6G54X}{LAPITZAGA} — \href{http://coord.info/GC6G54X\Number{}693646427}{240}}\cacheData{{2017/06/12 Cestasgirl, Traditional Cache (1.5/1.5)}}\begin{cacheText}En cette fin d'après-midi direction Ainharp en compagnie de chelette 64 et de Shasha. La cache est vite trouvée ( Merci Shasha).... mais le logbook est inutilisable car trempé. Nous avons signé sur un bout de papier..... mplc\end{cacheText}

\cacheNumber{241}\needspace{5\baselineskip}\cacheName{\href{http://coord.info/GC6H53B}{La fontaine Saint-Jean} — \href{http://coord.info/GC6H53B\Number{}693639315}{241}}\cacheData{{2017/06/12 Cestasgirl, Traditional Cache (1.5/1.5)}}\begin{cacheText}Après le château fort oú j'affiche un DNF j'espère trouver cette cache placée dans ce bel endroit. Chelette 64 dégote la cache en deux temps deux mouvements . Nous signons et replaçons la cache discrètement pendant qu' un moldu remplit les bouteilles à la fontaine. Merci\end{cacheText}

\cacheNumber{242}\needspace{5\baselineskip}\cacheName{\href{http://coord.info/GC6Q7MF}{\Number{}04 les passerelles de l’Aphatarena} — \href{http://coord.info/GC6Q7MF\Number{}693876791}{242}}\cacheData{{2017/06/13 matchouteam.com  cow040, Traditional Cache (2/2)}}\begin{cacheText}Après avoir validé \Number{}3 je file vers la\Number{}4 puisqu'elle a été réactivée. Je vais tenter ma chance. Grâce à l'indice nous repérons l'endroit mais on ne cherche pas du bon côté !!!! L'équilibriste entre en jeu et finis par trouver la cache!!!! Par contre pas de logbook : je signe donc un papier libre que je glisse dans la pochette. Merci 😊 pour ce bon moment.\end{cacheText}

\cacheNumber{243}\needspace{5\baselineskip}\cacheName{\href{http://coord.info/GC6QAH6}{\Number{}03 les passerelles de l’Aphatarena} — \href{http://coord.info/GC6QAH6\Number{}693876788}{243}}\cacheData{{2017/06/13 matchouteam.com, Traditional Cache (2/2)}}\begin{cacheText}Il me faut attendre une heure le bus de Bayonne....direction les passerelles de l'Aphatarena pour tenter de débusquer la Belle. Après avoir lu les commentaires je m'éloigne moi aussi duPZ et BINGO!!!!! Les précédents logeurs ont laissé le nécessaire et c'est parti.... trop rigolo on dirait une épreuve de Koh\Underscore{}Lanta 😂😂😂😂 cela mérite un point favoris merci pour cet excellent moment\end{cacheText}

\cacheNumber{244}\needspace{5\baselineskip}\cacheName{\href{http://coord.info/GC45JA1}{Le château d'Orthe} — \href{http://coord.info/GC45JA1\Number{}694110386}{244}}\cacheData{{2017/06/14 dorisbear, Traditional Cache (1.5/1.5)}}\begin{cacheText}De passage pour aller finir la série E B je m'arrête pour dénicher la Belle. Pas trop difficile grâce à l'indice. En cette fin d'après-midi il n'y a pas trop de moldus. Le logbook est en effet humide..... 😊 Merci de\end{cacheText}

\cacheNumber{245}\needspace{5\baselineskip}\cacheName{\href{http://coord.info/GC49F2A}{Baco blanc ou baco noir} — \href{http://coord.info/GC49F2A\Number{}694105148}{245}}\cacheData{{2017/06/14 dorisbear, Traditional Cache (1.5/2.5)}}\begin{cacheText}C'est mon second passage et je la débusque enfin!!!!!! Il faut prévoir les échasses !!!!!Je ne connaissais pas l'histoire de François Baco. Merci pour toutes ces informations.\end{cacheText}

\cacheNumber{246}\needspace{5\baselineskip}\cacheName{\href{http://coord.info/GC6JC60}{E.B.6. Wolpertinger} — \href{http://coord.info/GC6JC60\Number{}694084715}{246}}\cacheData{{2017/06/15 DorisBear, Traditional Cache (2/2)}}\begin{cacheText}Mplc\end{cacheText}

\cacheNumber{247}\needspace{5\baselineskip}\cacheName{\href{http://coord.info/GC6JBXH}{E.B.1. La montagne / The mountain} — \href{http://coord.info/GC6JBXH\Number{}694087776}{247}}\cacheData{{2017/06/15 DorisBear, Traditional Cache (2/3)}}\begin{cacheText}Mplc\end{cacheText}

\cacheNumber{248}\needspace{5\baselineskip}\cacheName{\href{http://coord.info/GC6JBYJ}{E.B.2. Le lac / The lake} — \href{http://coord.info/GC6JBYJ\Number{}694091808}{248}}\cacheData{{2017/06/15 DorisBear, Traditional Cache (2/2.5)}}\begin{cacheText}Mplc\end{cacheText}

\cacheNumber{249}\needspace{5\baselineskip}\cacheName{\href{http://coord.info/GC6JCAG}{E.B.7.Bonus. König Ludwig II} — \href{http://coord.info/GC6JCAG\Number{}694103268}{249}}\cacheData{{2017/06/15 DorisBear, Unknown Cache (2.5/2)}}\begin{cacheText}Enfin\end{cacheText}

\cacheNumber{250}\needspace{5\baselineskip}\cacheName{\href{http://coord.info/GC5KQ9A}{Oiseaux / Birds} — \href{http://coord.info/GC5KQ9A\Number{}694845655}{250}}\cacheData{{2017/06/17 dorisbear, Unknown Cache (2.5/1.5)}}\begin{cacheText}C est sur le retour bredouille de la bonus Muridae que je décide de chercher cette cache mystère. J avoue que je suis restée dubitative quelque temps devant cette énigme ..... et puis un jour....la lumière!!!!Comme d habitude très jolie cache. Merci pour ce bon moment\end{cacheText}

\cacheNumber{251}\needspace{5\baselineskip}\cacheName{\href{http://coord.info/GC661YH}{Mus musculus} — \href{http://coord.info/GC661YH\Number{}694804863}{251}}\cacheData{{2017/06/17 dorisbear, Traditional Cache (2/2)}}\begin{cacheText}C est la première de la série et annonce de belles caches travaillées comme je les aime.L endroit est vite repéré mais la belle me résiste ....En fait je l ai bougée sans m en rendre compte oufff j ai failli passer à coté.Je suis phobique des rongeurs et cette série est compliquée pour moi  pourtant il faut signer et relever l indice. Merci pour ce beau travail\end{cacheText}

\cacheNumber{252}\needspace{5\baselineskip}\cacheName{\href{http://coord.info/GC661YR}{Rattus norvegicus} — \href{http://coord.info/GC661YR\Number{}694806942}{252}}\cacheData{{2017/06/17 dorisbear, Traditional Cache (2/2)}}\begin{cacheText}Cette cache dans le bois est assez difficile à trouver ; elle se fond parfaitement dans le décor!!!!Un gros rat veille sur l indice mais je réussis tout de même à le noter . Merci\end{cacheText}

\cacheNumber{253}\needspace{5\baselineskip}\cacheName{\href{http://coord.info/GC661Z4}{Hydromys chrysogaster} — \href{http://coord.info/GC661Z4\Number{}694810619}{253}}\cacheData{{2017/06/17 dorisbear, Traditional Cache (2.5/2)}}\begin{cacheText}L endroit est très reposant et est habité par de petits rongeurs. L un d entre eux est bien camouflé et garde un précieux indice mais j en viens à bout ... MPLC\end{cacheText}

\cacheNumber{254}\needspace{5\baselineskip}\cacheName{\href{http://coord.info/GC661ZH}{Xeromys myoides} — \href{http://coord.info/GC661ZH\Number{}694815639}{254}}\cacheData{{2017/06/17 dorisbear, Traditional Cache (2.5/2.5)}}\begin{cacheText}Apres avoir admiré une vue superbe direction la dernière mais le GPS s affole : vraiment pas facile!!!!! A force de recherches et grâce au commentaire de Zeebrain je découvre enfin  le dernier rongeur et son indice. Merci les Dorisbear\end{cacheText}

\cacheNumber{255}\needspace{5\baselineskip}\cacheName{\href{http://coord.info/GC6CZDJ}{Ambrus  Le circuit des Totems  \Number{}7} — \href{http://coord.info/GC6CZDJ\Number{}695329514}{255}}\cacheData{{2017/06/19 jomatoju 47, Traditional Cache (1.5/1.5)}}\begin{cacheText}Mplc\end{cacheText}

\cacheNumber{256}\needspace{5\baselineskip}\cacheName{\href{http://coord.info/GC6CZDD}{Ambrus  Le circuit des Totems  \Number{}6} — \href{http://coord.info/GC6CZDD\Number{}695332086}{256}}\cacheData{{2017/06/19 jomatoju 47, Traditional Cache (1.5/1.5)}}\begin{cacheText}Mplc\end{cacheText}

\cacheNumber{257}\needspace{5\baselineskip}\cacheName{\href{http://coord.info/GC6CZ59}{Ambrus  Le circuit des Totems  \Number{}1} — \href{http://coord.info/GC6CZ59\Number{}695338392}{257}}\cacheData{{2017/06/19 jomatoju 47, Traditional Cache (1.5/1.5)}}\begin{cacheText}Des caches comme je les aime..... merci pour ce beau travail 👏👏👏\end{cacheText}

\cacheNumber{258}\needspace{5\baselineskip}\cacheName{\href{http://coord.info/GC43HJR}{CHAPELLE DE ST GERMAIN DE RIVIERE} — \href{http://coord.info/GC43HJR\Number{}695381854}{258}}\cacheData{{2017/06/19 ISOGARD Team, Traditional Cache (1/1.5)}}\begin{cacheText}En vacances aux alentours de Tonneins je décide de faire quelques caches.Le lieu est très surprenant !!! Un cimetière abandonné et un monument qui rappelle l'ancienne Chapelle ...l'endroit est paisible. La cache est vite trouvée Merci 😊\end{cacheText}

\cacheNumber{259}\needspace{5\baselineskip}\cacheName{\href{http://coord.info/GC5GHC5}{Notre Dame D AMBRUS} — \href{http://coord.info/GC5GHC5\Number{}695329481}{259}}\cacheData{{2017/06/19 jomatoju 47, Traditional Cache (1.5/1.5)}}\begin{cacheText}En vacances dans le Lot et Garonne je décide de faire ce matin tant qu'il ne fait pas trop chaud la série des Totems . Cette cache n'en fait pas parti mais elle m'appelle !!!! Je ne suis pas déçue l'endroit est très joli. La cache est vite trouvée grâce à l'indice. Merci 😊\end{cacheText}

\cacheNumber{260}\needspace{5\baselineskip}\cacheName{\href{http://coord.info/GC5Q9WG}{Chapelle Nationale des voitures anciennes} — \href{http://coord.info/GC5Q9WG\Number{}695384532}{260}}\cacheData{{2017/06/19 legolas47, Traditional Cache (1.5/2)}}\begin{cacheText}En vacances dans la région, j'en profite pour loguée quelques caches. Je n'avais jamais entendu parler de cette chapelle !!! L'endroit est désert et je peux déloger la cache sans difficulté grâce à l'indice. Merci pour la découverte de ce lieu 😊\end{cacheText}

\cacheNumber{261}\needspace{5\baselineskip}\cacheName{\href{http://coord.info/GC6CZ5M}{Ambrus  Le circuit des Totems\Number{}2} — \href{http://coord.info/GC6CZ5M\Number{}695356401}{261}}\cacheData{{2017/06/19 jomatoju 47, Traditional Cache (1.5/1.5)}}\begin{cacheText}En arrivant devant 

la maison j'ai pensé mettre trompé mais j'ai persévéré et continué à suivre le GPS!!!Encore une belle cache....indice noté.... à la suivante Merci\end{cacheText}

\cacheNumber{262}\needspace{5\baselineskip}\cacheName{\href{http://coord.info/GC6CZ5Y}{Ambrus  Le circuit des Totems  \Number{}3} — \href{http://coord.info/GC6CZ5Y\Number{}695341261}{262}}\cacheData{{2017/06/19 jomatoju 47, Traditional Cache (1.5/1.5)}}\begin{cacheText}Encore une jolie cache trouvée sans difficulté . Merci 😊\end{cacheText}

\cacheNumber{263}\needspace{5\baselineskip}\cacheName{\href{http://coord.info/GC6CZ62}{Ambrus  Le circuit des Totems  \Number{}4} — \href{http://coord.info/GC6CZ62\Number{}695345112}{263}}\cacheData{{2017/06/19 jomatoju 47, Traditional Cache (1.5/1.5)}}\begin{cacheText}Il faut trouver la cache pour comprendre l'indice ....Belle réalisation. Merci 😊\end{cacheText}

\cacheNumber{264}\needspace{5\baselineskip}\cacheName{\href{http://coord.info/GC6CZFT}{Ambrus  Le circuit des Totems  \Number{}8} — \href{http://coord.info/GC6CZFT\Number{}695360519}{264}}\cacheData{{2017/06/19 jomatoju 47, Unknown Cache (1.5/1.5)}}\begin{cacheText}En vacances dans le lot et Garonne je me laisse tenter par le circuit des Totems.... et je ne suis pas déçue par la qualité des caches. La cache bonus m'a donné plus de mal car je cherchais un endroit pouvant abriter une \Quoted{petite}. En fait il s'agit d'une XS!!!! Je l'ai trouvée grâce à l'indice. Merci pour tout ce travail 😊\end{cacheText}

\cacheNumber{265}\needspace{5\baselineskip}\cacheName{\href{http://coord.info/GC6YHP4}{Moulin de Fauillet} — \href{http://coord.info/GC6YHP4\Number{}695391674}{265}}\cacheData{{2017/06/19 zoïc, Traditional Cache (3.5/1.5)}}\begin{cacheText}En vacances aux alentours de Tonneins je m'échappe faire quelques caches sous ce soleil de plomb (37 degrés). Après avoir admiré les nénuphars je pars à la recherche de la cache qui est vite découverte ( coup de chance car elle est très très très bien dissimulée). Sur le retour je discute de ce beau moulin avec un agent communal qui me propose de le visiter !!!! Quelle chance !!!!Mplc 👌 superbe moulin\end{cacheText}

\cacheNumber{266}\needspace{5\baselineskip}\cacheName{\href{http://coord.info/GC6CZ69}{Ambrus  Le circuit des Totems  \Number{}5} — \href{http://coord.info/GC6CZ69\Number{}695347620}{266}}\cacheData{{2017/06/19 jomatoju 47, Traditional Cache (1.5/1.5)}}\begin{cacheText}Maintenant que j'y suis je comprend mieux l'indice 😂😂😂très original dans les bois !!!! Merci 😊\end{cacheText}

\cacheNumber{267}\needspace{5\baselineskip}\cacheName{\href{http://coord.info/GC5H6E1}{La fontaine bouillonnante de Caubeyre} — \href{http://coord.info/GC5H6E1\Number{}695539428}{267}}\cacheData{{2017/06/20 jomatoju 47, Traditional Cache (1.5/1.5)}}\begin{cacheText}En vacances dans le Lot et Garonne j'e profite pour faire quelques caches.La cache est vite repérée grâce à l'indice. Endroit charmant et paisible. Merci 😊\end{cacheText}

\cacheNumber{268}\needspace{5\baselineskip}\cacheName{\href{http://coord.info/GC5H764}{L’église de st Julien} — \href{http://coord.info/GC5H764\Number{}695540966}{268}}\cacheData{{2017/06/20 jomatoju 47, Traditional Cache (1.5/1.5)}}\begin{cacheText}Mplc\end{cacheText}

\cacheNumber{269}\needspace{5\baselineskip}\cacheName{\href{http://coord.info/GC5H69N}{Fargues la source du lavoir} — \href{http://coord.info/GC5H69N\Number{}695543017}{269}}\cacheData{{2017/06/20 jomatoju 47, Traditional Cache (1.5/1.5)}}\begin{cacheText}Mplc\end{cacheText}

\cacheNumber{270}\needspace{5\baselineskip}\cacheName{\href{http://coord.info/GC694F9}{Le Chateau de JEAN POTON} — \href{http://coord.info/GC694F9\Number{}695544783}{270}}\cacheData{{2017/06/20 jomatoju 47, Traditional Cache (1.5/1.5)}}\begin{cacheText}Mplc\end{cacheText}

\cacheNumber{271}\needspace{5\baselineskip}\cacheName{\href{http://coord.info/GC6CZ3V}{Eglise de Xaintrailles} — \href{http://coord.info/GC6CZ3V\Number{}695549063}{271}}\cacheData{{2017/06/20 jomatoju 47, Traditional Cache (1.5/1.5)}}\begin{cacheText}Effectivement il faut bien chercher mais qui cherche ..... trouve!!!! Mplc\end{cacheText}

\cacheNumber{272}\needspace{5\baselineskip}\cacheName{\href{http://coord.info/GC5GE25}{le site du moulin des tours} — \href{http://coord.info/GC5GE25\Number{}695558974}{272}}\cacheData{{2017/06/20 janokinou, Traditional Cache (2/2)}}\begin{cacheText}Depuis le temps que je voulais m'arrêter c est chose faite aujourd'hui !!!!Je l'ai trouvée avec difficulté .Le GPS n'est pas toujours très fiable!!! Heureusement qu'il y à l'indice. 😊 merci\end{cacheText}

\cacheNumber{273}\needspace{5\baselineskip}\cacheName{\href{http://coord.info/GC5K5Y4}{Chateau de SALLES} — \href{http://coord.info/GC5K5Y4\Number{}695578269}{273}}\cacheData{{2017/06/20 jomatoju 47, Traditional Cache (1.5/1.5)}}\begin{cacheText}Le château est superbe et je me gare sur le bas côté pour aller chercher la belle.Des lapins habitent les lieux.Jolie cache mais log humide difficile de signer  mplc 😊\end{cacheText}

\cacheNumber{274}\needspace{5\baselineskip}\cacheName{\href{http://coord.info/GC5K6EE}{EGLISE DE LIMON} — \href{http://coord.info/GC5K6EE\Number{}695576025}{274}}\cacheData{{2017/06/20 jomatoju 47, Traditional Cache (1.5/1.5)}}\begin{cacheText}Apres le cimetière direction l'église qui semble abandonnée.Je fais le tour de l'église et découvre un essaim d'abeilles je ne les dérange surtout pas et me dirige vers le monument commémoratif. La cache est débusquée grâce à l'indice. Merci 😊\end{cacheText}

\cacheNumber{275}\needspace{5\baselineskip}\cacheName{\href{http://coord.info/GC5K6ZB}{CIMETIÈRE DE LIMON} — \href{http://coord.info/GC5K6ZB\Number{}695574388}{275}}\cacheData{{2017/06/20 jomatoju 47, Traditional Cache (1.5/1.5)}}\begin{cacheText}C'est en effet un bel hommage rendu à ces hommes qui ont donné leur vie pour la liberté.Je ne connaissais pas du tout l'endroit. J'ai eu un peu de mal à trouver: GPS fantaisiste et chaleur ne m'ont pas aidé.... j'ai finalement mis la main dessus. Merci pour cette cache et pour la découverte de ce lieu 😊\end{cacheText}

\cacheNumber{276}\needspace{5\baselineskip}\cacheName{\href{http://coord.info/GC6BYZ9}{Eglises du Canton de Bouglon – 3} — \href{http://coord.info/GC6BYZ9\Number{}695619362}{276}}\cacheData{{2017/06/20 lerolay \And{} Aroukai, Traditional Cache (1/1.5)}}\begin{cacheText}Sous l'oarage et la pluie ... geocaching quand tu nous tiens!!!! Cache facile, Merci pour la découverte de cette Belle église 😃\end{cacheText}

\cacheNumber{277}\needspace{5\baselineskip}\cacheName{\href{http://coord.info/GC6BZ0B}{Eglises du Canton de Bouglon – 4} — \href{http://coord.info/GC6BZ0B\Number{}695615517}{277}}\cacheData{{2017/06/20 lerolay \And{} Aroukai, Traditional Cache (1.5/1.5)}}\begin{cacheText}En cette fin d'après-midi je décide de commencer cette série des églises.Arrivée sur les lieux je comprends vite l'indice : je regarde... je reregarde...RIEN j'étais sur le point d'abandonner lorsqu'enfin je la vois !!!!Indice relevé pour la bonus.La fatigue de la journée se fait sentir... c'est un classique du geocaching !!!!  L'orage gronde il est temps de passer à la suite.Merci pour ce bon moment\end{cacheText}

\cacheNumber{278}\needspace{5\baselineskip}\cacheName{\href{http://coord.info/GC6BZ1N}{Eglises du Canton de Bouglon – 5} — \href{http://coord.info/GC6BZ1N\Number{}695622637}{278}}\cacheData{{2017/06/20 lerolay \And{} Aroukai, Traditional Cache (3/2)}}\begin{cacheText}Très bonne cache... j'ai eu un peu de mal d'autant plus que l'orage gronde et que je ne suis pas tranquille mais je veux la trouver!!!Super .... le petit gris se décolle il faudrait l'arranger merci pour ce bon moment 😃\end{cacheText}

\cacheNumber{279}\needspace{5\baselineskip}\cacheName{\href{http://coord.info/GC6BZ2B}{Eglises du Canton de Bouglon – 6} — \href{http://coord.info/GC6BZ2B\Number{}695625984}{279}}\cacheData{{2017/06/20 lerolay \And{} Aroukai, Traditional Cache (2.5/2)}}\begin{cacheText}Encore sous la pluie et sous l'orage j'arrive à cette étrange église qui semble oubliée du monde!!!! L'endroit est très étrange : tout ces piquets qui attendent un plateau pour faire une table et ces blocs qui serviront de siège laissent supposé que beaucoup de personnes ont fréquenté les lieux!!!!! Étrange, étrange.... n'oubliez pas de jeter un œil au puit derrière le bosquet d'arbres. La cache est trouvée avec un peu de bon sens. Merci 😃\end{cacheText}

\cacheNumber{280}\needspace{5\baselineskip}\cacheName{\href{http://coord.info/GC5K713}{CHÂTEAU DE GUEYZE} — \href{http://coord.info/GC5K713\Number{}695582644}{280}}\cacheData{{2017/06/20 jomatoju 47, Traditional Cache (1.5/1.5)}}\begin{cacheText}Magnifique château..... je dégusterai bien un de leur petit 🍷 !!!! Merci pour la cache 😃\end{cacheText}

\cacheNumber{281}\needspace{5\baselineskip}\cacheName{\href{http://coord.info/GC5HWBT}{La chapelle de Calezun} — \href{http://coord.info/GC5HWBT\Number{}695587636}{281}}\cacheData{{2017/06/20 janokinou, Traditional Cache (1.5/2)}}\begin{cacheText}Mplc\end{cacheText}

\cacheNumber{282}\needspace{5\baselineskip}\cacheName{\href{http://coord.info/GC57ZMF}{Le four à pain de Montesquieu} — \href{http://coord.info/GC57ZMF\Number{}695735175}{282}}\cacheData{{2017/06/21 janokinou, Traditional Cache (2.5/1.5)}}\begin{cacheText}Quel patrimoine !!!! L'endroit est superbe..... moi qui suis férue de lecture ,vielles pierres et de geocaching je suis au paradis !!!!! La cache est vite repérée et signée. Merci 😊\end{cacheText}

\cacheNumber{283}\needspace{5\baselineskip}\cacheName{\href{http://coord.info/GC57ZK8}{La fontaine de Montesquieu} — \href{http://coord.info/GC57ZK8\Number{}695735904}{283}}\cacheData{{2017/06/21 janokinou, Traditional Cache (2/2)}}\begin{cacheText}Très belle fontaine qui permet de se rafraîchir par ces temps de canicule !!!! La cache m'attend ....Merci 😊\end{cacheText}

\cacheNumber{284}\needspace{5\baselineskip}\cacheName{\href{http://coord.info/GC5XY0X}{Vianne} — \href{http://coord.info/GC5XY0X\Number{}695740054}{284}}\cacheData{{2017/06/21 lidibrini, Traditional Cache (1.5/1.5)}}\begin{cacheText}Mini mini mini.... ne pas faire sous la pluie : Le logbook ne résistera pas!!!!Cette bastide est magnifique Merci 😊\end{cacheText}

\cacheNumber{285}\needspace{5\baselineskip}\cacheName{\href{http://coord.info/GC6GZPF}{Famille Longue-Vue} — \href{http://coord.info/GC6GZPF\Number{}695746263}{285}}\cacheData{{2017/06/21 Vallée du Lot 47, Traditional Cache (2/1.5)}}\begin{cacheText}Sans l'aide du gentil restaurateur, Je n'aurais pas trouvé , en effet le GPS n'indique pas les bonnes coordonnées et Je n'aurais pas osé aller devant les tables dressées !!!! Cache à la vue de tous .... surprenant qu'elle soit encore la!!!! Merci 😊\end{cacheText}

\cacheNumber{286}\needspace{5\baselineskip}\cacheName{\href{http://coord.info/GC4AW88}{Château de Lacaze} — \href{http://coord.info/GC4AW88\Number{}695749021}{286}}\cacheData{{2017/06/21 intuiva67, Traditional Cache (1.5/1.5)}}\begin{cacheText}Sous un soleil de plomb et après avoir lu les commentaires précédents je déniche la Belle .... le château est magnifique et son allée de platanes superbe. Merci 😊\end{cacheText}

\cacheNumber{287}\needspace{5\baselineskip}\cacheName{\href{http://coord.info/GC5AGNJ}{Le moulin de Montpeza d'agenais} — \href{http://coord.info/GC5AGNJ\Number{}695764258}{287}}\cacheData{{2017/06/21 sergeagen et Marie-flo, Traditional Cache (1/1)}}\begin{cacheText}Lieu extraordinaire et un panorama exceptionnel !!!! Sous un soleil de plomb je trouve facilement la Belle... sans démonter tout le mur . Merci 😊\end{cacheText}

\cacheNumber{288}\needspace{5\baselineskip}\cacheName{\href{http://coord.info/GC5M4MX}{Lavoir de Granges sur Lot} — \href{http://coord.info/GC5M4MX\Number{}695757118}{288}}\cacheData{{2017/06/21 brunolli, Traditional Cache (1.5/1.5)}}\begin{cacheText}Très beau cadre à l'ombre(apprécié par cette canicule)....  La cache est bien à sa place.... il faut juste un peu escalader.... Merci 😊\end{cacheText}

\cacheNumber{289}\needspace{5\baselineskip}\cacheName{\href{http://coord.info/GC5YH3H}{Famille Gabarre} — \href{http://coord.info/GC5YH3H\Number{}695759956}{289}}\cacheData{{2017/06/21 Vallée du Lot 47, Traditional Cache (1/1)}}\begin{cacheText}Superbe Hôtel de Ville et belles 🌺 Merci famille Gabarre 😃👏\end{cacheText}

\cacheNumber{290}\needspace{5\baselineskip}\cacheName{\href{http://coord.info/GC66JQ2}{Confluence du Lot et de la Garonne} — \href{http://coord.info/GC66JQ2\Number{}697041190}{290}}\cacheData{{2017/06/21 junkys, Earthcache (1.5/2)}}\begin{cacheText}Très belle découverte : un panorama exceptionnel !!!!!\end{cacheText}

\cacheNumber{291}\needspace{5\baselineskip}\cacheName{\href{http://coord.info/GC75AT5}{Le Pont du ruisseau du Pic} — \href{http://coord.info/GC75AT5\Number{}695767889}{291}}\cacheData{{2017/06/21 tatinou47, Traditional Cache (1.5/1.5)}}\begin{cacheText}Très jolie cache.... l'endroit n'attendait qu'une boîte !!!!! Merci pour la balade 😃\end{cacheText}

\cacheNumber{292}\needspace{5\baselineskip}\cacheName{\href{http://coord.info/GC75KN3}{Sur les voies du passé} — \href{http://coord.info/GC75KN3\Number{}695771375}{292}}\cacheData{{2017/06/21 tatinou47, Traditional Cache (1.5/1.5)}}\begin{cacheText}Horaires du club de randonnée du livradais affiché pour ceux que cela intéresse....Cache rapidement trouvée l'endroit est presque désert avec cette canicule!!! Merci 😊\end{cacheText}

\cacheNumber{293}\needspace{5\baselineskip}\cacheName{\href{http://coord.info/GC5GVXT}{L'église de Nomdieu} — \href{http://coord.info/GC5GVXT\Number{}695923887}{293}}\cacheData{{2017/06/22 janokinou, Traditional Cache (2/1.5)}}\begin{cacheText}Magnifique village fleuri : le lieu est superbe. Justement la commission des villages fleuris fait le tour des parterres!!!! Obligée d'attendre sous ce soleil de plomb !!!!! Enfin la cache..... pas évident de sortir le logbook mais je réussi Merci 😊 pour cette belle découverte\end{cacheText}

\cacheNumber{294}\needspace{5\baselineskip}\cacheName{\href{http://coord.info/GC596WA}{Le lavoir de Moncaut} — \href{http://coord.info/GC596WA\Number{}695927818}{294}}\cacheData{{2017/06/22 janokinou, Traditional Cache (2/1.5)}}\begin{cacheText}Splendide lavoir avec ses beaux nénuphars cache vite trouvée Merci 😊\end{cacheText}

\cacheNumber{295}\needspace{5\baselineskip}\cacheName{\href{http://coord.info/GC577C9}{L'église St Martin de Layrac, vue d'en bas !} — \href{http://coord.info/GC577C9\Number{}695956664}{295}}\cacheData{{2017/06/22 janokinou, Traditional Cache (2/2)}}\begin{cacheText}Trouvée sans difficulté..... l'attraper a été plus difficile. Merci pour la découverte de ce lieu 😃\end{cacheText}

\cacheNumber{296}\needspace{5\baselineskip}\cacheName{\href{http://coord.info/GC57EHF}{Eglise Saint Martin de Layrac, vue d'en haut} — \href{http://coord.info/GC57EHF\Number{}695953561}{296}}\cacheData{{2017/06/22 janokinou, Traditional Cache (2/1.5)}}\begin{cacheText}Très belle église cache vite repérée merci pour la découverte de ce lieu 😃\end{cacheText}

\cacheNumber{297}\needspace{5\baselineskip}\cacheName{\href{http://coord.info/GC57EPZ}{Le lavoir de Verdun à Layrac} — \href{http://coord.info/GC57EPZ\Number{}695954940}{297}}\cacheData{{2017/06/22 janokinou, Traditional Cache (1.5/1.5)}}\begin{cacheText}Lavoir très original, peu traditionnel !!!!cache vite repérée grâce à l'indice. Merci 😊\end{cacheText}

\cacheNumber{298}\needspace{5\baselineskip}\cacheName{\href{http://coord.info/GC58H9N}{Le château de Roquefort} — \href{http://coord.info/GC58H9N\Number{}695968976}{298}}\cacheData{{2017/06/22 janokinou, Traditional Cache (2/1.5)}}\begin{cacheText}Cache trouvée facilement  joli vestige merci 😊\end{cacheText}

\cacheNumber{299}\needspace{5\baselineskip}\cacheName{\href{http://coord.info/GC5EEF5}{Le 4ème lavoir de Layrac - Bords du Gers \Number{}11} — \href{http://coord.info/GC5EEF5\Number{}695960719}{299}}\cacheData{{2017/06/22 janokinou, Traditional Cache (2.5/2.5)}}\begin{cacheText}Tout d'abord il m'a fallu trouver la cache!!!!! Une bonne demi-heure !!!! En suite il m'a fallu l'attraper.... pas simple mais tout est question de détermination .... enfin je l'ai eu!!!! Merci 😊\end{cacheText}

\cacheNumber{300}\needspace{5\baselineskip}\cacheName{\href{http://coord.info/GC5KD5C}{Hommage à Elsa Cayat} — \href{http://coord.info/GC5KD5C\Number{}695974058}{300}}\cacheData{{2017/06/22 janokinou, Traditional Cache (1.5/1.5)}}\begin{cacheText}Un bien bel hommage pour ne pas oublier tous ces innocents 😭😭😭.  Merci pour la cache\end{cacheText}

\cacheNumber{301}\needspace{5\baselineskip}\cacheName{\href{http://coord.info/GC57XV8}{L'église Sainte Marie d'Aubiac} — \href{http://coord.info/GC57XV8\Number{}695930845}{301}}\cacheData{{2017/06/22 janokinou, Traditional Cache (2/1.5)}}\begin{cacheText}Encore une très belle église fortifiée cache vite trouvée grâce à l'indice Merci 😊\end{cacheText}

\cacheNumber{302}\needspace{5\baselineskip}\cacheName{\href{http://coord.info/GC57XTB}{Le vieux mur de Laplume} — \href{http://coord.info/GC57XTB\Number{}695934726}{302}}\cacheData{{2017/06/22 janokinou, Traditional Cache (2.5/1.5)}}\begin{cacheText}Cache facile grâce au spoiler vue magnifique sur la campagne environnante merci 😊\end{cacheText}

\cacheNumber{303}\needspace{5\baselineskip}\cacheName{\href{http://coord.info/GC57XP8}{L'église de Marmont-Pachas} — \href{http://coord.info/GC57XP8\Number{}695938488}{303}}\cacheData{{2017/06/22 janokinou, Traditional Cache (2/1.5)}}\begin{cacheText}Très belle église avec son clocher en pierres ....remarquable!!!! Merci pour cette cache 😃\end{cacheText}

\cacheNumber{304}\needspace{5\baselineskip}\cacheName{\href{http://coord.info/GC5EQRB}{Le lavoir des tanneries à Astaffort} — \href{http://coord.info/GC5EQRB\Number{}695942864}{304}}\cacheData{{2017/06/22 janokinou, Traditional Cache (1.5/1.5)}}\begin{cacheText}Très joli lavoir dommage qu' il ne soit pas rénové et mis en valeur. Cache vite repérée merci 😊\end{cacheText}

\cacheNumber{305}\needspace{5\baselineskip}\cacheName{\href{http://coord.info/GC57BZB}{Un lavoir d'Astaffort} — \href{http://coord.info/GC57BZB\Number{}695943954}{305}}\cacheData{{2017/06/22 janokinou, Traditional Cache (1.5/1.5)}}\begin{cacheText}Encore un autre beau lavoir dommage que l'on ne puisse pas y entrer!!!! La cache reste accessible... Merci 😊\end{cacheText}

\cacheNumber{306}\needspace{5\baselineskip}\cacheName{\href{http://coord.info/GC5ETJB}{Le moulin d'Astallort} — \href{http://coord.info/GC5ETJB\Number{}695945856}{306}}\cacheData{{2017/06/22 janokinou, Traditional Cache (2/2)}}\begin{cacheText}Je suis passée une première fois.... j'ai fait le tour du noisetier, me suis faite piquée par les orties, attaquée par les fourmis qui ont colonisé l'arbre et RIEN!!!! Je suis partie faire les 2 autres lavoirs et suis revenue encouragée par mes 2 Found.....et bingo du premier coup!!!!!  Merci 😊\end{cacheText}

\cacheNumber{307}\needspace{5\baselineskip}\cacheName{\href{http://coord.info/GC57XMM}{L'église Saint Pierre de Goulens} — \href{http://coord.info/GC57XMM\Number{}695947039}{307}}\cacheData{{2017/06/22 janokinou, Traditional Cache (1.5/1.5)}}\begin{cacheText}Des que je me suis garée j'ai deviné l'endroit de la cache belle église merci 😊\end{cacheText}

\cacheNumber{308}\needspace{5\baselineskip}\cacheName{\href{http://coord.info/GC5KT6X}{Série Hommage \And{} Découverte des bords du Gers \Number{}1} — \href{http://coord.info/GC5KT6X\Number{}695948368}{308}}\cacheData{{2017/06/22 janokinou, Traditional Cache (1.5/1.5)}}\begin{cacheText}Merci pour ce bel hommage à cet homme si courageux 😭😭😭\end{cacheText}

\cacheNumber{309}\needspace{5\baselineskip}\cacheName{\href{http://coord.info/GC57XKP}{Le lavoir de Salens à Layrac} — \href{http://coord.info/GC57XKP\Number{}695952036}{309}}\cacheData{{2017/06/22 janokinou, Traditional Cache (2/2)}}\begin{cacheText}Par ce temps de canicule il fait bon se rafraîchir au lavoir.... logbook signé mais j'ai eu du mal à le trouver!!!! Le terrain serait plutôt de 2 ou 2.5. : il faut pouvoir grimper!!!! Merci pour cette découverte 😃\end{cacheText}

\cacheNumber{310}\needspace{5\baselineskip}\cacheName{\href{http://coord.info/GC6BYVN}{Eglises du Canton de Bouglon – 1} — \href{http://coord.info/GC6BYVN\Number{}696073598}{310}}\cacheData{{2017/06/23 lerolay \And{} Aroukai, Traditional Cache (1/1.5)}}\begin{cacheText}Mplc\end{cacheText}

\cacheNumber{311}\needspace{5\baselineskip}\cacheName{\href{http://coord.info/GC6BYZ0}{Eglises du Canton de Bouglon – 2} — \href{http://coord.info/GC6BYZ0\Number{}696074174}{311}}\cacheData{{2017/06/23 lerolay \And{} Aroukai, Traditional Cache (2.5/1.5)}}\begin{cacheText}Mplc\end{cacheText}

\cacheNumber{312}\needspace{5\baselineskip}\cacheName{\href{http://coord.info/GC6BZ55}{Eglises du Canton de Bouglon – Bonus} — \href{http://coord.info/GC6BZ55\Number{}696111759}{312}}\cacheData{{2017/06/23 lerolay \And{} Aroukai, Unknown Cache (1.5/1.5)}}\begin{cacheText}Après avoir refait les calculs et avoir cherché 1 heure sur place enfin je la trouve!!!! Merci pour cette très belle série 😃\end{cacheText}

\cacheNumber{313}\needspace{5\baselineskip}\cacheName{\href{http://coord.info/GC6XFRP}{Le Parc d'Albret : ancien château de Casteljaloux} — \href{http://coord.info/GC6XFRP\Number{}696118552}{313}}\cacheData{{2017/06/23 Agex, Traditional Cache (2.5/2)}}\begin{cacheText}Très joli parc, très agréable à parcourir..... ne s'agit il pas d'un prunus???? Merci pour cette cache 😃\end{cacheText}

\cacheNumber{314}\needspace{5\baselineskip}\cacheName{\href{http://coord.info/GC72R6R}{Eglise \And{} Rives de L'Avance à Casteljaloux} — \href{http://coord.info/GC72R6R\Number{}696116762}{314}}\cacheData{{2017/06/23 Agex, Traditional Cache (2/1.5)}}\begin{cacheText}Encore un bien bel endroit très paisible: les petits ponts sont très jolis et la vue sur l'église idéale. Grâce au GPS la cache est vite repérée. Merci 😊\end{cacheText}

\cacheNumber{315}\needspace{5\baselineskip}\cacheName{\href{http://coord.info/GC5BGR6}{Eglise d Auterrive} — \href{http://coord.info/GC5BGR6\Number{}696374067}{315}}\cacheData{{2017/06/25 bixlou, Traditional Cache (1.5/1.5)}}\begin{cacheText}Il y a longtemps que je voulais faire cette cache mais connaissant le voisin et surtout sa grande curiosité j'hesitais...... Je m'en revenais de Salies lorsque je l'ai croisé en cette direction !!!!! Vite vite direction la cache que j'ai trouvé en deux temps trois mouvements. Merci 😊\end{cacheText}

\cacheNumber{316}\needspace{5\baselineskip}\cacheName{\href{http://coord.info/GC5JP1G}{Sentier botanique de la mine} — \href{http://coord.info/GC5JP1G\Number{}696760223}{316}}\cacheData{{2017/06/25 matchouteam.com, Traditional Cache (1.5/2.5)}}\begin{cacheText}L'endroit est charmant et très paisible.....je ne connaissais pas du tout le lieu. La cache est débusquée en deux temps trois mouvements grâce à l'indice très explicite. Merci pour cette découverte 😃\end{cacheText}

\cacheNumber{317}\needspace{5\baselineskip}\cacheName{\href{http://coord.info/GC5W5TK}{le lavoir d'orthevielle} — \href{http://coord.info/GC5W5TK\Number{}696770649}{317}}\cacheData{{2017/06/25 mizaga, Traditional Cache (1.5/1.5)}}\begin{cacheText}Quel joli bourg et quel joli lavoir fleuri : c'est charmant!!!! La cache est vite trouvée..... un classique !!!! Merci pour ce bon moment 😃\end{cacheText}

\cacheNumber{318}\needspace{5\baselineskip}\cacheName{\href{http://coord.info/GC6C3T5}{Happy Birthday Plume64 !!} — \href{http://coord.info/GC6C3T5\Number{}696634791}{318}}\cacheData{{2017/06/25 dorisbear, Traditional Cache (2.5/2.5)}}\begin{cacheText}Encore une jolie cache comme savent le faire les Dorisbear. Apres quelques recherches on met la main dessus, mais pas au point indiqué par le GPS!!!\end{cacheText}

\cacheNumber{319}\needspace{5\baselineskip}\cacheName{\href{http://coord.info/GC6H2GT}{Lavoir d'Orthevielle III} — \href{http://coord.info/GC6H2GT\Number{}696776276}{319}}\cacheData{{2017/06/25 mizaga, Traditional Cache (1.5/1.5)}}\begin{cacheText}Joli petit lavoir qui mériterait d'être plus mis en valeur ......La cache a été trouvée assez rapidement. Merci 😊\end{cacheText}

\cacheNumber{320}\needspace{5\baselineskip}\cacheName{\href{http://coord.info/GC6TP2V}{La Gaassoise} — \href{http://coord.info/GC6TP2V\Number{}696671597}{320}}\cacheData{{2017/06/25 Lolimomax, Traditional Cache (1.5/1.5)}}\begin{cacheText}Trouvée tres facilement malgré les moldus.... récupéré le TB Merci 😊\end{cacheText}

\cacheNumber{321}\needspace{5\baselineskip}\cacheName{\href{http://coord.info/GCV3RQ}{Lac de Luc} — \href{http://coord.info/GCV3RQ\Number{}696644381}{321}}\cacheData{{2017/06/25 f8pkp, Traditional Cache (1/1.5)}}\begin{cacheText}Cache trouvée facilement au pied de l'arbre!!!! Sur le retour nous faisons la connaissance de Valjojo qui cherche également....\end{cacheText}

\cacheNumber{322}\needspace{5\baselineskip}\cacheName{\href{http://coord.info/GC66205}{Muridae Bonus} — \href{http://coord.info/GC66205\Number{}696744441}{322}}\cacheData{{2017/06/26 dorisbear, Unknown Cache (2.5/2)}}\begin{cacheText}Après avoir recalculé au calme les coordonnées je reviens ce soir vérifier.... et BINGO je la trouve merci pour cette cache super sympa 😃\end{cacheText}

\cacheNumber{323}\needspace{5\baselineskip}\cacheName{\href{http://coord.info/GC67G6Q}{Une pour Gamboy} — \href{http://coord.info/GC67G6Q\Number{}696754018}{323}}\cacheData{{2017/06/26 dorisbear, Traditional Cache (3/1.5)}}\begin{cacheText}Mplc\end{cacheText}

\cacheNumber{324}\needspace{5\baselineskip}\cacheName{\href{http://coord.info/GC77XYV}{Un Belvédère a Labatut.} — \href{http://coord.info/GC77XYV\Number{}697615204}{324}}\cacheData{{2017/06/29 lauki3940, Traditional Cache (1.5/1.5)}}\begin{cacheText}TTF  En ce début de soirée le village est désert.  Il y a une très belle vue depuis le belvédère et la cache m'attend à sa place. L'indice aide bien à la découverte de la cache . Merci 😊\end{cacheText}

\cacheNumber{325}\needspace{5\baselineskip}\cacheName{\href{http://coord.info/GC7809K}{\Number{}1. Lac Des Glès} — \href{http://coord.info/GC7809K\Number{}697606092}{325}}\cacheData{{2017/06/29 lauki3940, Traditional Cache (1.5/1.5)}}\begin{cacheText}Après Cauneille, arrivée sur Labatut au lac des Grès. Je ne connaissais absolument pas l'endroit et c'est vraiment très joli. Le site est très calme, fréquenté par de rares pêcheurs.Les caches sont plutôt faciles et les indices aident bien. Ce circuit est parfait pour le faire avec des enfants. Je décroche 2 STF ( cache 2 et 7) et 4 TTF( cache 1,4,5 et 6). La cache 3 n'apparaissait pas sur l'application ( gros Bug) et elle vient de sortir😭😭😭😭maintenant que je suis chez moi!!!!!Merci pour l'ensemble du travail 😃\end{cacheText}

\cacheNumber{326}\needspace{5\baselineskip}\cacheName{\href{http://coord.info/GC780AA}{\Number{}2. Lac Des Grès} — \href{http://coord.info/GC780AA\Number{}697607008}{326}}\cacheData{{2017/06/29 lauki3940, Traditional Cache (1.5/1.5)}}\begin{cacheText}Après Cauneille, arrivée sur Labatut au lac des Grès. Je ne connaissais absolument pas l'endroit et c'est vraiment très joli. Le site est très calme, fréquenté par de rares pêcheurs.Les caches sont plutôt faciles et les indices aident bien. Ce circuit est parfait pour le faire avec des enfants. Je décroche 2 STF ( cache 2 et 7) et 4 TTF( cache 1,4,5 et 6). La cache 3 n'apparaissait pas sur l'application ( gros Bug) et elle vient de sortir😭😭😭😭maintenant que je suis chez moi!!!!!Merci pour l'ensemble du travail 😃\end{cacheText}

\cacheNumber{327}\needspace{5\baselineskip}\cacheName{\href{http://coord.info/GC780B8}{\Number{}4. Lac Des Glès} — \href{http://coord.info/GC780B8\Number{}697608731}{327}}\cacheData{{2017/06/29 lauki3940, Traditional Cache (1.5/1.5)}}\begin{cacheText}Après Cauneille, arrivée sur Labatut au lac des Grès. Je ne connaissais absolument pas l'endroit et c'est vraiment très joli. Le site est très calme, fréquenté par de rares pêcheurs.Les caches sont plutôt faciles et les indices aident bien. Ce circuit est parfait pour le faire avec des enfants. Je décroche 2 STF ( cache 2 et 7) et 4 TTF( cache 1,4,5 et 6). La cache 3 n'apparaissait pas sur l'application ( gros Bug) et elle vient de sortir😭😭😭😭maintenant que je suis chez moi!!!!!Merci pour l'ensemble du travail 😃\end{cacheText}

\cacheNumber{328}\needspace{5\baselineskip}\cacheName{\href{http://coord.info/GC780BH}{\Number{}5. Lac Des Glès} — \href{http://coord.info/GC780BH\Number{}697610280}{328}}\cacheData{{2017/06/29 lauki3940, Traditional Cache (1.5/1.5)}}\begin{cacheText}TTF    Après Cauneille, arrivée sur Labatut au lac des Grès. Je ne connaissais absolument pas l'endroit et c'est vraiment très joli. Le site est très calme, fréquenté par de rares pêcheurs.Les caches sont plutôt faciles et les indices aident bien. Ce circuit est parfait pour le faire avec des enfants. Je décroche 2 STF ( cache 2 et 7) et 4 TTF( cache 1,4,5 et 6). La cache 3 n'apparaissait pas sur l'application ( gros Bug) et elle vient de sortir😭😭😭😭maintenant que je suis chez moi!!!!!Merci pour l'ensemble du travail 😃\end{cacheText}

\cacheNumber{329}\needspace{5\baselineskip}\cacheName{\href{http://coord.info/GC780BV}{\Number{}6. Lac Des Glès} — \href{http://coord.info/GC780BV\Number{}697611083}{329}}\cacheData{{2017/06/29 lauki3940, Traditional Cache (1.5/1.5)}}\begin{cacheText}TTF  Après Cauneille, arrivée sur Labatut au lac des Grès. Je ne connaissais absolument pas l'endroit et c'est vraiment très joli. Le site est très calme, fréquenté par de rares pêcheurs.Les caches sont plutôt faciles et les indices aident bien. Ce circuit est parfait pour le faire avec des enfants. Je décroche 2 STF ( cache 2 et 7) et 4 TTF( cache 1,4,5 et 6). La cache 3 n'apparaissait pas sur l'application ( gros Bug) et elle vient de sortir😭😭😭😭maintenant que je suis chez moi!!!!!Merci pour l'ensemble du travail 😃\end{cacheText}

\cacheNumber{330}\needspace{5\baselineskip}\cacheName{\href{http://coord.info/GC780C1}{\Number{}7. Lac Des Glès} — \href{http://coord.info/GC780C1\Number{}697613269}{330}}\cacheData{{2017/06/29 lauki3940, Traditional Cache (1.5/1.5)}}\begin{cacheText}STF   Après Cauneille, arrivée sur Labatut au lac des Grès. Je ne connaissais absolument pas l'endroit et c'est vraiment très joli. Le site est très calme, fréquenté par de rares pêcheurs.Les caches sont plutôt faciles et les indices aident bien. Ce circuit est parfait pour le faire avec des enfants. Je décroche 2 STF ( cache 2 et 7) et 4 TTF( cache 1,4,5 et 6). La cache 3 n'apparaissait pas sur l'application ( gros Bug) et elle vient de sortir😭😭😭😭maintenant que je suis chez moi!!!!!Merci pour l'ensemble du travail 😃\end{cacheText}

\cacheNumber{331}\needspace{5\baselineskip}\cacheName{\href{http://coord.info/GC780CP}{Chateau de Cauneille} — \href{http://coord.info/GC780CP\Number{}697603082}{331}}\cacheData{{2017/06/29 lauki3940, Traditional Cache (1.5/1.5)}}\begin{cacheText}[{FTF}]

Après notre Dame d'Orthe je tente de découvrir la cache du château. Elle est facile à trouver....et surprise je suis à nouveau la première !!!!FTF sur cette cache : quel bonheur !!!! Merci Lauki3940 😃\end{cacheText}

\cacheNumber{332}\needspace{5\baselineskip}\cacheName{\href{http://coord.info/GC780DA}{Notre Dame du Pays d'Orthe} — \href{http://coord.info/GC780DA\Number{}697602408}{332}}\cacheData{{2017/06/29 lauki3940, Traditional Cache (1.5/1.5)}}\begin{cacheText}[{FTF}]

Cet après midi j'ai consulté mes mails et vu toutes ces nouvelles caches à proximité !!!Une seule idée en tête : partir apres le boulot tenter de décrocher un FTF. Arrivée sur  Cauneille vers 19 h...direction notre Dame du Pays d'Orthe. La cache est vite trouvée et .... surprise je suis la première. Merci pour la découverte de cette belle église 😃\end{cacheText}

\cacheNumber{333}\needspace{5\baselineskip}\cacheName{\href{http://coord.info/GC4T4NF}{Le mégalithe de Sainte Colombe} — \href{http://coord.info/GC4T4NF\Number{}697739570}{333}}\cacheData{{2017/06/30 vigneauetcie, Traditional Cache (1/2)}}\begin{cacheText}C'était en me rendant à Mont-de-Marsan que je décide enfin de m'arrêter pour trouver cette cache .Le dolmen est immense et la cache petite à côté !!!!Vite trouvée merci 😊\end{cacheText}

\cacheNumber{334}\needspace{5\baselineskip}\cacheName{\href{http://coord.info/GC76PG0}{Les Luso Mosellans 5 ans déjà !} — \href{http://coord.info/GC76PG0\Number{}698029398}{334}}\cacheData{{2017/07/01 senninhaN.R., Event Cache (1/1)}}\begin{cacheText}Merci à senninhaN pour l'organisation de cet évent qui nous a , en plus du plaisir procuré de rencontrer les géocacheurs locaux et d'ailleurs nous a permis d'avoir le souvenir Canada Day!!!Merci pour l'accueil très chaleureux de toute l'assemblée. J'ai récolté de nombreux indices qui vont m'aider à trouver quelques caches supplémentaires...et merci à Isaasia qui nous a donné l'astuce du dictaphone. Je pourrais ainsi faire un peu plus de commentaires qu'un simple\Quoted{MPLC}.J'ai été ravie de rencontrer Pépette . 😃\end{cacheText}

\cacheNumber{335}\needspace{5\baselineskip}\cacheName{\href{http://coord.info/GC5Z1H4}{Lieu de culte \Number{} 12 les Mormons} — \href{http://coord.info/GC5Z1H4\Number{}698378217}{335}}\cacheData{{2017/07/02 ManuMax, Traditional Cache (1.5/1.5)}}\begin{cacheText}C'est un rentrant sur Bayonne que j'aperçois un point vert sur la carte .Je m'arrête vite fait en cette fin de soirée et je cherche devant ce lieu de culte .La cache est vite trouvée grâce à l'indice.Merci 😃\end{cacheText}

\cacheNumber{336}\needspace{5\baselineskip}\cacheName{\href{http://coord.info/GC63ZFT}{Le mammouth à 3 pattes} — \href{http://coord.info/GC63ZFT\Number{}698445530}{336}}\cacheData{{2017/07/02 fdcdm, Traditional Cache (1.5/1.5)}}\begin{cacheText}Après mon cinquième passage je trouve enfin la cache grâce à l'aide de l'owner que j'ai rencontré hier lors de l'event Les Luso Mosellans 5 ans déjà !Sans lui je ne sais pas si je l'aurais trouvé : il me semblait pourtant avoir bien regardé partout!!!!il n'est pas toujours facile de chercher avec des moldus autour..  mais ça y est le point est passé au jaune. Merci 😊\end{cacheText}

\cacheNumber{337}\needspace{5\baselineskip}\cacheName{\href{http://coord.info/GC780AP}{\Number{}3. Lac Des Glès} — \href{http://coord.info/GC780AP\Number{}698230449}{337}}\cacheData{{2017/07/02 lauki3940, Traditional Cache (1.5/1.5)}}\begin{cacheText}TFT je reviens ce jour terminer le circuit du lac Des grès. La cache est rapidement repérée en descendant de la voiture merci 😊\end{cacheText}

\cacheNumber{338}\needspace{5\baselineskip}\cacheName{\href{http://coord.info/GC7857A}{île de pêche} — \href{http://coord.info/GC7857A\Number{}698241364}{338}}\cacheData{{2017/07/02 lauki3940, Traditional Cache (2.5/2.5)}}\begin{cacheText}Quelques difficultés difficultés pour arriver sur le site mais après quelques instants de recherche nous l'avons en faire en trouver merci pour la cache\end{cacheText}

\cacheNumber{339}\needspace{5\baselineskip}\cacheName{\href{http://coord.info/GC7857W}{eau de vie} — \href{http://coord.info/GC7857W\Number{}698247040}{339}}\cacheData{{2017/07/02 lauki3940, Traditional Cache (1.5/1.5)}}\begin{cacheText}TTF  La cache porte bien son nom .... aucun doute sur l'endroit où se trouve le trésor .Nous trouvons rapidement la cache et partons sur Saint Cricq du Gave ....merci 😃\end{cacheText}

\cacheNumber{340}\needspace{5\baselineskip}\cacheName{\href{http://coord.info/GC7858M}{le chateau} — \href{http://coord.info/GC7858M\Number{}698251815}{340}}\cacheData{{2017/07/02 lauki3940, Traditional Cache (1.5/2.5)}}\begin{cacheText}Arrivée devant ce superbe château je trouve la cache assez facilement grâce à l'indice Merci 😊\end{cacheText}

\cacheNumber{341}\needspace{5\baselineskip}\cacheName{\href{http://coord.info/GC7859E}{belvédère du lac} — \href{http://coord.info/GC7859E\Number{}698254461}{341}}\cacheData{{2017/07/02 lauki3940, Traditional Cache (2/2)}}\begin{cacheText}Je ne connaissais pas ce lac ...endroit très charmant!!! arrivée sur place j'ai vite trouvé la cache merci 😊\end{cacheText}

\cacheNumber{342}\needspace{5\baselineskip}\cacheName{\href{http://coord.info/GC785AR}{pont du gave de Pau} — \href{http://coord.info/GC785AR\Number{}698258984}{342}}\cacheData{{2017/07/02 lauki3940, Traditional Cache (1.5/3)}}\begin{cacheText}TTF  Arrivés sur les lieux nous allons tout d'abord en bas mais le GPS pointe 25 m vers l'eau!!!!puis nous remontons après avoir inspecté le long du pont enfin je découvre la belle bien cachée!!!bravo Merci 😊\end{cacheText}

\cacheNumber{343}\needspace{5\baselineskip}\cacheName{\href{http://coord.info/GC785CX}{lavoir de Saint-Cricq-du-Gave} — \href{http://coord.info/GC785CX\Number{}698265182}{343}}\cacheData{{2017/07/02 lauki3940, Traditional Cache (3/1.5)}}\begin{cacheText}TTF Charmant petit lavoir !Des que  nous sommes arrivés , nous avons trouvé ..... une Cyste!!!!Attention !!!!Nous avons repris nos recherches et enfin  ....le trésor. En effet ,il ne faut pas toucher aux tuiles .Merci pour la cache 😃\end{cacheText}

\cacheNumber{344}\needspace{5\baselineskip}\cacheName{\href{http://coord.info/GC3G859}{GR8\Number{}105} — \href{http://coord.info/GC3G859\Number{}698556267}{344}}\cacheData{{2017/07/03 Peyo64, Traditional Cache (2/2)}}\begin{cacheText}En ce jour de repos nous décidons de reprendre le GR huit la cacheest trouvée facilement: l'indice est très explicite merci 😃\end{cacheText}

\cacheNumber{345}\needspace{5\baselineskip}\cacheName{\href{http://coord.info/GC2ACAT}{Le mont Urzumu} — \href{http://coord.info/GC2ACAT\Number{}698558091}{345}}\cacheData{{2017/07/03 Peyo64, Traditional Cache (1.5/1.5)}}\begin{cacheText}Après cette ascension une pause s'impose merci pour le point de vue merci pour la cache 😃\end{cacheText}

\cacheNumber{346}\needspace{5\baselineskip}\cacheName{\href{http://coord.info/GC3G86Q}{GR8\Number{}109} — \href{http://coord.info/GC3G86Q\Number{}698560527}{346}}\cacheData{{2017/07/03 Peyo64, Traditional Cache (2/1.5)}}\begin{cacheText}La cache est très rapidement trouver grâce à l'indice la vue est magnifique merci\end{cacheText}

\cacheNumber{347}\needspace{5\baselineskip}\cacheName{\href{http://coord.info/GC3G85R}{GR8\Number{}107} — \href{http://coord.info/GC3G85R\Number{}698564078}{347}}\cacheData{{2017/07/03 Peyo64, Traditional Cache (1.5/2)}}\begin{cacheText}Arrivé sur le lieu on ne trouve pas dommage premier DNFde la journée 😭😭😭😭\end{cacheText}

\cacheNumber{348}\needspace{5\baselineskip}\cacheName{\href{http://coord.info/GC3G85G}{GR8\Number{}106} — \href{http://coord.info/GC3G85G\Number{}698565066}{348}}\cacheData{{2017/07/03 Peyo64, Traditional Cache (2/1.5)}}\begin{cacheText}Après cette ascension nous trouvons la cache assez rapidement  grâce à l'indice mais pas tout à fait au sommet !!!!!merci 😊\end{cacheText}

\cacheNumber{349}\needspace{5\baselineskip}\cacheName{\href{http://coord.info/GC3G86T}{GR8\Number{}110} — \href{http://coord.info/GC3G86T\Number{}698568292}{349}}\cacheData{{2017/07/03 Peyo64, Traditional Cache (2/1.5)}}\begin{cacheText}Encore une..... plutôt facile car placée à 1m50 Merci 😊\end{cacheText}

\cacheNumber{350}\needspace{5\baselineskip}\cacheName{\href{http://coord.info/GC3G872}{GR8\Number{}111} — \href{http://coord.info/GC3G872\Number{}698569086}{350}}\cacheData{{2017/07/03 Peyo64, Traditional Cache (2/1.5)}}\begin{cacheText}La cache a été remplacée merci 😊\end{cacheText}

\cacheNumber{351}\needspace{5\baselineskip}\cacheName{\href{http://coord.info/GC3G87G}{GR8\Number{}112} — \href{http://coord.info/GC3G87G\Number{}698569691}{351}}\cacheData{{2017/07/03 Peyo64, Traditional Cache (1.5/1.5)}}\begin{cacheText}Cache trouvée mais elle a été remplacée merci 😊\end{cacheText}

\cacheNumber{352}\needspace{5\baselineskip}\cacheName{\href{http://coord.info/GC3G881}{GR8\Number{}113} — \href{http://coord.info/GC3G881\Number{}698571262}{352}}\cacheData{{2017/07/03 Peyo64, Traditional Cache (1.5/1.5)}}\begin{cacheText}Un DNF ...zut mais on cherche quand même et ....on trouve !!!!! Le Log book est trempé je prends la photo pour preuve merci😃\end{cacheText}

\cacheNumber{353}\needspace{5\baselineskip}\cacheName{\href{http://coord.info/GC3DYMH}{Espelette 1} — \href{http://coord.info/GC3DYMH\Number{}698657405}{353}}\cacheData{{2017/07/03 Peyo64, Traditional Cache (2/1.5)}}\begin{cacheText}Sous l'œil expert de l'owner Peyo 64 nous trouvons le belle. Merci 😊\end{cacheText}

\cacheNumber{354}\needspace{5\baselineskip}\cacheName{\href{http://coord.info/GC3DYMR}{Espelette 2} — \href{http://coord.info/GC3DYMR\Number{}698653713}{354}}\cacheData{{2017/07/03 Peyo64, Traditional Cache (1.5/1.5)}}\begin{cacheText}En compagnie de Peyo 64 nous avons trouvé Merci 😊\end{cacheText}

\cacheNumber{355}\needspace{5\baselineskip}\cacheName{\href{http://coord.info/GC3E99V}{Ezpeleta : ITZULIA TTIKIA / Circuit des familles} — \href{http://coord.info/GC3E99V\Number{}698580296}{355}}\cacheData{{2017/07/03 TEAM rat\Underscore{}zmot, Traditional Cache (1.5/2)}}\begin{cacheText}Nous avons entrepris ce matin une partie du GR huit, en allant chercher de quoi se restaurer nous voyons ce petit point vert sur la carte nous trouvons très rapidement cette très jolie cache et nous laissons un objet voyageur merci😃\end{cacheText}

\cacheNumber{356}\needspace{5\baselineskip}\cacheName{\href{http://coord.info/GC3G88F}{GR8\Number{}114} — \href{http://coord.info/GC3G88F\Number{}698574657}{356}}\cacheData{{2017/07/03 Peyo64, Traditional Cache (2/1.5)}}\begin{cacheText}Arrivé sur les lieux  nous mettons un peu de temps pour la trouver mais elle est là .Elle nous attend :elle a été également réparée par  Spirou 43 et Poune 43 merci 😊\end{cacheText}

\cacheNumber{357}\needspace{5\baselineskip}\cacheName{\href{http://coord.info/GC3G88T}{GR8\Number{}115} — \href{http://coord.info/GC3G88T\Number{}698576064}{357}}\cacheData{{2017/07/03 Peyo64, Traditional Cache (2/1.5)}}\begin{cacheText}La cache est toujours en hauteur elle nous attend 😊\end{cacheText}

\cacheNumber{358}\needspace{5\baselineskip}\cacheName{\href{http://coord.info/GC3G892}{GR8\Number{}116} — \href{http://coord.info/GC3G892\Number{}698599034}{358}}\cacheData{{2017/07/03 Peyo64, Traditional Cache (2.5/2)}}\begin{cacheText}C'est une cache qui nous a donné du fil à retorde ah ah ah....mais nous avons fini par mettre la main dessus ,ne pas se décourager la cachette est toujours en place !!!!merci 😊\end{cacheText}

\cacheNumber{359}\needspace{5\baselineskip}\cacheName{\href{http://coord.info/GC3G89J}{GR8\Number{}117} — \href{http://coord.info/GC3G89J\Number{}698603685}{359}}\cacheData{{2017/07/03 Peyo64, Traditional Cache (1.5/1.5)}}\begin{cacheText}le spoiler reste d'une aide précieuse. Merci 😊Arrivé sur les lieux nous retournons toutes les pierres rien!!! nous commençons à nous décourager et puis finalement BINGO elle est là !!!Le spoiler reste d'une aide précieuse. Merci 😊\end{cacheText}

\cacheNumber{360}\needspace{5\baselineskip}\cacheName{\href{http://coord.info/GC3G8A1}{GR8\Number{}118} — \href{http://coord.info/GC3G8A1\Number{}698603040}{360}}\cacheData{{2017/07/03 Peyo64, Traditional Cache (1.5/2)}}\begin{cacheText}La boîte nous attend nichée entre les cinq bras. Un moldu passe la debrouissalleuse et nous signons discrètement.Merci 😊\end{cacheText}

\cacheNumber{361}\needspace{5\baselineskip}\cacheName{\href{http://coord.info/GC3G8AC}{GR8\Number{}119} — \href{http://coord.info/GC3G8AC\Number{}698607162}{361}}\cacheData{{2017/07/03 Peyo64, Traditional Cache (1.5/2)}}\begin{cacheText}L'ascension sur la digestion en terrain boueux  est très difficile mais malgré ça nous la dénichons merci 😊\end{cacheText}

\cacheNumber{362}\needspace{5\baselineskip}\cacheName{\href{http://coord.info/GC3G8CM}{GR8\Number{}121} — \href{http://coord.info/GC3G8CM\Number{}698616209}{362}}\cacheData{{2017/07/03 Peyo64, Traditional Cache (1.5/1.5)}}\begin{cacheText}Cache trouvée très facilement le plus dur a été de la décoincer . Le visiteur précédent l'avait bien coincé !!!!merci 😊\end{cacheText}

\cacheNumber{363}\needspace{5\baselineskip}\cacheName{\href{http://coord.info/GC3G8CY}{GR8\Number{}122} — \href{http://coord.info/GC3G8CY\Number{}698617950}{363}}\cacheData{{2017/07/03 Peyo64, Traditional Cache (2/2)}}\begin{cacheText}Nous arrivons sur les lieux et nous découvrons un Logbook tout neuf .... Merci 😊\end{cacheText}

\cacheNumber{364}\needspace{5\baselineskip}\cacheName{\href{http://coord.info/GC3G8D5}{GR8\Number{}123} — \href{http://coord.info/GC3G8D5\Number{}698633236}{364}}\cacheData{{2017/07/03 Peyo64, Traditional Cache (2/1.5)}}\begin{cacheText}Elle nous attend à l'endroit indiqué merci 😊\end{cacheText}

\cacheNumber{365}\needspace{5\baselineskip}\cacheName{\href{http://coord.info/GC3G8DF}{GR8\Number{}124} — \href{http://coord.info/GC3G8DF\Number{}698628020}{365}}\cacheData{{2017/07/03 Peyo64, Traditional Cache (1.5/1.5)}}\begin{cacheText}Cache bien en place Merci 😊\end{cacheText}

\cacheNumber{366}\needspace{5\baselineskip}\cacheName{\href{http://coord.info/GC3G8DW}{GR8\Number{}125} — \href{http://coord.info/GC3G8DW\Number{}698627867}{366}}\cacheData{{2017/07/03 Peyo64, Traditional Cache (1.5/1.5)}}\begin{cacheText}Sans problème, La cache est au poteau Merci 😊\end{cacheText}

\cacheNumber{367}\needspace{5\baselineskip}\cacheName{\href{http://coord.info/GC3G8E2}{GR8\Number{}126} — \href{http://coord.info/GC3G8E2\Number{}698643857}{367}}\cacheData{{2017/07/03 Peyo64, Traditional Cache (2/2)}}\begin{cacheText}Le cache est trouvée vite fait.... un grand classique du jeu merci 😊\end{cacheText}

\cacheNumber{368}\needspace{5\baselineskip}\cacheName{\href{http://coord.info/GC3G8FJ}{GR8\Number{}128} — \href{http://coord.info/GC3G8FJ\Number{}698637358}{368}}\cacheData{{2017/07/03 Peyo64, Traditional Cache (1.5/1.5)}}\begin{cacheText}La cache est trouvée sans problème,le paysage est magnifique merci 😊\end{cacheText}

\cacheNumber{369}\needspace{5\baselineskip}\cacheName{\href{http://coord.info/GC3G8G1}{GR8\Number{}129} — \href{http://coord.info/GC3G8G1\Number{}698638762}{369}}\cacheData{{2017/07/03 Peyo64, Traditional Cache (2/1.5)}}\begin{cacheText}Cache pas évidente

à trouver mais on a la debusque quand même !!! La vue est toujours aussi belle merci 😊\end{cacheText}

\cacheNumber{370}\needspace{5\baselineskip}\cacheName{\href{http://coord.info/GC3G8GJ}{GR8\Number{}130} — \href{http://coord.info/GC3G8GJ\Number{}698640255}{370}}\cacheData{{2017/07/03 Peyo64, Traditional Cache (1.5/2)}}\begin{cacheText}La cache Nous attend sous l'escalier. On a du d'abord se frayer un chemin dans la fougère....Merci 😃\end{cacheText}

\cacheNumber{371}\needspace{5\baselineskip}\cacheName{\href{http://coord.info/GC3G8H2}{GR8\Number{}131} — \href{http://coord.info/GC3G8H2\Number{}698641717}{371}}\cacheData{{2017/07/03 Peyo64, Traditional Cache (2/1.5)}}\begin{cacheText}La cache est bien à sa place : Le spoiler est très explicite !!!! Merci 😊\end{cacheText}

\cacheNumber{372}\needspace{5\baselineskip}\cacheName{\href{http://coord.info/GC78AE9}{le chateau abandonné} — \href{http://coord.info/GC78AE9\Number{}698703067}{372}}\cacheData{{2017/07/03 lauki3940, Traditional Cache (2/1.5)}}\begin{cacheText}[TTF] C'est au retour d'une journée geocaching sur le GR8 de Peyo 64 que j'ai consulté mes mails et que j'ai vu de nouvelles caches pas très loin de la maison!!! La tentation est trop grande ...je file à 19h40 vers les belles  on mangera plus tard!!!! J'arrive trop tard pour le FTF ... Merci 😊\end{cacheText}

\cacheNumber{373}\needspace{5\baselineskip}\cacheName{\href{http://coord.info/GC78AER}{pont de la marquise} — \href{http://coord.info/GC78AER\Number{}698753657}{373}}\cacheData{{2017/07/03 lauki3940, Traditional Cache (1.5/2)}}\begin{cacheText}{TTF} Super passerelle cache vite trouvée grâce à l'indice Merci 😊\end{cacheText}

\cacheNumber{374}\needspace{5\baselineskip}\cacheName{\href{http://coord.info/GC78AF4}{lavoir de Lahontan} — \href{http://coord.info/GC78AF4\Number{}698756270}{374}}\cacheData{{2017/07/03 lauki3940, Traditional Cache (3/1.5)}}\begin{cacheText}{STF} Arrivés sur les lieux je reconnais une silhouette familière : Isa asia est entrain de chercher !!!! Nous trouvons ensemble la jolie cache . Merci 😊\end{cacheText}

\cacheNumber{375}\needspace{5\baselineskip}\cacheName{\href{http://coord.info/GC78AFK}{la grande traverssée du gave de Pau} — \href{http://coord.info/GC78AFK\Number{}698756809}{375}}\cacheData{{2017/07/03 lauki3940, Traditional Cache (2/1.5)}}\begin{cacheText}{TTF} En compagnie de Isa asia nous continuons la recherche des petites boîtes 📦.... celle ci est vite trouvée Merci 😊\end{cacheText}

\cacheNumber{376}\needspace{5\baselineskip}\cacheName{\href{http://coord.info/GC78AFZ}{labatut vieux village} — \href{http://coord.info/GC78AFZ\Number{}698757300}{376}}\cacheData{{2017/07/03 lauki3940, Traditional Cache (1.5/1)}}\begin{cacheText}[(STF)] Nous arrivons sur les lieux avec Isa asia et loguons sans problème la belle. Merci 😃\end{cacheText}

\cacheNumber{377}\needspace{5\baselineskip}\cacheName{\href{http://coord.info/GC78AGC}{maison ancienne} — \href{http://coord.info/GC78AGC\Number{}698757648}{377}}\cacheData{{2017/07/03 lauki3940, Traditional Cache (2/1.5)}}\begin{cacheText}[(TTF)] Nous terminons la série des caches parues ce jour sans difficulté. Isa asia signe en seconde position et nous en troisième.Il est presque 21h00 maintenant À Table!!!! Merci 😊\end{cacheText}

\cacheNumber{378}\needspace{5\baselineskip}\cacheName{\href{http://coord.info/GC5JY0K}{Source de Bidas / Spring of Bidas} — \href{http://coord.info/GC5JY0K\Number{}700133839}{378}}\cacheData{{2017/07/09 dorisbear, Traditional Cache (2.5/2.5)}}\begin{cacheText}Encore un endroit que je ne connaissais pas ....réalisation trouvée grâce au coordonnées précises.merci pour la cache 😃\end{cacheText}

\cacheNumber{379}\needspace{5\baselineskip}\cacheName{\href{http://coord.info/GC66245}{38. Kereru  -  New Zealand Wood Pigeon} — \href{http://coord.info/GC66245\Number{}700137925}{379}}\cacheData{{2017/07/09 DorisBear, Traditional Cache (2.5/2)}}\begin{cacheText}C'est un bel oiseau que voilà !!!! Ainsi que son nichoir . Encore une belle réalisation qui nécessite un peu d'agilité pour l'ouverture !!! merci 😊\end{cacheText}

\cacheNumber{380}\needspace{5\baselineskip}\cacheName{\href{http://coord.info/GC66249}{39. Kiwi} — \href{http://coord.info/GC66249\Number{}700130426}{380}}\cacheData{{2017/07/09 DorisBear, Traditional Cache (2/2)}}\begin{cacheText}Encore une cache magnifique comme je les aime.... Merci 😊\end{cacheText}

\cacheNumber{381}\needspace{5\baselineskip}\cacheName{\href{http://coord.info/GC6NG1D}{Le lavoir de Mimbaste} — \href{http://coord.info/GC6NG1D\Number{}700171627}{381}}\cacheData{{2017/07/09 Opmb40, Traditional Cache (3.5/2)}}\begin{cacheText}Contrairement au viaduc , on trouve  la cache en arrivant!!!! Le logbook est tres humide, il faudrait prévoir un plastique pour le protéger. Merci 😊\end{cacheText}

\cacheNumber{382}\needspace{5\baselineskip}\cacheName{\href{http://coord.info/GC6PB9G}{L'église de MIMBASTE} — \href{http://coord.info/GC6PB9G\Number{}700146483}{382}}\cacheData{{2017/07/09 Opmb40, Traditional Cache (2/1)}}\begin{cacheText}Cache originale,trouvée sans difficulté grâce à l'indice. Merci 😊\end{cacheText}

\cacheNumber{383}\needspace{5\baselineskip}\cacheName{\href{http://coord.info/GC6PBDW}{Le viaduc de MIMBASTE} — \href{http://coord.info/GC6PBDW\Number{}700185220}{383}}\cacheData{{2017/07/09 Opmb40, Traditional Cache (2/2)}}\begin{cacheText}Nous avons mis 3/4 d'heure pour la découvrir : heureusement que la moitié est repassé sur mes pas pour la découvrir !!! Et heureusement que je connaissais le système pour atteindre le logbook !!! En effet, tout est dans Le texte et heureusement que j'avais, comme Clavitos , les commentaires précédents. Merci 😊\end{cacheText}

\cacheNumber{384}\needspace{5\baselineskip}\cacheName{\href{http://coord.info/GC725PE}{GTAQ3-26\Number{}26 Clermont-Ozourt\Number{}nature} — \href{http://coord.info/GC725PE\Number{}700513183}{384}}\cacheData{{2017/07/10 crispol40, Traditional Cache (1.5/1.5)}}\begin{cacheText}L'indice est très explicite.... l'endroit est très calme!!!! Merci 😊\end{cacheText}

\cacheNumber{385}\needspace{5\baselineskip}\cacheName{\href{http://coord.info/GC725PP}{GTAQ3-26\Number{}27 Clermont-Ozourt\Number{}Parcabe} — \href{http://coord.info/GC725PP\Number{}700511094}{385}}\cacheData{{2017/07/10 crispol40, Traditional Cache (1.5/1.5)}}\begin{cacheText}À l ombre du maïs la cache nous attend.... Merci 😊\end{cacheText}

\cacheNumber{386}\needspace{5\baselineskip}\cacheName{\href{http://coord.info/GC725PV}{GTAQ3-26\Number{}28 Clermont-Ozourt\Number{}croisement} — \href{http://coord.info/GC725PV\Number{}700510481}{386}}\cacheData{{2017/07/10 crispol40, Traditional Cache (1.5/1.5)}}\begin{cacheText}Epareuse passée mais la cache nous attend Merci 😊\end{cacheText}

\cacheNumber{387}\needspace{5\baselineskip}\cacheName{\href{http://coord.info/GC725PZ}{GTAQ3-26\Number{}29 Clermont-Ozourt\Number{}troll} — \href{http://coord.info/GC725PZ\Number{}700508984}{387}}\cacheData{{2017/07/10 crispol40, Traditional Cache (1.5/1.5)}}\begin{cacheText}La cache est remontée en rappel du gouffre.... 😊 merci\end{cacheText}

\cacheNumber{388}\needspace{5\baselineskip}\cacheName{\href{http://coord.info/GC725Q1}{GTAQ3-26\Number{}30 Clermont-Ozourt\Number{}Boussat} — \href{http://coord.info/GC725Q1\Number{}700507640}{388}}\cacheData{{2017/07/10 crispol40, Traditional Cache (1.5/1.5)}}\begin{cacheText}Près avoir inspecté tous les piquets en ciment on la découvre enfin!!! Merci 😃\end{cacheText}

\cacheNumber{389}\needspace{5\baselineskip}\cacheName{\href{http://coord.info/GC726B0}{GTAQ3-26\Number{}31 Clermont-Ozourt\Number{}Dailleux} — \href{http://coord.info/GC726B0\Number{}700505809}{389}}\cacheData{{2017/07/10 crispol40, Traditional Cache (1.5/1.5)}}\begin{cacheText}Ici l indice nous a bien aidé merci 😊\end{cacheText}

\cacheNumber{390}\needspace{5\baselineskip}\cacheName{\href{http://coord.info/GC726B9}{GTAQ3-26\Number{}33 Clermont-Ozourt\Number{}stop 1} — \href{http://coord.info/GC726B9\Number{}700495476}{390}}\cacheData{{2017/07/10 crispol40, Traditional Cache (1.5/1.5)}}\begin{cacheText}Nous avons fait une halte et avons trouvé sans difficulté !!! Merci 😊\end{cacheText}

\cacheNumber{391}\needspace{5\baselineskip}\cacheName{\href{http://coord.info/GC726BF}{GTAQ3-26\Number{}34 Clermont-Ozourt\Number{}stop 2} — \href{http://coord.info/GC726BF\Number{}700494128}{391}}\cacheData{{2017/07/10 crispol40, Traditional Cache (1.5/1.5)}}\begin{cacheText}Malgré la circulation cache trouvée en deux temps trois mouvements !!!! Merci 😊\end{cacheText}

\cacheNumber{392}\needspace{5\baselineskip}\cacheName{\href{http://coord.info/GC726BK}{GTAQ3-26\Number{}35 Clermont-Ozourt\Number{}Sarrail} — \href{http://coord.info/GC726BK\Number{}700491310}{392}}\cacheData{{2017/07/10 crispol40, Traditional Cache (2/1.5)}}\begin{cacheText}La nature reprend toujours ses droits ...la cache est en place .merci 😊\end{cacheText}

\cacheNumber{393}\needspace{5\baselineskip}\cacheName{\href{http://coord.info/GC726BV}{GTAQ3-26\Number{}36 Clermont-Ozourt\Number{}angle droit} — \href{http://coord.info/GC726BV\Number{}700489506}{393}}\cacheData{{2017/07/10 crispol40, Traditional Cache (1.5/1.5)}}\begin{cacheText}Ici aussi la cache l'a échappé belle : l'épareuse a nettoyé  les bas-côtés !!!!En cherchant bien on la trouve parterre .Nous la remettons en place  ( dans ce qu'il reste du buisson qui fait aïe aïe aïe) Merci 😊\end{cacheText}

\cacheNumber{394}\needspace{5\baselineskip}\cacheName{\href{http://coord.info/GC726BW}{GTAQ3-26\Number{}37 Clermont-Ozourt\Number{}Lauga} — \href{http://coord.info/GC726BW\Number{}700487926}{394}}\cacheData{{2017/07/10 crispol40, Traditional Cache (1.5/3)}}\begin{cacheText}Les joncs ont été broyés :la cache est  toujours en place mais de justesse !!!!nous l'avons camouflée tant bien que mal merci 😊\end{cacheText}

\cacheNumber{395}\needspace{5\baselineskip}\cacheName{\href{http://coord.info/GC726BZ}{GTAQ3-26\Number{}38 Clermont-Ozourt\Number{}terre} — \href{http://coord.info/GC726BZ\Number{}700483533}{395}}\cacheData{{2017/07/10 crispol40, Traditional Cache (1.5/1.5)}}\begin{cacheText}Balade très agréable dans ce sous-bois la cache nous attend bien à sa place !!!! merci 😊\end{cacheText}

\cacheNumber{396}\needspace{5\baselineskip}\cacheName{\href{http://coord.info/GC726C3}{GTAQ3-26\Number{}39 Clermont-Ozourt\Number{}terre 2} — \href{http://coord.info/GC726C3\Number{}700481848}{396}}\cacheData{{2017/07/10 crispol40, Traditional Cache (1.5/1.5)}}\begin{cacheText}La cache est bien à sa place elle est rapidement trouvée grâce à l'indice merci 😊\end{cacheText}

\cacheNumber{397}\needspace{5\baselineskip}\cacheName{\href{http://coord.info/GC72GQR}{GTAQ3-26\Number{}02Clermont-Ozourt\Number{} Marie Madeleine} — \href{http://coord.info/GC72GQR\Number{}700476609}{397}}\cacheData{{2017/07/10 crispol40, Traditional Cache (1.5/1.5)}}\begin{cacheText}Le lieu n'a rien de sacré effectivement !!!Cache trouvée grâce à l'aide l'owner rencontré lors de l'event les Luso Mossellan 5 ans déjà .....merci 😃\end{cacheText}

\cacheNumber{398}\needspace{5\baselineskip}\cacheName{\href{http://coord.info/GC73W3J}{La cache à -MAMA-} — \href{http://coord.info/GC73W3J\Number{}700514257}{398}}\cacheData{{2017/07/10 Opmb40, Traditional Cache (1.5/1.5)}}\begin{cacheText}Merci Maël pour cette jolie cache.... l'indice qui nous a fait rire, nous a bien aidé 😃\end{cacheText}

\cacheNumber{399}\needspace{5\baselineskip}\cacheName{\href{http://coord.info/GC75W3C}{Le panneau de MIMBASTE 2} — \href{http://coord.info/GC75W3C\Number{}700518232}{399}}\cacheData{{2017/07/10 Opmb40, Traditional Cache (2/1.5)}}\begin{cacheText}Gros gros coup de coeur ❤️ pour cette magnifique cache. Un PF pour l'originalité et la conception  du vrai geocaching Merci 😊\end{cacheText}

\cacheNumber{400}\needspace{5\baselineskip}\cacheName{\href{http://coord.info/GC71ZKB}{GTAQ3-26\Number{}14 Clermont-Ozourt\Number{}chicane} — \href{http://coord.info/GC71ZKB\Number{}700712225}{400}}\cacheData{{2017/07/11 crispol40, Traditional Cache (1.5/1.5)}}\begin{cacheText}La cache nous attend bien sagement dans la souche :quel plaisir de se promener dans ses sous bois !!!!merci 😊\end{cacheText}

\cacheNumber{401}\needspace{5\baselineskip}\cacheName{\href{http://coord.info/GC71ZKT}{GTAQ3-26\Number{}15 Clermont-Ozourt\Number{}piste 1} — \href{http://coord.info/GC71ZKT\Number{}700713133}{401}}\cacheData{{2017/07/11 crispol40, Traditional Cache (1.5/1.5)}}\begin{cacheText}Monsieur ,arrivé le premier sur les lieux ,trouve la cachette sans difficulté .Merci 😃\end{cacheText}

\cacheNumber{402}\needspace{5\baselineskip}\cacheName{\href{http://coord.info/GC71ZMA}{GTAQ3-26\Number{}16 Clermont-Ozourt\Number{}piste 2} — \href{http://coord.info/GC71ZMA\Number{}700720059}{402}}\cacheData{{2017/07/11 crispol40, Traditional Cache (1.5/1.5)}}\begin{cacheText}Après avoir demandé l'autorisation à un gentil monsieur de passer chez lui nous trouvons la cache dans le bois.Merci 😊\end{cacheText}

\cacheNumber{403}\needspace{5\baselineskip}\cacheName{\href{http://coord.info/GC721ZH}{GTAQ3-26\Number{}18 Clermont-Ozourt\Number{}L' entrée} — \href{http://coord.info/GC721ZH\Number{}700717308}{403}}\cacheData{{2017/07/11 crispol40, Traditional Cache (1.5/1.5)}}\begin{cacheText}La cache est au pied du pilone , elle est signée discrètement merci 😊\end{cacheText}

\cacheNumber{404}\needspace{5\baselineskip}\cacheName{\href{http://coord.info/GC725P2}{GTAQ3-26\Number{}24 Clermont-Ozourt\Number{}Mounicq} — \href{http://coord.info/GC725P2\Number{}700729004}{404}}\cacheData{{2017/07/11 crispol40, Traditional Cache (1.5/1.5)}}\begin{cacheText}En arrivant sur les lieux nous avons cherché dans le lierre.... rien puis après avoir lu les commentaires notamment celui de Crispol 40 nous avons cherché sur un arbre horizontal!!! La cache nous attend  Merci 😊\end{cacheText}

\cacheNumber{405}\needspace{5\baselineskip}\cacheName{\href{http://coord.info/GC725P6}{GTAQ3-26\Number{}25 Clermont-Ozourt\Number{}passerelle} — \href{http://coord.info/GC725P6\Number{}700730747}{405}}\cacheData{{2017/07/11 crispol40, Letterbox Hybrid (1.5/1.5)}}\begin{cacheText}Quel joli endroit ... sans le géocaching je ne l'aurai jamais connu . La letterbox est en place merci  😊\end{cacheText}

\cacheNumber{406}\needspace{5\baselineskip}\cacheName{\href{http://coord.info/GC78JY7}{chateaux d'eau de hinx} — \href{http://coord.info/GC78JY7\Number{}701018397}{406}}\cacheData{{2017/07/11 lauki3940, Traditional Cache (1.5/1.5)}}\begin{cacheText}En route pour faire la série Clermont -Ozourt on s'arrête pour signer .... la cache est très vite trouvée et Le logbook commence déjà à se remplir!!!! Merci 😊\end{cacheText}

\cacheNumber{407}\needspace{5\baselineskip}\cacheName{\href{http://coord.info/GC71ZGK}{GTAQ3-26\Number{}08 Clermont-Ozourt\Number{}Junca} — \href{http://coord.info/GC71ZGK\Number{}700694517}{407}}\cacheData{{2017/07/11 crispol40, Traditional Cache (1.5/1.5)}}\begin{cacheText}Reprise du circuit.... la cache nous attend avec l'indice Merci 😊\end{cacheText}

\cacheNumber{408}\needspace{5\baselineskip}\cacheName{\href{http://coord.info/GC71ZH2}{GTAQ3-26\Number{}09 Clermont-Ozourt\Number{}Sousset} — \href{http://coord.info/GC71ZH2\Number{}700696311}{408}}\cacheData{{2017/07/11 crispol40, Letterbox Hybrid (1.5/2)}}\begin{cacheText}La lecteur boxe ai trouvé grâce à l'indice au niveau de la souche comme indiqué par l'indice merci\end{cacheText}

\cacheNumber{409}\needspace{5\baselineskip}\cacheName{\href{http://coord.info/GC71ZHF}{GTAQ3-26\Number{}10 Clermont-Ozourt\Number{}Leborde} — \href{http://coord.info/GC71ZHF\Number{}700706365}{409}}\cacheData{{2017/07/11 crispol40, Traditional Cache (1.5/2)}}\begin{cacheText}Après 3/4 d l'Eure de recherches acharnées nous mettons enfin la main dessus!! Monsieur cherchait en bas Madame a trouvé en haut..... La cache est toujours au sein!!!! merci 😊\end{cacheText}

\cacheNumber{410}\needspace{5\baselineskip}\cacheName{\href{http://coord.info/GC71ZHW}{GTAQ3-26\Number{}11 Clermont-Ozourt\Number{}Esquirron} — \href{http://coord.info/GC71ZHW\Number{}700707822}{410}}\cacheData{{2017/07/11 crispol40, Traditional Cache (1.5/2)}}\begin{cacheText}Après s'être fait piquer par les nombreuses branches de houx nous mettant la main dessus en faire Monsieur mets la main dessus elle était par terre et la remise en place merci\end{cacheText}

\cacheNumber{411}\needspace{5\baselineskip}\cacheName{\href{http://coord.info/GC71ZJF}{GTAQ3-26\Number{}12 Clermont-Ozourt\Number{}impasse 1} — \href{http://coord.info/GC71ZJF\Number{}700708824}{411}}\cacheData{{2017/07/11 crispol40, Traditional Cache (1.5/1.5)}}\begin{cacheText}Arrivé sur les lieux Monsieur debusque la cache en deux temps trois mouvements et il repart merci\end{cacheText}

\cacheNumber{412}\needspace{5\baselineskip}\cacheName{\href{http://coord.info/GC71ZJV}{GTAQ3-26\Number{}13 Clermont-Ozourt\Number{}impasse 2} — \href{http://coord.info/GC71ZJV\Number{}700710821}{412}}\cacheData{{2017/07/11 crispol40, Traditional Cache (1.5/1.5)}}\begin{cacheText}La boîte est déloger sans difficulté et nous relevons les indices pour la petite bonus nous continuons la balade merci 😊\end{cacheText}

\cacheNumber{413}\needspace{5\baselineskip}\cacheName{\href{http://coord.info/GC721Z8}{GTAQ3-26\Number{}17 Clermont-Ozourt\Number{}Houssat} — \href{http://coord.info/GC721Z8\Number{}700718145}{413}}\cacheData{{2017/07/12 crispol40, Traditional Cache (1.5/1.5)}}\begin{cacheText}La cache est bien à sa place Merci 😊\end{cacheText}

\cacheNumber{414}\needspace{5\baselineskip}\cacheName{\href{http://coord.info/GC7220V}{GTAQ3-26\Number{}20 Clermont-Ozourt\Number{}Lasserre} — \href{http://coord.info/GC7220V\Number{}700723605}{414}}\cacheData{{2017/07/12 crispol40, Traditional Cache (1.5/1.5)}}\begin{cacheText}La cache semblait facile.... trop petite j'ai laissé tombé le tube dans le panneau!!! Monsieur a arraché le panneau afin de récupérer l'indice puis l'a remis en place!!! Dur dur.... Merci 😊\end{cacheText}

\cacheNumber{415}\needspace{5\baselineskip}\cacheName{\href{http://coord.info/GC72210}{GTAQ3-26\Number{}21 Clermont-Ozourt\Number{}Rousseau} — \href{http://coord.info/GC72210\Number{}700723977}{415}}\cacheData{{2017/07/12 crispol40, Traditional Cache (1.5/1.5)}}\begin{cacheText}Ici pas de difficulté.... on continue.... Merci 😊\end{cacheText}

\cacheNumber{416}\needspace{5\baselineskip}\cacheName{\href{http://coord.info/GC72217}{GTAQ3-26\Number{}22 Clermont-Ozourt\Number{}Mourtra} — \href{http://coord.info/GC72217\Number{}700724645}{416}}\cacheData{{2017/07/12 crispol40, Traditional Cache (1.5/1.5)}}\begin{cacheText}Celle çi est faite en voiture... Merci 😊\end{cacheText}

\cacheNumber{417}\needspace{5\baselineskip}\cacheName{\href{http://coord.info/GC725NP}{GTAQ3-26\Number{}23 Clermont-Ozourt\Number{}Labastide} — \href{http://coord.info/GC725NP\Number{}700725110}{417}}\cacheData{{2017/07/12 crispol40, Traditional Cache (1.5/1.5)}}\begin{cacheText}La cache nous attend au pied du pilone , couverte par les fougères. Merci 😊\end{cacheText}

\cacheNumber{418}\needspace{5\baselineskip}\cacheName{\href{http://coord.info/GC726D5}{GTAQ3-26\Number{}BONUS Clermont-Ozourt} — \href{http://coord.info/GC726D5\Number{}700737546}{418}}\cacheData{{2017/07/12 crispol40, Unknown Cache (1.5/2)}}\begin{cacheText}Nous avons pris un grand plaisir à faire ce joli parcours même si nous l'avons fait sur 3 jours. Nous avons un DNF mais nous reviendrons à l'occasion effacer ce point bleu sur la carte.La boite nous attend au creux de l'arbre.... il faut juste sauter le fossé. Merci Crispol pour tout ce travail 👏👏😃\end{cacheText}

\cacheNumber{419}\needspace{5\baselineskip}\cacheName{\href{http://coord.info/GC78F9D}{Voie verte 1 : les bambous} — \href{http://coord.info/GC78F9D\Number{}700739521}{419}}\cacheData{{2017/07/12 classedehinx, Traditional Cache (1.5/1.5)}}\begin{cacheText}Cache trouvée en allant faire le circuit de Clermont- Ozourt.... Merci les élèves de Hinx, la cache est sympa mais Le logbook est très humide. Il serait bien de le protéger dans une pochette en plastique merci 😊\end{cacheText}

\cacheNumber{420}\needspace{5\baselineskip}\cacheName{\href{http://coord.info/GC2NCEW}{L'ile aux trésors} — \href{http://coord.info/GC2NCEW\Number{}702085501}{420}}\cacheData{{2017/07/16 slipencarton, Traditional Cache (4/4)}}\begin{cacheText}C'est en pédalo et en compagnie de Isaasia que nous avons décidé d'aller chercher le trésor .Nous avons une heure à tuer avant le début de l'event un \Quoted{Ardennais en vacances} qui entre  dans le jeu du trésor de Marie Hyde . Parfaitement dans le thème !!!! Nous avons accosté rapidement sur la l'île et l'avons traversée rapidement  pour aller rejoindre les 4 pommes de pin. Quelle belle aventure....Merci 😃\end{cacheText}

\cacheNumber{421}\needspace{5\baselineskip}\cacheName{\href{http://coord.info/GC4WEY1}{Vieux-Boucau - Le Lavoir} — \href{http://coord.info/GC4WEY1\Number{}702045145}{421}}\cacheData{{2017/07/16 gilles64, Traditional Cache (1.5/1)}}\begin{cacheText}Beaucoup de moldus se promènent et pas facile d'être discret!!!! La cache est bien là mais plus du tout aux coordonnées indiquées cherchez le tamaris et vous la trouverez!!!! merci 😊\end{cacheText}

\cacheNumber{422}\needspace{5\baselineskip}\cacheName{\href{http://coord.info/GC4WNWB}{Vieux-Boucau - La Plage Nord} — \href{http://coord.info/GC4WNWB\Number{}702349470}{422}}\cacheData{{2017/07/16 gilles64, Traditional Cache (2/1.5)}}\begin{cacheText}Le super Meet and greet\Quoted{Un ardennais en vacances} a été suivi par un repas au restau qui nous a permis de sympathiser avec Fabilab. C'est en leur compagnie et celle de Isaasia que nous sommes partis à la recherche de ce dernier point vert. À la lumière des lampes torches la Belle  est débusquée vite fait... Merci pour la cache 😃\end{cacheText}

\cacheNumber{423}\needspace{5\baselineskip}\cacheName{\href{http://coord.info/GC52P16}{le jardin de messanges} — \href{http://coord.info/GC52P16\Number{}702036768}{423}}\cacheData{{2017/07/16 mizaga, Traditional Cache (1.5/1.5)}}\begin{cacheText}Cache originale trouvée par l'œil averti de Isa asia qui n'a pas hésité à se mouiller les mains!!!\end{cacheText}

\cacheNumber{424}\needspace{5\baselineskip}\cacheName{\href{http://coord.info/GC6J29T}{H027- Géodyssée 40-64} — \href{http://coord.info/GC6J29T\Number{}701980938}{424}}\cacheData{{2017/07/16 mizaga, Traditional Cache (1.5/1.5)}}\begin{cacheText}Mplc\end{cacheText}

\cacheNumber{425}\needspace{5\baselineskip}\cacheName{\href{http://coord.info/GC6J2A6}{H026- Géodyssée 40-64} — \href{http://coord.info/GC6J2A6\Number{}701985450}{425}}\cacheData{{2017/07/16 mizaga, Traditional Cache (1.5/1.5)}}\begin{cacheText}Toujours en compagnie d'Isa Azias nous deux puisqu'on la cache merci\end{cacheText}

\cacheNumber{426}\needspace{5\baselineskip}\cacheName{\href{http://coord.info/GC6J2AV}{H025- Géodyssée 40-64} — \href{http://coord.info/GC6J2AV\Number{}701989364}{426}}\cacheData{{2017/07/16 mizaga, Traditional Cache (1.5/1.5)}}\begin{cacheText}Toujours en compagnie d'Isa Azias nous on cherche en faite cachets qui est assez difficile à trouver au vu de la photo mais nous persévérant et nous finissons par mettre la main dessus merci pour la cache\end{cacheText}

\cacheNumber{427}\needspace{5\baselineskip}\cacheName{\href{http://coord.info/GC6J2BD}{H024- Géodyssée 40-64} — \href{http://coord.info/GC6J2BD\Number{}701992641}{427}}\cacheData{{2017/07/16 mizaga, Traditional Cache (1.5/1.5)}}\begin{cacheText}Cache rapidement trouvée par Isa Asia et moi-même ....nous continuons merci\end{cacheText}

\cacheNumber{428}\needspace{5\baselineskip}\cacheName{\href{http://coord.info/GC6J2C2}{H023- Géodyssée 40-64} — \href{http://coord.info/GC6J2C2\Number{}701994943}{428}}\cacheData{{2017/07/16 mizaga, Traditional Cache (1.5/1.5)}}\begin{cacheText}L indice est très explicite et la cache rapidement trouvée avec Isa Asia Merci 😊\end{cacheText}

\cacheNumber{429}\needspace{5\baselineskip}\cacheName{\href{http://coord.info/GC6J2CA}{H022- Géodyssée 40-64} — \href{http://coord.info/GC6J2CA\Number{}702001742}{429}}\cacheData{{2017/07/16 mizaga, Traditional Cache (1.5/1.5)}}\begin{cacheText}Sans difficulté nous trouvons le chêne liège et logons la cache  Merci 😊\end{cacheText}

\cacheNumber{430}\needspace{5\baselineskip}\cacheName{\href{http://coord.info/GC6JJ7G}{H021- Géodyssée 40-64} — \href{http://coord.info/GC6JJ7G\Number{}702003763}{430}}\cacheData{{2017/07/16 mizaga, Traditional Cache (1.5/2)}}\begin{cacheText}Trouvée avec l'odeur des pins et le chant des cigales quelque polué par le bruit des voitures toujours en compagnie de Isa asia Mplc 😃\end{cacheText}

\cacheNumber{431}\needspace{5\baselineskip}\cacheName{\href{http://coord.info/GC6JJ7R}{H020- Géodyssée 40-64} — \href{http://coord.info/GC6JJ7R\Number{}702005214}{431}}\cacheData{{2017/07/16 mizaga, Traditional Cache (1.5/1.5)}}\begin{cacheText}Cache super facile... pomme de pin au pied d'un chêne c' est suspect... Merci 😊\end{cacheText}

\cacheNumber{432}\needspace{5\baselineskip}\cacheName{\href{http://coord.info/GC6JJ81}{H019- Géodyssée 40-64} — \href{http://coord.info/GC6JJ81\Number{}702008449}{432}}\cacheData{{2017/07/16 mizaga, Traditional Cache (1.5/1.5)}}\begin{cacheText}Toujours en route pour l'event un ardennais en vacances nous logons en passant Merci 😊\end{cacheText}

\cacheNumber{433}\needspace{5\baselineskip}\cacheName{\href{http://coord.info/GC6JJ88}{H018- Géodyssée 40-64} — \href{http://coord.info/GC6JJ88\Number{}702010374}{433}}\cacheData{{2017/07/16 mizaga, Traditional Cache (1.5/1.5)}}\begin{cacheText}Encore un petite qui passe au jaune.... Merci 😊\end{cacheText}

\cacheNumber{434}\needspace{5\baselineskip}\cacheName{\href{http://coord.info/GC6JJ8J}{H017- Géodyssée 40-64} — \href{http://coord.info/GC6JJ8J\Number{}702012339}{434}}\cacheData{{2017/07/16 mizaga, Traditional Cache (1.5/1.5)}}\begin{cacheText}Pas de difficulté particulière... en route pour la prochaine...\end{cacheText}

\cacheNumber{435}\needspace{5\baselineskip}\cacheName{\href{http://coord.info/GC6JJ8T}{H016- Géodyssée 40-64} — \href{http://coord.info/GC6JJ8T\Number{}702017006}{435}}\cacheData{{2017/07/16 mizaga, Traditional Cache (1.5/1.5)}}\begin{cacheText}Grâce a l'œil perspicace de Isala cache est débusquer en deux temps trois mouvements merci\end{cacheText}

\cacheNumber{436}\needspace{5\baselineskip}\cacheName{\href{http://coord.info/GC6JJ9C}{H015- Géodyssée 40-64} — \href{http://coord.info/GC6JJ9C\Number{}702020303}{436}}\cacheData{{2017/07/16 mizaga, Traditional Cache (1.5/1.5)}}\begin{cacheText}Arrivés sur Vieux boucau ...celle-ci est trouvée sans problème merci 😊\end{cacheText}

\cacheNumber{437}\needspace{5\baselineskip}\cacheName{\href{http://coord.info/GC6JJ9M}{H014- Géodyssée 40-64} — \href{http://coord.info/GC6JJ9M\Number{}702023901}{437}}\cacheData{{2017/07/16 mizaga, Traditional Cache (1.5/1.5)}}\begin{cacheText}Sans problème particulier.... Merci 😊\end{cacheText}

\cacheNumber{438}\needspace{5\baselineskip}\cacheName{\href{http://coord.info/GC6JXNE}{H012- Géodyssée 40-64} — \href{http://coord.info/GC6JXNE\Number{}702040177}{438}}\cacheData{{2017/07/16 mizaga, Traditional Cache (1.5/1.5)}}\begin{cacheText}Après avoir cherché sous plusieurs arbres nous nous enfonçons et nous mettons la main sur la belle merci pour la cache 😃\end{cacheText}

\cacheNumber{439}\needspace{5\baselineskip}\cacheName{\href{http://coord.info/GC6JXNG}{H013- Géodyssée 40-64} — \href{http://coord.info/GC6JXNG\Number{}702029692}{439}}\cacheData{{2017/07/16 mizaga, Traditional Cache (1.5/1.5)}}\begin{cacheText}En arrivant sur les lieux nous tombons nez à nez avec FDCM !!!!nous trouvons la cache sans difficulté grâce a l'indice merci 😊\end{cacheText}

\cacheNumber{440}\needspace{5\baselineskip}\cacheName{\href{http://coord.info/GC6QAD9}{Eglise de Seignosse-le-Penon} — \href{http://coord.info/GC6QAD9\Number{}701998336}{440}}\cacheData{{2017/07/16 Bigorra65, Traditional Cache (1.5/1.5)}}\begin{cacheText}Au plus haut des cieux effectivement.... en route vers l'event de vieux boucau nous nous arrêtons avec Isa asia pour chercher la cache... heureusement que nous sommes deux!!!! Merci pour la découverte de ce lieu moderne\end{cacheText}

\cacheNumber{441}\needspace{5\baselineskip}\cacheName{\href{http://coord.info/GC75VNK}{Meet and greet - Un ardennais en vacance !} — \href{http://coord.info/GC75VNK\Number{}702267732}{441}}\cacheData{{2017/07/16 kentin08350, Event Cache (1/1)}}\begin{cacheText}Super moment passé en compagnie de geocacheurs venus d 'horizon divers !!!! Après avoir échangé diverses anecdotes et signé le logbook la majorité des personnes est repartie vers d'autres lieux. Pour finir la soirée en beauté nous avons décidé de manger avec Isaasia et Fabilab dans ce charmant restau. Peu pressé de se séparer nous avons pris la direction de la plage afin de débusquer une dernière petite boîte..... Merci 😊 pour ce bon moment\end{cacheText}

\cacheNumber{442}\needspace{5\baselineskip}\cacheName{\href{http://coord.info/GC5981H}{Le lavoir de tercis-les-bains} — \href{http://coord.info/GC5981H\Number{}703626699}{442}}\cacheData{{2017/07/21 kar@melos40, Traditional Cache (2/1.5)}}\begin{cacheText}Cache trouvée avec difficulté ...je ne m'attendais pas à cela vu l'indice !!!! Je pense qu'il s'agit d'une cache de dépannage. La photo pour preuve merci pour la cache\end{cacheText}

\cacheNumber{443}\needspace{5\baselineskip}\cacheName{\href{http://coord.info/GC5CKVB}{[BSD] \Number{}018} — \href{http://coord.info/GC5CKVB\Number{}703636660}{443}}\cacheData{{2017/07/21 GéoLandesTour, Traditional Cache (1.5/1.5)}}\begin{cacheText}Trouvée facilement grâce à la photo !!! Un panneau probablement placé par le propriétaire de la maison annonçant du bois de chauffage à vendre est devant la cache qui est parfaitement intégrée dans le mur!!! Merci 😊\end{cacheText}

\cacheNumber{444}\needspace{5\baselineskip}\cacheName{\href{http://coord.info/GC5CMAB}{[BSD] \Number{}020} — \href{http://coord.info/GC5CMAB\Number{}703634722}{444}}\cacheData{{2017/07/21 GéoLandesTour, Traditional Cache (1.5/1.5)}}\begin{cacheText}Difficile de déloger la cache ... beaucoup de circulation en cette fin d'après-midi Mplc\end{cacheText}

\cacheNumber{445}\needspace{5\baselineskip}\cacheName{\href{http://coord.info/GC5CMD8}{[BSD] \Number{}022} — \href{http://coord.info/GC5CMD8\Number{}703630115}{445}}\cacheData{{2017/07/21 GéoLandesTour, Traditional Cache (2/1.5)}}\begin{cacheText}Il suffit de comprendre l'indice pour trouver la cache ....merci 😊\end{cacheText}

\cacheNumber{446}\needspace{5\baselineskip}\cacheName{\href{http://coord.info/GC5CNFK}{[BSD] \Number{}023} — \href{http://coord.info/GC5CNFK\Number{}703628414}{446}}\cacheData{{2017/07/21 GéoLandesTour, Traditional Cache (1.5/1.5)}}\begin{cacheText}Une classique..... logbook plein; j'ai signé sur le côté mplc\end{cacheText}

\cacheNumber{447}\needspace{5\baselineskip}\cacheName{\href{http://coord.info/GC5CNFW}{[BSD] \Number{}024} — \href{http://coord.info/GC5CNFW\Number{}703639701}{447}}\cacheData{{2017/07/21 GéoLandesTour, Traditional Cache (1.5/1.5)}}\begin{cacheText}Les recherches ont été effectuées auprès de l'arbre indiqué sur la photo :non seulement il faut être grand mais il ne faut  pas avoir peur ni de se mouiller les mains pour attraper la cache ni avoir la phobie des fourmis !!!!!merci 😃\end{cacheText}

\cacheNumber{448}\needspace{5\baselineskip}\cacheName{\href{http://coord.info/GC5CNVV}{[BSD] \Number{}025} — \href{http://coord.info/GC5CNVV\Number{}703621841}{448}}\cacheData{{2017/07/21 GéoLandesTour, Traditional Cache (1.5/1.5)}}\begin{cacheText}Une classique...trouvée sans problème.Mplc 😃\end{cacheText}

\cacheNumber{449}\needspace{5\baselineskip}\cacheName{\href{http://coord.info/GC60YFT}{Les trois croix d'undurein} — \href{http://coord.info/GC60YFT\Number{}705831021}{449}}\cacheData{{2017/07/28 Cestasgirl, Traditional Cache (1.5/1.5)}}\begin{cacheText}La cache est toujours en place !!!!Nous l'avons vite trouvée grâce à l'indice merci 😊\end{cacheText}

\cacheNumber{450}\needspace{5\baselineskip}\cacheName{\href{http://coord.info/GC6G3EX}{Le fronton d'Espes} — \href{http://coord.info/GC6G3EX\Number{}705830833}{450}}\cacheData{{2017/07/28 Cestasgirl, Traditional Cache (1.5/1.5)}}\begin{cacheText}L'arbre n'est plus 😭😭😭 mais la cache est toujours là !!!! Le log book d'origine est trempé et il a été remplacé. Merci pour la cache\end{cacheText}

\cacheNumber{451}\needspace{5\baselineskip}\cacheName{\href{http://coord.info/GC3FYX7}{GR8\Number{}16} — \href{http://coord.info/GC3FYX7\Number{}706237977}{451}}\cacheData{{2017/07/29 Peyo64, Traditional Cache (2/1.5)}}\begin{cacheText}Par ce bel après-midi nous décidons de continuer le GR8 et ouf aujourd'hui le chemin est praticable !!!! La cache est vite repérée grâce à l'indice.Merci\end{cacheText}

\cacheNumber{452}\needspace{5\baselineskip}\cacheName{\href{http://coord.info/GC3FYXP}{GR8\Number{}17} — \href{http://coord.info/GC3FYXP\Number{}706249924}{452}}\cacheData{{2017/07/29 Peyo64, Traditional Cache (2/1.5)}}\begin{cacheText}La cache est bien la....mais au pied de l'arbre !!!!!Merci\end{cacheText}

\cacheNumber{453}\needspace{5\baselineskip}\cacheName{\href{http://coord.info/GC3FYYB}{GR8\Number{}18} — \href{http://coord.info/GC3FYYB\Number{}706251702}{453}}\cacheData{{2017/07/29 Peyo64, Traditional Cache (2/1.5)}}\begin{cacheText}Pour celle-ci la photo nous est bien utile,la cache  nous attend , posée Sur l'arbre. Merci\end{cacheText}

\cacheNumber{454}\needspace{5\baselineskip}\cacheName{\href{http://coord.info/GC3FYZ4}{GR8\Number{}19} — \href{http://coord.info/GC3FYZ4\Number{}706252497}{454}}\cacheData{{2017/07/29 Peyo64, Traditional Cache (2/1.5)}}\begin{cacheText}La cache est bien là mais à 40 cm du sol et non pas à 1 m !!!!Merci pour la cache 😃\end{cacheText}

\cacheNumber{455}\needspace{5\baselineskip}\cacheName{\href{http://coord.info/GC3FYZV}{GR8\Number{}20} — \href{http://coord.info/GC3FYZV\Number{}706255701}{455}}\cacheData{{2017/07/29 Peyo64, Traditional Cache (1.5/1.5)}}\begin{cacheText}Nous sommes dans un endroit très joli: le ruisseau est plein de nénuphars !!!! La cache est débusquée en deux temps trois mouvements. Merci 😊\end{cacheText}

\cacheNumber{456}\needspace{5\baselineskip}\cacheName{\href{http://coord.info/GC3FZ0H}{GR8\Number{}21} — \href{http://coord.info/GC3FZ0H\Number{}706258263}{456}}\cacheData{{2017/07/29 Peyo64, Traditional Cache (2/1.5)}}\begin{cacheText}Après avoir vu une superbe plombière, nous arrivons sur le lieux de la cache: elle est bien la au coeur de la souche merci 😊\end{cacheText}

\cacheNumber{457}\needspace{5\baselineskip}\cacheName{\href{http://coord.info/GC3FZ17}{GR8\Number{}22} — \href{http://coord.info/GC3FZ17\Number{}706258869}{457}}\cacheData{{2017/07/29 Peyo64, Traditional Cache (2/1.5)}}\begin{cacheText}celle-là est trouvée sans difficulté le lieu est propre merci 😃\end{cacheText}

\cacheNumber{458}\needspace{5\baselineskip}\cacheName{\href{http://coord.info/GC3FZ20}{GR8\Number{}23} — \href{http://coord.info/GC3FZ20\Number{}706262363}{458}}\cacheData{{2017/07/29 Peyo64, Traditional Cache (1.5/1.5)}}\begin{cacheText}Les containers à poubelle qui sont sur la photo ont disparu mais la cache est toujours dans son petit trou!!!! Merci 😊\end{cacheText}

\cacheNumber{459}\needspace{5\baselineskip}\cacheName{\href{http://coord.info/GC3FZ2Y}{GR8\Number{}24} — \href{http://coord.info/GC3FZ2Y\Number{}706263281}{459}}\cacheData{{2017/07/29 Peyo64, Traditional Cache (2/1.5)}}\begin{cacheText}La cache nous attend dans le laurier : nous n'oublions pas de relever l'indice pour la bonus. Merci 😊\end{cacheText}

\cacheNumber{460}\needspace{5\baselineskip}\cacheName{\href{http://coord.info/GC3FZ4K}{GR8\Number{}25} — \href{http://coord.info/GC3FZ4K\Number{}706263785}{460}}\cacheData{{2017/07/29 Peyo64, Traditional Cache (1.5/1.5)}}\begin{cacheText}C'est la dernière cache de la journée et elle est trouvée sans aucun problème . Merci 😊\end{cacheText}

\cacheNumber{461}\needspace{5\baselineskip}\cacheName{\href{http://coord.info/GC28A48}{Le belvédère de Mourenx} — \href{http://coord.info/GC28A48\Number{}707030331}{461}}\cacheData{{2017/07/31 Peyo64, Traditional Cache (1.5/1.5)}}\begin{cacheText}En route pour un rendez-vous sur Pau nous décidons de nous arrêter pour faire cette cache . L'horizon s'étend à perte de vue ,je ne connaissais pas du tout l'endroit !!! La cache est vite découverte grâce à l'indice . Le log book d'origine a disparu mais nous signons un logbook improvisé que nous trouvons la .Merci 😊\end{cacheText}

\cacheNumber{462}\needspace{5\baselineskip}\cacheName{\href{http://coord.info/GC5F3FN}{uzosienne} — \href{http://coord.info/GC5F3FN\Number{}707038361}{462}}\cacheData{{2017/07/31 coxigypaete, Traditional Cache (1.5/2)}}\begin{cacheText}En arrivant sur les lieux nous découvrons cette très belle école rénovée et cette jolie mairie sur l'eau. Arrivés devant la cache le GPS nous balade et nous nous arrêtons devant un arbre qui semble correspondre à l'indice .Heureusement que nous sommes deux pour faire la courte échelle et débusquer la cache !!!! Merci 😊 Out: 2 jetons                                       In: 1 smiley vert et une fève\end{cacheText}

\cacheNumber{463}\needspace{5\baselineskip}\cacheName{\href{http://coord.info/GC5N475}{Le rempart canon} — \href{http://coord.info/GC5N475\Number{}707028014}{463}}\cacheData{{2017/07/31 chantou 64, Traditional Cache (1/1.5)}}\begin{cacheText}Cette fois la cache est trouvée sans difficulté quoique un peu haute pour moi!!!!!mais je réussi à l'attraper quand même !!!!merci\end{cacheText}

\cacheNumber{464}\needspace{5\baselineskip}\cacheName{\href{http://coord.info/GC5QQJ8}{01\Number{}GAVE ETCOTEAUX} — \href{http://coord.info/GC5QQJ8\Number{}706953773}{464}}\cacheData{{2017/07/31 coxigypaete, Traditional Cache (1.5/1.5)}}\begin{cacheText}Nous démarrons aujourd'hui ce circuit, mais le finirons plus tard faute de temps !!! Cette cache est débusquée facilement grâce à l'indice. Merci 😊\end{cacheText}

\cacheNumber{465}\needspace{5\baselineskip}\cacheName{\href{http://coord.info/GC5QQJR}{02\Number{}GAVE ETCOTEAUX} — \href{http://coord.info/GC5QQJR\Number{}707054199}{465}}\cacheData{{2017/07/31 coxigypaete, Traditional Cache (1.5/1.5)}}\begin{cacheText}Arrivés sur les lieux, l'indice nous induit en erreur mais avec un peu d'intuition nous finissons par découvrir la belle. Ouff ...merci pour la cache 😃\end{cacheText}

\cacheNumber{466}\needspace{5\baselineskip}\cacheName{\href{http://coord.info/GC6PEAD}{Le bassin de rétention} — \href{http://coord.info/GC6PEAD\Number{}707033406}{466}}\cacheData{{2017/07/31 Votsh, Traditional Cache (1.5/1.5)}}\begin{cacheText}Un rendez-vous sur Pau ....nous avons le temps de faire quelques caches !!!Nous arrivons en ces lieux où nous effectuons une recherche minutieuse . Nous finissons par mettre la main dessus assez rapidement. Nous signons donc sur ce logbook tout neuf !!!Merci pour le travail de pose.\end{cacheText}

\cacheNumber{467}\needspace{5\baselineskip}\cacheName{\href{http://coord.info/GC55YEB}{Aux alentours de la colo...} — \href{http://coord.info/GC55YEB\Number{}708081981}{467}}\cacheData{{2017/08/04 Iron\Underscore{}Momo, Traditional Cache (2/3.5)}}\begin{cacheText}Apres un méga pique nique au col du Soulor l'envie nous prend de faire quelques caches. Nous voilà parti à la recherche du trésor. Le site est extraordinaire..... magique. Grâce à l'indice et la photo elle est découverte en deux temps trois mouvements.... seul bémol... le logbook est trempé et inutilisable !!!! Merci de faire la maintenance 😃😃\end{cacheText}

\cacheNumber{468}\needspace{5\baselineskip}\cacheName{\href{http://coord.info/GC3FZ5B}{GR8\Number{}26} — \href{http://coord.info/GC3FZ5B\Number{}709292249}{468}}\cacheData{{2017/08/07 Peyo64, Traditional Cache (1.5/1.5)}}\begin{cacheText}L'indice et la photo nous guident bien.... Merci 😊\end{cacheText}

\cacheNumber{469}\needspace{5\baselineskip}\cacheName{\href{http://coord.info/GC3FZ68}{GR8\Number{}27} — \href{http://coord.info/GC3FZ68\Number{}709292416}{469}}\cacheData{{2017/08/07 Peyo64, Traditional Cache (2/1.5)}}\begin{cacheText}La sapinette heberge toujours la Belle . Merci 😊\end{cacheText}

\cacheNumber{470}\needspace{5\baselineskip}\cacheName{\href{http://coord.info/GC3FZ94}{GR8\Number{}28} — \href{http://coord.info/GC3FZ94\Number{}709327797}{470}}\cacheData{{2017/08/07 Peyo64, Traditional Cache (1.5/1.5)}}\begin{cacheText}La cache nous attend à sa place.Indice relevé pour la bonus.Merci\end{cacheText}

\cacheNumber{471}\needspace{5\baselineskip}\cacheName{\href{http://coord.info/GC3FZBA}{GR8\Number{}29} — \href{http://coord.info/GC3FZBA\Number{}709291967}{471}}\cacheData{{2017/08/07 Peyo64, Traditional Cache (1.5/1.5)}}\begin{cacheText}Reprise aujourd'hui du GR 8.Cette cache est trouvée sans aucun problème :elle nous attend au milieu . Merci

 Merci 😊\end{cacheText}

\cacheNumber{472}\needspace{5\baselineskip}\cacheName{\href{http://coord.info/GC3FZC0}{GR8\Number{}30} — \href{http://coord.info/GC3FZC0\Number{}709329196}{472}}\cacheData{{2017/08/07 Peyo64, Traditional Cache (2/1.5)}}\begin{cacheText}Nous arrivons sur les lieux et découvrons les ruines ,probablement une maison , que nous fouillons. La cache est bien présente.Merci\end{cacheText}

\cacheNumber{473}\needspace{5\baselineskip}\cacheName{\href{http://coord.info/GC3FZCE}{GR8\Number{}31} — \href{http://coord.info/GC3FZCE\Number{}709329774}{473}}\cacheData{{2017/08/07 Peyo64, Traditional Cache (2/1.5)}}\begin{cacheText}Nous poursuivons notre promenade jusqu'à la cachette qui nous attend derrière la mousse.... merci\end{cacheText}

\cacheNumber{474}\needspace{5\baselineskip}\cacheName{\href{http://coord.info/GC3FZDC}{GR8\Number{}32} — \href{http://coord.info/GC3FZDC\Number{}709331364}{474}}\cacheData{{2017/08/07 Peyo64, Traditional Cache (2/1.5)}}\begin{cacheText}Arrivés sur les lieux l'endroit est désert. La caché est trouvée et l'indice noté. MerciMMMMMerci 😊\end{cacheText}

\cacheNumber{475}\needspace{5\baselineskip}\cacheName{\href{http://coord.info/GC3FZEW}{GR8\Number{}33} — \href{http://coord.info/GC3FZEW\Number{}709330534}{475}}\cacheData{{2017/08/07 Peyo64, Traditional Cache (1.5/1.5)}}\begin{cacheText}Grâce à la photo la cache est vite trouvée Merci 😊\end{cacheText}

\cacheNumber{476}\needspace{5\baselineskip}\cacheName{\href{http://coord.info/GC3FZFD}{GR8\Number{}34} — \href{http://coord.info/GC3FZFD\Number{}709335234}{476}}\cacheData{{2017/08/07 Peyo64, Traditional Cache (2/1.5)}}\begin{cacheText}Pas de difficulté particulière... Merci 😊\end{cacheText}

\cacheNumber{477}\needspace{5\baselineskip}\cacheName{\href{http://coord.info/GC3FZGC}{GR8\Number{}35} — \href{http://coord.info/GC3FZGC\Number{}709335473}{477}}\cacheData{{2017/08/07 Peyo64, Traditional Cache (2/1.5)}}\begin{cacheText}Nous la trouvons facilement et signons sur Un logbook de replacement. Merci 😊\end{cacheText}

\cacheNumber{478}\needspace{5\baselineskip}\cacheName{\href{http://coord.info/GC3FZJN}{GR8\Number{}36} — \href{http://coord.info/GC3FZJN\Number{}709336154}{478}}\cacheData{{2017/08/07 Peyo64, Traditional Cache (2/2)}}\begin{cacheText}Arrivés sur les lieux nous ne voyons pas trop de lierre mais la photo nous indique ou chercher!! Merci 😊 

INDIQUE LE LIEUX 0\end{cacheText}

\cacheNumber{479}\needspace{5\baselineskip}\cacheName{\href{http://coord.info/GC3FZJW}{GR8\Number{}37} — \href{http://coord.info/GC3FZJW\Number{}709336348}{479}}\cacheData{{2017/08/07 Peyo64, Traditional Cache (1.5/1.5)}}\begin{cacheText}Petite révision du code de la route..... cache au milieu des orties... monsieur s'y colle!!! Ici aussi nouveau logbook !!! Merci 😊\end{cacheText}

\cacheNumber{480}\needspace{5\baselineskip}\cacheName{\href{http://coord.info/GC3FZKJ}{GR8\Number{}39} — \href{http://coord.info/GC3FZKJ\Number{}709337253}{480}}\cacheData{{2017/08/07 Peyo64, Traditional Cache (2/1.5)}}\begin{cacheText}Etonnant... seulement  3 signatures avant moi sur le logbook dont une qui date de 2015;merci\end{cacheText}

\cacheNumber{481}\needspace{5\baselineskip}\cacheName{\href{http://coord.info/GC3FZM0}{GR8\Number{}40} — \href{http://coord.info/GC3FZM0\Number{}709337440}{481}}\cacheData{{2017/08/07 Peyo64, Traditional Cache (2/2)}}\begin{cacheText}Pas de bus à l'horizon mais la cache nous attend. Merci 😊\end{cacheText}

\cacheNumber{482}\needspace{5\baselineskip}\cacheName{\href{http://coord.info/GC3FZMD}{GR8\Number{}41} — \href{http://coord.info/GC3FZMD\Number{}709337844}{482}}\cacheData{{2017/08/07 Peyo64, Traditional Cache (2/2)}}\begin{cacheText}La cache nous attend bien à sa place.... la végétation n'est pas si intense que ça erci§§Mmerci 😊\end{cacheText}

\cacheNumber{483}\needspace{5\baselineskip}\cacheName{\href{http://coord.info/GC3FZN2}{GR8\Number{}42} — \href{http://coord.info/GC3FZN2\Number{}709338135}{483}}\cacheData{{2017/08/07 Peyo64, Traditional Cache (1.5/1.5)}}\begin{cacheText}La cache est rapidement dénichée grâce à l'indice, nous relevons l' indice qui nous mènera nous espérons à la bonus merci\end{cacheText}

\cacheNumber{484}\needspace{5\baselineskip}\cacheName{\href{http://coord.info/GC3FZNE}{GR8\Number{}43} — \href{http://coord.info/GC3FZNE\Number{}709338342}{484}}\cacheData{{2017/08/07 Peyo64, Traditional Cache (2/1.5)}}\begin{cacheText}Plutôt astucieuse celle la!!! Mais on la trouve Merci 😊\end{cacheText}

\cacheNumber{485}\needspace{5\baselineskip}\cacheName{\href{http://coord.info/GC3FZNR}{GR8\Number{}44} — \href{http://coord.info/GC3FZNR\Number{}709338510}{485}}\cacheData{{2017/08/07 Peyo64, Traditional Cache (2/1.5)}}\begin{cacheText}Pas de soucis pour celle la! Merci 😊\end{cacheText}

\cacheNumber{486}\needspace{5\baselineskip}\cacheName{\href{http://coord.info/GC3FZRQ}{GR8\Number{}45} — \href{http://coord.info/GC3FZRQ\Number{}709338767}{486}}\cacheData{{2017/08/07 Peyo64, Traditional Cache (2/1.5)}}\begin{cacheText}Cache trouvée avec difficulté.. terrain envahi de ronces!!! La persévérance finit toujours par payer Merci 😊\end{cacheText}

\cacheNumber{487}\needspace{5\baselineskip}\cacheName{\href{http://coord.info/GC3FZT4}{GR8\Number{}47} — \href{http://coord.info/GC3FZT4\Number{}709339136}{487}}\cacheData{{2017/08/07 Peyo64, Traditional Cache (1.5/1.5)}}\begin{cacheText}Ici l cache est vite repérée merci 😊\end{cacheText}

\cacheNumber{488}\needspace{5\baselineskip}\cacheName{\href{http://coord.info/GC3FZTF}{GR8\Number{}49} — \href{http://coord.info/GC3FZTF\Number{}709339559}{488}}\cacheData{{2017/08/07 Peyo64, Traditional Cache (2/1.5)}}\begin{cacheText}La dernière est trouvé sans difficulté devant cette très jolie église est devant ce beau pelerin de Saint Jacques merci 😊\end{cacheText}

\cacheNumber{489}\needspace{5\baselineskip}\cacheName{\href{http://coord.info/GC5WVAK}{\Number{}L2-03 Guiche - Le Comté du Labourd} — \href{http://coord.info/GC5WVAK\Number{}709928256}{489}}\cacheData{{2017/08/09 gilles64, Multi-cache (2/1.5)}}\begin{cacheText}En cette fin d' après-midi pluvieux nous décidons d'attaquer nos premières multis . Arrivés devant le presbytère et la maison nous relevons les indices et partons faire les calculs des coordonnées qui nous mènent tout droit à la cache. Merci Gilles pour ce bon moment 😀\end{cacheText}

\cacheNumber{490}\needspace{5\baselineskip}\cacheName{\href{http://coord.info/GC5XGMY}{\Number{}L2-10 Urt - Le Comté du Labourd} — \href{http://coord.info/GC5XGMY\Number{}710163580}{490}}\cacheData{{2017/08/10 gilles64, Multi-cache (2/1.5)}}\begin{cacheText}Apres avoir relevé les deux dates et établis les nouvelles coordonnées je débusque enfin la belle..... qui me résiste et qui est coincée au fond!!!!Impossible de la déloger!!!Avec l'accord de l'owner je la marque Found It et j'irai signer le logbook lorsque la maintenance sera effectuée. Merci Gilles pour ce splendide point de vu\end{cacheText}

\cacheNumber{491}\needspace{5\baselineskip}\cacheName{\href{http://coord.info/GC4X43W}{A10 - Aire de repos de Chermignac Est} — \href{http://coord.info/GC4X43W\Number{}710324051}{491}}\cacheData{{2017/08/11 gilles64, Traditional Cache (1.5/1.5)}}\begin{cacheText}En route pour l'Atlantic Évent de Saint Viaud petite pause et quelle surprise de trouver une cache sur cette aire de repos....de Gilles 64 qui plus est!!!!! Pas de difficulté pour la dénicher . Merci 😊\end{cacheText}

\cacheNumber{492}\needspace{5\baselineskip}\cacheName{\href{http://coord.info/GC5HCPX}{A10 - Aire de FENIOUX Est} — \href{http://coord.info/GC5HCPX\Number{}710328100}{492}}\cacheData{{2017/08/11 eureka17, Traditional Cache (1.5/1.5)}}\begin{cacheText}En route pour l'Atlantic Évent nous nous arrêtons faire cette cache. Pas de difficulté pour la trouver. Quelle surprise de signer à la suite de Titiger 39 et Lauki 3940!!!! Le sud ouest débarque.... Merci 😊\end{cacheText}

\cacheNumber{493}\needspace{5\baselineskip}\cacheName{\href{http://coord.info/GC49GFB}{Croix de Tartifume - Vue} — \href{http://coord.info/GC49GFB\Number{}710360274}{493}}\cacheData{{2017/08/11 FREDAGNES, Traditional Cache (1.5/2)}}\begin{cacheText}Cache rapidement trouvée mais en mauvais état :le log book est détrempé ,impossible de signer . Je joins la photo pour prouver ma découverte . Merci 😊\end{cacheText}

\cacheNumber{494}\needspace{5\baselineskip}\cacheName{\href{http://coord.info/GC6XDG4}{[HDV] 3/5 : Le vieux cimetière} — \href{http://coord.info/GC6XDG4\Number{}710511092}{494}}\cacheData{{2017/08/11 mattdevue, Traditional Cache (1.5/1.5)}}\begin{cacheText}Arrivés pour participer à l'Atlantic Évent de demain nous faisons quelques caches pour nous mettre en jambe. Le lieu est rapidement trouvé mais la boîte compliquée à ouvrir. Merci de nous faire découvrir cet endroit particulier😊\end{cacheText}

\cacheNumber{495}\needspace{5\baselineskip}\cacheName{\href{http://coord.info/GC6V6PB}{[HDV] 2/5 : La chapelle de la Blanchardais} — \href{http://coord.info/GC6V6PB\Number{}710510489}{495}}\cacheData{{2017/08/11 mattdevue, Traditional Cache (2/2.5)}}\begin{cacheText}La chapelle semble abandonnée ... quel dommage !!! La très jolie cache nous attend et nous prenons soin de ne pas l'abîmer. Belle réalisation.... Merci 😊\end{cacheText}

\cacheNumber{496}\needspace{5\baselineskip}\cacheName{\href{http://coord.info/GC6XEXQ}{[HDV] 4/5 : Le château de la Blanchardais} — \href{http://coord.info/GC6XEXQ\Number{}710509985}{496}}\cacheData{{2017/08/11 mattdevue, Traditional Cache (1.5/1.5)}}\begin{cacheText}Le château nous attend. Le mur est propre et un détail attire mon œil!!! La cache est bien la.Merci😊\end{cacheText}

\cacheNumber{497}\needspace{5\baselineskip}\cacheName{\href{http://coord.info/GC6PKF0}{LBP\Underscore{}PN21} — \href{http://coord.info/GC6PKF0\Number{}710379225}{497}}\cacheData{{2017/08/11 duj\Underscore{}family, Traditional Cache (1.5/1.5)}}\begin{cacheText}En route pour l'Atlantic Évent de Saint Viaud de demain nous nous arrêtons faire quelques caches: celle-ci est débusquée en deux temps trois mouvements merci 😊\end{cacheText}

\cacheNumber{498}\needspace{5\baselineskip}\cacheName{\href{http://coord.info/GC3DAK3}{Où sont Bobine de Cuivre \And{} Bob le Bricoleur \Number{}5} — \href{http://coord.info/GC3DAK3\Number{}710509590}{498}}\cacheData{{2017/08/12 bob le bricoleur, Traditional Cache (2.5/1.5)}}\begin{cacheText}La cache est vite trouvée ....Merci 😊\end{cacheText}

\cacheNumber{499}\needspace{5\baselineskip}\cacheName{\href{http://coord.info/GC4QBAF}{Au bout du bout} — \href{http://coord.info/GC4QBAF\Number{}710509405}{499}}\cacheData{{2017/08/12 MISSION ESTUAIRE, Traditional Cache (1.5/1.5)}}\begin{cacheText}Un GPS qui s'affole mais malgré tout on debusque la cache tombée d'un vaisseau spatial.Il n'y a aucun objet voyageur ...que des capsules de bière et des coquillages .Merci 😊\end{cacheText}

\cacheNumber{500}\needspace{5\baselineskip}\cacheName{\href{http://coord.info/GC3DAJX}{Où sont Bobine de Cuivre \And{} Bob le Bricoleur \Number{}3} — \href{http://coord.info/GC3DAJX\Number{}710509156}{500}}\cacheData{{2017/08/12 bob le bricoleur, Traditional Cache (2.5/1.5)}}\begin{cacheText}Le GPS continue à nous balader mais nous étendons la zone de nos recherches. Nous finissons par trouver la cache !!! Très beau dolmen.Merci 😀\end{cacheText}

\cacheNumber{501}\needspace{5\baselineskip}\cacheName{\href{http://coord.info/GC610DZ}{Souvenirs d'enfance} — \href{http://coord.info/GC610DZ\Number{}710428892}{501}}\cacheData{{2017/08/12 Q-D78, Traditional Cache (1.5/1.5)}}\begin{cacheText}Pas de problème pour débusquer la cache, l'indice est explicite merci 😊\end{cacheText}

\cacheNumber{502}\needspace{5\baselineskip}\cacheName{\href{http://coord.info/GC3DAJP}{Où sont Bobine de Cuivre \And{} Bob le Bricoleur \Number{}2} — \href{http://coord.info/GC3DAJP\Number{}710419685}{502}}\cacheData{{2017/08/12 bob le bricoleur, Traditional Cache (2.5/1.5)}}\begin{cacheText}La cache nous attend bien sagement aux pied. Le quartier est calme on peut loguer facilement.Merci 😀\end{cacheText}

\cacheNumber{503}\needspace{5\baselineskip}\cacheName{\href{http://coord.info/GC70WXQ}{Atlantic Event 2017 : Saint-Viaud} — \href{http://coord.info/GC70WXQ\Number{}710886667}{503}}\cacheData{{2017/08/12 duj\Underscore{}family, Event Cache (1/1)}}\begin{cacheText}Nous avons passé une excellente journée et avons appris une multitude de choses sur le geocaching.  Un grand merci aux organisateurs qui nous ont très bien accueilli et merci pour leur délicate attention .Nous rentrons avec de beaux souvenirs ( géocoins et nouvelle cache gagnée à la perombola!!!). Les circuits proposés étaient très variés et très plaisants.  Bon rétablissement à Vincent....À l'année prochaine\end{cacheText}

\cacheNumber{504}\needspace{5\baselineskip}\cacheName{\href{http://coord.info/GC75B37}{\Number{}01 Bords de Loire - Le Ponton [AE2017]} — \href{http://coord.info/GC75B37\Number{}710935401}{504}}\cacheData{{2017/08/12 lesdecouvreurs, Traditional Cache (2.5/2.5)}}\begin{cacheText}En compagnie de Fabilab et sur nos vélos nous attaquons la boucle des Bords de Loire. C'est l'œil avertit de Stephane qui repère le petit truc. L'indice nous a bien aidé. Merci pour le travail de pose 😀\end{cacheText}

\cacheNumber{505}\needspace{5\baselineskip}\cacheName{\href{http://coord.info/GC75B3K}{\Number{}02 Bords de Loire - Cui-cui junior [AE2017]} — \href{http://coord.info/GC75B3K\Number{}710936380}{505}}\cacheData{{2017/08/12 lesdecouvreurs, Traditional Cache (2/4.5)}}\begin{cacheText}[{STF}]

Nous avons mis un peu de temps avant de la découvrir ( trop pressés nous n'avions pas lu l'indice) mais avons finit par la repérer et c'est Stephane qui se dévout pour la récupérer.Cette cache mérite un PF Merci 😃\end{cacheText}

\cacheNumber{506}\needspace{5\baselineskip}\cacheName{\href{http://coord.info/GC75B44}{\Number{}03 Bords de Loire - Pin parasol [AE2017]} — \href{http://coord.info/GC75B44\Number{}711045408}{506}}\cacheData{{2017/08/12 lesdecouvreurs, Traditional Cache (2.5/2)}}\begin{cacheText}Nous avons cherché un long moment et étions sur le point d'abandonner quand enfin nous mettons la main dessus ouffff Merci 😊\end{cacheText}

\cacheNumber{507}\needspace{5\baselineskip}\cacheName{\href{http://coord.info/GC75B47}{\Number{}04 Bords de Loire - Point de vue [AE2017]} — \href{http://coord.info/GC75B47\Number{}711047507}{507}}\cacheData{{2017/08/12 lesdecouvreurs, Traditional Cache (2/2)}}\begin{cacheText}Le crachin s'est arrêté et nous commençons à avoir chaud avec nos vélos.... pas de problème particulier sur cette cache. Merci 😊\end{cacheText}

\cacheNumber{508}\needspace{5\baselineskip}\cacheName{\href{http://coord.info/GC75B4C}{\Number{}05 Bords de Loire [AE2017]} — \href{http://coord.info/GC75B4C\Number{}711049537}{508}}\cacheData{{2017/08/12 lesdecouvreurs, Traditional Cache (1.5/2)}}\begin{cacheText}Arrivés sur les lieux les premiers, les garçons découvrent la cache en deux temps trois mouvements.... et repartent vers d'autres caches nous laissant signer!!!! Merci 😊\end{cacheText}

\cacheNumber{509}\needspace{5\baselineskip}\cacheName{\href{http://coord.info/GC75B4H}{\Number{}06 Bords de Loire [AE2017]} — \href{http://coord.info/GC75B4H\Number{}711050798}{509}}\cacheData{{2017/08/12 lesdecouvreurs, Traditional Cache (2.5/1.5)}}\begin{cacheText}La cache nous attend bien à sa place ... Merci 😊\end{cacheText}

\cacheNumber{510}\needspace{5\baselineskip}\cacheName{\href{http://coord.info/GC75B4R}{\Number{}07 Bords de Loire [AE2017]} — \href{http://coord.info/GC75B4R\Number{}711061076}{510}}\cacheData{{2017/08/12 lesdecouvreurs, Multi-cache (2/1.5)}}\begin{cacheText}Nous trouvons très facilement les réponses de cette multi (4 cerveaux en ébullition !!!) et n'avons aucun problème pour trouver la cache Merci 😊\end{cacheText}

\cacheNumber{511}\needspace{5\baselineskip}\cacheName{\href{http://coord.info/GC75B4T}{\Number{}08 Bords de Loire [AE2017]} — \href{http://coord.info/GC75B4T\Number{}711127100}{511}}\cacheData{{2017/08/12 lesdecouvreurs, Traditional Cache (1.5/1.5)}}\begin{cacheText}[{STF}]

Nous arrivons au PZ et sortons le logbook de la cache... Nath44250 et Kallin 44 sont sur nos talons... on doit accélérer pour ne pas gâcher leurs découvertes. Merci 😊\end{cacheText}

\cacheNumber{512}\needspace{5\baselineskip}\cacheName{\href{http://coord.info/GC75B4Z}{\Number{}09 Bords de Loire [AE2017]} — \href{http://coord.info/GC75B4Z\Number{}711129698}{512}}\cacheData{{2017/08/12 lesdecouvreurs, Traditional Cache (3/1.5)}}\begin{cacheText}{[STF]}

On accélère le rythme... ouf la cache ne nous résiste pas bien longtemps. Merci 😊\end{cacheText}

\cacheNumber{513}\needspace{5\baselineskip}\cacheName{\href{http://coord.info/GC75B54}{\Number{}10 Bords de Loire - L'arrêt de bus [AE2017]} — \href{http://coord.info/GC75B54\Number{}711134033}{513}}\cacheData{{2017/08/12 lesdecouvreurs, Traditional Cache (1.5/1.5)}}\begin{cacheText}[{STF}]

Nous arrivons essoufflés sur les lieux mais nous mettons de la distance avec les autres géocacheurs . La cache est facilement trouvée et nous repartons sur les chapeaux de roues... Merci 😊\end{cacheText}

\cacheNumber{514}\needspace{5\baselineskip}\cacheName{\href{http://coord.info/GC75B5C}{\Number{}11 Bords de Loire - Ghostbuster [AE2017]} — \href{http://coord.info/GC75B5C\Number{}711120548}{514}}\cacheData{{2017/08/12 lesdecouvreurs, Traditional Cache (1.5/1.5)}}\begin{cacheText}[{FTF}]

Arrivés sur les lieux nous ne comprenons pas trop l'indice mais lorsque nous dénichons la Belle, Il prend tout son sens. Merci 😊\end{cacheText}

\cacheNumber{515}\needspace{5\baselineskip}\cacheName{\href{http://coord.info/GC75B5H}{\Number{}12 Bords de Loire [AE2017]} — \href{http://coord.info/GC75B5H\Number{}711123424}{515}}\cacheData{{2017/08/12 lesdecouvreurs, Traditional Cache (1.5/2)}}\begin{cacheText}[{STF}]

Cette cache ne nous cause aucun problème... Merci 😊\end{cacheText}

\cacheNumber{516}\needspace{5\baselineskip}\cacheName{\href{http://coord.info/GC75B5W}{\Number{}13 Bords de Loire [AE2017]} — \href{http://coord.info/GC75B5W\Number{}711141172}{516}}\cacheData{{2017/08/12 lesdecouvreurs, Traditional Cache (1.5/1.5)}}\begin{cacheText}[{STF]}

Les garçons sont arrivés les premiers et ont vite découvert la belle. Pas de soucis particulier. Merci 😊\end{cacheText}

\cacheNumber{517}\needspace{5\baselineskip}\cacheName{\href{http://coord.info/GC75B63}{\Number{}14 Bords de Loire [AE2017]} — \href{http://coord.info/GC75B63\Number{}711146964}{517}}\cacheData{{2017/08/12 lesdecouvreurs, Multi-cache (2/2)}}\begin{cacheText}[{FTF}]

Arrivés au PZ les 4 cerveaux se remettent en ébullition... la question du C nous fait quelque peu hésiter mais on tente.... BINGO , la cache nous attend. Merci 😊\end{cacheText}

\cacheNumber{518}\needspace{5\baselineskip}\cacheName{\href{http://coord.info/GC75B6B}{\Number{}15 Bords de Loire - Lettres de mon moulin[AE2017]} — \href{http://coord.info/GC75B6B\Number{}711151193}{518}}\cacheData{{2017/08/12 lesdecouvreurs, Letterbox Hybrid (1.5/1.5)}}\begin{cacheText}[{STF}]

Nous avons perdu pas mal de temps sur les Lettres de mon moulin. Nous allions renoncer lorsque Fabienne la découvre enfin à 21 mètres des coordonnées du GPS. Ouffff.A cause de la pluie du début d'après midi nous n'avons pas fait suivre les cartes postales et les bidules, donc pas de souvenir de notre passage . Merci 😊\end{cacheText}

\cacheNumber{519}\needspace{5\baselineskip}\cacheName{\href{http://coord.info/GC75B6F}{\Number{}16 Bords de Loire - Maître corbeau [AE2017]} — \href{http://coord.info/GC75B6F\Number{}711159277}{519}}\cacheData{{2017/08/12 lesdecouvreurs, Traditional Cache (2.5/3.5)}}\begin{cacheText}[{FTF}]

Et nous revoilà grimper aux arbres ... la cache est bien la. Merci 😊\end{cacheText}

\cacheNumber{520}\needspace{5\baselineskip}\cacheName{\href{http://coord.info/GC75B6N}{\Number{}17 Bords de Loire [AE2017]} — \href{http://coord.info/GC75B6N\Number{}711165293}{520}}\cacheData{{2017/08/12 lesdecouvreurs, Traditional Cache (1.5/2)}}\begin{cacheText}[{FTF}]

Pas de difficulté sur celle ci... on file pour essayer de finir le parcours avant le rendez-vous de 19 h. Merci 😊\end{cacheText}

\cacheNumber{521}\needspace{5\baselineskip}\cacheName{\href{http://coord.info/GC75B6Y}{\Number{}18 Bords de Loire - La Profissais [AE2017]} — \href{http://coord.info/GC75B6Y\Number{}711167489}{521}}\cacheData{{2017/08/12 lesdecouvreurs, Traditional Cache (1.5/1)}}\begin{cacheText}[{STF}]

La cache nous attend ... pas de soucis . Merci 😊\end{cacheText}

\cacheNumber{522}\needspace{5\baselineskip}\cacheName{\href{http://coord.info/GC75B73}{\Number{}19 Bords de Loire [AE2017]} — \href{http://coord.info/GC75B73\Number{}711169494}{522}}\cacheData{{2017/08/12 lesdecouvreurs, Traditional Cache (2/2)}}\begin{cacheText}[{FTF]}

Ici non plus pas de problème pour trouver la cache . Merci 😊\end{cacheText}

\cacheNumber{523}\needspace{5\baselineskip}\cacheName{\href{http://coord.info/GC75B79}{\Number{}20 Bords de Loire - Noé des prés [AE2017]} — \href{http://coord.info/GC75B79\Number{}711171851}{523}}\cacheData{{2017/08/12 lesdecouvreurs, Traditional Cache (1.5/2.5)}}\begin{cacheText}[{STF]}

Les garçons ,arrivés sur les lieux ,ont délogé la Belle et sont partis au point suivant. Il ne nous reste qu'à signer !!! Merci 😊\end{cacheText}

\cacheNumber{524}\needspace{5\baselineskip}\cacheName{\href{http://coord.info/GC75B7H}{\Number{}21 Bords de Loire [AE2017]} — \href{http://coord.info/GC75B7H\Number{}711173700}{524}}\cacheData{{2017/08/12 lesdecouvreurs, Traditional Cache (3/2)}}\begin{cacheText}[{STF}]

La cache est bien à sa place... nous continuons. Merci 😊\end{cacheText}

\cacheNumber{525}\needspace{5\baselineskip}\cacheName{\href{http://coord.info/GC75B7P}{\Number{}22 Bords de Loire - le puits [AE2017]} — \href{http://coord.info/GC75B7P\Number{}711177096}{525}}\cacheData{{2017/08/12 lesdecouvreurs, Traditional Cache (1.5/1.5)}}\begin{cacheText}[{FTF}]

Quel grand puits !!! Nous ,qui sommes assoiffés , aurions bien bu un coup!!  Ne pas se laisser disperser, le temps passe!!! La cache est bien la !!! Merci 😊\end{cacheText}

\cacheNumber{526}\needspace{5\baselineskip}\cacheName{\href{http://coord.info/GC75B7X}{\Number{}23 Bords de Loire - le tube [AE2017]} — \href{http://coord.info/GC75B7X\Number{}711179607}{526}}\cacheData{{2017/08/12 lesdecouvreurs, Unknown Cache (2.5/2.5)}}\begin{cacheText}[{FTF}]

Les vagues, les vagues.... ça y est!!!  Nous sommes 3 à compter les vagues. Ok pour le nombre de vagues la cache est vite trouvée. Merci 😊\end{cacheText}

\cacheNumber{527}\needspace{5\baselineskip}\cacheName{\href{http://coord.info/GC75B84}{\Number{}24 Bords de Loire - le puits [AE2017]} — \href{http://coord.info/GC75B84\Number{}711184211}{527}}\cacheData{{2017/08/12 lesdecouvreurs, Traditional Cache (1.5/1.5)}}\begin{cacheText}[{STF}]

Nous sommes complètement dessèchés et la présence d'eau non potable est une torture!!! Nous restons malgré tout concentré et trouvons la belle. Merci 😊\end{cacheText}

\cacheNumber{528}\needspace{5\baselineskip}\cacheName{\href{http://coord.info/GC75B8B}{\Number{}25 Bords de Loire - les cigognes [AE2017]} — \href{http://coord.info/GC75B8B\Number{}711186907}{528}}\cacheData{{2017/08/12 lesdecouvreurs, Multi-cache (2/2)}}\begin{cacheText}[{FTF}]

À nouveau nous nous concentrons pour trouver les coordonnées. Vite fait nous partons voir la Belle . Merci 😊\end{cacheText}

\cacheNumber{529}\needspace{5\baselineskip}\cacheName{\href{http://coord.info/GC75B8J}{\Number{}26 Bords de Loire [AE2017]} — \href{http://coord.info/GC75B8J\Number{}711191339}{529}}\cacheData{{2017/08/12 lesdecouvreurs, Traditional Cache (2/2.5)}}\begin{cacheText}[{STF}]

Celle ci est vite délogée .... et quelle surprise !!!! Un PF pour elle... on a bien rigolé. Merci 😊\end{cacheText}

\cacheNumber{530}\needspace{5\baselineskip}\cacheName{\href{http://coord.info/GC75B8P}{\Number{}27 Bords de Loire [AE2017]} — \href{http://coord.info/GC75B8P\Number{}711193573}{530}}\cacheData{{2017/08/12 lesdecouvreurs, Traditional Cache (1.5/1.5)}}\begin{cacheText}[{FTF}]

Heureusement que certaines personnes s'intéressent au foot!!! La cache nous attend bien sagement APDP. Mplc 😀\end{cacheText}

\cacheNumber{531}\needspace{5\baselineskip}\cacheName{\href{http://coord.info/GC75B8Y}{\Number{}28 Bords de Loire [AE2017]} — \href{http://coord.info/GC75B8Y\Number{}711196333}{531}}\cacheData{{2017/08/12 lesdecouvreurs, Traditional Cache (2/1.5)}}\begin{cacheText}[{STF}]

Un classique du jeu... on adore !!! Un PF pour elle également.Merci 😊\end{cacheText}

\cacheNumber{532}\needspace{5\baselineskip}\cacheName{\href{http://coord.info/GC75B92}{\Number{}44 Bords de Loire [AE2017]} — \href{http://coord.info/GC75B92\Number{}711228541}{532}}\cacheData{{2017/08/12 lesdecouvreurs, Traditional Cache (1.5/1.5)}}\begin{cacheText}Heureusement que celle ci n'est pas compliquée : l'équipage est mort de soif et rêve de pouvoir se désaltérer !!! Merci 😊\end{cacheText}

\cacheNumber{533}\needspace{5\baselineskip}\cacheName{\href{http://coord.info/GC75B98}{\Number{}45 Bords de Loire [AE2017]} — \href{http://coord.info/GC75B98\Number{}711229831}{533}}\cacheData{{2017/08/12 lesdecouvreurs, Traditional Cache (2.5/4)}}\begin{cacheText}Encore un ultime effort pour terminer le parcours... le logbook est vite signé et nous repartons... Merci 😊\end{cacheText}

\cacheNumber{534}\needspace{5\baselineskip}\cacheName{\href{http://coord.info/GC77FQ8}{PR LES HAUTEURS 02 [AE2017]} — \href{http://coord.info/GC77FQ8\Number{}710931941}{534}}\cacheData{{2017/08/12 Les GéotrouveTout44, Traditional Cache (1.5/2)}}\begin{cacheText}C est lors de l'Atlantic Évent et en attendant la cache de nuit que nous décidons avec Fabilab de chercher cette cache. Elle est débusquée en deux temps trois mouvements. ..Merci pour tout ce travail 😀\end{cacheText}

\cacheNumber{535}\needspace{5\baselineskip}\cacheName{\href{http://coord.info/GC77FQM}{PR LES HAUTEURS 03 [AE2017]} — \href{http://coord.info/GC77FQM\Number{}710931872}{535}}\cacheData{{2017/08/12 Les GéotrouveTout44, Traditional Cache (1.5/2)}}\begin{cacheText}Il reste un peu de temps avant la cache de nuit proposée lors de l'Atlantic Évent... en compagnie de Fabilab nous trouvons assez rapidement la cache. Merci pour le travail de pose\end{cacheText}

\cacheNumber{536}\needspace{5\baselineskip}\cacheName{\href{http://coord.info/GC78W27}{Patinodrome [Atlantic Event 2017]} — \href{http://coord.info/GC78W27\Number{}710934824}{536}}\cacheData{{2017/08/12 KST44, Traditional Cache (2/1)}}\begin{cacheText}{[FTT]}

C'est en compagnie de Fabilab et lors de l'Atlantic Évent que nous découvrons la cache. Un mariage se déroule à côté et nous devons être discret..... chose faite nous repartons vers d'autres caches à vélo sous le crachin !!! Merci 😊 pour le travail.\end{cacheText}

\cacheNumber{537}\needspace{5\baselineskip}\cacheName{\href{http://coord.info/GC78X6W}{Maison des Jeunes [Atlantic Event 2017]} — \href{http://coord.info/GC78X6W\Number{}710931773}{537}}\cacheData{{2017/08/12 KST44, Traditional Cache (1.5/1)}}\begin{cacheText}[{STF}]

Après avoir passé une matinée à l'abri à découvrir divers outils liés au géocaching et après avoir pique niqué, nous nous lançons dans l'aventure en compagnie de Fabilab. Le crachin est de la partie mais cela ne nous empêche pas de débusquer la Belle . Merci pour le travail 😃\end{cacheText}

\cacheNumber{538}\needspace{5\baselineskip}\cacheName{\href{http://coord.info/GC78X73}{Le lac de Saint-Viaud [Atlantic Event 2017]} — \href{http://coord.info/GC78X73\Number{}711372426}{538}}\cacheData{{2017/08/12 KST44, Traditional Cache (1.5/1)}}\begin{cacheText}En compagnie de Fabilab nous terminons le parcours du bord de Loire,nous nous devons de chercher celle la!!! Vite trouvée nous logons avant de partir à la pérombola.Merci pour ce travail de pose\end{cacheText}

\cacheNumber{539}\needspace{5\baselineskip}\cacheName{\href{http://coord.info/GC78X8E}{Atlantic Event 2017 (cache de nuit)} — \href{http://coord.info/GC78X8E\Number{}710932971}{539}}\cacheData{{2017/08/12 KST44, Unknown Cache (3.5/1.5)}}\begin{cacheText}23h00 ...enfin nous partons  en compagnie d'une vingtaine de personnes à la recherche de la night( 4 groupes)... nous faisons le tour du lac.... aucun indice !! Arrivés vers le restaurant nous trouvons un indice et nous suivons le chemin comme le Petit Poucet jusqu'au camion.Au final la cache est trouvée collectivement. Merci 😊 nous avons passés une excellente soirée\end{cacheText}

\cacheNumber{540}\needspace{5\baselineskip}\cacheName{\href{http://coord.info/GC79RQK}{\Number{}30 Bords de Loire - le petit chêne [AE2017]} — \href{http://coord.info/GC79RQK\Number{}711200885}{540}}\cacheData{{2017/08/12 lesdecouvreurs, Traditional Cache (2/1.5)}}\begin{cacheText}[{ FTF}]

La cache est très bien camouflée ... mais elle ne nous résiste pas!! Nous accélérons pour finir la boucle et aller boire... Merci 😊\end{cacheText}

\cacheNumber{541}\needspace{5\baselineskip}\cacheName{\href{http://coord.info/GC79RR2}{\Number{}31 Bords de Loire [AE2017]} — \href{http://coord.info/GC79RR2\Number{}711204901}{541}}\cacheData{{2017/08/12 lesdecouvreurs, Traditional Cache (2/1.5)}}\begin{cacheText}Cette cache a déjà été visitée.. fin des STF 😭😭😭. Pas de difficulté particulière. Merci 😊\end{cacheText}

\cacheNumber{542}\needspace{5\baselineskip}\cacheName{\href{http://coord.info/GC79RRC}{\Number{}32 Bords de Loire [AE2017]} — \href{http://coord.info/GC79RRC\Number{}711207345}{542}}\cacheData{{2017/08/12 lesdecouvreurs, Traditional Cache (1.5/1.5)}}\begin{cacheText}La course contre la montre à demarré... pas de temps à perdre !!! La Belle est vite délogée. Merci 😊\end{cacheText}

\cacheNumber{543}\needspace{5\baselineskip}\cacheName{\href{http://coord.info/GC79RRK}{\Number{}33 Bords de Loire [AE2017]} — \href{http://coord.info/GC79RRK\Number{}711208694}{543}}\cacheData{{2017/08/12 lesdecouvreurs, Traditional Cache (1.5/2)}}\begin{cacheText}Nous continuons nos recherches et restons concentrés : la cache ne nous résiste pas. Merci 😊\end{cacheText}

\cacheNumber{544}\needspace{5\baselineskip}\cacheName{\href{http://coord.info/GC79RRT}{\Number{}34 Bords de Loire [AE2017]} — \href{http://coord.info/GC79RRT\Number{}711209710}{544}}\cacheData{{2017/08/12 lesdecouvreurs, Traditional Cache (1.5/1.5)}}\begin{cacheText}Ici aussi nous trouvons la boîte facilement... Merci 😊\end{cacheText}

\cacheNumber{545}\needspace{5\baselineskip}\cacheName{\href{http://coord.info/GC79RT1}{\Number{}35 Bords de Loire [AE2017]} — \href{http://coord.info/GC79RT1\Number{}711211512}{545}}\cacheData{{2017/08/12 lesdecouvreurs, Traditional Cache (1.5/1.5)}}\begin{cacheText}L'indice nous indique l'endroit ou chercher ... au la suivante ... Merci 😊\end{cacheText}

\cacheNumber{546}\needspace{5\baselineskip}\cacheName{\href{http://coord.info/GC79RT7}{\Number{}36 Bords de Loire [AE2017]} — \href{http://coord.info/GC79RT7\Number{}711213154}{546}}\cacheData{{2017/08/12 lesdecouvreurs, Traditional Cache (2.5/2)}}\begin{cacheText}Cache vraiment sympa... Merci 😊\end{cacheText}

\cacheNumber{547}\needspace{5\baselineskip}\cacheName{\href{http://coord.info/GC79RTE}{\Number{}37 Bords de Loire [AE2017]} — \href{http://coord.info/GC79RTE\Number{}711215125}{547}}\cacheData{{2017/08/12 lesdecouvreurs, Traditional Cache (1.5/1.5)}}\begin{cacheText}Pas de problème majeur pour débusquer la Belle. Merci 😊\end{cacheText}

\cacheNumber{548}\needspace{5\baselineskip}\cacheName{\href{http://coord.info/GC79RTP}{\Number{}38 Bords de Loire - Ste Marie [AE2017]} — \href{http://coord.info/GC79RTP\Number{}711217630}{548}}\cacheData{{2017/08/12 lesdecouvreurs, Traditional Cache (1.5/1.5)}}\begin{cacheText}[{FTF}]

Grande surprise... les autres ont abandonné le circuit !!! Pas de soucis pour Ste Marie... Merci 😊\end{cacheText}

\cacheNumber{549}\needspace{5\baselineskip}\cacheName{\href{http://coord.info/GC79RTW}{\Number{}39 Bords de Loire [AE2017]} — \href{http://coord.info/GC79RTW\Number{}711218759}{549}}\cacheData{{2017/08/12 lesdecouvreurs, Traditional Cache (2/1.5)}}\begin{cacheText}[{STF}]

Excellent camouflage..... un PF pour cette cache. Merci 😊\end{cacheText}

\cacheNumber{550}\needspace{5\baselineskip}\cacheName{\href{http://coord.info/GC79RWD}{\Number{}41 Bords de Loire [AE2017]} — \href{http://coord.info/GC79RWD\Number{}711220724}{550}}\cacheData{{2017/08/12 lesdecouvreurs, Traditional Cache (1.5/1.5)}}\begin{cacheText}[{STF}]

La cache est bien APDP. MPLC 😂\end{cacheText}

\cacheNumber{551}\needspace{5\baselineskip}\cacheName{\href{http://coord.info/GC79RWN}{\Number{}42 Bords de Loire [AE2017]} — \href{http://coord.info/GC79RWN\Number{}711222366}{551}}\cacheData{{2017/08/12 lesdecouvreurs, Traditional Cache (1.5/1.5)}}\begin{cacheText}[{STF}]

Il nous semble avoir déjà fait une cache comme celle la.... elle attend au  même endroit\end{cacheText}

\cacheNumber{552}\needspace{5\baselineskip}\cacheName{\href{http://coord.info/GC79RWR}{\Number{}43 Bords de Loire [AE2017]} — \href{http://coord.info/GC79RWR\Number{}711226837}{552}}\cacheData{{2017/08/12 lesdecouvreurs, Traditional Cache (1.5/1.5)}}\begin{cacheText}[{FTF}]

Nous la trouvons en deux temps trois mouvements et serons à l'heure pour l'apéro !!! Merci 😊\end{cacheText}

\cacheNumber{553}\needspace{5\baselineskip}\cacheName{\href{http://coord.info/GC6J9XG}{Virée à Saint-Viaud [AE2016]} — \href{http://coord.info/GC6J9XG\Number{}711055665}{553}}\cacheData{{2017/08/12 duj\Underscore{}family, Multi-cache (5/2.5)}}\begin{cacheText}En arrivant sur les lieux nous trouvons Nath 44250 et Kallin 44 entrain de réfléchir à la cache.... c'est donc en équipe que nous trouvons la solution . Merci 😊\end{cacheText}

\cacheNumber{554}\needspace{5\baselineskip}\cacheName{\href{http://coord.info/GC6HXX1}{Vue sur l'Eolienne [AE2016]} — \href{http://coord.info/GC6HXX1\Number{}710933951}{554}}\cacheData{{2017/08/12 lesdecouvreurs, Traditional Cache (1/4)}}\begin{cacheText}C'est lors de l'Atlantic Évent et toujours en compagnie de Fabilab que nous débusquons la Belle. L'agilité est de mise pour récupérer la boîte mais nous y arrivons !!!! La vue de l'éolienne est vraiment chouette. Merci 😊\end{cacheText}

\cacheNumber{555}\needspace{5\baselineskip}\cacheName{\href{http://coord.info/GC6Q26A}{LBP\Underscore{}PN44} — \href{http://coord.info/GC6Q26A\Number{}710933336}{555}}\cacheData{{2017/08/13 duj\Underscore{}family, Traditional Cache (1.5/1.5)}}\begin{cacheText}C'est après le pique nique du soir et en attendant la cache de nuit de l'Atlantic Évent que nous décidons de faire cette cache. Les coordonnées sont précises et nous permettent de découvrir rapidement la Belle. Merci 😊\end{cacheText}

\cacheNumber{556}\needspace{5\baselineskip}\cacheName{\href{http://coord.info/GC6Q266}{LBP\Underscore{}PN42} — \href{http://coord.info/GC6Q266\Number{}710817389}{556}}\cacheData{{2017/08/13 duj\Underscore{}family, Traditional Cache (1.5/1.5)}}\begin{cacheText}Nous continuons nos recherches en attendant les nights caches de l'event de Saint Viaud. Celle ci est trouvée sans problème. Merci 😊\end{cacheText}

\cacheNumber{557}\needspace{5\baselineskip}\cacheName{\href{http://coord.info/GC2VQEX}{La tour FL 241 de st-brévin} — \href{http://coord.info/GC2VQEX\Number{}711387198}{557}}\cacheData{{2017/08/13 les cinq, Traditional Cache (1.5/1.5)}}\begin{cacheText}Après l'Atlantic Évent d'hier, nous en profitons pour faire d'autres caches sur la région avant notre départ. La cache est vite trouvée et signée à l'abris des regards indiscrets !!! La villa de la duchesse Anne qui jouée la cache, est magnifique . Merci 😊\end{cacheText}

\cacheNumber{558}\needspace{5\baselineskip}\cacheName{\href{http://coord.info/GC3DAK4}{Où sont Bobine de Cuivre \And{} Bob le Bricoleur \Number{}6} — \href{http://coord.info/GC3DAK4\Number{}711389701}{558}}\cacheData{{2017/08/13 bob le bricoleur, Traditional Cache (3/1.5)}}\begin{cacheText}En week-end pour participer à l Atlantic Évent nous en profitons pour faire encore quelques caches avant le départ.Très joli endroit : le serpent de mer est impressionnant !!! Nous trouvons très vite la cache qui est visible,le collier en plastique a été cassé (besoin de maintenance). Je la cache comme je peux!!!! Merci 😊\end{cacheText}

\cacheNumber{559}\needspace{5\baselineskip}\cacheName{\href{http://coord.info/GC3N2KK}{La porte du Lazaret de mindin} — \href{http://coord.info/GC3N2KK\Number{}711389720}{559}}\cacheData{{2017/08/13 les cinq, Traditional Cache (2/1.5)}}\begin{cacheText}Dernières caches avant notre départ, nous sommes venus pour l'Atlantic Évent. Ces portes sont magnifiques , dommage qu'elles ne soient pas mieux entretenues !!! La cache nous donné du fil à retordre mais nous finissons par la déloger. Merci pour cette cache.\end{cacheText}

\cacheNumber{560}\needspace{5\baselineskip}\cacheName{\href{http://coord.info/GC3N2P9}{Le cimetière des pecheries} — \href{http://coord.info/GC3N2P9\Number{}711389816}{560}}\cacheData{{2017/08/13 les cinq, Traditional Cache (2/2)}}\begin{cacheText}Dernières caches avant notre départ... venus pour l'Atlantic Évent nous en redemandons!!!! Ici la cache est bien dissimulée mais nous finissons par mettre la main dessus. Merci pour la découverte de ce lieu\end{cacheText}

\cacheNumber{561}\needspace{5\baselineskip}\cacheName{\href{http://coord.info/GC50FET}{Communes de Loire-Atlantique : St Brevin-les-Pins} — \href{http://coord.info/GC50FET\Number{}711396020}{561}}\cacheData{{2017/08/13 vinc20100, Traditional Cache (2/1.5)}}\begin{cacheText}Dernière cache avant notre départ !!! Quelle surprise que de trouver ce menhir !!! La cache est vite trouvée : elle est parterre. Nous mettons le logbook dans un plastique et essayons de la camoufler. Merci 😊\end{cacheText}

\cacheNumber{562}\needspace{5\baselineskip}\cacheName{\href{http://coord.info/GC1H4KV}{AutoStop : A83 Sud - Grissay} — \href{http://coord.info/GC1H4KV\Number{}711071824}{562}}\cacheData{{2017/08/13 njda78, Traditional Cache (2/1.5)}}\begin{cacheText}c'est sur le retour de l'Atlantic Évent que nous nous arrêtons faire une petite pause ....chouette une cache !!!Celle-ci est vite trouvée. Merci 😊\end{cacheText}

\cacheNumber{563}\needspace{5\baselineskip}\cacheName{\href{http://coord.info/GC4BNM2}{Les Fesses du Diable – Saint Brévin les Pins} — \href{http://coord.info/GC4BNM2\Number{}711086944}{563}}\cacheData{{2017/08/13 FREDAGNES, Multi-cache (2/1)}}\begin{cacheText}Arrivés sur Saint Brevin pour l'Atlantic Évent on ne peut s'empêcher de visiter la ville. Les Fabilab nous servent de guide..... et nous montrent quelques caches. Cette multi est vite résolue.... trouver la nano n'est pas aussi simple mais on finit par mettre la main dessus. Merci 😊\end{cacheText}

\cacheNumber{564}\needspace{5\baselineskip}\cacheName{\href{http://coord.info/GC79RTZ}{\Number{}40 Bords de Loire [AE2017]} — \href{http://coord.info/GC79RTZ\Number{}711219947}{564}}\cacheData{{2017/08/13 lesdecouvreurs, Traditional Cache (2/2.5)}}\begin{cacheText}[{STF}]

Nous cherchons un petit moment et enfin la Belle se dévoile . Merci 😊\end{cacheText}

\cacheNumber{565}\needspace{5\baselineskip}\cacheName{\href{http://coord.info/GC3G24A}{GR8\Number{}50} — \href{http://coord.info/GC3G24A\Number{}711795490}{565}}\cacheData{{2017/08/14 Peyo64, Traditional Cache (2/1.5)}}\begin{cacheText}Après la bonus de la partie 1 du GR8 nous reprenons la deuxième partie du GR 8. Cette cache est trouvée sans aucune difficulté . Merci 😊\end{cacheText}

\cacheNumber{566}\needspace{5\baselineskip}\cacheName{\href{http://coord.info/GC3G2W9}{GR8\Number{}53} — \href{http://coord.info/GC3G2W9\Number{}711795752}{566}}\cacheData{{2017/08/14 Peyo64, Traditional Cache (2/1.5)}}\begin{cacheText}ici,il n'y a pas de difficulté :nous trouvons le logbook à sa place .Merci 😊\end{cacheText}

\cacheNumber{567}\needspace{5\baselineskip}\cacheName{\href{http://coord.info/GC3G2WD}{GR8\Number{}54} — \href{http://coord.info/GC3G2WD\Number{}711796204}{567}}\cacheData{{2017/08/14 Peyo64, Traditional Cache (1.5/2)}}\begin{cacheText}Arrivés sur les lieux nous cherchons désespérément une souche : absolument rien !!!! C'est en lisant les commentaires précédents que nous trouvons la solution . Merci Thibaldus 6667: l'indice donné n'est plus valable!!!Merci 😊\end{cacheText}

\cacheNumber{568}\needspace{5\baselineskip}\cacheName{\href{http://coord.info/GC3G2WG}{GR8\Number{}55} — \href{http://coord.info/GC3G2WG\Number{}711796446}{568}}\cacheData{{2017/08/14 Peyo64, Traditional Cache (2/1.5)}}\begin{cacheText}Arrivés sur les lieux ,nous sommes obligés d'attendre car les ouvriers de l'équipement font leur pause syndicale :celle-ci semble s'éterniser... Après avoir fait la cache suivant nous revenons sur nos pas et il n'y a plus personne .Cela nous permet de chercher et de trouver tranquillement .Merci 😊\end{cacheText}

\cacheNumber{569}\needspace{5\baselineskip}\cacheName{\href{http://coord.info/GC3G2WK}{GR8\Number{}56} — \href{http://coord.info/GC3G2WK\Number{}711796900}{569}}\cacheData{{2017/08/14 Peyo64, Traditional Cache (1.5/1.5)}}\begin{cacheText}Nous continuons le parcours ...la cache est bien en place . Merci 😊\end{cacheText}

\cacheNumber{570}\needspace{5\baselineskip}\cacheName{\href{http://coord.info/GC3G2WQ}{GR8\Number{}57} — \href{http://coord.info/GC3G2WQ\Number{}711796901}{570}}\cacheData{{2017/08/14 Peyo64, Traditional Cache (2/1.5)}}\begin{cacheText}Le tour de l'arbre est jonché d'excréments !!!! Pas très sympa !!!!Nous finissons par mettre la main dessus,malgré tout ,avec une paire de gants !!!!Merci 😊\end{cacheText}

\cacheNumber{571}\needspace{5\baselineskip}\cacheName{\href{http://coord.info/GC3G2WT}{GR8\Number{}58} — \href{http://coord.info/GC3G2WT\Number{}711796902}{571}}\cacheData{{2017/08/14 Peyo64, Traditional Cache (1.5/1.5)}}\begin{cacheText}Nous redoutons un DNF .... mais après avoir dégoté l'arbre couché et bien fouillé la zone nous finissons par déloger la Belle. Merci 😊\end{cacheText}

\cacheNumber{572}\needspace{5\baselineskip}\cacheName{\href{http://coord.info/GC3G2X1}{GR8\Number{}60} — \href{http://coord.info/GC3G2X1\Number{}711796961}{572}}\cacheData{{2017/08/14 Peyo64, Traditional Cache (1.5/1.5)}}\begin{cacheText}Pas de difficulté grâce a l'indice :nous trouvons la cache en deux temps trois mouvements . Le logbook n'est pas d'origine ,la maintenance a été faite par Spirou 43 et Poune 43 merci à tous 😃\end{cacheText}

\cacheNumber{573}\needspace{5\baselineskip}\cacheName{\href{http://coord.info/GC3G2X5}{GR8\Number{}61} — \href{http://coord.info/GC3G2X5\Number{}711796997}{573}}\cacheData{{2017/08/14 Peyo64, Traditional Cache (1.5/1.5)}}\begin{cacheText}Arrivés devant les sextuplés nous délogeons sans problème la Belle . Merci 😊\end{cacheText}

\cacheNumber{574}\needspace{5\baselineskip}\cacheName{\href{http://coord.info/GC3G2X7}{GR8\Number{}62} — \href{http://coord.info/GC3G2X7\Number{}711797127}{574}}\cacheData{{2017/08/14 Peyo64, Traditional Cache (2/1.5)}}\begin{cacheText}Ici ,nous reprenons le drive .Avec la chaleur qu'il fait nous apprécions la clim ...la cache est vite repérée  grâce a la photo . Merci 😊\end{cacheText}

\cacheNumber{575}\needspace{5\baselineskip}\cacheName{\href{http://coord.info/GC3G2X9}{GR8\Number{}63} — \href{http://coord.info/GC3G2X9\Number{}711797128}{575}}\cacheData{{2017/08/14 Peyo64, Traditional Cache (1.5/1.5)}}\begin{cacheText}Vite fait ,bien fait ....le log book est plein je signe sur le côté . Merci 😊\end{cacheText}

\cacheNumber{576}\needspace{5\baselineskip}\cacheName{\href{http://coord.info/GC3G2XD}{GR8\Number{}64} — \href{http://coord.info/GC3G2XD\Number{}711797234}{576}}\cacheData{{2017/08/14 Peyo64, Traditional Cache (2/1.5)}}\begin{cacheText}Sous un soleil de plomb nous affairons à retourner toutes les pierres  le plus discrètement possible...et enfin nous mettons la main dessus !!!!Quel bonheur!!!! Merci 😊\end{cacheText}

\cacheNumber{577}\needspace{5\baselineskip}\cacheName{\href{http://coord.info/GC3G2XF}{GR8\Number{}65} — \href{http://coord.info/GC3G2XF\Number{}711797288}{577}}\cacheData{{2017/08/14 Peyo64, Traditional Cache (1.5/1.5)}}\begin{cacheText}Arrivés sur les lieux nous redoutons le DNF... grâce a l'indice et surtout la photo nous découvrons la cache rapidement . Merci 😊\end{cacheText}

\cacheNumber{578}\needspace{5\baselineskip}\cacheName{\href{http://coord.info/GC3G2XK}{GR8\Number{}66} — \href{http://coord.info/GC3G2XK\Number{}711797322}{578}}\cacheData{{2017/08/14 Peyo64, Traditional Cache (1.5/1.5)}}\begin{cacheText}Ici aussi c'est vite fait bien fait ....l'aventure continue . Merci 😊\end{cacheText}

\cacheNumber{579}\needspace{5\baselineskip}\cacheName{\href{http://coord.info/GC3G2Y2}{GR8\Number{}68} — \href{http://coord.info/GC3G2Y2\Number{}711797561}{579}}\cacheData{{2017/08/14 Peyo64, Traditional Cache (1.5/1.5)}}\begin{cacheText}Arrivés à la croix nous voyons le banc improvisé et nous cherchons ....et finissons par trouver !!!! Nous relevons l' indice .Merci 😊\end{cacheText}

\cacheNumber{580}\needspace{5\baselineskip}\cacheName{\href{http://coord.info/GC3G4N1}{Bonus GR8 1} — \href{http://coord.info/GC3G4N1\Number{}711636695}{580}}\cacheData{{2017/08/14 Peyo64, Unknown Cache (3.5/3.5)}}\begin{cacheText}Je découvre enfin la belle bien nichée dans sa souche.La première partie du GR 8 a été un vrai plaisir . 

Out : un géocoin 

In : rondelle de bois 

Merci Peyo pour tout ce travail de pose et de maintenance. Un PF pour l'ensemble du circuit 😊\end{cacheText}

\cacheNumber{581}\needspace{5\baselineskip}\cacheName{\href{http://coord.info/GC3G255}{GR8\Number{}51} — \href{http://coord.info/GC3G255\Number{}711795589}{581}}\cacheData{{2017/08/14 Peyo64, Traditional Cache (2/1.5)}}\begin{cacheText}Cette cache ne présente pas de difficultés particulières ,nous sommes en mode Power Trail :monsieur conduis et madame signe !!!Merci 😊\end{cacheText}

\cacheNumber{582}\needspace{5\baselineskip}\cacheName{\href{http://coord.info/GC3A34W}{\Number{}20 Sentier du Littoral} — \href{http://coord.info/GC3A34W\Number{}712021669}{582}}\cacheData{{2017/08/15 Peyo64, Traditional Cache (2/1.5)}}\begin{cacheText}Invitée sur Biarritz pour passer la journée je ne peux m'empêcher de faire quelques cache celle-ci est rapidement trouvée . Merci 😊\end{cacheText}

\cacheNumber{583}\needspace{5\baselineskip}\cacheName{\href{http://coord.info/GC2AC9P}{La citadelle} — \href{http://coord.info/GC2AC9P\Number{}712376060}{583}}\cacheData{{2017/08/16 Peyo64, Traditional Cache (2/2)}}\begin{cacheText}Arrivés sur les yeux on ne trouve pas !!!Il nous a suffit de lever la tête pour comprendre :mission réussie !!!Par contre pas d'objet voyageur dans le cache merci 😊\end{cacheText}

\cacheNumber{584}\needspace{5\baselineskip}\cacheName{\href{http://coord.info/GC5V5A2}{Art... Nez... Guy...} — \href{http://coord.info/GC5V5A2\Number{}712322077}{584}}\cacheData{{2017/08/17 plume64, Multi-cache (1.5/1.5)}}\begin{cacheText}En route pour le col de Ronceveaux nous nous arrêtons pour faire cette petite Multi :pas de difficultés particulières . Merci plume 64 😃\end{cacheText}

\cacheNumber{585}\needspace{5\baselineskip}\cacheName{\href{http://coord.info/GC2BZJK}{La porte d'Espagne} — \href{http://coord.info/GC2BZJK\Number{}712375265}{585}}\cacheData{{2017/08/17 Peyo64, Traditional Cache (1.5/1.5)}}\begin{cacheText}Ici...bonne surprise nous trouvons très rapidement ,il n'y a plus personne à 19h et nous pouvons signer discrètement .C'est super ...merci Peyo 😀\end{cacheText}

\cacheNumber{586}\needspace{5\baselineskip}\cacheName{\href{http://coord.info/GC425P6}{Event SdG GR10 06-01} — \href{http://coord.info/GC425P6\Number{}712379708}{586}}\cacheData{{2017/08/17 Charnègues, Traditional Cache (1.5/1.5)}}\begin{cacheText}Une des dernières caches de la journée ...oufff on la trouve très facilement car elle est visible depuis la voiture!!! En repartant on la camoufle du mieux possible merci 😊\end{cacheText}

\cacheNumber{587}\needspace{5\baselineskip}\cacheName{\href{http://coord.info/GC7707G}{Batalla de Roncesvalles BR5} — \href{http://coord.info/GC7707G\Number{}712366992}{587}}\cacheData{{2017/08/17 gemacarlos, Traditional Cache (1.5/1.5)}}\begin{cacheText}Nous commençons le circuit dans le désordre nous n'avons pas vu qu'il s'agissait d'une boucle. La cache est rapidement trouvée grâce a l'indice. Très jolie boîte pleine de bidules ....j'adore...

In: une rondelle de bois

Out: un pendentif tennis rouge

C'est un très bel endroit que l'on ne connaissait absolument pas. Je remarque des entrées un peu bizarre… Je ne sais pas ce que c'est...Merci pour cette découverte\end{cacheText}

\cacheNumber{588}\needspace{5\baselineskip}\cacheName{\href{http://coord.info/GC77092}{Batalla de Roncesvalles BR6} — \href{http://coord.info/GC77092\Number{}712368684}{588}}\cacheData{{2017/08/17 gemacarlos, Traditional Cache (1.5/1.5)}}\begin{cacheText}Le lieu est toujours aussi beau et nous trouvons rapidement la jolie cache remplie de bidules comme je les aime Merci 😊\end{cacheText}

\cacheNumber{589}\needspace{5\baselineskip}\cacheName{\href{http://coord.info/GC6XCJW}{Maisons à Colombages - Casteljaloux - Rue Posterne} — \href{http://coord.info/GC6XCJW\Number{}712825003}{589}}\cacheData{{2017/08/18 Agex, Traditional Cache (1.5/1.5)}}\begin{cacheText}En week-end dans le Lot-et-Garonne nous en profitons pour faire quelques caches qu'il nous manque sur Casteljaloux. Celle-ci est trouvée sans difficulté et nous fait découvrir une superbe maison à colombages.Merci 😄\end{cacheText}

\cacheNumber{590}\needspace{5\baselineskip}\cacheName{\href{http://coord.info/GC6XFFE}{Les Remparts de Casteljaloux - les vestiges} — \href{http://coord.info/GC6XFFE\Number{}712826978}{590}}\cacheData{{2017/08/18 Agex, Traditional Cache (2/1)}}\begin{cacheText}Nous continuons notre quête vers les remparts de Casteljaloux .Après avoir lu les logs précédent un petit indice nous aiguille enfin...nous dénichons la cache .Merci\end{cacheText}

\cacheNumber{591}\needspace{5\baselineskip}\cacheName{\href{http://coord.info/GC4YJ2G}{Alphabet Lot et Garonnais... C pour Casteljaloux} — \href{http://coord.info/GC4YJ2G\Number{}712831396}{591}}\cacheData{{2017/08/18 nmns26, Traditional Cache (2/1)}}\begin{cacheText}Nous poursuivons la découverte des monuments de la ville :la maison du roi est très jolie. La cache est vite repérée.Merci pour cette découverte 😃\end{cacheText}

\cacheNumber{592}\needspace{5\baselineskip}\cacheName{\href{http://coord.info/GC6XCEQ}{Gare de Casteljaloux - Les Vestiges visibles} — \href{http://coord.info/GC6XCEQ\Number{}712831731}{592}}\cacheData{{2017/08/18 Agex, Traditional Cache (1/1)}}\begin{cacheText}Très bonne idée pour se rappeler du passé :la cache est vite trouvée.Merci 😃\end{cacheText}

\cacheNumber{593}\needspace{5\baselineskip}\cacheName{\href{http://coord.info/GC6ZWF1}{Lac de Clarens \Number{}1 : L'entrée - Casteljaloux(47)} — \href{http://coord.info/GC6ZWF1\Number{}712832454}{593}}\cacheData{{2017/08/18 Agex, Traditional Cache (1.5/1)}}\begin{cacheText}Heureusement l'endroit n'est pas trop fréquenté par ce jour nuageux ,cela nous permet de chercher sans être repéré .Merci 😊\end{cacheText}

\cacheNumber{594}\needspace{5\baselineskip}\cacheName{\href{http://coord.info/GC4W3ZY}{Alphabet Lot et Garonnais... A pour Avance} — \href{http://coord.info/GC4W3ZY\Number{}712834313}{594}}\cacheData{{2017/08/18 nmns26, Traditional Cache (2.5/2)}}\begin{cacheText}En week-end sur le Lot-et-Garonne nous venons faire quelques caches. Arrivés sur les lieux nous cherchons tout d'abord l'arbre creux que nous trouvons, puis nous nous attaquons à l'ingénieux système. Pas facile d'attraper la cache mais on n'y parvient : une limace faisait ventouse !!!Logbook signé nous remettons tout en place. Merci pour ce beau travail ( qui mérite un PF)😊\end{cacheText}

\cacheNumber{595}\needspace{5\baselineskip}\cacheName{\href{http://coord.info/GC6ZAVY}{La Bartère - Parc et Centre Thermal - Casteljaloux} — \href{http://coord.info/GC6ZAVY\Number{}712832231}{595}}\cacheData{{2017/08/18 Agex, Traditional Cache (2.5/2.5)}}\begin{cacheText}Voilà un joli petit sentier de découverte qui passe devant les jardins partagés de Casteljaloux (endroit pittoresque). La cache nous attend à la cicatrice,Merci pour cette belle découverte\end{cacheText}

\cacheNumber{596}\needspace{5\baselineskip}\cacheName{\href{http://coord.info/GC5V4V2}{eglise de samazan} — \href{http://coord.info/GC5V4V2\Number{}712843380}{596}}\cacheData{{2017/08/18 gmarlet, Traditional Cache (1/1.5)}}\begin{cacheText}Arrivé devant cette très belle église nous devons attendre car une moldu  promène son chien !!! La cache est ensuite rapidement localisée. Merci 😊\end{cacheText}

\cacheNumber{597}\needspace{5\baselineskip}\cacheName{\href{http://coord.info/GC76D10}{COCUMONT \Number{} 14 Le Vigneron} — \href{http://coord.info/GC76D10\Number{}712849201}{597}}\cacheData{{2017/08/19 PhT47 DidiFR, Traditional Cache (1.5/1.5)}}\begin{cacheText}Endroit très sympa: la cache nous attend bien sagement au pied. ..Merci 😊\end{cacheText}

\cacheNumber{598}\needspace{5\baselineskip}\cacheName{\href{http://coord.info/GC6QY05}{COCUMONT\Number{}7 Ruisseau de Constans} — \href{http://coord.info/GC6QY05\Number{}712845266}{598}}\cacheData{{2017/08/19 PhNTD, Traditional Cache (1.5/1.5)}}\begin{cacheText}Arrivés sur les lieux l' indice prend tout son sens . Nous découvrons la cache en deux temps trois mouvements ...charmant petit endroit merci\end{cacheText}

\cacheNumber{599}\needspace{5\baselineskip}\cacheName{\href{http://coord.info/GC6QYCN}{COCUMONT\Number{}8 Le solitaire} — \href{http://coord.info/GC6QYCN\Number{}712848687}{599}}\cacheData{{2017/08/19 PhNTD, Traditional Cache (1.5/1.5)}}\begin{cacheText}Arrivés sur les lieux nous cherchons parmi les branches et toutes les pommes !!! Nous mettons la main dessus et au passage je mange une pomme 🍎 . Merci 😊\end{cacheText}

\cacheNumber{600}\needspace{5\baselineskip}\cacheName{\href{http://coord.info/GC6QWEA}{COCUMONT\Number{}6 Le Clos de l'Ane} — \href{http://coord.info/GC6QWEA\Number{}712844900}{600}}\cacheData{{2017/08/19 PhNTD, Traditional Cache (1.5/1.5)}}\begin{cacheText}Ici le GPS nous a bien baladé :il nous a fallu chercher auprès de chaque arbre chaque pierre  et chaque table pour enfin mettre la main dessus ...et ne pas faire un DNF . L'âne s'est bien amusé de nous voir faire!!!Merci pour ce bon moment 😃\end{cacheText}

\cacheNumber{601}\needspace{5\baselineskip}\cacheName{\href{http://coord.info/GC6QWEV}{COCUMONT\Number{}5 La Forge} — \href{http://coord.info/GC6QWEV\Number{}712844212}{601}}\cacheData{{2017/08/19 PhNTD, Traditional Cache (1.5/1.5)}}\begin{cacheText}Aucune difficulté pour cette cache… nous la trouvons parterre!!! Après avoir signé nous la remettons à 1 m20 du sol. Merci 😊\end{cacheText}

\cacheNumber{602}\needspace{5\baselineskip}\cacheName{\href{http://coord.info/GC6QWDX}{COCUMONT\Number{}4 Le Lavoir 2} — \href{http://coord.info/GC6QWDX\Number{}712844202}{602}}\cacheData{{2017/08/19 PhNTD, Traditional Cache (1.5/1.5)}}\begin{cacheText}L'endroit est très paisible et nous en profitons pour nous asseoir un peu....La cache nous attend au pied de ce joli lavoir merci😁\end{cacheText}

\cacheNumber{603}\needspace{5\baselineskip}\cacheName{\href{http://coord.info/GC6QR54}{COCUMONT\Number{}3 La Vieille Eglise} — \href{http://coord.info/GC6QR54\Number{}712843804}{603}}\cacheData{{2017/08/19 PhNTD, Traditional Cache (1.5/1.5)}}\begin{cacheText}Cette cache nous permet de découvrir une très belle église . La cache est bien à l'extérieur. Merci 😊\end{cacheText}

\cacheNumber{604}\needspace{5\baselineskip}\cacheName{\href{http://coord.info/GC542ZW}{VV Le Mas - Villeton \Number{}10 - Halte nautique Villeton} — \href{http://coord.info/GC542ZW\Number{}713060319}{604}}\cacheData{{2017/08/19 nmns26, Traditional Cache (3/1.5)}}\begin{cacheText}Nous décidons ce matin d'attaquer le canal de la Garonne :cette première cache nous donne du fil à retordre .Le GPS nous promène !!! Heureusement il est tôt et il n'y a aucun moldu à l'horizon. Nous finissons par mettre la main dessus :cache originale parfaitement intégrée dans le décor. Cela mérite un PF 

In: une rondelle de bois

Out : un autocollant 

Merci pour ce bon moment\end{cacheText}

\cacheNumber{605}\needspace{5\baselineskip}\cacheName{\href{http://coord.info/GC542Z6}{VV Le Mas - Villeton \Number{}7 - Pont de Jeanserre} — \href{http://coord.info/GC542Z6\Number{}713063400}{605}}\cacheData{{2017/08/19 nmns26, Traditional Cache (2.5/1.5)}}\begin{cacheText}Arrivé sur le pont il nous a fallu un peu de temps pour déloger la belle ...bien cachée!!! L'endroit est calme et paisible :c'est très agréable .Merci 😃\end{cacheText}

\cacheNumber{606}\needspace{5\baselineskip}\cacheName{\href{http://coord.info/GC542YW}{VV Le Mas - Villeton \Number{}6 - Halte de Lagruère} — \href{http://coord.info/GC542YW\Number{}713066680}{606}}\cacheData{{2017/08/19 nmns26, Multi-cache (1.5/1.5)}}\begin{cacheText}Après avoir résolu les l'énigme qui nous donne les nouvelles coordonnées nous arrivons sur ce lieu superbe. Grâce a l'indice nous trouvons rapidement la cache 

In : coquillage 

Out : un autocollant 

Merci pour ce bon moment\end{cacheText}

\cacheNumber{607}\needspace{5\baselineskip}\cacheName{\href{http://coord.info/GC542Y0}{VV Le Mas - Villeton \Number{}5 - Canton} — \href{http://coord.info/GC542Y0\Number{}713067915}{607}}\cacheData{{2017/08/19 nmns26, Traditional Cache (1.5/1.5)}}\begin{cacheText}Nous continuons cette jolie balade le long du canal de la Garonne nous ne croisons personne ce qui nous permet de loguer très rapidement cette cache face à la péniche\end{cacheText}

\cacheNumber{608}\needspace{5\baselineskip}\cacheName{\href{http://coord.info/GC542XP}{VV Le Mas - Villeton \Number{}4 - Pont de Lagruère} — \href{http://coord.info/GC542XP\Number{}713067919}{608}}\cacheData{{2017/08/19 nmns26, Traditional Cache (1.5/1.5)}}\begin{cacheText}Arrivés au PZ la cache est évidente (trouvée grâce à l'indice )nous continuons notre promenade . Merci 😃\end{cacheText}

\cacheNumber{609}\needspace{5\baselineskip}\cacheName{\href{http://coord.info/GC542XF}{VV Le Mas - Villeton \Number{}3 - Beauséjour} — \href{http://coord.info/GC542XF\Number{}713068489}{609}}\cacheData{{2017/08/19 nmns26, Traditional Cache (1.5/1.5)}}\begin{cacheText}Ici aussi la endroit est paisible ,la cache est rapidement repérée nous poursuivons notre collecte  de caches.  Merci 😊\end{cacheText}

\cacheNumber{610}\needspace{5\baselineskip}\cacheName{\href{http://coord.info/GC542XB}{VV Le Mas - Villeton \Number{}2 - Perrot} — \href{http://coord.info/GC542XB\Number{}713069048}{610}}\cacheData{{2017/08/19 nmns26, Traditional Cache (1.5/1.5)}}\begin{cacheText}Les platanes sont toujours aussi magnifiques!!! et la cache bien à l'abri Merci 😊\end{cacheText}

\cacheNumber{611}\needspace{5\baselineskip}\cacheName{\href{http://coord.info/GC542X1}{VV Le Mas - Villeton \Number{}1 - Le pont suspendu} — \href{http://coord.info/GC542X1\Number{}713069640}{611}}\cacheData{{2017/08/19 nmns26, Traditional Cache (1.5/2)}}\begin{cacheText}Ici le cadre est beaucoup moins bucolique nous trouvons la cache et partons vers d'autres cieux ...Merci 😊\end{cacheText}

\cacheNumber{612}\needspace{5\baselineskip}\cacheName{\href{http://coord.info/GC4ZVH7}{VV Pont des Sables - Le Mas \Number{}11 - Pichot} — \href{http://coord.info/GC4ZVH7\Number{}713104293}{612}}\cacheData{{2017/08/19 nmns26, Traditional Cache (1.5/1.5)}}\begin{cacheText}Après avoir relu l'indice ,la cache est délogée facilement merci pour le travail 😃\end{cacheText}

\cacheNumber{613}\needspace{5\baselineskip}\cacheName{\href{http://coord.info/GC4ZVGE}{VV Pont des Sables - Le Mas \Number{}10 - Pont Larriveau} — \href{http://coord.info/GC4ZVGE\Number{}713104880}{613}}\cacheData{{2017/08/19 nmns26, Traditional Cache (2/1.5)}}\begin{cacheText}Ici aussi la cache est facile à trouver .L'endroit est toujours aussi paisible .Merci 😊\end{cacheText}

\cacheNumber{614}\needspace{5\baselineskip}\cacheName{\href{http://coord.info/GC4ZVG6}{VV Pont des Sables - Le Mas \Number{}9 - Le déversoir} — \href{http://coord.info/GC4ZVG6\Number{}713105258}{614}}\cacheData{{2017/08/19 nmns26, Traditional Cache (2/1.5)}}\begin{cacheText}Je confirme ...un petit miroir peut aider !!! La cache est bien à sa place merci 😊\end{cacheText}

\cacheNumber{615}\needspace{5\baselineskip}\cacheName{\href{http://coord.info/GC4ZVFV}{VV Pont des Sables - Le Mas \Number{}8 - Pont de Larroque} — \href{http://coord.info/GC4ZVFV\Number{}713105546}{615}}\cacheData{{2017/08/19 nmns26, Traditional Cache (1/1.5)}}\begin{cacheText}Pas de surprise sur cette cache  Elle nous attend à l'endroit indiqué . Nous continuons notre promenade vers la prochaine merci 😊\end{cacheText}

\cacheNumber{616}\needspace{5\baselineskip}\cacheName{\href{http://coord.info/GC4ZVF8}{VV Pont des Sables - Le Mas \Number{}7 - Plantion} — \href{http://coord.info/GC4ZVF8\Number{}713105798}{616}}\cacheData{{2017/08/19 nmns26, Traditional Cache (1.5/1.5)}}\begin{cacheText}La balade est toujours aussi agréable et nous trouvons çette cache rapidement merci 😊\end{cacheText}

\cacheNumber{617}\needspace{5\baselineskip}\cacheName{\href{http://coord.info/GC4ZVEQ}{VV Pont des Sables - Le Mas \Number{}6 - Caumont} — \href{http://coord.info/GC4ZVEQ\Number{}713106198}{617}}\cacheData{{2017/08/19 nmns26, Traditional Cache (3/2.5)}}\begin{cacheText}Ici la cache est assez compliquée.... nous décidons de faire la halte pique-nique et en profitons pour chercher. Bingo ...la cache est repérée. Le log book est détrempé

In : une figurine

Out : Rien

Merci 😊\end{cacheText}

\cacheNumber{618}\needspace{5\baselineskip}\cacheName{\href{http://coord.info/GC4ZVED}{VV Pont des Sables - Le Mas \Number{}5 - Pont de l'église} — \href{http://coord.info/GC4ZVED\Number{}713106358}{618}}\cacheData{{2017/08/19 nmns26, Traditional Cache (1/1.5)}}\begin{cacheText}Il commence à faire chaud sur les bords de la Garonne et nous trouvons la cache sans aucune difficulté merci 😊\end{cacheText}

\cacheNumber{619}\needspace{5\baselineskip}\cacheName{\href{http://coord.info/GC4ZVE3}{VV Pont des Sables - Le Mas \Number{}4 - Fourques village} — \href{http://coord.info/GC4ZVE3\Number{}713107014}{619}}\cacheData{{2017/08/19 nmns26, Traditional Cache (1/1.5)}}\begin{cacheText}Grâce à l'indice nous trouvons la cache qui est effectivement bien tapie au fond . Merci 😊\end{cacheText}

\cacheNumber{620}\needspace{5\baselineskip}\cacheName{\href{http://coord.info/GC55WDW}{VV Villeton - Damazan \Number{}2 - Pont de la Falotte} — \href{http://coord.info/GC55WDW\Number{}713108471}{620}}\cacheData{{2017/08/19 nmns26, Traditional Cache (1/1.5)}}\begin{cacheText}Et nous reprenons notre promenade dans l'autre sens ,la cache est vite  détectée . Nous admirons les splendides maisons et le beau bateau.Le paysage est magnifique. Merci 😊\end{cacheText}

\cacheNumber{621}\needspace{5\baselineskip}\cacheName{\href{http://coord.info/GC55WE8}{VV Villeton - Damazan \Number{}3 - Pont de Monheurt} — \href{http://coord.info/GC55WE8\Number{}713110051}{621}}\cacheData{{2017/08/19 nmns26, Traditional Cache (2/1.5)}}\begin{cacheText}Nous arrivons au pont de Monheurt . L'indice nous aiguille et nous trouvons la cache en deux temps trois mouvements . Nous ne rencontrons personne le long de ce bord de Garonne pour l'instant !!!! Merci 😊\end{cacheText}

\cacheNumber{622}\needspace{5\baselineskip}\cacheName{\href{http://coord.info/GC55WJE}{VV Villeton - Damazan \Number{}4 - Pont de Vigneau} — \href{http://coord.info/GC55WJE\Number{}713111058}{622}}\cacheData{{2017/08/19 nmns26, Traditional Cache (2/1.5)}}\begin{cacheText}Ici il nous a fallu faire appel au tournevis pour réussir à atteindre la belle.  Merci 😊\end{cacheText}

\cacheNumber{623}\needspace{5\baselineskip}\cacheName{\href{http://coord.info/GC55WJT}{VV Villeton - Damazan \Number{}5 - Pont de Morin} — \href{http://coord.info/GC55WJT\Number{}713111514}{623}}\cacheData{{2017/08/19 nmns26, Traditional Cache (1.5/1.5)}}\begin{cacheText}Malgré le précédent DNF nous arrivons motivéssur le lieu de la cache  et à force de recherche nous trouvons !!! Merci 😊\end{cacheText}

\cacheNumber{624}\needspace{5\baselineskip}\cacheName{\href{http://coord.info/GC55WQN}{VV Villeton - Damazan \Number{}8 - Chateau de Pellet} — \href{http://coord.info/GC55WQN\Number{}713111787}{624}}\cacheData{{2017/08/19 nmns26, Traditional Cache (1.5/1.5)}}\begin{cacheText}La balade est toujours aussi agréable et la cache est vite trouvée.Merci 😊\end{cacheText}

\cacheNumber{625}\needspace{5\baselineskip}\cacheName{\href{http://coord.info/GC55WQY}{VV Villeton - Damazan \Number{}9 - Le réservoir} — \href{http://coord.info/GC55WQY\Number{}713112579}{625}}\cacheData{{2017/08/19 nmns26, Traditional Cache (1.5/1.5)}}\begin{cacheText}Arrivés au réservoir nous délogeons la belle vite fait bien fait!!

In : un coquillage 

out : rien

Merci 😊\end{cacheText}

\cacheNumber{626}\needspace{5\baselineskip}\cacheName{\href{http://coord.info/GC6XBX4}{DAMAZAN \Number{} 1 - Le Lac} — \href{http://coord.info/GC6XBX4\Number{}713113198}{626}}\cacheData{{2017/08/19 PhNTD, Traditional Cache (1.5/1.5)}}\begin{cacheText}En week-end dans le Lot-et-Garonne nous faisons le long de la Garonne puis nous voyons cette cache en dehors du parcours et naturellement nous décidons de la faire. Ceci n'est qu'une simple formalité, la cache est vite trouvée merci 😊\end{cacheText}

\cacheNumber{627}\needspace{5\baselineskip}\cacheName{\href{http://coord.info/GC77Q3D}{Eglise Notre Dame de Buzet-sur-Baïse (47)} — \href{http://coord.info/GC77Q3D\Number{}713113682}{627}}\cacheData{{2017/08/19 Agex, Traditional Cache (2.5/1)}}\begin{cacheText}Nous faisons rapidement le tour de cette très belle église et admirons les végétaux qui poussent sauvagement . La cache est vite trouvée ,merci pour ce moment 😃\end{cacheText}

\cacheNumber{628}\needspace{5\baselineskip}\cacheName{\href{http://coord.info/GC789VC}{Halte Nautique - Canal de Garonne -Buzet-sur-Baïse} — \href{http://coord.info/GC789VC\Number{}713114785}{628}}\cacheData{{2017/08/19 Agex, Traditional Cache (1.5/1.5)}}\begin{cacheText}Nous arrivons sur l'endroit que nous trouvons charmant !!!l La cache est rapidement localisée :les coordonnées sont précises . Merci pour la découverte de ce lieu 😃\end{cacheText}

\cacheNumber{629}\needspace{5\baselineskip}\cacheName{\href{http://coord.info/GC52E24}{Alphabet Lot-et-Garonnais... F comme église\Quoted{Forte}} — \href{http://coord.info/GC52E24\Number{}713131562}{629}}\cacheData{{2017/08/19 -Matter-, Traditional Cache (1.5/1.5)}}\begin{cacheText}Nous trouvons la une très belle église comme il y en a tant dans le Lot-et-Garonne. Nous cherchons la croix grise et de ce fait nous trouvons la boîte . Merci pour la découverte de ce beau petit village et de son musée de l'école 😃\end{cacheText}

\cacheNumber{630}\needspace{5\baselineskip}\cacheName{\href{http://coord.info/GC687J5}{Le Lavoir de Xaintrailles} — \href{http://coord.info/GC687J5\Number{}713430239}{630}}\cacheData{{2017/08/20 jomatoju 47, Traditional Cache (1.5/1.5)}}\begin{cacheText}Retour au lavoir ,deuxième passage : cette fois la cache est délogée !!! Pas de difficulté grâce à l'indice et aux lors  précédents et attention aux guêpes ,Merci  😊 pour la découverte de ce lieu\end{cacheText}

\cacheNumber{631}\needspace{5\baselineskip}\cacheName{\href{http://coord.info/GC5GGVH}{EGLISE D ESTUSSAN} — \href{http://coord.info/GC5GGVH\Number{}713479820}{631}}\cacheData{{2017/08/20 jomatoju 47, Traditional Cache (1.5/1.5)}}\begin{cacheText}Première cache pour mon frère que l'on initie.... la cache a été vite débusquée. Merci 😊\end{cacheText}

\cacheNumber{632}\needspace{5\baselineskip}\cacheName{\href{http://coord.info/GC5JYAM}{place du griffon} — \href{http://coord.info/GC5JYAM\Number{}713484422}{632}}\cacheData{{2017/08/20 lomimaju47, Traditional Cache (1.5/1.5)}}\begin{cacheText}Nous continuons les recherches dans ce joli village de Lavardac .Arrivés sur les lieux et aidés d'un plombier nous debusquons la bouche à clé très rapidement. Jolie cache , du vrai géocaching !!!! Heureusement que nous faisons suivre les ustensiles.... et qu'il n'y a pas trop de moldus à cette heure ci !!!!merci pour ce bon moment\end{cacheText}

\cacheNumber{633}\needspace{5\baselineskip}\cacheName{\href{http://coord.info/GC767RE}{Mosaïque aquatique} — \href{http://coord.info/GC767RE\Number{}713490199}{633}}\cacheData{{2017/08/20 marlowar, Traditional Cache (1.5/1.5)}}\begin{cacheText}Après avoir cherché  un bon quart d'heure notre plombier favoris débusque  La cache.  Les histoires d'eau ça lui parlent!!!! Merci 😊\end{cacheText}

\cacheNumber{634}\needspace{5\baselineskip}\cacheName{\href{http://coord.info/GC5GGTR}{Lavoir de Dame POUPOSECO} — \href{http://coord.info/GC5GGTR\Number{}713470828}{634}}\cacheData{{2017/08/20 jomatoju 47, Traditional Cache (1.5/1.5)}}\begin{cacheText}Arrivés sur les lieux le GPS nous envoie à l'opposé .Après avoir lu plusieurs fois les commentaires et regardé les photos nous déterminons une zone et après une demi-heure de recherches ....enfin nous mettons la main dessus . Elle est très bien intégrée au paysage et mérite un point favoris .Merci 😊\end{cacheText}

\cacheNumber{635}\needspace{5\baselineskip}\cacheName{\href{http://coord.info/GC2JX9Q}{La Tour du Chateau de Sault de Navailles} — \href{http://coord.info/GC2JX9Q\Number{}715027431}{635}}\cacheData{{2017/08/25 lokateo64, Traditional Cache (1.5/1.5)}}\begin{cacheText}Ayant une course à faire dans le coin j'en profite pour faire cette cache qui me tente depuis un certain temps .Arrivée sur le site je découvre les vestiges d'un  magnifique château. Je pars à la recherche de cette cache que je découvre très rapidement. Ce n'est pas celle d'origine car le log book est quasiment neuf. La cache se perpétue merci pour la pause et pour ceux qui ont effectué la maintenance.\end{cacheText}

\cacheNumber{636}\needspace{5\baselineskip}\cacheName{\href{http://coord.info/GC3G8HB}{GR8\Number{}132} — \href{http://coord.info/GC3G8HB\Number{}715755597}{636}}\cacheData{{2017/08/27 Peyo64, Traditional Cache (2/1.5)}}\begin{cacheText}Reprise du GR8 : lendemain de fête c'est compliqué d'autant plus que les montées sont ardues!!!! La cache est trouvée rapidement grâce au spoiler . Nous rencontrons un chercheur de champignons (il a quelques cèpes dans son joli panier ) et nous faisons un brin de causette . Nous reprenons notre route ....merci pour ce bon moment 😃\end{cacheText}

\cacheNumber{637}\needspace{5\baselineskip}\cacheName{\href{http://coord.info/GC3G8HH}{GR8\Number{}133} — \href{http://coord.info/GC3G8HH\Number{}715759189}{637}}\cacheData{{2017/08/27 Peyo64, Traditional Cache (2/1.5)}}\begin{cacheText}Après avoir aidé un pèlerin espagnol qui a perdu la route d'Ainhoa ( 3 heures qu'il tourne en rond sur ces chemins ... parce que le panneau arraché ne donne plus la bonne direction) nous trouvons cette jolie cache . Nous continuons notre périple, il commence à pleuvoir ...merci 😄\end{cacheText}

\cacheNumber{638}\needspace{5\baselineskip}\cacheName{\href{http://coord.info/GC3G8HR}{GR8\Number{}134} — \href{http://coord.info/GC3G8HR\Number{}715775530}{638}}\cacheData{{2017/08/27 Peyo64, Traditional Cache (2/1.5)}}\begin{cacheText}Ici aussi la cache est vite trouvée et effectivement il ne s'agit pas de la cache d'origine . Tant pis pour l'indice s' il y en avait un!!! La vue est magnifique et je ne connaissais pas cette partie de l'histoire du Pays Basque.Merci pour cette belle promenade 😄\end{cacheText}

\cacheNumber{639}\needspace{5\baselineskip}\cacheName{\href{http://coord.info/GC3G8J6}{GR8\Number{}135} — \href{http://coord.info/GC3G8J6\Number{}715761326}{639}}\cacheData{{2017/08/27 Peyo64, Traditional Cache (1.5/1.5)}}\begin{cacheText}Arrivé sur les lieux nous sommes plutôt sceptiques . Nous trouvons effectivement l'attache verte sur le lierre et à tout hasard nous cherchons au pied du lierre et... au miracle nous la trouvons couverte par les feuilles !!!Nous la remettons en place ,très content de notre découverte.  Merci 😊\end{cacheText}

\cacheNumber{640}\needspace{5\baselineskip}\cacheName{\href{http://coord.info/GC3G8MG}{GR8\Number{}139} — \href{http://coord.info/GC3G8MG\Number{}715778025}{640}}\cacheData{{2017/08/27 Peyo64, Traditional Cache (1.5/1.5)}}\begin{cacheText}Il fait de plus en plus lourd !!! À a fin de cette longue descente nous arrivons à l'endroit où se trouvent les quelques pierres . Nous débusquons la cache rapidement . Merci 😊\end{cacheText}

\cacheNumber{641}\needspace{5\baselineskip}\cacheName{\href{http://coord.info/GC3G8MC}{GR8\Number{}138} — \href{http://coord.info/GC3G8MC\Number{}715777056}{641}}\cacheData{{2017/08/27 Peyo64, Traditional Cache (2/1.5)}}\begin{cacheText}Ici aussi la vue est exceptionnelle. L'indice et la photo nous aident bien ...pas de problème . Merci\end{cacheText}

\cacheNumber{642}\needspace{5\baselineskip}\cacheName{\href{http://coord.info/GC3G8KK}{GR8\Number{}137} — \href{http://coord.info/GC3G8KK\Number{}715776430}{642}}\cacheData{{2017/08/27 Peyo64, Traditional Cache (2/1.5)}}\begin{cacheText}Ici les arbres sont majestueux ,ils ont certainement des centaines d'années. La cache est rapidement repérée par l'œil, maintenant ,expert de monsieur !!! Merci pour la découverte de ce lieu 😃\end{cacheText}

\cacheNumber{643}\needspace{5\baselineskip}\cacheName{\href{http://coord.info/GC3G8JD}{GR8\Number{}136} — \href{http://coord.info/GC3G8JD\Number{}715775539}{643}}\cacheData{{2017/08/27 Peyo64, Traditional Cache (1.5/1.5)}}\begin{cacheText}Le parcours continue et la vue est toujours aussi superbe la cache est trouvée sans difficulté . Nous poursuivons .... Merci 😊\end{cacheText}

\cacheNumber{644}\needspace{5\baselineskip}\cacheName{\href{http://coord.info/GC3G8NF}{GR8\Number{}142} — \href{http://coord.info/GC3G8NF\Number{}715784520}{644}}\cacheData{{2017/08/27 Peyo64, Traditional Cache (2/1.5)}}\begin{cacheText}Grâce à l'indice et la photo nous repérons très facilement la cache . C'est un très joli endroit .  Merci 😊\end{cacheText}

\cacheNumber{645}\needspace{5\baselineskip}\cacheName{\href{http://coord.info/GC3G8N6}{GR8\Number{}141} — \href{http://coord.info/GC3G8N6\Number{}715783235}{645}}\cacheData{{2017/08/27 Peyo64, Traditional Cache (2/1.5)}}\begin{cacheText}Une fois de plus nous arrivons sans ci conviction sur le lieu . Nous cherchons malgré tout et BINGO...,la cache se dévoile . Nous accélérons le pas : des gouttes tombent et l'orage gronde !!!merci 😊\end{cacheText}

\cacheNumber{646}\needspace{5\baselineskip}\cacheName{\href{http://coord.info/GC3G8MZ}{GR8\Number{}140} — \href{http://coord.info/GC3G8MZ\Number{}715780951}{646}}\cacheData{{2017/08/27 Peyo64, Traditional Cache (2/1.5)}}\begin{cacheText}Après plusieurs minutes de recherche nous nous approchons du DNF .... mais nous avons du mal à croire qu'elle ait disparu dans un endroit aussi peu fréquenté.La cache nous apparaît enfin... effectivement personne ne l'a touchée depuis août 2016 et la nature a repris ses droits!!!! Grande satisfaction merci 😊\end{cacheText}

\cacheNumber{647}\needspace{5\baselineskip}\cacheName{\href{http://coord.info/GC5KRJG}{Forêt Ustaritz - Saint Pée sur Nivelle} — \href{http://coord.info/GC5KRJG\Number{}715785221}{647}}\cacheData{{2017/08/27 brunolli, Traditional Cache (1.5/1.5)}}\begin{cacheText}Sur le retour du GR8 de Peyo64 nous nous arrêtons pour faire cette petite cache. Endroit sympa et cache rapidement trouvée.Merci pour la découverte 😄\end{cacheText}

\cacheNumber{648}\needspace{5\baselineskip}\cacheName{\href{http://coord.info/GC68CD7}{Le Chateau Lota} — \href{http://coord.info/GC68CD7\Number{}715786228}{648}}\cacheData{{2017/08/27 gamboy, Traditional Cache (1.5/1.5)}}\begin{cacheText}Sur le retour du GR 8 de Peyo 64 nous nous arrêtons au château de Lota pour faire cette petite . Elle est découverte en deux temps trois mouvements . Le château est superbe et mérite une restauration !!! Merci 😊\end{cacheText}

\cacheNumber{649}\needspace{5\baselineskip}\cacheName{\href{http://coord.info/GC3G2Y5}{GR8\Number{}69} — \href{http://coord.info/GC3G2Y5\Number{}715803478}{649}}\cacheData{{2017/08/27 Peyo64, Traditional Cache (1.5/1.5)}}\begin{cacheText}Il nous manquait deux caches suce tronçon. Nous découvrons celle-ci en deux temps trois mouvements elle est bien cachée dans le tronc. .. et accélèrons le pas car il commence à sérieusement pleuvoir.Merci 😊\end{cacheText}

\cacheNumber{650}\needspace{5\baselineskip}\cacheName{\href{http://coord.info/GC3G2Y8}{GR8\Number{}70} — \href{http://coord.info/GC3G2Y8\Number{}715802580}{650}}\cacheData{{2017/08/27 Peyo64, Traditional Cache (1.5/2)}}\begin{cacheText}Dernière cache de la deuxième partie du GR 8. Nous avons regardé au pied des arbres et nous l'avons trouvé  rapidement heureusement car la pluie tombe à grosses gouttes et les éclairs sont de plus en plus violent!!!! Nous courrons nous mettre à l'abri dans la voiture et finissons la journée. Merci 😊\end{cacheText}

\cacheNumber{651}\needspace{5\baselineskip}\cacheName{\href{http://coord.info/GC7BGVD}{Le Quillet du lac de luc} — \href{http://coord.info/GC7BGVD\Number{}716223099}{651}}\cacheData{{2017/08/29 zeebrain, Traditional Cache (2/2)}}\begin{cacheText}[{FTF}]

Installée confortablement dans mon canapé je consulte mes mails et constate qu'une alerte est arrivée !!!! La cache n'est pas trop loin de mon domicile.... mais monsieur est monté se coucher. Je n'ai pas eu à insister: 5 min plus tard nous sommes dans la géomobile équipés de torches et de parkas car il  pleut pas mal!!!!! Espérons qu'aucune autre personne  ne soit arrivé sur les lieux avant nous!!!!

 La cache a été rapidement repérée à la torche car nous en avons déjà vu des similaires. Elle est parfaitement intégrée : du super bricolage. Bravo !!! Oufff nous sommes les premiers et donc nous remportons le géocoin . Dans la précipitation je n'ai pas pris mes bidules pour en laisser .... je reviendrai à l'occasion en déposer un.Merci pour cette super soirée 😃\end{cacheText}

\cacheNumber{652}\needspace{5\baselineskip}\cacheName{\href{http://coord.info/GC69J63}{Lavoir de Houndaro - Anglet} — \href{http://coord.info/GC69J63\Number{}718867210}{652}}\cacheData{{2017/09/08 gilles64, Traditional Cache (2/1.5)}}\begin{cacheText}Encore un lavoir oublié de tous ..,quel dommage !!!!La cacher est vite trouvée grâce à l'indice .Merci Gilles de nous montrer de si jolis endroits .\end{cacheText}

\cacheNumber{653}\needspace{5\baselineskip}\cacheName{\href{http://coord.info/GC6DVY0}{Les \Quoted{Oiseaux Migrateurs} !} — \href{http://coord.info/GC6DVY0\Number{}718829749}{653}}\cacheData{{2017/09/08 gilles64, Traditional Cache (1.5/1.5)}}\begin{cacheText}Ayant un peu de temps libre avant la sortie des classes je me décide pour venir faire celle-ci. Les oiseaux passent et repassent pendant que je cherche la cache. C'est une véritable boîte à bidules comme je les aime. Merci Gilles 64\end{cacheText}

\cacheNumber{654}\needspace{5\baselineskip}\cacheName{\href{http://coord.info/GC6DMTT}{Pigeonnier de la Peña - Anglet} — \href{http://coord.info/GC6DMTT\Number{}719128553}{654}}\cacheData{{2017/09/09 gilles64, Traditional Cache (1.5/1.5)}}\begin{cacheText}Encore un bien bel endroit que je n'aurais probablement jamais découvert sans le géocaching. Le pigeonnier est superbe et il est fort regrettable que\Quoted{des artistes} aient apposé leurs œuvres sur ces murs!!! La cache nous attend entre les bras...pas de difficulté. Merci Gilles64\end{cacheText}

\cacheNumber{655}\needspace{5\baselineskip}\cacheName{\href{http://coord.info/GC6DNKN}{Alors on danse et on prie !} — \href{http://coord.info/GC6DNKN\Number{}719106989}{655}}\cacheData{{2017/09/09 gilles64, Traditional Cache (1.5/1.5)}}\begin{cacheText}Une petite virée sur Bayonne pour l'achat des fournitures scolaires et hop une cache pour se détendre!!!! Cette jolie église nous livre rapidement son secret grâce à l'indice. Merci pour la découverte de ce lieu et toutes ces informations sur la dance et l'église.\end{cacheText}

\cacheNumber{656}\needspace{5\baselineskip}\cacheName{\href{http://coord.info/GC5CGD1}{Eglise de Saint Martin de Seignanx} — \href{http://coord.info/GC5CGD1\Number{}719312977}{656}}\cacheData{{2017/09/10 Aino4064, Traditional Cache (1.5/2.5)}}\begin{cacheText}Très jolie église, pas de difficulté particulière. .. La cache était parterre :nous l'avons remise en  place et l'avons camouflée un petit peu;l'indice n'est plus d'actualité. Merci pour la découverte de ce lieu\end{cacheText}

\cacheNumber{657}\needspace{5\baselineskip}\cacheName{\href{http://coord.info/GC5CGEE}{Chateau rouge Saint Martin de Seignanx} — \href{http://coord.info/GC5CGEE\Number{}719316717}{657}}\cacheData{{2017/09/10 Aino4064, Traditional Cache (1.5/1.5)}}\begin{cacheText}C'est en arrivant sur les lieux que l'on comprend l'indice. La cache est vite débusquée...Merci\end{cacheText}

\cacheNumber{658}\needspace{5\baselineskip}\cacheName{\href{http://coord.info/GC5F887}{ORX - Eglise Saint-Martin} — \href{http://coord.info/GC5F887\Number{}719311598}{658}}\cacheData{{2017/09/10 gilles64, Traditional Cache (1.5/1.5)}}\begin{cacheText}La cache a été rapidement découverte !! Merci pour la découverte de cette petite église.\end{cacheText}

\cacheNumber{659}\needspace{5\baselineskip}\cacheName{\href{http://coord.info/GC6F2D3}{ORX-O1} — \href{http://coord.info/GC6F2D3\Number{}719324509}{659}}\cacheData{{2017/09/10 mizaga, Unknown Cache (1.5/1.5)}}\begin{cacheText}Quel étrange animal perdu dans ce marais !!!La vue est superbe et le logbook est tout neuf .Merci pour la cache ,un PF pour le récompenser le travail\end{cacheText}

\cacheNumber{660}\needspace{5\baselineskip}\cacheName{\href{http://coord.info/GC6F2DD}{ORX-O2} — \href{http://coord.info/GC6F2DD\Number{}719324964}{660}}\cacheData{{2017/09/10 mizaga, Unknown Cache (1.5/1.5)}}\begin{cacheText}L'indice est d'une aide précieuse .La cache est trouvée rapidement ...merci 😊\end{cacheText}

\cacheNumber{661}\needspace{5\baselineskip}\cacheName{\href{http://coord.info/GC6F2DN}{ORX-O3} — \href{http://coord.info/GC6F2DN\Number{}719523518}{661}}\cacheData{{2017/09/10 mizaga, Unknown Cache (1.5/1.5)}}\begin{cacheText}La belle a été rapidement trouvée, l'indice est rigolo... Merci pour la cache.\end{cacheText}

\cacheNumber{662}\needspace{5\baselineskip}\cacheName{\href{http://coord.info/GC6F2DR}{ORX-O4} — \href{http://coord.info/GC6F2DR\Number{}719524490}{662}}\cacheData{{2017/09/10 mizaga, Unknown Cache (1.5/1.5)}}\begin{cacheText}Nous avons trouvé facilement la cache mais c'est du mini mini. Merci\end{cacheText}

\cacheNumber{663}\needspace{5\baselineskip}\cacheName{\href{http://coord.info/GC6F2DZ}{ORX-O5} — \href{http://coord.info/GC6F2DZ\Number{}719326697}{663}}\cacheData{{2017/09/10 mizaga, Unknown Cache (2/2)}}\begin{cacheText}Une cache comme je les aime du vrai Géocaching!!!! Nous avons également mal interprété l'indice mais à force de recherches nous avons finis par mettre la main ou plutot le pied dessus!!!! Merci\end{cacheText}

\cacheNumber{664}\needspace{5\baselineskip}\cacheName{\href{http://coord.info/GC6F2E3}{ORX-O6} — \href{http://coord.info/GC6F2E3\Number{}719527555}{664}}\cacheData{{2017/09/10 mizaga, Unknown Cache (2/2)}}\begin{cacheText}Cette cache est excellente....Nous avons mis un peu de temps pour savoir comment récupérer le logbook  mais nous avons réussi!!!Oufff!!! Un PF suplementaire voir deux!!!Nous avons passé un bon moment, merci.\end{cacheText}

\cacheNumber{665}\needspace{5\baselineskip}\cacheName{\href{http://coord.info/GC6F2EM}{Orx-O8} — \href{http://coord.info/GC6F2EM\Number{}719328133}{665}}\cacheData{{2017/09/10 DorisBear, Unknown Cache (3/1.5)}}\begin{cacheText}La cache est trouvée en deux temps trois mouvements merci\end{cacheText}

\cacheNumber{666}\needspace{5\baselineskip}\cacheName{\href{http://coord.info/GC6F2EV}{Orx-O9} — \href{http://coord.info/GC6F2EV\Number{}719330474}{666}}\cacheData{{2017/09/10 DorisBear, Unknown Cache (2/2)}}\begin{cacheText}J'adore... Cette coquine nous donne du fil à retordre: elle est parfaitement intégrée au paysage!!! Enfin nous avons fini par mettre la main dessus... quel bonheur !!!Merci pour tout ce travail et un PF pour le récompenser\end{cacheText}

\cacheNumber{667}\needspace{5\baselineskip}\cacheName{\href{http://coord.info/GC6F2EZ}{Orx-R6} — \href{http://coord.info/GC6F2EZ\Number{}719538110}{667}}\cacheData{{2017/09/10 DorisBear, Unknown Cache (2/1.5)}}\begin{cacheText}Encore une très belle réalisation ...que du plaisir sur ce parcours!!! La cache est vite trouvée car le GPS est précis sur ce coup là. Merci un

Point Favoris de plus\end{cacheText}

\cacheNumber{668}\needspace{5\baselineskip}\cacheName{\href{http://coord.info/GC6F2F2}{Orx-R11} — \href{http://coord.info/GC6F2F2\Number{}719543220}{668}}\cacheData{{2017/09/10 DorisBear, Unknown Cache (2/1.5)}}\begin{cacheText}C'est une cache élaborée comme je les aime. C'est encore une belle surprise .Merci et un autre PF\end{cacheText}

\cacheNumber{669}\needspace{5\baselineskip}\cacheName{\href{http://coord.info/GC6F2F7}{Orx-X1} — \href{http://coord.info/GC6F2F7\Number{}719539628}{669}}\cacheData{{2017/09/10 DorisBear, Unknown Cache (2/2)}}\begin{cacheText}Ici aussi une belle surprise nous attend, superbe réalisation, que du bonheur!!! Du vrai géocaching!!! Un PF supplémentaire .Merci pour ce bon moment\end{cacheText}

\cacheNumber{670}\needspace{5\baselineskip}\cacheName{\href{http://coord.info/GC6F2X5}{ORX-O7} — \href{http://coord.info/GC6F2X5\Number{}719327571}{670}}\cacheData{{2017/09/10 mizaga, Unknown Cache (1.5/1.5)}}\begin{cacheText}La vue est toujours aussi belle. Ici la cache ne présente pas de problème.Merci\end{cacheText}

\cacheNumber{671}\needspace{5\baselineskip}\cacheName{\href{http://coord.info/GC6F2XY}{ORX-O10} — \href{http://coord.info/GC6F2XY\Number{}719333018}{671}}\cacheData{{2017/09/10 mizaga, Unknown Cache (2/2)}}\begin{cacheText}Ici aussi nous trouvons une cache très originale je n'en dirais pas plus!!! Merci pour tout ce travaiL et un PF\end{cacheText}

\cacheNumber{672}\needspace{5\baselineskip}\cacheName{\href{http://coord.info/GC6F2YD}{ORX-O11} — \href{http://coord.info/GC6F2YD\Number{}719339502}{672}}\cacheData{{2017/09/10 mizaga, Unknown Cache (2/2)}}\begin{cacheText}Ici aussi du vrai géocaching comme je l'aime .La cache nous a pris un peu de temps mais nous avons fini par la trouver!!! L'indice peut être mal interprété ,je parlerais plus de buche que de souche!! Merci pour le travail ....et un PF\end{cacheText}

\cacheNumber{673}\needspace{5\baselineskip}\cacheName{\href{http://coord.info/GC6F2ZG}{ORX-O12} — \href{http://coord.info/GC6F2ZG\Number{}719341495}{673}}\cacheData{{2017/09/10 mizaga, Unknown Cache (2/2)}}\begin{cacheText}Nous avons retrouvé Shaun le .... l'indice est très explicite et il nous a bien aidé dans les recherches car le GPS n'est pas très fiable merci pour ce travail et un PF\end{cacheText}

\cacheNumber{674}\needspace{5\baselineskip}\cacheName{\href{http://coord.info/GC6F3RH}{ORX-R1} — \href{http://coord.info/GC6F3RH\Number{}719322630}{674}}\cacheData{{2017/09/10 mizaga, Unknown Cache (2/1.5)}}\begin{cacheText}Un intrus se trouve dans les bras...Nous avons trouvé facilement.Une maintenance est à effectuer car le log book est archi clair merci\end{cacheText}

\cacheNumber{675}\needspace{5\baselineskip}\cacheName{\href{http://coord.info/GC6F3T2}{ORX-R2} — \href{http://coord.info/GC6F3T2\Number{}719321504}{675}}\cacheData{{2017/09/10 mizaga, Unknown Cache (2/1.5)}}\begin{cacheText}Le cadre est vraiment superbe. Malgré l'absence d'indices nous trouvons facilement l'endroit. Merci pour la cache\end{cacheText}

\cacheNumber{676}\needspace{5\baselineskip}\cacheName{\href{http://coord.info/GC6F3TV}{ORX-R3} — \href{http://coord.info/GC6F3TV\Number{}719320005}{676}}\cacheData{{2017/09/10 mizaga, Unknown Cache (1.5/1.5)}}\begin{cacheText}Pas de difficulté particulière pour celle-ci .C'est un classique...je confirme que le logbook est plein(j'ai signé sur le coté).Merci\end{cacheText}

\cacheNumber{677}\needspace{5\baselineskip}\cacheName{\href{http://coord.info/GC6F3X8}{ORX-R8} — \href{http://coord.info/GC6F3X8\Number{}719535411}{677}}\cacheData{{2017/09/10 mizaga, Unknown Cache (2/2)}}\begin{cacheText}Pas mal de moldus ont décidé de se promener et ce n'est pas évident de débusquer la belle!!! Mais avec de la patience nous arrivons à mettre la main dessus!!! attention les frelons sont toujours là!!! Merci pour cette decouverte\end{cacheText}

\cacheNumber{678}\needspace{5\baselineskip}\cacheName{\href{http://coord.info/GC6F3Y1}{ORX-R9} — \href{http://coord.info/GC6F3Y1\Number{}719531107}{678}}\cacheData{{2017/09/10 mizaga, Unknown Cache (2/1.5)}}\begin{cacheText}C'est dans un cadre magnifique que nous débusquons la belle merci pour cette belle promenade et ce grand plaisir\end{cacheText}

\cacheNumber{679}\needspace{5\baselineskip}\cacheName{\href{http://coord.info/GC6F3YA}{ORX-R10} — \href{http://coord.info/GC6F3YA\Number{}719542380}{679}}\cacheData{{2017/09/10 mizaga, Unknown Cache (2/2)}}\begin{cacheText}Encore une superbe réalisation que l'on trouve sans difficulté.Malheureusement l'arbre est à terre. Nous camouflons comme nous pouvons le dit pioupiou; Merci et un PF pour cette belle réalisation\end{cacheText}

\cacheNumber{680}\needspace{5\baselineskip}\cacheName{\href{http://coord.info/GC6F3YW}{ORX-R12} — \href{http://coord.info/GC6F3YW\Number{}719544648}{680}}\cacheData{{2017/09/10 mizaga, Unknown Cache (2/2)}}\begin{cacheText}Cache trouvée très rapidement, certainement un coup de chance!!!C'est un  logbook de maintenance, espérons qu'il n'y ait pas d'indice . Merci\end{cacheText}

\cacheNumber{681}\needspace{5\baselineskip}\cacheName{\href{http://coord.info/GC6F3Z7}{ORX-X2} — \href{http://coord.info/GC6F3Z7\Number{}719540465}{681}}\cacheData{{2017/09/10 mizaga, Unknown Cache (2/2)}}\begin{cacheText}Excellente ...en effet cette cache est dur dur. Monsieur s'en est vu mais a fini par la débusquer encore un PF merci\end{cacheText}

\cacheNumber{682}\needspace{5\baselineskip}\cacheName{\href{http://coord.info/GC668CP}{Le premier du Boudigau} — \href{http://coord.info/GC668CP\Number{}719736083}{682}}\cacheData{{2017/09/11 benfx, Traditional Cache (1.5/1)}}\begin{cacheText}La nouvelle boîte a rapidement été localisée. Nous avons fait celle-ci à la fin du parcours des Mysteries du Marais d' Orx. Maintenant direction la bonus merci\end{cacheText}

\cacheNumber{683}\needspace{5\baselineskip}\cacheName{\href{http://coord.info/GC6F2FC}{Orx-X4} — \href{http://coord.info/GC6F2FC\Number{}719735305}{683}}\cacheData{{2017/09/11 DorisBear, Unknown Cache (1.5/1.5)}}\begin{cacheText}Après avoir bien examiné les lieux, il nous semble apercevoir quelque chose....C'est Monsieur qui s'y colle ...pas de problème !!!! Merci 😊 et un PF\end{cacheText}

\cacheNumber{684}\needspace{5\baselineskip}\cacheName{\href{http://coord.info/GC6F3W1}{ORX-R5} — \href{http://coord.info/GC6F3W1\Number{}719736030}{684}}\cacheData{{2017/09/11 mizaga, Unknown Cache (2/2)}}\begin{cacheText}C'est notre deuxième passage :hier il y avait trop de monde à la cabane et le temps nous était conté !!!!! Aujourd'hui ,une aubaine ,il n'y a personne . Nous avons le temps de faire le tour, de regarder ,d'observer et  BINGO on trouve le truc anormal merci pour la cache un PF\end{cacheText}

\cacheNumber{685}\needspace{5\baselineskip}\cacheName{\href{http://coord.info/GC6F403}{ORX-X5 BONUS} — \href{http://coord.info/GC6F403\Number{}719739672}{685}}\cacheData{{2017/09/11 mizaga, Unknown Cache (2/1.5)}}\begin{cacheText}Nous avons finis ce superbe parcours ( 26 Found,3 DNF, 2 désactivées) et il nous manque 1 indice!!! C'est au hasard que l'on tape les coordonnées et au bout de 5 minutes le geocheck passé au vert.... direction La bonus. Il ne nous a fallu qu'une minute pour la trouver. Que du bonheur !!! Échange de TB et Log signé. Merci  😊 pour tout ce travail et PF pour récompenser l'ensemble.\end{cacheText}

\cacheNumber{686}\needspace{5\baselineskip}\cacheName{\href{http://coord.info/GC6F40J}{ORX-X6} — \href{http://coord.info/GC6F40J\Number{}719734466}{686}}\cacheData{{2017/09/11 mizaga, Unknown Cache (2/2)}}\begin{cacheText}Nous faisons de belles découvertes ...des canards sauvages ,des spatules ,des oiseaux divers et variés et nous trouvons même des papillons !!! Quel bel endroit !!!Merci pour la cache que nous avons vite aperçu depuis le Chemin.\end{cacheText}

\cacheNumber{687}\needspace{5\baselineskip}\cacheName{\href{http://coord.info/GC6F40Y}{ORX-X7} — \href{http://coord.info/GC6F40Y\Number{}719706425}{687}}\cacheData{{2017/09/11 mizaga, Unknown Cache (2/2.5)}}\begin{cacheText}Nous reprenons ce jour la suite du parcours .La belle est dénichée en deux temps trois mouvements ;belle réalisation!!! Merci\end{cacheText}

\cacheNumber{688}\needspace{5\baselineskip}\cacheName{\href{http://coord.info/GC7C8RJ}{GO02 Gaineko Ordokia - Iparlatze} — \href{http://coord.info/GC7C8RJ\Number{}720120898}{688}}\cacheData{{2017/09/13 gilles64, Traditional Cache (1.5/2.5)}}\begin{cacheText}[{FTF}]

Nous avons continué la promenade au milieu de quelques carcasses de brebis( pas très rassurant !!!) et sommes arrivés au PZ. Les tentacules n'ont pas livré facilement la Belle mais nous avons finis par l'avoir. Ici aussi nous sommes FTF, quel plaisir !!! La vue est à couper le souffle mais nous apercevons au loin des vautours !!!! Il est temps de partir\end{cacheText}

\cacheNumber{689}\needspace{5\baselineskip}\cacheName{\href{http://coord.info/GC7C8R5}{GO01 Gaineko Ordokia - Iparlatze} — \href{http://coord.info/GC7C8R5\Number{}720116850}{689}}\cacheData{{2017/09/14 gilles64, Traditional Cache (1.5/2.5)}}\begin{cacheText}[{FTF}]

L'alerte a sonné hier vers 18h00 mais il était un peu tard et le temps très pluvieux . Tant pis pour le FTF..... En sortant du boulot un rapide coup d'œil à l'application et je constate que personne n'a encore logué . De plus aujourd'hui le temps est superbe pour une promenade. Nous arrivons un peu essoufflé sur le lieu et repérons facilement la cache. Fébriles, nous déroulons le logbook et surprise nous sommes les premiers!!! Merci pour cette balade, la vue est magnifique .\end{cacheText}

\cacheNumber{690}\needspace{5\baselineskip}\cacheName{\href{http://coord.info/GC677Y8}{Col d'Iparlatze} — \href{http://coord.info/GC677Y8\Number{}720184287}{690}}\cacheData{{2017/09/14 gilles64, Traditional Cache (1.5/1.5)}}\begin{cacheText}La cache est vite localisée mais encore faut-il l'attraper!!!!! C'est chose faite mais avec difficulté !!! Merci Gilles pour la découverte de ce magnifique col\end{cacheText}

\cacheNumber{691}\needspace{5\baselineskip}\cacheName{\href{http://coord.info/GC67WNA}{Chateau Laxague - Ostabat-Asme} — \href{http://coord.info/GC67WNA\Number{}720184289}{691}}\cacheData{{2017/09/14 gilles64, Traditional Cache (2/1.5)}}\begin{cacheText}Quelle belle bâtisse perdue en pleine campagne !!! Nous repérons facilement la cache grâce à l'indice mais La locataire des lieux est dans les parages. Nous avons regardé le château et elle nous a invité à visiter la cour. Vraiment très sympa ces murs chargés d'histoire !!! C'est en repartant que nous loguons discrètement . Merci Gilles pour cette belle découverte. 😃\end{cacheText}

\cacheNumber{692}\needspace{5\baselineskip}\cacheName{\href{http://coord.info/GC4MVM0}{AutoStop - A83 - Aire La Canepetière} — \href{http://coord.info/GC4MVM0\Number{}720315659}{692}}\cacheData{{2017/09/15 FlavignyTeam, Traditional Cache (1.5/2)}}\begin{cacheText}En route pour l'Event Entre Mer et Marais de Saint André des Eaux nous ne pouvons pas faire autrement que s'arrêter pour faire cette cache.. Pas de difficulté tout est dit!!! Merci 😊\end{cacheText}

\cacheNumber{693}\needspace{5\baselineskip}\cacheName{\href{http://coord.info/GC6A79F}{Fort-Boyard, le conseil 30 (flechettes)} — \href{http://coord.info/GC6A79F\Number{}721071596}{693}}\cacheData{{2017/09/15 Bonnet rouge, Traditional Cache (2/1.5)}}\begin{cacheText}Dans la région pour participer à l' Évent entre Mer et Marais de Saint André des Eaux, nous en profitons pour faire quelques caches.Nous démarrons le circuit , qui nous a été recommandé ,un peu dans le désordre en compagnie de Fabilab .Cette boite est vite trouvée par Monsieur . Merci\end{cacheText}

\cacheNumber{694}\needspace{5\baselineskip}\cacheName{\href{http://coord.info/GC6A79T}{Fort-Boyard, le conseil 31 (compteur)} — \href{http://coord.info/GC6A79T\Number{}721072634}{694}}\cacheData{{2017/09/15 Bonnet rouge, Traditional Cache (3.5/2.5)}}\begin{cacheText}Comme notre prédécesseur nous trouvons un reste de la cache (il ne reste plus rien si ce n'est le caillou et le couvercle ). Dommage une maintenance serait nécessaire . Merci 😊\end{cacheText}

\cacheNumber{695}\needspace{5\baselineskip}\cacheName{\href{http://coord.info/GC6A7A5}{Fort-Boyard, le conseil 32 (paires)} — \href{http://coord.info/GC6A7A5\Number{}721074412}{695}}\cacheData{{2017/09/15 Bonnet rouge, Traditional Cache (3.5/1.5)}}\begin{cacheText}Après avoir lu les indices nous cherchons la Belle  en compagnie de Fabilab,et c'est Stéphane qui la dégote !!!!un très bel ouvrage ....du plaisir merci\end{cacheText}

\cacheNumber{696}\needspace{5\baselineskip}\cacheName{\href{http://coord.info/GC6A5KK}{Fort-Boyard 1 l'entrée au fort (la grille)} — \href{http://coord.info/GC6A5KK\Number{}721189464}{696}}\cacheData{{2017/09/15 Bonnet rouge, Traditional Cache (2/1.5)}}\begin{cacheText}Arrivé sur les lieux ,Stéphane découvre vite fait la cache. Elle est vraiment très sympa merci\end{cacheText}

\cacheNumber{697}\needspace{5\baselineskip}\cacheName{\href{http://coord.info/GC6ZE8X}{entre vents et marées 01} — \href{http://coord.info/GC6ZE8X\Number{}720329498}{697}}\cacheData{{2017/09/15 Bonnet rouge, Traditional Cache (1.5/1.5)}}\begin{cacheText}Trouvée en compagnie de Fabilab .Venus pour l'Event entre Mer et Marais nous décidons de faire quelques caches en attendant 19h!!!!! Très jolie boîte et merci pour la découverte de cette belle côte atlantique.\end{cacheText}

\cacheNumber{698}\needspace{5\baselineskip}\cacheName{\href{http://coord.info/GC7CBXQ}{PTTTN - Night 1} — \href{http://coord.info/GC7CBXQ\Number{}721219427}{698}}\cacheData{{2017/09/15 lazuli44, les cinq \And{} ptitloup, Unknown Cache (1.5/2)}}\begin{cacheText}(FTF)

Après la présentation de l Évent, du géoblabla accompagné du géoapéro et des ripailles ,nous partons ,à la nuit tombée , dans la bonne humeur réviser le code de la route. Cache super sympa trouvée avec la Night Team. Un grand merci pour ce super moment.😃😃😃😃\end{cacheText}

\cacheNumber{699}\needspace{5\baselineskip}\cacheName{\href{http://coord.info/GC6A5M5}{Fort-Boyard, epreuve 2 (le musée)} — \href{http://coord.info/GC6A5M5\Number{}721204673}{699}}\cacheData{{2017/09/15 Bonnet rouge, Traditional Cache (3.5/2.5)}}\begin{cacheText}En arrivant sur les lieux la cache est rapidement repérée :très belle surprise!!!! Mais comment l'ouvrir?après réflexion nous arrivons à décoder et à ouvrir la boîte .Merci Bonnet Rouge ,nous avons passé un très bon moment grâce à toi. Élle mérite un PF évidemment...\end{cacheText}

\cacheNumber{700}\needspace{5\baselineskip}\cacheName{\href{http://coord.info/GC6A5MD}{Fort-Boyard, epreuve 3 (Magic Academy)} — \href{http://coord.info/GC6A5MD\Number{}721206206}{700}}\cacheData{{2017/09/15 Bonnet rouge, Traditional Cache (2/4)}}\begin{cacheText}Les belles caches se suivent mais ne se ressemblent pas !!!! Encore une super trouvaille !!! Merci pour la découverte de ce lieu.\end{cacheText}

\cacheNumber{701}\needspace{5\baselineskip}\cacheName{\href{http://coord.info/GC6A5R6}{Fort-Boyard, epreuve 4 (le casino)} — \href{http://coord.info/GC6A5R6\Number{}721206496}{701}}\cacheData{{2017/09/15 Bonnet rouge, Traditional Cache (2/2.5)}}\begin{cacheText}Cache rapidement trouvée car elle était à l'envers :dommage !!!! Géocacheurs,pensez à remettre la cache dans son état initial !!!!Merci pour tous\end{cacheText}

\cacheNumber{702}\needspace{5\baselineskip}\cacheName{\href{http://coord.info/GC6A5RA}{Fort-Boyard, epreuve 5 (ketchuperie)} — \href{http://coord.info/GC6A5RA\Number{}721207033}{702}}\cacheData{{2017/09/15 Bonnet rouge, Traditional Cache (3/4)}}\begin{cacheText}Cache découverte mais malheureusement des prédécesseurs ont abîmé en ouvrant avec le tournevis au lieu de chercher le code !!!!C'est vraiment dommage pour la boîte et peu respectueux du travail de Bonnet Rouge!!! Merci pour le travail.\end{cacheText}

\cacheNumber{703}\needspace{5\baselineskip}\cacheName{\href{http://coord.info/GC6A5RR}{Fort-Boyard, epreuve 6 (Cellule interactive)} — \href{http://coord.info/GC6A5RR\Number{}721208037}{703}}\cacheData{{2017/09/15 Bonnet rouge, Traditional Cache (2/2.5)}}\begin{cacheText}Arrivés près de la zone,nous repérons un petit passage et nous nous y engouffrons. En deux temps trois mouvements nous repérons la cache . Il n'existe plus aucun mécanisme pour l'ouvrir, bien dommage ,l'idée était vraiment très sympa . Merci\end{cacheText}

\cacheNumber{704}\needspace{5\baselineskip}\cacheName{\href{http://coord.info/GC6A5RW}{Fort-Boyard, epreuve 7 (Alerte rouge)} — \href{http://coord.info/GC6A5RW\Number{}721209274}{704}}\cacheData{{2017/09/15 Bonnet rouge, Traditional Cache (2/2)}}\begin{cacheText}Après avoir mangé un excellent repas au Belem, nous attendons que la pluie passe puis  nous reprenons le circuit .  Cette cache est ,elle aussi ,parfaitement intégrée dans le décor. Cela mérite un PF merci Bonnet Rouge\end{cacheText}

\cacheNumber{705}\needspace{5\baselineskip}\cacheName{\href{http://coord.info/GC6A5RZ}{Fort-Boyard, epreuve 8 (Plateau 215)} — \href{http://coord.info/GC6A5RZ\Number{}721209625}{705}}\cacheData{{2017/09/15 Bonnet rouge, Traditional Cache (3/2.5)}}\begin{cacheText}Arrivés sur les lieux nous avons un peu de mal à trouver ...les coordonnées GPS nous baladent et c'est Fabienne qui finit par la découvrir à 9 m du point indiqué. Merci\end{cacheText}

\cacheNumber{706}\needspace{5\baselineskip}\cacheName{\href{http://coord.info/GC6A5V0}{Fort-Boyard, epreuve 9 (Menotte)} — \href{http://coord.info/GC6A5V0\Number{}721209926}{706}}\cacheData{{2017/09/15 Bonnet rouge, Traditional Cache (3.5/3)}}\begin{cacheText}La cache a été dénichée rapidement mais nous avons mis beaucoup plus de temps pour trouver le code !!!Nous avons enfin réussi merci pour cette belle réalisation . Un PF de plus et dépôt de TB\end{cacheText}

\cacheNumber{707}\needspace{5\baselineskip}\cacheName{\href{http://coord.info/GC6A5WM}{Fort-Boyard, epreuve 11 (les cylindres)} — \href{http://coord.info/GC6A5WM\Number{}721210812}{707}}\cacheData{{2017/09/15 Bonnet rouge, Traditional Cache (2.5/2.5)}}\begin{cacheText}Ici aussi la cache est rapidement decouverte par Fabienne et Pierre .Vite fait ,bien fait nous logons et repartons.  Merci\end{cacheText}

\cacheNumber{708}\needspace{5\baselineskip}\cacheName{\href{http://coord.info/GC6A5Y7}{Fort-Boyard, aventure 19 (Manoir hanté)} — \href{http://coord.info/GC6A5Y7\Number{}721211159}{708}}\cacheData{{2017/09/15 Bonnet rouge, Traditional Cache (2/2.5)}}\begin{cacheText}L'heure tourne et le ciel menace, nous décidons de couper le circuit.Celle-ci est trouvée sans difficulté . Encore une très belle cache ,merci.\end{cacheText}

\cacheNumber{709}\needspace{5\baselineskip}\cacheName{\href{http://coord.info/GC6A5YE}{Fort-Boyard, aventure 20 (Tyrolienne inversé)} — \href{http://coord.info/GC6A5YE\Number{}721212564}{709}}\cacheData{{2017/09/15 Bonnet rouge, Traditional Cache (3/2.5)}}\begin{cacheText}Encore une belle trouvaille du géocaching comme je l'aime !!! Merci pour ce bel ouvrage. Un PF de plus.\end{cacheText}

\cacheNumber{710}\needspace{5\baselineskip}\cacheName{\href{http://coord.info/GC6A5YW}{Fort-Boyard, aventure 22 (entrainement sous-marin)} — \href{http://coord.info/GC6A5YW\Number{}721213347}{710}}\cacheData{{2017/09/16 Bonnet rouge, Traditional Cache (2/2.5)}}\begin{cacheText}Enfin, une cache de trouver !!! Ca donne un peu de baume au cœur !!!C'est Fabienne qui nous la déniche :encore une belle réalisation . Merci\end{cacheText}

\cacheNumber{711}\needspace{5\baselineskip}\cacheName{\href{http://coord.info/GC6A77Q}{Fort-Boyard, le conseil 23 (aquarium)} — \href{http://coord.info/GC6A77Q\Number{}721213836}{711}}\cacheData{{2017/09/16 Bonnet rouge, Traditional Cache (2.5/2.5)}}\begin{cacheText}Et une cache de plus trouvée très rapidement . Elle est résolue tout aussi vite . Merci pour ce bel ouvrage, un PF de plus. Nous continuons...\end{cacheText}

\cacheNumber{712}\needspace{5\baselineskip}\cacheName{\href{http://coord.info/GC6A77Y}{Fort-Boyard, le conseil 24 (clous en equilibre)} — \href{http://coord.info/GC6A77Y\Number{}721214351}{712}}\cacheData{{2017/09/16 Bonnet rouge, Traditional Cache (2/2.5)}}\begin{cacheText}Arrivés sur les lieux Fabienne nous déniche la belle en deux temps trois mouvements . Sans tournevis elle sort le Logbook et nous signons . Encore une belle réalisation ,que du plaisir sur ce parcours. Merci\end{cacheText}

\cacheNumber{713}\needspace{5\baselineskip}\cacheName{\href{http://coord.info/GC6A787}{Fort-Boyard, le conseil 25 (souffle)} — \href{http://coord.info/GC6A787\Number{}721214805}{713}}\cacheData{{2017/09/16 Bonnet rouge, Traditional Cache (2/2)}}\begin{cacheText}Sur celle-ci nous mettons du temps :après avoir cherché dans le sous-bois c'est l'ami Stéphane qui  s'eloigne et qui met la main dessus !!!! Nous avons dansé sur la passerelle pour fêter ça !!! Merci pour cette cache astucieuse.\end{cacheText}

\cacheNumber{714}\needspace{5\baselineskip}\cacheName{\href{http://coord.info/GC6A78K}{Fort-Boyard, le conseil 26 (reflexe)} — \href{http://coord.info/GC6A78K\Number{}721215181}{714}}\cacheData{{2017/09/16 Bonnet rouge, Traditional Cache (1.5/4.5)}}\begin{cacheText}C'est PIERRE qui trouve celle-ci . Le GPS ne nous envoie pas trop loin, c'est une cache originale. On n'est vraiment pas déçu !!!! Merci\end{cacheText}

\cacheNumber{715}\needspace{5\baselineskip}\cacheName{\href{http://coord.info/GC6A792}{Fort-Boyard, le conseil 27 (marteau)} — \href{http://coord.info/GC6A792\Number{}721215488}{715}}\cacheData{{2017/09/16 Bonnet rouge, Traditional Cache (2/2.5)}}\begin{cacheText}C'est l'ami Stéphane qui arrive le premier sur les lieux et qui nous dégote la belle et. Les quatre cerveaux sont en ébullition et notre perspicace Stéphane découvre la solution. C'est parfait encore une super cache .Merci\end{cacheText}

\cacheNumber{716}\needspace{5\baselineskip}\cacheName{\href{http://coord.info/GC6A795}{Fort-Boyard, le conseil 29 (Dés)} — \href{http://coord.info/GC6A795\Number{}721216320}{716}}\cacheData{{2017/09/16 Bonnet rouge, Traditional Cache (2/2)}}\begin{cacheText}C'est la dernière de la journée ,nous allons nous approcher de l' Event entre mer et marais. Ici c'est STEPHANE qui la découvre .Encore une belle réalisation . 

Merci pour ce beau circuit.\end{cacheText}

\cacheNumber{717}\needspace{5\baselineskip}\cacheName{\href{http://coord.info/GC710HY}{Entre mer et marais : Acte I} — \href{http://coord.info/GC710HY\Number{}720481786}{717}}\cacheData{{2017/09/16 lazuli44, les cinq \And{} ptitloup, Event Cache (1/1)}}\begin{cacheText}Superbe soirée passée en compagnie de supers geocacheurs . Vivement demain pour partir sur les Chemins...\end{cacheText}

\cacheNumber{718}\needspace{5\baselineskip}\cacheName{\href{http://coord.info/GC3JF6N}{La croix de la ville JAU} — \href{http://coord.info/GC3JF6N\Number{}721220504}{718}}\cacheData{{2017/09/16 les cinq, Traditional Cache (2/1.5)}}\begin{cacheText}En route pour l'Event entre mer et marais nous sommes un peu en avance et nous nous arrêtons pour faire quelques cache sur Saint-André des eaux . Celle-ci est trouvée rapidement . Merci pour la découverte de ce calvaire.\end{cacheText}

\cacheNumber{719}\needspace{5\baselineskip}\cacheName{\href{http://coord.info/GC6CGDW}{PTTTN - M55} — \href{http://coord.info/GC6CGDW\Number{}721967426}{719}}\cacheData{{2017/09/16 les cinq, Traditional Cache (2/2)}}\begin{cacheText}Ici aussi nous trouvons la belle rapidement, heureusement car les batteries arrivent à zéro. De plus, l'heure du pique nique approche à grands pas. Pas le temps d' en faire une autre ...Mplc\end{cacheText}

\cacheNumber{720}\needspace{5\baselineskip}\cacheName{\href{http://coord.info/GC6CGEH}{PTTTN - M56} — \href{http://coord.info/GC6CGEH\Number{}721966213}{720}}\cacheData{{2017/09/16 les cinq, Multi-cache (2/1.5)}}\begin{cacheText}Les batteries du téléphone sont presque déchargées ! ! ! Il nous faut faire vite pour trouver la Multi. Grâce à l'aide des DAK et de Thomy56 nous délogeons la Belle. Merci pour la cache.\end{cacheText}

\cacheNumber{721}\needspace{5\baselineskip}\cacheName{\href{http://coord.info/GC6CGEJ}{PTTTN - M57} — \href{http://coord.info/GC6CGEJ\Number{}721965577}{721}}\cacheData{{2017/09/16 les cinq, Traditional Cache (1.5/2)}}\begin{cacheText}[{TTF}]

Vraiment, on trouve de tout sur ce circuit!!!Cache très originale ,trouvée le long du chemin. Un PF supplémentaire. Merci pour tout ce travail.\end{cacheText}

\cacheNumber{722}\needspace{5\baselineskip}\cacheName{\href{http://coord.info/GC6CGEM}{PTTTN - M58} — \href{http://coord.info/GC6CGEM\Number{}721965348}{722}}\cacheData{{2017/09/16 les cinq, Traditional Cache (1.5/1.5)}}\begin{cacheText}[{TTF}]

La cache nous attend au centre. Nous reprenons la route. Merci\end{cacheText}

\cacheNumber{723}\needspace{5\baselineskip}\cacheName{\href{http://coord.info/GC6CGEV}{PTTTN - M59} — \href{http://coord.info/GC6CGEV\Number{}721964986}{723}}\cacheData{{2017/09/16 les cinq, Traditional Cache (2/2)}}\begin{cacheText}[{STF}]

Nous nous enfonçons dans le bois et finissons par repérer l'arbre du spoiler. Cache super sympa que nous adorons ainsi que son système. Un PF évidemment.Merci\end{cacheText}

\cacheNumber{724}\needspace{5\baselineskip}\cacheName{\href{http://coord.info/GC6CGEW}{PTTTN - M60} — \href{http://coord.info/GC6CGEW\Number{}721964423}{724}}\cacheData{{2017/09/16 les cinq, Traditional Cache (2/2)}}\begin{cacheText}[{STF}]

Arrivés sur les lieux il nous faut trouver le bon trou. C'est chose faite par Monsieur et qui dit qu'il ne peut pas l'attraper, c'est impossible!!!! La fatigue se fait vraiment sentir… je regarde la belle et tire le petit fill!!!Ah ah ah ah . Merci pour ce bon moment\end{cacheText}

\cacheNumber{725}\needspace{5\baselineskip}\cacheName{\href{http://coord.info/GC6CGF0}{PTTTN - M61} — \href{http://coord.info/GC6CGF0\Number{}721960542}{725}}\cacheData{{2017/09/16 les cinq, Traditional Cache (1.5/1.5)}}\begin{cacheText}[{TTF}]

Nous montons en haut du talus et découvrons la belle sans difficulté. Merci.\end{cacheText}

\cacheNumber{726}\needspace{5\baselineskip}\cacheName{\href{http://coord.info/GC6CGF1}{PTTTN - M62} — \href{http://coord.info/GC6CGF1\Number{}721960304}{726}}\cacheData{{2017/09/16 les cinq, Traditional Cache (1.5/1.5)}}\begin{cacheText}[{STF}]

L' indice et le spoiler sont d'une aide précieuse. Merci pour la cache.\end{cacheText}

\cacheNumber{727}\needspace{5\baselineskip}\cacheName{\href{http://coord.info/GC6CGF3}{PTTTN - M63} — \href{http://coord.info/GC6CGF3\Number{}721960047}{727}}\cacheData{{2017/09/16 les cinq, Traditional Cache (1.5/1.5)}}\begin{cacheText}[{STF}]

La cache nous attend sagement à sa place. Merci\end{cacheText}

\cacheNumber{728}\needspace{5\baselineskip}\cacheName{\href{http://coord.info/GC6CGF4}{PTTTN - M64} — \href{http://coord.info/GC6CGF4\Number{}721959681}{728}}\cacheData{{2017/09/16 les cinq, Traditional Cache (1.5/1.5)}}\begin{cacheText}[{STF}]

Nous trouvons et logons la cache discrètement et rapidement. Merci.\end{cacheText}

\cacheNumber{729}\needspace{5\baselineskip}\cacheName{\href{http://coord.info/GC6CGF8}{PTTTN - M65} — \href{http://coord.info/GC6CGF8\Number{}721959369}{729}}\cacheData{{2017/09/16 les cinq, Traditional Cache (1.5/1.5)}}\begin{cacheText}[{FTF}]

Arrivés sur le PZ, le GPS nous fait tourner en rond. À l'aide du spoiler nous finissons par déloger la belle. Il manque le log book, nous le remplaçons par un bout de papier. Merci 😊\end{cacheText}

\cacheNumber{730}\needspace{5\baselineskip}\cacheName{\href{http://coord.info/GC6CGFB}{PTTTN - M66} — \href{http://coord.info/GC6CGFB\Number{}721959115}{730}}\cacheData{{2017/09/16 les cinq, Traditional Cache (1.5/1.5)}}\begin{cacheText}[{FTF}]

Nous complétons notre toute petite collection de FTF en arrivant les premiers. La cache nous attend bien sagement au pied des trois. Merci.\end{cacheText}

\cacheNumber{731}\needspace{5\baselineskip}\cacheName{\href{http://coord.info/GC6CGFD}{PTTTN - M67} — \href{http://coord.info/GC6CGFD\Number{}721958371}{731}}\cacheData{{2017/09/16 les cinq, Traditional Cache (1.5/1.5)}}\begin{cacheText}[{FTF}]

Surprise !!!Nous sommes arrivés les premiers sur celle-là. Une cache trouvée facilement. Merci\end{cacheText}

\cacheNumber{732}\needspace{5\baselineskip}\cacheName{\href{http://coord.info/GC6CGFF}{PTTTN - M68} — \href{http://coord.info/GC6CGFF\Number{}721958162}{732}}\cacheData{{2017/09/16 les cinq, Traditional Cache (1.5/1.5)}}\begin{cacheText}[{STF}]

Pas de difficulté, nous trouvons la cache rapidement. Nous continuons notre route...Merci\end{cacheText}

\cacheNumber{733}\needspace{5\baselineskip}\cacheName{\href{http://coord.info/GC6CGGA}{PTTTN - M69} — \href{http://coord.info/GC6CGGA\Number{}721957540}{733}}\cacheData{{2017/09/16 les cinq, Traditional Cache (1.5/1.5)}}\begin{cacheText}[{STF}]

Et hop une de plus !!!Merci pour la cache.\end{cacheText}

\cacheNumber{734}\needspace{5\baselineskip}\cacheName{\href{http://coord.info/GC6CGGF}{PTTTN - M70} — \href{http://coord.info/GC6CGGF\Number{}721956875}{734}}\cacheData{{2017/09/16 les cinq, Traditional Cache (1.5/2)}}\begin{cacheText}[{STF}]

Une autre très belle cache de découverte. Quel beau travail ! Nous sommes vraiment gâté sur ce circuit. Félicitations et merci à l'owner. Un PF en plus ....\end{cacheText}

\cacheNumber{735}\needspace{5\baselineskip}\cacheName{\href{http://coord.info/GC6CGGM}{PTTTN - M71} — \href{http://coord.info/GC6CGGM\Number{}721808166}{735}}\cacheData{{2017/09/16 les cinq, Traditional Cache (2/1.5)}}\begin{cacheText}[{STF}]

Nous trouvons ici une cache avec un très bon camouflage . Un grand bravo et un grand merci pour tout ce travail.\end{cacheText}

\cacheNumber{736}\needspace{5\baselineskip}\cacheName{\href{http://coord.info/GC6CGY6}{PTTTN - M72} — \href{http://coord.info/GC6CGY6\Number{}721808025}{736}}\cacheData{{2017/09/16 les cinq, Traditional Cache (1.5/1.5)}}\begin{cacheText}[{FTF}]

Cette cache nous a donné du fil à retordre. Elle est bien intégrée dans le paysage mais nous avons fini par mettre la main dessus ouffff.Merci pour la cache.\end{cacheText}

\cacheNumber{737}\needspace{5\baselineskip}\cacheName{\href{http://coord.info/GC6CGYN}{PTTTN - M73} — \href{http://coord.info/GC6CGYN\Number{}721807926}{737}}\cacheData{{2017/09/16 les cinq, Traditional Cache (1.5/1.5)}}\begin{cacheText}[{STF}]

Nous découvrons la cache après Papybiker. Elle ne nous a pas posé de problème. Merci 😊\end{cacheText}

\cacheNumber{738}\needspace{5\baselineskip}\cacheName{\href{http://coord.info/GC6CGZ0}{PTTTN - M74} — \href{http://coord.info/GC6CGZ0\Number{}721807752}{738}}\cacheData{{2017/09/16 les cinq, Traditional Cache (1.5/1.5)}}\begin{cacheText}[{FTF}]

Encore une très belle cache, comme je les aime!!! Un grand merci 😃pour ce bel ouvrage et un PF\end{cacheText}

\cacheNumber{739}\needspace{5\baselineskip}\cacheName{\href{http://coord.info/GC6CGZ6}{PTTTN - M75} — \href{http://coord.info/GC6CGZ6\Number{}721807646}{739}}\cacheData{{2017/09/16 les cinq, Traditional Cache (1.5/1.5)}}\begin{cacheText}[{FTF}]

Grosse surprise en découvrant la cache. Superbe réalisation et un grand merci 😊. Un PF évidemment.\end{cacheText}

\cacheNumber{740}\needspace{5\baselineskip}\cacheName{\href{http://coord.info/GC6CGZJ}{PTTTN - M77} — \href{http://coord.info/GC6CGZJ\Number{}721807350}{740}}\cacheData{{2017/09/16 les cinq, Traditional Cache (1.5/1.5)}}\begin{cacheText}[{FTF}]

Nous découvrons la cache rapidement et ... surprise nous sommes premiers!!! Super contents nous repartons. Merci pour le travail\end{cacheText}

\cacheNumber{741}\needspace{5\baselineskip}\cacheName{\href{http://coord.info/GC6CGZP}{PTTTN - M76} — \href{http://coord.info/GC6CGZP\Number{}721807517}{741}}\cacheData{{2017/09/16 les cinq, Traditional Cache (2/1.5)}}\begin{cacheText}[{FTF}]

La cache nous attend bien sagement au pied!!!Super promenade et Merci pour la cache.\end{cacheText}

\cacheNumber{742}\needspace{5\baselineskip}\cacheName{\href{http://coord.info/GC6CGZZ}{PTTTN - M78} — \href{http://coord.info/GC6CGZZ\Number{}721806886}{742}}\cacheData{{2017/09/16 les cinq, Traditional Cache (1.5/1.5)}}\begin{cacheText}[{STF}]

Sur cette cache Papybiker nous a devancé…Les caches dans le lierre ne sont pas nos préférées mais nous finissons par la découvrir.Mplc\end{cacheText}

\cacheNumber{743}\needspace{5\baselineskip}\cacheName{\href{http://coord.info/GC6E92P}{PTTTN - M79} — \href{http://coord.info/GC6E92P\Number{}721806413}{743}}\cacheData{{2017/09/16 les cinq, Traditional Cache (1.5/1.5)}}\begin{cacheText}[{FTF}]

Enfin le calme de la campagne... la belle est découverte en deux temps trois mouvements. Merci pour la cache\end{cacheText}

\cacheNumber{744}\needspace{5\baselineskip}\cacheName{\href{http://coord.info/GC6E94K}{PTTTN - M80} — \href{http://coord.info/GC6E94K\Number{}721806212}{744}}\cacheData{{2017/09/16 les cinq, Traditional Cache (1.5/1.5)}}\begin{cacheText}[{FTF}]

Nous démarrons le circuit un peu dans le désordre .Sous l'œil amusé des moutons nous nous évertuons à chercher la cache. Elle est trouvée sans trop de difficulté et grande surprise nous sommes les premiers.Merci pour le travail.\end{cacheText}

\cacheNumber{745}\needspace{5\baselineskip}\cacheName{\href{http://coord.info/GC7BKRC}{PTTTN - N1} — \href{http://coord.info/GC7BKRC\Number{}721451675}{745}}\cacheData{{2017/09/16 les cinq, Traditional Cache (2/1.5)}}\begin{cacheText}[{STF}]

Il nous a fallu du temps pour choisir le circuit que nous allions faire. Finalement nous optons pour le circuit N . Nous localisons très rapidement la cache mais une équipe plus efficace est devant.... Tant qu'à venir à un Évent autant avoir quelques FTF ...nous décidons d'attaquer le circuit par la dernière. En route. ..Merci pour le travail\end{cacheText}

\cacheNumber{746}\needspace{5\baselineskip}\cacheName{\href{http://coord.info/GC7BKRN}{PTTTN - N2} — \href{http://coord.info/GC7BKRN\Number{}721793385}{746}}\cacheData{{2017/09/16 les cinq, Traditional Cache (1.5/1.5)}}\begin{cacheText}[{STF}]

Ici nous rattrapons le temps perdu auparavant. La cache est vite dénichée. Merci\end{cacheText}

\cacheNumber{747}\needspace{5\baselineskip}\cacheName{\href{http://coord.info/GC7BKRX}{PTTTN - N3} — \href{http://coord.info/GC7BKRX\Number{}721793169}{747}}\cacheData{{2017/09/16 les cinq, Traditional Cache (2/3)}}\begin{cacheText}[{STF}]

Arrivés aux coordonnées indiquées et à la lecture de l'indice il nous semblait que la cache serait facile à trouver !!!!Et bien pas du tout!!!Il nous a fallu une demi-heure de recherches à deux pour enfin la dénicher.Oufff. Merci pour la cache\end{cacheText}

\cacheNumber{748}\needspace{5\baselineskip}\cacheName{\href{http://coord.info/GC7BKT7}{PTTTN - N4} — \href{http://coord.info/GC7BKT7\Number{}721792948}{748}}\cacheData{{2017/09/16 les cinq, Traditional Cache (1.5/1.5)}}\begin{cacheText}[{STF}]

Waouh… La cache est super originale. Merci pour ce beau travail.\end{cacheText}

\cacheNumber{749}\needspace{5\baselineskip}\cacheName{\href{http://coord.info/GC7BKTG}{PTTTN - N5} — \href{http://coord.info/GC7BKTG\Number{}721792737}{749}}\cacheData{{2017/09/16 les cinq, Traditional Cache (1.5/1.5)}}\begin{cacheText}[{STF}]

Le chemin n'est pas vraiment propre : les détritus jonchent le sol. Quel dommage !!! Il serait intéressant d'organiser un CITO …ceci dit nous trouvons la cache rapidement. Merci\end{cacheText}

\cacheNumber{750}\needspace{5\baselineskip}\cacheName{\href{http://coord.info/GC7BKTW}{PTTTN - N6} — \href{http://coord.info/GC7BKTW\Number{}721792530}{750}}\cacheData{{2017/09/16 les cinq, Traditional Cache (1.5/1.5)}}\begin{cacheText}[{STF}]

Nous retrouvons la voie routière bruyante...Grâce à l'indice nous délogeons la belle et repartons vers d'autres caches. Merci\end{cacheText}

\cacheNumber{751}\needspace{5\baselineskip}\cacheName{\href{http://coord.info/GC7BKV3}{PTTTN - N7} — \href{http://coord.info/GC7BKV3\Number{}721792000}{751}}\cacheData{{2017/09/16 les cinq, Traditional Cache (1.5/1.5)}}\begin{cacheText}[{STF}]

Nous arrivons sur une zone un peu glauque:un scooter semble plus ou moins camouflé contre le transformateur électrique. .. c'est un peu suspect!!!Nous nous concentrons sur la recherche de la boîte et nous finissons par mettre la main dessus. Bonne cache. Merci\end{cacheText}

\cacheNumber{752}\needspace{5\baselineskip}\cacheName{\href{http://coord.info/GC7BKV7}{PTTTN - N8} — \href{http://coord.info/GC7BKV7\Number{}721627321}{752}}\cacheData{{2017/09/16 les cinq, Traditional Cache (1.5/1.5)}}\begin{cacheText}(STF)

Au niveau du rond-point, il y a beaucoup de passage. Nous devons rester discret mais nous arrivons à trouver la cache. Merci\end{cacheText}

\cacheNumber{753}\needspace{5\baselineskip}\cacheName{\href{http://coord.info/GC7BKVD}{PTTTN - N9} — \href{http://coord.info/GC7BKVD\Number{}721627103}{753}}\cacheData{{2017/09/16 les cinq, Traditional Cache (1.5/1.5)}}\begin{cacheText}(STF)

Ici, nous trouvons une jolie surprise : une boîte à bidules comme je les aime. J'y dépose d'ailleurs une fève en souvenir. Merci pour la cache.\end{cacheText}

\cacheNumber{754}\needspace{5\baselineskip}\cacheName{\href{http://coord.info/GC7BKVK}{PTTTN - N10} — \href{http://coord.info/GC7BKVK\Number{}721626808}{754}}\cacheData{{2017/09/16 les cinq, Traditional Cache (1.5/1.5)}}\begin{cacheText}[{STF}]

Nous essayons d'accélérer le pas : le ciel se charge et les nuages deviennent menaçants. Heureusement nous trouvons vite fait la cache. Merci\end{cacheText}

\cacheNumber{755}\needspace{5\baselineskip}\cacheName{\href{http://coord.info/GC7BKVT}{PTTTN - N11} — \href{http://coord.info/GC7BKVT\Number{}721626189}{755}}\cacheData{{2017/09/16 les cinq, Traditional Cache (1.5/1.5)}}\begin{cacheText}(STF)

L'endroit est très bruyant à cause de la route à proximité. Cela ne nous empêche pas de débusquer la belle. Nous repartons vers la suivante. Merci\end{cacheText}

\cacheNumber{756}\needspace{5\baselineskip}\cacheName{\href{http://coord.info/GC7BKVZ}{PTTTN - N12} — \href{http://coord.info/GC7BKVZ\Number{}721625742}{756}}\cacheData{{2017/09/16 les cinq, Traditional Cache (1.5/1.5)}}\begin{cacheText}(STF)

La cache ,ici ,nous attend bien sagement dans le V. Merci.\end{cacheText}

\cacheNumber{757}\needspace{5\baselineskip}\cacheName{\href{http://coord.info/GC7BKWA}{PTTTN - N13} — \href{http://coord.info/GC7BKWA\Number{}721467478}{757}}\cacheData{{2017/09/16 les cinq, Traditional Cache (1.5/1.5)}}\begin{cacheText}[{STF}]

Les caches se suivent et les points jaunes aussi sur la carte. L' indice et le spoiler nous aident bien. Merci pour le travail.\end{cacheText}

\cacheNumber{758}\needspace{5\baselineskip}\cacheName{\href{http://coord.info/GC7BKWE}{PTTTN - N14} — \href{http://coord.info/GC7BKWE\Number{}721465533}{758}}\cacheData{{2017/09/16 les cinq, Traditional Cache (1.5/1.5)}}\begin{cacheText}[{STF}]

Une classique du genre que nous délogeons facilement. Ici, l'endroit est très bruyant. Nous repartons .Merci\end{cacheText}

\cacheNumber{759}\needspace{5\baselineskip}\cacheName{\href{http://coord.info/GC7BKWH}{PTTTN - N15} — \href{http://coord.info/GC7BKWH\Number{}721464896}{759}}\cacheData{{2017/09/16 les cinq, Traditional Cache (1.5/1.5)}}\begin{cacheText}[{STF}]

Et encore une de trouvée . Merci pour la cache.\end{cacheText}

\cacheNumber{760}\needspace{5\baselineskip}\cacheName{\href{http://coord.info/GC7BKWR}{PTTTN - N16} — \href{http://coord.info/GC7BKWR\Number{}721464448}{760}}\cacheData{{2017/09/16 les cinq, Traditional Cache (1.5/1.5)}}\begin{cacheText}[{STF}]

Ici aussi l'indice est d'une aide précieuse. Nous débusquons la cache rapidement . Merci\end{cacheText}

\cacheNumber{761}\needspace{5\baselineskip}\cacheName{\href{http://coord.info/GC7BKWW}{PTTTN - N17} — \href{http://coord.info/GC7BKWW\Number{}721462498}{761}}\cacheData{{2017/09/16 les cinq, Traditional Cache (1.5/1.5)}}\begin{cacheText}[{ STF}]

En chemin nous avons croisé une autre team. Ici ,nous découvrons irapidement la cache. MPLC\end{cacheText}

\cacheNumber{762}\needspace{5\baselineskip}\cacheName{\href{http://coord.info/GC7BKX3}{PTTTN - N18} — \href{http://coord.info/GC7BKX3\Number{}721461212}{762}}\cacheData{{2017/09/16 les cinq, Traditional Cache (1.5/1.5)}}\begin{cacheText}[{FTF}]

Arrivés sur les lieux ,le chien des voisins n'arrête pas d'aboyer . Pas vraiment discret!!!! Nous arrivons à loguer tant bien que mal. MPLC\end{cacheText}

\cacheNumber{763}\needspace{5\baselineskip}\cacheName{\href{http://coord.info/GC7BKX8}{PTTTN - N19} — \href{http://coord.info/GC7BKX8\Number{}721459412}{763}}\cacheData{{2017/09/16 les cinq, Traditional Cache (1.5/1.5)}}\begin{cacheText}[{FTF}]

Arrivés sur les lieux, le spoiler nous guide bien . La cache est découverte discrètement car le voisinage observe. Encore un bon camouflage. Merci\end{cacheText}

\cacheNumber{764}\needspace{5\baselineskip}\cacheName{\href{http://coord.info/GC7BKXK}{PTTTN - N20} — \href{http://coord.info/GC7BKXK\Number{}721476338}{764}}\cacheData{{2017/09/16 les cinq, Traditional Cache (1.5/1.5)}}\begin{cacheText}(FTF)

Perdus au milieu du champ, Le GPS nous promène d'un endroit à l'autre. La photo de la barrière et des ronces nous permet de localiser la cache qui est bien camouflée . Merci.\end{cacheText}

\cacheNumber{765}\needspace{5\baselineskip}\cacheName{\href{http://coord.info/GC7BKXV}{PTTTN - N21} — \href{http://coord.info/GC7BKXV\Number{}721457806}{765}}\cacheData{{2017/09/16 les cinq, Traditional Cache (1.5/1.5)}}\begin{cacheText}[{FTF}]

Celle-ci est très bien camouflée .Elle se fond dans le paysage. Merci\end{cacheText}

\cacheNumber{766}\needspace{5\baselineskip}\cacheName{\href{http://coord.info/GC7BKY1}{PTTTN - N22} — \href{http://coord.info/GC7BKY1\Number{}721457309}{766}}\cacheData{{2017/09/16 les cinq, Traditional Cache (1.5/2)}}\begin{cacheText}[{FTF}]

Cette cache nous donne du fil à retordre . Le GPS nous balade d'un côté de la route puis de l'autre. Nous finissons par la découvrir...beau camouflage. Bravo et merci.\end{cacheText}

\cacheNumber{767}\needspace{5\baselineskip}\cacheName{\href{http://coord.info/GC7BKY8}{PTTTN - N23} — \href{http://coord.info/GC7BKY8\Number{}721456217}{767}}\cacheData{{2017/09/16 les cinq, Traditional Cache (1.5/2)}}\begin{cacheText}[{FTF}]

Sous l'œil étonné des vaches ,nous trouvons la belle sans difficulté. Merci pour la cache.\end{cacheText}

\cacheNumber{768}\needspace{5\baselineskip}\cacheName{\href{http://coord.info/GC7BKYN}{PTTTN - N24} — \href{http://coord.info/GC7BKYN\Number{}721455532}{768}}\cacheData{{2017/09/16 les cinq, Traditional Cache (1.5/2)}}\begin{cacheText}[{FTF}]

Cette cache ne nous pose pas de problème particulier. Nous continuons la route . Merci.\end{cacheText}

\cacheNumber{769}\needspace{5\baselineskip}\cacheName{\href{http://coord.info/GC7BKYR}{PTTTN - N25} — \href{http://coord.info/GC7BKYR\Number{}721454665}{769}}\cacheData{{2017/09/16 les cinq, Traditional Cache (1.5/1.5)}}\begin{cacheText}[{FTF}]

Waouh quelle cache!!!Les cerveaux en ébullition ,nous arrivons à ouvrir le trésor. Cette cache est super ,elle mérite un PF. Merci pour le travail.\end{cacheText}

\cacheNumber{770}\needspace{5\baselineskip}\cacheName{\href{http://coord.info/GC7BKYW}{PTTTN - N26} — \href{http://coord.info/GC7BKYW\Number{}721452931}{770}}\cacheData{{2017/09/16 les cinq, Traditional Cache (1.5/1.5)}}\begin{cacheText}[{FTF}]

Notre premier FTF..... nous avons bien fais d'attaquer le circuit par la fin. La cache est rapidement trouvée,elle est très bien signalée. Merci pour le travail.\end{cacheText}

\cacheNumber{771}\needspace{5\baselineskip}\cacheName{\href{http://coord.info/GC7CBY5}{PTTTN -  NIGHT 2} — \href{http://coord.info/GC7CBY5\Number{}722168761}{771}}\cacheData{{2017/09/16 les cinq, Unknown Cache (2.5/2)}}\begin{cacheText}[{FTF}]

Trouvée avec la Team Évent 

C'est après une extraordinaire journée ponctuée de fabuleuses caches que nous nous retrouvons au bois Joelland pour l'apéro et le pique nique. À la nuit tombée, le troupeau ,équipé de lampes torches et de lampes UV, s'élance , dans la bonne humeur à la recherche de la première cache de la soirée. Après avoir arpenté les bois chargé d'Histoire et relevé les précieux indices nous arrivons au but. Et quel final !!!! Un très grand Merci et un PF évidemment\end{cacheText}

\cacheNumber{772}\needspace{5\baselineskip}\cacheName{\href{http://coord.info/GC7CBYW}{PTTTN - NIGHT 3} — \href{http://coord.info/GC7CBYW\Number{}722169215}{772}}\cacheData{{2017/09/16 ptitloup, Unknown Cache (2.5/3)}}\begin{cacheText}[{FTF}]

Trouvée avec la Team Event.

Nous revoilà partis , toujours dans la bonne humeur, avec les nouvelles consignes ,à la recherche des indices. Les enfants, très efficaces, les trouvent sans difficulté. Après avoir effectué les calculs nous arrivons au point final qui est tout simplement EXTRAORDINAIRE. Quelle idée !!!! Un très grand Merci et un PF évidemment.\end{cacheText}

\cacheNumber{773}\needspace{5\baselineskip}\cacheName{\href{http://coord.info/GC73CZJ}{Entre mer et marais : Acte II} — \href{http://coord.info/GC73CZJ\Number{}720604700}{773}}\cacheData{{2017/09/17 les cinq;ptitloup;lazuli44, Event Cache (1/1)}}\begin{cacheText}Super journée avec un festival de belles caches. Merci à tous pour cette journée\end{cacheText}

\cacheNumber{774}\needspace{5\baselineskip}\cacheName{\href{http://coord.info/GC6CEZB}{PTTTN - M7} — \href{http://coord.info/GC6CEZB\Number{}722190311}{774}}\cacheData{{2017/09/17 les cinq, Traditional Cache (1.5/1.5)}}\begin{cacheText}Encore une bien jolie cache .Une jolie surprise sur ce parcours .Ici nous sommes septième... le circuit est fort fréquenté !!!Merci\end{cacheText}

\cacheNumber{775}\needspace{5\baselineskip}\cacheName{\href{http://coord.info/GC45VN4}{Une nuit magique dans le marais briéron} — \href{http://coord.info/GC45VN4\Number{}722852820}{775}}\cacheData{{2017/09/17 les cinq, Unknown Cache (3/1.5)}}\begin{cacheText}Dernière cache de la journée( en compagnie de Fabilab)que j'attendais avec impatience. Nous n'avons pas été déçu !!! Elle est vraiment magique. On a la banane !! Une nuit magique au marais nous a rendu le sourire !!! Un PF  super mérité\end{cacheText}

\cacheNumber{776}\needspace{5\baselineskip}\cacheName{\href{http://coord.info/GC4Z5NT}{POUR LES PLUS PETITS} — \href{http://coord.info/GC4Z5NT\Number{}722181639}{776}}\cacheData{{2017/09/17 tytoutiti, Traditional Cache (1.5/1.5)}}\begin{cacheText}Venus pour assister à l'Event entre Mer et MARAIS on ne peut pas passer à côté de cette cache sans la trouver.C'est chose faite!!! Merci\end{cacheText}

\cacheNumber{777}\needspace{5\baselineskip}\cacheName{\href{http://coord.info/GC5BBZT}{La ville rouello !} — \href{http://coord.info/GC5BBZT\Number{}722851061}{777}}\cacheData{{2017/09/17 jujudesaintmarc, Traditional Cache (1.5/1.5)}}\begin{cacheText}Participant à l'Event Entre Mer et Marais nous avons choisi avec les Fabilab de nous lancer sur le parcours Q. La cache n'étant pas loin du circuit nous décidons de la chercher. Ayant la même verte c'est tout naturellement que nous découvrons la belle. Quel pot !\end{cacheText}

\cacheNumber{778}\needspace{5\baselineskip}\cacheName{\href{http://coord.info/GC6CEVE}{PTTTN - M1} — \href{http://coord.info/GC6CEVE\Number{}721968595}{778}}\cacheData{{2017/09/17 les cinq, Traditional Cache (1.5/1.5)}}\begin{cacheText}Nous reprenons ce jour ce très beau circuit que nous aimerions compléter avant de partir!!!Nous arrivons au PZ et heureusement que la photo est là pour nous guider .Merci pour la cache\end{cacheText}

\cacheNumber{779}\needspace{5\baselineskip}\cacheName{\href{http://coord.info/GC6CEVY}{PTTTN - M2} — \href{http://coord.info/GC6CEVY\Number{}721969448}{779}}\cacheData{{2017/09/17 les cinq, Traditional Cache (2/1.5)}}\begin{cacheText}Bon arrivés sur le PZ, nous pinallons alors nous nous aidons de la photo et.... enfin découvrons la cache. Belle réalisation et merci\end{cacheText}

\cacheNumber{780}\needspace{5\baselineskip}\cacheName{\href{http://coord.info/GC6CEW4}{PTTTN - M3} — \href{http://coord.info/GC6CEW4\Number{}721972226}{780}}\cacheData{{2017/09/17 les cinq, Traditional Cache (1.5/1.5)}}\begin{cacheText}[{TTF}]

Il y a beaucoup de fréquentation sur ce chemin du bord du lac. Nous attendons un moment calme pour nous faufiler et découvrir la belle. Merci pour la cache.\end{cacheText}

\cacheNumber{781}\needspace{5\baselineskip}\cacheName{\href{http://coord.info/GC6CEW6}{PTTTN - M4} — \href{http://coord.info/GC6CEW6\Number{}721971967}{781}}\cacheData{{2017/09/17 les cinq, Traditional Cache (1.5/1.5)}}\begin{cacheText}[{TTF}]

Nous devons nous faire discret en ce dimanche matin, la zone est envahie de pêcheurs .Nous découvrons en deux temps trois mouvements la cache et regrettons de ne pas avoir pris la petite pince à épiler .Merci pour la cache.\end{cacheText}

\cacheNumber{782}\needspace{5\baselineskip}\cacheName{\href{http://coord.info/GC6CEWK}{PTTTN - M5} — \href{http://coord.info/GC6CEWK\Number{}721185907}{782}}\cacheData{{2017/09/17 les cinq, Letterbox Hybrid (1.5/1.5)}}\begin{cacheText}[{TTF}]

Belle surprise en arrivant: cette cache est super chouette !!!Nous loguons et prenons un objet voyageur. En repartant nous tombons nez à nez avec des cèpes dans le fossé .Que du bonheur....\end{cacheText}

\cacheNumber{783}\needspace{5\baselineskip}\cacheName{\href{http://coord.info/GC6CEZ6}{PTTTN - M6} — \href{http://coord.info/GC6CEZ6\Number{}722182501}{783}}\cacheData{{2017/09/17 les cinq, Traditional Cache (1.5/1.5)}}\begin{cacheText}Nous continuons le chemin et trouvons facilement cette cache. Merci pour le travail\end{cacheText}

\cacheNumber{784}\needspace{5\baselineskip}\cacheName{\href{http://coord.info/GC6CEZJ}{PTTTN - M8} — \href{http://coord.info/GC6CEZJ\Number{}722192096}{784}}\cacheData{{2017/09/17 les cinq, Traditional Cache (1.5/1.5)}}\begin{cacheText}Nous trouvons ici une belle réalisation. Cache sympa. Merci\end{cacheText}

\cacheNumber{785}\needspace{5\baselineskip}\cacheName{\href{http://coord.info/GC6CEZX}{PTTTN - M10} — \href{http://coord.info/GC6CEZX\Number{}722197306}{785}}\cacheData{{2017/09/17 les cinq, Traditional Cache (1.5/1.5)}}\begin{cacheText}Très jolie cache mais le système est cassé !!!! Les prédécesseurs ont dû tirer trop fort!!Dommage le système était super ...merci pour ce beau travail et un PF pour l'ingéniosité.\end{cacheText}

\cacheNumber{786}\needspace{5\baselineskip}\cacheName{\href{http://coord.info/GC6CF01}{PTTTN - M11} — \href{http://coord.info/GC6CF01\Number{}722199234}{786}}\cacheData{{2017/09/17 les cinq, Traditional Cache (1.5/1.5)}}\begin{cacheText}Quelle belle surprise en arrivant !!! Cache qui nécessite de la patience mais nous y arrivons . Vraiment très heureux ,un gros PF . Merci\end{cacheText}

\cacheNumber{787}\needspace{5\baselineskip}\cacheName{\href{http://coord.info/GC6CF05}{PTTTN - M12} — \href{http://coord.info/GC6CF05\Number{}722215018}{787}}\cacheData{{2017/09/17 les cinq, Traditional Cache (2/1.5)}}\begin{cacheText}Nous avons eu du mal à la voir mais nous avons fini par la trouver . Pas évidente!!!!Merci pour le travail.\end{cacheText}

\cacheNumber{788}\needspace{5\baselineskip}\cacheName{\href{http://coord.info/GC6CF0D}{PTTTN - M13} — \href{http://coord.info/GC6CF0D\Number{}722219046}{788}}\cacheData{{2017/09/17 les cinq, Traditional Cache (1.5/1.5)}}\begin{cacheText}Cache bien intégrée dans le paysage ,nous trouvons assez facilement et il n'y a pas de voisins pour nous déranger!!!Merci\end{cacheText}

\cacheNumber{789}\needspace{5\baselineskip}\cacheName{\href{http://coord.info/GC6CF0J}{PTTTN - M14} — \href{http://coord.info/GC6CF0J\Number{}722219975}{789}}\cacheData{{2017/09/17 les cinq, Traditional Cache (2/1.5)}}\begin{cacheText}On ne comprend l'indice qu'en arrivant sur le lieu . Encore une belle réalisation . Merci\end{cacheText}

\cacheNumber{790}\needspace{5\baselineskip}\cacheName{\href{http://coord.info/GC6CF0R}{PTTTN - M15} — \href{http://coord.info/GC6CF0R\Number{}722222775}{790}}\cacheData{{2017/09/17 les cinq, Traditional Cache (1.5/1.5)}}\begin{cacheText}Une petite bien intégrée au milieu des fourrés que nous finissons par dénicher. Merci 😊\end{cacheText}

\cacheNumber{791}\needspace{5\baselineskip}\cacheName{\href{http://coord.info/GC6CF0W}{PTTTN - M16} — \href{http://coord.info/GC6CF0W\Number{}722225431}{791}}\cacheData{{2017/09/17 les cinq, Traditional Cache (1.5/1.5)}}\begin{cacheText}L'indice est précieux : il nous mène à la cache. Merci\end{cacheText}

\cacheNumber{792}\needspace{5\baselineskip}\cacheName{\href{http://coord.info/GC6CF10}{PTTTN - M17} — \href{http://coord.info/GC6CF10\Number{}722229313}{792}}\cacheData{{2017/09/17 les cinq, Traditional Cache (1.5/1.5)}}\begin{cacheText}Et hop une de plus ! Une jolie qui fait du bruit!!!! Merci pour la cache, on continue.\end{cacheText}

\cacheNumber{793}\needspace{5\baselineskip}\cacheName{\href{http://coord.info/GC6CF13}{PTTTN - M18} — \href{http://coord.info/GC6CF13\Number{}722233394}{793}}\cacheData{{2017/09/17 les cinq, Traditional Cache (2/1.5)}}\begin{cacheText}Arrivés au PZ la faim se fait sentir. Cela tombe bien, il y a des pommes 🍎 partout.C'est l'embarras du choix. Merci pour la cache . Un PF car cette cache se fond dans le décor.\end{cacheText}

\cacheNumber{794}\needspace{5\baselineskip}\cacheName{\href{http://coord.info/GC6CF19}{PTTTN - M19} — \href{http://coord.info/GC6CF19\Number{}722234946}{794}}\cacheData{{2017/09/17 les cinq, Traditional Cache (2.5/2)}}\begin{cacheText}Cette cache nous a donné du fil à retordre. Mais grâce a la photo, nous arrivons à mettre la main dessus. Elle est parfaitement intégrée.Bravo et merci.\end{cacheText}

\cacheNumber{795}\needspace{5\baselineskip}\cacheName{\href{http://coord.info/GC6CF1C}{PTTTN - M20} — \href{http://coord.info/GC6CF1C\Number{}722236561}{795}}\cacheData{{2017/09/17 les cinq, Traditional Cache (2/2.5)}}\begin{cacheText}Celle-là n'est pas évidente non plus!!!Heureusement qu'il y a des traces ..nous suivons le chemin au milieu des ronces.  Dans un an ,elle sera très compliquée!!!Merci 😊\end{cacheText}

\cacheNumber{796}\needspace{5\baselineskip}\cacheName{\href{http://coord.info/GC6CF1H}{PTTTN - M21} — \href{http://coord.info/GC6CF1H\Number{}722238822}{796}}\cacheData{{2017/09/17 les cinq, Multi-cache (2/2)}}\begin{cacheText}Encore une cache super sympa et originale.  L'owner a visé juste!!! Il fallait y penser… Un PF supplémentaire  ( ils descendent à vu d'œil !!!). Merci pour la cache.\end{cacheText}

\cacheNumber{797}\needspace{5\baselineskip}\cacheName{\href{http://coord.info/GC6CF25}{PTTTN - M22} — \href{http://coord.info/GC6CF25\Number{}722241565}{797}}\cacheData{{2017/09/17 les cinq, Traditional Cache (1.5/1.5)}}\begin{cacheText}Waouh encore une cache super originale que nous trouvons sans difficulté grâce à l'indice. Merci pour ce travail. Un PF\end{cacheText}

\cacheNumber{798}\needspace{5\baselineskip}\cacheName{\href{http://coord.info/GC6CF29}{PTTTN - M23} — \href{http://coord.info/GC6CF29\Number{}722244209}{798}}\cacheData{{2017/09/17 les cinq, Traditional Cache (1.5/1.5)}}\begin{cacheText}Ici aussi nous avons eu un peu de mal . Après avoir fait  X fois le tour de l'arbre et des arbustes environnant nous prenons un peu de recul et BINGO elle nous saute aux yeux, L'idée est très originale , je dirais même brillante. Encore bravo  et un PF.  Merci\end{cacheText}

\cacheNumber{799}\needspace{5\baselineskip}\cacheName{\href{http://coord.info/GC6CF4D}{PTTTN - M24} — \href{http://coord.info/GC6CF4D\Number{}722246894}{799}}\cacheData{{2017/09/17 les cinq, Traditional Cache (1.5/1.5)}}\begin{cacheText}Encore une très bonne idée, nous la trouvons ce coup ci assez facilement. Merci\end{cacheText}

\cacheNumber{800}\needspace{5\baselineskip}\cacheName{\href{http://coord.info/GC6CGA9}{PTTTN - M26} — \href{http://coord.info/GC6CGA9\Number{}722255331}{800}}\cacheData{{2017/09/17 les cinq, Traditional Cache (1.5/1.5)}}\begin{cacheText}Ici nous trouvons la cache sans difficulté. Nous gagnons un peu de temps pour avancer sur le circuit.Merci\end{cacheText}

\cacheNumber{801}\needspace{5\baselineskip}\cacheName{\href{http://coord.info/GC6CGAA}{PTTTN - M27} — \href{http://coord.info/GC6CGAA\Number{}722257118}{801}}\cacheData{{2017/09/17 les cinq, Traditional Cache (1.5/1.5)}}\begin{cacheText}Encore une excellente cache. Nous avons insisté autour des pierres mauvaise idée. Enfin nous mettons la main dessus très bon camouflage un PF. Merci\end{cacheText}

\cacheNumber{802}\needspace{5\baselineskip}\cacheName{\href{http://coord.info/GC6CGAB}{PTTTN - M28} — \href{http://coord.info/GC6CGAB\Number{}722707045}{802}}\cacheData{{2017/09/17 les cinq, Traditional Cache (1.5/1.5)}}\begin{cacheText}Encore un très bon camouflage!!!!Nous repérons la cache mais un moldu , certainement le gardien de l'entreprise,nous interpelle pour nous demander,en s'excusant ,ce que l'on fait car il voit défiler pas mal de personnes dans cette impasse.Nous le rassurons et lui expliquons le principe du jeu.. Il semble intéressé... nous loguons et repartons vers d'autres caches.... Merci 😊\end{cacheText}

\cacheNumber{803}\needspace{5\baselineskip}\cacheName{\href{http://coord.info/GC6CGAF}{PTTTN - M29} — \href{http://coord.info/GC6CGAF\Number{}722707103}{803}}\cacheData{{2017/09/17 les cinq, Traditional Cache (2/1.5)}}\begin{cacheText}Pas facile, facile celle-là !!!! Le GPS nous promène et nous cherchons à 1m60. Nous finissons par mettre la main dessus : bien camouflée à plus d'1m60!!!! Merci 😊\end{cacheText}

\cacheNumber{804}\needspace{5\baselineskip}\cacheName{\href{http://coord.info/GC6CGAH}{PTTTN - M30} — \href{http://coord.info/GC6CGAH\Number{}722707226}{804}}\cacheData{{2017/09/17 les cinq, Traditional Cache (1.5/1.5)}}\begin{cacheText}Nous avons trouvé du bois..dans du bois!!!!Merci pour la cache\end{cacheText}

\cacheNumber{805}\needspace{5\baselineskip}\cacheName{\href{http://coord.info/GC6CGAJ}{PTTTN - M31} — \href{http://coord.info/GC6CGAJ\Number{}722707523}{805}}\cacheData{{2017/09/17 les cinq, Letterbox Hybrid (1.5/1.5)}}\begin{cacheText}Encore une très belle surprise. ..du beau travail!!!!Nous recuperons un objet voyageur et je m4en veux d'avoir oublié mes cartes postales!!!! Merci pour tout ce plaisir et un PF de plus.\end{cacheText}

\cacheNumber{806}\needspace{5\baselineskip}\cacheName{\href{http://coord.info/GC6CGAR}{PTTTN - M32} — \href{http://coord.info/GC6CGAR\Number{}722707920}{806}}\cacheData{{2017/09/17 les cinq, Traditional Cache (1.5/1.5)}}\begin{cacheText}C'est une très bonne idée,la cache  fait partie du paysage!!! Merci pour le travail\end{cacheText}

\cacheNumber{807}\needspace{5\baselineskip}\cacheName{\href{http://coord.info/GC6CGAT}{PTTTN - M33} — \href{http://coord.info/GC6CGAT\Number{}722708236}{807}}\cacheData{{2017/09/17 les cinq, Traditional Cache (1.5/1.5)}}\begin{cacheText}Toujours dans le calme de la campagne, nous délogeons tranquillement la belle. Merci pour la cache.\end{cacheText}

\cacheNumber{808}\needspace{5\baselineskip}\cacheName{\href{http://coord.info/GC6CGAW}{PTTTN - M34} — \href{http://coord.info/GC6CGAW\Number{}722708551}{808}}\cacheData{{2017/09/17 les cinq, Traditional Cache (1.5/1.5)}}\begin{cacheText}Ici la photo nous guide bien. Elle nous aide à trouver la cache. Merci\end{cacheText}

\cacheNumber{809}\needspace{5\baselineskip}\cacheName{\href{http://coord.info/GC6CGAZ}{PTTTN - M35} — \href{http://coord.info/GC6CGAZ\Number{}722708863}{809}}\cacheData{{2017/09/17 les cinq, Traditional Cache (2/1.5)}}\begin{cacheText}Waouh j'adore... La cache est rapidement trouvée mais sortir le logbook est une autre histoire. Avec de la patience, nous y arrivons merci. Un PF évidemment\end{cacheText}

\cacheNumber{810}\needspace{5\baselineskip}\cacheName{\href{http://coord.info/GC6CGB1}{PTTTN - M36} — \href{http://coord.info/GC6CGB1\Number{}722709191}{810}}\cacheData{{2017/09/17 les cinq, Traditional Cache (2/2)}}\begin{cacheText}Aïe aïe aïe nous sommes un peu petit... mais trouvons le moyen d'attraper la belle.Encore une belle réalisation:il y a de l'idée.... Merci pour la cache et un PF\end{cacheText}

\cacheNumber{811}\needspace{5\baselineskip}\cacheName{\href{http://coord.info/GC6CGB3}{PTTTN - M37} — \href{http://coord.info/GC6CGB3\Number{}722709543}{811}}\cacheData{{2017/09/17 les cinq, Traditional Cache (1.5/1.5)}}\begin{cacheText}Encore une cache qui est parfaitement intégrée. Nous finissons par mettre la main dessus. Merci pour le travail.\end{cacheText}

\cacheNumber{812}\needspace{5\baselineskip}\cacheName{\href{http://coord.info/GC6CGB5}{PTTTN - M39} — \href{http://coord.info/GC6CGB5\Number{}722710501}{812}}\cacheData{{2017/09/17 les cinq, Traditional Cache (1.5/1.5)}}\begin{cacheText}Voici une cache bien rigolote!!!!C' est super chouette. Merci pour le travail et un PF en plus. Nous poursuivons...\end{cacheText}

\cacheNumber{813}\needspace{5\baselineskip}\cacheName{\href{http://coord.info/GC6CGB9}{PTTTN - M40} — \href{http://coord.info/GC6CGB9\Number{}722710835}{813}}\cacheData{{2017/09/17 les cinq, Traditional Cache (1.5/2)}}\begin{cacheText}Belle surprise, encore du bel ouvrage, du super bricolage. Un grand merci et un PF\end{cacheText}

\cacheNumber{814}\needspace{5\baselineskip}\cacheName{\href{http://coord.info/GC6CGBC}{PTTTN - M41} — \href{http://coord.info/GC6CGBC\Number{}722711194}{814}}\cacheData{{2017/09/17 les cinq, Traditional Cache (2/1.5)}}\begin{cacheText}Magnifique réalisation ...coup de chance extraordinaire qui nous permet d'ouvrir!!! Merci et un PF évidemment. Il est temps pour nous d'arrêter le parcours car nous avons rendez vous avec les Fabilab pour faire quelques caches ensemble.\end{cacheText}

\cacheNumber{815}\needspace{5\baselineskip}\cacheName{\href{http://coord.info/GC771JD}{PTTTN - Q4} — \href{http://coord.info/GC771JD\Number{}722742687}{815}}\cacheData{{2017/09/17 les cinq, Traditional Cache (1.5/1.5)}}\begin{cacheText}L'indice nous guide pour savoir ce qu'il faut chercher . Cache vite repérée. Merci pour le travail.\end{cacheText}

\cacheNumber{816}\needspace{5\baselineskip}\cacheName{\href{http://coord.info/GC771JG}{PTTTN - Q7} — \href{http://coord.info/GC771JG\Number{}722839394}{816}}\cacheData{{2017/09/17 les cinq, Traditional Cache (1.5/1.5)}}\begin{cacheText}Nous avons regardé à 2 fois avant de la trouver!!!! Pas évidente à repérer mais nous y sommes arrivé!!!!Merci pour la cache.\end{cacheText}

\cacheNumber{817}\needspace{5\baselineskip}\cacheName{\href{http://coord.info/GC771JK}{PTTTN - Q8} — \href{http://coord.info/GC771JK\Number{}722840159}{817}}\cacheData{{2017/09/17 les cinq, Traditional Cache (1.5/1.5)}}\begin{cacheText}Toujours en compagnie de Fabilab, nous trouvons sans difficulté la Belle. À la suivante... Merci pour la cache.\end{cacheText}

\cacheNumber{818}\needspace{5\baselineskip}\cacheName{\href{http://coord.info/GC771JQ}{PTTTN - Q11} — \href{http://coord.info/GC771JQ\Number{}722851909}{818}}\cacheData{{2017/09/17 les cinq, Traditional Cache (1.5/1.5)}}\begin{cacheText}Arrivés sur le PZ, nous ne tardons pas pour débusquer la cache . Merci 😊\end{cacheText}

\cacheNumber{819}\needspace{5\baselineskip}\cacheName{\href{http://coord.info/GC771K0}{PTTTN - Q13} — \href{http://coord.info/GC771K0\Number{}722852369}{819}}\cacheData{{2017/09/17 les cinq, Traditional Cache (1.5/1.5)}}\begin{cacheText}Nous terminons le circuit en beauté en trouvant cette cache. Merci pour ce très joli parcours\end{cacheText}

\cacheNumber{820}\needspace{5\baselineskip}\cacheName{\href{http://coord.info/GC77B6B}{PTTTN - Q3} — \href{http://coord.info/GC77B6B\Number{}722742339}{820}}\cacheData{{2017/09/17 les cinq, Traditional Cache (1.5/1.5)}}\begin{cacheText}La cache est bien intégrée dans le paysage… Mais elle ne nous résiste pas très longtemps. Merci pour le travail.\end{cacheText}

\cacheNumber{821}\needspace{5\baselineskip}\cacheName{\href{http://coord.info/GC77E37}{PTTTN - Q1} — \href{http://coord.info/GC77E37\Number{}722742032}{821}}\cacheData{{2017/09/17 ptitloup, Traditional Cache (1.5/1.5)}}\begin{cacheText}Nous nous retrouvons avec Fabilab qui n'a pas pu participer à l'Event pour faire un petit circuit ensemble. Nous optons pour le fameux Q qui a tant fait parlé à l' Évent !!!! Cette cache est rapidement découverte. Merci 😊\end{cacheText}

\cacheNumber{822}\needspace{5\baselineskip}\cacheName{\href{http://coord.info/GC77E39}{PTTTN - Q2} — \href{http://coord.info/GC77E39\Number{}722742158}{822}}\cacheData{{2017/09/17 ptitloup, Traditional Cache (1.5/1.5)}}\begin{cacheText}Arrivés sur le lieu du PZ, nous trouvons rapidement la cache : quatre paires d'yeux cela sert!!!!Merci\end{cacheText}

\cacheNumber{823}\needspace{5\baselineskip}\cacheName{\href{http://coord.info/GC77E3B}{PTTTN - Q5} — \href{http://coord.info/GC77E3B\Number{}722837559}{823}}\cacheData{{2017/09/17 ptitloup, Traditional Cache (1.5/1.5)}}\begin{cacheText}Ici, la cache nous attend bien sagement aux pieds. C'est un excellent camouflage . Merci pour la cache.\end{cacheText}

\cacheNumber{824}\needspace{5\baselineskip}\cacheName{\href{http://coord.info/GC77E3D}{PTTTN - Q6} — \href{http://coord.info/GC77E3D\Number{}722837865}{824}}\cacheData{{2017/09/17 ptitloup, Traditional Cache (1.5/1.5)}}\begin{cacheText}Ici nous cherchons une cache magnétique… qui ne l'est pas vraiment ! Nous finissons par mettre la main dessus grâce au côté métallique. Merci pour la cache\end{cacheText}

\cacheNumber{825}\needspace{5\baselineskip}\cacheName{\href{http://coord.info/GC77E3E}{PTTTN - Q9} — \href{http://coord.info/GC77E3E\Number{}722840428}{825}}\cacheData{{2017/09/17 ptitloup, Traditional Cache (2/2)}}\begin{cacheText}Je dirais magnétique... on peut dire que la belle nous a fait chercher!!! Serait ce la fatigue de la journée qui se fait sentir? Merci pour la cache.\end{cacheText}

\cacheNumber{826}\needspace{5\baselineskip}\cacheName{\href{http://coord.info/GC77E3F}{PTTTN - Q10} — \href{http://coord.info/GC77E3F\Number{}722851629}{826}}\cacheData{{2017/09/17 ptitloup, Traditional Cache (1.5/1.5)}}\begin{cacheText}Sous l'œil intrigué d'une promeneuse ( nous sommes assez bruyant) nous inspectons les arbustes environnant. Une fois le champ libre nous loguons et repartons. Merci pour la cache.\end{cacheText}

\cacheNumber{827}\needspace{5\baselineskip}\cacheName{\href{http://coord.info/GC6CGA8}{PTTTN - M25} — \href{http://coord.info/GC6CGA8\Number{}722253501}{827}}\cacheData{{2017/09/17 les cinq, Traditional Cache (1.5/1.5)}}\begin{cacheText}Pour cette cache nous nous sommes fait avoir!!!  En regardant de plus près on fini par la trouver .Merci\end{cacheText}

\cacheNumber{828}\needspace{5\baselineskip}\cacheName{\href{http://coord.info/GC73CZP}{Entre mer et marais : Acte III} — \href{http://coord.info/GC73CZP\Number{}721066468}{828}}\cacheData{{2017/09/18 ptitloup ; les cinq ; lazuli44, Event Cache (1/1)}}\begin{cacheText}Accueil plus que chaleureux, de la bonne humeur et des caches à profusion pour ce super évent . La team Nazarienne nous a gâté avec pas loin de 650 caches. Un travail de titan !!! Un grand MERCI à toute l'équipe. Nous serons obligés de revenir pour terminer les circuits....\end{cacheText}

\cacheNumber{829}\needspace{5\baselineskip}\cacheName{\href{http://coord.info/GC4MTJR}{AutoStop - A83 - La Chateaudrie} — \href{http://coord.info/GC4MTJR\Number{}722871749}{829}}\cacheData{{2017/09/18 FlavignyTeam, Traditional Cache (1.5/1.5)}}\begin{cacheText}Sur le retour de l'Event entre Mer et Marais de Saint André des Eaux,nous nous arrêtons pour la pause pique-nique. Quelle chance il en reste une sur cette aire d'autoroute. Elle est loguée en deux temps trois mouvements. Merci pour la cache\end{cacheText}

\cacheNumber{830}\needspace{5\baselineskip}\cacheName{\href{http://coord.info/GC6CGBJ}{PTTTN - M42} — \href{http://coord.info/GC6CGBJ\Number{}722856376}{830}}\cacheData{{2017/09/18 les cinq, Traditional Cache (1.5/1.5)}}\begin{cacheText}Avant de reprendre la route vers les Pyrénées Atlantiques ,nous décidons de continuer quelques caches sur ce très très beau circuit .Celle-là est encore originale ,une très bonne idée. Merci.\end{cacheText}

\cacheNumber{831}\needspace{5\baselineskip}\cacheName{\href{http://coord.info/GC6CGBK}{PTTTN - M43} — \href{http://coord.info/GC6CGBK\Number{}722859081}{831}}\cacheData{{2017/09/18 les cinq, Letterbox Hybrid (2/1.5)}}\begin{cacheText}Très belle surprise en arrivant sur les lieux :la boîte est superbe. Mais c'est encore un jeu de patience .C'est madame qui s'y colle avec réussite .Merci pour ce splendide travail (nous piquerons l'idée certainement). Un PF évidemment;\end{cacheText}

\cacheNumber{832}\needspace{5\baselineskip}\cacheName{\href{http://coord.info/GC6CGBN}{PTTTN - M44} — \href{http://coord.info/GC6CGBN\Number{}722859972}{832}}\cacheData{{2017/09/18 les cinq, Traditional Cache (1.5/1.5)}}\begin{cacheText}Pas de difficulté particulière pour cette cache:c'est très bien indiqué.Merci\end{cacheText}

\cacheNumber{833}\needspace{5\baselineskip}\cacheName{\href{http://coord.info/GC6CGBT}{PTTTN - M45} — \href{http://coord.info/GC6CGBT\Number{}722861829}{833}}\cacheData{{2017/09/18 les cinq, Letterbox Hybrid (1.5/1.5)}}\begin{cacheText}Quelle belle boîte encore.... vraiment du super bricolage .Je regrette de ne pas avoir ramené de cartes postales( j'ai complètement oublié de renouveler mon petit stock)Merci pour ce beau travail... que du bonheur .Un PF évidemment\end{cacheText}

\cacheNumber{834}\needspace{5\baselineskip}\cacheName{\href{http://coord.info/GC6CGCM}{PTTTN - M46} — \href{http://coord.info/GC6CGCM\Number{}722862446}{834}}\cacheData{{2017/09/18 les cinq, Traditional Cache (1.5/1.5)}}\begin{cacheText}Nous continuons cette très belle balade le long de ce parcours .Nous découvrons la belle sans difficulté sous un beau soleil et au son des oiseaux qui piaillent .Merci pour la cache\end{cacheText}

\cacheNumber{835}\needspace{5\baselineskip}\cacheName{\href{http://coord.info/GC6CGCR}{PTTTN - M47} — \href{http://coord.info/GC6CGCR\Number{}722863115}{835}}\cacheData{{2017/09/18 les cinq, Traditional Cache (2/2)}}\begin{cacheText}Waouh!!Ici le port du casque est obligatoire... on ne sait pas ce qu'il peut vous tomber sur la tête .Vraiment très très sympa, j'adore l'idée. Un PF et merci pour tout ce travail\end{cacheText}

\cacheNumber{836}\needspace{5\baselineskip}\cacheName{\href{http://coord.info/GC6CGDM}{PTTTN - M50} — \href{http://coord.info/GC6CGDM\Number{}722866252}{836}}\cacheData{{2017/09/18 les cinq, Traditional Cache (2/1.5)}}\begin{cacheText}Contrairement à l'autre cache, le terrain est propre donc nous la trouvons en deux temps trois mouvements . Merci pour le travail\end{cacheText}

\cacheNumber{837}\needspace{5\baselineskip}\cacheName{\href{http://coord.info/GC6CGDP}{PTTTN - M51} — \href{http://coord.info/GC6CGDP\Number{}722867223}{837}}\cacheData{{2017/09/18 les cinq, Traditional Cache (1.5/1.5)}}\begin{cacheText}Ici le chemin est moins tranquille, il y a pas mal de passage de voiture. Nous découvrons la cache et repartons. Merci\end{cacheText}

\cacheNumber{838}\needspace{5\baselineskip}\cacheName{\href{http://coord.info/GC6CGDQ}{PTTTN - M52} — \href{http://coord.info/GC6CGDQ\Number{}722867938}{838}}\cacheData{{2017/09/18 les cinq, Traditional Cache (1.5/1.5)}}\begin{cacheText}Ici, nous avons cherché un bon moment et tout d'un coup... l'illumination .Bingo elle est bien là .Merci, une bonne trouvaille .Encore un PF\end{cacheText}

\cacheNumber{839}\needspace{5\baselineskip}\cacheName{\href{http://coord.info/GC6CGDT}{PTTTN - M53} — \href{http://coord.info/GC6CGDT\Number{}722868444}{839}}\cacheData{{2017/09/18 les cinq, Traditional Cache (1.5/1.5)}}\begin{cacheText}Ici rien à signaler. Le spoiler nous guide. Il faut juste attendre que les voitures passent. Merci pour la cache.\end{cacheText}

\cacheNumber{840}\needspace{5\baselineskip}\cacheName{\href{http://coord.info/GC6CGDV}{PTTTN - M54} — \href{http://coord.info/GC6CGDV\Number{}722868779}{840}}\cacheData{{2017/09/18 les cinq, Traditional Cache (1.5/1.5)}}\begin{cacheText}Trouvée  sans difficulté auprès de la jolie maison au toit de chaume. Merci pour la cache et la balade.\end{cacheText}

\cacheNumber{841}\needspace{5\baselineskip}\cacheName{\href{http://coord.info/GC77A57}{PTTTN - M84} — \href{http://coord.info/GC77A57\Number{}722871253}{841}}\cacheData{{2017/09/18 les cinq, Traditional Cache (1.5/1.5)}}\begin{cacheText}Bien cachée mais elle ne nous résiste pas!!! Mais malheureusement l'heure approche nous devons partir. Merci pour ces bons moments et ces belles caches.\end{cacheText}

\cacheNumber{842}\needspace{5\baselineskip}\cacheName{\href{http://coord.info/GC1F3Q8}{lac de la glacière} — \href{http://coord.info/GC1F3Q8\Number{}724229061}{842}}\cacheData{{2017/09/30 gadiax, Traditional Cache (1.5/1.5)}}\begin{cacheText}Cache rapidement trouvée en compagnie de la team❤️rassemblée pour le CITO Love Love Glacière. Merci 😊\end{cacheText}

\cacheNumber{843}\needspace{5\baselineskip}\cacheName{\href{http://coord.info/GC2P2DY}{Les barthes de l'Adour - le parcours botanique} — \href{http://coord.info/GC2P2DY\Number{}724310870}{843}}\cacheData{{2017/09/30 fdcdm, Traditional Cache (1.5/1.5)}}\begin{cacheText}C'est en compagnie de la Team Love Cœur, rassemblée pour l'occasion du CITO Love Love Glacière que nous découvrons cette cache. Elle se trouve sur le parcours de la Night et cela tombe bien car nous commençons à tourner en rond !!!!.Au premier abord elle semblait avoir disparu mais finalement BINGO ...elle est bien à l' abris. Merci pour la cache\end{cacheText}

\cacheNumber{844}\needspace{5\baselineskip}\cacheName{\href{http://coord.info/GC466N1}{La Gazelle -- Dax} — \href{http://coord.info/GC466N1\Number{}723930826}{844}}\cacheData{{2017/09/30 Sod@'s, Traditional Cache (2/1.5)}}\begin{cacheText}En route pour le CITO LOVE LOVE GLACIERE organisé par les Mizaga,nous nous arrêtons pour faire cette cache. Elle nous résiste!!!! Un quart d'heure que nous cherchons mais après avoir lu et relu les logs des prédécesseurs, nous finissons enfin par la déloger. Très bonne cache. Merci\end{cacheText}

\cacheNumber{845}\needspace{5\baselineskip}\cacheName{\href{http://coord.info/GC5CNW9}{[BSD] \Number{}026} — \href{http://coord.info/GC5CNW9\Number{}723927594}{845}}\cacheData{{2017/09/30 GéoLandesTour, Traditional Cache (1.5/1.5)}}\begin{cacheText}En route pour le CITO LOVE LOVE Glacière organisé  ce soir par Mizaga, nous en profitons pour faire quelques caches, notamment avancer la BSD que nous avions commencé il y a quelques temps. Cache trouvée facilement grâce au spoiler. Merci pour la cache.\end{cacheText}

\cacheNumber{846}\needspace{5\baselineskip}\cacheName{\href{http://coord.info/GC5CV2G}{[BSD] \Number{}027} — \href{http://coord.info/GC5CV2G\Number{}723927709}{846}}\cacheData{{2017/09/30 GéoLandesTour, Traditional Cache (1.5/1.5)}}\begin{cacheText}Nous continuons la route et trouvons cette cache facilement. Nous avons de la chance : il n'y a pas de passage en cette journée pluvieuse. Merci pour la cache\end{cacheText}

\cacheNumber{847}\needspace{5\baselineskip}\cacheName{\href{http://coord.info/GC5CV43}{[BSD] \Number{}028} — \href{http://coord.info/GC5CV43\Number{}723928090}{847}}\cacheData{{2017/09/30 GéoLandesTour, Traditional Cache (1.5/1.5)}}\begin{cacheText}Pour cette cache il faut avoir le sens de l'équilibre! Nous l'avons trouvée au milieu du lierre mais il faut prévoir un nouveau logbook car l'autre est plein.Merci pour la cache.\end{cacheText}

\cacheNumber{848}\needspace{5\baselineskip}\cacheName{\href{http://coord.info/GC5CVRD}{[BSD] \Number{}029} — \href{http://coord.info/GC5CVRD\Number{}723928560}{848}}\cacheData{{2017/09/30 GéoLandesTour, Traditional Cache (1.5/1.5)}}\begin{cacheText}La cache est trouvée facilement . La maintenance est nécessaire: il n'y a plus de bouchon et le logbook est dans un sale état!!!!Merci pour la cache\end{cacheText}

\cacheNumber{849}\needspace{5\baselineskip}\cacheName{\href{http://coord.info/GC5PXTA}{Parc Botanique du Sarrat} — \href{http://coord.info/GC5PXTA\Number{}723989413}{849}}\cacheData{{2017/09/30 mizaga, Traditional Cache (1/1)}}\begin{cacheText}En route pour ton CITO LOVE LOVE GLACIERE nous décidons de faire quelques caches en attendant l'heure.Nous devons patienter avant de signer car des promeneurs avec leur chien se baladent.C'est une cache super ingénieuse que nous trouvons là. Merci pour ce bon moment.\end{cacheText}

\cacheNumber{850}\needspace{5\baselineskip}\cacheName{\href{http://coord.info/GC5Y1BT}{LK collège Danielle  Mitterrand 01} — \href{http://coord.info/GC5Y1BT\Number{}723993712}{850}}\cacheData{{2017/09/30 lauki3940, Traditional Cache (1.5/1.5)}}\begin{cacheText}Dernière avant le CITO....En ce samedi pluvieux il n'y a personne à l'horizon. Le GPS nous promène et nous finissons par repérer le lieu où se trouve la cache. Cache signée et logbook signé.,nous repartons .Merci pour la cache\end{cacheText}

\cacheNumber{851}\needspace{5\baselineskip}\cacheName{\href{http://coord.info/GC65EXM}{Horloge de oeyreluy} — \href{http://coord.info/GC65EXM\Number{}723929103}{851}}\cacheData{{2017/09/30 lauki3940, Traditional Cache (1.5/1.5)}}\begin{cacheText}Quelle belle horloge ! Nous mettons un peu de temps pour trouver la belle mais grâce aux indices laissés par les prédécesseurs dans les logs nous finissons par mettre la main dessus. La persévérance finit toujours par payer!!! Bonne cache ,pas facile à trouver. Merci\end{cacheText}

\cacheNumber{852}\needspace{5\baselineskip}\cacheName{\href{http://coord.info/GC66NF0}{Une pour Mizaga} — \href{http://coord.info/GC66NF0\Number{}723932524}{852}}\cacheData{{2017/09/30 dorisbear, Traditional Cache (2.5/2)}}\begin{cacheText}En route pour le CITO LOVE LOVE GLACIERE organisé par les Mizaga, nous ne pouvons pas faire sans s'arrêter sur cette cache qui leur rend hommage!!!Nous avons d'abord hésité à nous engager sur ce chemin privé mais la tentation était trop grande!!!Arrivé au PZ nous découvrons rapidement la cache et nous ne sommes vraiment pas déçus.....encore du super travail!!! Merci\end{cacheText}

\cacheNumber{853}\needspace{5\baselineskip}\cacheName{\href{http://coord.info/GC6RCKE}{05-Love, Love Glacière- point info} — \href{http://coord.info/GC6RCKE\Number{}735098842}{853}}\cacheData{{2017/09/30 mizaga, Earthcache (1.5/1)}}\begin{cacheText}Co (FTF)

C'est lors de l'Event Love Love Glacière que nous avons découvert le phénomène du Puits Artésien. Merci pour ce bon moment de connaissances.\end{cacheText}

\cacheNumber{854}\needspace{5\baselineskip}\cacheName{\href{http://coord.info/GC71JEZ}{Les jardins familiaux \Number{}4} — \href{http://coord.info/GC71JEZ\Number{}723987961}{854}}\cacheData{{2017/09/30 Lolo2d@x, Traditional Cache (1.5/1.5)}}\begin{cacheText}En route pour le CITO LOVE LOVE GLACIERE organisé par les Mizaga ,nous décidons de faire quelques caches en attendant l'heure. L' endroit est super sympa et la cache très bien intégrée...nous avons cherché un bon moment!!!!Super cache,merci.\end{cacheText}

\cacheNumber{855}\needspace{5\baselineskip}\cacheName{\href{http://coord.info/GC750TH}{[GTAQ 26] - Cache cache avec Lisa [NC]} — \href{http://coord.info/GC750TH\Number{}724310104}{855}}\cacheData{{2017/09/30 tichivi, Unknown Cache (3/2)}}\begin{cacheText}Trouvée avec la Team ❤️composée lors du CITO Love Love Glacière .C'est en cortège que nous arrivons sur les coordonnées indiquées . Après avoir relevé 9 indices sur 12 nous ne retrouvons plus la suite des agrès ... frustrant !!!!! Mais heureusement Isa qui a déjà fait la cache lors du Gtaq nous mène à destination. Merci pour ce beau travail et un PF\end{cacheText}

\cacheNumber{856}\needspace{5\baselineskip}\cacheName{\href{http://coord.info/GC754VD}{[GTAQ 26] Le tueur au canard [NC]} — \href{http://coord.info/GC754VD\Number{}724310956}{856}}\cacheData{{2017/09/30 DorisBear, Unknown Cache (2/2.5)}}\begin{cacheText}C'est avec la Team ❤️ et toujours dans une super ambiance que nous découvrons cette superbe cache de nuit.... mais pas sans difficulté!!!! Merci pour tout ce travail et un PF\end{cacheText}

\cacheNumber{857}\needspace{5\baselineskip}\cacheName{\href{http://coord.info/GC77NED}{Aire de jeux à Oeyreluy.} — \href{http://coord.info/GC77NED\Number{}723928749}{857}}\cacheData{{2017/09/30 lauki3940, Traditional Cache (1.5/1.5)}}\begin{cacheText}En route pour le CITO LOVE LOVE GLACIERE organisé par les Mizaga nous décidons de faire quelques caches sur le pays Daquois.Nous arrivons sur le PZ et repérons facilement l'endroit où se trouve la cache. Seul bémol : un rassemblement de chasseurs se trouve à proximité ! Il nous faut être discret pour tirer la ficelle. Nous y arrivons!!! Merci pour la cache.\end{cacheText}

\cacheNumber{858}\needspace{5\baselineskip}\cacheName{\href{http://coord.info/GC78213}{Observatoire de Dax} — \href{http://coord.info/GC78213\Number{}723991720}{858}}\cacheData{{2017/09/30 lauki3940, Traditional Cache (1.5/1.5)}}\begin{cacheText}Nous avons encore un peu de temps avant l'heure du CITO....Nous trouvons la cache sans difficulté ....le plus dur est de rester discret face à la circulation. Bel observatoire ...merci pour la découverte.\end{cacheText}

\cacheNumber{859}\needspace{5\baselineskip}\cacheName{\href{http://coord.info/GC7CA1H}{Love, Love Glacière CITO III} — \href{http://coord.info/GC7CA1H\Number{}724223291}{859}}\cacheData{{2017/09/30 mizaga, Cache In Trash Out Event (1/1.5)}}\begin{cacheText}Ce n'était pas gagné mais le temps était de la partie en ce samedi après-midi...oufff. Cela nous a permis, dans la bonne humeur, de ramasser les quelques déchets qui jonchaient le tour de ce joli lac tout en dénichant des caches super sympa ,mises en place par les Mizaga. Après le méga pique nique ou les plats délicieux ont été partagés par tous , nous avons assisté à la démonstration du patator créé par Crispol40 qui a laissé les spectateurs sans voix!!! Et pour clore cette excellente journée une partie de la troupe est partie faire les 2 night caches choisies par les Mizaga. Merci à tous pour votre accueil et particulièrement aux Mizaga.\end{cacheText}

\cacheNumber{860}\needspace{5\baselineskip}\cacheName{\href{http://coord.info/GC7CRWX}{01-Love, Love Glacière- une pause} — \href{http://coord.info/GC7CRWX\Number{}724227465}{860}}\cacheData{{2017/09/30 mizaga, Traditional Cache (1.5/1.5)}}\begin{cacheText}[{FTF}]

Trouvée en groupe,Team ❤️, lors du CITO Love Love Glacière.

Dans la bonne humeur nous nous lançons à la recherche de cette cache qui est très bien intégrée. Pas facile à voir.... Bravo les Mizaga ...et un PF\end{cacheText}

\cacheNumber{861}\needspace{5\baselineskip}\cacheName{\href{http://coord.info/GC7CRWY}{02-Love, Love Glacière- coin tranquile} — \href{http://coord.info/GC7CRWY\Number{}724231937}{861}}\cacheData{{2017/09/30 mizaga, Traditional Cache (1.5/1.5)}}\begin{cacheText}[{FTF}]

Encore une cache difficile à déloger mais la team ❤️ est redoutable et finit par mettre la main dessus. Merci pour ce beau bricolage.\end{cacheText}

\cacheNumber{862}\needspace{5\baselineskip}\cacheName{\href{http://coord.info/GC7CRX0}{03-Love, Love Glacière- le ruisseau} — \href{http://coord.info/GC7CRX0\Number{}724238689}{862}}\cacheData{{2017/09/30 mizaga, Traditional Cache (2/2)}}\begin{cacheText}[{FTF}]

Arrivés sur le PZ, la team ❤️finit par découvrir la Belle mais nous n'étions pas vraiment équipé… Sauf Sujiva qui avait prévu les bottes!!! Après réflexion et un peu d'aide ... nous loguons la belle. Merci pour ce beau travail et un PF en plus\end{cacheText}

\cacheNumber{863}\needspace{5\baselineskip}\cacheName{\href{http://coord.info/GC7CRX3}{04-Love, Love Glacière- l'arbre} — \href{http://coord.info/GC7CRX3\Number{}724240675}{863}}\cacheData{{2017/09/30 mizaga, Traditional Cache (1.5/1.5)}}\begin{cacheText}[{FTF}]

Toujours avec la Team ❤️nous découvrons cette belle réalisation après avoir cherché au mauvais endroit !!! Merci et un PF\end{cacheText}

\cacheNumber{864}\needspace{5\baselineskip}\cacheName{\href{http://coord.info/GC7CRXN}{06-Love, Love Glacière- massif forestier} — \href{http://coord.info/GC7CRXN\Number{}724250369}{864}}\cacheData{{2017/09/30 mizaga, Multi-cache (1.5/1.5)}}\begin{cacheText}[{FTF}]

Avec l'aide de tous les membres de l'équipe de la team ❤️ , nous découvrons les indices nécessaires pour déloger la belle. Encore une super cache. Merci les Mizaga et un PF\end{cacheText}

\cacheNumber{865}\needspace{5\baselineskip}\cacheName{\href{http://coord.info/GC5VK3F}{Musée de l'alat} — \href{http://coord.info/GC5VK3F\Number{}723930370}{865}}\cacheData{{2017/09/30 mizaga, Traditional Cache (1.5/1.5)}}\begin{cacheText}En route pour le CITO LOVE LOVE GLACIERE, nous décidons de faire quelques caches sur Dax. C'est en deux temps trois mouvements que nous dénichons la belle. Un coup de chance car elle est très bien camouflée. Merci pour la cache.\end{cacheText}

\cacheNumber{866}\needspace{5\baselineskip}\cacheName{\href{http://coord.info/GC47F0R}{La mare de Pompogne} — \href{http://coord.info/GC47F0R\Number{}724412033}{866}}\cacheData{{2017/10/02 nmns26, Multi-cache (1/1.5)}}\begin{cacheText}Contrairement à mes prédécesseurs, j'ai fini par trouver la cache mais à force de persévérer. En effet , j'ai mal interprété la dernière question qui m'a donné de mauvaises coordonnées !!!Je me suis promenée autour de la mare et soudain j'ai compris mon erreur ...Oufff. Merci pour la découverte de ce lieu extraordinaire\end{cacheText}

\cacheNumber{867}\needspace{5\baselineskip}\cacheName{\href{http://coord.info/GC4VQ0E}{Le clocher au pin} — \href{http://coord.info/GC4VQ0E\Number{}724410659}{867}}\cacheData{{2017/10/02 kar@melos40, Traditional Cache (1.5/1.5)}}\begin{cacheText}Sur la route de Casteljaloux je ne peux m'empêcher de m'arrêter pour faire ces quelques caches sur Pompogne. Celle-ci est vite trouvée et me fait découvrir une superbe église et son pin hors du commun. Une maintenance est nécessaire car le log book est plein, je signe sur le papier jaune qui a été rajouté provisoirement. Merci pour la cache 😃\end{cacheText}

\cacheNumber{868}\needspace{5\baselineskip}\cacheName{\href{http://coord.info/GC76QV4}{Lac de Clarens \Number{}2 : La Plage - Casteljaloux(47)} — \href{http://coord.info/GC76QV4\Number{}724460579}{868}}\cacheData{{2017/10/02 Agex, Traditional Cache (2.5/1.5)}}\begin{cacheText}Arrivée sur le PZ, l'indice prend tout son sens. Toute la difficulté réside dans le fait de l'attraper!!!!Heureusement, j'ai, dans la voiture qui est garée à proximité, une caisse qui va me servir de tabouret. Il n'y a personne dans les parages....ouffff. Merci pour cette micro cache.\end{cacheText}

\cacheNumber{869}\needspace{5\baselineskip}\cacheName{\href{http://coord.info/GC7BQ4E}{VFDQ : Le Lavoir} — \href{http://coord.info/GC7BQ4E\Number{}724467753}{869}}\cacheData{{2017/10/02 Bugs\And{}Co, Traditional Cache (2/1.5)}}\begin{cacheText}( STF)

Il y a longtemps que je voulais faire cette cache… De retour dans le coin pour la semaine j'en profite!!!C'est chose faite mais j'arrive en seconde position. Dommage ! C'est un très joli lavoir et effectivement les bons geocacheurs n'ont pas besoin de tuyaux. Je récupère le TB. Merci pour ce bon moment et la découverte de ce lieux.\end{cacheText}

\cacheNumber{870}\needspace{5\baselineskip}\cacheName{\href{http://coord.info/GC5E1YJ}{Le lavoir de Port Sainte Marie} — \href{http://coord.info/GC5E1YJ\Number{}724678508}{870}}\cacheData{{2017/10/03 janokinou, Traditional Cache (1.5/1.5)}}\begin{cacheText}C'est un lavoir hors du commun, très joli. La cache est trouvée rapidement car elle est visible de loin. Merci pour la découverte de ce lieu.\end{cacheText}

\cacheNumber{871}\needspace{5\baselineskip}\cacheName{\href{http://coord.info/GC5GW00}{L'église de Mazères} — \href{http://coord.info/GC5GW00\Number{}724675344}{871}}\cacheData{{2017/10/03 janokinou, Traditional Cache (1.5/1.5)}}\begin{cacheText}De passage dans le Lot-et-Garonne j'en profite pour faire quelques caches. Celle-ci est très vite délogée . Elle nous fait découvrir une très belle église. Merci pour la cache.\end{cacheText}

\cacheNumber{872}\needspace{5\baselineskip}\cacheName{\href{http://coord.info/GC5KRZR}{TB hôtel du lot et garonne} — \href{http://coord.info/GC5KRZR\Number{}724681102}{872}}\cacheData{{2017/10/03 corentin47450, Traditional Cache (2.5/2.5)}}\begin{cacheText}Il ne m'a pas fallu longtemps pour localiser le PZ mais j'avais beau regarder au-dessus de ma tête… Rien J'allais repartir quand quelque chose a attiré mon attention dans l'herbe. C'etait bien ça ...l'indice qu'il me manquait!!!!Échange de TB. Merci pour la cache. Il est nécessaire de faire la maintenance car je n'ai pas réussi à rattacher l'indice.\end{cacheText}

\cacheNumber{873}\needspace{5\baselineskip}\cacheName{\href{http://coord.info/GC5XPP4}{Clermont Dessous} — \href{http://coord.info/GC5XPP4\Number{}724678888}{873}}\cacheData{{2017/10/03 lidibrini, Traditional Cache (1.5/1.5)}}\begin{cacheText}J'ai passé plus de temps à visiter ce village pittoresque et merveilleux qu'à chercher la cache. Merci pour la découverte de ce village extraordinaire.\end{cacheText}

\cacheNumber{874}\needspace{5\baselineskip}\cacheName{\href{http://coord.info/GC56C5A}{Randonnée Colayrac n°1} — \href{http://coord.info/GC56C5A\Number{}724891523}{874}}\cacheData{{2017/10/04 nikko44, Traditional Cache (2/1)}}\begin{cacheText}Cache trouvée grâce aux photos des prédécesseurs. Mon petit GPS se trompe de 10 m!!!!Merci pour la cache\end{cacheText}

\cacheNumber{875}\needspace{5\baselineskip}\cacheName{\href{http://coord.info/GC56C94}{Randonnée Colayrac n°2} — \href{http://coord.info/GC56C94\Number{}724891269}{875}}\cacheData{{2017/10/04 nikko44, Traditional Cache (2.5/2)}}\begin{cacheText}Trouvée avec beaucoup de difficulté malgré les photos des prédécesseurs.J'ai regardé plusieurs fois au même endroit sans jamais la trouver j'allais renoncer lorsqu'enfin je la découvre!!!!!Merci\end{cacheText}

\cacheNumber{876}\needspace{5\baselineskip}\cacheName{\href{http://coord.info/GC56CB1}{Randonnée Colayrac n°3} — \href{http://coord.info/GC56CB1\Number{}724892323}{876}}\cacheData{{2017/10/04 nikko44, Traditional Cache (1/2.5)}}\begin{cacheText}Ici pas de difficulté, grâce a l'indice la cache est vite délogée. Merci\end{cacheText}

\cacheNumber{877}\needspace{5\baselineskip}\cacheName{\href{http://coord.info/GC56CKE}{Randonnée Colayrac n°11} — \href{http://coord.info/GC56CKE\Number{}724889793}{877}}\cacheData{{2017/10/04 corentin47450, Traditional Cache (2.5/1.5)}}\begin{cacheText}Certes ,l'endroit n'est pas très agréable mais la cache a le mérite d'exister!!!D'autant plus qu'il ne s'agit pas d'un simple tube… Il n'y a pas besoin de chercher dans les arbustes pour la dénicher. Merci pour la cache\end{cacheText}

\cacheNumber{878}\needspace{5\baselineskip}\cacheName{\href{http://coord.info/GC5AM9M}{Cache du camping} — \href{http://coord.info/GC5AM9M\Number{}724890493}{878}}\cacheData{{2017/10/04 corentin47450, Traditional Cache (2.5/1.5)}}\begin{cacheText}J'ai cherché un petit moment avant de mettre la main dessus. C'est une cache vraiment originale,bon recyclage, c'est la première que je vois ce genre. Merci pour la cache\end{cacheText}

\cacheNumber{879}\needspace{5\baselineskip}\cacheName{\href{http://coord.info/GC5AMBF}{Salle des fêtes} — \href{http://coord.info/GC5AMBF\Number{}724892501}{879}}\cacheData{{2017/10/04 corentin47450, Traditional Cache (2.5/2)}}\begin{cacheText}Belle réalisation : la cache est parfaitement intégrée au paysage . Merci pour ce bon moment. Un PF\end{cacheText}

\cacheNumber{880}\needspace{5\baselineskip}\cacheName{\href{http://coord.info/GC5AMC2}{Eglise de cardonnet} — \href{http://coord.info/GC5AMC2\Number{}724902956}{880}}\cacheData{{2017/10/04 corentin47450, Traditional Cache (2/1)}}\begin{cacheText}Pas de soucis particulier pour déloger la belle.. Elle nous fait découvrir encore une magnifique église. Quel riche patrimoine !!! Merci pour la cache 😃\end{cacheText}

\cacheNumber{881}\needspace{5\baselineskip}\cacheName{\href{http://coord.info/GC5AMCE}{Château de Madaillan} — \href{http://coord.info/GC5AMCE\Number{}724903517}{881}}\cacheData{{2017/10/04 corentin47450, Traditional Cache (2/1.5)}}\begin{cacheText}Cache vite délogée.... un classique du genre. On aperçoit au loin le magnifique château féodal de Madaillan.  Merci pour le travai\end{cacheText}

\cacheNumber{882}\needspace{5\baselineskip}\cacheName{\href{http://coord.info/GC5B9TT}{lavoir de lary} — \href{http://coord.info/GC5B9TT\Number{}724905323}{882}}\cacheData{{2017/10/04 corentin47450, Traditional Cache (2.5/2)}}\begin{cacheText}Cette cache n'est pas facile à trouver mais en persévérant j'arrive au résultat. Merci pour la découverte de ce joli lavoir.\end{cacheText}

\cacheNumber{883}\needspace{5\baselineskip}\cacheName{\href{http://coord.info/GC5EYAT}{L'église de Lusignan Grand} — \href{http://coord.info/GC5EYAT\Number{}724886921}{883}}\cacheData{{2017/10/04 janokinou, Traditional Cache (2/1.5)}}\begin{cacheText}Quelle belle église… Quel beau patrimoine ! Je finis par déloger la belle avec beaucoup de difficulté mais un doute persiste je ne suis pas sûre que cela soit la bonne cache. Je suis la premiere à signer sur le log book !\end{cacheText}

\cacheNumber{884}\needspace{5\baselineskip}\cacheName{\href{http://coord.info/GC5G9Z6}{Cache abri bus} — \href{http://coord.info/GC5G9Z6\Number{}724889354}{884}}\cacheData{{2017/10/04 corentin47450, Traditional Cache (2/2)}}\begin{cacheText}Arrivée sur le PZ c'est la tuile… Il me faut attendre 10 bonnes minutes pour pouvoir commencer à chercher car une personne jette des bouteilles dans le containers. Arrive une autre personne qui cherche sa route Saint-Hilaire de Lusignan… N'étant pas du coin je suis obligée de prendre le GPS  pour lui expliquer!!!!Enfin la route est libre, je peux chercher et trouver tranquillement... Merci pour la cache\end{cacheText}

\cacheNumber{885}\needspace{5\baselineskip}\cacheName{\href{http://coord.info/GC5GA0Q}{église de saint hilaire de lusignan} — \href{http://coord.info/GC5GA0Q\Number{}724884580}{885}}\cacheData{{2017/10/04 corentin47450, Traditional Cache (3/1.5)}}\begin{cacheText}La caest est difficile à trouver mais grâce aux indices laissés par les prédécesseurs je finis par mettre la main dessus. Merci pour la découverte de cette très belle église qui aurait besoin d'être rénovée.\end{cacheText}

\cacheNumber{886}\needspace{5\baselineskip}\cacheName{\href{http://coord.info/GC5GX63}{Lusignan Grand : souvenir du passé} — \href{http://coord.info/GC5GX63\Number{}724885739}{886}}\cacheData{{2017/10/04 janokinou, Traditional Cache (2/1.5)}}\begin{cacheText}Voici un très bel ouvrage que je n'ai jamais vu ailleurs. Arrivée sur le site je lis les commentaires et pense à repartir mais finalement la curiosité l'emporte !!Je cherche et bingo je trouve un  logbook de remplacement. Merci pour la cache et pour la maintenance\end{cacheText}

\cacheNumber{887}\needspace{5\baselineskip}\cacheName{\href{http://coord.info/GC5GX6D}{Lusignan Grand : la falaise} — \href{http://coord.info/GC5GX6D\Number{}724888063}{887}}\cacheData{{2017/10/04 janokinou, Traditional Cache (2/2)}}\begin{cacheText}Cette falaise est impressionnante. La cache est rapidement délogée grâce à l'indice et aux coordonnées précises. Merci pour la cache\end{cacheText}

\cacheNumber{888}\needspace{5\baselineskip}\cacheName{\href{http://coord.info/GC5GX95}{Lusignan Grand : au bord de la route} — \href{http://coord.info/GC5GX95\Number{}724888473}{888}}\cacheData{{2017/10/04 janokinou, Traditional Cache (2/2)}}\begin{cacheText}Arrivée sur les lieux, mon instinct me guide tout droit à la cache. Elle est très bien camouflée. Merci\end{cacheText}

\cacheNumber{889}\needspace{5\baselineskip}\cacheName{\href{http://coord.info/GC5H9RN}{Cap Cauderoue} — \href{http://coord.info/GC5H9RN\Number{}724986119}{889}}\cacheData{{2017/10/05 jomatoju 47, Traditional Cache (1.5/1.5)}}\begin{cacheText}La cache est  facilement repérée mais l'attraper est une autre histoire!!!! Heureusement l'endroit n'est pas fréquenté en cette période. Je peux donc me servir de la caisse pour l'attraper :le logbook est trempe,impossible de signer.!!! Ci-joint la photo pour valider. Merci pour la cache\end{cacheText}

\cacheNumber{890}\needspace{5\baselineskip}\cacheName{\href{http://coord.info/GC5HXGW}{L’église de Mazeret} — \href{http://coord.info/GC5HXGW\Number{}725752268}{890}}\cacheData{{2017/10/05 jomatoju 47, Traditional Cache (1.5/1.5)}}\begin{cacheText}Enfin trouvée… Difficile car le GPS me promène, il n'est pas du tout précis. Mais en persévérant je finis par la déloger.Magnifique église perdue dans la campagne et qui  inspire au repos.... Merci 😊\end{cacheText}

\cacheNumber{891}\needspace{5\baselineskip}\cacheName{\href{http://coord.info/GC5JMZE}{HAMEAU DU BEAS} — \href{http://coord.info/GC5JMZE\Number{}724976340}{891}}\cacheData{{2017/10/05 jomatoju 47, Traditional Cache (1.5/1.5)}}\begin{cacheText}En effet, les caches les plus apparentes aux yeux des geocacheurs sont les moins visibles aux yeux des moldus!!!! Merci pour la découverte de cette belle église 😃\end{cacheText}

\cacheNumber{892}\needspace{5\baselineskip}\cacheName{\href{http://coord.info/GC5JN09}{HAMEAU DU BEAS \Quoted{LE LAC}} — \href{http://coord.info/GC5JN09\Number{}724977595}{892}}\cacheData{{2017/10/05 jomatoju 47, Traditional Cache (1.5/1.5)}}\begin{cacheText}Nous voici dans un endroit très paisible. La cache est rapidement trouvée grâce aux indices. Merci\end{cacheText}

\cacheNumber{893}\needspace{5\baselineskip}\cacheName{\href{http://coord.info/GC66G78}{le lavoir de péchet} — \href{http://coord.info/GC66G78\Number{}724988167}{893}}\cacheData{{2017/10/05 lomimaju47, Traditional Cache (1.5/1.5)}}\begin{cacheText}Un joli petit lavoir qui semble assez récent. Le GPS et l'indice me mènent tout droit à la cache. Cache sympa.Ici aussi le log book est dégradé. Je le change. Merci pour la cache.\end{cacheText}

\cacheNumber{894}\needspace{5\baselineskip}\cacheName{\href{http://coord.info/GC6AKGE}{LA SOURCE DE POMPOGNE} — \href{http://coord.info/GC6AKGE\Number{}725747533}{894}}\cacheData{{2017/10/08 MKL33210, Earthcache (1.5/2)}}\begin{cacheText}La multi m'a mené tout droit à la source. Très bel endroit....Merci pour cette découverte.\end{cacheText}

\cacheNumber{895}\needspace{5\baselineskip}\cacheName{\href{http://coord.info/GC3YMFM}{Le jardin des sources} — \href{http://coord.info/GC3YMFM\Number{}726111401}{895}}\cacheData{{2017/10/09 Degreze, Traditional Cache (2/2)}}\begin{cacheText}L'endroit est fort agréable. En ce bel après-midi il n'y a personne… Tant mieux pour nous, nous pouvons déloger la belle sans difficulté. Il n'y a pas d'objet voyageur présent dans la boîte mais j'en dépose un.  Prenez en soin. Merci pour la cache\end{cacheText}

\cacheNumber{896}\needspace{5\baselineskip}\cacheName{\href{http://coord.info/GC7085R}{SAINT PARDOUX \Number{} 01} — \href{http://coord.info/GC7085R\Number{}726101590}{896}}\cacheData{{2017/10/09 PhT47, Traditional Cache (1.5/1.5)}}\begin{cacheText}Cache dénichée au bout d'un certain temps . Nous  n'avons pas tout compris mais nous avons fini par mettre la main dessus. C'est une cache instructive mais je ne comprends pas le lien entre la date à certifier et le lieu de la cache!!!!Merci\end{cacheText}

\cacheNumber{897}\needspace{5\baselineskip}\cacheName{\href{http://coord.info/GC70ACT}{SAINT PARDOUX \Number{} 10} — \href{http://coord.info/GC70ACT\Number{}726108573}{897}}\cacheData{{2017/10/09 PhT47, Traditional Cache (1.5/1.5)}}\begin{cacheText}L'endroit est charmant et calme pour poser une cache. Le système trouvé est ingénieux. Bravo !!!Attention aux piquants.Merci pour la cache.\end{cacheText}

\cacheNumber{898}\needspace{5\baselineskip}\cacheName{\href{http://coord.info/GC70B4G}{SAINT PARDOUX \Number{} 04} — \href{http://coord.info/GC70B4G\Number{}726105601}{898}}\cacheData{{2017/10/09 PhT47, Traditional Cache (1.5/1.5)}}\begin{cacheText}Arrivés sur le PZ nous devons patienter un petit moment pour que des promeneurs s'éloignent. Nous délogeons la belle en deux temps trois mouvements grâce à l'indice . Merci pour la cache\end{cacheText}

\cacheNumber{899}\needspace{5\baselineskip}\cacheName{\href{http://coord.info/GC70H2K}{SAINT PARDOUX \Number{} 05} — \href{http://coord.info/GC70H2K\Number{}726105965}{899}}\cacheData{{2017/10/09 PhT47, Traditional Cache (1.5/1.5)}}\begin{cacheText}La cache est rapidement repérée mais il faut passer toutes les orties!!!C'est madame qui s'y colle et qui trouve la cache. Merci\end{cacheText}

\cacheNumber{900}\needspace{5\baselineskip}\cacheName{\href{http://coord.info/GC70MK2}{SAINT PARDOUX \Number{} 02} — \href{http://coord.info/GC70MK2\Number{}726103853}{900}}\cacheData{{2017/10/09 PhT47, Traditional Cache (1.5/1.5)}}\begin{cacheText}Grâce à l'indice nous trouvons facilement la cache qui est bien camouflée... Merci pour la Belle promenade. Un pont plus que douteux se trouve pas très loin... heureusement qu'il ne faut pas l'emprunter!!!\end{cacheText}

\cacheNumber{901}\needspace{5\baselineskip}\cacheName{\href{http://coord.info/GC70MKH}{SAINT PARDOUX \Number{} 03} — \href{http://coord.info/GC70MKH\Number{}726105651}{901}}\cacheData{{2017/10/09 PhT47, Traditional Cache (1.5/1.5)}}\begin{cacheText}De passage dans le Lot-et-Garonne nous nous attaquons au circuit de Saint Pardoux . Ici un DNF est annoncé. Nous cherchons quand même : effectivement le fils de fer est en place mais en regardant bien parterre nous finissons par trouver le tube. Nous le remettons en place sommairement, une maintenance est nécessaire. Merci pour la cache\end{cacheText}

\cacheNumber{902}\needspace{5\baselineskip}\cacheName{\href{http://coord.info/GC70RRA}{SAINT PARDOUX \Number{} 07} — \href{http://coord.info/GC70RRA\Number{}726106601}{902}}\cacheData{{2017/10/09 PhT47, Traditional Cache (1.5/1.5)}}\begin{cacheText}Cache trouvée sans aucune difficulté .Un classique du genre !!!!Merci\end{cacheText}

\cacheNumber{903}\needspace{5\baselineskip}\cacheName{\href{http://coord.info/GC70RTK}{SAINT PARDOUX \Number{} 08} — \href{http://coord.info/GC70RTK\Number{}726106911}{903}}\cacheData{{2017/10/09 PhT47, Traditional Cache (1.5/1.5)}}\begin{cacheText}Une cache facile… Vite fait bien fait et nous repartons merci pour le travail.\end{cacheText}

\cacheNumber{904}\needspace{5\baselineskip}\cacheName{\href{http://coord.info/GC70RWG}{SAINT PARDOUX \Number{} 09} — \href{http://coord.info/GC70RWG\Number{}726108001}{904}}\cacheData{{2017/10/09 PhT47, Traditional Cache (1.5/1.5)}}\begin{cacheText}Arrivés sur les lieux l'indice prend tout son sens!!!Bonne idée. Merci pour la cache\end{cacheText}

\cacheNumber{905}\needspace{5\baselineskip}\cacheName{\href{http://coord.info/GC70Y1R}{SAINT PARDOUX \Number{} 11} — \href{http://coord.info/GC70Y1R\Number{}726108582}{905}}\cacheData{{2017/10/09 PhT47, Traditional Cache (2/1.5)}}\begin{cacheText}Encore une super cache comme nous les aimons. Du très bon travail merci pour ce bon moment.\end{cacheText}

\cacheNumber{906}\needspace{5\baselineskip}\cacheName{\href{http://coord.info/GC71Z4N}{SAINT PARDOUX \Number{} 12} — \href{http://coord.info/GC71Z4N\Number{}726111390}{906}}\cacheData{{2017/10/09 PhT47, Traditional Cache (1.5/1.5)}}\begin{cacheText}Avec l'aide de l'indice, nous arrivons à trouver la cache malgré quelques difficultés. Cachette très sympa qui mérite un PF. Merci pour la cache.\end{cacheText}

\cacheNumber{907}\needspace{5\baselineskip}\cacheName{\href{http://coord.info/GC71Z5V}{SAINT PARDOUX \Number{} 14} — \href{http://coord.info/GC71Z5V\Number{}726109881}{907}}\cacheData{{2017/10/09 PhT47, Traditional Cache (1.5/1.5)}}\begin{cacheText}Ici l'indice nous aide bien à repérer l'endroit précis de la cache. Pas de difficulté. Merci pour cette jolie balade.\end{cacheText}

\cacheNumber{908}\needspace{5\baselineskip}\cacheName{\href{http://coord.info/GC71Z6F}{SAINT PARDOUX \Number{} 15} — \href{http://coord.info/GC71Z6F\Number{}726109883}{908}}\cacheData{{2017/10/09 PhT47, Traditional Cache (1.5/1.5)}}\begin{cacheText}Arrivés  sur les lieux, la cache nous attend bien à sa place.Merci\end{cacheText}

\cacheNumber{909}\needspace{5\baselineskip}\cacheName{\href{http://coord.info/GC73YB5}{Caliméro - un rejeton de Rosalie} — \href{http://coord.info/GC73YB5\Number{}726167244}{909}}\cacheData{{2017/10/10 fdcdm, Traditional Cache (1.5/1.5)}}\begin{cacheText}[{FTF}]

Ce matin au petit déjeuner, une annonce. Nouvelle cache sur Mont de Marsan de l'ami FDCDM! Cela tombe bien nous rentrons du Lot et Garonne et pouvons nous arrêter en passant!!!! Nous découvrons un joli bois et un très bel élevage à proximité. Le GPS nous guide tout droit à Calimero. Merci pour cette jolie cache\end{cacheText}

\cacheNumber{910}\needspace{5\baselineskip}\cacheName{\href{http://coord.info/GC3F3AE}{GR10\Number{}69} — \href{http://coord.info/GC3F3AE\Number{}728475834}{910}}\cacheData{{2017/10/20 Peyo64, Traditional Cache (2/1.5)}}\begin{cacheText}Ayant finis les caches traditionnelles du GR 8 nous faisons un petit tour sur Sare. Nous nous arrêtons donc à celle du GR 10 que nous trouvons facilement. Merci pour la cache\end{cacheText}

\cacheNumber{911}\needspace{5\baselineskip}\cacheName{\href{http://coord.info/GC3G8ET}{GR8\Number{}127} — \href{http://coord.info/GC3G8ET\Number{}728462080}{911}}\cacheData{{2017/10/20 Peyo64, Traditional Cache (1.5/1.5)}}\begin{cacheText}Ce matin nous décidons de finir le GR… Bonne surprise les Dorisbear ont fait de la maintenance sur certaines caches que nous n'avions pas trouvées!!!C'est chose faite pour celle la!!. Merci pour la cache et la maintenance\end{cacheText}

\cacheNumber{912}\needspace{5\baselineskip}\cacheName{\href{http://coord.info/GC3G8NW}{GR8\Number{}143} — \href{http://coord.info/GC3G8NW\Number{}728464726}{912}}\cacheData{{2017/10/20 Peyo64, Traditional Cache (2/1.5)}}\begin{cacheText}Bien décidés à finir le circuit, nous attaquons motivés par celle ci qui ne nous résiste pas longtemps. Nous signons le logbook de maintenance et repartons...Merci\end{cacheText}

\cacheNumber{913}\needspace{5\baselineskip}\cacheName{\href{http://coord.info/GC3G8PG}{GR8\Number{}144} — \href{http://coord.info/GC3G8PG\Number{}728465241}{913}}\cacheData{{2017/10/20 Peyo64, Traditional Cache (2/1.5)}}\begin{cacheText}Nous trouvons ici la cache qui est parfaitement confondue dans le lierre. Nous continuons la route. Merci\end{cacheText}

\cacheNumber{914}\needspace{5\baselineskip}\cacheName{\href{http://coord.info/GC3G8Q8}{GR8\Number{}145} — \href{http://coord.info/GC3G8Q8\Number{}728465888}{914}}\cacheData{{2017/10/20 Peyo64, Traditional Cache (2/1.5)}}\begin{cacheText}La cache est trouvée sans difficulté. Elle est bien apparente. Merci pour la cache et sa maintenance. La vue est superbe,bravo c'est magnifique!!!\end{cacheText}

\cacheNumber{915}\needspace{5\baselineskip}\cacheName{\href{http://coord.info/GC3G8QY}{GR8\Number{}146} — \href{http://coord.info/GC3G8QY\Number{}728466069}{915}}\cacheData{{2017/10/20 Peyo64, Traditional Cache (1.5/1.5)}}\begin{cacheText}Après de longues recherches nous finissons par trouver le log book qui est d'origine. Nous sommes très contents. Merci pour la cache\end{cacheText}

\cacheNumber{916}\needspace{5\baselineskip}\cacheName{\href{http://coord.info/GC3G8RG}{GR8\Number{}147} — \href{http://coord.info/GC3G8RG\Number{}728466424}{916}}\cacheData{{2017/10/20 Peyo64, Traditional Cache (2/1.5)}}\begin{cacheText}Grâce à l'indice, la cache est rapidement délogée. Merci pour ce beau parcours et cette vue magnifique.\end{cacheText}

\cacheNumber{917}\needspace{5\baselineskip}\cacheName{\href{http://coord.info/GC3G8T3}{GR8\Number{}148} — \href{http://coord.info/GC3G8T3\Number{}728466908}{917}}\cacheData{{2017/10/20 Peyo64, Traditional Cache (1.5/1.5)}}\begin{cacheText}Nous trouvons facilement et rapidement la cache de maintenance. Merci\end{cacheText}

\cacheNumber{918}\needspace{5\baselineskip}\cacheName{\href{http://coord.info/GC3G8TN}{GR8\Number{}149} — \href{http://coord.info/GC3G8TN\Number{}728467221}{918}}\cacheData{{2017/10/20 Peyo64, Traditional Cache (1.5/1.5)}}\begin{cacheText}Arrivés sur les lieux et grâce au spoiler nous découvrons la cache avec son log book  d'origine. Merci\end{cacheText}

\cacheNumber{919}\needspace{5\baselineskip}\cacheName{\href{http://coord.info/GC3G8TZ}{GR8\Number{}150} — \href{http://coord.info/GC3G8TZ\Number{}728469285}{919}}\cacheData{{2017/10/20 Peyo64, Traditional Cache (1.5/1.5)}}\begin{cacheText}Aussitôt dit aussitôt fait nous trouvons le log book de maintenance en deux temps trois mouvements. Merci pour la cache et sa maintenance.\end{cacheText}

\cacheNumber{920}\needspace{5\baselineskip}\cacheName{\href{http://coord.info/GC3G8VJ}{GR8\Number{}151} — \href{http://coord.info/GC3G8VJ\Number{}728469537}{920}}\cacheData{{2017/10/20 Peyo64, Traditional Cache (1.5/2)}}\begin{cacheText}Le GPS nous guide jusqu'à la cache. Merci pour cette vue superbe. Nous reprenons la route vers cette longue ascension.\end{cacheText}

\cacheNumber{921}\needspace{5\baselineskip}\cacheName{\href{http://coord.info/GC3G8VY}{GR8\Number{}152} — \href{http://coord.info/GC3G8VY\Number{}728470950}{921}}\cacheData{{2017/10/20 Peyo64, Traditional Cache (2/1.5)}}\begin{cacheText}Ici la vue est époustouflante, exceptionnelle. Nous ne serions probablement jamais venus sans le géocaching. La cache est vite trouvée et c'est un logbook de maintenance que nous signons. Merci pour la cache et sa maintenance\end{cacheText}

\cacheNumber{922}\needspace{5\baselineskip}\cacheName{\href{http://coord.info/GC3G8WH}{GR8\Number{}153} — \href{http://coord.info/GC3G8WH\Number{}728471454}{922}}\cacheData{{2017/10/20 Peyo64, Traditional Cache (2/1.5)}}\begin{cacheText}Arrivés aux coordonnées nous trouvons sous les pierres le log book de maintenance des DorisBear. Merci pour la cache et sa maintenance\end{cacheText}

\cacheNumber{923}\needspace{5\baselineskip}\cacheName{\href{http://coord.info/GC3G8WT}{GR8\Number{}154} — \href{http://coord.info/GC3G8WT\Number{}728471792}{923}}\cacheData{{2017/10/20 Peyo64, Traditional Cache (2/1.5)}}\begin{cacheText}Avec ce soleil radieux et ces couleurs automnales, la balade est un pur plaisir. La cache est vite découverte grâce à l'indice et au spoiler. Merci\end{cacheText}

\cacheNumber{924}\needspace{5\baselineskip}\cacheName{\href{http://coord.info/GC3G8X1}{GR8\Number{}155} — \href{http://coord.info/GC3G8X1\Number{}728472030}{924}}\cacheData{{2017/10/20 Peyo64, Traditional Cache (1.5/1.5)}}\begin{cacheText}Pas de soucis,la cache nous attend au pied. Merci\end{cacheText}

\cacheNumber{925}\needspace{5\baselineskip}\cacheName{\href{http://coord.info/GC3G8X4}{GR8\Number{}156} — \href{http://coord.info/GC3G8X4\Number{}728472238}{925}}\cacheData{{2017/10/20 Peyo64, Traditional Cache (2/1.5)}}\begin{cacheText}Arrivés aux coordonnées nous repérons facilement le tronc et trouvons le log book d'origine. Merci pour cette belle balade.\end{cacheText}

\cacheNumber{926}\needspace{5\baselineskip}\cacheName{\href{http://coord.info/GC3G8XF}{GR8\Number{}157} — \href{http://coord.info/GC3G8XF\Number{}728472411}{926}}\cacheData{{2017/10/20 Peyo64, Traditional Cache (2/1.5)}}\begin{cacheText}Pas de soucis pour découvrir la belle grâce à l'indice .Le lieu est très paisible et charmant. Merci pour cette découverte\end{cacheText}

\cacheNumber{927}\needspace{5\baselineskip}\cacheName{\href{http://coord.info/GC3G8XQ}{GR8\Number{}158} — \href{http://coord.info/GC3G8XQ\Number{}728472873}{927}}\cacheData{{2017/10/20 Peyo64, Traditional Cache (2/1.5)}}\begin{cacheText}Nous continuons  sous ce soleil éblouissant. Journée très agréable. Nous découvrons et nous signons le logbook de maintenance. Merci pour la cache et son entretien.\end{cacheText}

\cacheNumber{928}\needspace{5\baselineskip}\cacheName{\href{http://coord.info/GC3G8Y1}{GR8\Number{}159} — \href{http://coord.info/GC3G8Y1\Number{}728473732}{928}}\cacheData{{2017/10/20 Peyo64, Traditional Cache (2/1.5)}}\begin{cacheText}C'est un magnifique Oratoire que nous découvrons ici en l'hommage de Saint Eloi .La cache est trouvée sans problème grâce à l'indice. Merci Peyo pour tout ce travail et pour les merveilleux endroits que nous avons découverts.\end{cacheText}

\cacheNumber{929}\needspace{5\baselineskip}\cacheName{\href{http://coord.info/GC3T20Q}{Un pas pour Lokateo64 ! Mouhahahaha !} — \href{http://coord.info/GC3T20Q\Number{}728473960}{929}}\cacheData{{2017/10/20 Peyo64, Traditional Cache (1.5/1.5)}}\begin{cacheText}De passage pour faire les cache du GR huit nous nous arrêtons à ce charmant petit lavoir. La cache est rapidement trouvée grâce au GPS. L'endroit est magnifique. Merci pour la cache\end{cacheText}

\cacheNumber{930}\needspace{5\baselineskip}\cacheName{\href{http://coord.info/GC4D19F}{Moulin Plazako Errota - St Pé} — \href{http://coord.info/GC4D19F\Number{}728478067}{930}}\cacheData{{2017/10/20 gilles64, Traditional Cache (1.5/1.5)}}\begin{cacheText}Le GPS nous mène tout droit à la cache qui nous fait découvrir un très joli moulin habité. Merci\end{cacheText}

\cacheNumber{931}\needspace{5\baselineskip}\cacheName{\href{http://coord.info/GC5CZ49}{Vierge du quartier Olha} — \href{http://coord.info/GC5CZ49\Number{}728476057}{931}}\cacheData{{2017/10/20 brunolli, Traditional Cache (1.5/1.5)}}\begin{cacheText}Ici nous sommes à Saint-Pée-sur-Nivelle et je ne suis pas sûre que l'eau soit bénite… Mais bon nous trouvons la cache rapidement. Merci pour cette cache sympa.\end{cacheText}

\cacheNumber{932}\needspace{5\baselineskip}\cacheName{\href{http://coord.info/GC5R1K9}{\Number{}SO01 Sare - Oratoire Marie Mère de Dieu} — \href{http://coord.info/GC5R1K9\Number{}728474622}{932}}\cacheData{{2017/10/20 gilles64, Traditional Cache (2.5/1.5)}}\begin{cacheText}Nous découvrons un magnifique oratoire dédié à Marie. La cache est trouvée après avoir compris l'indice....Nous avons la chance de signer un logbook de maintenance sur lequel nous relevons l'indice pour la bonus. Merci pour la cache\end{cacheText}

\cacheNumber{933}\needspace{5\baselineskip}\cacheName{\href{http://coord.info/GC5RWJX}{Sare - Pont Romain \Quoted{Granadako-Zubia}} — \href{http://coord.info/GC5RWJX\Number{}728475462}{933}}\cacheData{{2017/10/20 gilles64, Traditional Cache (1.5/1.5)}}\begin{cacheText}La cache est trouvée assez rapidement et elle nous permet de découvrir un magnifique pont. MPLC\end{cacheText}

\cacheNumber{934}\needspace{5\baselineskip}\cacheName{\href{http://coord.info/GC5T49M}{\Number{}SO10 Sare - Oratoire Saint Nicolas} — \href{http://coord.info/GC5T49M\Number{}728475175}{934}}\cacheData{{2017/10/20 gilles64, Traditional Cache (1.5/1.5)}}\begin{cacheText}Sur cette cache, l'indice nous a bien aidé et guidé . Encore un superbe Oratoire découvert grâce au Géocaching. Merci pour la cache\end{cacheText}

\cacheNumber{935}\needspace{5\baselineskip}\cacheName{\href{http://coord.info/GC6FX9E}{L'ARSAGUAISE \Number{} 1} — \href{http://coord.info/GC6FX9E\Number{}730255391}{935}}\cacheData{{2017/10/28 crispol40, Traditional Cache (1.5/1.5)}}\begin{cacheText}Par ce bel après midi l'envie de faire quelques caches se fait sentir.... nous décidons de faire un petit circuit et optons pour celui d'Arsague. Arrivés au PZ ,nous découvrons une très jolie église. La cache est découverte rapidement. Merci\end{cacheText}

\cacheNumber{936}\needspace{5\baselineskip}\cacheName{\href{http://coord.info/GC6FXAB}{L'ARSAGUAISE \Number{}2} — \href{http://coord.info/GC6FXAB\Number{}730259929}{936}}\cacheData{{2017/10/28 crispol40, Traditional Cache (3/1.5)}}\begin{cacheText}Arrivés sur le PZ une battue de chasse était en cours!!!! Nous avons donc décidé de faire le parcours dans l'autre sens. Arrivés une seconde fois sur les lieux les chasseurs plient bagages !!!! Ouffff!!! Le GPS nous promène encore. Nous mettons du temps à chercher ....à droite, à gauche… Finalement c'est grâce à la persévérance et à la perspicacité de Pierre que l'on finit par la dégoter. Du grand travail. Merci pour la cache\end{cacheText}

\cacheNumber{937}\needspace{5\baselineskip}\cacheName{\href{http://coord.info/GC6FXB6}{L'ARSAGUAISE \Number{}3} — \href{http://coord.info/GC6FXB6\Number{}730259822}{937}}\cacheData{{2017/10/28 crispol40, Traditional Cache (1.5/1.5)}}\begin{cacheText}Ici nous trouvons une cache parfaitement intégrée au paysage. Encore une cache comme je les aime du vrai geocaching . Merci Paul pour ce très joli circuit.\end{cacheText}

\cacheNumber{938}\needspace{5\baselineskip}\cacheName{\href{http://coord.info/GC6FXBM}{L'ARSAGUAISE \Number{}4} — \href{http://coord.info/GC6FXBM\Number{}730259685}{938}}\cacheData{{2017/10/28 crispol40, Traditional Cache (2/1.5)}}\begin{cacheText}Dure,dure à trouver celle-ci. Elle est très petite et bien cachée ,il faut ouvrir les yeux!!!!Nous finissons par la déloger et repartons. Merci pour la cache.\end{cacheText}

\cacheNumber{939}\needspace{5\baselineskip}\cacheName{\href{http://coord.info/GC6FXCE}{L'ARSAGUAISE \Number{}5} — \href{http://coord.info/GC6FXCE\Number{}730259580}{939}}\cacheData{{2017/10/28 crispol40, Traditional Cache (1.5/2)}}\begin{cacheText}Nous trouvons ici sans trop de difficulté une jolie petite boîte à bidules. Nous déposons un petit personnage, un petit coup de tampon sur le log book, et nous repartons. Merci pour la cache\end{cacheText}

\cacheNumber{940}\needspace{5\baselineskip}\cacheName{\href{http://coord.info/GC6FXD0}{L'ARSAGUAISE \Number{}6} — \href{http://coord.info/GC6FXD0\Number{}730259483}{940}}\cacheData{{2017/10/28 crispol40, Traditional Cache (1.5/1.5)}}\begin{cacheText}Pas de problème pour celle-ci.... merci\end{cacheText}

\cacheNumber{941}\needspace{5\baselineskip}\cacheName{\href{http://coord.info/GC6G362}{L'ARSAGUAISE \Number{}7} — \href{http://coord.info/GC6G362\Number{}730259397}{941}}\cacheData{{2017/10/28 crispol40, Traditional Cache (1.5/1.5)}}\begin{cacheText}Ici nous trouvons une cache très astucieuse que nous avons repérée grâce à l'indice. Merci\end{cacheText}

\cacheNumber{942}\needspace{5\baselineskip}\cacheName{\href{http://coord.info/GC6G371}{L'ARSAGUAISE \Number{}8} — \href{http://coord.info/GC6G371\Number{}730259293}{942}}\cacheData{{2017/10/28 crispol40, Traditional Cache (1.5/1.5)}}\begin{cacheText}Cette cache a été trouvée assez rapidement malgré un GPS très changeant. Merci pour cette belle balade.\end{cacheText}

\cacheNumber{943}\needspace{5\baselineskip}\cacheName{\href{http://coord.info/GC6G37B}{L'ARSAGUAISE \Number{}9} — \href{http://coord.info/GC6G37B\Number{}730259197}{943}}\cacheData{{2017/10/28 crispol40, Traditional Cache (1.5/1.5)}}\begin{cacheText}Encore une très belle surprise :nous débusquons la cache après un petit moment de recherche. Un PF évidemment. Merci\end{cacheText}

\cacheNumber{944}\needspace{5\baselineskip}\cacheName{\href{http://coord.info/GC6G37T}{L'ARSAGUAISE \Number{}10} — \href{http://coord.info/GC6G37T\Number{}730259112}{944}}\cacheData{{2017/10/28 crispol40, Traditional Cache (1.5/1.5)}}\begin{cacheText}Malgré les coordonnées imprécises du GPS nous finissons par débusquer la belle qui est très bien camouflée mais attention à l'owner c'est un farceur!!!!Merci pour cette cache super sympa qui mérite 1 PF\end{cacheText}

\cacheNumber{945}\needspace{5\baselineskip}\cacheName{\href{http://coord.info/GC6G385}{L'ARSAGUAISE \Number{}11} — \href{http://coord.info/GC6G385\Number{}730258885}{945}}\cacheData{{2017/10/28 crispol40, Traditional Cache (1.5/1.5)}}\begin{cacheText}La cache est bien camouflée mais elle  ne nous résiste pas..... Merci pour la cache.\end{cacheText}

\cacheNumber{946}\needspace{5\baselineskip}\cacheName{\href{http://coord.info/GC6G38K}{L'ARSAGUAISE \Number{}12} — \href{http://coord.info/GC6G38K\Number{}730258558}{946}}\cacheData{{2017/10/28 crispol40, Traditional Cache (1.5/1.5)}}\begin{cacheText}Arrivés sur les lieux nous approchons de l'arbre qui est infesté de guêpes et de frelons...Je commence à raler me disant que la malchance est avec nous mais ouf nous finissons par la débusquer sans prendre de risques. Merci pour la cache\end{cacheText}

\cacheNumber{947}\needspace{5\baselineskip}\cacheName{\href{http://coord.info/GC6G3B6}{L'ARSAGUAISE \Number{} BONUS} — \href{http://coord.info/GC6G3B6\Number{}730260029}{947}}\cacheData{{2017/10/28 crispol40, Unknown Cache (1.5/1.5)}}\begin{cacheText}Les indices relevés et les bonnes coordonnées en poche nous arrivons au PZ où des personnes jouent à la pétanque !!!! Vraiment très fréquenté ce charmant village.... il nous faut être discret pour déloger la Belle. Nous déposons un TB . Un grand merci à Paul pour ce parcours très sympa.\end{cacheText}

\cacheNumber{948}\needspace{5\baselineskip}\cacheName{\href{http://coord.info/GC6PMG3}{Agglomération Pays Basque Sud} — \href{http://coord.info/GC6PMG3\Number{}734007740}{948}}\cacheData{{2017/11/18 Fidel Gómez, Traditional Cache (1.5/1.5)}}\begin{cacheText}En avance pour assister au match du petit dernier nous en profitons pour faire quelques caches. Nous commençons par celle-ci. Grâce aux nombreux indices donnés par les commentaires précédents nous mettons la main dessus en 2 temps 3 mouvements. Merci pour la cache\end{cacheText}

\cacheNumber{949}\needspace{5\baselineskip}\cacheName{\href{http://coord.info/GC6PMG6}{Les oiseaux de l'Untxin} — \href{http://coord.info/GC6PMG6\Number{}734008734}{949}}\cacheData{{2017/11/18 Fidel Gómez, Traditional Cache (1.5/1.5)}}\begin{cacheText}Arrivés sur les lieux nous avons délogé très rapidement la cache. Le log book est trempé et inutilisable. Nous marquons malgré tout un Found It. Nous n'avons rien pour remplacer le log book il faut faire une maintenance en urgence . Merci pour la cache\end{cacheText}

\cacheNumber{950}\needspace{5\baselineskip}\cacheName{\href{http://coord.info/GC6EB26}{Le pont sur l'Untxin} — \href{http://coord.info/GC6EB26\Number{}734011289}{950}}\cacheData{{2017/11/18 Fidel Gómez , Traditional Cache (1/1.5)}}\begin{cacheText}Dans la région pour assister à un match de rugby nous nous attelons à ces quelques caches. Celle-ci est rapidement trouvée. Merci pour cette découverte.\end{cacheText}

\cacheNumber{951}\needspace{5\baselineskip}\cacheName{\href{http://coord.info/GC6RRTA}{J032	 - Géodyssée 40 64} — \href{http://coord.info/GC6RRTA\Number{}734012383}{951}}\cacheData{{2017/11/18 gilles64, Traditional Cache (1.5/1.5)}}\begin{cacheText}En route pour assister à un match sur Hendaye, nous ne pouvons pas résister à la tentation de cette cache. Celle-ci est rapidement repérée grâce à l'indice et au spoiler. Merci pour la cache\end{cacheText}

\cacheNumber{952}\needspace{5\baselineskip}\cacheName{\href{http://coord.info/GC6V6PA}{J035	 - Géodyssée 40 64} — \href{http://coord.info/GC6V6PA\Number{}734013393}{952}}\cacheData{{2017/11/18 gilles64, Traditional Cache (1.5/1.5)}}\begin{cacheText}En route pour Hendaye nous dégotons la cache très rapidement: heureusement car l'heure tourne...Merci pour la cache\end{cacheText}

\cacheNumber{953}\needspace{5\baselineskip}\cacheName{\href{http://coord.info/GC6V6QV}{J036	 - Géodyssée 40 64} — \href{http://coord.info/GC6V6QV\Number{}734015044}{953}}\cacheData{{2017/11/18 gilles64, Traditional Cache (1.5/1.5)}}\begin{cacheText}Cache toujours en place ... elle nous attend bien sagement . Merci pour la découverte du château observatoire d'Abadia.\end{cacheText}

\cacheNumber{954}\needspace{5\baselineskip}\cacheName{\href{http://coord.info/GC3CW0J}{GR10\Number{}06} — \href{http://coord.info/GC3CW0J\Number{}734016237}{954}}\cacheData{{2017/11/18 Peyo64, Traditional Cache (1.5/1.5)}}\begin{cacheText}En attendant le début du match de rugby nous en profitons pour faire quelques cache sur Hendaye. Celle-ci est vite repérée. Malheureusement le banc est occupé… Il nous faut patienter et par chance les personnes s'en vont assez vite ouffff. Merci pour la cache\end{cacheText}

\cacheNumber{955}\needspace{5\baselineskip}\cacheName{\href{http://coord.info/GC6V6V0}{J039	 - Géodyssée 40 64} — \href{http://coord.info/GC6V6V0\Number{}734017143}{955}}\cacheData{{2017/11/18 gilles64, Traditional Cache (1.5/1.5)}}\begin{cacheText}La jolie boîboîte est découverte en deux temps trois mouvements grâce à la photo. Merci pour cette superbe vue sur le port.\end{cacheText}

\cacheNumber{956}\needspace{5\baselineskip}\cacheName{\href{http://coord.info/GC3CW1P}{GR10\Number{}07} — \href{http://coord.info/GC3CW1P\Number{}734018205}{956}}\cacheData{{2017/11/18 Peyo64, Traditional Cache (1.5/1.5)}}\begin{cacheText}Et une de plus au compteur. Merci pour la cache et cette superbe vue\end{cacheText}

\cacheNumber{957}\needspace{5\baselineskip}\cacheName{\href{http://coord.info/GC3JAC8}{📷 Bordeaux touristique - Place des Quinconces} — \href{http://coord.info/GC3JAC8\Number{}734423461}{957}}\cacheData{{2017/11/19 Calimero33, Traditional Cache (1.5/1)}}\begin{cacheText}Ici aussi, nous trouvons la cache rapidement mais le logbook est plein!!Victime du succès!!!  Je n'ai plus de papier.... une petite maintenance s'impose. Merci pour la découverte de cette superbe place des Quinconces.\end{cacheText}

\cacheNumber{958}\needspace{5\baselineskip}\cacheName{\href{http://coord.info/GC3JACX}{Cours Xavier Arnozan} — \href{http://coord.info/GC3JACX\Number{}734422908}{958}}\cacheData{{2017/11/19 Calimero33, Traditional Cache (1.5/1.5)}}\begin{cacheText}Nous continuons la quête et trouvons la belle. Merci pour la découverte des lieux.\end{cacheText}

\cacheNumber{959}\needspace{5\baselineskip}\cacheName{\href{http://coord.info/GC3JADH}{Place Paul Doumer} — \href{http://coord.info/GC3JADH\Number{}734422771}{959}}\cacheData{{2017/11/19 Calimero33, Traditional Cache (1.5/1)}}\begin{cacheText}Nous arrivons juste car le tramway redémarre. Zut nous devons patienter des personnes occupent les bancs ,nous allons faire une petite pause!!!Merci pour la cache\end{cacheText}

\cacheNumber{960}\needspace{5\baselineskip}\cacheName{\href{http://coord.info/GC3JAE9}{Halle des Chartrons et Eglise Saint Louis} — \href{http://coord.info/GC3JAE9\Number{}734422609}{960}}\cacheData{{2017/11/19 Calimero33, Traditional Cache (1.5/1.5)}}\begin{cacheText}Cette cache est effectivement très bien dissimulée… mais elle ne nous a pas résistée. Merci pour la cache\end{cacheText}

\cacheNumber{961}\needspace{5\baselineskip}\cacheName{\href{http://coord.info/GC3JAEK}{La cité mondiale du vin} — \href{http://coord.info/GC3JAEK\Number{}734423574}{961}}\cacheData{{2017/11/19 Calimero33, Traditional Cache (1.5/1.5)}}\begin{cacheText}La place n'est pas trop fréquentée en ce dimanche de début d'après-midi. À force de recherches nous finissons par mettre la main dessus grâce aux photos des logs précédents. Merci pour la découverte de cette belle cité et ce parvis des Chartrons.\end{cacheText}

\cacheNumber{962}\needspace{5\baselineskip}\cacheName{\href{http://coord.info/GC3M32H}{Terrasse du Jardin Public} — \href{http://coord.info/GC3M32H\Number{}734422961}{962}}\cacheData{{2017/11/19 Calimero33, Traditional Cache (2/1)}}\begin{cacheText}Nous arrivons dans ce superbe parc qui est aujourd'hui très fréquenté mais par chance l'endroit que nous recherchons est libre. Merci pour la cache\end{cacheText}

\cacheNumber{963}\needspace{5\baselineskip}\cacheName{\href{http://coord.info/GC41BRV}{Faubourg des Arts} — \href{http://coord.info/GC41BRV\Number{}734422426}{963}}\cacheData{{2017/11/19 Calimero33, Traditional Cache (1.5/1)}}\begin{cacheText}Après avoir bien mangé et bien bu, nous découvrons cette cache qui nous a fait tourner en rond un petit moment. Merci\end{cacheText}

\cacheNumber{964}\needspace{5\baselineskip}\cacheName{\href{http://coord.info/GC4FX29}{Place Picard} — \href{http://coord.info/GC4FX29\Number{}734422564}{964}}\cacheData{{2017/11/19 Calimero33, Traditional Cache (1.5/1)}}\begin{cacheText}Nous avions besoin d'une petite pause… Cela tombe à pic. Merci pour la cache\end{cacheText}

\cacheNumber{965}\needspace{5\baselineskip}\cacheName{\href{http://coord.info/GC4N6TH}{Le pont Ba-Ba} — \href{http://coord.info/GC4N6TH\Number{}734421303}{965}}\cacheData{{2017/11/19 Calimero33, Traditional Cache (1.5/1.5)}}\begin{cacheText}Malgré un trafic très intense en ce dimanche matin, nous finissons par déloger la belle. Merci pour la découverte de ce superbe pont et de cette magnifique vue sur Bordeaux.\end{cacheText}

\cacheNumber{966}\needspace{5\baselineskip}\cacheName{\href{http://coord.info/GC4NB1R}{Le Jardin imaginaire} — \href{http://coord.info/GC4NB1R\Number{}734421225}{966}}\cacheData{{2017/11/19 Calimero33, Traditional Cache (1.5/1.5)}}\begin{cacheText}Arrivés sur Bordeaux de bonne heure pour assister au Choc Girondin /Marseille nous nous lançons à la recherche de nos premières caches sur ce département. En ce dimanche matin, les lieux sont très fréquentés mais nous mettons la main dessus discrètement. Merci\end{cacheText}

\cacheNumber{967}\needspace{5\baselineskip}\cacheName{\href{http://coord.info/GC4RM84}{SkatePark} — \href{http://coord.info/GC4RM84\Number{}734422260}{967}}\cacheData{{2017/11/19 Calimero33, Traditional Cache (1.5/1)}}\begin{cacheText}Pas de difficultés pour cette cache...Les indices sont explicites. Merci pour la découverte de ce joli skate Park qui est très fréquenté en ce dimanche matin. Merci pour la cache.\end{cacheText}

\cacheNumber{968}\needspace{5\baselineskip}\cacheName{\href{http://coord.info/GC5HAZ8}{Le Temple} — \href{http://coord.info/GC5HAZ8\Number{}734423652}{968}}\cacheData{{2017/11/19 Calimero33, Traditional Cache (1.5/1.5)}}\begin{cacheText}Nous découvrons, ici, un joli temple perdu au milieu des ruelles. Merci pour la cache\end{cacheText}

\cacheNumber{969}\needspace{5\baselineskip}\cacheName{\href{http://coord.info/GC5HAZC}{Entre Place et Jardin} — \href{http://coord.info/GC5HAZC\Number{}734423219}{969}}\cacheData{{2017/11/19 Calimero33, Traditional Cache (1.5/1.5)}}\begin{cacheText}Pas de difficultés pour dénicher la belle mais le log book est plein .Nous rajoutons un papier. Merci pour la cache.\end{cacheText}

\cacheNumber{970}\needspace{5\baselineskip}\cacheName{\href{http://coord.info/GC6F0K9}{Les animaux citadins} — \href{http://coord.info/GC6F0K9\Number{}734422543}{970}}\cacheData{{2017/11/19 Calimero33, Traditional Cache (1.5/1.5)}}\begin{cacheText}Et voici une vraie cache comme je les aime…intégrée!!!! Merci pour la cache.\end{cacheText}

\cacheNumber{971}\needspace{5\baselineskip}\cacheName{\href{http://coord.info/GC6RKC9}{\Number{}2 Le lac de Bordeaux 2 - Le parc des expositions} — \href{http://coord.info/GC6RKC9\Number{}734424500}{971}}\cacheData{{2017/11/19 nonnnno, Traditional Cache (1.5/2)}}\begin{cacheText}Arrivés sur le bord du lac des moldus patientent autour d'une bière en attendant le match Bordeaux Marseille. Ce n'est pas facile mais nous finissons par mettre la main dessus. Merci pour la découverte de ce très joli lac.\end{cacheText}

\cacheNumber{972}\needspace{5\baselineskip}\cacheName{\href{http://coord.info/GC6RKCH}{\Number{}3 Le lac de Bordeaux 2 - Les portes drapeaux} — \href{http://coord.info/GC6RKCH\Number{}734424381}{972}}\cacheData{{2017/11/19 nonnnno, Traditional Cache (2/1.5)}}\begin{cacheText}Encore une cache rapidement délogée. Merci pour ce très joli circuit.\end{cacheText}

\cacheNumber{973}\needspace{5\baselineskip}\cacheName{\href{http://coord.info/GC6RKDG}{\Number{}4 Le lac de Bordeaux 2 - Le parc des expositions} — \href{http://coord.info/GC6RKDG\Number{}734424340}{973}}\cacheData{{2017/11/19 nonnnno, Traditional Cache (1.5/1.5)}}\begin{cacheText}Nous continuons notre quête et trouvons le Graal à la Porte D. Merci pour la cache.\end{cacheText}

\cacheNumber{974}\needspace{5\baselineskip}\cacheName{\href{http://coord.info/GC6RKDV}{\Number{}5 Le lac de Bordeaux 2 - Le stade} — \href{http://coord.info/GC6RKDV\Number{}734424266}{974}}\cacheData{{2017/11/19 nonnnno, Traditional Cache (1.5/1.5)}}\begin{cacheText}En attendant le match Bordeaux Marseille nous continuons nos petites caches avant la tombée de la nuit. Celle-ci est trouvée au nez et à la barbe des moldus. Merci pour la cache\end{cacheText}

\cacheNumber{975}\needspace{5\baselineskip}\cacheName{\href{http://coord.info/GC6RKG6}{\Number{}6 Le lac de Bordeaux 2 - Le parc floral} — \href{http://coord.info/GC6RKG6\Number{}734423893}{975}}\cacheData{{2017/11/19 nonnnno, Traditional Cache (4/1.5)}}\begin{cacheText}En cette fin d'après-midi alors que nous attendons le match prévue à 21h Bordeaux Marseille nous attaquons les quelques taches alentours. Celle-ci a été repérée… Elle est toujours en place. Merci pour la cache et un PF\end{cacheText}

\cacheNumber{976}\needspace{5\baselineskip}\cacheName{\href{http://coord.info/GC6RKGC}{\Number{}7 Le lac de Bordeaux 2 - Plaine des sport Besson} — \href{http://coord.info/GC6RKGC\Number{}734423769}{976}}\cacheData{{2017/11/19 nonnnno, Traditional Cache (2/1.5)}}\begin{cacheText}En attendant le match Bordeaux Marseille nous avons le temps de faire quelques caches . Celle-ci est trouvée sans difficulté et nous permet de profiter du soleil. Merci pour la cache\end{cacheText}

\cacheNumber{977}\needspace{5\baselineskip}\cacheName{\href{http://coord.info/GC6VZHF}{Et ta soeur ?} — \href{http://coord.info/GC6VZHF\Number{}734422172}{977}}\cacheData{{2017/11/19 GAVbrioche33 \And{} jojoarca, Traditional Cache (1.5/1.5)}}\begin{cacheText}Sur Bordeaux pour assister au Choc Girondin/Marseille, nous en profitons pour dénicher quelques caches et ainsi découvrir cette belle ville. Après avoir cherché la belle un bon moment ,nous finissons par mettre la main dessus. La boite est assez grande, nous en profitons pour déposer un TB et un bouchon Merci pour la découverte de ce joli parc.\end{cacheText}

\cacheNumber{978}\needspace{5\baselineskip}\cacheName{\href{http://coord.info/GC6WF5P}{⚒ Bordeaux 2030 - 01 - Cité du Vin} — \href{http://coord.info/GC6WF5P\Number{}734421544}{978}}\cacheData{{2017/11/19 Calimero33, Traditional Cache (1.5/1.5)}}\begin{cacheText}Nous arrivons sur la cité du Vin. ..vraiment exceptionnel. Nous trouvons la cache assez rapidement. Merci pour la découverte de ce lieu.\end{cacheText}

\cacheNumber{979}\needspace{5\baselineskip}\cacheName{\href{http://coord.info/GC7CECW}{ALDDB - Entre écluses et silos} — \href{http://coord.info/GC7CECW\Number{}734421666}{979}}\cacheData{{2017/11/19 nonnnno, Traditional Cache (1.5/1.5)}}\begin{cacheText}C'est un classique du genre que nous trouvons en deux temps trois mouvements. Merci pour la cache.\end{cacheText}

\cacheNumber{980}\needspace{5\baselineskip}\cacheName{\href{http://coord.info/GC7CG4B}{ALDDB - Ancien hangar des dockers} — \href{http://coord.info/GC7CG4B\Number{}734421443}{980}}\cacheData{{2017/11/19 nonnnno, Traditional Cache (1.5/1.5)}}\begin{cacheText}Apres avoir traversé le splendide pont nous voici au milieu des anciens hangars....belle reconversion!!!La belle ,bien cachée, ne nous résiste pas. Merci pour la découverte de ces lieux.\end{cacheText}

\cacheNumber{981}\needspace{5\baselineskip}\cacheName{\href{http://coord.info/GC4WM0W}{Parc Camille DUSSARTHOU -- St Paul les Dax} — \href{http://coord.info/GC4WM0W\Number{}734611315}{981}}\cacheData{{2017/11/20 Sod@'s, Traditional Cache (3/1.5)}}\begin{cacheText}Sur Dax pour participer à l'Event  la Tibby Team est encore de retour nous en profitons pour faire quelques caches .Nous commençons par celle-ci  et découvrons du vrai géocaching comme nous l'aimons. Heureusement nous avons ce qu'il faut pour sortir le logbook. Merci pour cette cache\end{cacheText}

\cacheNumber{982}\needspace{5\baselineskip}\cacheName{\href{http://coord.info/GC657NR}{EVDA II} — \href{http://coord.info/GC657NR\Number{}734617691}{982}}\cacheData{{2017/11/20 mizaga, Traditional Cache (1.5/1.5)}}\begin{cacheText}Cache trouvée en allant manger au Richelieu lors de l'Event la Tibby Team est encore de retour. C'est avec l'aide de CLePhan que nous débusquons la cache. Merci .\end{cacheText}

\cacheNumber{983}\needspace{5\baselineskip}\cacheName{\href{http://coord.info/GC65G7Z}{Ecole de musique} — \href{http://coord.info/GC65G7Z\Number{}734611498}{983}}\cacheData{{2017/11/20 lauki3940, Traditional Cache (2/2)}}\begin{cacheText}Nous continuons nos recherches et arrivons devant l'école municipale de musique et de danse. La cache est débusquée en deux temps trois mouvements. Merci pour la cache\end{cacheText}

\cacheNumber{984}\needspace{5\baselineskip}\cacheName{\href{http://coord.info/GC65G86}{La pause des tout petits} — \href{http://coord.info/GC65G86\Number{}734614822}{984}}\cacheData{{2017/11/20 lauki3940, Traditional Cache (2/2)}}\begin{cacheText}Enfin!!! Arrivés sur les lieux nous trouvons rapidement la belle qui est à l'abris des regards. Merci pour la cache\end{cacheText}

\cacheNumber{985}\needspace{5\baselineskip}\cacheName{\href{http://coord.info/GC698E2}{Arènes de Dax} — \href{http://coord.info/GC698E2\Number{}734611987}{985}}\cacheData{{2017/11/20 mizaga, Traditional Cache (1/1.5)}}\begin{cacheText}Grâce à l'indice et aux logs précédents… nous délogons la belle. Merci pour la découverte des arènes.\end{cacheText}

\cacheNumber{986}\needspace{5\baselineskip}\cacheName{\href{http://coord.info/GC6C2PC}{CHRISTUS V- aire de pic-nic} — \href{http://coord.info/GC6C2PC\Number{}734616661}{986}}\cacheData{{2017/11/20 mizaga, Traditional Cache (1.5/1.5)}}\begin{cacheText}Nous avons également cherché un bon moment cette cache… Le GPS s'affole mais nous finissons par mettre la main dessus .Oufff. Merci pour la cache\end{cacheText}

\cacheNumber{987}\needspace{5\baselineskip}\cacheName{\href{http://coord.info/GC6CD36}{CHRISTUS VII- les thermes} — \href{http://coord.info/GC6CD36\Number{}734616134}{987}}\cacheData{{2017/11/20 mizaga, Traditional Cache (1.5/1.5)}}\begin{cacheText}Elle nous a fait tourner en rond mais nous avons finis par l'avoir!!! Nous trouvons ici une cache parfaitement intégrée. Merci pour la découverte.\end{cacheText}

\cacheNumber{988}\needspace{5\baselineskip}\cacheName{\href{http://coord.info/GC74N9G}{Chateau d'eau de Gaston Larieux} — \href{http://coord.info/GC74N9G\Number{}734615569}{988}}\cacheData{{2017/11/20 lauki3940, Traditional Cache (1.5/1.5)}}\begin{cacheText}La belle est bien camouflée mais l'endroit est assez calme pour chercher .Du bon travail, merci pour la cache.\end{cacheText}

\cacheNumber{989}\needspace{5\baselineskip}\cacheName{\href{http://coord.info/GC7E8ZV}{La Tibby Team est encore de retour... by Tib} — \href{http://coord.info/GC7E8ZV\Number{}734619876}{989}}\cacheData{{2017/11/20 Team, Event Cache (1/1)}}\begin{cacheText}Un grand merci à la Tibby Team pour l'organisation de ce super Event. Nous avons passé une excellente soirée en très bonne compagnie ,dégusté un excellent repas et après avoir fait fumer les méninges, découvert une dernière  cache avant d'aller se coucher.

Au plaisir de se revoir...\end{cacheText}

\cacheNumber{990}\needspace{5\baselineskip}\cacheName{\href{http://coord.info/GC7EZ4M}{Agur Jaunak by Tib} — \href{http://coord.info/GC7EZ4M\Number{}735091465}{990}}\cacheData{{2017/11/20 Team, Unknown Cache (4/1.5)}}\begin{cacheText}Enigme résolue en commun lors du repas de l'Event la Tibby Team est encore de retour. Avant de prendre le chemin du retour, nous partons à la recherche de la belle qui est vite délogée. Milesker  hanitz la Tibby Team pour cette belle énigme et un PF.\end{cacheText}

\cacheNumber{991}\needspace{5\baselineskip}\cacheName{\href{http://coord.info/GC3VANV}{La source de marcerin} — \href{http://coord.info/GC3VANV\Number{}735401236}{991}}\cacheData{{2017/11/26 dakacora, Traditional Cache (1/1.5)}}\begin{cacheText}Nous découvrons ici une petite source. La belle est bien cachée mais nous la trouvons rapidement. Je dépose le TB L'eau c'est la vie. Merci pour la cache.\end{cacheText}

\cacheNumber{992}\needspace{5\baselineskip}\cacheName{\href{http://coord.info/GC5XDBE}{LA FONTAINE DES CAGOTS - ARTHEZ-DE-BEARN} — \href{http://coord.info/GC5XDBE\Number{}735400615}{992}}\cacheData{{2017/11/26 CCLO, Traditional Cache (1.5/1.5)}}\begin{cacheText}La cache est rapidement trouvée . Il n'y a pas trop d'orties en cette fin novembre.Nous découvrons une superbe petite fontaine oubliée de tous . Merci pour la découverte de ce lieu.\end{cacheText}

\cacheNumber{993}\needspace{5\baselineskip}\cacheName{\href{http://coord.info/GC6EWZK}{La passerelle} — \href{http://coord.info/GC6EWZK\Number{}735401653}{993}}\cacheData{{2017/11/26 echohs, Traditional Cache (1/1.5)}}\begin{cacheText}L'endroit est très sympa et nous n'avons pas de mal à trouver la cache. La boîte de céleri vide nous attend bien à sa place. Merci pour la découverte de ce lieu.\end{cacheText}

\cacheNumber{994}\needspace{5\baselineskip}\cacheName{\href{http://coord.info/GC7BNF8}{LA BOUCLE DU LAC DE DOAZON \Number{}01} — \href{http://coord.info/GC7BNF8\Number{}735370997}{994}}\cacheData{{2017/11/26 denisb64, Traditional Cache (1.5/2)}}\begin{cacheText}En mode sanglier nous arrivons jusqu'à la cache. Elle nous attend bien sagement à sa place.Merci pour la découverte de ce lac.\end{cacheText}

\cacheNumber{995}\needspace{5\baselineskip}\cacheName{\href{http://coord.info/GC7BNHZ}{LA BOUCLE DU LAC DE DOAZON \Number{}02} — \href{http://coord.info/GC7BNHZ\Number{}735388308}{995}}\cacheData{{2017/11/26 denisb64, Traditional Cache (2/2)}}\begin{cacheText}Équipés de bottes nous arrivons jusqu'à la cache.L'automne est là et les feuilles tombées nous laissent entrevoir la Belle. Mplc\end{cacheText}

\cacheNumber{996}\needspace{5\baselineskip}\cacheName{\href{http://coord.info/GC7BNJD}{LA BOUCLE DU LAC DE DOAZON \Number{}03} — \href{http://coord.info/GC7BNJD\Number{}735388322}{996}}\cacheData{{2017/11/26 denisb64, Traditional Cache (2/2)}}\begin{cacheText}Le temps est idéal pour faire ce tour du lac. Les coordonnées sont précises et nous découvrons la Belle en deux temps trois mouvements. Merci\end{cacheText}

\cacheNumber{997}\needspace{5\baselineskip}\cacheName{\href{http://coord.info/GC7BNKA}{LA BOUCLE DU LAC DE DOAZON \Number{}04} — \href{http://coord.info/GC7BNKA\Number{}735390681}{997}}\cacheData{{2017/11/26 denisb64, Traditional Cache (1.5/2)}}\begin{cacheText}Ici la cache nous donne du fil à retordre … Le GPS nous promène ! Heureusement que le spoiler est là pour nous guider vers la cache. Dépôt d'une petite figurine.Merci\end{cacheText}

\cacheNumber{998}\needspace{5\baselineskip}\cacheName{\href{http://coord.info/GC7BNKT}{LA BOUCLE DU LAC DE DOAZON \Number{}05} — \href{http://coord.info/GC7BNKT\Number{}735393029}{998}}\cacheData{{2017/11/26 denisb64, Traditional Cache (2/1.5)}}\begin{cacheText}Arrivés sur les lieux il ne nous a pas fallu longtemps pour repérer la Belle. La vue sur le lac est super chouette. Merci pour la cache qui est parfaitement intégrée au paysage. Nous apprécions cette belle balade en ce dimanche après-midi. Un PF évidemment\end{cacheText}

\cacheNumber{999}\needspace{5\baselineskip}\cacheName{\href{http://coord.info/GC7BNMM}{LA BOUCLE DU LAC DE DOAZON \Number{}06} — \href{http://coord.info/GC7BNMM\Number{}735394183}{999}}\cacheData{{2017/11/26 denisb64, Traditional Cache (2/1.5)}}\begin{cacheText}La balade se poursuit ,toujours aussi agréable, autour de ce lac. Cette cache est débusquée sans difficulté. Merci\end{cacheText}

\cacheNumber{1000}\needspace{5\baselineskip}\cacheName{\href{http://coord.info/GC7BNPB}{LA BOUCLE DU LAC DE DOAZON \Number{}07} — \href{http://coord.info/GC7BNPB\Number{}735395707}{1000}}\cacheData{{2017/11/26 denisb64, Traditional Cache (1.5/1.5)}}\begin{cacheText}Nous découvrons ici une cache fort ingénieuse. Il n'y a personne aux alentours , nous pouvons loguer tranquillement. Merci pour la découverte de ces lieux.\end{cacheText}

\cacheNumber{1001}\needspace{5\baselineskip}\cacheName{\href{http://coord.info/GC7BNPQ}{LA BOUCLE DU LAC DE DOAZON \Number{}08} — \href{http://coord.info/GC7BNPQ\Number{}735396286}{1001}}\cacheData{{2017/11/26 denisb64, Traditional Cache (1.5/3)}}\begin{cacheText}Arrivé sur le PZ nous comprenons pourquoi la cache est classée en terrain trois… Monsieur la déloge avec plus ou moins de facilité. Merci pour la cache\end{cacheText}

\cacheNumber{1002}\needspace{5\baselineskip}\cacheName{\href{http://coord.info/GC7BNX6}{MYSTERE DU LAC DE DOAZON \Number{}03} — \href{http://coord.info/GC7BNX6\Number{}735398716}{1002}}\cacheData{{2017/11/26 denisb64, Unknown Cache (2.5/1.5)}}\begin{cacheText}Seulement deux énigmes ont été résolues . Nous nous dirigeons donc vers la première....Alors que nous cherchions désespérément du mauvais côté, l'œil a été attiré par quelque chose qui brillait ! Hourra...nous trouvons mais ne sommes pas au bout de nos peines !!! La boîte a été forcée ( aucun respect c'est lamentable !!!) mais jouons le jeu jusqu'à trouver la bonne.... Un gros PF et un grand merci pour cette belle énigme et cette belle cache.\end{cacheText}

\cacheNumber{1003}\needspace{5\baselineskip}\cacheName{\href{http://coord.info/GC7BPRV}{MYSTERE DU LAC DE DOAZON \Number{}05} — \href{http://coord.info/GC7BPRV\Number{}735399859}{1003}}\cacheData{{2017/11/26 denisb64, Unknown Cache (2/1.5)}}\begin{cacheText}1000

La cache est trouvée rapidement le long de l'observatoire. Merci pour la découverte de ce lieu.\end{cacheText}

\cacheNumber{1004}\needspace{5\baselineskip}\cacheName{\href{http://coord.info/GC7EG58}{Calendrier de l'avent 1} — \href{http://coord.info/GC7EG58\Number{}736400892}{1004}}\cacheData{{2017/12/02 denisb64, Traditional Cache (1.5/1.5)}}\begin{cacheText}Apres avoir déjeuné ,nous nous divisons en deux groupes. Nous avons la chance d'être avec les denisb64 qui nous donnent un coup de mains pour décoder les Mystères sorties la veille!!!Trouvée avec la Team Évent Noël .La cache est délogée malgré des GPS qui nous promènent...les indices induisent parfois en erreur!!!Merci pour la cache.\end{cacheText}

\cacheNumber{1005}\needspace{5\baselineskip}\cacheName{\href{http://coord.info/GC7EG6J}{Calendrier de l'avent 2} — \href{http://coord.info/GC7EG6J\Number{}736400409}{1005}}\cacheData{{2017/12/02 denisb64, Traditional Cache (3/1.5)}}\begin{cacheText}Apres avoir déjeuné ,nous nous divisons en deux groupes. Nous avons la chance d'être avec les denisb64 qui nous donnent un coup de mains pour décoder les Mystères sorties la veille!!!Trouvée avec la Team Évent Noël .Les denisb se sont bien amusés à nous voir passer et repasser ...Cache comme nous les aimons. Un PF supplémentaire.\end{cacheText}

\cacheNumber{1006}\needspace{5\baselineskip}\cacheName{\href{http://coord.info/GC7EG7J}{Calendrier de l'avent 3} — \href{http://coord.info/GC7EG7J\Number{}736399163}{1006}}\cacheData{{2017/12/02 denisb64, Traditional Cache (2/4)}}\begin{cacheText}Apres avoir déjeuné ,nous nous divisons en deux groupes. Nous avons la chance d'être avec les denisb64 qui nous donnent un coup de mains pour décoder les Mystères sorties la veille!!!Trouvée avec la Team Évent Noël .Heureusement que le propriétaire ,agile, est parmi nous pour déloger la belle!!!Merci pour la cache.Un PF évidemment...\end{cacheText}

\cacheNumber{1007}\needspace{5\baselineskip}\cacheName{\href{http://coord.info/GC7EG8B}{Calendrier de l'avent 4} — \href{http://coord.info/GC7EG8B\Number{}736398515}{1007}}\cacheData{{2017/12/02 denisb64, Traditional Cache (1.5/1.5)}}\begin{cacheText}Apres avoir déjeuné ,nous nous divisons en deux groupes. Nous avons la chance d'être avec les denisb64 qui nous donnent un coup de mains pour décoder les Mystères sorties la veille!!!Trouvée avec la Team Évent Noël .Encore une belle surprise. Merci pour la cache.\end{cacheText}

\cacheNumber{1008}\needspace{5\baselineskip}\cacheName{\href{http://coord.info/GC7EG9N}{Calendrier de l'avent 5} — \href{http://coord.info/GC7EG9N\Number{}736397861}{1008}}\cacheData{{2017/12/02 denisb64, Traditional Cache (1.5/2)}}\begin{cacheText}Apres avoir déjeuné ,nous nous divisons en deux groupes. Nous avons la chance d'être avec les denisb64 qui nous donnent un coup de mains pour décoder les Mystères sorties la veille!!!Trouvée avec la Team Évent Noël .La belle est vite découverte. Merci pour la cache\end{cacheText}

\cacheNumber{1009}\needspace{5\baselineskip}\cacheName{\href{http://coord.info/GC7EG9Z}{Calendrier de l'avent 6} — \href{http://coord.info/GC7EG9Z\Number{}736397640}{1009}}\cacheData{{2017/12/02 denisb64, Traditional Cache (1.5/2)}}\begin{cacheText}Apres avoir déjeuné ,nous nous divisons en deux groupes. Nous avons la chance d'être avec les denisb64 qui nous donnent un coup de mains pour décoder les Mystères sorties la veille!!!Trouvée avec la Team Évent Noël .La belle nous attend bien sagement....Merci pour la cache.\end{cacheText}

\cacheNumber{1010}\needspace{5\baselineskip}\cacheName{\href{http://coord.info/GC7EGAQ}{Calendrier de l'avent 14} — \href{http://coord.info/GC7EGAQ\Number{}736144818}{1010}}\cacheData{{2017/12/02 denisb64, Traditional Cache (1.5/2)}}\begin{cacheText}Apres avoir déjeuné ,nous nous divisons en deux groupes. Nous avons la chance d'être avec les denisb64 qui nous donnent un coup de mains pour décoder les Mystères sorties la veille!!!Trouvée avec la Team Évent Noël .Pour celle ci il faut aller au plus simple!!!Merci pour la cache.\end{cacheText}

\cacheNumber{1011}\needspace{5\baselineskip}\cacheName{\href{http://coord.info/GC7EGBA}{Calendrier de l'avent 15} — \href{http://coord.info/GC7EGBA\Number{}736144701}{1011}}\cacheData{{2017/12/02 denisb64, Traditional Cache (2/1.5)}}\begin{cacheText}Apres avoir déjeuné ,nous nous divisons en deux groupes. Nous avons la chance d'être avec les denisb64 qui nous donnent un coup de mains pour décoder les Mystères sorties la veille!!!Trouvée avec la Team Évent Noël . Elle est fort bien dissimulée.Merci pour la cache.\end{cacheText}

\cacheNumber{1012}\needspace{5\baselineskip}\cacheName{\href{http://coord.info/GC7EGCG}{Calendrier de l'avent 16} — \href{http://coord.info/GC7EGCG\Number{}736144548}{1012}}\cacheData{{2017/12/02 denisb64, Traditional Cache (4/2)}}\begin{cacheText}Apres avoir déjeuné ,nous nous divisons en deux groupes. Nous avons la chance d'être avec les denisb64 qui nous donnent un coup de mains pour décoder les Mystères sorties la veille!!!Trouvée avec la Team Évent Noël  .Merci pour la cache.\end{cacheText}

\cacheNumber{1013}\needspace{5\baselineskip}\cacheName{\href{http://coord.info/GC7EGD0}{Calendrier de l'avent 17} — \href{http://coord.info/GC7EGD0\Number{}736144448}{1013}}\cacheData{{2017/12/02 denisb64, Traditional Cache (2/2.5)}}\begin{cacheText}Apres avoir déjeuné ,nous nous divisons en deux groupes. Nous avons la chance d'être avec les denisb64 qui nous donnent un coup de mains pour décoder les Mystères sorties la veille!!!Trouvée avec la Team Évent Noël  .Merci pour la cache.\end{cacheText}

\cacheNumber{1014}\needspace{5\baselineskip}\cacheName{\href{http://coord.info/GC7EGDE}{Calendrier de l'avent 18} — \href{http://coord.info/GC7EGDE\Number{}736143835}{1014}}\cacheData{{2017/12/02 denisb64, Traditional Cache (2/2)}}\begin{cacheText}Apres avoir déjeuné ,nous nous divisons en deux groupes. Nous avons la chance d'être avec les denisb64 qui nous donnent un coup de mains pour décoder les Mystères sorties la veille!!!Trouvée avec la Team Évent Noël. La belle est bien camouflée. Merci pour la cache.\end{cacheText}

\cacheNumber{1015}\needspace{5\baselineskip}\cacheName{\href{http://coord.info/GC7EGDY}{Calendrier de l'avent 19} — \href{http://coord.info/GC7EGDY\Number{}736143531}{1015}}\cacheData{{2017/12/02 denisb64, Traditional Cache (1.5/1.5)}}\begin{cacheText}Apres avoir déjeuné ,nous nous divisons en deux groupes. Nous avons la chance d'être avec les denisb64 qui nous donnent un coup de mains pour décoder les Mystères sorties la veille!!!Trouvée avec la Team Évent Noël  .Merci pour la cache.\end{cacheText}

\cacheNumber{1016}\needspace{5\baselineskip}\cacheName{\href{http://coord.info/GC7EGFN}{Calendrier de l'avent 7} — \href{http://coord.info/GC7EGFN\Number{}736396135}{1016}}\cacheData{{2017/12/02 denisb64, Unknown Cache (1.5/2.5)}}\begin{cacheText}Apres avoir déjeuné ,nous nous divisons en deux groupes. Nous avons la chance d'être avec les denisb64 qui nous donnent un coup de mains pour décoder les Mystères sorties la veille!!!Trouvée avec la Team Évent Noël .Cette cache nous réserve une petite surprise...c'est dans la boite!!!Merci pour la cache et son énigme sympa. Un PF bien mérité!!!\end{cacheText}

\cacheNumber{1017}\needspace{5\baselineskip}\cacheName{\href{http://coord.info/GC7EGME}{calendrier de l'avent 8} — \href{http://coord.info/GC7EGME\Number{}736391797}{1017}}\cacheData{{2017/12/02 denisb64, Unknown Cache (2/1.5)}}\begin{cacheText}Apres avoir déjeuné ,nous nous divisons en deux groupes. Nous avons la chance d'être avec les denisb64 qui nous donnent un coup de mains pour décoder les Mystères sorties la veille!!!Trouvée avec la Team Évent Noël .Pas de problème pour celle ci. Merci pour la cache et son énigme.\end{cacheText}

\cacheNumber{1018}\needspace{5\baselineskip}\cacheName{\href{http://coord.info/GC7EGMV}{calendrier de l'avent 9} — \href{http://coord.info/GC7EGMV\Number{}736391336}{1018}}\cacheData{{2017/12/02 denisb64, Unknown Cache (2.5/1.5)}}\begin{cacheText}Apres avoir déjeuné ,nous nous divisons en deux groupes. Nous avons la chance d'être avec les denisb64 qui nous donnent un coup de mains pour décoder les Mystères sorties la veille!!!Trouvée avec la Team Évent Noël .Trouvée mais elle est très bien intégrée... dur dur!!!!! Merci pour la cache et son énigme.\end{cacheText}

\cacheNumber{1019}\needspace{5\baselineskip}\cacheName{\href{http://coord.info/GC7EGN4}{calendrier de l'avent 10} — \href{http://coord.info/GC7EGN4\Number{}736390869}{1019}}\cacheData{{2017/12/02 denisb64, Unknown Cache (1.5/1.5)}}\begin{cacheText}Apres avoir déjeuné ,nous nous divisons en deux groupes. Nous avons la chance d'être avec les denisb64 qui nous donnent un coup de mains pour décoder les Mystères sorties la veille!!!Trouvée avec la Team Évent Noël .Ici il faut ouvrir l'œil!!!Merci pour la cache et son énigme !!!\end{cacheText}

\cacheNumber{1020}\needspace{5\baselineskip}\cacheName{\href{http://coord.info/GC7EGNF}{calendrier de l'avent 11} — \href{http://coord.info/GC7EGNF\Number{}736231500}{1020}}\cacheData{{2017/12/02 denisb64, Unknown Cache (4/2.5)}}\begin{cacheText}Apres avoir déjeuné ,nous nous divisons en deux groupes. Nous avons la chance d'être avec les denisb64 qui nous donnent un coup de mains pour décoder les Mystères sorties la veille!!!Trouvée avec la Team Évent Noël .Cette cache est parfaitement intégrée et nous a donné du fil à retordre!!!Un PF bien mérité. Merci\end{cacheText}

\cacheNumber{1021}\needspace{5\baselineskip}\cacheName{\href{http://coord.info/GC7EGPX}{Calendrier de l'avent 12} — \href{http://coord.info/GC7EGPX\Number{}736231143}{1021}}\cacheData{{2017/12/02 denisb64, Unknown Cache (2.5/1.5)}}\begin{cacheText}Apres avoir déjeuné ,nous nous divisons en deux groupes. Nous avons la chance d'être avec les denisb64 qui nous donnent un coup de mains pour décoder les Mystères sorties la veille!!!Trouvée avec la Team Évent Noël .Merci pour la cache\end{cacheText}

\cacheNumber{1022}\needspace{5\baselineskip}\cacheName{\href{http://coord.info/GC7EGQF}{calendrier de l'avent 13} — \href{http://coord.info/GC7EGQF\Number{}736229629}{1022}}\cacheData{{2017/12/02 denisb64, Unknown Cache (3/2)}}\begin{cacheText}Apres avoir déjeuné ,nous nous divisons en deux groupes. Nous avons la chance d'être avec les denisb64 qui nous donnent un coup de mains pour décoder les Mystères sorties la veille!!!Trouvée avec la Team Évent Noël .Merci pour tout ce travail de recherches et de pose.\end{cacheText}

\cacheNumber{1023}\needspace{5\baselineskip}\cacheName{\href{http://coord.info/GC7EGQQ}{calendrier de l'avent 20} — \href{http://coord.info/GC7EGQQ\Number{}736401928}{1023}}\cacheData{{2017/12/02 denisb64, Unknown Cache (2.5/1.5)}}\begin{cacheText}Apres avoir déjeuné ,nous nous divisons en deux groupes. Nous avons la chance d'être avec les denisb64 qui nous donnent un coup de mains pour décoder les Mystères sorties la veille!!!Trouvée avec la Team Évent Noël .Pas de difficulté pour déloger la belle, une fois l'énigme décodée. Nous terminons le parcours autour d'un gouter improvisé.Merci pour tout le travail et cette excellente journée.\end{cacheText}

\cacheNumber{1024}\needspace{5\baselineskip}\cacheName{\href{http://coord.info/GC7EGRJ}{calendrier de l'avent 21} — \href{http://coord.info/GC7EGRJ\Number{}736405576}{1024}}\cacheData{{2017/12/02 denisb64, Unknown Cache (2/1.5)}}\begin{cacheText}Apres avoir déjeuné ,nous nous divisons en deux groupes. Nous avons la chance d'être avec les denisb64 qui nous donnent un coup de mains pour décoder les Mystères sorties la veille!!!Trouvée avec la Team Évent Noël. Les nombreux chercheurs ont vite débusqué la cache après avoir résolu l'énigme. Merci pour la cache .\end{cacheText}

\cacheNumber{1025}\needspace{5\baselineskip}\cacheName{\href{http://coord.info/GC7EGRR}{calendrier de l'avent 22} — \href{http://coord.info/GC7EGRR\Number{}736404271}{1025}}\cacheData{{2017/12/02 denisb64, Unknown Cache (1.5/1.5)}}\begin{cacheText}Apres avoir déjeuné ,nous nous divisons en deux groupes. Nous avons la chance d'être avec les denisb64 qui nous donnent un coup de mains pour décoder les Mystères sorties la veille!!!Trouvée avec la Team Évent Noël. La belle ne résiste pas longtemps aux chercheurs....l'énigme a été plus longue à décoder!!!Merci pour tout le travail\end{cacheText}

\cacheNumber{1026}\needspace{5\baselineskip}\cacheName{\href{http://coord.info/GC7EGT1}{calendrier de l'avent 23} — \href{http://coord.info/GC7EGT1\Number{}736402457}{1026}}\cacheData{{2017/12/02 denisb64, Unknown Cache (2/1.5)}}\begin{cacheText}Apres avoir déjeuné ,nous nous divisons en deux groupes. Nous avons la chance d'être avec les denisb64 qui nous donnent un coup de mains pour décoder les Mystères sorties la veille!!!Trouvée avec la Team Évent Noël .La cache est très bien dissimulée mais elle ne nous a pas résistée longtemps!!!Apres avoir expliqué le principe du géocaching à une voisine nous reprenons la route. Merci pour la cache.\end{cacheText}

\cacheNumber{1027}\needspace{5\baselineskip}\cacheName{\href{http://coord.info/GC7EGTD}{calendrier de l'avent 24} — \href{http://coord.info/GC7EGTD\Number{}736407007}{1027}}\cacheData{{2017/12/02 denisb64, Unknown Cache (2/2)}}\begin{cacheText}Apres avoir déjeuné ,nous nous divisons en deux groupes. Nous avons la chance d'être avec les denisb64 qui nous donnent un coup de mains pour décoder les Mystères sorties la veille!!!Trouvée avec la Team Évent Noël. La cache est vite découverte par le groupe....Merci\end{cacheText}

\cacheNumber{1028}\needspace{5\baselineskip}\cacheName{\href{http://coord.info/GC7EZ9D}{en attendant noel} — \href{http://coord.info/GC7EZ9D\Number{}736142273}{1028}}\cacheData{{2017/12/02 denisb64, Event Cache (1.5/1.5)}}\begin{cacheText}Il faut être fou .....fou de géocaching pour pique niquer un 2 décembre avec un petit 2 au thermomètre!!!!!!Après avoir fait connaissance avec les nombreux chercheurs et partagé nos victuailles, nous sommes partis nous réchauffer avec les owners  sur ce superbe parcours du Calendrier de l' Avant qui nous a réservé bien des surprises. Un GRAND merci aux denisb64 pour l'organisation parfaite de cette journée.\end{cacheText}

\cacheNumber{1029}\needspace{5\baselineskip}\cacheName{\href{http://coord.info/GC2EKZ6}{ND du RUGBY} — \href{http://coord.info/GC2EKZ6\Number{}736930623}{1029}}\cacheData{{2017/12/09 MLF40, Traditional Cache (2/1.5)}}\begin{cacheText}Très bel endroit et très belle chapelle consacrée au rugby. La cache est bien camouflée et j'ai eu du mal à la trouver!!!! Merci MLF 40 pour cette très jolie découverte.\end{cacheText}

\cacheNumber{1030}\needspace{5\baselineskip}\cacheName{\href{http://coord.info/GC3A95M}{Le puits miraculeux de Saint Amand} — \href{http://coord.info/GC3A95M\Number{}736928653}{1030}}\cacheData{{2017/12/09 fdcdm, Traditional Cache (1.5/1.5)}}\begin{cacheText}Quelle belle surprise de découvrir ce lieu!!!L'indice nous guide tout droit à la belle.Merci pour la cache.\end{cacheText}

\cacheNumber{1031}\needspace{5\baselineskip}\cacheName{\href{http://coord.info/GC3BG45}{Avé Maria} — \href{http://coord.info/GC3BG45\Number{}736927389}{1031}}\cacheData{{2017/12/09 fdcdm, Traditional Cache (1.5/1.5)}}\begin{cacheText}En route pour Grenade sur l'Adour nous nous arrêtons pour dénicher la belle qui se dévoile à nous sans difficulté. Merci pour la découverte de cette sainte vierge.\end{cacheText}

\cacheNumber{1032}\needspace{5\baselineskip}\cacheName{\href{http://coord.info/GC3BKN3}{Le refuge des pécheurs} — \href{http://coord.info/GC3BKN3\Number{}736927637}{1032}}\cacheData{{2017/12/09 fdcdm, Traditional Cache (1.5/1.5)}}\begin{cacheText}Il était grand temps de venir dépoussiérer la Belle. C'est chose faite merci pour la découverte de notre Dame du refuge des pêcheurs.\end{cacheText}

\cacheNumber{1033}\needspace{5\baselineskip}\cacheName{\href{http://coord.info/GC3EKP6}{Avé Nanard} — \href{http://coord.info/GC3EKP6\Number{}736929249}{1033}}\cacheData{{2017/12/09 fdcdm, Traditional Cache (1.5/1.5)}}\begin{cacheText}La cache ou plutôt les caches sont rapidement trouvées!!! Elles nous attendaient  à la vue de tous, parterre.!!!Nous signons les 2 logbooks et camouflons les belles au sol.Merci pour la découverte du site.\end{cacheText}

\cacheNumber{1034}\needspace{5\baselineskip}\cacheName{\href{http://coord.info/GC4XK5R}{Passerelle \Quoted{LOU PIN DOU PEYRE}} — \href{http://coord.info/GC4XK5R\Number{}736913822}{1034}}\cacheData{{2017/12/09 minion40, Traditional Cache (1.5/2)}}\begin{cacheText}De passage dans la région nous en profitons pour faire quelques caches. Celle ci  est rapidement trouvée grâce à l'indice . Cette passerelle est parfaitement intégrée dans le paysage.Merci pour la cache\end{cacheText}

\cacheNumber{1035}\needspace{5\baselineskip}\cacheName{\href{http://coord.info/GC5FF0W}{Le Grand Moun} — \href{http://coord.info/GC5FF0W\Number{}736925407}{1035}}\cacheData{{2017/12/09 perenoel65, Traditional Cache (1/1)}}\begin{cacheText}Pas de difficulté pour cette cache: elle est bien en place. Merci pour la découverte du Grand Moun.\end{cacheText}

\cacheNumber{1036}\needspace{5\baselineskip}\cacheName{\href{http://coord.info/GC5RFC5}{Le couvent des Jacobins} — \href{http://coord.info/GC5RFC5\Number{}736917044}{1036}}\cacheData{{2017/12/09 vigneauetcie, Traditional Cache (1.5/1.5)}}\begin{cacheText}De passage dans la région nous découvrons un superbe couvent. La cache ne nous pose aucun problème. Merci pour la découverte de ce lieu magnifique chargé d'histoire.\end{cacheText}

\cacheNumber{1037}\needspace{5\baselineskip}\cacheName{\href{http://coord.info/GC5WTKD}{Le Passage Couvert} — \href{http://coord.info/GC5WTKD\Number{}736925242}{1037}}\cacheData{{2017/12/09 NATH40, Traditional Cache (1.5/1.5)}}\begin{cacheText}La cache  n'est pas trop compliquée… Il suffit de chercher au bon endroit .Merci pour la découverte de cette petite fontaine.\end{cacheText}

\cacheNumber{1038}\needspace{5\baselineskip}\cacheName{\href{http://coord.info/GC5WTMR}{Ancienne Manufacture Plumes et Duvet} — \href{http://coord.info/GC5WTMR\Number{}736919484}{1038}}\cacheData{{2017/12/09 NATH40, Traditional Cache (1.5/1.5)}}\begin{cacheText}Pas de difficulté pour trouver la belle:l'indice est très explicite. Le log book est humide car la boîte n'est pas étanche. Merci pour la découverte de l'ancienne manufacture de la famille Crabos, famille originaire de notre village.\end{cacheText}

\cacheNumber{1039}\needspace{5\baselineskip}\cacheName{\href{http://coord.info/GC63K5J}{Dolmen 2 du Grand Moun} — \href{http://coord.info/GC63K5J\Number{}736926013}{1039}}\cacheData{{2017/12/09 ManuMax, Traditional Cache (1.5/1.5)}}\begin{cacheText}La belle est trouvée sans difficulté car il n'y a pas grand monde à cette heure ci. Merci pour la cache.\end{cacheText}

\cacheNumber{1040}\needspace{5\baselineskip}\cacheName{\href{http://coord.info/GC48JWJ}{Collégiale d'Ibos} — \href{http://coord.info/GC48JWJ\Number{}738397573}{1040}}\cacheData{{2017/12/23 lokateo64, Traditional Cache (1.5/1.5)}}\begin{cacheText}De passage sur Tarbes nous nous arrêtons pour visiter cette superbe collégiale . Merci pour la cache.\end{cacheText}

\cacheNumber{1041}\needspace{5\baselineskip}\cacheName{\href{http://coord.info/GC5BC6G}{Eglise Saint-André de Cabanac} — \href{http://coord.info/GC5BC6G\Number{}738397390}{1041}}\cacheData{{2017/12/23 painpinette, Traditional Cache (1/1.5)}}\begin{cacheText}De passage sur Cabanac ,nous trouvons facilement la Belle grâce à l'indice.Merci Painpinette.\end{cacheText}

\cacheNumber{1042}\needspace{5\baselineskip}\cacheName{\href{http://coord.info/GC64NH6}{Bordères sur l'Echez} — \href{http://coord.info/GC64NH6\Number{}738398199}{1042}}\cacheData{{2017/12/23 sc65, Traditional Cache (1.5/1)}}\begin{cacheText}De passage dans la région nous en profitons pour faire quelques caches.La belle est bien cachée mais elle ne nous résiste pas longtemps.Il n'y a pas de boite ,juste une pochette en plastique.Merci\end{cacheText}

\cacheNumber{1043}\needspace{5\baselineskip}\cacheName{\href{http://coord.info/GC6KFEA}{L ECHEZ / BORDERES} — \href{http://coord.info/GC6KFEA\Number{}738475477}{1043}}\cacheData{{2017/12/23 jb65, Traditional Cache (1/1.5)}}\begin{cacheText}De passage sur Tarbes nous décidons de dénicher quelques  caches dans les environs.L'indice nous mène tout droit à la Belle. Merci pour la découverte de l'Echez.\end{cacheText}

\cacheNumber{1044}\needspace{5\baselineskip}\cacheName{\href{http://coord.info/GC6KJPK}{Transit TB (retour maison) ;)))} — \href{http://coord.info/GC6KJPK\Number{}738476064}{1044}}\cacheData{{2017/12/23 jb65vic, Traditional Cache (1/1.5)}}\begin{cacheText}Une cache dans un lotissement ce n'est pas facile à trouver discrètement mais heureusement pour nous il n' y a personne dans les parages !!!La cache est vite débusquée.Attention à la tête en se relevant!!!! Merci pour la cache.\end{cacheText}

\cacheNumber{1045}\needspace{5\baselineskip}\cacheName{\href{http://coord.info/GC6NG0X}{Geoshopping-Ibos} — \href{http://coord.info/GC6NG0X\Number{}738397472}{1045}}\cacheData{{2017/12/23 Bigorra65, Unknown Cache (1.5/1.5)}}\begin{cacheText}De passage sur Tarbes nous décidons de manger à la Pataterie. C'est un savourant notre repas que nous commençons à décrypter l'énigme. À la sortie, nous allons chercher les dernières informations manquantes. Les calculs sont vite faits et nous délogeons la belle en deux temps trois mouvements. Merci Bigorra65 pour cette cache.\end{cacheText}

\cacheNumber{1046}\needspace{5\baselineskip}\cacheName{\href{http://coord.info/GC6T2R7}{d'ousbelille à borderes \Number{}7} — \href{http://coord.info/GC6T2R7\Number{}738479974}{1046}}\cacheData{{2017/12/23 gobfamilly, Traditional Cache (2/1.5)}}\begin{cacheText}Ici pas de difficultés particulières, nous délogeons vite fait la Belle. Merci pour la cache.\end{cacheText}

\cacheNumber{1047}\needspace{5\baselineskip}\cacheName{\href{http://coord.info/GC6Y1E3}{Mystère à Bordères} — \href{http://coord.info/GC6Y1E3\Number{}738398023}{1047}}\cacheData{{2017/12/23 l.a.f.l.e.u.r, Unknown Cache (1.5/1.5)}}\begin{cacheText}Chose étonnante,cette mystère a été vite résolue !!!De passage sur Tarbes nous faisons un crochet pour déloger la Belle.Nous devons patienter un peu car il y a quelques moldus dans les parages mais une fois le champ libre nous la découvrons vite.Merci\end{cacheText}

\cacheNumber{1048}\needspace{5\baselineskip}\cacheName{\href{http://coord.info/GC721FM}{La croix de Monjolhe} — \href{http://coord.info/GC721FM\Number{}738397728}{1048}}\cacheData{{2017/12/23 larrylqx, Traditional Cache (2/2)}}\begin{cacheText}Arrivés sur les lieux, nous devons attendre que les promeneurs passent...Nous trouvons la Belle bien intégrée au paysage.Merci pour la cache\end{cacheText}

\cacheNumber{1049}\needspace{5\baselineskip}\cacheName{\href{http://coord.info/GC74TGZ}{presque drive in \Number{}5 gobfamilly} — \href{http://coord.info/GC74TGZ\Number{}738478183}{1049}}\cacheData{{2017/12/23 gobfamilly, Traditional Cache (1.5/1.5)}}\begin{cacheText}Très belle réalisation sur ce rond-point. Cette cache nous a donné un peu de fil à retordre mais nous avons fini par la trouver. Merci pour cette decouverte\end{cacheText}

\cacheNumber{1050}\needspace{5\baselineskip}\cacheName{\href{http://coord.info/GC75WJ6}{presque drive in \Number{}3 gobfamilly} — \href{http://coord.info/GC75WJ6\Number{}738480055}{1050}}\cacheData{{2017/12/23 gobfamilly, Traditional Cache (1.5/1)}}\begin{cacheText}En ce samedi après-midi heureusement pour nous il n'y a pas de match. Quelques enfants jouent sur le parking mais ne nous prête aucune attention. Nous sommes tranquilles pour déloger la belle. Merci pour la cache\end{cacheText}

\cacheNumber{1051}\needspace{5\baselineskip}\cacheName{\href{http://coord.info/GC76BVW}{presque drive in \Number{}6 gobfamilly} — \href{http://coord.info/GC76BVW\Number{}738397808}{1051}}\cacheData{{2017/12/23 gobfamilly, Traditional Cache (2/1.5)}}\begin{cacheText}\Quoted{La Croix guidera} effectivement !!!!Mais laquelle??? Que de vices ..... mais nous adorons. Mplc\end{cacheText}

\cacheNumber{1052}\needspace{5\baselineskip}\cacheName{\href{http://coord.info/GC22B1A}{Illumination pour Betty} — \href{http://coord.info/GC22B1A\Number{}739386269}{1052}}\cacheData{{2017/12/28 escornecrabe, Multi-cache (2/1.5)}}\begin{cacheText}Après l'Event Petit Restau de fin d'année, nous partons en compagnie de Domino50, Dune33 et JB65Vic à la recherche de la Cache sous la pluie. Après avoir plusieurs fois compté les pales ,nous tombons d'accord et partons loguer la Belle. Merci pour la découverte de ce lieu qui est superbe de nuit.Un PF évidemment.\end{cacheText}

\cacheNumber{1053}\needspace{5\baselineskip}\cacheName{\href{http://coord.info/GC7FYW6}{Petit restau de fin d'année ;-)} — \href{http://coord.info/GC7FYW6\Number{}739385980}{1053}}\cacheData{{2017/12/28 Vincelo ❤ ❤ ❤, Event Cache (1/1)}}\begin{cacheText}Merci Elodie et Vincent pour ce sympathique Évent.Nous avons eu grand plaisir à retrouver des connaissances et faire de nouvelles rencontres. Nous avons très bien mangé et avons passé une excellente soirée...Au plaisir de vous revoir\end{cacheText}

\cacheNumber{1054}\needspace{5\baselineskip}\cacheName{\href{http://coord.info/GC2DJ7V}{Grand Hotel de Lourdes - TB Hotel} — \href{http://coord.info/GC2DJ7V\Number{}739387946}{1054}}\cacheData{{2017/12/29 Vinceo, Traditional Cache (2/2)}}\begin{cacheText}C'est en rentrant d'Argeles Gazost que nous décidons de découvrir cette cache.Les coordonnées , précises, nous mènent tout droit à la Belle qui malheureusement ne contient aucun TB.Merci Vinceo pour la cache.\end{cacheText}

\cacheNumber{1055}\needspace{5\baselineskip}\cacheName{\href{http://coord.info/GC3TKE8}{La tour du Garnavie} — \href{http://coord.info/GC3TKE8\Number{}739458954}{1055}}\cacheData{{2017/12/29 VinceO, Traditional Cache (2/1.5)}}\begin{cacheText}Arrivés au PZ en compagnie de domino50, nous découvrons cette superbe tour. La cache est vite localisée et constatons qu'il n' y a plus de logbook mais un simple bout de papier publicitaire. Il pleut des cordes et le nécessaire de dépannage est resté à l'abri dans la voiture. La cache est sympa , joli travail.....Merci\end{cacheText}

\cacheNumber{1056}\needspace{5\baselineskip}\cacheName{\href{http://coord.info/GC443YM}{Event SdG GR653 M\Underscore{}L-25} — \href{http://coord.info/GC443YM\Number{}739605212}{1056}}\cacheData{{2017/12/29 Charnègues, Traditional Cache (1.5/1.5)}}\begin{cacheText}Nous suivons notre parcours et celle ci sera la dernière de la journée .Pas de difficultés pour celle-ci. Merci pour la cache.\end{cacheText}

\cacheNumber{1057}\needspace{5\baselineskip}\cacheName{\href{http://coord.info/GC443Z8}{Event SdG GR653 M\Underscore{}L-26} — \href{http://coord.info/GC443Z8\Number{}739604563}{1057}}\cacheData{{2017/12/29 Charnègues, Traditional Cache (1.5/1.5)}}\begin{cacheText}Malgré le monde qui sort de l'hippodrome nous arrivons à loguer discrètement. Malheureusement, le log book est trempé. Il aurait besoin d'une maintenance rapide. Merci pour la cache.\end{cacheText}

\cacheNumber{1058}\needspace{5\baselineskip}\cacheName{\href{http://coord.info/GC443ZR}{Event SdG GR653 M\Underscore{}L-27} — \href{http://coord.info/GC443ZR\Number{}739604282}{1058}}\cacheData{{2017/12/29 Charnègues, Traditional Cache (1.5/1.5)}}\begin{cacheText}Arrivés au PZ, la photo et le GPS  nous mènent tout droit à la cache. Merci\end{cacheText}

\cacheNumber{1059}\needspace{5\baselineskip}\cacheName{\href{http://coord.info/GC44407}{Event SdG GR653 M\Underscore{}L-28} — \href{http://coord.info/GC44407\Number{}739600666}{1059}}\cacheData{{2017/12/29 Charnègues, Traditional Cache (1.5/1.5)}}\begin{cacheText}Ici, toute la difficulté est de loguer discrètement. Il y a énormément de passage mais nous arrivons à loguer. Merci pour la cache\end{cacheText}

\cacheNumber{1060}\needspace{5\baselineskip}\cacheName{\href{http://coord.info/GC4440M}{Event SdG GR653 M\Underscore{}L-29} — \href{http://coord.info/GC4440M\Number{}739600292}{1060}}\cacheData{{2017/12/29 Charnègues, Traditional Cache (1.5/1.5)}}\begin{cacheText}Ici aussi la cache nous attend bien sagement à sa place. Pas trop de circulation nous pouvons loguer tranquillement. Merci\end{cacheText}

\cacheNumber{1061}\needspace{5\baselineskip}\cacheName{\href{http://coord.info/GC44418}{Event SdG GR653 M\Underscore{}L-30} — \href{http://coord.info/GC44418\Number{}739599884}{1061}}\cacheData{{2017/12/29 Charnègues, Traditional Cache (1.5/1.5)}}\begin{cacheText}Le logbook est bien en place. Il faut bien chercher... le spoiler est excellent. Merci pour la cache\end{cacheText}

\cacheNumber{1062}\needspace{5\baselineskip}\cacheName{\href{http://coord.info/GC4441Q}{Event SdG GR653 M\Underscore{}L-31} — \href{http://coord.info/GC4441Q\Number{}739599427}{1062}}\cacheData{{2017/12/29 Charnègues, Traditional Cache (1.5/1.5)}}\begin{cacheText}La cache est vite repérée grâce à l'indice. Merci pour ce joli circuit.\end{cacheText}

\cacheNumber{1063}\needspace{5\baselineskip}\cacheName{\href{http://coord.info/GC4442C}{Event SdG GR653 M\Underscore{}L-32} — \href{http://coord.info/GC4442C\Number{}739598901}{1063}}\cacheData{{2017/12/29 Charnègues, Traditional Cache (1.5/2)}}\begin{cacheText}Après cet épisode pluvieux nous rencontrons quelques petites difficultés pour accéder à la cache… Les fossés sont gorgés d'eau mais nous réussissons à sauter. Merci pour la cache\end{cacheText}

\cacheNumber{1064}\needspace{5\baselineskip}\cacheName{\href{http://coord.info/GC4444E}{Event SdG GR653 M\Underscore{}L-33} — \href{http://coord.info/GC4444E\Number{}739598208}{1064}}\cacheData{{2017/12/29 Charnègues, Traditional Cache (1.5/1.5)}}\begin{cacheText}La cache est rapidement trouvée grâce à l'indice. Merci\end{cacheText}

\cacheNumber{1065}\needspace{5\baselineskip}\cacheName{\href{http://coord.info/GC4444Z}{Event SdG GR653 M\Underscore{}L-34} — \href{http://coord.info/GC4444Z\Number{}739597772}{1065}}\cacheData{{2017/12/29 Charnègues, Traditional Cache (1.5/1.5)}}\begin{cacheText}La Belle nous attend bien à sa place. Personne à l'horizon...Nous pouvons signer tranquillement.Merci\end{cacheText}

\cacheNumber{1066}\needspace{5\baselineskip}\cacheName{\href{http://coord.info/GC4445C}{Event SdG GR653 M\Underscore{}L-35} — \href{http://coord.info/GC4445C\Number{}739597128}{1066}}\cacheData{{2017/12/29 Charnègues, Traditional Cache (1.5/1.5)}}\begin{cacheText}L' indice ,très explicite, nous mène tout droit à la cache. Merci pour ce bon moment.\end{cacheText}

\cacheNumber{1067}\needspace{5\baselineskip}\cacheName{\href{http://coord.info/GC4445E}{Event SdG GR653 M\Underscore{}L-36} — \href{http://coord.info/GC4445E\Number{}739596786}{1067}}\cacheData{{2017/12/29 Charnègues, Traditional Cache (1.5/1.5)}}\begin{cacheText}Arrivés sur les lieux , nous trouvons très rapidement la cache qui est en bon état. Nous ne trainons pas car le site est envahi par les chenilles urticantes.Merci pour la cache.\end{cacheText}

\cacheNumber{1068}\needspace{5\baselineskip}\cacheName{\href{http://coord.info/GC5F5M1}{Le boulevard de  la grotte} — \href{http://coord.info/GC5F5M1\Number{}739459731}{1068}}\cacheData{{2017/12/29 raphio, Traditional Cache (2/1.5)}}\begin{cacheText}Malgré 2 DNF nous décidons, avec Domino50, de rejoindre le PZ....qui ne tente rien n'a rien!!!!!BINGO... la Belle nous attend à sa place. Merci raphio\end{cacheText}

\cacheNumber{1069}\needspace{5\baselineskip}\cacheName{\href{http://coord.info/GC696CG}{\Quoted{Le garde barrière}} — \href{http://coord.info/GC696CG\Number{}739402547}{1069}}\cacheData{{2017/12/29 raphio, Traditional Cache (1.5/1)}}\begin{cacheText}Arrivée au PZ ,pendant que monsieur gare la géomobile, je rencontre domino 50 qui s’apprête à repartir.Cache logée ,nous décidons de faire quelques caches ensemble.Merci pour la cache.\end{cacheText}

\cacheNumber{1070}\needspace{5\baselineskip}\cacheName{\href{http://coord.info/GC6AAN8}{Une cache pour Coxigypaete} — \href{http://coord.info/GC6AAN8\Number{}739601358}{1070}}\cacheData{{2017/12/29 l.a.f.l.e.u.r, Traditional Cache (1/1.5)}}\begin{cacheText}De passage sur Pau nous nous attelons à faire quelques caches. Celle ci est vite repérée car l' indice nous a bien guidé. Merci pour la cache.\end{cacheText}

\cacheNumber{1071}\needspace{5\baselineskip}\cacheName{\href{http://coord.info/GC73YCH}{\Quoted{Mini Multi Lourdaise}} — \href{http://coord.info/GC73YCH\Number{}739459618}{1071}}\cacheData{{2017/12/29 raphio, Multi-cache (1.5/1.5)}}\begin{cacheText}En compagnie de domino 50 nous nous attaquons à cette petite multi lourdaise sous la pluie. Les indices se laissent relever facilement sauf celui de l'aigle et la truite!!!Impossible de mettre la main sur cette date!!!Grace au coup de fil à un ami (jb65vic) nous finissons par avoir les coordonnées finales. Arrivés au PZ ,domino50 qui a compris l'indice cherche mais ne trouve rien!!!! Nous avons eu la chance de trouver la cache parterre!!!!Une petite maintenance s'imposait....La cache est à nouveau en place. Merci pour cette super promenade.\end{cacheText}

\cacheNumber{1072}\needspace{5\baselineskip}\cacheName{\href{http://coord.info/GC7EEKP}{Le Lavoir de Pontacq} — \href{http://coord.info/GC7EEKP\Number{}739459760}{1072}}\cacheData{{2017/12/29 famchok, Traditional Cache (2/2)}}\begin{cacheText}C'est au retour de Lourdes que nous nous arrêtons loguer cette cache. Pas de difficulté. Très jolie lavoir. Merci pour la découverte.\end{cacheText}

\cacheNumber{1073}\needspace{5\baselineskip}\cacheName{\href{http://coord.info/GC71VYF}{[GTAQ26]01-Sort en Chalosse-L'aire de jeux} — \href{http://coord.info/GC71VYF\Number{}739614858}{1073}}\cacheData{{2017/12/30 Opmb40, Traditional Cache (1.5/3.5)}}\begin{cacheText}C'est un mode Boot Camp que Monsieur va chercher la belle. Super cache. Merci pour la découverte de ce joli endroit.\end{cacheText}

\cacheNumber{1074}\needspace{5\baselineskip}\cacheName{\href{http://coord.info/GC71VYP}{[GTAQ26]02-Sort en Chalosse-Les Palmipèdes} — \href{http://coord.info/GC71VYP\Number{}739617667}{1074}}\cacheData{{2017/12/30 Opmb40, Traditional Cache (3/1.5)}}\begin{cacheText}Nous passons une première fois : l'éleveur est en train de donner à manger à ses palmipèdes. Second passage, la voie est libre .Après avoir cherché un bon moment nous finissons par débusquer la Belle et c'est encore une cache originale. Merci\end{cacheText}

\cacheNumber{1075}\needspace{5\baselineskip}\cacheName{\href{http://coord.info/GC71VZ5}{[GTAQ26]03-Sort en Chalosse-Une pour la route} — \href{http://coord.info/GC71VZ5\Number{}739617123}{1075}}\cacheData{{2017/12/30 Opmb40, Traditional Cache (1.5/1.5)}}\begin{cacheText}Nous trouvons ici une cache qui ne nous pose aucun problème… Assez rare sur le circuit!!!Mais nous adorons!!!Le log book est un peu humide. Merci pour la cache\end{cacheText}

\cacheNumber{1076}\needspace{5\baselineskip}\cacheName{\href{http://coord.info/GC71W0E}{[GTAQ26]04-Sort en Chalosse-Le baradeau} — \href{http://coord.info/GC71W0E\Number{}739616647}{1076}}\cacheData{{2017/12/30 Opmb40, Traditional Cache (2.5/2)}}\begin{cacheText}Arrivés sur les lieux on tourne, on vire à droite, à gauche: le GPS nous promène. Sur le point d'abandonner, l'œil est attiré par… Bravo pour la cache ,du geocaching comme nous l'aimons. Il n'y a plus de logbook:je le remplace et je le met dans un petit sachet. Merci\end{cacheText}

\cacheNumber{1077}\needspace{5\baselineskip}\cacheName{\href{http://coord.info/GC71WBM}{[GTAQ26]06-Sort en Chalosse-Les gallinacés} — \href{http://coord.info/GC71WBM\Number{}739618187}{1077}}\cacheData{{2017/12/30 Opmb40, Traditional Cache (1.5/1.5)}}\begin{cacheText}Ici la cache est repérée en 2 temps 3 mouvements. Il n'y a pas de gallinacés en ce jour d'hiver. Merci pour la cache\end{cacheText}

\cacheNumber{1078}\needspace{5\baselineskip}\cacheName{\href{http://coord.info/GC71XCY}{[GTAQ26]14-Sort en Chalosse-Thuja occidentalis} — \href{http://coord.info/GC71XCY\Number{}739620449}{1078}}\cacheData{{2017/12/30 Opmb40, Traditional Cache (2/1.5)}}\begin{cacheText}Arrivés au PZ, nous avons inspecté les Thujas occidentalis un par un. Jusqu'à ce que tout d'un coup l'indice prenne tout son sens. BINGO... quelle superbe cache ?un vrai régal. Très bonne idée. Merci pour la cache et un PF évidemment\end{cacheText}

\cacheNumber{1079}\needspace{5\baselineskip}\cacheName{\href{http://coord.info/GC71XDA}{[GTAQ26]15-Sort en Chalosse-Une pour la route 3} — \href{http://coord.info/GC71XDA\Number{}739625819}{1079}}\cacheData{{2017/12/30 Opmb40, Traditional Cache (1.5/1.5)}}\begin{cacheText}Ici la cache est très rapidement trouvée par mon frère ,moldu, qui s'est joint à nous. Il y est allé en courant!!! Nous avons frotté la lampe et Aladdin est apparu. Merci pour la cache.\end{cacheText}

\cacheNumber{1080}\needspace{5\baselineskip}\cacheName{\href{http://coord.info/GC71XDN}{[GTAQ26]16-Sort en Chalosse-Le bassin d'irrigation} — \href{http://coord.info/GC71XDN\Number{}739627564}{1080}}\cacheData{{2017/12/30 Opmb40, Traditional Cache (2.5/2)}}\begin{cacheText}Arrivés au PZ ,nous cherchons dans tous les coins et recoins… Rien!!!Mais ce \Quoted{tut tut} nous intrigue .Tous ces fils électriques ne sont pas très rassurants, nous sommes sur le point de repartir et bingo on met la main dessus.Il fallait oser!!!Merci pour cette super cache. Pas de photo car la nuit est tombée , il est temps de rentrer. Merci pour cette super cache et un PF\end{cacheText}

\cacheNumber{1081}\needspace{5\baselineskip}\cacheName{\href{http://coord.info/GC71XEG}{[GTAQ26]20-Sort en Chalosse-Fiat lux} — \href{http://coord.info/GC71XEG\Number{}739614189}{1081}}\cacheData{{2017/12/30 Opmb40, Traditional Cache (1.5/1.5)}}\begin{cacheText}Arrivés sur le PZ,la belle se laisse débusquer assez rapidement. Elle est parfaitement intégrée. Bravo pour le travail et merci .\end{cacheText}

\cacheNumber{1082}\needspace{5\baselineskip}\cacheName{\href{http://coord.info/GC71Z64}{[GTAQ26]23-Sort en Chalosse-La côte de la Banère} — \href{http://coord.info/GC71Z64\Number{}739608939}{1082}}\cacheData{{2017/12/30 Opmb40, Traditional Cache (2.5/1)}}\begin{cacheText}Nous avions fait une cache (19) en rentrant du GTAQ le dimanche en fin d'après midi et nous étions trop fatigués pour continuer. Mais ce smiley nous fait de l'œil depuis!!! Arrivés sur le PZ, nous trouvons un endroit de cache très original. Nous signons entre 2 passages de voiture. Merci pour cette cache.\end{cacheText}

\cacheNumber{1083}\needspace{5\baselineskip}\cacheName{\href{http://coord.info/GC71ZEG}{[GTAQ26]21-Sort en Chalosse-Notre Dame des Champs} — \href{http://coord.info/GC71ZEG\Number{}739624451}{1083}}\cacheData{{2017/12/30 Opmb40, Traditional Cache (3/1)}}\begin{cacheText}Quelle aventure!!! Il nous en fallu du temps et de la persévérance pour la déloger!!!Nous étions sur le PZ depuis un bon moment entrain de virer et de tourner lorsqu'une voiture de gendarmerie s'est arrêtée pour nous demander ce que nous faisions!!!Apres avoir expliqué le principe du jeu ils ont également cherché avec nous mais... RIEN. Restés seuls sur le PZ nous allions renoncer lorsque tout à coup...BINGO  le point G!!!Merci Opmb40 pour cet excellent moment. Un PF évidemment\end{cacheText}

\cacheNumber{1084}\needspace{5\baselineskip}\cacheName{\href{http://coord.info/GC3Z0FD}{Péristyle de Villa Quieta} — \href{http://coord.info/GC3Z0FD\Number{}739874174}{1084}}\cacheData{{2017/12/31 gilles64, Traditional Cache (1.5/1.5)}}\begin{cacheText}Nous avons fini par mettre la main dessus grâce à un commentaire précédent. En effet ,les coordonnées ne sont pas très précises. Merci pour la découverte de ces vestiges.\end{cacheText}

\cacheNumber{1085}\needspace{5\baselineskip}\cacheName{\href{http://coord.info/GC4R8W2}{Hiriburu : Le Renard Rale, ses Dragons Veillent} — \href{http://coord.info/GC4R8W2\Number{}739870690}{1085}}\cacheData{{2017/12/31 Gamboy, Traditional Cache (2/1.5)}}\begin{cacheText}Arrivés sur le PZ nous découvrons deux jolies réalisations. Après avoir lu l'indice nous nous dirigeons vers la belle que nous dénichons en deux temps trois mouvements. Le log book a beaucoup souffert mais nous parvenons à signer. Merci les Gamboy pour cette jolie cache.\end{cacheText}

\cacheNumber{1086}\needspace{5\baselineskip}\cacheName{\href{http://coord.info/GC4R8WW}{Hiriburu : La Maison du Caramel} — \href{http://coord.info/GC4R8WW\Number{}739872038}{1086}}\cacheData{{2017/12/31 Gamboy, Traditional Cache (1.5/1.5)}}\begin{cacheText}Nous trouvons très facilement la cache mais malheureusement le log book est détrempé .Nous le remplaçons donc par un papier de substitution pour les suivants. Merci pour la découverte de ce lieu très insolite.\end{cacheText}

\cacheNumber{1087}\needspace{5\baselineskip}\cacheName{\href{http://coord.info/GC4RA15}{Hiriburu : Tissage-Tulipe} — \href{http://coord.info/GC4RA15\Number{}739874948}{1087}}\cacheData{{2017/12/31 Gamboy, Traditional Cache (1.5/1.5)}}\begin{cacheText}L'indice prend tout son sens quand on arrive sur les lieux. Elle n'est pas très accessible aux petites personnes mais nous arrivons tout de même ,discrètement , malgré le passage ,à loguer. Merci pour cette cache\end{cacheText}

\cacheNumber{1088}\needspace{5\baselineskip}\cacheName{\href{http://coord.info/GC4RA1H}{Hiriburu : La Tour Vaurien} — \href{http://coord.info/GC4RA1H\Number{}739873803}{1088}}\cacheData{{2017/12/31 Gamboy, Traditional Cache (1.5/2)}}\begin{cacheText}Quel joli lavoir nous trouvons ici!!!L'indice nous mène tout droit à la cache. Merci pour cette jolie découverte\end{cacheText}

\cacheNumber{1089}\needspace{5\baselineskip}\cacheName{\href{http://coord.info/GC4YQ7P}{Hiriburu : J'empierre} — \href{http://coord.info/GC4YQ7P\Number{}739878285}{1089}}\cacheData{{2017/12/31 Gamboy, Traditional Cache (1.5/1.5)}}\begin{cacheText}Celle-ci nous a donné du mal !!!Après avoir retourné toutes les pierres ou presque et s'être éloignés du PZ Monsieur fini par mettre la main dessus. Ouffff il était temps car la pluie arrive. Merci pour ce bon moment\end{cacheText}

\cacheNumber{1090}\needspace{5\baselineskip}\cacheName{\href{http://coord.info/GC50XBW}{Hiriburu : Val Italique} — \href{http://coord.info/GC50XBW\Number{}739875519}{1090}}\cacheData{{2017/12/31 Gamboy, Traditional Cache (2/1.5)}}\begin{cacheText}Arrivés sur le site il n'y a pas 50 endroitspour pouvoir camoufler la belle. Rapidement trouvée ,nous allons voir cette jolie chapelle perdue au milieu des immeubles. Merci pour la découverte.\end{cacheText}

\cacheNumber{1091}\needspace{5\baselineskip}\cacheName{\href{http://coord.info/GC6526C}{Älgen Ulf} — \href{http://coord.info/GC6526C\Number{}739753843}{1091}}\cacheData{{2017/12/31 dorisbear, Unknown Cache (2/1.5)}}\begin{cacheText}Après avoir résolu cette petite mystère, nous voici, en ce 31 décembre ,sur le terrain pour déloger la belle. Personne dans les parages aujourd'hui... Un moment d'hésitation puis un détail nous confirme la cache  Encore une magnifique réalisation de nos \Quoted{ours}préférés. Merci pour ce bon moment qui nous permet de gagner le souvenir.\end{cacheText}

\cacheNumber{1092}\needspace{5\baselineskip}\cacheName{\href{http://coord.info/GC76TBC}{ROUTE DES CIMES} — \href{http://coord.info/GC76TBC\Number{}739871296}{1092}}\cacheData{{2017/12/31 Graftitou64, Traditional Cache (1/1)}}\begin{cacheText}La cache est rapidement trouvée malgré la circulation. Le log book est composé de post it… Curieux ! Merci pour la cache.\end{cacheText}

\cacheNumber{1093}\needspace{5\baselineskip}\cacheName{\href{http://coord.info/GC7F6N3}{Ongi Etorri} — \href{http://coord.info/GC7F6N3\Number{}739869804}{1093}}\cacheData{{2017/12/31 DorisBear, Traditional Cache (1.5/1.5)}}\begin{cacheText}Nous continuons notre quête, en ce jour du 31 décembre, et nous débusquons sans trop de difficulté cette cache. Il n'y a pas trop de circulation heureusement!!!Compliqué d' être discret. Merci les Doris Bear pour cette cache.\end{cacheText}

\cacheNumber{1094}\needspace{5\baselineskip}\cacheName{\href{http://coord.info/GC2WPWM}{Les vignes Royales} — \href{http://coord.info/GC2WPWM\Number{}740139538}{1094}}\cacheData{{2018/01/01 lokateo64, Traditional Cache (1.5/1.5)}}\begin{cacheText}Il m'a fallu attendre la fin d'après midi pour espérer déloger la Belle et gagner le souvenir. C'est chose faite , la cache était très accessible. Merci pour la découverte des vignes et du château.\end{cacheText}

\cacheNumber{1095}\needspace{5\baselineskip}\cacheName{\href{http://coord.info/GC50DG6}{St Pé - Le Chistera} — \href{http://coord.info/GC50DG6\Number{}741419163}{1095}}\cacheData{{2018/01/08 gilles64, Traditional Cache (2/1.5)}}\begin{cacheText}De passage dans le coin ,on ne peut pas résister à la tentation. La cache est bien camouflée mais on finit par mettre la main dessus. Le chistera est magnifique.Merci\end{cacheText}

\cacheNumber{1096}\needspace{5\baselineskip}\cacheName{\href{http://coord.info/GC5CZ6X}{Pique-nique au quartier Urguri} — \href{http://coord.info/GC5CZ6X\Number{}741423012}{1096}}\cacheData{{2018/01/08 brunolli, Traditional Cache (1.5/1.5)}}\begin{cacheText}De passage dans le coin, nous repérons ce point vert sur la carte. Nous découvrons un endroit charmant dans lequel coule un joli ruisseau. La Belle est vite repérée et le smiley apparaît sur nos visages et sur la carte. Merci pour la cache.\end{cacheText}

\cacheNumber{1097}\needspace{5\baselineskip}\cacheName{\href{http://coord.info/GC6VYN5}{I010	 - Géodyssée 40 64} — \href{http://coord.info/GC6VYN5\Number{}741829641}{1097}}\cacheData{{2018/01/12 gilles64, Traditional Cache (1.5/1.5)}}\begin{cacheText}Et une autre au palmarès!!!!Pas de difficulté grâce à l'indice ....merci pour la cache.\end{cacheText}

\cacheNumber{1098}\needspace{5\baselineskip}\cacheName{\href{http://coord.info/GC6VYN9}{I011	 - Géodyssée 40 64} — \href{http://coord.info/GC6VYN9\Number{}741829448}{1098}}\cacheData{{2018/01/12 gilles64, Traditional Cache (1.5/1.5)}}\begin{cacheText}Ici la cache est toujours en place.Nous laissons passer les marcheurs et loguons la cache en deux temps trois mouvements .Merci\end{cacheText}

\cacheNumber{1099}\needspace{5\baselineskip}\cacheName{\href{http://coord.info/GC6VYNB}{I012	 - Géodyssée 40 64} — \href{http://coord.info/GC6VYNB\Number{}741828628}{1099}}\cacheData{{2018/01/12 gilles64, Traditional Cache (1.5/1.5)}}\begin{cacheText}Les travaux forestiers ont eu raison de la cache. Nous avons récupéré l'ancien log book et mis en place une nouvelle cache les coordonnées sont les suivantes:N 43°34,035 O 001°28,213. Nouveau spoiler proposé. Nous continuons la quête.Merci\end{cacheText}

\cacheNumber{1100}\needspace{5\baselineskip}\cacheName{\href{http://coord.info/GC6VYNR}{I013	 - Géodyssée 40 64} — \href{http://coord.info/GC6VYNR\Number{}741824498}{1100}}\cacheData{{2018/01/12 gilles64, Traditional Cache (1.5/1.5)}}\begin{cacheText}Nous profitons de cet après-midi ensoleillé pour faire quelques cache de la Géodyssée. Celle-ci est trouvée sans aucune difficulté. Ce parcours est vraiment très agréable. Merci pour la cache\end{cacheText}

\cacheNumber{1101}\needspace{5\baselineskip}\cacheName{\href{http://coord.info/GC6VYNW}{I014	 - Géodyssée 40 64} — \href{http://coord.info/GC6VYNW\Number{}741824152}{1101}}\cacheData{{2018/01/12 gilles64, Traditional Cache (1.5/1.5)}}\begin{cacheText}Dans le secteur pour trouver la cache du cadran solaire de Tarnos nous décidons de faire quelques caches supplémentaires.Celle ci est loguée rapidement entre le passage de deux voitures.Les coordonnées précises et le spoiler nous ont menés tout droit à la cache .Merci\end{cacheText}

\cacheNumber{1102}\needspace{5\baselineskip}\cacheName{\href{http://coord.info/GC6WR85}{Forêt d'Ondres} — \href{http://coord.info/GC6WR85\Number{}741835099}{1102}}\cacheData{{2018/01/12 ilonaellaaudrey, Traditional Cache (3/2)}}\begin{cacheText}Après avoir cherché dans la zone pendant un bon petit moment,le GPS nous promène, nous finissons par mettre la main dessus. C'est une jolie boite à bidules qui malheureusement ne contient aucun TB. Dommage.!!!Merci pour la cache\end{cacheText}

\cacheNumber{1103}\needspace{5\baselineskip}\cacheName{\href{http://coord.info/GC6X7MB}{Une pour le Cardinal} — \href{http://coord.info/GC6X7MB\Number{}741834932}{1103}}\cacheData{{2018/01/12 Gamboy, Traditional Cache (1.5/1.5)}}\begin{cacheText}Par ce bel après-midi ensoleillé nous découvrons cette petite cache.Une boite à bidules comme nous les aimons.Pas de moldu à l'horizon.... nous loguons tranquillement Merci\end{cacheText}

\cacheNumber{1104}\needspace{5\baselineskip}\cacheName{\href{http://coord.info/GC7DDCD}{Quel heure est il ??} — \href{http://coord.info/GC7DDCD\Number{}741823198}{1104}}\cacheData{{2018/01/12 isaasia, Traditional Cache (2.5/1.5)}}\begin{cacheText}Aujourd'hui c'est repos et, enfin,le soleil brille .Nous voici donc partis pour faire cette cache qui nous tient tant à cœur. Arrivés sur le site ,il y a beaucoup de circulation :ce n'est pas facile de chercher en étant discret. D'autant plus que la Belle se fait désirer!!!! Au moment de renoncer... BINGO!!! Merci Isa pour cette super cache( un PF évidemment!!!) et pour la découverte de ce cadran solaire.\end{cacheText}

\cacheNumber{1105}\needspace{5\baselineskip}\cacheName{\href{http://coord.info/GC4ZPMP}{Le facteur a des plumes / Pigeon Post - letterbox} — \href{http://coord.info/GC4ZPMP\Number{}744006424}{1105}}\cacheData{{2018/01/27 DorisBear, Letterbox Hybrid (1.5/1.5)}}\begin{cacheText}C'est pour l'Australian day que nous découvrons cette superbe cache avec ces objets voyageurs. Il faut effectivement bien regarder partout!!!Un grand merci pour cette très jolie cache.Un PF évidemment...\end{cacheText}

\cacheNumber{1106}\needspace{5\baselineskip}\cacheName{\href{http://coord.info/GC7GCV7}{Australia Day 2018} — \href{http://coord.info/GC7GCV7\Number{}744004702}{1106}}\cacheData{{2018/01/27 DorisBear, Event Cache (1/1)}}\begin{cacheText}Accueillis par les autochtones australiens locaux ,nous sommes arrivés à temps pour le concours de\Quoted{lancer de tongs} qui s'est déroulé sous les drapeaux Australien.Nous avons bien rit et après la remise des superbes prix (kangourou ,crocodile et TB tong ,nous sommes tous parti à la recherche de la Belle publiée pour l'occasion.La promenade nous a fait beaucoup de bien, et la cache ,comme d'habitude avec les Ours, a été à la hauteur de nos espérances( bien dans le thème Australien!!!! ). La journée s'est terminée autour du délicieux burger australien (Bun,sauce,salad,tomato,steak,cheese,pineapple,beet and egg) confectionné par nos amis les Ours, et des innombrables succulents desserts apportés par les participants.Un 20/20 pour cette inoubliable journée...MERCI\end{cacheText}

\cacheNumber{1107}\needspace{5\baselineskip}\cacheName{\href{http://coord.info/GC7H3TF}{Australia Day Cache} — \href{http://coord.info/GC7H3TF\Number{}744005671}{1107}}\cacheData{{2018/01/27 DorisBear, Unknown Cache (3/1.5)}}\begin{cacheText}Co (FTF)

L’énigme, résolue dans la soirée, nous a permis d'en apprendre un peu plus sur ce pays lointain .C'est avec la Team Australian Day que nous avons découvert cette cache élaborée qui est parfaitement dans le thème de la journée...Un PF évidemment pour ce superbe travail d'information et de fabrication .Un grand merci.\end{cacheText}

\cacheNumber{1108}\needspace{5\baselineskip}\cacheName{\href{http://coord.info/GC5MC0A}{Une cache pour Patricia64 / A cache for Patricia64} — \href{http://coord.info/GC5MC0A\Number{}744187590}{1108}}\cacheData{{2018/01/28 dorisbear, Multi-cache (2/2)}}\begin{cacheText}Par ce très bel après-midi ensoleillé, pas question de rester à la maison!!!Nous voilà donc partis pour une des caches de Dorisbear. Nous comprenons pourquoi elle est dédiée à Patricia64 en arrivant devant le panneau.Cette Multi ne nous déçoit pas: la cache finale est à la hauteur de nos attentes. Parfaitement intégrée dans le décor,c'est encore du très beau travail . Nous aurions bien distribué un point favoris mais hélas nos réserves sont épuisées!!! Nous l'attribuons virtuellement.\end{cacheText}

\cacheNumber{1109}\needspace{5\baselineskip}\cacheName{\href{http://coord.info/GC7ABW8}{Le trésor de Matouba} — \href{http://coord.info/GC7ABW8\Number{}744030683}{1109}}\cacheData{{2018/01/28 Dominique. Duriez, Traditional Cache (1/1)}}\begin{cacheText}Sur Navarrenx pour la foire agricole avec des amis, je m'éclipse pour aller trouver le trésor de Matouba. Arrivée près du PZ mon œil est attiré par une boîte semi- ouverte. C'est bien lui!!!Malheureusement, tout est trempé et je n'ai rien pour faire la maintenance (mon sac géocaching est resté à la maison). Je ne peux donc pas signer le log book. J'envoie une photo. À l'occasion je reviendrai signer. Merci pour la cache\end{cacheText}

\cacheNumber{1110}\needspace{5\baselineskip}\cacheName{\href{http://coord.info/GC3Y7BZ}{Voie Verte ~ Tronçon Laroin-Tarsacq (3)} — \href{http://coord.info/GC3Y7BZ\Number{}745274639}{1110}}\cacheData{{2018/02/04 Lluis64, Traditional Cache (2.5/1.5)}}\begin{cacheText}Après cette longue période de pluie,le moindre rayon de soleil nous pousse à la recherche des caches.Nous jetons notre dévolu sur la VVLT .Nous garons notre carrosse à mi parcours(VVLT\Number{}27 )et descendons les vélos. 

Cette cache n'est pas dans la série mais nous la loguons en passant.L'indice et les coordonnées nous ménent tout droit à la belle.Une jolie boîte à bidules!!!Merci pour la cache.\end{cacheText}

\cacheNumber{1111}\needspace{5\baselineskip}\cacheName{\href{http://coord.info/GC42KP0}{Event SdG GR653 L\Underscore{}L-08} — \href{http://coord.info/GC42KP0\Number{}745231622}{1111}}\cacheData{{2018/02/04 Charnègues, Traditional Cache (1.5/1.5)}}\begin{cacheText}Après cette longue période de pluie,le moindre rayon de soleil nous pousse à la recherche des caches.Nous jetons notre dévolu sur la VVLT .Nous garons notre carrosse à mi parcours(VVLT\Number{}27 )et descendons les vélos. 

Nous en profitons pour faire celle ci en passant.La belle est trouvée en deux temps trois mouvements grâce au spoiler. Merci pour la cache.\end{cacheText}

\cacheNumber{1112}\needspace{5\baselineskip}\cacheName{\href{http://coord.info/GC70YB6}{VVLT \Number{}27} — \href{http://coord.info/GC70YB6\Number{}745229397}{1112}}\cacheData{{2018/02/04 l.a.f.l.e.u.r, Traditional Cache (1.5/1.5)}}\begin{cacheText}Après cette longue période de pluie,le moindre rayon de soleil nous pousse à la recherche des caches.Nous jetons notre dévolu sur la VVLT .Nous garons notre carrosse à mi parcours(VVLT\Number{}27 )et descendons les vélos. 

Alors que nous nous apprêtons à partir après avoir vite repéré la belle,une famille s'approche du PZ,portable à la main, et nous observe.Ne serait ce pas des géocacheurs??? Hé bien oui!!!C'est ainsi que nous faisons connaissance avec les lafrogne.Après le traditionnel géoblabla nous partons à l'opposé.

 Merci pour la cache\end{cacheText}

\cacheNumber{1113}\needspace{5\baselineskip}\cacheName{\href{http://coord.info/GC70YBB}{VVLT \Number{}28} — \href{http://coord.info/GC70YBB\Number{}745230573}{1113}}\cacheData{{2018/02/04 l.a.f.l.e.u.r, Traditional Cache (1.5/1.5)}}\begin{cacheText}Après cette longue période de pluie,le moindre rayon de soleil nous pousse à la recherche des caches.Nous jetons notre dévolu sur la VVLT .Nous garons notre carrosse à mi parcours(VVLT\Number{}27 )et descendons les vélos. 

Arrivés au PZ et après une petite recherche nous trouvons un superbe camouflage. Le logbook est trempé mais nous réussissons à tamponner.Merci pour cette super cache.\end{cacheText}

\cacheNumber{1114}\needspace{5\baselineskip}\cacheName{\href{http://coord.info/GC70YBE}{VVLT \Number{}29} — \href{http://coord.info/GC70YBE\Number{}745233059}{1114}}\cacheData{{2018/02/04 l.a.f.l.e.u.r, Traditional Cache (1.5/1.5)}}\begin{cacheText}Après cette longue période de pluie,le moindre rayon de soleil nous pousse à la recherche des caches.Nous jetons notre dévolu sur la VVLT .Nous garons notre carrosse à mi parcours(VVLT\Number{}27 )et descendons les vélos. 

Les coordonnées précises et le spoiler nous mènent tout droit à la cache .Merci pour tout ce travail de pose.\end{cacheText}

\cacheNumber{1115}\needspace{5\baselineskip}\cacheName{\href{http://coord.info/GC70YBJ}{VVLT \Number{}30} — \href{http://coord.info/GC70YBJ\Number{}745234920}{1115}}\cacheData{{2018/02/04 l.a.f.l.e.u.r, Traditional Cache (1.5/1.5)}}\begin{cacheText}Après cette longue période de pluie,le moindre rayon de soleil nous pousse à la recherche des caches.Nous jetons notre dévolu sur la VVLT .Nous garons notre carrosse à mi parcours(VVLT\Number{}27 )et descendons les vélos. 

La Belle ne nous a pas résisté longtemps .Merci pour cette super promenade.\end{cacheText}

\cacheNumber{1116}\needspace{5\baselineskip}\cacheName{\href{http://coord.info/GC70YBQ}{VVLT \Number{}31} — \href{http://coord.info/GC70YBQ\Number{}745236337}{1116}}\cacheData{{2018/02/04 l.a.f.l.e.u.r, Traditional Cache (1.5/1.5)}}\begin{cacheText}Après cette longue période de pluie,le moindre rayon de soleil nous pousse à la recherche des caches.Nous jetons notre dévolu sur la VVLT .Nous garons notre carrosse à mi parcours(VVLT\Number{}27 )et descendons les vélos. 

Nous nous sommes concentré sur le spoiler et...RIEN!!!Nous avons élargi la zone et... enfin la SURPRISE!!!Cache super sympa,merci.Un PF évidemment\end{cacheText}

\cacheNumber{1117}\needspace{5\baselineskip}\cacheName{\href{http://coord.info/GC70YBV}{VVLT \Number{}32} — \href{http://coord.info/GC70YBV\Number{}745237221}{1117}}\cacheData{{2018/02/04 l.a.f.l.e.u.r, Traditional Cache (1.5/1.5)}}\begin{cacheText}Après cette longue période de pluie,le moindre rayon de soleil nous pousse à la recherche des caches.Nous jetons notre dévolu sur la VVLT .Nous garons notre carrosse à mi parcours(VVLT\Number{}27 )et descendons les vélos. 

La Belle est vite débusquée.Le logbook est trempé mais nous réussissons à tamponner.Merci pour la cache.\end{cacheText}

\cacheNumber{1118}\needspace{5\baselineskip}\cacheName{\href{http://coord.info/GC70YBZ}{VVLT \Number{}33} — \href{http://coord.info/GC70YBZ\Number{}745271952}{1118}}\cacheData{{2018/02/04 l.a.f.l.e.u.r, Traditional Cache (1.5/1.5)}}\begin{cacheText}Après cette longue période de pluie,le moindre rayon de soleil nous pousse à la recherche des caches.Nous jetons notre dévolu sur la VVLT .Nous garons notre carrosse à mi parcours(VVLT\Number{}27 )et descendons les vélos. 

Aucune difficulté pour déloger la belle qui nous attend sagement.Encore un excellent camouflage...que du bonheur!!!!Merci pour la cache.\end{cacheText}

\cacheNumber{1119}\needspace{5\baselineskip}\cacheName{\href{http://coord.info/GC70YC1}{VVLT \Number{}34} — \href{http://coord.info/GC70YC1\Number{}745272354}{1119}}\cacheData{{2018/02/04 l.a.f.l.e.u.r, Traditional Cache (1.5/1.5)}}\begin{cacheText}Après cette longue période de pluie,le moindre rayon de soleil nous pousse à la recherche des caches.Nous jetons notre dévolu sur la VVLT .Nous garons notre carrosse à mi parcours(VVLT\Number{}27 )et descendons les vélos. 

Nous arrivons ensemble,avec la famille lafrogne, sur le PZ et c'est madame lafrogne qui déniche la Belle bien cachée!!!Le logbook est détrempé:nous le remplaçons.Merci pour ce bon moment.\end{cacheText}

\cacheNumber{1120}\needspace{5\baselineskip}\cacheName{\href{http://coord.info/GC70YC6}{VVLT \Number{}35} — \href{http://coord.info/GC70YC6\Number{}745272580}{1120}}\cacheData{{2018/02/04 l.a.f.l.e.u.r, Traditional Cache (1.5/1.5)}}\begin{cacheText}Après cette longue période de pluie,le moindre rayon de soleil nous pousse à la recherche des caches.Nous jetons notre dévolu sur la VVLT .Nous garons notre carrosse à mi parcours(VVLT\Number{}27 )et descendons les vélos. 

Toujours en compagnie des lafrogne ,qui avaient préféré continuer leur chemin pour cause de chasse,nous découvrons la Belle.Encore une belle surprise.Merci les l.a.f.l.e.u.r pour ce super circuit.\end{cacheText}

\cacheNumber{1121}\needspace{5\baselineskip}\cacheName{\href{http://coord.info/GC70YC8}{VVLT \Number{}36} — \href{http://coord.info/GC70YC8\Number{}745274248}{1121}}\cacheData{{2018/02/04 l.a.f.l.e.u.r, Traditional Cache (1.5/1.5)}}\begin{cacheText}Après cette longue période de pluie,le moindre rayon de soleil nous pousse à la recherche des caches.Nous jetons notre dévolu sur la VVLT .Nous garons notre carrosse à mi parcours(VVLT\Number{}27 )et descendons les vélos. 

Ici nous découvrons une jolie boite travaillée ,parfaitement intégrée.Nous continuons notre géoblabla avec les lafrogne avant leur départ.

Merci pour la cache.\end{cacheText}

\cacheNumber{1122}\needspace{5\baselineskip}\cacheName{\href{http://coord.info/GC70YCD}{VVLT \Number{}37} — \href{http://coord.info/GC70YCD\Number{}745274423}{1122}}\cacheData{{2018/02/04 l.a.f.l.e.u.r, Traditional Cache (1.5/1.5)}}\begin{cacheText}Après cette longue période de pluie,le moindre rayon de soleil nous pousse à la recherche des caches.Nous jetons notre dévolu sur la VVLT .Nous garons notre carrosse à mi parcours(VVLT\Number{}27 )et descendons les vélos. 

Arrivés au PZ, il ne nous a pas fallu longtemps pour mettre la main dessus.Merci pour tout ce travail.\end{cacheText}

\cacheNumber{1123}\needspace{5\baselineskip}\cacheName{\href{http://coord.info/GC70YCG}{VVLT \Number{}38} — \href{http://coord.info/GC70YCG\Number{}745328242}{1123}}\cacheData{{2018/02/04 l.a.f.l.e.u.r, Traditional Cache (1.5/1.5)}}\begin{cacheText}Après cette longue période de pluie,le moindre rayon de soleil nous pousse à la recherche des caches.Nous jetons notre dévolu sur la VVLT .Nous garons notre carrosse à mi parcours(VVLT\Number{}27 )et descendons les vélos. 

Pas de difficultés pour déloger la Belle :nous attendons patiemment que les promeneurs du dimanche passent sur le chemin. Merci pour la cache\end{cacheText}

\cacheNumber{1124}\needspace{5\baselineskip}\cacheName{\href{http://coord.info/GC70YCK}{VVLT \Number{}39} — \href{http://coord.info/GC70YCK\Number{}745329212}{1124}}\cacheData{{2018/02/04 l.a.f.l.e.u.r, Traditional Cache (1.5/1.5)}}\begin{cacheText}Après cette longue période de pluie,le moindre rayon de soleil nous pousse à la recherche des caches.Nous jetons notre dévolu sur la VVLT .Nous garons notre carrosse à mi parcours(VVLT\Number{}27 )et descendons les vélos.

 La cache est bien indiquée et nous découvrons une jolie boite à bidules.Merci pour la cache.\end{cacheText}

\cacheNumber{1125}\needspace{5\baselineskip}\cacheName{\href{http://coord.info/GC70YCR}{VVLT \Number{}40} — \href{http://coord.info/GC70YCR\Number{}745330226}{1125}}\cacheData{{2018/02/04 l.a.f.l.e.u.r, Traditional Cache (1.5/1.5)}}\begin{cacheText}Après cette longue période de pluie,le moindre rayon de soleil nous pousse à la recherche des caches.Nous jetons notre dévolu sur la VVLT .Nous garons notre carrosse à mi parcours(VVLT\Number{}27 )et descendons les vélos. 

Nous avons mis du temps pour la trouver car nous ne cherchions pas au bon endroit!!!En partant ,un autre tronc a attiré notre attention et BINGO !!!Il n'y a pas de boite mais un simple sachet en plastique.Merci pour la cache.\end{cacheText}

\cacheNumber{1126}\needspace{5\baselineskip}\cacheName{\href{http://coord.info/GC70YCX}{VVLT \Number{}41} — \href{http://coord.info/GC70YCX\Number{}745330720}{1126}}\cacheData{{2018/02/04 l.a.f.l.e.u.r, Traditional Cache (1.5/1.5)}}\begin{cacheText}Après cette longue période de pluie,le moindre rayon de soleil nous pousse à la recherche des caches.Nous jetons notre dévolu sur la VVLT .Nous garons notre carrosse à mi parcours(VVLT\Number{}27 )et descendons les vélos. 

La belle nous attend au pied, pendant qu'une belle blonde surveille son petit de près!!!\end{cacheText}

\cacheNumber{1127}\needspace{5\baselineskip}\cacheName{\href{http://coord.info/GC70YCZ}{VVLT \Number{}42} — \href{http://coord.info/GC70YCZ\Number{}745332131}{1127}}\cacheData{{2018/02/04 l.a.f.l.e.u.r, Traditional Cache (1.5/1.5)}}\begin{cacheText}Après cette longue période de pluie,le moindre rayon de soleil nous pousse à la recherche des caches.Nous jetons notre dévolu sur la VVLT .Nous garons notre carrosse à mi parcours(VVLT\Number{}27 )et descendons les vélos. 

Le GPS nous promène dans les bois mais nous finissons par découvrir la Belle. Merci pour la cache\end{cacheText}

\cacheNumber{1128}\needspace{5\baselineskip}\cacheName{\href{http://coord.info/GC70YD0}{VVLT \Number{}43} — \href{http://coord.info/GC70YD0\Number{}745333173}{1128}}\cacheData{{2018/02/04 l.a.f.l.e.u.r, Traditional Cache (1.5/1.5)}}\begin{cacheText}Après cette longue période de pluie,le moindre rayon de soleil nous pousse à la recherche des caches.Nous jetons notre dévolu sur la VVLT .Nous garons notre carrosse à mi parcours(VVLT\Number{}27 )et descendons les vélos. 

La Cache est trouvée sans trop de difficulté. Le log book est très humide (il ne m' en reste qu'un de remplacement ,je n'en ai pas d'autres pour rechanger dommage )Merci pour la cache\end{cacheText}

\cacheNumber{1129}\needspace{5\baselineskip}\cacheName{\href{http://coord.info/GC70YD4}{VVLT \Number{}44} — \href{http://coord.info/GC70YD4\Number{}745333481}{1129}}\cacheData{{2018/02/04 l.a.f.l.e.u.r, Traditional Cache (1.5/1.5)}}\begin{cacheText}Après cette longue période de pluie,le moindre rayon de soleil nous pousse à la recherche des caches.Nous jetons notre dévolu sur la VVLT .Nous garons notre carrosse à mi parcours(VVLT\Number{}27 )et descendons les vélos. 

La cache est trouvée très facilement grâce à l'indice. Merci pour la cache\end{cacheText}

\cacheNumber{1130}\needspace{5\baselineskip}\cacheName{\href{http://coord.info/GC70YD5}{VVLT \Number{}45} — \href{http://coord.info/GC70YD5\Number{}745333960}{1130}}\cacheData{{2018/02/04 l.a.f.l.e.u.r, Traditional Cache (1.5/1.5)}}\begin{cacheText}Après cette longue période de pluie,le moindre rayon de soleil nous pousse à la recherche des caches.Nous jetons notre dévolu sur la VVLT .Nous garons notre carrosse à mi parcours(VVLT\Number{}27 )et descendons les vélos. 

On finit par trouver la Belle  après quelques recherches. Merci pour la cache.\end{cacheText}

\cacheNumber{1131}\needspace{5\baselineskip}\cacheName{\href{http://coord.info/GC70YD7}{VVLT \Number{}46} — \href{http://coord.info/GC70YD7\Number{}745334564}{1131}}\cacheData{{2018/02/04 l.a.f.l.e.u.r, Traditional Cache (1.5/1.5)}}\begin{cacheText}Après cette longue période de pluie,le moindre rayon de soleil nous pousse à la recherche des caches.Nous jetons notre dévolu sur la VVLT .Nous garons notre carrosse à mi parcours(VVLT\Number{}27 )et descendons les vélos. 

Pas de soucis particulier pour celle-ci qui se laisse découvrir facilement. Merci pour la cache.\end{cacheText}

\cacheNumber{1132}\needspace{5\baselineskip}\cacheName{\href{http://coord.info/GC70YDA}{VVLT \Number{}47} — \href{http://coord.info/GC70YDA\Number{}745334933}{1132}}\cacheData{{2018/02/04 l.a.f.l.e.u.r, Traditional Cache (1.5/1.5)}}\begin{cacheText}Après cette longue période de pluie,le moindre rayon de soleil nous pousse à la recherche des caches.Nous jetons notre dévolu sur la VVLT .Nous garons notre carrosse à mi parcours(VVLT\Number{}27 )et descendons les vélos. 

La cache bien à l’abri est trouvée sans problème. Merci pour la cache.\end{cacheText}

\cacheNumber{1133}\needspace{5\baselineskip}\cacheName{\href{http://coord.info/GC70YDB}{VVLT \Number{}48} — \href{http://coord.info/GC70YDB\Number{}745338041}{1133}}\cacheData{{2018/02/04 l.a.f.l.e.u.r, Traditional Cache (1.5/1.5)}}\begin{cacheText}Après cette longue période de pluie,le moindre rayon de soleil nous pousse à la recherche des caches.Nous jetons notre dévolu sur la VVLT .Nous garons notre carrosse à mi parcours(VVLT\Number{}27 )et descendons les vélos. 

Nous finissons par mettre la main sur le trésor mais le logbook est fichu!!!Nous laissons notre dernier logbook de remplacement en place.Merci pour la cache.\end{cacheText}

\cacheNumber{1134}\needspace{5\baselineskip}\cacheName{\href{http://coord.info/GC70YDF}{VVLT} — \href{http://coord.info/GC70YDF\Number{}745340900}{1134}}\cacheData{{2018/02/04 l.a.f.l.e.u.r, Letterbox Hybrid (1.5/1.5)}}\begin{cacheText}Après cette longue période de pluie,le moindre rayon de soleil nous pousse à la recherche des caches.Nous jetons notre dévolu sur la VVLT .Nous garons notre carrosse à mi parcours(VVLT\Number{}27 )et descendons les vélos. 

Arrivés au PZ nous découvrons la letterbox:belle surprise dans les bois!!!!Seul bémol le tampon \Quoted{bave}.Merci pour tout ce travail

Pas de photo pour l'effet de surprise!!!!\end{cacheText}

\cacheNumber{1135}\needspace{5\baselineskip}\cacheName{\href{http://coord.info/GC70YDN}{VVLT \Number{}50} — \href{http://coord.info/GC70YDN\Number{}745341516}{1135}}\cacheData{{2018/02/04 l.a.f.l.e.u.r, Traditional Cache (1.5/1.5)}}\begin{cacheText}Après cette longue période de pluie,le moindre rayon de soleil nous pousse à la recherche des caches.Nous jetons notre dévolu sur la VVLT .Nous garons notre carrosse à mi parcours(VVLT\Number{}27 )et descendons les vélos.

Arrivés au PZ ,grosse déception!!!Le panneau d'information a disparu!!!Mais on ne sait jamais... apres quelques investigations nous découvrons la Belle quiest toujours en place . Merci pour la cache.\end{cacheText}

\cacheNumber{1136}\needspace{5\baselineskip}\cacheName{\href{http://coord.info/GC70YDV}{VVLT \Number{}51} — \href{http://coord.info/GC70YDV\Number{}745342596}{1136}}\cacheData{{2018/02/04 l.a.f.l.e.u.r, Traditional Cache (1.5/1.5)}}\begin{cacheText}Après cette longue période de pluie,le moindre rayon de soleil nous pousse à la recherche des caches.Nous jetons notre dévolu sur la VVLT .Nous garons notre carrosse à mi parcours(VVLT\Number{}27 )et descendons les vélos. 

Arrivés sur les lieux nous devons patienter car une famille avec enfants donne du pain aux chevaux et poneys.Une fois partis, il ne nous faut pas longtemps pour déloger la Belle!!!

Et voila... ce tronçon est terminé;Nous nous sommes régalés avec cette super série .Un grand merci et un PF pour l'ensemble des caches\end{cacheText}

\cacheNumber{1137}\needspace{5\baselineskip}\cacheName{\href{http://coord.info/GC4DKG5}{Tour du Lac du GABAS ~ 03} — \href{http://coord.info/GC4DKG5\Number{}747920877}{1137}}\cacheData{{2018/02/21 MaryeLoup, Traditional Cache (1.5/2)}}\begin{cacheText}Le rendez vous est pris depuis un moment avec Bigorra65 pour faire ce tour du Lac de Gabas.Malgré une météo incertaine nous maintenons la sortie et nous nous retrouvons sur le parking. Bien chaussées ,équipées de nos sacs à dos remplis d'outils(au cas ou...) et de victuailles ,nous nous élançons à la recherche des Belles...

Nous papotons,nous papotons et nous avons passé le PZ!!!!Retour sur nos pas pour chercher la Belle qui nous attend au pied du superbe chêne.Merci pour la cache\end{cacheText}

\cacheNumber{1138}\needspace{5\baselineskip}\cacheName{\href{http://coord.info/GC4DKGP}{Tour du Lac du GABAS ~ 05} — \href{http://coord.info/GC4DKGP\Number{}747921464}{1138}}\cacheData{{2018/02/21 MaryeLoup, Traditional Cache (1.5/2)}}\begin{cacheText}Le rendez vous est pris depuis un moment avec Bigorra65 pour faire ce tour du Lac de Gabas.Malgré une météo incertaine nous maintenons la sortie et nous nous retrouvons sur le parking. Bien chaussées ,équipées de nos sacs à dos remplis d'outils(au cas ou...) et de victuailles ,nous nous élançons à la recherche des Belles...

Malgré un DNF, nous tentons de déloger la Belle .Bingo ...elle est bien là à sa place!!!Merci pour la cache.\end{cacheText}

\cacheNumber{1139}\needspace{5\baselineskip}\cacheName{\href{http://coord.info/GC4DKGY}{Tour du Lac du GABAS ~ 06} — \href{http://coord.info/GC4DKGY\Number{}747922047}{1139}}\cacheData{{2018/02/21 MaryeLoup, Traditional Cache (1.5/2)}}\begin{cacheText}Le rendez vous est pris depuis un moment avec Bigorra65 pour faire ce tour du Lac de Gabas.Malgré une météo incertaine nous maintenons la sortie et nous nous retrouvons sur le parking. Bien chaussées ,équipées de nos sacs à dos remplis d'outils(au cas ou...) et de victuailles ,nous nous élançons à la recherche des Belles...

Effectivement l'indice est très explicite et il n'y a pas beaucoup de solution.La Belle est vite trouvée.Merci.\end{cacheText}

\cacheNumber{1140}\needspace{5\baselineskip}\cacheName{\href{http://coord.info/GC4DKH6}{Tour du Lac du GABAS ~ 08} — \href{http://coord.info/GC4DKH6\Number{}747922866}{1140}}\cacheData{{2018/02/21 MaryeLoup, Traditional Cache (1.5/2)}}\begin{cacheText}Le rendez vous est pris depuis un moment avec Bigorra65 pour faire ce tour du Lac de Gabas.Malgré une météo incertaine nous maintenons la sortie et nous nous retrouvons sur le parking. Bien chaussées ,équipées de nos sacs à dos remplis d'outils(au cas ou...) et de victuailles ,nous nous élançons à la recherche des Belles...

Alors que le GPS fait des siennes nous faisons le tour des supposés vases pour découvrir la cache.C'est chose faite...à la suite!!!Merci.\end{cacheText}

\cacheNumber{1141}\needspace{5\baselineskip}\cacheName{\href{http://coord.info/GC4DKHA}{Tour du Lac du GABAS ~ 09} — \href{http://coord.info/GC4DKHA\Number{}747923286}{1141}}\cacheData{{2018/02/21 MaryeLoup, Traditional Cache (1.5/2)}}\begin{cacheText}Le rendez vous est pris depuis un moment avec Bigorra65 pour faire ce tour du Lac de Gabas.Malgré une météo incertaine nous maintenons la sortie et nous nous retrouvons sur le parking. Bien chaussées ,équipées de nos sacs à dos remplis d'outils(au cas ou...) et de victuailles ,nous nous élançons à la recherche des Belles...

Nous trouvons rapidement un tube pas très étanche.Nous supposons que c'est une cache de remplacement car elle n'a aucun rapport avec le hint.Merci.\end{cacheText}

\cacheNumber{1142}\needspace{5\baselineskip}\cacheName{\href{http://coord.info/GC4DKHC}{Tour du Lac du GABAS ~ 10} — \href{http://coord.info/GC4DKHC\Number{}747923750}{1142}}\cacheData{{2018/02/21 MaryeLoup, Traditional Cache (1.5/2)}}\begin{cacheText}Le rendez vous est pris depuis un moment avec Bigorra65 pour faire ce tour du Lac de Gabas.Malgré une météo incertaine nous maintenons la sortie et nous nous retrouvons sur le parking. Bien chaussées ,équipées de nos sacs à dos remplis d'outils(au cas ou...) et de victuailles ,nous nous élançons à la recherche des Belles...

La Belle est très bien cachée mais elle ne résiste pas à l’œil de lynx de Bigorra65. Merci.\end{cacheText}

\cacheNumber{1143}\needspace{5\baselineskip}\cacheName{\href{http://coord.info/GC4DKHF}{Tour du Lac du GABAS ~ 11} — \href{http://coord.info/GC4DKHF\Number{}747924181}{1143}}\cacheData{{2018/02/21 MaryeLoup, Traditional Cache (1.5/2)}}\begin{cacheText}Le rendez vous est pris depuis un moment avec Bigorra65 pour faire ce tour du Lac de Gabas.Malgré une météo incertaine nous maintenons la sortie et nous nous retrouvons sur le parking. Bien chaussées ,équipées de nos sacs à dos remplis d'outils(au cas ou...) et de victuailles ,nous nous élançons à la recherche des Belles...

Le jeunot nous tend les bras et nous livre la Belle.Merci pour la cache.\end{cacheText}

\cacheNumber{1144}\needspace{5\baselineskip}\cacheName{\href{http://coord.info/GC4DKN6}{Tour du Lac du GABAS ~ 16} — \href{http://coord.info/GC4DKN6\Number{}747926334}{1144}}\cacheData{{2018/02/21 MaryeLoup, Traditional Cache (1.5/2)}}\begin{cacheText}Le rendez vous est pris depuis un moment avec Bigorra65 pour faire ce tour du Lac de Gabas.Malgré une météo incertaine nous maintenons la sortie et nous nous retrouvons sur le parking. Bien chaussées ,équipées de nos sacs à dos remplis d'outils(au cas ou...) et de victuailles ,nous nous élançons à la recherche des Belles...

La dernière cache de cette très agréable ballade nous attend à la vue de tous sur le rocher!!!Bigorra65 la remet à l’abri des regards afin de la préserver.Dommage qu'il y ait tant de caches manquantes  car le parcours est  vraiment très plaisant.Merci MaryeLoup pour tout ce travail de pose..\end{cacheText}

\cacheNumber{1145}\needspace{5\baselineskip}\cacheName{\href{http://coord.info/GC4DKN8}{Tour du Lac du GABAS ~ 17} — \href{http://coord.info/GC4DKN8\Number{}747925278}{1145}}\cacheData{{2018/02/21 MaryeLoup, Traditional Cache (1.5/2)}}\begin{cacheText}Le rendez vous est pris depuis un moment avec Bigorra65 pour faire ce tour du Lac de Gabas.Malgré une météo incertaine nous maintenons la sortie et nous nous retrouvons sur le parking. Bien chaussées ,équipées de nos sacs à dos remplis d'outils(au cas ou...) et de victuailles ,nous nous élançons à la recherche des Belles...

Il nous semblait que cette cache serait facile ...Hé bien non !!!Elle nous a fait tourner et re-tourner mais nous avons eu le dernier mot!!!Les chaussettes ont presque disparues ...Merci.\end{cacheText}

\cacheNumber{1146}\needspace{5\baselineskip}\cacheName{\href{http://coord.info/GC4DKNF}{Tour du Lac du GABAS ~ 19} — \href{http://coord.info/GC4DKNF\Number{}747790731}{1146}}\cacheData{{2018/02/21 MaryeLoup, Traditional Cache (1.5/2)}}\begin{cacheText}Le rendez vous est pris depuis un moment avec Bigorra65 pour faire ce tour du Lac de Gabas.Malgré une météo incertaine nous maintenons la sortie et nous nous retrouvons sur le parking. Bien chaussées ,équipées de nos sacs à dos remplis d'outils(au cas ou...) et de victuailles ,nous nous élançons à la recherche des Belles...

Celle ci est découverte en deux temps trois mouvements grâce à l’indice et à la photo.;..au pied du sapin!!!!Merci\end{cacheText}

\cacheNumber{1147}\needspace{5\baselineskip}\cacheName{\href{http://coord.info/GC4DKNQ}{Tour du Lac du GABAS ~ 21} — \href{http://coord.info/GC4DKNQ\Number{}747793502}{1147}}\cacheData{{2018/02/21 MaryeLoup, Traditional Cache (1.5/2)}}\begin{cacheText}Le rendez vous est pris depuis un moment avec Bigorra65 pour faire ce tour du Lac de Gabas.Malgré une météo incertaine nous maintenons la sortie et nous nous retrouvons sur le parking. Bien chaussées ,équipées de nos sacs à dos remplis d'outils(au cas ou...) et de victuailles ,nous nous élançons à la recherche des Belles...

Arrivées sur le PZ  il nous a fallu du temps pour déloger la cache.Sur le point d'abandonner elle s'est laissée apercevoir...oufff.Merci\end{cacheText}

\cacheNumber{1148}\needspace{5\baselineskip}\cacheName{\href{http://coord.info/GC4DKP2}{Tour du Lac du GABAS ~ 24} — \href{http://coord.info/GC4DKP2\Number{}747793990}{1148}}\cacheData{{2018/02/21 MaryeLoup, Traditional Cache (1.5/2)}}\begin{cacheText}Le rendez vous est pris depuis un moment avec Bigorra65 pour faire ce tour du Lac de Gabas.Malgré une météo incertaine nous maintenons la sortie et nous nous retrouvons sur le parking. Bien chaussées ,équipées de nos sacs à dos remplis d'outils(au cas ou...) et de victuailles ,nous nous élançons à la recherche des Belles...

Nous avons eu du mal pour ne trouver ...que le bouchon!!!.Nous remettons donc un logbook de remplacement. Merci pour la cache.\end{cacheText}

\cacheNumber{1149}\needspace{5\baselineskip}\cacheName{\href{http://coord.info/GC4DKPA}{Tour du Lac du GABAS ~ 26} — \href{http://coord.info/GC4DKPA\Number{}747795193}{1149}}\cacheData{{2018/02/21 MaryeLoup, Traditional Cache (1.5/2)}}\begin{cacheText}Le rendez vous est pris depuis un moment avec Bigorra65 pour faire ce tour du Lac de Gabas.Malgré une météo incertaine nous maintenons la sortie et nous nous retrouvons sur le parking. Bien chaussées ,équipées de nos sacs à dos remplis d'outils(au cas ou...) et de victuailles ,nous nous élançons à la recherche des Belles...

L'indice et le spoiler sont très explicites et nous trouvons la cache sans difficulté.Elle est remise à temps à sa place lorsqu'un jogger fait son apparition!!! Merci\end{cacheText}

\cacheNumber{1150}\needspace{5\baselineskip}\cacheName{\href{http://coord.info/GC4DKPG}{Tour du Lac du GABAS ~ 27} — \href{http://coord.info/GC4DKPG\Number{}747796160}{1150}}\cacheData{{2018/02/21 MaryeLoup, Traditional Cache (1.5/2)}}\begin{cacheText}Le rendez vous est pris depuis un moment avec Bigorra65 pour faire ce tour du Lac de Gabas.Malgré une météo incertaine nous maintenons la sortie et nous nous retrouvons sur le parking. Bien chaussées ,équipées de nos sacs à dos remplis d'outils(au cas ou...) et de victuailles ,nous nous élançons à la recherche des Belles...

Un bonhomme,deux bonhomme ...on ne sait pas ou donner de la tête!!!La Belle finit par se montrer.Merci\end{cacheText}

\cacheNumber{1151}\needspace{5\baselineskip}\cacheName{\href{http://coord.info/GC4DKPN}{Tour du Lac du GABAS ~ 28} — \href{http://coord.info/GC4DKPN\Number{}747797780}{1151}}\cacheData{{2018/02/21 MaryeLoup, Traditional Cache (1.5/2)}}\begin{cacheText}\smallskip



 

\smallskip





\smallskip



 

\smallskip



Le rendez vous est pris depuis un moment avec Bigorra65 pour faire ce tour du Lac de Gabas.Malgré une météo incertaine nous maintenons la sortie et nous nous retrouvons sur le parking. Bien chaussées ,équipées de nos sacs à dos remplis d'outils(au cas ou...) et de victuailles ,nous nous élançons à la recherche des Belles...

Arrivées sur le PZ nous devinons de suite ou se trouve la Belle.Seul soucis nous sommes trop petites!!!Qu'à cela ne tienne nous trouvons la solution du marche pied improvisé!!!Merci\end{cacheText}

\cacheNumber{1152}\needspace{5\baselineskip}\cacheName{\href{http://coord.info/GC4DKPT}{Tour du Lac du GABAS ~ 29} — \href{http://coord.info/GC4DKPT\Number{}747798602}{1152}}\cacheData{{2018/02/21 MaryeLoup, Traditional Cache (1.5/2)}}\begin{cacheText}Le rendez vous est pris depuis un moment avec Bigorra65 pour faire ce tour du Lac de Gabas.Malgré une météo incertaine nous maintenons la sortie et nous nous retrouvons sur le parking. Bien chaussées ,équipées de nos sacs à dos remplis d'outils(au cas ou...) et de victuailles ,nous nous élançons à la recherche des Belles...

Nous découvrons ici un très joli arbre qui abrite la Belle.Merci\end{cacheText}

\cacheNumber{1153}\needspace{5\baselineskip}\cacheName{\href{http://coord.info/GC4DKPZ}{Tour du Lac du GABAS ~ 30} — \href{http://coord.info/GC4DKPZ\Number{}747914673}{1153}}\cacheData{{2018/02/21 MaryeLoup, Traditional Cache (1.5/2)}}\begin{cacheText}Le rendez vous est pris depuis un moment avec Bigorra65 pour faire ce tour du Lac de Gabas.Malgré une météo incertaine nous maintenons la sortie et nous nous retrouvons sur le parking. Bien chaussées ,équipées de nos sacs à dos remplis d'outils(au cas ou...) et de victuailles ,nous nous élançons à la recherche des Belles...

Après la pause pique nique bien agrémentée de géoblabla nous reprenons la route.Deux moldus s'occupent de leur bateau et ne prêtent pas attention à notre présence.Tant mieux!!!Nous loguons sans difficulté.Merci pour la cache.\end{cacheText}

\cacheNumber{1154}\needspace{5\baselineskip}\cacheName{\href{http://coord.info/GC4DKQ2}{Tour du Lac du GABAS ~ 31} — \href{http://coord.info/GC4DKQ2\Number{}747917196}{1154}}\cacheData{{2018/02/21 MaryeLoup, Traditional Cache (1.5/2)}}\begin{cacheText}Le rendez vous est pris depuis un moment avec Bigorra65 pour faire ce tour du Lac de Gabas.Malgré une météo incertaine nous maintenons la sortie et nous nous retrouvons sur le parking. Bien chaussées ,équipées de nos sacs à dos remplis d'outils(au cas ou...) et de victuailles ,nous nous élançons à la recherche des Belles...

Arrivées sur le PZ, nous avons beau chercher il n'y a plus de cache!!!Nous procédons donc à une maintenance dans les règles...merci\end{cacheText}

\cacheNumber{1155}\needspace{5\baselineskip}\cacheName{\href{http://coord.info/GC4DKQ5}{Tour du Lac du GABAS ~ 32} — \href{http://coord.info/GC4DKQ5\Number{}747918623}{1155}}\cacheData{{2018/02/21 MaryeLoup, Traditional Cache (1.5/2)}}\begin{cacheText}Le rendez vous est pris depuis un moment avec Bigorra65 pour faire ce tour du Lac de Gabas.Malgré une météo incertaine nous maintenons la sortie et nous nous retrouvons sur le parking. Bien chaussées ,équipées de nos sacs à dos remplis d'outils(au cas ou...) et de victuailles ,nous nous élançons à la recherche des Belles...

Le plus compliqué est de retrouver les 4 pieds car le GPS s'affole.Une fois repéré les arbres la Belle est vite trouvée.Merci\end{cacheText}

\cacheNumber{1156}\needspace{5\baselineskip}\cacheName{\href{http://coord.info/GC4DKQE}{Tour du Lac du GABAS ~ 34} — \href{http://coord.info/GC4DKQE\Number{}747919651}{1156}}\cacheData{{2018/02/21 MaryeLoup, Traditional Cache (1.5/2)}}\begin{cacheText}Le rendez vous est pris depuis un moment avec Bigorra65 pour faire ce tour du Lac de Gabas.Malgré une météo incertaine nous maintenons la sortie et nous nous retrouvons sur le parking. Bien chaussées ,équipées de nos sacs à dos remplis d'outils(au cas ou...) et de victuailles ,nous nous élançons à la recherche des Belles...

Ici la vue est superbe... surtout avec le soleil.Le toboggan,grandiose ,nous interpelle mais nous nous concentrons sur la recherche de la Belle.Le logbook est rempli:nous le remplaçons.Merci pour la cache.\end{cacheText}

\cacheNumber{1157}\needspace{5\baselineskip}\cacheName{\href{http://coord.info/GC4DKQG}{Tour du Lac du GABAS ~ 35} — \href{http://coord.info/GC4DKQG\Number{}747920347}{1157}}\cacheData{{2018/02/21 MaryeLoup, Traditional Cache (1.5/2)}}\begin{cacheText}Le rendez vous est pris depuis un moment avec Bigorra65 pour faire ce tour du Lac de Gabas.Malgré une météo incertaine nous maintenons la sortie et nous nous retrouvons sur le parking. Bien chaussées ,équipées de nos sacs à dos remplis d'outils(au cas ou...) et de victuailles ,nous nous élançons à la recherche des Belles...

Aidées de l'indice et de la photo nous n'avons pas de mal à repérer la Belle.Nous avons juste eu le temps de la remettre en place lorsque des moldus ont fait leur apparition sur le parking!!!Oufff!!!Merci\end{cacheText}

\cacheNumber{1158}\needspace{5\baselineskip}\cacheName{\href{http://coord.info/GC4Y1TM}{CR1 06 Chiberta - Balade du Pignada} — \href{http://coord.info/GC4Y1TM\Number{}748000913}{1158}}\cacheData{{2018/02/22 patricia64, Traditional Cache (1.5/1.5)}}\begin{cacheText}Le moral remonte… La cache est évidente à trouver ici. Merci Patricia.\end{cacheText}

\cacheNumber{1159}\needspace{5\baselineskip}\cacheName{\href{http://coord.info/GC76DT5}{CR2 01 Chiberta - Ballade du Pignada} — \href{http://coord.info/GC76DT5\Number{}747998595}{1159}}\cacheData{{2018/02/22 gilles64, Traditional Cache (1.5/1.5)}}\begin{cacheText}(STF)

Le nez dans le laurier(quelle bonne odeur!!), nous délogeons la Belle en deux temps trois mouvements. Merci pour la cache.\end{cacheText}

\cacheNumber{1160}\needspace{5\baselineskip}\cacheName{\href{http://coord.info/GC76GZM}{CR2 02 Chiberta - Ballade du Pignada} — \href{http://coord.info/GC76GZM\Number{}747998833}{1160}}\cacheData{{2018/02/22 gilles64, Traditional Cache (1.5/1.5)}}\begin{cacheText}(STF)

Malgré un GPS qui s'affole, nous découvrons la belle. Il est vrai que l'endroit s'y prête bien. Merci pour la cache.\end{cacheText}

\cacheNumber{1161}\needspace{5\baselineskip}\cacheName{\href{http://coord.info/GC76GZN}{CR2 03 Chiberta - Ballade du Pignada} — \href{http://coord.info/GC76GZN\Number{}748000114}{1161}}\cacheData{{2018/02/22 gilles64, Traditional Cache (1.5/1.5)}}\begin{cacheText}(STF)

Ici les choses se corsent ! Le GPS s'affole et nous promène de droite à gauche. En persévérant et grâce à la photo ,nous arrivons tout de même à la trouver. Merci Gilles pour ce bon moment.\end{cacheText}

\cacheNumber{1162}\needspace{5\baselineskip}\cacheName{\href{http://coord.info/GC76H04}{CR2 09 Chiberta - Ballade du Pignada} — \href{http://coord.info/GC76H04\Number{}748041272}{1162}}\cacheData{{2018/02/22 gilles64, Traditional Cache (1.5/1.5)}}\begin{cacheText}( STF)

La cache est délogée assez facilement... heureusement car mon compère commence à s'impatienter .Merci pour la cache.\end{cacheText}

\cacheNumber{1163}\needspace{5\baselineskip}\cacheName{\href{http://coord.info/GC76H0H}{CR2 13 Chiberta - Ballade du Pignada} — \href{http://coord.info/GC76H0H\Number{}747991452}{1163}}\cacheData{{2018/02/22 gilles64, Traditional Cache (1.5/1.5)}}\begin{cacheText}(STF)

Sur Bayonne pour faire des achats je convaincs mon petit dernier de faire quelques caches avant de rentrer. Arrivés sur les lieux, pas de difficultés pour trouver la belle.Elle est bien camouflée et, belle surprise ,nous sommes STF. Merci pour la cache.\end{cacheText}

\cacheNumber{1164}\needspace{5\baselineskip}\cacheName{\href{http://coord.info/GC76H0M}{CR2 14 Chiberta - Ballade du Pignada} — \href{http://coord.info/GC76H0M\Number{}747994684}{1164}}\cacheData{{2018/02/22 gilles64, Traditional Cache (1.5/1.5)}}\begin{cacheText}(STF)

En arrivant sur les lieux, nous commençons à chercher et grâce à l'indice nous voyons la cache. Nous laissons passer quelques promeneurs  et reprenons notre parcours. Merci\end{cacheText}

\cacheNumber{1165}\needspace{5\baselineskip}\cacheName{\href{http://coord.info/GC76H0T}{CR2 15 Chiberta - Ballade du Pignada} — \href{http://coord.info/GC76H0T\Number{}747997753}{1165}}\cacheData{{2018/02/22 gilles64, Traditional Cache (1.5/1.5)}}\begin{cacheText}(TTF)

Arrivés sur le PZ nous laissons passer quelques moldus puis nous nous laissons guider par l' indice. La cache nous attend bien sagement. Merci pour la ballade.\end{cacheText}

\cacheNumber{1166}\needspace{5\baselineskip}\cacheName{\href{http://coord.info/GC76H0W}{CR2 16 Chiberta - Ballade du Pignada} — \href{http://coord.info/GC76H0W\Number{}747998331}{1166}}\cacheData{{2018/02/22 gilles64, Traditional Cache (1.5/1.5)}}\begin{cacheText}(TTF)

Le temps se prête vraiment à cette promenade super sympa au milieu des pins… Que des pins !!!! La cache nous attend bien sagement: la photo et l'indice nous guide bien. Merci\end{cacheText}

\cacheNumber{1167}\needspace{5\baselineskip}\cacheName{\href{http://coord.info/GC5D0EZ}{[BSD] \Number{}047} — \href{http://coord.info/GC5D0EZ\Number{}748042928}{1167}}\cacheData{{2018/02/23 GéoLandesTour, Traditional Cache (1.5/1.5)}}\begin{cacheText}De passage sur Dax, j'en profite pour faire une ou deux caches. Après le DNF de la cache de Mizaga je suis heureuse de trouver celle-ci. Pas de passage, c'est facile de loguer. Merci pour la cache\end{cacheText}

\cacheNumber{1168}\needspace{5\baselineskip}\cacheName{\href{http://coord.info/GC5D0F9}{[BSD] \Number{}049} — \href{http://coord.info/GC5D0F9\Number{}748043488}{1168}}\cacheData{{2018/02/23 GéoLandesTour, Traditional Cache (1.5/1.5)}}\begin{cacheText}De passage sur Dax j'en profite pour faire quelques caches et continuer la boucle. Celle-ci est bien à sa place. L'indice et la photo me guident. Merci pour la cache.\end{cacheText}

\cacheNumber{1169}\needspace{5\baselineskip}\cacheName{\href{http://coord.info/GC6WFXD}{Meeting Point \Number{}4 - France/Portugal: Les voyageurs} — \href{http://coord.info/GC6WFXD\Number{}748046336}{1169}}\cacheData{{2018/02/23 mizaga, Unknown Cache (3/2)}}\begin{cacheText}Medieval Team/Fleura64



Contactés par la Medieval Team nous nous intéressons d'un peu plus près à cette cache. Après avoir décrypté et trouvé les différentes coordonnées ( en France et au Portugal ), nous finissons par nous rendre à la cache finale.Le temps passe et toujours RIEN !!!C' est grâce au message envoyé au FTF français que nous arrivons à avoir le détail qui nous permet de mettre la main sur le précieux trésor.Merci les mizaga pour cette super cache collaborative et pour tout ce travail de pose. Un PF bien mérité.\end{cacheText}

\cacheNumber{1170}\needspace{5\baselineskip}\cacheName{\href{http://coord.info/GC42HZR}{Le Pavillon Chinois} — \href{http://coord.info/GC42HZR\Number{}748206708}{1170}}\cacheData{{2018/02/24 gilles64, Unknown Cache (3/2)}}\begin{cacheText}Le petit ayant un match de rugby dans le coin, j'en profite pour faire quelques caches avant le coup d'envoi. Me revoilà donc dans cette forêt de Chiberta à la recherche de cette mystérieuse cache. Après avoir tourné et retourné dans les bambous enfin j'arrive sur le PZ. Après une bonne recherche active, je finis par mettre la main sur le Graal superbement camouflé. Très beau travail pour cette mystère. Merci Gilles64 et un Pf évidemment.\end{cacheText}

\cacheNumber{1171}\needspace{5\baselineskip}\cacheName{\href{http://coord.info/GC4Y1T8}{CR1 04 Chiberta - Balade du Pignada} — \href{http://coord.info/GC4Y1T8\Number{}748217675}{1171}}\cacheData{{2018/02/24 patricia64, Traditional Cache (1.5/1.5)}}\begin{cacheText}La cache est, comme le dit la chanson… adossée à un chêne-liège. Pas de problème.Merci pour la cache\end{cacheText}

\cacheNumber{1172}\needspace{5\baselineskip}\cacheName{\href{http://coord.info/GC4Y1TC}{CR1 05 Chiberta - Balade du Pignada} — \href{http://coord.info/GC4Y1TC\Number{}748218212}{1172}}\cacheData{{2018/02/24 patricia64, Traditional Cache (1.5/1.5)}}\begin{cacheText}Arrivé sur le PZ, le GPS s'affole. Il me promène et je suis donc obligée de faire tous les pieds de pin de la zone. Je fini par la trouver. Merci.\end{cacheText}

\cacheNumber{1173}\needspace{5\baselineskip}\cacheName{\href{http://coord.info/GC5WH9X}{\Number{}L2-01 Bardos - Le Comté du Labourd} — \href{http://coord.info/GC5WH9X\Number{}748226101}{1173}}\cacheData{{2018/02/24 gilles64, Multi-cache (2/1.5)}}\begin{cacheText}Il y a longtemps que je voulais faire cette Multi mais j'attendais d'avoir un peu de temps. Au retour de Bayonne je m'arrête tranquillement pour relever les indices, faire les calculs et arriver à trouver la belle. Superbe point de vue. Merci Gilles pour cette petite ballade.\end{cacheText}

\cacheNumber{1174}\needspace{5\baselineskip}\cacheName{\href{http://coord.info/GC6V8YA}{La passerelle de Lahonce} — \href{http://coord.info/GC6V8YA\Number{}748222305}{1174}}\cacheData{{2018/02/24 DorisBear, Traditional Cache (2/1.5)}}\begin{cacheText}C'est en revenant de voir mon fiston jouer au rugby que je m'arrête faire cette cache qui est encore verte sur la carte. Arrivée sur le PZ, le GPS  me désigne un endroit. Tiens bizarre... pas grand endroit pour cacher!!! Et pourtant, oui c'est bien là. Encore une très bonne idée de nos amis les ours. Merci pour la découverte de cette jolie passerelle.\end{cacheText}

\cacheNumber{1175}\needspace{5\baselineskip}\cacheName{\href{http://coord.info/GC76H01}{CR2 08 Chiberta - Ballade du Pignada} — \href{http://coord.info/GC76H01\Number{}748216026}{1175}}\cacheData{{2018/02/24 gilles64, Traditional Cache (1.5/1.5)}}\begin{cacheText}Ici, j'ai cherché un bon moment. Cette cache m'a donné un peu plus de mal que les autres. La persévérance finit par payer j'ai enfin trouvé la Belle. Merci\end{cacheText}

\cacheNumber{1176}\needspace{5\baselineskip}\cacheName{\href{http://coord.info/GC76H09}{CR2 10 Chiberta - Ballade du Pignada} — \href{http://coord.info/GC76H09\Number{}748209435}{1176}}\cacheData{{2018/02/24 gilles64, Traditional Cache (1.5/1.5)}}\begin{cacheText}Arrivée sur le PZ,  la cache est rapidement localisée. En effet, la photo et les indices sont très précis. Merci Gilles.\end{cacheText}

\cacheNumber{1177}\needspace{5\baselineskip}\cacheName{\href{http://coord.info/GC76H0B}{CR2 11 Chiberta - Ballade du Pignada} — \href{http://coord.info/GC76H0B\Number{}748211910}{1177}}\cacheData{{2018/02/24 gilles64, Traditional Cache (1.5/1.5)}}\begin{cacheText}La promenade se poursuit et je trouve la cache facilement. Elle m' attend bien sagement dans sa souche. À zut j'ai failli oublier… C'est bon je peux repartir. Merci pour la cache.\end{cacheText}

\cacheNumber{1178}\needspace{5\baselineskip}\cacheName{\href{http://coord.info/GC76H0E}{CR2 12 Chiberta - Ballade du Pignada} — \href{http://coord.info/GC76H0E\Number{}748212438}{1178}}\cacheData{{2018/02/24 gilles64, Traditional Cache (1.5/1.5)}}\begin{cacheText}Arrivée sur la zone, le GPS me mène à la cache . J' y trouve un très joli arbre envahi de chenilles.Merci pour la cache.\end{cacheText}

\cacheNumber{1179}\needspace{5\baselineskip}\cacheName{\href{http://coord.info/GC7J94J}{CR2 08a Chiberta - Ballade du Pignada} — \href{http://coord.info/GC7J94J\Number{}748213848}{1179}}\cacheData{{2018/02/24 gilles64, Traditional Cache (1.5/1.5)}}\begin{cacheText}Ici aussi, je découvre une cache bien camouflée. Quelque promeneurs… j'attend pour sortir, signer et remettre en place en toute sérénité. Merci pour la ballade.\end{cacheText}

\cacheNumber{1180}\needspace{5\baselineskip}\cacheName{\href{http://coord.info/GC4Y17T}{CR1 01 Chiberta - Balade du Pignada} — \href{http://coord.info/GC4Y17T\Number{}748504701}{1180}}\cacheData{{2018/02/25 patricia64, Traditional Cache (1.5/1.5)}}\begin{cacheText}Arrivée sur la zone, le GPS n'est pas très précis. Me voilà donc à faire le tour de tous les pins pour enfin débusquer la Belle. Merci.\end{cacheText}

\cacheNumber{1181}\needspace{5\baselineskip}\cacheName{\href{http://coord.info/GC4Y1RR}{CR1 02 Chiberta - Balade du Pignada} — \href{http://coord.info/GC4Y1RR\Number{}748502279}{1181}}\cacheData{{2018/02/25 patricia64, Traditional Cache (1.5/1.5)}}\begin{cacheText}La cache est bien à l'abri du chêne liège. Pas de difficultés pour la trouver, le log book est bien remplie preuve du succès de cette super promenade. Merci Patricia pour le circuit.\end{cacheText}

\cacheNumber{1182}\needspace{5\baselineskip}\cacheName{\href{http://coord.info/GC4Y1T4}{CR1 03 Chiberta - Balade du Pignada} — \href{http://coord.info/GC4Y1T4\Number{}748504157}{1182}}\cacheData{{2018/02/25 patricia64, Traditional Cache (2/1.5)}}\begin{cacheText}Arrivée au PZ, je découvre une vue superbe de l'Adour. La cache est vite repérée et signée. Je suis obligée d'attendre: des moldus admirent le paysage. Merci pour la cache.\end{cacheText}

\cacheNumber{1183}\needspace{5\baselineskip}\cacheName{\href{http://coord.info/GC4Y1TT}{CR1 07 Chiberta - Balade du Pignada} — \href{http://coord.info/GC4Y1TT\Number{}748500733}{1183}}\cacheData{{2018/02/25 patricia64, Traditional Cache (2.5/1.5)}}\begin{cacheText}Je décide de repartir sur cette cache qui m'avait posé problème la dernière fois. À force de persévérer je finis par mettre la main dessus. Les yeux ne suffisent pas ...il faut s'aider des mains!!! Merci pour la cache.\end{cacheText}

\cacheNumber{1184}\needspace{5\baselineskip}\cacheName{\href{http://coord.info/GC4Y1TX}{CR1 08 Chiberta - Balade du Pignada} — \href{http://coord.info/GC4Y1TX\Number{}748500104}{1184}}\cacheData{{2018/02/25 patricia64, Traditional Cache (1.5/1.5)}}\begin{cacheText}Le spoiler et le Hint sont extrêmement précis. La cache est rapidement trouvée. Merci Patricia pour ce joli parcours.\end{cacheText}

\cacheNumber{1185}\needspace{5\baselineskip}\cacheName{\href{http://coord.info/GC4Y1V2}{CR1 09 Chiberta - Balade du Pignada} — \href{http://coord.info/GC4Y1V2\Number{}748498803}{1185}}\cacheData{{2018/02/25 patricia64, Traditional Cache (1.5/1.5)}}\begin{cacheText}Arrivée sur le PZ, les lauriers sont vite identifiés : la cache m'attend au pied. Quelques moldus profitent de ce dimanche pour promener leur chien ,j'attend qu'ils passent et reprend ma route.Merci pour cette ballade.\end{cacheText}

\cacheNumber{1186}\needspace{5\baselineskip}\cacheName{\href{http://coord.info/GC4Y1V7}{CR1 10 Chiberta - Balade du Pignada} — \href{http://coord.info/GC4Y1V7\Number{}748496616}{1186}}\cacheData{{2018/02/25 patricia64, Traditional Cache (2/1.5)}}\begin{cacheText}Aujourd'hui il fait encore beau… Pas question de laisser le circuit en plan. Je reprends la route vers ces caches qu'il me manque pour trouver les bonus. Ici le GPS me mène tout droit à la Belle. Merci pour la cache\end{cacheText}

\cacheNumber{1187}\needspace{5\baselineskip}\cacheName{\href{http://coord.info/GC6J9CK}{CR1 02a Chiberta - Ballade du Pignada} — \href{http://coord.info/GC6J9CK\Number{}748503147}{1187}}\cacheData{{2018/02/25 patricia64 gilles64, Traditional Cache (2/2)}}\begin{cacheText}Arrivée sur les lieux, je constate que les forestiers ont déblayé la zone. Mais le chêne reste un bois noble, il est toujours là abritant la Belle. Superbe vue de la cathédrale (?) au loin. Merci pour la cache .\end{cacheText}

\cacheNumber{1188}\needspace{5\baselineskip}\cacheName{\href{http://coord.info/GC76GZR}{CR2 04 Chiberta - Ballade du Pignada} — \href{http://coord.info/GC76GZR\Number{}748499754}{1188}}\cacheData{{2018/02/25 gilles64, Traditional Cache (2/1.5)}}\begin{cacheText}Ici, la cache est repérée grâce à l'indice. Effectivement ,un arbre à bouchons au milieu des pins... cela se voit! Merci pour la cache.\end{cacheText}

\cacheNumber{1189}\needspace{5\baselineskip}\cacheName{\href{http://coord.info/GC76GZV}{CR2 05 Chiberta - Ballade du Pignada} — \href{http://coord.info/GC76GZV\Number{}748499290}{1189}}\cacheData{{2018/02/25 gilles64, Traditional Cache (1.5/1.5)}}\begin{cacheText}En arrivant sur le PZ, on pourrait se dire que la cache a disparu : des travaux forestiers ont été effectués récemment. Mais non, elle est bien à sa place au pied des\Quoted{bouchonnées}. Merci pour la cache.\end{cacheText}

\cacheNumber{1190}\needspace{5\baselineskip}\cacheName{\href{http://coord.info/GC76GZY}{CR2 06 Chiberta - Ballade du Pignada} — \href{http://coord.info/GC76GZY\Number{}748501266}{1190}}\cacheData{{2018/02/25 gilles64, Traditional Cache (1.5/1.5)}}\begin{cacheText}Ici quelques promeneurs profitent du dimanche. Je patiente pour loguer en toute tranquillité. Merci pour la cache.\end{cacheText}

\cacheNumber{1191}\needspace{5\baselineskip}\cacheName{\href{http://coord.info/GC76GZZ}{CR2 07 Chiberta - Ballade du Pignada} — \href{http://coord.info/GC76GZZ\Number{}748502011}{1191}}\cacheData{{2018/02/25 gilles64, Traditional Cache (1.5/1.5)}}\begin{cacheText}La cache est rapidement découverte. Il est difficile de ne pas voir le laurier!!! C'est la dernière cache qu' il me manquait. Je calcule les coordonnées pour aller vers la bonus ...le checker passe au vert ouffff. Merci Gilles.\end{cacheText}

\cacheNumber{1192}\needspace{5\baselineskip}\cacheName{\href{http://coord.info/GC76H1F}{CR2 17 Bonus Chiberta - Ballade du Pignada} — \href{http://coord.info/GC76H1F\Number{}748505820}{1192}}\cacheData{{2018/02/25 gilles64, Unknown Cache (3/1.5)}}\begin{cacheText}Tous les indices en poche, je calcule les coordonnées de la bonus. Et me revoilà partie sur les sentiers. La Belle est vite repérée grâce à l'indice. Merci Gilles pour tout ce travail et cette super ballade à travers les pins.Un PF évidemment pour récompense car je me suis régalée!!!\end{cacheText}

\cacheNumber{1193}\needspace{5\baselineskip}\cacheName{\href{http://coord.info/GC5X2DM}{Lavoir et Fontaine Notre Dame} — \href{http://coord.info/GC5X2DM\Number{}749215853}{1193}}\cacheData{{2018/03/03 mizaga, Traditional Cache (1.5/1.5)}}\begin{cacheText}La cache est trouvée très facilement. Quelques moldus se promènent, j'attends qu'ils passent et je logue. Cet endroit est magnifique. Merci pour la découverte des lieux. Évidemment il n'y a pas d'objet voyageur dans la boîte. C'est vraiment dommage. Merci.\end{cacheText}

\cacheNumber{1194}\needspace{5\baselineskip}\cacheName{\href{http://coord.info/GC6F7JK}{La Vélodyssée} — \href{http://coord.info/GC6F7JK\Number{}749221522}{1194}}\cacheData{{2018/03/03 Eriick, Traditional Cache (1.5/1.5)}}\begin{cacheText}De passage pour voir mon fils jouer un match de rugby sur Léon, j'en profite pour faire quelques caches aux alentours. L'indice et la photo me mènent tout droit à la cache. Super cache. Merci pour le travail.\end{cacheText}

\cacheNumber{1195}\needspace{5\baselineskip}\cacheName{\href{http://coord.info/GC6JX4E}{H001- Géodyssée 40-64} — \href{http://coord.info/GC6JX4E\Number{}749225478}{1195}}\cacheData{{2018/03/03 mizaga, Traditional Cache (1.5/1.5)}}\begin{cacheText}La belle nous attend bien à sa place à l'entrée de la Géodyssée. Parfaitement au sec, j'arrive à signer sans problème.Les indices sont tous relevés ,à l'occasion j'irais signer la Bonus. Merci les Mizaga pour cette belle promenade.\end{cacheText}

\cacheNumber{1196}\needspace{5\baselineskip}\cacheName{\href{http://coord.info/GC6JX4M}{H002- Géodyssée 40-64} — \href{http://coord.info/GC6JX4M\Number{}749224704}{1196}}\cacheData{{2018/03/03 mizaga, Traditional Cache (1.5/1.5)}}\begin{cacheText}La cache est rapidement découverte. Malheureusement ici aussi le log book est détrempé .La maintenance serait nécessaire. Merci pour la cache.\end{cacheText}

\cacheNumber{1197}\needspace{5\baselineskip}\cacheName{\href{http://coord.info/GC6JX51}{H003- Géodyssée 40-64} — \href{http://coord.info/GC6JX51\Number{}749220905}{1197}}\cacheData{{2018/03/03 mizaga, Traditional Cache (1.5/1.5)}}\begin{cacheText}C' est sans trop de mal que je découvre la Belle. Ici aussi le log book est rempli et trempé mais… En fouillant dans la boîte à gants, j'ai trouvé un log book de remplacement ! Ouf je peux ainsi assurer la maintenance. Merci les Mizaga pour la cache.\end{cacheText}

\cacheNumber{1198}\needspace{5\baselineskip}\cacheName{\href{http://coord.info/GC6JX59}{H004- Géodyssée 40-64} — \href{http://coord.info/GC6JX59\Number{}749218822}{1198}}\cacheData{{2018/03/03 mizaga, Traditional Cache (1.5/1.5)}}\begin{cacheText}La cache est vite trouvée. Le log book est rempli et inutilisable .Quelqu'un a laissé sa liste de courses : je signe dans la continuité. Je râle de ne pas avoir le nécessaire pour la maintenance... désoléE. Merci pour la cache\end{cacheText}

\cacheNumber{1199}\needspace{5\baselineskip}\cacheName{\href{http://coord.info/GC6JX5P}{H006- Géodyssée 40-64} — \href{http://coord.info/GC6JX5P\Number{}749218182}{1199}}\cacheData{{2018/03/03 mizaga, Traditional Cache (1.5/1.5)}}\begin{cacheText}Cache trouvée malgré des voisins un peu curieux. Beaucoup de monde sur la Géodyssée par ce bel après-midi ensoleillé. Je continue ma recherche. Merci pour la cache.\end{cacheText}

\cacheNumber{1200}\needspace{5\baselineskip}\cacheName{\href{http://coord.info/GC6JXN5}{H007- Géodyssée 40-64} — \href{http://coord.info/GC6JXN5\Number{}749214031}{1200}}\cacheData{{2018/03/03 mizaga, Traditional Cache (1.5/1.5)}}\begin{cacheText}Pas de difficulté pour trouver la boîte. Mais le log book est quasiment complet il va falloir penser à une maintenance. Merci pour la cache\end{cacheText}

\cacheNumber{1201}\needspace{5\baselineskip}\cacheName{\href{http://coord.info/GC6JXN6}{H008- Géodyssée 40-64} — \href{http://coord.info/GC6JXN6\Number{}749213456}{1201}}\cacheData{{2018/03/03 mizaga, Traditional Cache (1.5/1.5)}}\begin{cacheText}Nous avons cherché un moment autour du PZ et nous avons fini par la trouver parterre. Le log book est également remplie et très humide je n'ai plus de quoi faire de maintenance.  Merci pour la cache.\end{cacheText}

\cacheNumber{1202}\needspace{5\baselineskip}\cacheName{\href{http://coord.info/GC6JXN8}{H009- Géodyssée 40-64} — \href{http://coord.info/GC6JXN8\Number{}749212986}{1202}}\cacheData{{2018/03/03 mizaga, Traditional Cache (1.5/1.5)}}\begin{cacheText}Ici aussi, la cache est localisée très facilement grâce à l'indice et la photo.Le Log book est plein( preuve du succés de la géodyssée) je me permets donc de le remplacer.Ne cherchez pas, il n'y a pas d'indices sur cette cache. Merci\end{cacheText}

\cacheNumber{1203}\needspace{5\baselineskip}\cacheName{\href{http://coord.info/GC6JXNA}{H010- Géodyssée 40-64} — \href{http://coord.info/GC6JXNA\Number{}749210958}{1203}}\cacheData{{2018/03/03 mizaga, Traditional Cache (1.5/1.5)}}\begin{cacheText}Le Hint et le spoiler me mènent tout droit à la Belle qui patiente à l'abri. Indice relevé, je repars vers d'autres cache. Merci\end{cacheText}

\cacheNumber{1204}\needspace{5\baselineskip}\cacheName{\href{http://coord.info/GC6JXNC}{H011- Géodyssée 40-64} — \href{http://coord.info/GC6JXNC\Number{}749205959}{1204}}\cacheData{{2018/03/03 mizaga, Traditional Cache (1.5/1.5)}}\begin{cacheText}En route pour aller voir le petit qui joue un match de rugby à Léon nous reprenons le circuit de la Géodyssée que nous avions entamé il y a quelques temps. La cache est rapidement trouvée mais malheureusement il nous a fallu attendre car des moldus arrivaient pour faire du vélo. Je patiente pour remettre  la Belle en place et reprenons la route. Merci pour la cache.\end{cacheText}

\cacheNumber{1205}\needspace{5\baselineskip}\cacheName{\href{http://coord.info/GC4VPZP}{Saubusse, l'eglise} — \href{http://coord.info/GC4VPZP\Number{}749477031}{1205}}\cacheData{{2018/03/04 kar@melos40, Traditional Cache (1.5/1.5)}}\begin{cacheText}Aucune difficulté pour déloger la belle. Je découvre une très belle église qui malheureusement est en train de s'effondrer. Heureusement les travaux sont en cours.  Le logbook est archi plein il, faudrait effectuer une maintenance. Merci pour la cache.\end{cacheText}

\cacheNumber{1206}\needspace{5\baselineskip}\cacheName{\href{http://coord.info/GC5JY01}{Orist : le lavoir / the wash house} — \href{http://coord.info/GC5JY01\Number{}749477705}{1206}}\cacheData{{2018/03/04 dorisbear, Traditional Cache (2/2)}}\begin{cacheText}Et voilà ,encore une cache qui me fait tourner en rond!!! Une cache qui est parfaitement intégrée. Nul besoin de s'enfoncer très loin dans le sous-bois… Merci pour la découverte de ce joli lavoir qui mériterait un peu plus d'entretien. Merci les amis pour ce super moment.\end{cacheText}

\cacheNumber{1207}\needspace{5\baselineskip}\cacheName{\href{http://coord.info/GC65ZY9}{Covoiturage Rivière / Carpooling Rivière} — \href{http://coord.info/GC65ZY9\Number{}749474589}{1207}}\cacheData{{2018/03/04 dorisbear, Traditional Cache (2/1.5)}}\begin{cacheText}Toujours de belles réalisations avec les Dorisbear , du beau travail!!! Arrivée prés du PZ  je repère quelque chose qui pourrait accueillir la Belle.Bingo mais je suis obligée de patienter car des personnes discutent sur le parking. Heureusement car je n'en finis pas de tirer!!!!Pourvu qu'aucune voiture ne passe.....ouff je finis par loguer et je dois recommencer l'opération en sens inverse...J'ai eu de la chance car personne n'est passé et je peux reprendre ma route. Merci pour cette super cache.\end{cacheText}

\cacheNumber{1208}\needspace{5\baselineskip}\cacheName{\href{http://coord.info/GC27K99}{La tour de Bordagain} — \href{http://coord.info/GC27K99\Number{}749802299}{1208}}\cacheData{{2018/03/06 Peyo64, Traditional Cache (2/2)}}\begin{cacheText}Dans le coin pour raisons professionnelles ,Monsieur en profite pour dénicher la Belle.L' endroit est superbe .Merci pour la cache\end{cacheText}

\cacheNumber{1209}\needspace{5\baselineskip}\cacheName{\href{http://coord.info/GC66V9P}{La Chapelle oubliée !} — \href{http://coord.info/GC66V9P\Number{}750701794}{1209}}\cacheData{{2018/03/09 gilles64, Traditional Cache (2.5/1.5)}}\begin{cacheText}Sur le chemin de la géodyssée,je vois cette petite tache verte sur la carte... direction la chapelle oubliée. Au départ je n'ose pas rentrer puis finalement je me faufile derrière les grilles.Personne à l'horizon, je peux chercher tranquillement cette super cache parfaitement intégrée. Merci  Gilles pour la découverte de cette chapelle oubliée qui visiblement va être restaurée.Un PF pour l'endroit et la cache.\end{cacheText}

\cacheNumber{1210}\needspace{5\baselineskip}\cacheName{\href{http://coord.info/GC66WQM}{Les Baïnes de Labenne} — \href{http://coord.info/GC66WQM\Number{}750715427}{1210}}\cacheData{{2018/03/09 Gamboy, Earthcache (1.5/1.5)}}\begin{cacheText}Après avoir lu le panneau sur les dunes ,je m’intéresse à celui des baïnes.Le sujet est fort instructif .Merci pour la Earthcache.\end{cacheText}

\cacheNumber{1211}\needspace{5\baselineskip}\cacheName{\href{http://coord.info/GC66WRD}{Dune Land} — \href{http://coord.info/GC66WRD\Number{}750706261}{1211}}\cacheData{{2018/03/09 Gamboy, Earthcache (1.5/1.5)}}\begin{cacheText}Petite journée de repos, j'en profite pour compléter la géodyssée.Un petit écart me permet d'en apprendre un peu plus sur les dunes.Merci les Gamboy pour cette earthcache très instructive.\end{cacheText}

\cacheNumber{1212}\needspace{5\baselineskip}\cacheName{\href{http://coord.info/GC6J28D}{H028 BONUS- Géodyssée 40-64} — \href{http://coord.info/GC6J28D\Number{}750690042}{1212}}\cacheData{{2018/03/09 mizaga, Unknown Cache (1.5/2)}}\begin{cacheText}Aujourd'hui, c'est jour de repos et le soleil est de la partie. Direction Capbreton pour trouver cette belle bonus. Le GPS est hyper précis et la photo aide bien.Indice relevé pour la super bonus et direction les autres caches. Merci les mizaga et 1 PF pour l'ensemble de la série.\end{cacheText}

\cacheNumber{1213}\needspace{5\baselineskip}\cacheName{\href{http://coord.info/GC6VYJZ}{I001	 - Géodyssée 40 64} — \href{http://coord.info/GC6VYJZ\Number{}750691061}{1213}}\cacheData{{2018/03/09 gilles64, Traditional Cache (1.5/1.5)}}\begin{cacheText}Après avoir fait plusieurs fois le tour du rond-point et inspecté tous les panneaux je finis par trouver la belle. Je n'avais pas les yeux bien ouverts!!!!Merci pour la cache.\end{cacheText}

\cacheNumber{1214}\needspace{5\baselineskip}\cacheName{\href{http://coord.info/GC6VYKC}{I003	 - Géodyssée 40 64} — \href{http://coord.info/GC6VYKC\Number{}750692098}{1214}}\cacheData{{2018/03/09 gilles64, Traditional Cache (1.5/1.5)}}\begin{cacheText}Arrivée sur le PZ, les yeux grands ouverts, je découvre La cache en deux temps trois mouvements.Oufff!!! Merci Gilles\end{cacheText}

\cacheNumber{1215}\needspace{5\baselineskip}\cacheName{\href{http://coord.info/GC6VYKG}{I004	 - Géodyssée 40 64} — \href{http://coord.info/GC6VYKG\Number{}750694669}{1215}}\cacheData{{2018/03/09 gilles64, Traditional Cache (1.5/1.5)}}\begin{cacheText}Peu de moldus ,en cette fin de matinée, fréquente la vélodyssée. Tant mieux, je peux chercher tranquille. La Belle est bien à sa place.Merci pour la promenade.\end{cacheText}

\cacheNumber{1216}\needspace{5\baselineskip}\cacheName{\href{http://coord.info/GC6VYKQ}{I005	 - Géodyssée 40 64} — \href{http://coord.info/GC6VYKQ\Number{}750695203}{1216}}\cacheData{{2018/03/09 gilles64, Traditional Cache (1.5/1.5)}}\begin{cacheText}Pas de soucis pour cette cache, elle est bien à sa place entre le liège et le lierre. Merci.\end{cacheText}

\cacheNumber{1217}\needspace{5\baselineskip}\cacheName{\href{http://coord.info/GC6VYM0}{I006	 - Géodyssée 40 64} — \href{http://coord.info/GC6VYM0\Number{}750697939}{1217}}\cacheData{{2018/03/09 gilles64, Traditional Cache (1.5/1.5)}}\begin{cacheText}Il commençe à pleuvoir mais je continue ma route: je ne me décourage pas.Je croise plusieurs moldus qui pressent le pas. Arrivée sur le PZ pas de difficultés mais au moment de remettre la Belle en place des moldus!!!!! Je patiente et je repars. Merci pour la belle.\end{cacheText}

\cacheNumber{1218}\needspace{5\baselineskip}\cacheName{\href{http://coord.info/GC6VYMB}{I007	 - Géodyssée 40 64} — \href{http://coord.info/GC6VYMB\Number{}750695984}{1218}}\cacheData{{2018/03/09 gilles64, Traditional Cache (1.5/1.5)}}\begin{cacheText}Arrivée sur le PZ, personne en vue!!! Il est facile de chercher la Belle car l'indice et la photo sont très clairs. Merci pour la cache.\end{cacheText}

\cacheNumber{1219}\needspace{5\baselineskip}\cacheName{\href{http://coord.info/GC6VYMR}{I008	 - Géodyssée 40 64} — \href{http://coord.info/GC6VYMR\Number{}750699068}{1219}}\cacheData{{2018/03/09 gilles64, Traditional Cache (1.5/1.5)}}\begin{cacheText}Arrivée sur le PZ je me laisse guider par le GPS qui me mène tout droit à la cache. Je repars en reniflant l'odeur fort agréable de tous les mimosas environnants. Merci pour la cache.\end{cacheText}

\cacheNumber{1220}\needspace{5\baselineskip}\cacheName{\href{http://coord.info/GC6VYN2}{I009	 - Géodyssée 40 64} — \href{http://coord.info/GC6VYN2\Number{}750703336}{1220}}\cacheData{{2018/03/09 gilles64, Traditional Cache (1.5/1.5)}}\begin{cacheText}Le titi est vite localisé depuis la géodyssée : le plus dur est d'escalader ! Le log book signé je reprends la route. Finalement je ne sais plus s' il y avait un indice je repars à l'escalade. … Le doute enlevé je reprends la route. Merci pour la cache\end{cacheText}

\cacheNumber{1221}\needspace{5\baselineskip}\cacheName{\href{http://coord.info/GC64NK9}{Alphabet Landais ... P comme Pouillon} — \href{http://coord.info/GC64NK9\Number{}750717521}{1221}}\cacheData{{2018/03/11 dorisbear, Traditional Cache (3/1.5)}}\begin{cacheText}Par ce bel après-midi ensoleillé j'ai décidé d'aller faire quelques caches. Mon dévolu se porte sur cette cache de Pouillon qui me résiste. C'est chose faite, la Belle est loguée mais pas facile à trouver. Elle est parfaitement intégrée ,bravo.Merci pour la cache.\end{cacheText}

\cacheNumber{1222}\needspace{5\baselineskip}\cacheName{\href{http://coord.info/GC6624K}{Bag End} — \href{http://coord.info/GC6624K\Number{}750718811}{1222}}\cacheData{{2018/03/11 DorisBear, Traditional Cache (2/2)}}\begin{cacheText}Après avoir garé la géomobile sur le parking je m'aventure sur le chemin à la recherche de la cache. Arrivée sur le PZ  je commence à penser que peut-être elle a disparu. En effet, plusieurs arbres ont été coupé. Mais non… Elle est bien la ..magnifique. Que du bonheur!!! Merci les Dorisbear pour cette superbe cache: un PF évidemment.\end{cacheText}

\cacheNumber{1223}\needspace{5\baselineskip}\cacheName{\href{http://coord.info/GC74XWC}{\Number{}1 Harry Potter à l'école des sorciers} — \href{http://coord.info/GC74XWC\Number{}752177464}{1223}}\cacheData{{2018/03/17 Bigorra65, Unknown Cache (3/1.5)}}\begin{cacheText}Après avoir fais connaissance avec les élèves de Poudlard ,le checker finit par passer au vert grâce à l'indice;

Arrivée sur le PZ , la belle n'est pas facile à déloger.Grâce à l'indice je sais qu'elle est par-là mais où??? Mes petits doigts cherchent et finissent par trouver ...direction la Bonus.Merci Bigorra65 pour cet excellent travail.\end{cacheText}

\cacheNumber{1224}\needspace{5\baselineskip}\cacheName{\href{http://coord.info/GC7JP8M}{\Number{}2 Harry Potter et la chambre des secrets} — \href{http://coord.info/GC7JP8M\Number{}752173673}{1224}}\cacheData{{2018/03/17 Bigorra65, Unknown Cache (2.5/1.5)}}\begin{cacheText}(TTF)

Le monde d'Harry Potter regorge de créatures extraordinaires: il m'a fallu du temps pour retrouver les animaux concernés ... 

En arrivant sur le PZ il pleut toujours et il n'y a personne à l'horizon .Cela n'a pas été facile de trouver la Belle au milieu de toutes les feuilles mortes, je voyais le moment ou j'allais rentrer bredouille mais ouff j'aperçois...Encore du super travail. Merci Bigorra65\end{cacheText}

\cacheNumber{1225}\needspace{5\baselineskip}\cacheName{\href{http://coord.info/GC7JQ3D}{\Number{}3 Harry Potter et le prisonnier d'Azkaban} — \href{http://coord.info/GC7JQ3D\Number{}752166540}{1225}}\cacheData{{2018/03/17 Bigorra65, Unknown Cache (3/1.5)}}\begin{cacheText}(FTF)

Cette énigme m'a donné bien du mal pourtant elle n'est pas si compliquée… L'indice nous ouvre la voie...

Arrivée au PZ il pleut des cordes et personne n'est en vue. Tant mieux!!! je peux chercher tranquillement et je finis par dégoter la Belle. Encore du très beau travail. Merci Bigorra65 pour cet excellent moment passé.\end{cacheText}

\cacheNumber{1226}\needspace{5\baselineskip}\cacheName{\href{http://coord.info/GC7JQ6F}{\Number{}4 Harry Potter et la coupe de feu} — \href{http://coord.info/GC7JQ6F\Number{}752163636}{1226}}\cacheData{{2018/03/17 Bigorra65, Unknown Cache (2.5/1.5)}}\begin{cacheText}C'est la première énigme de la série que j'ai résolue et elle m'a bien occupée. Les cours du professeur Babbling sont très instructifs.

Le PZ est assez fréquenté et ce n'est pas facile de rester discrète. J'attend le bon moment et finis par trouver la belle. Elle est super chouette. Merci Bigorra65 pour ce super travail.\end{cacheText}

\cacheNumber{1227}\needspace{5\baselineskip}\cacheName{\href{http://coord.info/GC7JQ8N}{\Number{}5 Harry Potter et l'ordre du phénix} — \href{http://coord.info/GC7JQ8N\Number{}752160096}{1227}}\cacheData{{2018/03/17 Bigorra65, Unknown Cache (2.5/1.5)}}\begin{cacheText}(STF)

Encore une super énigme qui m'a donné du fil à retordre jusqu'à ce que je m'intéresse à l'indice!!!.

 Arrivée sur le PZ  je suis obligée de patienter 20 min car un moldu attend dans sa voiture devant la cache. Il finit par partir et je fonce découvrir la Belle .Chapeau bas Valérie :encore une belle découverte et un joli travail effectué. Merci .\end{cacheText}

\cacheNumber{1228}\needspace{5\baselineskip}\cacheName{\href{http://coord.info/GC7JQB2}{\Number{}6 Harry Potter et le prince de sang mêlé} — \href{http://coord.info/GC7JQB2\Number{}752157212}{1228}}\cacheData{{2018/03/17 Bigorra65, Unknown Cache (3/1.5)}}\begin{cacheText}L'énigme a été très rigolote à résoudre. Quelques coups de baguette magique et me voilà devant la Belle. Encore du très beau travail.J'arrive en quatrième position: les géocacheurs locaux sont déjà passés. Merci Bigorra65 pour cette super cache.\end{cacheText}

\cacheNumber{1229}\needspace{5\baselineskip}\cacheName{\href{http://coord.info/GC7JQC6}{\Number{}7 Harry Potter et les reliques de la mort} — \href{http://coord.info/GC7JQC6\Number{}752155838}{1229}}\cacheData{{2018/03/17 Bigorra65, Unknown Cache (3.5/1.5)}}\begin{cacheText}(FTF)

La résolution de l'énigme n'a pas été simple .Après avoir essayé je me suis rendue compte que sans avoir vu ou lu la série je n'y arriverai pas!!!Je suis donc partie à la recherche de fans et j'ai trouvé la perle rare qui a relié les éléments en deux temps trois mouvements sans l'indice initial (ex churros) qui prend tout son sens à la résolution.  Je remercie Lucie qui est fan d'Harry Potter et qui m'a bien aidé sur ce coup là . Toutes les coordonnées en poche ,je peux enfin faire le déplacement sur Tarbes.

Arrivée sur le PZ il n'y a aucun moldu à l'horizon...La potion est vite trouvée et surprise , je suis FTF sur une de la série !!!Un grand merci Bigorra65 pour cette énigme et cette jolie boite.\end{cacheText}

\cacheNumber{1230}\needspace{5\baselineskip}\cacheName{\href{http://coord.info/GC7JQEA}{Harry Potter retour au château de Poudlard \Number{} BONUS} — \href{http://coord.info/GC7JQEA\Number{}752181395}{1230}}\cacheData{{2018/03/17 Bigorra65, Unknown Cache (1.5/1.5)}}\begin{cacheText}Les indices en poche ,le checker au vert ,me voilà partie sur mon Nimbus2000 en direction de Poudlard. Pas déçue du château et encore moins de la partie de Quidditch : désignée comme attrapeur je finis par capturer le v...Quel travail !Quel souci du détail !Je suis heureuse d'avoir fait cette série qui est vraiment superbe tant dans les énigmes que dans la fabrication des caches .Un PF pour l'ensemble de la série et encore un grand merci.\end{cacheText}

\cacheNumber{1231}\needspace{5\baselineskip}\cacheName{\href{http://coord.info/GC69ZE2}{\Number{}L2-06 Lahonce - Le Comté du Labourd} — \href{http://coord.info/GC69ZE2\Number{}752186513}{1231}}\cacheData{{2018/03/20 gilles64, Multi-cache (2.5/1.5)}}\begin{cacheText}De retour de Bayonne, j'ai un peu de temps devant moi . Charmante promenade dans Lahonce où je  découvre de bien beaux monuments. Après avoir refait les calculs les coordonnées finales apparaissent. La vue est magnifique. Merci Gilles pour ce bon moment.\end{cacheText}

\cacheNumber{1232}\needspace{5\baselineskip}\cacheName{\href{http://coord.info/GC773F5}{La boîte à lire} — \href{http://coord.info/GC773F5\Number{}753013356}{1232}}\cacheData{{2018/03/23 yvain64, Traditional Cache (1/1.5)}}\begin{cacheText}L'attraction des planètes est trop forte...me voila partie sur la super série La Landaise de crispol40 et c'est en revenant que je repère cette cache. J’adore les livres et la belle est parfaitement intégrée mais l'indice me parle bien!!! Très jolie cache et un PF évidemment. Un grand merci\end{cacheText}

\cacheNumber{1233}\needspace{5\baselineskip}\cacheName{\href{http://coord.info/GC7CYFG}{LA LANDAISE \Number{} 02} — \href{http://coord.info/GC7CYFG\Number{}752977686}{1233}}\cacheData{{2018/03/23 crispol40, Unknown Cache (2.5/1.5)}}\begin{cacheText}L'attraction des planètes est trop forte...me voila partie sur cette super série qui est décodée depuis au moment .Je prévois de la faire en deux fois.

Le GPS me joue des tours mais les nanard40270 sont la pour me mettre sur le droit chemin .Ici je trouve une autre oeuvre de crispol40....Bravo et merci.\end{cacheText}

\cacheNumber{1234}\needspace{5\baselineskip}\cacheName{\href{http://coord.info/GC7CYG9}{LA LANDAISE \Number{} 03} — \href{http://coord.info/GC7CYG9\Number{}752984414}{1234}}\cacheData{{2018/03/23 crispol40, Unknown Cache (2.5/1.5)}}\begin{cacheText}L'attraction des planètes est trop forte...me voila partie sur cette super série qui est décodée depuis au moment .Je prévois de la faire en deux fois.

Ici la cache est trouvée rapidement mais le logbook n'est pas facile à sortir!!!Une fois signé nous partons vers la prochaine qui semble assez difficile.Merci pour la cache.\end{cacheText}

\cacheNumber{1235}\needspace{5\baselineskip}\cacheName{\href{http://coord.info/GC7CZ56}{LA LANDAISE \Number{} 19} — \href{http://coord.info/GC7CZ56\Number{}752685592}{1235}}\cacheData{{2018/03/23 crispol40, Unknown Cache (2.5/1.5)}}\begin{cacheText}L'attraction des planètes est trop forte...me voila partie sur cette super série qui est décodée depuis au moment .Je prévois de la faire en deux fois.

Et voilà encore un super camouflage fabriqué par crispol40, du vrai géocaching comme je l’aime.Merci\end{cacheText}

\cacheNumber{1236}\needspace{5\baselineskip}\cacheName{\href{http://coord.info/GC7CZ5A}{LA LANDAISE \Number{} 20} — \href{http://coord.info/GC7CZ5A\Number{}752685244}{1236}}\cacheData{{2018/03/23 crispol40, Unknown Cache (2.5/2.5)}}\begin{cacheText}L'attraction des planètes est trop forte...me voila partie sur cette super série qui est décodée depuis au moment .Je prévois de la faire en deux fois.

Arrivée dans la zone, le GPS me promène à droite... à gauche... et puis finalement en regardant les berges je finis par pécher la Belle. Merci pour la cache.\end{cacheText}

\cacheNumber{1237}\needspace{5\baselineskip}\cacheName{\href{http://coord.info/GC7CZ73}{LA LANDAISE \Number{} 21} — \href{http://coord.info/GC7CZ73\Number{}752684816}{1237}}\cacheData{{2018/03/23 crispol40, Unknown Cache (4.5/1.5)}}\begin{cacheText}L'attraction des planètes est trop forte...me voila partie sur cette super série qui est décodée depuis au moment .Je prévois de la faire en deux fois.

Arrivée sur le PZ, il n’y a pas 50 possibilités ! Le camouflage est très original. Alors que je suis sur le point de repartir une voiture arrive : quelqu’un descend et s'approche du PZ… Je pars à sa rencontre: nanard40270! Ils explorent également les planètes. Un peu de discussion et je repars. Bravo crispol40 pour cette cache Merci.\end{cacheText}

\cacheNumber{1238}\needspace{5\baselineskip}\cacheName{\href{http://coord.info/GC7CZ76}{LA LANDAISE \Number{} 22} — \href{http://coord.info/GC7CZ76\Number{}752683629}{1238}}\cacheData{{2018/03/23 crispol40, Unknown Cache (2/1.5)}}\begin{cacheText}L'attraction des planètes est trop forte...me voila partie sur cette super série qui est décodée depuis au moment .Je prévois de la faire en deux fois.

Ici, la cache est tout à fait à ma portée. L'indice est très explicite. Pas de moldu en vu mais il commence à pleuvoir. Merci pour la cache.\end{cacheText}

\cacheNumber{1239}\needspace{5\baselineskip}\cacheName{\href{http://coord.info/GC7CZ79}{LA LANDAISE \Number{} 23} — \href{http://coord.info/GC7CZ79\Number{}752682701}{1239}}\cacheData{{2018/03/23 crispol40, Unknown Cache (3.5/1.5)}}\begin{cacheText}L'attraction des planètes est trop forte...me voila partie sur cette super série qui est décodée depuis au moment .Je prévois de la faire en deux fois.

Ouf l’honneur est sauf ! Il m’a la fallu du temps pour enfin la déloger pourtant… Le GPS me joue des tours :je ne suis pas sortie de l’auberge. Je récupère le TB pour de prochaines aventures. Merci Paul pour la cache.\end{cacheText}

\cacheNumber{1240}\needspace{5\baselineskip}\cacheName{\href{http://coord.info/GC7CZ7E}{LA LANDAISE \Number{} 24} — \href{http://coord.info/GC7CZ7E\Number{}752919566}{1240}}\cacheData{{2018/03/23 crispol40, Unknown Cache (5/3.5)}}\begin{cacheText}L'attraction des planètes est trop forte...me voila partie sur cette super série qui est décodée depuis au moment .Je prévois de la faire en deux fois.

Je reviens sur mes pas afin de voir si les nanard40270 ont délogé les caches. Celle

ci n'a pas échappé à l'œil perspicace de Nadine qui me sert de guide. L' indice, me dit elle ,il faut analyser l'indice! Et ...bingo la Belle se dévoile lentement....Merci pour cette super cache.\end{cacheText}

\cacheNumber{1241}\needspace{5\baselineskip}\cacheName{\href{http://coord.info/GC7CZ7J}{LA LANDAISE \Number{} 25} — \href{http://coord.info/GC7CZ7J\Number{}752925916}{1241}}\cacheData{{2018/03/23 crispol40, Unknown Cache (3/1.5)}}\begin{cacheText}L'attraction des planètes est trop forte...me voila partie sur cette super série qui est décodée depuis au moment .Je prévois de la faire en deux fois.

Alors que nous étions parties sur la cache \Number{}24, Bernard cherche la\Quoted{Belle}. A notre retour , toujours RIEN!!! Ici, le GPS nous promène : nous cherchons à droite à gauche et au bout d’un certain temps  nous décidons de faire appel à Dune33 qui se fait un plaisir de nous guider .Enfin nous mettons la main dessus.....à 10 M des coordonnées. Je ne l'aurais jamais trouvée sans aide!!! Ce n'est qu'en la voyant que je comprend l'indice!!!!Merci Crispol40 pour cette cache squelettique....\end{cacheText}

\cacheNumber{1242}\needspace{5\baselineskip}\cacheName{\href{http://coord.info/GC7CZ7T}{LA LANDAISE \Number{} 26} — \href{http://coord.info/GC7CZ7T\Number{}752680860}{1242}}\cacheData{{2018/03/23 crispol40, Unknown Cache (3/1.5)}}\begin{cacheText}L'attraction des planètes est trop forte...me voila partie sur cette super série qui est décodée depuis au moment .Je prévois de la faire en deux fois.

Arrivée sur le PZ pas de difficulté : la boîte est apparente. Merci crispol40.\end{cacheText}

\cacheNumber{1243}\needspace{5\baselineskip}\cacheName{\href{http://coord.info/GC7CZ81}{LA LANDAISE \Number{} 27} — \href{http://coord.info/GC7CZ81\Number{}752947069}{1243}}\cacheData{{2018/03/23 crispol40, Unknown Cache (3.5/2)}}\begin{cacheText}L'attraction des planètes est trop forte...me voila partie sur cette super série qui est décodée depuis au moment .Je prévois de la faire en deux fois.

En compagnie de Nanard40270 je furète dans la mousse et nous finissons par découvrir le...C'est en trouvant la cache que l'indice prend tout son sens. Merci pour la cache.\end{cacheText}

\cacheNumber{1244}\needspace{5\baselineskip}\cacheName{\href{http://coord.info/GC7CZ89}{LA LANDAISE \Number{} 28} — \href{http://coord.info/GC7CZ89\Number{}752951665}{1244}}\cacheData{{2018/03/23 crispol40, Unknown Cache (3/1.5)}}\begin{cacheText}L'attraction des planètes est trop forte...me voila partie sur cette super série qui est décodée depuis au moment .Je prévois de la faire en deux fois.

C' est Nanard40270 qui la trouve en premier mais il a la gentillesse de me laisser la découvrir. Encore une jolie cache travaillée ...merci crispol40.\end{cacheText}

\cacheNumber{1245}\needspace{5\baselineskip}\cacheName{\href{http://coord.info/GC7CZ8F}{LA LANDAISE \Number{} 29} — \href{http://coord.info/GC7CZ8F\Number{}752957242}{1245}}\cacheData{{2018/03/23 crispol40, Unknown Cache (3.5/1.5)}}\begin{cacheText}L'attraction des planètes est trop forte...me voila partie sur cette super série qui est décodée depuis au moment .Je prévois de la faire en deux fois.

Arrivés au PZ, la cache nous saute aux yeux!!!Nous signons mais nous pensons qu’il ne s'agit pas de la cache d'origine (le logbook est un bout de papier signé par nos prédécesseurs). Cela ne ressemble pas aux caches de crispol40!!!!Merci\end{cacheText}

\cacheNumber{1246}\needspace{5\baselineskip}\cacheName{\href{http://coord.info/GC7CZ8T}{LA LANDAISE \Number{} 30} — \href{http://coord.info/GC7CZ8T\Number{}752959325}{1246}}\cacheData{{2018/03/23 crispol40, Unknown Cache (3.5/1.5)}}\begin{cacheText}L'attraction des planètes est trop forte...me voila partie sur cette super série qui est décodée depuis au moment .Je prévois de la faire en deux fois.

Le GPS de mon téléphone n’est pas très précis mais heureusement que celui des nanard40270 oui!!! La cache est débarrassées des feuilles en deux temps trois mouvements. Il faut fouiller mais elle est bien là aux pieds. Merci pour la cache.\end{cacheText}

\cacheNumber{1247}\needspace{5\baselineskip}\cacheName{\href{http://coord.info/GC7CZAF}{LA LANDAISE \Number{} 31} — \href{http://coord.info/GC7CZAF\Number{}752965593}{1247}}\cacheData{{2018/03/23 crispol40, Unknown Cache (3.5/1.5)}}\begin{cacheText}L'attraction des planètes est trop forte...me voila partie sur cette super série qui est décodée depuis au moment .Je prévois de la faire en deux fois.

Ici le GPS est précis et il nous guide tout droit à la cache qui est très bien aimantée. Logbook signé nous repartons vers la prochaine. Merci pour la cache.\end{cacheText}

\cacheNumber{1248}\needspace{5\baselineskip}\cacheName{\href{http://coord.info/GC7CZDA}{LA LANDAISE \Number{} 32} — \href{http://coord.info/GC7CZDA\Number{}752967961}{1248}}\cacheData{{2018/03/23 crispol40, Unknown Cache (4/1.5)}}\begin{cacheText}L'attraction des planètes est trop forte...me voila partie sur cette super série qui est décodée depuis au moment .Je prévois de la faire en deux fois.

Ici nous distinguons un petit quelque chose depuis la route.....Ouiiii c'est bien elle !!!Merci pour la cache.\end{cacheText}

\cacheNumber{1249}\needspace{5\baselineskip}\cacheName{\href{http://coord.info/GC7CZDT}{LA LANDAISE \Number{} 33} — \href{http://coord.info/GC7CZDT\Number{}752969901}{1249}}\cacheData{{2018/03/23 crispol40, Unknown Cache (5/2)}}\begin{cacheText}L'attraction des planètes est trop forte...me voila partie sur cette super série qui est décodée depuis au moment .Je prévois de la faire en deux fois.

Nous découvrons la Belle accrochée en arrivant sur le PZ. Pas de problème. Merci pour la cache.\end{cacheText}

\cacheNumber{1250}\needspace{5\baselineskip}\cacheName{\href{http://coord.info/GC7CZE0}{LA LANDAISE \Number{} 34} — \href{http://coord.info/GC7CZE0\Number{}752973126}{1250}}\cacheData{{2018/03/23 crispol40, Unknown Cache (5/1.5)}}\begin{cacheText}L'attraction des planètes est trop forte...me voila partie sur cette super série qui est décodée depuis au moment .Je prévois de la faire en deux fois.

Ici aussi la cache est visible mais pas facile à ouvrir.Merci pour ce beau travail.\end{cacheText}

\cacheNumber{1251}\needspace{5\baselineskip}\cacheName{\href{http://coord.info/GC2E6JW}{À la Porte Mousserolles} — \href{http://coord.info/GC2E6JW\Number{}753818305}{1251}}\cacheData{{2018/03/26 Gamboy, Unknown Cache (1.5/1.5)}}\begin{cacheText}L’énigme est résolue depuis fort longtemps… Il ne me restait qu’à prendre un peu de temps pour venir la chercher. C’est chose faite aujourd’hui : une jolie boîte pleine de bidules. Merci les Gamboy pour la découverte de cette Porte Mousseroles.\end{cacheText}

\cacheNumber{1252}\needspace{5\baselineskip}\cacheName{\href{http://coord.info/GC3FYT1}{GR8\Number{}11} — \href{http://coord.info/GC3FYT1\Number{}753819881}{1252}}\cacheData{{2018/03/26 Peyo64, Traditional Cache (1.5/1.5)}}\begin{cacheText}En revenant de Bayonne, je passe devant ce petit point bleu qui m’agace. J’ai un peu de temps devant moi, je m’arrête et enfin je découvre la Belle. Elle n’était pas très compliquée!!! Comment l’avons nous manqué à notre premier passage ? La fatigue peut-être… Merci Peyo.\end{cacheText}

\cacheNumber{1253}\needspace{5\baselineskip}\cacheName{\href{http://coord.info/GC6RRGV}{J001 - Géodyssée 40 64} — \href{http://coord.info/GC6RRGV\Number{}753819197}{1253}}\cacheData{{2018/03/26 gilles64, Traditional Cache (1.5/1.5)}}\begin{cacheText}Pas facile de rester discrète avec tous ces moldus qui vont et qui viennent. Cela a été assez compliqué car le GPS m’a bien promené mais au bout de 20 minutes... Ouff l' honneur est sauf je mets la main sur la Belle. Merci Gilles\end{cacheText}

\cacheNumber{1254}\needspace{5\baselineskip}\cacheName{\href{http://coord.info/GC6RRHN}{J003	 - Géodyssée 40 64} — \href{http://coord.info/GC6RRHN\Number{}753817652}{1254}}\cacheData{{2018/03/26 gilles64, Traditional Cache (1.5/1.5)}}\begin{cacheText}Quelques caches en passant sur Bayonne. Celle ci est trouvée sans difficulté : les coordonnées sont \Quoted{pile poil}.Elle nous attend bien à sa place mais je dois m'étirer!!!Pas très discrète!!!!. Merci pour la cache\end{cacheText}

\cacheNumber{1255}\needspace{5\baselineskip}\cacheName{\href{http://coord.info/GC6KQN7}{En route pour le Pic du Jer \Number{}1} — \href{http://coord.info/GC6KQN7\Number{}754267492}{1255}}\cacheData{{2018/03/27 l.a.f.l.e.u.r, Traditional Cache (1.5/2)}}\begin{cacheText}Le rendez-vous est pris depuis la semaine dernière avec Dune33, et DorisBear pour faire une virée sur Lourdes. Apres le lac de Lourdes de GOLIATH65,nous déjeunons en vitesse et partons prendre le funiculaire pour faire les caches en descendant. Le vent pluie redouble en sortant de la cabine et nous ne sommes pas très fiers. Mais maintenant que nous sommes la... Apres la deuxième cache le temps se calme enfin et le soleil fait même une petite apparition. Peu de moldus, quelques VTT et même un superbe husky croiseront notre route.Toutes les caches sont trouvées et nous logons sous le pseudo DuFleDo. La descente est très agréable et le paysage superbe. Merci Lafleur pour cet excellent moment et un PF(sur la 12 :la vue est à couper le souffle) pour l'ensemble des caches.                         Ici la cache a disparu : nous la remplaçons.\end{cacheText}

\cacheNumber{1256}\needspace{5\baselineskip}\cacheName{\href{http://coord.info/GC6KQNP}{En route pour le Pic du Jer \Number{}2} — \href{http://coord.info/GC6KQNP\Number{}754267184}{1256}}\cacheData{{2018/03/27 l.a.f.l.e.u.r, Traditional Cache (1.5/2)}}\begin{cacheText}Le rendez-vous est pris depuis la semaine dernière avec Dune33, et DorisBear pour faire une virée sur Lourdes. Apres le lac de Lourdes de GOLIATH65,nous déjeunons en vitesse et partons prendre le funiculaire pour faire les caches en descendant. Le vent pluie redouble en sortant de la cabine et nous ne sommes pas très fiers. Mais maintenant que nous sommes la... Apres la deuxième cache le temps se calme enfin et le soleil fait même une petite apparition. Peu de moldus, quelques VTT et même un superbe husky croiseront notre route.Toutes les caches sont trouvées et nous logons sous le pseudo DuFleDo. La descente est très agréable et le paysage superbe. Merci Lafleur pour cet excellent moment et un PF(sur la 12 :la vue est à couper le souffle) pour l'ensemble des caches.\end{cacheText}

\cacheNumber{1257}\needspace{5\baselineskip}\cacheName{\href{http://coord.info/GC6KQNY}{En route pour le Pic du Jer \Number{}3} — \href{http://coord.info/GC6KQNY\Number{}754266964}{1257}}\cacheData{{2018/03/27 l.a.f.l.e.u.r, Traditional Cache (1.5/2)}}\begin{cacheText}Le rendez-vous est pris depuis la semaine dernière avec Dune33, et DorisBear pour faire une virée sur Lourdes. Apres le lac de Lourdes de GOLIATH65,nous déjeunons en vitesse et partons prendre le funiculaire pour faire les caches en descendant. Le vent pluie redouble en sortant de la cabine et nous ne sommes pas très fiers. Mais maintenant que nous sommes la... Apres la deuxième cache le temps se calme enfin et le soleil fait même une petite apparition. Peu de moldus, quelques VTT et même un superbe husky croiseront notre route.Toutes les caches sont trouvées et nous logons sous le pseudo DuFleDo. La descente est très agréable et le paysage superbe. Merci Lafleur pour cet excellent moment et un PF(sur la 12 :la vue est à couper le souffle) pour l'ensemble des caches.\end{cacheText}

\cacheNumber{1258}\needspace{5\baselineskip}\cacheName{\href{http://coord.info/GC6KQP0}{En route pour le Pic du Jer \Number{}4} — \href{http://coord.info/GC6KQP0\Number{}754266865}{1258}}\cacheData{{2018/03/27 l.a.f.l.e.u.r, Traditional Cache (1.5/2)}}\begin{cacheText}Le rendez-vous est pris depuis la semaine dernière avec Dune33, et DorisBear pour faire une virée sur Lourdes. Apres le lac de Lourdes de GOLIATH65,nous déjeunons en vitesse et partons prendre le funiculaire pour faire les caches en descendant. Le vent pluie redouble en sortant de la cabine et nous ne sommes pas très fiers. Mais maintenant que nous sommes la... Apres la deuxième cache le temps se calme enfin et le soleil fait même une petite apparition. Peu de moldus, quelques VTT et même un superbe husky croiseront notre route.Toutes les caches sont trouvées et nous logons sous le pseudo DuFleDo. La descente est très agréable et le paysage superbe. Merci Lafleur pour cet excellent moment et un PF(sur la 12 :la vue est à couper le souffle) pour l'ensemble des caches.\end{cacheText}

\cacheNumber{1259}\needspace{5\baselineskip}\cacheName{\href{http://coord.info/GC6KQP6}{En route pour le Pic du Jer \Number{}5} — \href{http://coord.info/GC6KQP6\Number{}754266714}{1259}}\cacheData{{2018/03/27 l.a.f.l.e.u.r, Traditional Cache (1.5/2)}}\begin{cacheText}Le rendez-vous est pris depuis la semaine dernière avec Dune33, et DorisBear pour faire une virée sur Lourdes. Apres le lac de Lourdes de GOLIATH65,nous déjeunons en vitesse et partons prendre le funiculaire pour faire les caches en descendant. Le vent pluie redouble en sortant de la cabine et nous ne sommes pas très fiers. Mais maintenant que nous sommes la... Apres la deuxième cache le temps se calme enfin et le soleil fait même une petite apparition. Peu de moldus, quelques VTT et même un superbe husky croiseront notre route.Toutes les caches sont trouvées et nous logons sous le pseudo DuFleDo. La descente est très agréable et le paysage superbe. Merci Lafleur pour cet excellent moment et un PF(sur la 12 :la vue est à couper le souffle) pour l'ensemble des caches.\end{cacheText}

\cacheNumber{1260}\needspace{5\baselineskip}\cacheName{\href{http://coord.info/GC6KQPF}{En route pour le Pic du Jer \Number{}6} — \href{http://coord.info/GC6KQPF\Number{}754266605}{1260}}\cacheData{{2018/03/27 l.a.f.l.e.u.r, Traditional Cache (1.5/2)}}\begin{cacheText}Le rendez-vous est pris depuis la semaine dernière avec Dune33, et DorisBear pour faire une virée sur Lourdes. Apres le lac de Lourdes de GOLIATH65,nous déjeunons en vitesse et partons prendre le funiculaire pour faire les caches en descendant. Le vent pluie redouble en sortant de la cabine et nous ne sommes pas très fiers. Mais maintenant que nous sommes la... Apres la deuxième cache le temps se calme enfin et le soleil fait même une petite apparition. Peu de moldus, quelques VTT et même un superbe husky croiseront notre route.Toutes les caches sont trouvées et nous logons sous le pseudo DuFleDo. La descente est très agréable et le paysage superbe. Merci Lafleur pour cet excellent moment et un PF(sur la 12 :la vue est à couper le souffle) pour l'ensemble des caches.\end{cacheText}

\cacheNumber{1261}\needspace{5\baselineskip}\cacheName{\href{http://coord.info/GC6KQPH}{En route pour le Pic du Jer \Number{}7} — \href{http://coord.info/GC6KQPH\Number{}754266502}{1261}}\cacheData{{2018/03/27 l.a.f.l.e.u.r, Traditional Cache (1.5/2)}}\begin{cacheText}Le rendez-vous est pris depuis la semaine dernière avec Dune33, et DorisBear pour faire une virée sur Lourdes. Apres le lac de Lourdes de GOLIATH65,nous déjeunons en vitesse et partons prendre le funiculaire pour faire les caches en descendant. Le vent pluie redouble en sortant de la cabine et nous ne sommes pas très fiers. Mais maintenant que nous sommes la... Apres la deuxième cache le temps se calme enfin et le soleil fait même une petite apparition. Peu de moldus, quelques VTT et même un superbe husky croiseront notre route.Toutes les caches sont trouvées et nous logons sous le pseudo DuFleDo. La descente est très agréable et le paysage superbe. Merci Lafleur pour cet excellent moment et un PF(sur la 12 :la vue est à couper le souffle) pour l'ensemble des caches.\end{cacheText}

\cacheNumber{1262}\needspace{5\baselineskip}\cacheName{\href{http://coord.info/GC6KQPN}{En route pour le Pic du Jer \Number{}8} — \href{http://coord.info/GC6KQPN\Number{}754262591}{1262}}\cacheData{{2018/03/27 l.a.f.l.e.u.r, Traditional Cache (1.5/2)}}\begin{cacheText}Le rendez-vous est pris depuis la semaine dernière avec Dune33, et DorisBear pour faire une virée sur Lourdes. Apres le lac de Lourdes de GOLIATH65,nous déjeunons en vitesse et partons prendre le funiculaire pour faire les caches en descendant. Le vent pluie redouble en sortant de la cabine et nous ne sommes pas très fiers. Mais maintenant que nous sommes la... Apres la deuxième cache le temps se calme enfin et le soleil fait même une petite apparition. Peu de moldus, quelques VTT et même un superbe husky croiseront notre route.Toutes les caches sont trouvées et nous logons sous le pseudo DuFleDo. La descente est très agréable et le paysage superbe. Merci Lafleur pour cet excellent moment et un PF(sur la 12 :la vue est à couper le souffle) pour l'ensemble des caches.\end{cacheText}

\cacheNumber{1263}\needspace{5\baselineskip}\cacheName{\href{http://coord.info/GC6KQPQ}{En route pour le Pic du Jer \Number{}9} — \href{http://coord.info/GC6KQPQ\Number{}754262382}{1263}}\cacheData{{2018/03/27 l.a.f.l.e.u.r, Traditional Cache (1.5/2)}}\begin{cacheText}Le rendez-vous est pris depuis la semaine dernière avec Dune33, et DorisBear pour faire une virée sur Lourdes. Apres le lac de Lourdes de GOLIATH65,nous déjeunons en vitesse et partons prendre le funiculaire pour faire les caches en descendant. Le vent pluie redouble en sortant de la cabine et nous ne sommes pas très fiers. Mais maintenant que nous sommes la... Apres la deuxième cache le temps se calme enfin et le soleil fait même une petite apparition. Peu de moldus, quelques VTT et même un superbe husky croiseront notre route.Toutes les caches sont trouvées et nous logons sous le pseudo DuFleDo. La descente est très agréable et le paysage superbe. Merci Lafleur pour cet excellent moment et un PF(sur la 12 :la vue est à couper le souffle) pour l'ensemble des caches.\end{cacheText}

\cacheNumber{1264}\needspace{5\baselineskip}\cacheName{\href{http://coord.info/GC6KQPW}{En route pour le Pic du Jer \Number{}10} — \href{http://coord.info/GC6KQPW\Number{}754261700}{1264}}\cacheData{{2018/03/27 l.a.f.l.e.u.r, Traditional Cache (1.5/2)}}\begin{cacheText}Le rendez-vous est pris depuis la semaine dernière avec Dune33, et DorisBear pour faire une virée sur Lourdes. Apres le lac de Lourdes de GOLIATH65,nous déjeunons en vitesse et partons prendre le funiculaire pour faire les caches en descendant. Le vent pluie redouble en sortant de la cabine et nous ne sommes pas très fiers. Mais maintenant que nous sommes la... Apres la deuxième cache le temps se calme enfin et le soleil fait même une petite apparition. Peu de moldus, quelques VTT et même un superbe husky croiseront notre route.Toutes les caches sont trouvées et nous logons sous le pseudo DuFleDo. La descente est très agréable et le paysage superbe. Merci Lafleur pour cet excellent moment et un PF(sur la 12 :la vue est à couper le souffle) pour l'ensemble des caches.                                                                                    Ici le logbook est trempé ,nous le remplaçons.\end{cacheText}

\cacheNumber{1265}\needspace{5\baselineskip}\cacheName{\href{http://coord.info/GC6KQQ2}{En route pour le Pic du Jer \Number{}11} — \href{http://coord.info/GC6KQQ2\Number{}754260488}{1265}}\cacheData{{2018/03/27 l.a.f.l.e.u.r, Traditional Cache (1.5/2)}}\begin{cacheText}Le rendez-vous est pris depuis la semaine dernière avec Dune33, et DorisBear pour faire une virée sur Lourdes. Apres le lac de Lourdes de GOLIATH65,nous déjeunons en vitesse et partons prendre le funiculaire pour faire les caches en descendant. Le vent pluie redouble en sortant de la cabine et nous ne sommes pas très fiers. Mais maintenant que nous sommes la... Apres la deuxième cache le temps se calme enfin et le soleil fait même une petite apparition. Peu de moldus, quelques VTT et même un superbe husky croiseront notre route.Toutes les caches sont trouvées et nous logons sous le pseudo DuFleDo. La descente est très agréable et le paysage superbe. Merci Lafleur pour cet excellent moment et un PF(sur la 12 :la vue est à couper le souffle) pour l'ensemble des caches.\end{cacheText}

\cacheNumber{1266}\needspace{5\baselineskip}\cacheName{\href{http://coord.info/GC6KQQD}{En route pour le Pic du Jer \Number{}12} — \href{http://coord.info/GC6KQQD\Number{}754266314}{1266}}\cacheData{{2018/03/27 l.a.f.l.e.u.r, Multi-cache (1.5/2)}}\begin{cacheText}Le rendez-vous est pris depuis la semaine dernière avec Dune33, et DorisBear pour faire une virée sur Lourdes. Apres le lac de Lourdes de GOLIATH65,nous déjeunons en vitesse et partons prendre le funiculaire pour faire les caches en descendant. Le vent pluie redouble en sortant de la cabine et nous ne sommes pas très fiers. Mais maintenant que nous sommes la... Apres la deuxième cache le temps se calme enfin et le soleil fait même une petite apparition. Peu de moldus, quelques VTT et même un superbe husky croiseront notre route.Toutes les caches sont trouvées et nous logons sous le pseudo DuFleDo. La descente est très agréable et le paysage superbe. Merci Lafleur pour cet excellent moment et un PF(sur la 12 :la vue est à couper le souffle) pour l'ensemble des caches.\end{cacheText}

\cacheNumber{1267}\needspace{5\baselineskip}\cacheName{\href{http://coord.info/GC7CPJE}{Lac de Lourdes \Number{}11} — \href{http://coord.info/GC7CPJE\Number{}754248690}{1267}}\cacheData{{2018/03/27 GOLIATH65, Traditional Cache (1.5/1.5)}}\begin{cacheText}Le rendez-vous est pris depuis la semaine dernière avec Dune33, et DorisBear pour faire cette série. La pluie est annoncée mais cela ne nous décourage pas.

Bien couverts nous nous lançons sur le ....mauvais chemin!!!!Dans l’excitation, nous n’avons pas bien regardé le plan!!!Les caches sont plus ou moins vite trouvées grâce aux 8 yeux et nous logons sous le pseudo DuFleDo. Le circuit est très agréable et les blablas vont bon train...le temps passe très vite .Les indices relevés et le checker au vert nous découvrons la Bonus sans difficulté  avant la pose repas.Merci GOLIATH 65 (et Bigorra65) pour ce bon moment et un PF pour l'ensemble de la série.\end{cacheText}

\cacheNumber{1268}\needspace{5\baselineskip}\cacheName{\href{http://coord.info/GC7CPQF}{Lac de Lourdes  \Number{} Bonus} — \href{http://coord.info/GC7CPQF\Number{}754249124}{1268}}\cacheData{{2018/03/27 GOLIATH65, Unknown Cache (2.5/1)}}\begin{cacheText}Le rendez-vous est pris depuis la semaine dernière avec Dune33, et DorisBear pour faire cette série. La pluie est annoncée mais cela ne nous décourage pas.

Bien couverts nous nous lançons sur le ....mauvais chemin!!!!Dans l’excitation, nous n’avons pas bien regardé le plan!!!Les caches sont plus ou moins vite trouvées grâce aux 8 yeux et nous logons sous le pseudo DuFleDo. Le circuit est très agréable et les blablas vont bon train...le temps passe très vite .Les indices relevés et le checker au vert nous découvrons la Bonus sans difficulté  avant la pose repas.Merci GOLIATH 65 (et Bigorra65) pour ce bon moment et un PF pour l'ensemble de la série.\end{cacheText}

\cacheNumber{1269}\needspace{5\baselineskip}\cacheName{\href{http://coord.info/GC7CQD4}{Lac de Lourdes \Number{} 2} — \href{http://coord.info/GC7CQD4\Number{}754247285}{1269}}\cacheData{{2018/03/27 GOLIATH65, Traditional Cache (1.5/1.5)}}\begin{cacheText}Le rendez-vous est pris depuis la semaine dernière avec Dune33, et DorisBear pour faire cette série. La pluie est annoncée mais cela ne nous décourage pas.

Bien couverts nous nous lançons sur le ....mauvais chemin!!!!Dans l’excitation, nous n’avons pas bien regardé le plan!!!Les caches sont plus ou moins vite trouvées grâce aux 8 yeux et nous logons sous le pseudo DuFleDo. Le circuit est très agréable et les blablas vont bon train...le temps passe très vite .Les indices relevés et le checker au vert nous découvrons la Bonus sans difficulté  avant la pose repas.Merci GOLIATH 65 (et Bigorra65) pour ce bon moment et un PF pour l'ensemble de la série.\end{cacheText}

\cacheNumber{1270}\needspace{5\baselineskip}\cacheName{\href{http://coord.info/GC7CQJY}{Lac de Lourdes \Number{} 3} — \href{http://coord.info/GC7CQJY\Number{}754247470}{1270}}\cacheData{{2018/03/27 GOLIATH65, Traditional Cache (1.5/1.5)}}\begin{cacheText}Le rendez-vous est pris depuis la semaine dernière avec Dune33, et DorisBear pour faire cette série. La pluie est annoncée mais cela ne nous décourage pas.

Bien couverts nous nous lançons sur le ....mauvais chemin!!!!Dans l’excitation, nous n’avons pas bien regardé le plan!!!Les caches sont plus ou moins vite trouvées grâce aux 8 yeux et nous logons sous le pseudo DuFleDo. Le circuit est très agréable et les blablas vont bon train...le temps passe très vite .Les indices relevés et le checker au vert nous découvrons la Bonus sans difficulté  avant la pose repas.Merci GOLIATH 65 (et Bigorra65) pour ce bon moment et un PF pour l'ensemble de la série.\end{cacheText}

\cacheNumber{1271}\needspace{5\baselineskip}\cacheName{\href{http://coord.info/GC7CQKJ}{Lac de Lourdes \Number{} 4} — \href{http://coord.info/GC7CQKJ\Number{}754247653}{1271}}\cacheData{{2018/03/27 GOLIATH65, Traditional Cache (1.5/1.5)}}\begin{cacheText}Le rendez-vous est pris depuis la semaine dernière avec Dune33, et DorisBear pour faire cette série. La pluie est annoncée mais cela ne nous décourage pas.

Bien couverts nous nous lançons sur le ....mauvais chemin!!!!Dans l’excitation, nous n’avons pas bien regardé le plan!!!Les caches sont plus ou moins vite trouvées grâce aux 8 yeux et nous logons sous le pseudo DuFleDo. Le circuit est très agréable et les blablas vont bon train...le temps passe très vite .Les indices relevés et le checker au vert nous découvrons la Bonus sans difficulté  avant la pose repas.Merci GOLIATH 65 (et Bigorra65) pour ce bon moment et un PF pour l'ensemble de la série.\end{cacheText}

\cacheNumber{1272}\needspace{5\baselineskip}\cacheName{\href{http://coord.info/GC7CQPZ}{Lac de Lourdes \Number{} 5} — \href{http://coord.info/GC7CQPZ\Number{}754247771}{1272}}\cacheData{{2018/03/27 GOLIATH65, Traditional Cache (1.5/1.5)}}\begin{cacheText}Le rendez-vous est pris depuis la semaine dernière avec Dune33, et DorisBear pour faire cette série. La pluie est annoncée mais cela ne nous décourage pas.

Bien couverts nous nous lançons sur le ....mauvais chemin!!!!Dans l’excitation, nous n’avons pas bien regardé le plan!!!Les caches sont plus ou moins vite trouvées grâce aux 8 yeux et nous logons sous le pseudo DuFleDo. Le circuit est très agréable et les blablas vont bon train...le temps passe très vite .Les indices relevés et le checker au vert nous découvrons la Bonus sans difficulté  avant la pose repas.Merci GOLIATH 65 (et Bigorra65) pour ce bon moment et un PF pour l'ensemble de la série.\end{cacheText}

\cacheNumber{1273}\needspace{5\baselineskip}\cacheName{\href{http://coord.info/GC7CQXF}{Lac de Lourdes \Number{} 6} — \href{http://coord.info/GC7CQXF\Number{}754247904}{1273}}\cacheData{{2018/03/27 GOLIATH65, Traditional Cache (1.5/1.5)}}\begin{cacheText}Le rendez-vous est pris depuis la semaine dernière avec Dune33, et DorisBear pour faire cette série. La pluie est annoncée mais cela ne nous décourage pas.

Bien couverts nous nous lançons sur le ....mauvais chemin!!!!Dans l’excitation, nous n’avons pas bien regardé le plan!!!Les caches sont plus ou moins vite trouvées grâce aux 8 yeux et nous logons sous le pseudo DuFleDo. Le circuit est très agréable et les blablas vont bon train...le temps passe très vite .Les indices relevés et le checker au vert nous découvrons la Bonus sans difficulté  avant la pose repas.Merci GOLIATH 65 (et Bigorra65) pour ce bon moment et un PF pour l'ensemble de la série.\end{cacheText}

\cacheNumber{1274}\needspace{5\baselineskip}\cacheName{\href{http://coord.info/GC7CQXR}{Lac de Lourdes \Number{} 7} — \href{http://coord.info/GC7CQXR\Number{}754248061}{1274}}\cacheData{{2018/03/27 GOLIATH65, Traditional Cache (1.5/2)}}\begin{cacheText}Le rendez-vous est pris depuis la semaine dernière avec Dune33, et DorisBear pour faire cette série. La pluie est annoncée mais cela ne nous décourage pas.

Bien couverts nous nous lançons sur le ....mauvais chemin!!!!Dans l’excitation, nous n’avons pas bien regardé le plan!!!Les caches sont plus ou moins vite trouvées grâce aux 8 yeux et nous logons sous le pseudo DuFleDo. Le circuit est très agréable et les blablas vont bon train...le temps passe très vite .Les indices relevés et le checker au vert nous découvrons la Bonus sans difficulté  avant la pose repas.Merci GOLIATH 65 (et Bigorra65) pour ce bon moment et un PF pour l'ensemble de la série.\end{cacheText}

\cacheNumber{1275}\needspace{5\baselineskip}\cacheName{\href{http://coord.info/GC7CR33}{Lac de Lourdes \Number{} 1} — \href{http://coord.info/GC7CR33\Number{}754247113}{1275}}\cacheData{{2018/03/27 GOLIATH65, Traditional Cache (1.5/1.5)}}\begin{cacheText}Le rendez-vous est pris depuis la semaine dernière avec Dune33, et DorisBear pour faire cette série. La pluie est annoncée mais cela ne nous décourage pas.

Bien couverts nous nous lançons sur le ....mauvais chemin!!!!Dans l’excitation, nous n’avons pas bien regardé le plan!!!Les caches sont plus ou moins vite trouvées grâce aux 8 yeux et nous logons sous le pseudo DuFleDo. Le circuit est très agréable et les blablas vont bon train...le temps passe très vite .Les indices relevés et le checker au vert nous découvrons la Bonus sans difficulté  avant la pose repas.Merci GOLIATH 65 (et Bigorra65) pour ce bon moment et un PF pour l'ensemble de la série.\end{cacheText}

\cacheNumber{1276}\needspace{5\baselineskip}\cacheName{\href{http://coord.info/GC7CR3X}{Lac de Lourdes \Number{} 8} — \href{http://coord.info/GC7CR3X\Number{}754248250}{1276}}\cacheData{{2018/03/27 GOLIATH65, Traditional Cache (1.5/1.5)}}\begin{cacheText}Le rendez-vous est pris depuis la semaine dernière avec Dune33, et DorisBear pour faire cette série. La pluie est annoncée mais cela ne nous décourage pas.

Bien couverts nous nous lançons sur le ....mauvais chemin!!!!Dans l’excitation, nous n’avons pas bien regardé le plan!!!Les caches sont plus ou moins vite trouvées grâce aux 8 yeux et nous logons sous le pseudo DuFleDo. Le circuit est très agréable et les blablas vont bon train...le temps passe très vite .Les indices relevés et le checker au vert nous découvrons la Bonus sans difficulté  avant la pose repas.Merci GOLIATH 65 (et Bigorra65) pour ce bon moment et un PF pour l'ensemble de la série.\end{cacheText}

\cacheNumber{1277}\needspace{5\baselineskip}\cacheName{\href{http://coord.info/GC7CR4E}{Lac de Lourdes \Number{} 9} — \href{http://coord.info/GC7CR4E\Number{}754248410}{1277}}\cacheData{{2018/03/27 GOLIATH65, Traditional Cache (1.5/2)}}\begin{cacheText}Le rendez-vous est pris depuis la semaine dernière avec Dune33, et DorisBear pour faire cette série. La pluie est annoncée mais cela ne nous décourage pas.

Bien couverts nous nous lançons sur le ....mauvais chemin!!!!Dans l’excitation, nous n’avons pas bien regardé le plan!!!Les caches sont plus ou moins vite trouvées grâce aux 8 yeux et nous logons sous le pseudo DuFleDo. Le circuit est très agréable et les blablas vont bon train...le temps passe très vite .Les indices relevés et le checker au vert nous découvrons la Bonus sans difficulté  avant la pose repas.Merci GOLIATH 65 (et Bigorra65) pour ce bon moment et un PF pour l'ensemble de la série.\end{cacheText}

\cacheNumber{1278}\needspace{5\baselineskip}\cacheName{\href{http://coord.info/GC7CR4P}{Lac de Lourdes \Number{} 10} — \href{http://coord.info/GC7CR4P\Number{}754248544}{1278}}\cacheData{{2018/03/27 GOLIATH65, Traditional Cache (1.5/1.5)}}\begin{cacheText}Le rendez-vous est pris depuis la semaine dernière avec Dune33, et DorisBear pour faire cette série. La pluie est annoncée mais cela ne nous décourage pas.

Bien couverts nous nous lançons sur le ....mauvais chemin!!!!Dans l’excitation, nous n’avons pas bien regardé le plan!!!Les caches sont plus ou moins vite trouvées grâce aux 8 yeux et nous logons sous le pseudo DuFleDo. Le circuit est très agréable et les blablas vont bon train...le temps passe très vite .Les indices relevés et le checker au vert nous découvrons la Bonus sans difficulté  avant la pose repas.Merci GOLIATH 65 (et Bigorra65) pour ce bon moment et un PF pour l'ensemble de la série.\end{cacheText}

\cacheNumber{1279}\needspace{5\baselineskip}\cacheName{\href{http://coord.info/GC7CR5C}{Lac de Lourdes \Number{} 12} — \href{http://coord.info/GC7CR5C\Number{}754248819}{1279}}\cacheData{{2018/03/27 GOLIATH65, Traditional Cache (1.5/1.5)}}\begin{cacheText}Le rendez-vous est pris depuis la semaine dernière avec Dune33, et DorisBear pour faire cette série. La pluie est annoncée mais cela ne nous décourage pas.

Bien couverts nous nous lançons sur le ....mauvais chemin!!!!Dans l’excitation, nous n’avons pas bien regardé le plan!!!Les caches sont plus ou moins vite trouvées grâce aux 8 yeux et nous logons sous le pseudo DuFleDo. Le circuit est très agréable et les blablas vont bon train...le temps passe très vite .Les indices relevés et le checker au vert nous découvrons la Bonus sans difficulté  avant la pose repas.Merci GOLIATH 65 (et Bigorra65) pour ce bon moment et un PF pour l'ensemble de la série.\end{cacheText}

\cacheNumber{1280}\needspace{5\baselineskip}\cacheName{\href{http://coord.info/GC1JME1}{PAU \Number{}3 - Place de la libération} — \href{http://coord.info/GC1JME1\Number{}754449592}{1280}}\cacheData{{2018/03/29 Gaingain, Traditional Cache (2.5/1.5)}}\begin{cacheText}Petite journée géocaching en compagnie des Fabilab qui sont en vacances dans les PA;C'est en faisant la mystère Tourner autour du PAU que nous trouvons la Belle. Merci\end{cacheText}

\cacheNumber{1281}\needspace{5\baselineskip}\cacheName{\href{http://coord.info/GC1JMF7}{PAU \Number{}4 - Vers la place Gramont} — \href{http://coord.info/GC1JMF7\Number{}754462749}{1281}}\cacheData{{2018/03/29 Gaingain, Multi-cache (2.5/1.5)}}\begin{cacheText}Petite journée géocaching en compagnie des Fabilab qui sont en vacances dans les PA. Cette multi est vite résolue et vite trouvée après le déjeuner. Merci pour la découverte de Bernadotte et de cette place.\end{cacheText}

\cacheNumber{1282}\needspace{5\baselineskip}\cacheName{\href{http://coord.info/GC1JN1H}{PAU \Number{}5 - La tour de la monnaie} — \href{http://coord.info/GC1JN1H\Number{}754464998}{1282}}\cacheData{{2018/03/29 Gaingain, Multi-cache (2.5/1.5)}}\begin{cacheText}Petite journée géocaching en compagnie des Fabilab qui sont en vacances dans les PA. Cette multi est vite résolue et vite trouvée après le déjeuner.Elle nous permet de collecter un indice pour la Mystere Tourner au fond du PAU!!!Celui la nous avait échappé lors de la visite. Arrivés au PZ nous nous fions à l'indice qui nous mène tout droit à la cache.Merci pour la découverte de cette superbe tour.\end{cacheText}

\cacheNumber{1283}\needspace{5\baselineskip}\cacheName{\href{http://coord.info/GC1JN1R}{PAU \Number{}6 - Le boulevard des Pyrénées} — \href{http://coord.info/GC1JN1R\Number{}754468386}{1283}}\cacheData{{2018/03/29 Gaingain, Multi-cache (2.5/2)}}\begin{cacheText}Petite journée géocaching en compagnie des Fabilab qui sont en vacances dans les PA. Cette multi nous permet de visiter la rambarde des Pyrénées assez fréquentée en cette journée ensoleillée. Merci pour la découverte de ce lieu\end{cacheText}

\cacheNumber{1284}\needspace{5\baselineskip}\cacheName{\href{http://coord.info/GC42KND}{Event SdG GR653 L\Underscore{}L-04} — \href{http://coord.info/GC42KND\Number{}754472325}{1284}}\cacheData{{2018/03/29 Charnègues, Traditional Cache (1.5/1.5)}}\begin{cacheText}Au retour de Pau , je m’arrête faire quelques caches supplémentaires. Je continue ce circuit que j’ai commencé il y a quelques temps. Cette cache est trouvée sans aucun problème. Merci pour la cache.\end{cacheText}

\cacheNumber{1285}\needspace{5\baselineskip}\cacheName{\href{http://coord.info/GC42KNF}{Event SdG GR653 L\Underscore{}L-05} — \href{http://coord.info/GC42KNF\Number{}754472835}{1285}}\cacheData{{2018/03/29 Charnègues, Traditional Cache (1.5/1.5)}}\begin{cacheText}Pas de problème pour trouver la Belle. Le log book est détrempé. J’en mets un autre de remplacement. Merci pour la cache.\end{cacheText}

\cacheNumber{1286}\needspace{5\baselineskip}\cacheName{\href{http://coord.info/GC6NY0T}{Parc Beaumont 2} — \href{http://coord.info/GC6NY0T\Number{}754471698}{1286}}\cacheData{{2018/03/29 lolo-64, Traditional Cache (1.5/1.5)}}\begin{cacheText}Petite journée géocaching en compagnie des Fabilab qui sont en vacances dans les PA. Une petite dernière avant de se séparer....La cache est vite trouvée grâce à l'indice et la photo. Nous signons tranquillement :aucun moldu à l'horizon.Merci pour la cache.\end{cacheText}

\cacheNumber{1287}\needspace{5\baselineskip}\cacheName{\href{http://coord.info/GC6RW7X}{CEZERACQ\Number{}1 La Fresque} — \href{http://coord.info/GC6RW7X\Number{}754444396}{1287}}\cacheData{{2018/03/29 PhNTD, Traditional Cache (1.5/1)}}\begin{cacheText}Aujourd’hui direction Pau ou j’ai rendez-vous avec les Fabilab qui arrivent de Loire-Atlantique pour faire quelques caches dans les Pyrénées. En passant je m’arrête faire cette cache. Pas de difficulté , elle est bien en place Les fresques sont magnifiques merci pour la découverte de ce lieu.\end{cacheText}

\cacheNumber{1288}\needspace{5\baselineskip}\cacheName{\href{http://coord.info/GC2BPWE}{Notre Dame des cyclistes} — \href{http://coord.info/GC2BPWE\Number{}754848945}{1288}}\cacheData{{2018/03/30 Samlepirate, Traditional Cache (1/1)}}\begin{cacheText}En arrivant je découvre une superbe église. Le site est magnifique. La boîte est vite trouvée mais comme d’habitude il n’y a aucun TB ....dommage!!!! Merci pour la découverte de ce superbe endroit.\end{cacheText}

\cacheNumber{1289}\needspace{5\baselineskip}\cacheName{\href{http://coord.info/GC2DR03}{confluence 44/0} — \href{http://coord.info/GC2DR03\Number{}754852813}{1289}}\cacheData{{2018/03/30 pap's, Traditional Cache (2/1.5)}}\begin{cacheText}En route pour le Lot-et-Garonne j’ai décidé de m’arrêter faire quelques caches. Celle-ci est encore verte sur la carte et je décide d’y aller. Arrivée au PZ aucune difficulté pour mettre la main dessus. Le log book est trempé il n’y a plus de couvercle. J’assure la maintenance avec un logbook tout neuf. Merci pour la cache\end{cacheText}

\cacheNumber{1290}\needspace{5\baselineskip}\cacheName{\href{http://coord.info/GC5KTG9}{La fontaine Saint Pierre} — \href{http://coord.info/GC5KTG9\Number{}754843796}{1290}}\cacheData{{2018/03/30 fdcdm, Traditional Cache (1.5/2)}}\begin{cacheText}En partance pour le Lot-et-Garonne, je ne peux m’empêcher de m’arrêter à cette cache qui me fait de l’œil. Arrivée au PZ je fouille à l’endroit et je vois le moment où je vais rentrer bredouille. Ouffff elle est bien là dans les bras du Picata. Merci pour la découverte de cette superbe fontaine.\end{cacheText}

\cacheNumber{1291}\needspace{5\baselineskip}\cacheName{\href{http://coord.info/GC6T7MZ}{Le lavoir du Tay de Saint Justin au Frêche} — \href{http://coord.info/GC6T7MZ\Number{}754845078}{1291}}\cacheData{{2018/03/30 fdcdm, Traditional Cache (1.5/1.5)}}\begin{cacheText}J’ai mis un peu de temps pour dénicher la Belle. Elle est bien planquée sous les tôles  à l’abri de l’eau. Merci pour la cache et la découverte de ce  très joli petit lavoir.\end{cacheText}

\cacheNumber{1292}\needspace{5\baselineskip}\cacheName{\href{http://coord.info/GC6WE8H}{Le lavoir de Bourron} — \href{http://coord.info/GC6WE8H\Number{}754846527}{1292}}\cacheData{{2018/03/30 fdcdm, Traditional Cache (1.5/2)}}\begin{cacheText}Je continue ma quête et j’arrive à ce charmant petit lavoir qui mériterait un peu d’attention. La cache est rapidement trouvée :il faut se fier à l'indice. Merci Eric.\end{cacheText}

\cacheNumber{1293}\needspace{5\baselineskip}\cacheName{\href{http://coord.info/GC7F9W3}{VFDQ : Eglise Saint Savin} — \href{http://coord.info/GC7F9W3\Number{}754854304}{1293}}\cacheData{{2018/03/30 Bugs\And{}Co, Traditional Cache (1.5/1.5)}}\begin{cacheText}De retour sur les terres Lot-et-Garonnaise j’en profite pour loguer la deuxième cache de VFDQ. Le GPS me mène tout droit à la cache. Merci pour la découverte de cette très belle église.\end{cacheText}

\cacheNumber{1294}\needspace{5\baselineskip}\cacheName{\href{http://coord.info/GC3NRYT}{Fougueyrolles} — \href{http://coord.info/GC3NRYT\Number{}755400234}{1294}}\cacheData{{2018/03/31 SI MGL, Traditional Cache (1.5/1.5)}}\begin{cacheText}En Dordogne pour assister à un match de rugby de mon petit fiston j’en profite pour faire quelques caches aux alentours. Je découvre ce charmant petit village où il n’y a aucun aucun moldu : l’église et le petit square sont charmants. La Belle est découverte sans aucun problème. Merci pour la cache et pour la découverte de ce lieu.\end{cacheText}

\cacheNumber{1295}\needspace{5\baselineskip}\cacheName{\href{http://coord.info/GC61EKD}{Vue sur le golf} — \href{http://coord.info/GC61EKD\Number{}755420922}{1295}}\cacheData{{2018/03/31 vivic54, Traditional Cache (1.5/1.5)}}\begin{cacheText}De passage en Dordogne pour voir mon fils jouer un match de rugby j’en profite pour faire quelques caches aux alentours. Arrivée au PZ la vue sur le golf est très sympa , les fruitiers sont en fleurs et les maisons environnantes anciennes. Après avoir lu les commentaires je me dirige vers la Belle que je trouve du premier coup. Merci pour la cache et la découverte de ce lieu.\end{cacheText}

\cacheNumber{1296}\needspace{5\baselineskip}\cacheName{\href{http://coord.info/GC6FMKD}{Aliénor d'Aquitaine} — \href{http://coord.info/GC6FMKD\Number{}755390224}{1296}}\cacheData{{2018/03/31 Madie2611, Traditional Cache (1/1.5)}}\begin{cacheText}De passage en Dordogne pour assister à un match de rugby du petit, j’en profite pour faire quelques caches sur Sainte-Foy. Celle-ci est trouvée sans aucun problème :l'indice est explicite. Merci pour la cache\end{cacheText}

\cacheNumber{1297}\needspace{5\baselineskip}\cacheName{\href{http://coord.info/GC6FRH8}{LA BRECHE} — \href{http://coord.info/GC6FRH8\Number{}755399029}{1297}}\cacheData{{2018/03/31 Madie2611, Traditional Cache (1/1.5)}}\begin{cacheText}Arrivée au PZ je ne mets pas beaucoup de temps pour déloger la Belle. Merci pour la découverte de ses bords de Dordogne. Et pour la cache.\end{cacheText}

\cacheNumber{1298}\needspace{5\baselineskip}\cacheName{\href{http://coord.info/GC6FRKF}{LES POMPIERS} — \href{http://coord.info/GC6FRKF\Number{}755393758}{1298}}\cacheData{{2018/03/31 Madie2611, Traditional Cache (1/1)}}\begin{cacheText}Après la gare, me voici devant la caserne de Sainte-Foy. La cache est rapidement trouvée. Pas de moldu à l'horizon je signe et repart. Merci pour la cache\end{cacheText}

\cacheNumber{1299}\needspace{5\baselineskip}\cacheName{\href{http://coord.info/GC6G06Q}{LA ROCADE} — \href{http://coord.info/GC6G06Q\Number{}755388424}{1299}}\cacheData{{2018/03/31 Madie2611, Traditional Cache (1/1.5)}}\begin{cacheText}Arrivée au PZ il pleut des cordes et le vent souffle fort. J'hésite un peu avant de sortir de la voiture....puis je fonce. Les recherches ne sont pas faciles mais je finis par mettre la main sur le Graal.Des arbres ont été coupés, espérons que la cache ne disparaisse pas!!!Merci pour la cache.\end{cacheText}

\cacheNumber{1300}\needspace{5\baselineskip}\cacheName{\href{http://coord.info/GC6H5Z9}{Aérodrome} — \href{http://coord.info/GC6H5Z9\Number{}755427617}{1300}}\cacheData{{2018/03/31 Madie2611, Traditional Cache (1/1)}}\begin{cacheText}Encore quelques caches sur le retour. Je découverte dans le cyprès blessé la Belle Il y a beaucoup de vent sur ces hauteurs mais une superbe vue. Merci pour la découverte de l’aérodrome et de sa cache.\end{cacheText}

\cacheNumber{1301}\needspace{5\baselineskip}\cacheName{\href{http://coord.info/GC6KVJH}{LE PONT FERRE} — \href{http://coord.info/GC6KVJH\Number{}755398499}{1301}}\cacheData{{2018/03/31 Madie2611, Traditional Cache (1.5/1.5)}}\begin{cacheText}Arrivée au PZ, l’indice est très explicite et il me mène tout droit à la cache. Pas trop de moldus à cette heure ci ,je peux loguer discrètement. Merci pour la cache\end{cacheText}

\cacheNumber{1302}\needspace{5\baselineskip}\cacheName{\href{http://coord.info/GC758G7}{eglise de st etienne de londre} — \href{http://coord.info/GC758G7\Number{}755386602}{1302}}\cacheData{{2018/03/31 reronim, Traditional Cache (2/1.5)}}\begin{cacheText}De passage dans le Lot-et-Garonne, j’en profite pour faire quelques caches. Ici je découvre une très jolie église qui mériterait une petite restauration. La cache est délogée non pas sans mal!!!Très bon camouflage qui se fond dans le décor. Merci pour la découverte de ce lieu.\end{cacheText}

\cacheNumber{1303}\needspace{5\baselineskip}\cacheName{\href{http://coord.info/GC778TY}{LAVOIR \Quoted{LES BOURNETS}} — \href{http://coord.info/GC778TY\Number{}755387715}{1303}}\cacheData{{2018/03/31 Madie2611, Traditional Cache (1.5/1.5)}}\begin{cacheText}Arrivée au PZ c'est une jolie surprise!!!Le lavoir est aménagé avec beaucoup de gout. Le GPS me mène tout droit à la cache.Merci pour cette belle découverte.\end{cacheText}

\cacheNumber{1304}\needspace{5\baselineskip}\cacheName{\href{http://coord.info/GC778W4}{LAVOIR \Quoted{LA RAYRE}} — \href{http://coord.info/GC778W4\Number{}755386967}{1304}}\cacheData{{2018/03/31 Madie2611, Traditional Cache (1.5/1.5)}}\begin{cacheText}De passage dans le Lot-et-Garonne, j’en profite pour faire quelques caches. C’est un très joli lavoir que je découvre ici. La cache est trouvée sans aucun problème grâce à l’indice. Merci pour la cache et la découverte de ce lieu.\end{cacheText}

\cacheNumber{1305}\needspace{5\baselineskip}\cacheName{\href{http://coord.info/GC7AB19}{Les Bardoulets} — \href{http://coord.info/GC7AB19\Number{}755399349}{1305}}\cacheData{{2018/03/31 gdrara, Traditional Cache (1.5/2)}}\begin{cacheText}De passage en Dordogne pour aller voir le fiston qui  dispute un match de rugby j’en profite pour faire quelques caches sur Sainte-Foy-la-Grande et les environs. Arrivée près du PZ aucun doute possible la cache dois être là. Effectivement je la trouve parterre :elle aurait besoin d’une petite maintenance. Je la repose mais j’ai peur qu’elle ne reste pas en place très longtemps. Merci pour la découverte de ses bords de Dordogne\end{cacheText}

\cacheNumber{1306}\needspace{5\baselineskip}\cacheName{\href{http://coord.info/GC7KVJ6}{BIENVENUE A SAINTE FOY LA GRANDE} — \href{http://coord.info/GC7KVJ6\Number{}755393487}{1306}}\cacheData{{2018/03/31 Madie2611, Traditional Cache (1.5/1.5)}}\begin{cacheText}(STF)

De passage dans le coin j’en profite pour faire quelques caches. L'indice me montre le chemin. Belle surprise j’arrive en second sur le log book ....un petit STF. La cache est apparente , je la camoufle avec des feuilles. Merci pour la cache\end{cacheText}

\cacheNumber{1307}\needspace{5\baselineskip}\cacheName{\href{http://coord.info/GC49F3G}{Raymond le Vicomte} — \href{http://coord.info/GC49F3G\Number{}755589351}{1307}}\cacheData{{2018/04/01 dorisbear, Multi-cache (1.5/1.5)}}\begin{cacheText}Avec ce beau dimanche ensoleillé, il est difficile de rester à la maison. Direction cagnotte et sa Multi. Je n’étais jamais venu voir l’abbaye je ne suis pas déçu. J’en apprends un peu plus sur les pratiques funéraires du Moyen Âge et je commence à glaner les indices. Les bonnes coordonnées empoche je me dirige vers le pays aide. Pas de difficultés pour trouver la petite boîte. Merci les DorisBear pour ce bon moment\end{cacheText}

\cacheNumber{1308}\needspace{5\baselineskip}\cacheName{\href{http://coord.info/GC79V79}{Le théâtre de verdure} — \href{http://coord.info/GC79V79\Number{}755592697}{1308}}\cacheData{{2018/04/01 Ethano 40 featuring zee brain, Multi-cache (2.5/1.5)}}\begin{cacheText}Après avoir trouvé la plaque commémorative, compté les portes drapeaux et les talanquères je calcule les coordonnées qui m’amènent au PZ. Ici les choses se compliquent :cela fait presque une demi-heure que je cherche, plus ou moins dérangée par les moldus qui profitent du soleil pour se promener.J’allais abandonner lorsque BINGO je finis par mettre la main dessus. Merci Ethano 40  pour cette cache.\end{cacheText}

\cacheNumber{1309}\needspace{5\baselineskip}\cacheName{\href{http://coord.info/GC7K3FG}{aire camionneur} — \href{http://coord.info/GC7K3FG\Number{}755597207}{1309}}\cacheData{{2018/04/01 lauki3940, Traditional Cache (1.5/1.5)}}\begin{cacheText}(TTF)

C’est un rentrant de Pouillon et pour la énième fois que je passe devant le PZ. Le parking est toujours occupé par les camions et aujourd'hui aussi il y en a un stationné juste devant.... la portière grande ouverte!!!!Tant pis j'y vais ....au pire je lui expliquerai le jeu. J'ai pu chercher et loguer tranquille en deux temps trois mouvements. Heureuse surprise je suis troisième. Merci lauki3940 pour la cache\end{cacheText}

\cacheNumber{1310}\needspace{5\baselineskip}\cacheName{\href{http://coord.info/GC33QFJ}{la Cascade Fleurie} — \href{http://coord.info/GC33QFJ\Number{}756960751}{1310}}\cacheData{{2018/04/05 dune33, Traditional Cache (1.5/2)}}\begin{cacheText}C' est lors de l'Event La cote de Jade en visite à Eugénie-les-Bains que nous nous échappons avec la famille la fleur pour découvrir la cache. Arrivés au PZ nous explorons l'endroit et finissons par mettre la main sur la Belle. Ouffff personne n'a fait Plouffff !!!!L'endroit superbe, est à visiter absolument. Merci les dune33 pour cette belle découverte et un PF.\end{cacheText}

\cacheNumber{1311}\needspace{5\baselineskip}\cacheName{\href{http://coord.info/GC4PT5T}{Alphabet Landais ... E pour Eugénie les Bains} — \href{http://coord.info/GC4PT5T\Number{}757077023}{1311}}\cacheData{{2018/04/05 nanard40270, Traditional Cache (1.5/1.5)}}\begin{cacheText}C' est lors de l'Event La cote de Jade en visite à Eugénie-les-Bains que nous nous échappons avec la famille la fleur pour découvrir les caches environnantes. Arrivés au PZ nous découvrons un superbe parc tout fleuri. L'endroit est magnifique et inspire le repos. Pendant que je m'extasie le petit Lafleur déloge la Belle et évidemment il n'y a pas de TB comme annoncé!!!.Merci les nanard40270 et un PF pour la découverte de cette magnifique ferme landaise.\end{cacheText}

\cacheNumber{1312}\needspace{5\baselineskip}\cacheName{\href{http://coord.info/GC6QNVD}{CHATEAU DES TEMPS MODERNES} — \href{http://coord.info/GC6QNVD\Number{}757077902}{1312}}\cacheData{{2018/04/05 DuPaTé, Traditional Cache (1.5/1.5)}}\begin{cacheText}C'est au retour de l'Event La Cote de Jade en visite à Eugénie-les-Bains que je m'arrête au château pour déloger la Belle. Il n'y a pas beaucoup d'endroit ou la planquer au PZ et elle est vite trouvée. Heureusement car la nuit tombe!!!Merci pour la cache.\end{cacheText}

\cacheNumber{1313}\needspace{5\baselineskip}\cacheName{\href{http://coord.info/GC7K5Y6}{La Côte de Jade en visite à Eugénie-les -Bains} — \href{http://coord.info/GC7K5Y6\Number{}757247980}{1313}}\cacheData{{2018/04/05 HelPat, Event Cache (1/1)}}\begin{cacheText}17h00 ...enfin l'heure de débaucher et partir rejoindre les géocopains. Pas Le temps de s'arrêter faire quelques caches car le temps est compté. Arrivée sur le lieu du rendez vous les locaux sont déjà entrain de blablater avec Hélène et Patrick. Après avoir dégusté les madeleines et le cidre offert par les organisateurs, échangé les TB et fait quelques caches il est temps de rentrer. Le temps est passé beaucoup trop vite encore une fois. Merci les HelPat pour ce bon moment et au plaisir de vous revoir (très certainement à l'AE).\end{cacheText}

\cacheNumber{1314}\needspace{5\baselineskip}\cacheName{\href{http://coord.info/GC208KH}{La Main du Pignada} — \href{http://coord.info/GC208KH\Number{}757261458}{1314}}\cacheData{{2018/04/06 gilles64, Multi-cache (2/2)}}\begin{cacheText}Sur le retour de la superbonus je regarde la multi d'un peu plus près.Il faut bien relire la description de la cache car les waypoints sont tous la!!!! Après avoir fait les différents calculs  je me dirige vers la cache. Arrivée au PZ il me faut peu de temps pour repérer le multi tronc . Je cherche au milieu des aiguilles et des feuilles et je découvre enfn la Belle. Le logbook est inutilisable malgré qu’il soit bien à l’abri dans le pot de Nutella. Je le change. Merci Gilles pour cette belle promenade\end{cacheText}

\cacheNumber{1315}\needspace{5\baselineskip}\cacheName{\href{http://coord.info/GC4Y1VT}{CR1 11 Bonus  Chiberta - Balade du Pignada} — \href{http://coord.info/GC4Y1VT\Number{}757255898}{1315}}\cacheData{{2018/04/06 patricia64, Unknown Cache (2.5/1.5)}}\begin{cacheText}Après avoir fait x fois les calculs et avoir eu confirmation de l'owner,j'ai enfin les bonnes coordonnées. La cache est rapidement trouvée grâce à l’indice. Je n’étais pas si loin que ça la dernière fois...Merci pour la découverte de cette superbe forêt de Chiberta.\end{cacheText}

\cacheNumber{1316}\needspace{5\baselineskip}\cacheName{\href{http://coord.info/GC54FMG}{L'Adour Nautique} — \href{http://coord.info/GC54FMG\Number{}757253854}{1316}}\cacheData{{2018/04/06 Gamboy, Letterbox Hybrid (2.5/1.5)}}\begin{cacheText}Enfin trouvée... au troisième passage!!!! La Multi a été vite résolue et le checker est passé au vert .Par deux fois je suis venue mais l’endroit était occupé par des promeneurs ou par des personnes qui s’affairaient auprès de leur bateau. Aujourd’hui 13h30 il n’y a personne et je peux enfin chercher tranquille. Vite délogée je tamponne mon carnet ( le tampon est mal-en-point je mets de l’encre partout !) Merci les Gamboy pour ce bon moment passé.\end{cacheText}

\cacheNumber{1317}\needspace{5\baselineskip}\cacheName{\href{http://coord.info/GC76H1Y}{CR1/CR2 Super Bonus Chiberta - Ballade du Pignada} — \href{http://coord.info/GC76H1Y\Number{}757257547}{1317}}\cacheData{{2018/04/06 gilles64, Unknown Cache (4/1.5)}}\begin{cacheText}Enfin j’arrive à la conclusion du parcours. Champagne ! ! ! Merci Gilles et Patricia pour cette série que je recommande particulièrement. J’ai adoré.Un PF pour récompense.\end{cacheText}

\cacheNumber{1318}\needspace{5\baselineskip}\cacheName{\href{http://coord.info/GC2FF84}{Le Lavoir Couvert} — \href{http://coord.info/GC2FF84\Number{}759624116}{1318}}\cacheData{{2018/04/14 lokateo64, Traditional Cache (1.5/1.5)}}\begin{cacheText}En arrivant au PZ, je découvre un joli petit lavoir entretenu. La cache ne me résiste pas longtemps et est vite attrapée. Merci pour la découverte de ce joli lavoir.\end{cacheText}

\cacheNumber{1319}\needspace{5\baselineskip}\cacheName{\href{http://coord.info/GC2FFCA}{La Vierge du Casterot} — \href{http://coord.info/GC2FFCA\Number{}759625596}{1319}}\cacheData{{2018/04/14 lokateo64, Traditional Cache (1.5/3.5)}}\begin{cacheText}Sur le retour de Monein, je m’arrête faire cette cache. L’ascension est assez difficile mais arrivée au sommet je découvre une magnifique vue ainsi qu’une superbe statue de la vierge. Les coordonnées sont précises et la cache rapidement trouvée. J’assure la maintenance avec un logbook tout neuf car l’autre est tout trempé. Merci pour la découverte de ce lieu.\end{cacheText}

\cacheNumber{1320}\needspace{5\baselineskip}\cacheName{\href{http://coord.info/GC2FFCJ}{Eglise Saint-Girons} — \href{http://coord.info/GC2FFCJ\Number{}759623148}{1320}}\cacheData{{2018/04/14 lokateo64, Traditional Cache (1.5/1.5)}}\begin{cacheText}Sur Monein pour assister à un match de rugby, je m’éclipse pour faire quelques caches .L’église Saint girons est splendide et serait l'œuvre de cagots...Il y a quelques moldus dans les environs mais avec de la patience j'arrive à sortir la cache discrètement.. Merci pour la découverte de ce lieu.\end{cacheText}

\cacheNumber{1321}\needspace{5\baselineskip}\cacheName{\href{http://coord.info/GC2GY5F}{La table d'orientation d'Ogenne-Camptort} — \href{http://coord.info/GC2GY5F\Number{}759626424}{1321}}\cacheData{{2018/04/14 lokateo64, Traditional Cache (1.5/1.5)}}\begin{cacheText}Arrivée au PZ, l’indice me mène tout droit à la cache. Elle est en parfait état et est parfaitement protégée. Cette table d’orientation est très surprenante perdue au milieu de la campagne. Très belle découverte. Merci pour la cache\end{cacheText}

\cacheNumber{1322}\needspace{5\baselineskip}\cacheName{\href{http://coord.info/GC4GF2F}{Alphabet des Pyrénées-Atlantiques A pour Argagnon} — \href{http://coord.info/GC4GF2F\Number{}760056741}{1322}}\cacheData{{2018/04/15 MaryeLoup, Traditional Cache (1.5/1.5)}}\begin{cacheText}Après Maslacq, direction Argagnon ou la maintenance a été effectuée. Après avoir lu le log précédent je découvre la Belle en deux temps trois mouvements. Merci pour la découverte de cette jolie église .\end{cacheText}

\cacheNumber{1323}\needspace{5\baselineskip}\cacheName{\href{http://coord.info/GC7BNDR}{ToutencarMont \Number{}1} — \href{http://coord.info/GC7BNDR\Number{}760060056}{1323}}\cacheData{{2018/04/15 ayous, Traditional Cache (1.5/1.5)}}\begin{cacheText}Après Maslacq et Argagnon le temps semble changer. Je choisis donc ce circuit en drive pour être sure de ne pas me mouiller. Arrivée au PZ  je me laisse guider par le GPS qui me mène tout droit à la cache. Du joli bricolage : bravo. Merci pour la cache.\end{cacheText}

\cacheNumber{1324}\needspace{5\baselineskip}\cacheName{\href{http://coord.info/GC7BNJE}{ToutencarMont \Number{}2} — \href{http://coord.info/GC7BNJE\Number{}760085546}{1324}}\cacheData{{2018/04/15 ayous, Traditional Cache (1.5/1.5)}}\begin{cacheText}Garée presque devant la cache, je n’ai pas de difficultés pour la dénicher.  Merci pour la cache.\end{cacheText}

\cacheNumber{1325}\needspace{5\baselineskip}\cacheName{\href{http://coord.info/GC7BNK0}{ToutencarMont \Number{}4} — \href{http://coord.info/GC7BNK0\Number{}760089528}{1325}}\cacheData{{2018/04/15 ayous, Traditional Cache (1.5/1.5)}}\begin{cacheText}Arrivée au PZ, il n'y a personne à l’horizon .Je me gare et logue en toute tranquillité. Merci pour la cache.\end{cacheText}

\cacheNumber{1326}\needspace{5\baselineskip}\cacheName{\href{http://coord.info/GC7BNNN}{ToutencarMont \Number{}5} — \href{http://coord.info/GC7BNNN\Number{}760091275}{1326}}\cacheData{{2018/04/15 ayous, Traditional Cache (1.5/1.5)}}\begin{cacheText}Le GPS me joue des tours ....peu importe la cache est vite trouvée . Merci\end{cacheText}

\cacheNumber{1327}\needspace{5\baselineskip}\cacheName{\href{http://coord.info/GC7BNNV}{ToutencarMont \Number{}6} — \href{http://coord.info/GC7BNNV\Number{}760091682}{1327}}\cacheData{{2018/04/15 ayous, Traditional Cache (1.5/1.5)}}\begin{cacheText}Ici aussi la cache est bien là où je l’attendais. Merci pour la découverte de ce village de Mont.\end{cacheText}

\cacheNumber{1328}\needspace{5\baselineskip}\cacheName{\href{http://coord.info/GC7N1FE}{La tour de Maslacq} — \href{http://coord.info/GC7N1FE\Number{}760054528}{1328}}\cacheData{{2018/04/15 Titiger39, Traditional Cache (1.5/1.5)}}\begin{cacheText}(FTF)

Ce matin des nouvelles caches viennent de paraître mais malheureusement d’autres occupations sont prioritaires. Cet après-midi il fait beau pourquoi pas !!!En route pour Masclacq!!! Arrivée au PZ je ne mets pas longtemps pour découvrir la Belle grâce à l'indice. Et... bonne surprise je suis FTF. Merci Titiger39 pour cette cache qui me permet de découvrir cette jolie tour.\end{cacheText}

\cacheNumber{1329}\needspace{5\baselineskip}\cacheName{\href{http://coord.info/GC1JMDY}{PAU \Number{}2 - Du château au Hédas} — \href{http://coord.info/GC1JMDY\Number{}760712645}{1329}}\cacheData{{2018/04/17 Gaingain, Multi-cache (2.5/1.5)}}\begin{cacheText}Rendez vous est pris avec Bigorra65 et le soleil pour faire quelques caches sur Pau. Après la visite du Parc du château nous nous lançons sur cette multi et tombons sur un indice qu'il nous manquait pour la mystère Béarnais\Number{}6.Chouette!!!!

Après avoir compté les marches nous rejoignons le PZ. Un groupe de jeunes moldus semble bien installé devant le spot et à force de tourner et virer( et avec beaucoup de patience )nous arrivons à les faire fuir. La cache est rapidement trouvée car elle est à la vue de tous. Merci pour la visite du Hédas et pour la cache.

:)1336:)\end{cacheText}

\cacheNumber{1330}\needspace{5\baselineskip}\cacheName{\href{http://coord.info/GC7EBCK}{Parc du Chateau de Pau \Number{}1} — \href{http://coord.info/GC7EBCK\Number{}760527127}{1330}}\cacheData{{2018/04/17 TeamPilar64, Traditional Cache (1.5/1.5)}}\begin{cacheText}Rendez vous est pris avec Bigorra65 et le soleil pour faire quelques caches sur Pau. Ici les jardiniers sont à l'œuvre pour traiter les buis du parc. Il nous faut donc patienter pour accéder aux divers PZ;

Le GPS n'est pas très précis mais la photo nous guide vers la Belle. Je récupère le TB et remet tout en place.Merci



:)1329 :)\end{cacheText}

\cacheNumber{1331}\needspace{5\baselineskip}\cacheName{\href{http://coord.info/GC7EBDD}{Parc du Chateau de Pau \Number{}2} — \href{http://coord.info/GC7EBDD\Number{}760584788}{1331}}\cacheData{{2018/04/17 TeamPilar64, Traditional Cache (1.5/1.5)}}\begin{cacheText}Rendez vous est pris avec Bigorra65 et le soleil pour faire quelques caches sur Pau. Ici les jardiniers sont à l'œuvre pour traiter les buis du parc. Il nous faut donc patienter pour accéder aux divers PZ.

Arrivées dans la zone, il n'y a aucun moldu. Les recherches vont bon train et la cache se dévoile rapidement dans la souche.Merci



:) 1330 :)\end{cacheText}

\cacheNumber{1332}\needspace{5\baselineskip}\cacheName{\href{http://coord.info/GC7EBDT}{Parc du Chateau de Pau \Number{}3} — \href{http://coord.info/GC7EBDT\Number{}760585245}{1332}}\cacheData{{2018/04/17 TeamPilar64, Traditional Cache (1.5/1.5)}}\begin{cacheText}Rendez vous est pris avec Bigorra65 et le soleil pour faire quelques caches sur Pau. Ici les jardiniers sont à l'œuvre pour traiter les buis du parc. Il nous faut donc patienter pour accéder aux divers PZ.

 De nombreux Moldus profitent du soleil, et ce PZ est très fréquenté:s ce n’est pas facile de rester discrètes. Nous simulons une pause sur la souche pour loguer la Belle. Le logbook a été mal replié et nous nous en voyons pour le sortir intact!!!!Pour les suivans merci de remettre tout bien en place correctement.

Merci pour la cache.



:) 1331 :)\end{cacheText}

\cacheNumber{1333}\needspace{5\baselineskip}\cacheName{\href{http://coord.info/GC7EBE3}{Parc du Chateau de Pau \Number{}4} — \href{http://coord.info/GC7EBE3\Number{}760586186}{1333}}\cacheData{{2018/04/17 TeamPilar64, Traditional Cache (1.5/1.5)}}\begin{cacheText}Rendez vous est pris avec Bigorra65 et le soleil pour faire quelques caches sur Pau. Ici les jardiniers sont à l'œuvre pour traiter les buis du parc. Il nous faut donc patienter pour accéder aux divers PZ.

Une très belle souche abrite la boite .Ici les géocacheurs précédents ne se sont pas donné la peine de mette le plastique autour du logbook , ni même de le ranger dans le sachet en plastique. Vraiment pas cool pour la survie des caches!!!Merci  TeamPilar64 pour la découverte de ce superbe parc du château.

:) 1332 :)\end{cacheText}

\cacheNumber{1334}\needspace{5\baselineskip}\cacheName{\href{http://coord.info/GC7EBEC}{Parc du Chateau de Pau \Number{}5} — \href{http://coord.info/GC7EBEC\Number{}760704780}{1334}}\cacheData{{2018/04/17 TeamPilar64, Traditional Cache (1.5/1.5)}}\begin{cacheText}Rendez vous est pris avec Bigorra65 et le soleil pour faire quelques caches sur Pau. Ici les jardiniers sont à l'œuvre pour traiter les buis du parc. Il nous faut donc patienter pour accéder aux divers PZ.

Il y a beaucoup de promeneurs, il faut être rapide et discrète .Mission accomplie nous continuons...Merci pour la cache.

:) 1333 :)\end{cacheText}

\cacheNumber{1335}\needspace{5\baselineskip}\cacheName{\href{http://coord.info/GC7EBEQ}{Parc du Chateau de Pau \Number{}6} — \href{http://coord.info/GC7EBEQ\Number{}760707582}{1335}}\cacheData{{2018/04/17 TeamPilar64, Traditional Cache (1.5/1.5)}}\begin{cacheText}Rendez vous est pris avec Bigorra65 et le soleil pour faire quelques caches sur Pau. Ici les jardiniers sont à l'œuvre pour traiter les buis du parc. Il nous faut donc patienter pour accéder aux divers PZ.

La Belle est vite trouvée par Bigorra65.Des moldus s'attardent devant le PZ...zut pas de chance il faut encore attendre!!!Finalement nous pouvons tout remettre en place.Merci pour la cache.

:)1334:)\end{cacheText}

\cacheNumber{1336}\needspace{5\baselineskip}\cacheName{\href{http://coord.info/GC7EBEY}{Parc du Chateau de Pau \Number{}7} — \href{http://coord.info/GC7EBEY\Number{}760708970}{1336}}\cacheData{{2018/04/17 TeamPilar64, Traditional Cache (1.5/1.5)}}\begin{cacheText}Rendez vous est pris avec Bigorra65 et le soleil pour faire quelques caches sur Pau. Ici les jardiniers sont à l'œuvre pour traiter les buis du parc. Il nous faut donc patienter pour accéder aux divers PZ.

Pas facile à trouver la coquine mais nous finissons par mettre la main dessus.Merci pour la cache.

:)1335:)\end{cacheText}

\cacheNumber{1337}\needspace{5\baselineskip}\cacheName{\href{http://coord.info/GC7GC3A}{Béarnais \Number{}6 - Tourner autour du \Quoted{Pau}} — \href{http://coord.info/GC7GC3A\Number{}760725080}{1337}}\cacheData{{2018/04/17 ayous, Unknown Cache (3/3)}}\begin{cacheText}Rendez vous est pris avec Bigorra65 et le soleil pour faire quelques caches sur Pau. Nous commençons par faire un premier tour :il nous manque 3 photos. Grrrr ...nous repartons et nous ne voyons rien de plus. Nous décidons donc de faire la pause déjeuner et d'enchainer avec les tradis du Parc du Château( nous n'allons tout de même pas rentrer bredouille!!!!).Alors que l'on se lance sur une multi nous tombons nez à nez avec une photo!!!Comment avons nous pu ne pas la voir????Grace à une précieuse aide nous finissons par obtenir les renseignements et passer le checker au vert. Arrivées au PZ qui n'est pas trop fréquenté ,nous trouvons la Belle en deux temps trois mouvements.

Un grand merci Ayous pour avoir importé ce super concept de caches en béarn.J'adoooore et un PF pour récompense.

:)1337:)\end{cacheText}

\cacheNumber{1338}\needspace{5\baselineskip}\cacheName{\href{http://coord.info/GC7J74H}{Tourner au fond du PAU} — \href{http://coord.info/GC7J74H\Number{}760789787}{1338}}\cacheData{{2018/04/17 ayous, Unknown Cache (2/2.5)}}\begin{cacheText}C'est en compagnie de Fabilab et de leurs amis moldus, que j'ai découvert le fond de Pau : 17 des 18 indices ont été repérés ,dont un tout à fait par hasard,et grâce à l'owner nous obtiendrons l'indice manquant. Malheureusement nous étions sur le retour...

Aujourd'hui c'est avec Bigorra65 que je parcours les rues de Pau et je fais un léger détour pour déloger la Belle. Un sms de Fabilab me confirme le PZ malgré les doutes de Bigorra65.

Encore bravo Ayous pour avoir adapté le concept à la ville de Pau.J'ai passé une super journée à découvrir le patrimoine palois.Un PF évidemment.

:)1338:)\end{cacheText}

\cacheNumber{1339}\needspace{5\baselineskip}\cacheName{\href{http://coord.info/GC20DN8}{Chemin du Port} — \href{http://coord.info/GC20DN8\Number{}761760825}{1339}}\cacheData{{2018/04/22 TTS et Betty P, Traditional Cache (2/1.5)}}\begin{cacheText}Arrivée au PZ, je n’ai aucun mal à trouver la Belle qui est bien accrochée à son arbre. La Garonne peux monter, elle restera… Merci pour la cache\end{cacheText}

\cacheNumber{1340}\needspace{5\baselineskip}\cacheName{\href{http://coord.info/GC20DNN}{Le parc aux nénuphars} — \href{http://coord.info/GC20DNN\Number{}761760130}{1340}}\cacheData{{2018/04/22 TTS et Betty P, Traditional Cache (1.5/1.5)}}\begin{cacheText}Je continue les recherches par cette belle journée ensoleillée. Les gens qui font un pique-nique au loin , ne m’empêchent pas de trouver la cache. Merci pour la découverte de ce lieu et de ses bords de Garonne.\end{cacheText}

\cacheNumber{1341}\needspace{5\baselineskip}\cacheName{\href{http://coord.info/GC20DQ8}{Le pont du Diable / Tersac} — \href{http://coord.info/GC20DQ8\Number{}762005597}{1341}}\cacheData{{2018/04/22 TTS et Betty P, Traditional Cache (2/1.5)}}\begin{cacheText}Ici l'indice est hyper précis et le  GPS me mène tout droit à la Belle. Il n’y a rien à dire sinon que le lieu est superbe. Merci pour la découverte de ce pont et pour la cache.\end{cacheText}

\cacheNumber{1342}\needspace{5\baselineskip}\cacheName{\href{http://coord.info/GC20E83}{Vestiges de four à tuiles} — \href{http://coord.info/GC20E83\Number{}761762643}{1342}}\cacheData{{2018/04/22 TTS et Betty P, Traditional Cache (1.5/1.5)}}\begin{cacheText}Ici , je découvre un magnifique  four à tuiles. Dommage qu’il ne soit pas mieux entretenu. La Belle m'a donné du mal mais j'ai finit par la déloger. Merci pour la découverte de ce lieu\end{cacheText}

\cacheNumber{1343}\needspace{5\baselineskip}\cacheName{\href{http://coord.info/GC20FXG}{Le Château de Rachac} — \href{http://coord.info/GC20FXG\Number{}761755891}{1343}}\cacheData{{2018/04/22 TTS et Betty P, Traditional Cache (1.5/1.5)}}\begin{cacheText}De passage sur Cazeres, j’en profite pour faire quelques caches et pour valider le département de Haute-Garonne. La cache est très rapidement trouvée grâce à l’indice. Je découvre ici une superbe maison. Merci pour la découverte de ce lieu et de la cache.\end{cacheText}

\cacheNumber{1344}\needspace{5\baselineskip}\cacheName{\href{http://coord.info/GC20FXN}{Cazères} — \href{http://coord.info/GC20FXN\Number{}761756472}{1344}}\cacheData{{2018/04/22 TTS et Betty P, Traditional Cache (1.5/1.5)}}\begin{cacheText}De passage sur cazeres j’en profite pour faire les quelques caches de la ville. Celle-ci est rapidement trouvée: elle est toujours en place. Je découvre un superbe bâtiment. Merci pour la découverte de ce lieu.\end{cacheText}

\cacheNumber{1345}\needspace{5\baselineskip}\cacheName{\href{http://coord.info/GC20WEQ}{Le Moulin des Donades} — \href{http://coord.info/GC20WEQ\Number{}762007601}{1345}}\cacheData{{2018/04/22 TTS et Betty P, Traditional Cache (1.5/1.5)}}\begin{cacheText}Heureusement pour moi, le chemin est tout tracé par les chercheurs précédents. Pas sûre que j’aurais osé m’engager au milieu de tout ça !!!! Merci pour la cache.\end{cacheText}

\cacheNumber{1346}\needspace{5\baselineskip}\cacheName{\href{http://coord.info/GC20WFA}{Le Moulin CAPLA} — \href{http://coord.info/GC20WFA\Number{}762006033}{1346}}\cacheData{{2018/04/22 TTS et Betty P, Traditional Cache (1.5/2)}}\begin{cacheText}Arrivée dans la zone, le GPS me fait tourner en rond mais au vu des commentaires précédents  je trouve rapidement le PZ et ...la Belle. Tout est en place. Merci pour la découverte de ce lieu.\end{cacheText}

\cacheNumber{1347}\needspace{5\baselineskip}\cacheName{\href{http://coord.info/GC4ZF2R}{Moulins de Lézat: \Quoted{les 2 tours}} — \href{http://coord.info/GC4ZF2R\Number{}762014103}{1347}}\cacheData{{2018/04/22 DD\And{}Ludo, Traditional Cache (2/3)}}\begin{cacheText}A Lézat pour assister à un match de rugby des juniors de l'USSP, j’en profite pour faire quelques caches. Après cette dure ascension je suis récompensée:l'endroit est  juste sublime. Les Moulins sont magnifiques et la vue extraordinaire. Je ne comprenais pas trop l’indice qui a pris tout son sens en arrivant sur le spot. Merci pour cette cache.\end{cacheText}

\cacheNumber{1348}\needspace{5\baselineskip}\cacheName{\href{http://coord.info/GC54NTW}{AutoStop - A64 :Aire Du Pic Du Midi} — \href{http://coord.info/GC54NTW\Number{}762016764}{1348}}\cacheData{{2018/04/22 Evid3nce, Traditional Cache (2/1.5)}}\begin{cacheText}Sur le retour de l’Ariège, je m’arrête faire une petite pause. Super !!!!Une cache m’attend.Elle est découverte en deux temps trois mouvements grâce au spoiler. Merci pour la cache.\end{cacheText}

\cacheNumber{1349}\needspace{5\baselineskip}\cacheName{\href{http://coord.info/GC55WTA}{L'église de Lézat-sur-Lèze} — \href{http://coord.info/GC55WTA\Number{}762010095}{1349}}\cacheData{{2018/04/22 DD\And{}Ludo, Traditional Cache (1.5/1.5)}}\begin{cacheText}Sur Lézat pour assister un match de rugby des juniors de l'USSP  j’en profite pour faire quelques caches. Arrivée au PZ, pas plus de succès que mes  prédécesseurs. Je pense chercher au bon endroit : je laisse un logbook de maintenance pour la survie de la cache. Merci pour la découverte de cette superbe église qui est malheureusement envahie de mouches et de pigeons. Quel dommage ! Merci pour la cache.\end{cacheText}

\cacheNumber{1350}\needspace{5\baselineskip}\cacheName{\href{http://coord.info/GC5YRTX}{Chemin des Soupirs... La Paroisse vous Observe...} — \href{http://coord.info/GC5YRTX\Number{}762016252}{1350}}\cacheData{{2018/04/22 Petischtroumpf, Traditional Cache (1.5/1.5)}}\begin{cacheText}En Ariège pour assister un match de rugby j’en profite pour faire quelques caches sur la route. Je découvre ici un endroit très tranquille et très paisible avec une superbe vue sur l’église. L’indice est parlant : la cache est vite trouvée mais malheureusement elle est renversée ,et sans bouchon, dans l'écorce. J’essaie tant bien que mal de récupérer le bouchon pour fermer la Belle mais les divers objets sont coincés dans l'écorces et je n’ai rien pour les attraper. Merci de prendre soin des boites!!!! Merci Petischtroumpf pour la cache.\end{cacheText}

\cacheNumber{1351}\needspace{5\baselineskip}\cacheName{\href{http://coord.info/GC697T1}{GALETS DE LA GARONNE} — \href{http://coord.info/GC697T1\Number{}762003414}{1351}}\cacheData{{2018/04/22 LouZoeCamNic, Earthcache (1.5/1)}}\begin{cacheText}De passage sur Cazeres j’en profite pour faire cette petite Earthcache. Je découvre un superbe bord de Garonne très paisible.. Merci pour toutes les informations.\end{cacheText}

\cacheNumber{1352}\needspace{5\baselineskip}\cacheName{\href{http://coord.info/GC6KWAK}{A64 - Aire des Pyrénées} — \href{http://coord.info/GC6KWAK\Number{}761754210}{1352}}\cacheData{{2018/04/22 l.a.f.l.e.u.r, Traditional Cache (1.5/1.5)}}\begin{cacheText}En route pour assister à un match dans l’Ariège, une petite pause s’impose sur cette aire d’autoroute. Je découvre ici une magnifique sculpture et diverses explications sur le tour de France. La belle nous attend bien à sa place. Merci pour la cache\end{cacheText}

\cacheNumber{1353}\needspace{5\baselineskip}\cacheName{\href{http://coord.info/GC6KWAV}{A64 - Aire de Serres-Morlaas} — \href{http://coord.info/GC6KWAV\Number{}762018189}{1353}}\cacheData{{2018/04/22 l.a.f.l.e.u.r, Traditional Cache (1.5/1.5)}}\begin{cacheText}C’est sur le retour de l’Ariège que je m'arrête faire une petite pause. Autant joindre l’utile à l’agréable !!!! Le spoiler est bien précis et la Belle ,envahie de fourmis, est vite délogée. Merci pour la cache .\end{cacheText}

\cacheNumber{1354}\needspace{5\baselineskip}\cacheName{\href{http://coord.info/GC71JY5}{H-G Géocaching \Number{}135 : CAZERES} — \href{http://coord.info/GC71JY5\Number{}761755072}{1354}}\cacheData{{2018/04/22 LouZoeCamNic, Traditional Cache (1/1.5)}}\begin{cacheText}De passage en Haute-Garonne, j’en profite pour faire quelques caches. Celle-ci est rapidement trouvée .Je découvre un très joli Oratoire. Merci pour la découverte de ce lieu.\end{cacheText}

\cacheNumber{1355}\needspace{5\baselineskip}\cacheName{\href{http://coord.info/GC760BR}{H-G Géocaching \Number{}153 : COULADERE} — \href{http://coord.info/GC760BR\Number{}761758662}{1355}}\cacheData{{2018/04/22 A, Multi-cache (1.5/1.5)}}\begin{cacheText}De passage dans la région, j’en profite pour faire quelques caches. Je découvre un charmant petit village et une super Lague ou croassent de nombreuses grenouilles au milieu des nénuphars qui flottent. Toutes les informations sont collectées sur le panneau. Les calculs rapidement fait... le plus dur reste à faire!!! Le GPS m’amène dans une zone non entretenue mais à force de recherches la Belle se découvre. Merci pour cette belle cache.\end{cacheText}

\cacheNumber{1356}\needspace{5\baselineskip}\cacheName{\href{http://coord.info/GC32HQW}{Ombrières photovoltaïques} — \href{http://coord.info/GC32HQW\Number{}764407672}{1356}}\cacheData{{2018/04/30 Lluís64, Unknown Cache (1.5/1.5)}}\begin{cacheText}C'est lors de l'Event Pinpin84 au Pays d'Henri IV que je découvre la Belle. Elle est bien camouflée mais les 6 yeux présents la délogent en deux temps trois mouvements. Merci pour la cache.\end{cacheText}

\cacheNumber{1357}\needspace{5\baselineskip}\cacheName{\href{http://coord.info/GC6ZJ20}{I ❤️ GC !} — \href{http://coord.info/GC6ZJ20\Number{}764413737}{1357}}\cacheData{{2018/04/30 l.a.f.l.e.u.r, Unknown Cache (2.5/1.5)}}\begin{cacheText}Apres l'Event Pinpin84 au Pays d'Henri IV nous nous retrouvons autour de la table du restau chinois .A la fin du repas, nous décryptons l'énigme avec l'aide de Julie .Nous partons, en délégation ,avec l'owner vers le GC. La team Chinois ne met pas longtemps pour trouver la Belle. Merci les Lafleur pour la cache et l'énigme super sympa.\end{cacheText}

\cacheNumber{1358}\needspace{5\baselineskip}\cacheName{\href{http://coord.info/GC7A8F0}{Joyeux anniversaire Mme \Quoted{L.A.F.L.E.U.R} ;)))} — \href{http://coord.info/GC7A8F0\Number{}764405676}{1358}}\cacheData{{2018/04/30 jb65vic, Unknown Cache (2/1.5)}}\begin{cacheText}C'est en compagnie des Dune33 que je rejoins l'Event Pinpin84 au Pays d'Henri IV et c'est sur le chemin que nous découvrons la Belle. Merci JB pour la cache et pour cette super idée.\end{cacheText}

\cacheNumber{1359}\needspace{5\baselineskip}\cacheName{\href{http://coord.info/GC7KCJC}{01 - En attendant le printemps ...} — \href{http://coord.info/GC7KCJC\Number{}764410536}{1359}}\cacheData{{2018/04/30 l.a.f.l.e.u.r, Traditional Cache (1.5/1.5)}}\begin{cacheText}Apres l'Event Pinpin84 au Pays d'Henri IV et le super restau chinois nous partons en délégation avec l'owner faire quelques caches. La team Chinois ne met pas longtemps pour trouver la Belle. Nous découvrons une super jolie cache bien nichée dans le noir. Merci les Lafleur pour la cache.\end{cacheText}

\cacheNumber{1360}\needspace{5\baselineskip}\cacheName{\href{http://coord.info/GC7KCQW}{03 - En attendant le printemps} — \href{http://coord.info/GC7KCQW\Number{}764411708}{1360}}\cacheData{{2018/04/30 l.a.f.l.e.u.r, Traditional Cache (1.5/1.5)}}\begin{cacheText}Apres l'Event Pinpin84 au Pays d'Henri IV et le super restau chinois nous partons en délégation avec l'owner faire quelques caches. La team Chinois ne met pas longtemps pour trouver la Belle. Nous découvrons une super jolie cache bien nichée dans le noir. Merci les Lafleur pour la cache.\end{cacheText}

\cacheNumber{1361}\needspace{5\baselineskip}\cacheName{\href{http://coord.info/GC7N4BD}{Pinpin84 au Pays d'Henri IV} — \href{http://coord.info/GC7N4BD\Number{}764118997}{1361}}\cacheData{{2018/05/01 pinpin84, Event Cache (1/1)}}\begin{cacheText}Un grand merci les pinpin84 pour l'organisation de ce sympathique évent. De l'apéro en passant par le restau chinois et pour finir la recherche des caches, tout était parfait. Il n'y a plus qu'à remettre ça autour d'un atelier mysterie....\end{cacheText}

\cacheNumber{1362}\needspace{5\baselineskip}\cacheName{\href{http://coord.info/GC7P171}{le cheval de fer} — \href{http://coord.info/GC7P171\Number{}764421683}{1362}}\cacheData{{2018/05/02 lauki3940 \And{} titiger39, Traditional Cache (2.5/3.5)}}\begin{cacheText}[{STF}]

Il n’y a pas grand monde à cet heure-ci qui circule. J’ai de la chance et je peux chercher tranquillement la Belle. Le GPS et le spoiler me mènent tout droit à la cache .Très bonne idée. Merci Lauki et titiger pour la cache.

:)1366:)\end{cacheText}

\cacheNumber{1363}\needspace{5\baselineskip}\cacheName{\href{http://coord.info/GC7P19F}{Pont suspendu pyrenéenne A64} — \href{http://coord.info/GC7P19F\Number{}764419634}{1363}}\cacheData{{2018/05/02 lauki3940, Traditional Cache (2/2)}}\begin{cacheText}[{STF}]

Arrivée au PZ le GPS ne semble pas très précis. Mais grâce aux indices je parviens à dénicher la belle. Merci lauki pour la cache



:)1365:)\end{cacheText}

\cacheNumber{1364}\needspace{5\baselineskip}\cacheName{\href{http://coord.info/GC7P1AZ}{CHATEAU  DE GUICHE} — \href{http://coord.info/GC7P1AZ\Number{}764418678}{1364}}\cacheData{{2018/05/02 lauki3940, Traditional Cache (3/3)}}\begin{cacheText}[{STF}]

Je continue ma quête… Mais les redoutables chasseurs de FTF sont passés par là. La cache est bien à l'abris et il faut monter,monter. Je découvre de près ce magnifique château des ducs de Gramont. Merci pour la cache



:)1364:)\end{cacheText}

\cacheNumber{1365}\needspace{5\baselineskip}\cacheName{\href{http://coord.info/GC7P1BT}{RUINE DE GUICHE} — \href{http://coord.info/GC7P1BT\Number{}764416371}{1365}}\cacheData{{2018/05/02 lauki3940, Traditional Cache (3/1.5)}}\begin{cacheText}{[FTF]}

Ici je découvre une très jolie ruine. Mais un DNF est annoncé : les gamboy qui sont bien habitués n'ont pas trouvé !!!! J’ai bien peur de rentrer bredouille. Mais le GPS me guide et grâce à l’indice bingo je découvre la belle. Merci lauki pour la cache et la découverte de cette ruine.



:)1363:)\end{cacheText}

\cacheNumber{1366}\needspace{5\baselineskip}\cacheName{\href{http://coord.info/GC7P1CP}{LAVOIR DE GUICHE} — \href{http://coord.info/GC7P1CP\Number{}764415604}{1366}}\cacheData{{2018/05/02 lauki3940, Traditional Cache (2/2)}}\begin{cacheText}(FTF)

Ce matin en consultant mes mails je m’aperçois que de nouvelles caches sortent  à quelques dizaines de kilomètres de la maison. Mais il faut partir travailler!!!!Je surveille toute la journée les mouvements de la cache. Toujours rien !!!Sortie du boulot direction le lavoir. De jeunes moldus s'affairent au lavoir mais ils ne me prêtent pas attention .Je file au PZ et découvre vite fait bien fait la Belle. Suspense... super je suis FTF. Merci lauki3940 pour cette  cache.



:)1362:)\end{cacheText}

\cacheNumber{1367}\needspace{5\baselineskip}\cacheName{\href{http://coord.info/GC2WBWG}{A la Croisée des Chemins} — \href{http://coord.info/GC2WBWG\Number{}765562800}{1367}}\cacheData{{2018/05/04 kik-64, Multi-cache (2/2)}}\begin{cacheText}Le rendez-vous est pris depuis \Quoted{l’Event Pinpin au pays d’Henri 4} avec Dune33,DorisBear et la fleur pour aller faire quelques caches dans la forêt domaniale du Bastard.

Apres avoir relevés les nombreux indices ( et en avoir appris un peu plus sur les arbres et les chemins)nous finissons par mettre la main sur la Belle.Merci pour la cache.\end{cacheText}

\cacheNumber{1368}\needspace{5\baselineskip}\cacheName{\href{http://coord.info/GC2Y9RH}{Pont et Passerelle} — \href{http://coord.info/GC2Y9RH\Number{}765557267}{1368}}\cacheData{{2018/05/04 nathalie64, Traditional Cache (1.5/1.5)}}\begin{cacheText}Le rendez-vous est pris depuis \Quoted{l’Event Pinpin au pays d’Henri 4} avec Dune33,DorisBear et la fleur pour aller faire quelques caches dans la forêt domaniale du Bastard.

Nous passons pas mal de temps à chercher la Belle: il y a pas mal de passage et ce nest pas tres facile.Nous ne trouvons rien et laissons un logbook de remplacement.

Merci pour la cache.\end{cacheText}

\cacheNumber{1369}\needspace{5\baselineskip}\cacheName{\href{http://coord.info/GC44161}{Event SdG GR653 M\Underscore{}L-22} — \href{http://coord.info/GC44161\Number{}765563702}{1369}}\cacheData{{2018/05/04 Charnègues, Traditional Cache (1.5/1.5)}}\begin{cacheText}Le rendez-vous est pris depuis \Quoted{l’Event Pinpin au pays d’Henri 4} avec Dune33,DorisBear et la fleur pour aller faire quelques caches dans la forêt domaniale du Bastard.

La cache bien dissimulée est vite découverte par les 8 yeux experts.Merci pour la cache.\end{cacheText}

\cacheNumber{1370}\needspace{5\baselineskip}\cacheName{\href{http://coord.info/GC6MEDH}{L'écureuil de Bastard} — \href{http://coord.info/GC6MEDH\Number{}765523437}{1370}}\cacheData{{2018/05/04 l.a.f.l.e.u.r, Unknown Cache (2/4)}}\begin{cacheText}Le rendez-vous est pris depuis \Quoted{l’Event  Pinpin au pays d’Henri 4} avec Dune33,DorisBear et la fleur pour aller faire quelques caches dans la forêt domaniale du Bastard.

Arrivés au PZ l'ensemble de l'équipe décide que je suis la plus \Quoted{sportive} pour monter dire bonjour à Scrat.Avec l'aide de Dédé et de mde lafleur et encouragée par Dune33 et mde Dorisbear je réussis à me hisser et à attraper la noisette!!!!Le plus dur restait à faire!!!!!Avec moult précautions je redescend et finis dans les bras de Dédé.

Merci les Lafleur pour cette énigme et cette super cache. Un PF pour récompense.\end{cacheText}

\cacheNumber{1371}\needspace{5\baselineskip}\cacheName{\href{http://coord.info/GC7KCJW}{02 - En attendant le printemps} — \href{http://coord.info/GC7KCJW\Number{}765506119}{1371}}\cacheData{{2018/05/04 l.a.f.l.e.u.r, Traditional Cache (1.5/1.5)}}\begin{cacheText}Le rendez-vous est pris depuis \Quoted{l’Event  Pinpin au pays d’Henri 4} avec Dune33,DorisBear et la fleur pour aller faire quelques caches dans la forêt domaniale du Bastard.

Aussitôt descendues de 🚗 ,les féminines de l’étape se mettent en quête de la Belle pendant que le gentil Dédé prépare le café et le gâteau 🍰 .Les yeux avertis de DorisBear repèrent rapidement  la super cache juste avant l'arrivée de madame Lafleur que nous félicitons .\end{cacheText}

\cacheNumber{1372}\needspace{5\baselineskip}\cacheName{\href{http://coord.info/GC7KCRB}{04 - En attendant le printemps} — \href{http://coord.info/GC7KCRB\Number{}765538558}{1372}}\cacheData{{2018/05/04 l.a.f.l.e.u.r, Traditional Cache (1.5/1.5)}}\begin{cacheText}Le rendez-vous est pris depuis \Quoted{l’Event  Pinpin au pays d’Henri 4} avec Dune33,DorisBear et la fleur pour aller faire quelques caches dans la forêt domaniale du Bastard.

Le GPS nous mène au pied de la Belle.Encore une super découverte.Merci pour la cache.\end{cacheText}

\cacheNumber{1373}\needspace{5\baselineskip}\cacheName{\href{http://coord.info/GC7KCRQ}{05 - En attendant le printemps} — \href{http://coord.info/GC7KCRQ\Number{}765532357}{1373}}\cacheData{{2018/05/04 l.a.f.l.e.u.r, Traditional Cache (1.5/1.5)}}\begin{cacheText}Le rendez-vous est pris depuis \Quoted{l’Event  Pinpin au pays d’Henri 4} avec Dune33,DorisBear et la fleur pour aller faire quelques caches dans la forêt domaniale du Bastard.

Guidés par mde Lafleur nous arrivons au PZ. Le décor a bien changé mais nous repérons la Belle sans trop de difficultés.

Merci pour la cache.\end{cacheText}

\cacheNumber{1374}\needspace{5\baselineskip}\cacheName{\href{http://coord.info/GC7KCT2}{06 - En attendant le printemps} — \href{http://coord.info/GC7KCT2\Number{}765526518}{1374}}\cacheData{{2018/05/04 l.a.f.l.e.u.r, Traditional Cache (1.5/1.5)}}\begin{cacheText}Le rendez-vous est pris depuis \Quoted{l’Event  Pinpin au pays d’Henri 4} avec Dune33,DorisBear et la fleur pour aller faire quelques caches dans la forêt domaniale du Bastard.

Les géoblablas vont bon train jusqu'à manquer le PZ!!!Retour sur nos pas et nous découvrons sous l'oeil amusé de mde lafleur la trés jolie cache.

Merci pour la cache\end{cacheText}

\cacheNumber{1375}\needspace{5\baselineskip}\cacheName{\href{http://coord.info/GC7M084}{07 - En attendant le printemps} — \href{http://coord.info/GC7M084\Number{}765551419}{1375}}\cacheData{{2018/05/04 l.a.f.l.e.u.r, Traditional Cache (1.5/1.5)}}\begin{cacheText}Le rendez-vous est pris depuis \Quoted{l’Event Pinpin au pays d’Henri 4} avec Dune33,DorisBear et la fleur pour aller faire quelques caches dans la forêt domaniale du Bastard.

la Belle est toujours en place...elle est tres bien dissimulées.Merci pour la cache.\end{cacheText}

\cacheNumber{1376}\needspace{5\baselineskip}\cacheName{\href{http://coord.info/GC7M08T}{09 - En attendant le printemps} — \href{http://coord.info/GC7M08T\Number{}765553669}{1376}}\cacheData{{2018/05/04 l.a.f.l.e.u.r, Traditional Cache (1.5/1.5)}}\begin{cacheText}Le rendez-vous est pris depuis \Quoted{l’Event Pinpin au pays d’Henri 4} avec Dune33,DorisBear et la fleur pour aller faire quelques caches dans la forêt domaniale du Bastard.

Mde lafleur nous quitte pour vaquer à ses occupations et nous continuons à découvrir les caches qui ont été dissilulées dans le bois. Grace au GPS et au spoiler nous trouvons le joli \Quoted{citron}(n'est ce pas Denise).

Merci pour la cache.\end{cacheText}

\cacheNumber{1377}\needspace{5\baselineskip}\cacheName{\href{http://coord.info/GC7M092}{08 - En attendant le printemps} — \href{http://coord.info/GC7M092\Number{}765546053}{1377}}\cacheData{{2018/05/04 l.a.f.l.e.u.r, Multi-cache (2.5/1.5)}}\begin{cacheText}Le rendez-vous est pris depuis \Quoted{l’Event  Pinpin au pays d’Henri 4} avec Dune33,DorisBear et la fleur pour aller faire quelques caches dans la forêt domaniale du Bastard.

Arrivés au PZ, Dédé met la main sur l'indice (il faut bien regarder!!!) et nous filons vers le deuxième waypoint.Le second indice en poche nous nous lancons dans le calcul final et arrivons à dénicher la Belle.

Merci les lafleur pour tout ce travail.\end{cacheText}

\cacheNumber{1378}\needspace{5\baselineskip}\cacheName{\href{http://coord.info/GC4VRRJ}{A63 - Aire de repos Lugos Est} — \href{http://coord.info/GC4VRRJ\Number{}765916188}{1378}}\cacheData{{2018/05/08 gilles64, Traditional Cache (1.5/1.5)}}\begin{cacheText}En partance pour le GTAQ 4 une petite pause s’impose. Cela tombe bien une petite cache nous attend. Pas de problème pour la dénicher :l’indice nous dit tout . Merci Gilles pour la cache.\end{cacheText}

\cacheNumber{1379}\needspace{5\baselineskip}\cacheName{\href{http://coord.info/GC3B81C}{📷 Bordeaux touristique - La cathédrale} — \href{http://coord.info/GC3B81C\Number{}766328676}{1379}}\cacheData{{2018/05/08 Calimero33 \And{} divers33, Traditional Cache (2.5/1)}}\begin{cacheText}Apres le pique nique de l'Event ,la troupe s'élance dans la visite de la très belle ville de Bordeaux qui n'a pas grand secret pour notre guide et son assistant. La cache est rouvée en groupe pendant la balade et est loguée sous le pseudo Team Bordeaux Event ...

Merci\end{cacheText}

\cacheNumber{1380}\needspace{5\baselineskip}\cacheName{\href{http://coord.info/GC3J1QM}{📷 Bordeaux touristique - Porte Cailhau} — \href{http://coord.info/GC3J1QM\Number{}766327923}{1380}}\cacheData{{2018/05/08 Calimero33, Traditional Cache (2.5/1)}}\begin{cacheText}Apres le pique nique de l'Event ,la troupe s'élance dans la visite de la très belle ville de Bordeaux qui n'a pas grand secret pour notre guide et son assistant. La cache est rouvée en groupe pendant la balade et est loguée sous le pseudo Team Bordeaux Event ...

Merci\end{cacheText}

\cacheNumber{1381}\needspace{5\baselineskip}\cacheName{\href{http://coord.info/GC3JQYE}{📷 Bordeaux touristique - La grosse Cloche} — \href{http://coord.info/GC3JQYE\Number{}766328323}{1381}}\cacheData{{2018/05/08 Calimero33, Traditional Cache (2/1)}}\begin{cacheText}Apres le pique nique de l'Event ,la troupe s'élance dans la visite de la très belle ville de Bordeaux qui n'a pas grand secret pour notre guide et son assistant. La cache est rouvée en groupe pendant la balade et est loguée sous le pseudo Team Bordeaux Event ...

Merci\end{cacheText}

\cacheNumber{1382}\needspace{5\baselineskip}\cacheName{\href{http://coord.info/GC3M339}{Jardin botanique du Jardin Public} — \href{http://coord.info/GC3M339\Number{}766354172}{1382}}\cacheData{{2018/05/08 Calimero33, Traditional Cache (2/1)}}\begin{cacheText}Apres le pique nique de l'Event ,la troupe s'élance dans la visite de la très belle ville de Bordeaux qui n'a pas grand secret pour notre guide et son assistant. La cache est rouvée en groupe pendant la balade et est loguée sous le pseudo Team Bordeaux Event ...

Merci\end{cacheText}

\cacheNumber{1383}\needspace{5\baselineskip}\cacheName{\href{http://coord.info/GC3MF75}{📷 Bordeaux touristique - Pont de pierre} — \href{http://coord.info/GC3MF75\Number{}766327062}{1383}}\cacheData{{2018/05/08 Calimero33, Traditional Cache (1.5/1.5)}}\begin{cacheText}Apres le pique nique de l'Event ,la troupe s'élance dans la visite de la très belle ville de Bordeaux qui n'a pas grand secret pour notre guide et son assistant. La cache est rouvée en groupe pendant la balade et est loguée sous le pseudo Team Bordeaux Event ...

Merci\end{cacheText}

\cacheNumber{1384}\needspace{5\baselineskip}\cacheName{\href{http://coord.info/GC3MPB4}{Event Bordelais - 1 - Gare d'Orléans} — \href{http://coord.info/GC3MPB4\Number{}766263616}{1384}}\cacheData{{2018/05/08 Calimero33, Traditional Cache (2/1.5)}}\begin{cacheText}De passage pour nous rendre à l'Event le GTAQ en balade dans la capitale nous nous arrêtons faire quelques caches. Beaucoup de moldu en ce mardi matin!!! Pas facile d’être discrète pour chercher mais je finis par mettre la main sur la Belle. Merci pour cette cache et cette superbe vue.\end{cacheText}

\cacheNumber{1385}\needspace{5\baselineskip}\cacheName{\href{http://coord.info/GC3MPC4}{Event Bordelais - 2 - Face à la Bourse} — \href{http://coord.info/GC3MPC4\Number{}766263939}{1385}}\cacheData{{2018/05/08 Calimero33, Traditional Cache (1.5/1.5)}}\begin{cacheText}Nous continuons notre quête mais une petite pause s’impose. Pas de difficulté. Le log book est archiplein cela nécessite une maintenance. Merci pour la cache.\end{cacheText}

\cacheNumber{1386}\needspace{5\baselineskip}\cacheName{\href{http://coord.info/GC3MPE8}{Event Bordelais - 3 - Toussaint Louverture} — \href{http://coord.info/GC3MPE8\Number{}766266400}{1386}}\cacheData{{2018/05/08 Calimero33, Traditional Cache (2.5/1.5)}}\begin{cacheText}Pendant l' Event le GTAQ en balade dans la capitale ,nous nous échappons pour loguer la Belle qui est à proximité. Merci pour la cache.\end{cacheText}

\cacheNumber{1387}\needspace{5\baselineskip}\cacheName{\href{http://coord.info/GC6ATT0}{[33 TOUR] - SAINT PEY D'ARMENS} — \href{http://coord.info/GC6ATT0\Number{}766354326}{1387}}\cacheData{{2018/05/08 Grim's, Traditional Cache (1.5/1)}}\begin{cacheText}Sur le retour de l'Event le GTAQ en balade dans la capitale nous nous arrêtons faire cette petite cache sur la route du gîte. La cache est trouvée en deux temps trois mouvements .Effectivement pas besoin d'entrer dans le cimetiere. Merci pour la cache\end{cacheText}

\cacheNumber{1388}\needspace{5\baselineskip}\cacheName{\href{http://coord.info/GC6T7HZ}{TTD\Number{}3 Le miroir d'eau} — \href{http://coord.info/GC6T7HZ\Number{}766354221}{1388}}\cacheData{{2018/05/08 athena033 \And{} Mrmaria, Traditional Cache (1.5/1)}}\begin{cacheText}Apres le pique nique de l'Event ,la troupe s'élance dans la visite de la très belle ville de Bordeaux qui n'a pas grand secret pour notre guide et son assistant. La cache est rouvée en groupe pendant la balade et est loguée sous le pseudo Team Bordeaux Event ...

Merci\end{cacheText}

\cacheNumber{1389}\needspace{5\baselineskip}\cacheName{\href{http://coord.info/GC6VX9V}{De Niel à Darwin} — \href{http://coord.info/GC6VX9V\Number{}766264628}{1389}}\cacheData{{2018/05/08 MariVaLou, Traditional Cache (2/1)}}\begin{cacheText}Nous cherchons une place pour nous stationner pour l'Event le GTAQ en balade dans la capitale et nous arretons pile poil devant la Belle qui est parfaitement intégrée au décor. Merci pour la découverte du site et pour cette cache ingénieuse.\end{cacheText}

\cacheNumber{1390}\needspace{5\baselineskip}\cacheName{\href{http://coord.info/GC6WF81}{⚒ Bordeaux 2030 - 05 - Palais des Sports} — \href{http://coord.info/GC6WF81\Number{}766328448}{1390}}\cacheData{{2018/05/08 Calimero33, Traditional Cache (1.5/1.5)}}\begin{cacheText}Apres le pique nique de l'Event ,la troupe s'élance dans la visite de la très belle ville de Bordeaux qui n'a pas grand secret pour notre guide et son assistant. La cache est rouvée en groupe pendant la balade et est loguée sous le pseudo Team Bordeaux Event ...

Merci\end{cacheText}

\cacheNumber{1391}\needspace{5\baselineskip}\cacheName{\href{http://coord.info/GC6WMRY}{🔎 Bordeaux insolite - La vielle dame} — \href{http://coord.info/GC6WMRY\Number{}766328147}{1391}}\cacheData{{2018/05/08 Pleins de geocacheurs Girondins, Traditional Cache (1.5/1.5)}}\begin{cacheText}Apres le pique nique de l'Event ,la troupe s'élance dans la visite de la très belle ville de Bordeaux qui n'a pas grand secret pour notre guide et son assistant. La cache est rouvée en groupe pendant la balade et est loguée sous le pseudo Team Bordeaux Event ...

Merci\end{cacheText}

\cacheNumber{1392}\needspace{5\baselineskip}\cacheName{\href{http://coord.info/GC7BA67}{LE CHATEAU TROMPETTE - Virtual Reward 2017/2018} — \href{http://coord.info/GC7BA67\Number{}767455427}{1392}}\cacheData{{2018/05/08 GTAQ, Virtual Cache (1.5/1)}}\begin{cacheText}C'est lors de la visite guidée proposée lors de l'Event le GTAQ en balade dans la capitale que nous validons la virtuelle. Superbe place envahie de moldus !!!Merci l'Auguste. Un PF pour le château trompette .\end{cacheText}

\cacheNumber{1393}\needspace{5\baselineskip}\cacheName{\href{http://coord.info/GC7BBE2}{💖 02 - Puzzle de Bordeaux} — \href{http://coord.info/GC7BBE2\Number{}766353924}{1393}}\cacheData{{2018/05/08 Calimero33, Unknown Cache (4/1)}}\begin{cacheText}Les puzzle de Bordeaux sont trouvés depuis un moment au cas ou...C'est avec la Team Bordeaux que nous loguons la Belle lors de la balade dans la capitale. Merci pour la cache.\end{cacheText}

\cacheNumber{1394}\needspace{5\baselineskip}\cacheName{\href{http://coord.info/GC7BBEN}{💖 01 - Puzzle de Bordeaux} — \href{http://coord.info/GC7BBEN\Number{}766353982}{1394}}\cacheData{{2018/05/08 Calimero33, Unknown Cache (4/1)}}\begin{cacheText}Les puzzle de Bordeaux sont trouvés depuis un moment au cas ou...C'est avec la Team Bordeaux que nous loguons la Belle lors de la balade dans la capitale. Merci pour la cache.\end{cacheText}

\cacheNumber{1395}\needspace{5\baselineskip}\cacheName{\href{http://coord.info/GC7M27B}{Bord’Eaux \Number{}0 : La caserne de la Benauge} — \href{http://coord.info/GC7M27B\Number{}766263226}{1395}}\cacheData{{2018/05/08 GAVbrioche33, Traditional Cache (1.5/1.5)}}\begin{cacheText}De passage sur Bordeaux pour participer à l’Event le GTAQ en balade dans la capitale, nous en profitons pour faire quelques cacheS. Nous découvrons celle-ci sans difficulté grâce a l’indice.Merci pour la cache.\end{cacheText}

\cacheNumber{1396}\needspace{5\baselineskip}\cacheName{\href{http://coord.info/GC7M29A}{Bord’Eaux \Number{}2 : Le Parc aux Angéliques} — \href{http://coord.info/GC7M29A\Number{}766262727}{1396}}\cacheData{{2018/05/08 GAVbrioche33, Traditional Cache (1.5/1.5)}}\begin{cacheText}Nous continuons notre promenade dans ce très joli jardin et découvrons la belle au pied du romarin. Merci pour la cache.\end{cacheText}

\cacheNumber{1397}\needspace{5\baselineskip}\cacheName{\href{http://coord.info/GC7M3ZX}{Bord’Eaux \Number{}3 : Le Pont Saint Jean} — \href{http://coord.info/GC7M3ZX\Number{}766262011}{1397}}\cacheData{{2018/05/08 GAVbrioche33, Traditional Cache (1.5/1.5)}}\begin{cacheText}En partance pour l'Event GTAQ en balade dans la capitale nous nous arrêtons faire cette cache en passant .Malgré un DNF annoncé, nous tentons notre chance. Bien nous appris puisque nous mettons la main sur la Belle qui est à terre. Nous la remettons en place. Merci pour la découverte de ce très joli jardin.\end{cacheText}

\cacheNumber{1398}\needspace{5\baselineskip}\cacheName{\href{http://coord.info/GC7NHM9}{LE GTAQ EN BALADE DANS LA CAPITALE AQUITAINE} — \href{http://coord.info/GC7NHM9\Number{}767465255}{1398}}\cacheData{{2018/05/08 GTAQ, Event Cache (1/1.5)}}\begin{cacheText}C'est le premier Event d'une longue série pour notre plus grande joie. Nous sommes heureux de retrouver les amis et de faire de nouvelles connaissances. Tout est parfait: l'apéro ,le pique nique, la visite guidée ponctuée de nombreuses caches, les cannelés et pour finir la virtuelle!!!!Un grand merci pour cet excellent moment passé ensemble.\end{cacheText}

\cacheNumber{1399}\needspace{5\baselineskip}\cacheName{\href{http://coord.info/GC5N7V3}{[33 TOUR] - RAUZAN} — \href{http://coord.info/GC5N7V3\Number{}767507304}{1399}}\cacheData{{2018/05/09 Grim's, Traditional Cache (1.5/1)}}\begin{cacheText}Au retour de la série le vignoble de domino50, nous arrêtons avec Fabilab pour faire quelques caches supplémentaires. Arrivés au PZ ,les indices ne correspondent pas du tout avec les cordonnées. Nous élargissons les recherches et finissons par mettre la main sur la Belle.Nous loguons sous le nom Team 64 44.Merci pour la cache.\end{cacheText}

\cacheNumber{1400}\needspace{5\baselineskip}\cacheName{\href{http://coord.info/GC5NE40}{Eglise Saint Martin de Jugazan} — \href{http://coord.info/GC5NE40\Number{}767501661}{1400}}\cacheData{{2018/05/09 Grim's, Traditional Cache (3.5/1.5)}}\begin{cacheText}Au retour de la série le vignoble de domino50, nous arrêtons avec Fabilab pour faire quelques caches supplémentaires. Arrivés au PZ je pars direct à la cache je monte mais Stéphane qui veut à tout prix l’attraper se propose pour le faire. Je lui laisse la place.Nous loguons sous le nom Team 64 44.Merci pour la cache.\end{cacheText}

\cacheNumber{1401}\needspace{5\baselineskip}\cacheName{\href{http://coord.info/GC5NE68}{[33 TOUR] - NAUJAN ET POSTIAC} — \href{http://coord.info/GC5NE68\Number{}767495934}{1401}}\cacheData{{2018/05/09 Grim's, Traditional Cache (1.5/1.5)}}\begin{cacheText}C'est lors du circuit des Mystères du vignoble que nous découvrons la cache .Nous loguons sous le nom Team Bouteille .Merci pour la cache.\end{cacheText}

\cacheNumber{1402}\needspace{5\baselineskip}\cacheName{\href{http://coord.info/GC5RJKB}{Jugazan - Domaine de Peyrelongue} — \href{http://coord.info/GC5RJKB\Number{}767502540}{1402}}\cacheData{{2018/05/09 Grim's, Traditional Cache (1.5/1.5)}}\begin{cacheText}C'est lors du circuit des Mystères du vignoble que nous découvrons la cache .Sitôt arrivés sur le PZ : Pierre dégote la belle. Nous repartons. Merci pour la cache. Nous signons Team 6444\end{cacheText}

\cacheNumber{1403}\needspace{5\baselineskip}\cacheName{\href{http://coord.info/GC5ZADR}{Rauzan Tour alt\Number{}1 Le Camping Du Vieux Chateau} — \href{http://coord.info/GC5ZADR\Number{}767520802}{1403}}\cacheData{{2018/05/09 Grim's, Traditional Cache (1.5/2)}}\begin{cacheText}La Belle est trouvée en sortant du Meet \And{} Greet à Rauzan. C'est en compagnie de Fabilab ,de Dune33 que nous débutons les recherches. Jet41 nous rejoint lorsque nous loguons. La Team Night part se coucher dans la bonne humeur .Merci pour la cache.\end{cacheText}

\cacheNumber{1404}\needspace{5\baselineskip}\cacheName{\href{http://coord.info/GC651T9}{[33 TOUR] - SAINT AUBIN DE BRANNE} — \href{http://coord.info/GC651T9\Number{}766357022}{1404}}\cacheData{{2018/05/09 Grim's, Traditional Cache (1.5/1.5)}}\begin{cacheText}C'est lors du circuit des Mystères du vignoble que nous découvrons la cache .Nous loguons sous le nom Team Bouteille .Merci pour la cache.\end{cacheText}

\cacheNumber{1405}\needspace{5\baselineskip}\cacheName{\href{http://coord.info/GC6FD3C}{Rauzan Tour \Number{}20 La Grange} — \href{http://coord.info/GC6FD3C\Number{}767504483}{1405}}\cacheData{{2018/05/09 Grim's, Traditional Cache (1.5/2)}}\begin{cacheText}Au retour de la série le vignoble de domino50, nous arrêtons avec Fabilab pour faire quelques caches supplémentaires. Nous loguons sous le nom Team 64 44.Merci pour la cache.\end{cacheText}

\cacheNumber{1406}\needspace{5\baselineskip}\cacheName{\href{http://coord.info/GC6FD47}{Rauzan Tour \Number{}21 Tropagat} — \href{http://coord.info/GC6FD47\Number{}767504825}{1406}}\cacheData{{2018/05/09 Grim's, Traditional Cache (1.5/1.5)}}\begin{cacheText}Au retour de la série le vignoble de domino50, nous arrêtons avec Fabilab pour faire quelques caches supplémentaires. .Nous loguons sous le nom Team 64 44.Merci pour la cache.\end{cacheText}

\cacheNumber{1407}\needspace{5\baselineskip}\cacheName{\href{http://coord.info/GC7KGCK}{[GTAQ]01-Les Mystères du vignoble} — \href{http://coord.info/GC7KGCK\Number{}766355966}{1407}}\cacheData{{2018/05/09 domino50, Unknown Cache (2.5/2.5)}}\begin{cacheText}La bouteille m'a donné quelques maux de tête et a occupé quelques soirée. Aujourd'hui...C'est le grand jour!!!Une vingtaine de personnes prennent le départ pour former la Team Bouteille.  La balade est faite avec la présence de l'owner dans une super  ambiance et avec beaucoup de géoblabla. Toutes les caches sont trouvées (merci les enfants) sous l'œil bienveillant de Domino. La pause pique nique, très agréable nous a permis de prendre des forces pour attaquer les 2 montées!!!Un grand merci à domino 50 pour la qualité des caches et la découverte de ces jolis paysages(un PF pour l'ensemble du ''chai''d'oeuvre).\end{cacheText}

\cacheNumber{1408}\needspace{5\baselineskip}\cacheName{\href{http://coord.info/GC7KKYQ}{[GTAQ]02-Les Mystères du vignoble} — \href{http://coord.info/GC7KKYQ\Number{}766356049}{1408}}\cacheData{{2018/05/09 domino50, Unknown Cache (2.5/2.5)}}\begin{cacheText}La bouteille m'a donné quelques maux de tête et a occupé quelques soirée. Aujourd'hui...C'est le grand jour!!!Une vingtaine de personnes prennent le départ pour former la Team Bouteille. La balade est faite avec la présence de l'owner dans une super ambiance et avec beaucoup de géoblabla. Toutes les caches sont trouvées (merci les enfants) sous l'œil bienveillant de Domino. La pause pique nique, très agréable nous a permis de prendre des forces pour attaquer les 2 montées!!!Un grand merci à domino 50 pour la qualité des caches et la découverte de ces jolis paysages(un PF pour l'ensemble du ''chai''d'oeuvre).\end{cacheText}

\cacheNumber{1409}\needspace{5\baselineskip}\cacheName{\href{http://coord.info/GC7KM44}{[GTAQ]20-Les Mystères du vignoble} — \href{http://coord.info/GC7KM44\Number{}766362392}{1409}}\cacheData{{2018/05/09 domino50, Unknown Cache (2.5/2.5)}}\begin{cacheText}La bouteille m'a donné quelques maux de tête et a occupé quelques soirées. Aujourd'hui...C'est le grand jour!!!Une vingtaine de personnes prennent le départ pour former la Team Bouteille. La balade est faite avec la présence de l'owner dans une super ambiance et avec beaucoup de géoblabla. Toutes les caches sont trouvées (merci les enfants) sous l'œil bienveillant de Domino. La pause pique nique, très agréable nous a permis de prendre des forces pour attaquer les 2 montées!!!Un grand merci à domino 50 pour la qualité des caches et la découverte de ces jolis paysages(un PF pour l'ensemble du ''chai''d'oeuvre).\end{cacheText}

\cacheNumber{1410}\needspace{5\baselineskip}\cacheName{\href{http://coord.info/GC7KM4J}{[GTAQ]03-Les Mystères du vignoble} — \href{http://coord.info/GC7KM4J\Number{}766356512}{1410}}\cacheData{{2018/05/09 domino50, Unknown Cache (2.5/2.5)}}\begin{cacheText}La bouteille m'a donné quelques maux de tête et a occupé quelques soirée. Aujourd'hui...C'est le grand jour!!!Une vingtaine de personnes prennent le départ pour former la Team Bouteille. La balade est faite avec la présence de l'owner dans une super ambiance et avec beaucoup de géoblabla. Toutes les caches sont trouvées (merci les enfants) sous l'œil bienveillant de Domino. La pause pique nique, très agréable nous a permis de prendre des forces pour attaquer les 2 montées!!!Un grand merci à domino 50 pour la qualité des caches et la découverte de ces jolis paysages(un PF pour l'ensemble du ''chai''d'oeuvre).\end{cacheText}

\cacheNumber{1411}\needspace{5\baselineskip}\cacheName{\href{http://coord.info/GC7KM51}{[GTAQ]04-Les Mystères du vignoble} — \href{http://coord.info/GC7KM51\Number{}766356564}{1411}}\cacheData{{2018/05/09 domino50, Unknown Cache (3/2.5)}}\begin{cacheText}La bouteille m'a donné quelques maux de tête et a occupé quelques soirée. Aujourd'hui...C'est le grand jour!!!Une vingtaine de personnes prennent le départ pour former la Team Bouteille. La balade est faite avec la présence de l'owner dans une super ambiance et avec beaucoup de géoblabla. Toutes les caches sont trouvées (merci les enfants) sous l'œil bienveillant de Domino. La pause pique nique, très agréable nous a permis de prendre des forces pour attaquer les 2 montées!!!Un grand merci à domino 50 pour la qualité des caches et la découverte de ces jolis paysages(un PF pour l'ensemble du ''chai''d'oeuvre).\end{cacheText}

\cacheNumber{1412}\needspace{5\baselineskip}\cacheName{\href{http://coord.info/GC7KM5H}{[GTAQ]05-Les Mystères du vignoble} — \href{http://coord.info/GC7KM5H\Number{}766356605}{1412}}\cacheData{{2018/05/09 domino50, Unknown Cache (2.5/2.5)}}\begin{cacheText}La bouteille m'a donné quelques maux de tête et a occupé quelques soirée. Aujourd'hui...C'est le grand jour!!!Une vingtaine de personnes prennent le départ pour former la Team Bouteille. La balade est faite avec la présence de l'owner dans une super ambiance et avec beaucoup de géoblabla. Toutes les caches sont trouvées (merci les enfants) sous l'œil bienveillant de Domino. La pause pique nique, très agréable nous a permis de prendre des forces pour attaquer les 2 montées!!!Un grand merci à domino 50 pour la qualité des caches et la découverte de ces jolis paysages(un PF pour l'ensemble du ''chai''d'oeuvre).\end{cacheText}

\cacheNumber{1413}\needspace{5\baselineskip}\cacheName{\href{http://coord.info/GC7KNCV}{[GTAQ]06-Les Mystères du vignoble} — \href{http://coord.info/GC7KNCV\Number{}766356646}{1413}}\cacheData{{2018/05/09 domino50, Unknown Cache (2.5/2.5)}}\begin{cacheText}La bouteille m'a donné quelques maux de tête et a occupé quelques soirée. Aujourd'hui...C'est le grand jour!!!Une vingtaine de personnes prennent le départ pour former la Team Bouteille. La balade est faite avec la présence de l'owner dans une super ambiance et avec beaucoup de géoblabla. Toutes les caches sont trouvées (merci les enfants) sous l'œil bienveillant de Domino. La pause pique nique, très agréable nous a permis de prendre des forces pour attaquer les 2 montées!!!Un grand merci à domino 50 pour la qualité des caches et la découverte de ces jolis paysages(un PF pour l'ensemble du ''chai''d'oeuvre).\end{cacheText}

\cacheNumber{1414}\needspace{5\baselineskip}\cacheName{\href{http://coord.info/GC7KND8}{[GTAQ]07-Les Mystères du vignoble} — \href{http://coord.info/GC7KND8\Number{}766356687}{1414}}\cacheData{{2018/05/09 domino50, Unknown Cache (2.5/2.5)}}\begin{cacheText}La bouteille m'a donné quelques maux de tête et a occupé quelques soirée. Aujourd'hui...C'est le grand jour!!!Une vingtaine de personnes prennent le départ pour former la Team Bouteille. La balade est faite avec la présence de l'owner dans une super ambiance et avec beaucoup de géoblabla. Toutes les caches sont trouvées (merci les enfants) sous l'œil bienveillant de Domino. La pause pique nique, très agréable nous a permis de prendre des forces pour attaquer les 2 montées!!!Un grand merci à domino 50 pour la qualité des caches et la découverte de ces jolis paysages(un PF pour l'ensemble du ''chai''d'oeuvre).\end{cacheText}

\cacheNumber{1415}\needspace{5\baselineskip}\cacheName{\href{http://coord.info/GC7KNDG}{[GTAQ]08-Les Mystères du vignoble} — \href{http://coord.info/GC7KNDG\Number{}766356720}{1415}}\cacheData{{2018/05/09 domino50, Unknown Cache (2.5/2.5)}}\begin{cacheText}La bouteille m'a donné quelques maux de tête et a occupé quelques soirée. Aujourd'hui...C'est le grand jour!!!Une vingtaine de personnes prennent le départ pour former la Team Bouteille. La balade est faite avec la présence de l'owner dans une super ambiance et avec beaucoup de géoblabla. Toutes les caches sont trouvées (merci les enfants) sous l'œil bienveillant de Domino. La pause pique nique, très agréable nous a permis de prendre des forces pour attaquer les 2 montées!!!Un grand merci à domino 50 pour la qualité des caches et la découverte de ces jolis paysages(un PF pour l'ensemble du ''chai''d'oeuvre).\end{cacheText}

\cacheNumber{1416}\needspace{5\baselineskip}\cacheName{\href{http://coord.info/GC7KNDP}{[GTAQ]09-Les Mystères du vignoble} — \href{http://coord.info/GC7KNDP\Number{}766356756}{1416}}\cacheData{{2018/05/09 domino50, Unknown Cache (2.5/2.5)}}\begin{cacheText}La bouteille m'a donné quelques maux de tête et a occupé quelques soirée. Aujourd'hui...C'est le grand jour!!!Une vingtaine de personnes prennent le départ pour former la Team Bouteille. La balade est faite avec la présence de l'owner dans une super ambiance et avec beaucoup de géoblabla. Toutes les caches sont trouvées (merci les enfants) sous l'œil bienveillant de Domino. La pause pique nique, très agréable nous a permis de prendre des forces pour attaquer les 2 montées!!!Un grand merci à domino 50 pour la qualité des caches et la découverte de ces jolis paysages(un PF pour l'ensemble du ''chai''d'oeuvre).\end{cacheText}

\cacheNumber{1417}\needspace{5\baselineskip}\cacheName{\href{http://coord.info/GC7KNF2}{[GTAQ]10-Les Mystères du vignoble} — \href{http://coord.info/GC7KNF2\Number{}766356806}{1417}}\cacheData{{2018/05/09 domino50, Unknown Cache (2.5/2.5)}}\begin{cacheText}La bouteille m'a donné quelques maux de tête et a occupé quelques soirée. Aujourd'hui...C'est le grand jour!!!Une vingtaine de personnes prennent le départ pour former la Team Bouteille. La balade est faite avec la présence de l'owner dans une super ambiance et avec beaucoup de géoblabla. Toutes les caches sont trouvées (merci les enfants) sous l'œil bienveillant de Domino. La pause pique nique, très agréable nous a permis de prendre des forces pour attaquer les 2 montées!!!Un grand merci à domino 50 pour la qualité des caches et la découverte de ces jolis paysages(un PF pour l'ensemble du ''chai''d'oeuvre).\end{cacheText}

\cacheNumber{1418}\needspace{5\baselineskip}\cacheName{\href{http://coord.info/GC7KNG1}{[GTAQ]11-Les Mystères du vignoble} — \href{http://coord.info/GC7KNG1\Number{}766356868}{1418}}\cacheData{{2018/05/09 domino50, Unknown Cache (2.5/2.5)}}\begin{cacheText}La bouteille m'a donné quelques maux de tête et a occupé quelques soirée. Aujourd'hui...C'est le grand jour!!!Une vingtaine de personnes prennent le départ pour former la Team Bouteille. La balade est faite avec la présence de l'owner dans une super ambiance et avec beaucoup de géoblabla. Toutes les caches sont trouvées (merci les enfants) sous l'œil bienveillant de Domino. La pause pique nique, très agréable nous a permis de prendre des forces pour attaquer les 2 montées!!!Un grand merci à domino 50 pour la qualité des caches et la découverte de ces jolis paysages(un PF pour l'ensemble du ''chai''d'oeuvre).\end{cacheText}

\cacheNumber{1419}\needspace{5\baselineskip}\cacheName{\href{http://coord.info/GC7KPTE}{[GTAQ]12-Les Mystères du vignoble} — \href{http://coord.info/GC7KPTE\Number{}766356908}{1419}}\cacheData{{2018/05/09 domino50, Unknown Cache (2.5/2.5)}}\begin{cacheText}La bouteille m'a donné quelques maux de tête et a occupé quelques soirée. Aujourd'hui...C'est le grand jour!!!Une vingtaine de personnes prennent le départ pour former la Team Bouteille. La balade est faite avec la présence de l'owner dans une super ambiance et avec beaucoup de géoblabla. Toutes les caches sont trouvées (merci les enfants) sous l'œil bienveillant de Domino. La pause pique nique, très agréable nous a permis de prendre des forces pour attaquer les 2 montées!!!Un grand merci à domino 50 pour la qualité des caches et la découverte de ces jolis paysages(un PF pour l'ensemble du ''chai''d'oeuvre).\end{cacheText}

\cacheNumber{1420}\needspace{5\baselineskip}\cacheName{\href{http://coord.info/GC7KPTW}{[GTAQ]13-Les Mystères du vignoble} — \href{http://coord.info/GC7KPTW\Number{}766357048}{1420}}\cacheData{{2018/05/09 domino50, Unknown Cache (2.5/2.5)}}\begin{cacheText}La bouteille m'a donné quelques maux de tête et a occupé quelques soirée. Aujourd'hui...C'est le grand jour!!!Une vingtaine de personnes prennent le départ pour former la Team Bouteille. La balade est faite avec la présence de l'owner dans une super ambiance et avec beaucoup de géoblabla. Toutes les caches sont trouvées (merci les enfants) sous l'œil bienveillant de Domino. La pause pique nique, très agréable nous a permis de prendre des forces pour attaquer les 2 montées!!!Un grand merci à domino 50 pour la qualité des caches et la découverte de ces jolis paysages(un PF pour l'ensemble du ''chai''d'oeuvre).\end{cacheText}

\cacheNumber{1421}\needspace{5\baselineskip}\cacheName{\href{http://coord.info/GC7KPV7}{[GTAQ]14-Les Mystères du vignoble} — \href{http://coord.info/GC7KPV7\Number{}766357075}{1421}}\cacheData{{2018/05/09 domino50, Unknown Cache (2.5/2.5)}}\begin{cacheText}La bouteille m'a donné quelques maux de tête et a occupé quelques soirée. Aujourd'hui...C'est le grand jour!!!Une vingtaine de personnes prennent le départ pour former la Team Bouteille. La balade est faite avec la présence de l'owner dans une super ambiance et avec beaucoup de géoblabla. Toutes les caches sont trouvées (merci les enfants) sous l'œil bienveillant de Domino. La pause pique nique, très agréable nous a permis de prendre des forces pour attaquer les 2 montées!!!Un grand merci à domino 50 pour la qualité des caches et la découverte de ces jolis paysages(un PF pour l'ensemble du ''chai''d'oeuvre).\end{cacheText}

\cacheNumber{1422}\needspace{5\baselineskip}\cacheName{\href{http://coord.info/GC7KPVA}{[GTAQ]15-Les Mystères du vignoble} — \href{http://coord.info/GC7KPVA\Number{}766357109}{1422}}\cacheData{{2018/05/09 domino50, Unknown Cache (2.5/2.5)}}\begin{cacheText}La bouteille m'a donné quelques maux de tête et a occupé quelques soirée. Aujourd'hui...C'est le grand jour!!!Une vingtaine de personnes prennent le départ pour former la Team Bouteille. La balade est faite avec la présence de l'owner dans une super ambiance et avec beaucoup de géoblabla. Toutes les caches sont trouvées (merci les enfants) sous l'œil bienveillant de Domino. La pause pique nique, très agréable nous a permis de prendre des forces pour attaquer les 2 montées!!!Un grand merci à domino 50 pour la qualité des caches et la découverte de ces jolis paysages(un PF pour l'ensemble du ''chai''d'oeuvre).\end{cacheText}

\cacheNumber{1423}\needspace{5\baselineskip}\cacheName{\href{http://coord.info/GC7KPVC}{[GTAQ]16-Les Mystères du vignoble} — \href{http://coord.info/GC7KPVC\Number{}766357142}{1423}}\cacheData{{2018/05/09 domino50, Unknown Cache (2.5/2.5)}}\begin{cacheText}La bouteille m'a donné quelques maux de tête et a occupé quelques soirée. Aujourd'hui...C'est le grand jour!!!Une vingtaine de personnes prennent le départ pour former la Team Bouteille. La balade est faite avec la présence de l'owner dans une super ambiance et avec beaucoup de géoblabla. Toutes les caches sont trouvées (merci les enfants) sous l'œil bienveillant de Domino. La pause pique nique, très agréable nous a permis de prendre des forces pour attaquer les 2 montées!!!Un grand merci à domino 50 pour la qualité des caches et la découverte de ces jolis paysages(un PF pour l'ensemble du ''chai''d'oeuvre).\end{cacheText}

\cacheNumber{1424}\needspace{5\baselineskip}\cacheName{\href{http://coord.info/GC7KPZK}{[GTAQ]17-Les Mystères du vignoble} — \href{http://coord.info/GC7KPZK\Number{}766357182}{1424}}\cacheData{{2018/05/09 domino50, Unknown Cache (3/2.5)}}\begin{cacheText}La bouteille m'a donné quelques maux de tête et a occupé quelques soirée. Aujourd'hui...C'est le grand jour!!!Une vingtaine de personnes prennent le départ pour former la Team Bouteille. La balade est faite avec la présence de l'owner dans une super ambiance et avec beaucoup de géoblabla. Toutes les caches sont trouvées (merci les enfants) sous l'œil bienveillant de Domino. La pause pique nique, très agréable nous a permis de prendre des forces pour attaquer les 2 montées!!!Un grand merci à domino 50 pour la qualité des caches et la découverte de ces jolis paysages(un PF pour l'ensemble du ''chai''d'oeuvre).\end{cacheText}

\cacheNumber{1425}\needspace{5\baselineskip}\cacheName{\href{http://coord.info/GC7KQ05}{[GTAQ]18-Les Mystères du vignoble} — \href{http://coord.info/GC7KQ05\Number{}766362287}{1425}}\cacheData{{2018/05/09 domino50, Unknown Cache (2.5/2.5)}}\begin{cacheText}La bouteille m'a donné quelques maux de tête et a occupé quelques soirée. Aujourd'hui...C'est le grand jour!!!Une vingtaine de personnes prennent le départ pour former la Team Bouteille. La balade est faite avec la présence de l'owner dans une super ambiance et avec beaucoup de géoblabla. Toutes les caches sont trouvées (merci les enfants) sous l'œil bienveillant de Domino. La pause pique nique, très agréable nous a permis de prendre des forces pour attaquer les 2 montées!!!Un grand merci à domino 50 pour la qualité des caches et la découverte de ces jolis paysages(un PF pour l'ensemble du ''chai''d'oeuvre).\end{cacheText}

\cacheNumber{1426}\needspace{5\baselineskip}\cacheName{\href{http://coord.info/GC7KQ0C}{[GTAQ]19-Les Mystères du vignoble} — \href{http://coord.info/GC7KQ0C\Number{}766362347}{1426}}\cacheData{{2018/05/09 domino50, Unknown Cache (3/2.5)}}\begin{cacheText}La bouteille m'a donné quelques maux de tête et a occupé quelques soirée. Aujourd'hui...C'est le grand jour!!!Une vingtaine de personnes prennent le départ pour former la Team Bouteille. La balade est faite avec la présence de l'owner dans une super ambiance et avec beaucoup de géoblabla. Toutes les caches sont trouvées (merci les enfants) sous l'œil bienveillant de Domino. La pause pique nique, très agréable nous a permis de prendre des forces pour attaquer les 2 montées!!!Un grand merci à domino 50 pour la qualité des caches et la découverte de ces jolis paysages(un PF pour l'ensemble du ''chai''d'oeuvre).\end{cacheText}

\cacheNumber{1427}\needspace{5\baselineskip}\cacheName{\href{http://coord.info/GC7KQ11}{[GTAQ]21-Les Mystères du vignoble} — \href{http://coord.info/GC7KQ11\Number{}766362443}{1427}}\cacheData{{2018/05/09 domino50, Unknown Cache (2.5/2.5)}}\begin{cacheText}La bouteille m'a donné quelques maux de tête et a occupé quelques soirées. Aujourd'hui...C'est le grand jour!!!Une vingtaine de personnes prennent le départ pour former la Team Bouteille. La balade est faite avec la présence de l'owner dans une super ambiance et avec beaucoup de géoblabla. Toutes les caches sont trouvées (merci les enfants) sous l'œil bienveillant de Domino. La pause pique nique, très agréable nous a permis de prendre des forces pour attaquer les 2 montées!!!Un grand merci à domino 50 pour la qualité des caches et la découverte de ces jolis paysages(un PF pour l'ensemble du ''chai''d'oeuvre).\end{cacheText}

\cacheNumber{1428}\needspace{5\baselineskip}\cacheName{\href{http://coord.info/GC7KQ1C}{[GTAQ]22-Les Mystères du vignoble} — \href{http://coord.info/GC7KQ1C\Number{}766362498}{1428}}\cacheData{{2018/05/09 domino50, Unknown Cache (2.5/2.5)}}\begin{cacheText}La bouteille m'a donné quelques maux de tête et a occupé quelques soirées. Aujourd'hui...C'est le grand jour!!!Une vingtaine de personnes prennent le départ pour former la Team Bouteille. La balade est faite avec la présence de l'owner dans une super ambiance et avec beaucoup de géoblabla. Toutes les caches sont trouvées (merci les enfants) sous l'œil bienveillant de Domino. La pause pique nique, très agréable nous a permis de prendre des forces pour attaquer les 2 montées!!!Un grand merci à domino 50 pour la qualité des caches et la découverte de ces jolis paysages(un PF pour l'ensemble du ''chai''d'oeuvre).\end{cacheText}

\cacheNumber{1429}\needspace{5\baselineskip}\cacheName{\href{http://coord.info/GC7KQ1G}{[GTAQ]23-Les Mystères du vignoble} — \href{http://coord.info/GC7KQ1G\Number{}766362569}{1429}}\cacheData{{2018/05/09 domino50, Unknown Cache (2.5/2.5)}}\begin{cacheText}La bouteille m'a donné quelques maux de tête et a occupé quelques soirées. Aujourd'hui...C'est le grand jour!!!Une vingtaine de personnes prennent le départ pour former la Team Bouteille. La balade est faite avec la présence de l'owner dans une super ambiance et avec beaucoup de géoblabla. Toutes les caches sont trouvées (merci les enfants) sous l'œil bienveillant de Domino. La pause pique nique, très agréable nous a permis de prendre des forces pour attaquer les 2 montées!!!Un grand merci à domino 50 pour la qualité des caches et la découverte de ces jolis paysages(un PF pour l'ensemble du ''chai''d'oeuvre).\end{cacheText}

\cacheNumber{1430}\needspace{5\baselineskip}\cacheName{\href{http://coord.info/GC7KQ1P}{[GTAQ]24-Les Mystères du vignoble} — \href{http://coord.info/GC7KQ1P\Number{}766362622}{1430}}\cacheData{{2018/05/09 domino50, Unknown Cache (2.5/2.5)}}\begin{cacheText}La bouteille m'a donné quelques maux de tête et a occupé quelques soirées. Aujourd'hui...C'est le grand jour!!!Une vingtaine de personnes prennent le départ pour former la Team Bouteille. La balade est faite avec la présence de l'owner dans une super ambiance et avec beaucoup de géoblabla. Toutes les caches sont trouvées (merci les enfants) sous l'œil bienveillant de Domino. La pause pique nique, très agréable nous a permis de prendre des forces pour attaquer les 2 montées!!!Un grand merci à domino 50 pour la qualité des caches et la découverte de ces jolis paysages(un PF pour l'ensemble du ''chai''d'oeuvre).\end{cacheText}

\cacheNumber{1431}\needspace{5\baselineskip}\cacheName{\href{http://coord.info/GC7KQ1W}{[GTAQ]25-Les Mystères du vignoble} — \href{http://coord.info/GC7KQ1W\Number{}766362714}{1431}}\cacheData{{2018/05/09 domino50, Unknown Cache (2.5/2.5)}}\begin{cacheText}La bouteille m'a donné quelques maux de tête et a occupé quelques soirées. Aujourd'hui...C'est le grand jour!!!Une vingtaine de personnes prennent le départ pour former la Team Bouteille. La balade est faite avec la présence de l'owner dans une super ambiance et avec beaucoup de géoblabla. Toutes les caches sont trouvées (merci les enfants) sous l'œil bienveillant de Domino. La pause pique nique, très agréable nous a permis de prendre des forces pour attaquer les 2 montées!!!Un grand merci à domino 50 pour la qualité des caches et la découverte de ces jolis paysages(un PF pour l'ensemble du ''chai''d'oeuvre).\end{cacheText}

\cacheNumber{1432}\needspace{5\baselineskip}\cacheName{\href{http://coord.info/GC7KQ26}{[GTAQ]26-Les Mystères du vignoble} — \href{http://coord.info/GC7KQ26\Number{}766362766}{1432}}\cacheData{{2018/05/09 domino50, Unknown Cache (2.5/2.5)}}\begin{cacheText}La bouteille m'a donné quelques maux de tête et a occupé quelques soirées. Aujourd'hui...C'est le grand jour!!!Une vingtaine de personnes prennent le départ pour former la Team Bouteille. La balade est faite avec la présence de l'owner dans une super ambiance et avec beaucoup de géoblabla. Toutes les caches sont trouvées (merci les enfants) sous l'œil bienveillant de Domino. La pause pique nique, très agréable nous a permis de prendre des forces pour attaquer les 2 montées!!!Un grand merci à domino 50 pour la qualité des caches et la découverte de ces jolis paysages(un PF pour l'ensemble du ''chai''d'oeuvre).\end{cacheText}

\cacheNumber{1433}\needspace{5\baselineskip}\cacheName{\href{http://coord.info/GC7KQ2C}{[GTAQ]27-Les Mystères du vignoble} — \href{http://coord.info/GC7KQ2C\Number{}766362819}{1433}}\cacheData{{2018/05/09 domino50, Unknown Cache (2.5/2.5)}}\begin{cacheText}La bouteille m'a donné quelques maux de tête et a occupé quelques soirées. Aujourd'hui...C'est le grand jour!!!Une vingtaine de personnes prennent le départ pour former la Team Bouteille. La balade est faite avec la présence de l'owner dans une super ambiance et avec beaucoup de géoblabla. Toutes les caches sont trouvées (merci les enfants) sous l'œil bienveillant de Domino. La pause pique nique, très agréable nous a permis de prendre des forces pour attaquer les 2 montées!!!Un grand merci à domino 50 pour la qualité des caches et la découverte de ces jolis paysages(un PF pour l'ensemble du ''chai''d'oeuvre).\end{cacheText}

\cacheNumber{1434}\needspace{5\baselineskip}\cacheName{\href{http://coord.info/GC7KQ2H}{[GTAQ]28-Les Mystères du vignoble} — \href{http://coord.info/GC7KQ2H\Number{}766362872}{1434}}\cacheData{{2018/05/09 domino50, Unknown Cache (2.5/2.5)}}\begin{cacheText}La bouteille m'a donné quelques maux de tête et a occupé quelques soirées. Aujourd'hui...C'est le grand jour!!!Une vingtaine de personnes prennent le départ pour former la Team Bouteille. La balade est faite avec la présence de l'owner dans une super ambiance et avec beaucoup de géoblabla. Toutes les caches sont trouvées (merci les enfants) sous l'œil bienveillant de Domino. La pause pique nique, très agréable nous a permis de prendre des forces pour attaquer les 2 montées!!!Un grand merci à domino 50 pour la qualité des caches et la découverte de ces jolis paysages(un PF pour l'ensemble du ''chai''d'oeuvre).\end{cacheText}

\cacheNumber{1435}\needspace{5\baselineskip}\cacheName{\href{http://coord.info/GC7KQ2N}{[GTAQ]29-Les Mystères du vignoble} — \href{http://coord.info/GC7KQ2N\Number{}766362917}{1435}}\cacheData{{2018/05/09 domino50, Unknown Cache (2.5/2.5)}}\begin{cacheText}La bouteille m'a donné quelques maux de tête et a occupé quelques soirées. Aujourd'hui...C'est le grand jour!!!Une vingtaine de personnes prennent le départ pour former la Team Bouteille. La balade est faite avec la présence de l'owner dans une super ambiance et avec beaucoup de géoblabla. Toutes les caches sont trouvées (merci les enfants) sous l'œil bienveillant de Domino. La pause pique nique, très agréable nous a permis de prendre des forces pour attaquer les 2 montées!!!Un grand merci à domino 50 pour la qualité des caches et la découverte de ces jolis paysages(un PF pour l'ensemble du ''chai''d'oeuvre).\end{cacheText}

\cacheNumber{1436}\needspace{5\baselineskip}\cacheName{\href{http://coord.info/GC7KQ2T}{[GTAQ]30-Les Mystères du vignoble} — \href{http://coord.info/GC7KQ2T\Number{}766363002}{1436}}\cacheData{{2018/05/09 domino50, Unknown Cache (2.5/2.5)}}\begin{cacheText}La bouteille m'a donné quelques maux de tête et a occupé quelques soirées. Aujourd'hui...C'est le grand jour!!!Une vingtaine de personnes prennent le départ pour former la Team Bouteille. La balade est faite avec la présence de l'owner dans une super ambiance et avec beaucoup de géoblabla. Toutes les caches sont trouvées (merci les enfants) sous l'œil bienveillant de Domino. La pause pique nique, très agréable nous a permis de prendre des forces pour attaquer les 2 montées!!!Un grand merci à domino 50 pour la qualité des caches et la découverte de ces jolis paysages(un PF pour l'ensemble du ''chai''d'oeuvre).\end{cacheText}

\cacheNumber{1437}\needspace{5\baselineskip}\cacheName{\href{http://coord.info/GC7KQ2Y}{[GTAQ]31-Les Mystères du vignoble} — \href{http://coord.info/GC7KQ2Y\Number{}766363205}{1437}}\cacheData{{2018/05/09 domino50, Unknown Cache (2.5/2.5)}}\begin{cacheText}La bouteille m'a donné quelques maux de tête et a occupé quelques soirées. Aujourd'hui...C'est le grand jour!!!Une vingtaine de personnes prennent le départ pour former la Team Bouteille. La balade est faite avec la présence de l'owner dans une super ambiance et avec beaucoup de géoblabla. Toutes les caches sont trouvées (merci les enfants) sous l'œil bienveillant de Domino. La pause pique nique, très agréable nous a permis de prendre des forces pour attaquer les 2 montées!!!Un grand merci à domino 50 pour la qualité des caches et la découverte de ces jolis paysages(un PF pour l'ensemble du ''chai''d'oeuvre).\end{cacheText}

\cacheNumber{1438}\needspace{5\baselineskip}\cacheName{\href{http://coord.info/GC7KQ34}{[GTAQ]32-Les Mystères du vignoble} — \href{http://coord.info/GC7KQ34\Number{}766541467}{1438}}\cacheData{{2018/05/09 domino50, Unknown Cache (2.5/2.5)}}\begin{cacheText}La bouteille m'a donné quelques maux de tête et a occupé quelques soirées. Aujourd'hui...C'est le grand jour!!!Une vingtaine de personnes prennent le départ pour former la Team Bouteille. La balade est faite avec la présence de l'owner dans une super ambiance et avec beaucoup de géoblabla. Toutes les caches sont trouvées (merci les enfants) sous l'œil bienveillant de Domino. La pause pique nique, très agréable nous a permis de prendre des forces pour attaquer les 2 montées!!!Un grand merci à domino 50 pour la qualité des caches et la découverte de ces jolis paysages(un PF pour l'ensemble du ''chai''d'oeuvre).\end{cacheText}

\cacheNumber{1439}\needspace{5\baselineskip}\cacheName{\href{http://coord.info/GC7KQ3D}{[GTAQ]33-Les Mystères du vignoble} — \href{http://coord.info/GC7KQ3D\Number{}767488288}{1439}}\cacheData{{2018/05/09 domino50, Unknown Cache (2.5/2.5)}}\begin{cacheText}La bouteille m'a donné quelques maux de tête et a occupé quelques soirées. Aujourd'hui...C'est le grand jour!!!Une vingtaine de personnes prennent le départ pour former la Team Bouteille. La balade est faite avec la présence de l'owner dans une super ambiance et avec beaucoup de géoblabla. Toutes les caches sont trouvées (merci les enfants) sous l'œil bienveillant de Domino. La pause pique nique, très agréable nous a permis de prendre des forces pour attaquer les 2 montées!!!Un grand merci à domino 50 pour la qualité des caches et la découverte de ces jolis paysages(un PF pour l'ensemble du ''chai''d'oeuvre).\end{cacheText}

\cacheNumber{1440}\needspace{5\baselineskip}\cacheName{\href{http://coord.info/GC7N8Y0}{Meet \And{} Greet à Rauzan} — \href{http://coord.info/GC7N8Y0\Number{}767513134}{1440}}\cacheData{{2018/05/09 crick69 \And{} Carem38, Event Cache (1/1.5)}}\begin{cacheText}Après une bonne journée dans le Vignoble de Domino50,et quelques caches supplémentaires, nous avons juste le temps de passer au gite nous doucher. A  notre arrivée ,les géoblabla vont bon train. Les tables sont remplies de spécialités régionales...Que de bonnes choses!!!!

C'est une très bonne idée de se retrouver juste avant le GTAQ 4.Merci pour l'accueil et l'organisation de ce super Event.\end{cacheText}

\cacheNumber{1441}\needspace{5\baselineskip}\cacheName{\href{http://coord.info/GC7J926}{GTAQ - SAISON 4 - Episode 1 : Les Fleuves} — \href{http://coord.info/GC7J926\Number{}767550944}{1441}}\cacheData{{2018/05/10 GTAQ, Event Cache (1/1)}}\begin{cacheText}Enfin le moment tant attendu arrive : c'est le début du GTAQ 4 !Le petit café avalé ,le roadbook en poche et les GPS chargés nous choisissons le circuit de Sainte Terre pour le matin et celui de Bayat pour l'après midi. La Team 4 (64/44/40) est formée de Fabilab,Dune33 et moi même.Nous avons découverts de superbes endroits ,de très belles caches et fait de chouettes rencontres.

La soirée a bien débuté avec le traditionnel chou vert et le pique nique partagé a été trés animé .La soirée s'est terminée avec une petite night avant d'aller se reposer pour attaquer la journée 2 du GTAQ.

Uu grand merci aux organisateurs pour cette formidable journée.\end{cacheText}

\cacheNumber{1442}\needspace{5\baselineskip}\cacheName{\href{http://coord.info/GC7KRRN}{[GTAQ10] 01 - Sainte-Terre} — \href{http://coord.info/GC7KRRN\Number{}767595035}{1442}}\cacheData{{2018/05/10 clavelitos, Traditional Cache (1.5/1.5)}}\begin{cacheText}Les caches seront loguées \Quoted{team 4} pour les départements 40, 64 \And{} 44 (Dune33/Fleura64/Fabilab).

Nous avons d'abord choisi l'emplacement pour le pique-nique et avons commencé par la cache à proximité .Bonne surprise nous avons été les premiers sur quelques caches!!!!Le parcours au bord de la Dordogne est vraiment très sympa. Nous découvrons des caches travaillées et croisons de nombreuses teams sur le circuit. Les indices récoltés nous permettent d'obtenir les coordonnées de la finale.

Un grand merci clavelitos pour ces belles caches.\end{cacheText}

\cacheNumber{1443}\needspace{5\baselineskip}\cacheName{\href{http://coord.info/GC7KRTJ}{[GTAQ10] 02 - Sainte-Terre} — \href{http://coord.info/GC7KRTJ\Number{}767593191}{1443}}\cacheData{{2018/05/10 clavelitos, Traditional Cache (1.5/1.5)}}\begin{cacheText}Les caches seront loguées \Quoted{team 4} pour les départements 40, 64 \And{} 44 (Dune33/Fleura64/Fabilab).

Nous avons d'abord choisi l'emplacement pour le pique-nique et avons commencé par la cache à proximité .Bonne surprise nous avons été les premiers sur quelques caches!!!!Le parcours au bord de la Dordogne est vraiment très sympa. Nous découvrons des caches travaillées et croisons de nombreuses teams sur le circuit. Les indices récoltés nous permettent d'obtenir les coordonnées de la finale.

Un grand merci clavelitos pour ces belles caches.\end{cacheText}

\cacheNumber{1444}\needspace{5\baselineskip}\cacheName{\href{http://coord.info/GC7KRTV}{[GTAQ10] 03 - Sainte-Terre} — \href{http://coord.info/GC7KRTV\Number{}767592722}{1444}}\cacheData{{2018/05/10 clavelitos, Traditional Cache (1.5/1.5)}}\begin{cacheText}Les caches seront loguées \Quoted{team 4} pour les départements 40, 64 \And{} 44 (Dune33/Fleura64/Fabilab).

Nous avons d'abord choisi l'emplacement pour le pique-nique et avons commencé par la cache à proximité .Bonne surprise nous avons été les premiers sur quelques caches!!!!Le parcours au bord de la Dordogne est vraiment très sympa. Nous découvrons des caches travaillées et croisons de nombreuses teams sur le circuit. Les indices récoltés nous permettent d'obtenir les coordonnées de la finale.

Un grand merci clavelitos pour ces belles caches.\end{cacheText}

\cacheNumber{1445}\needspace{5\baselineskip}\cacheName{\href{http://coord.info/GC7KRVB}{[GTAQ10] 04 - Sainte-Terre} — \href{http://coord.info/GC7KRVB\Number{}767590342}{1445}}\cacheData{{2018/05/10 clavelitos, Traditional Cache (1.5/1.5)}}\begin{cacheText}Les caches seront loguées \Quoted{team 4} pour les départements 40, 64 \And{} 44 (Dune33/Fleura64/Fabilab).

Nous avons d'abord choisi l'emplacement pour le pique-nique et avons commencé par la cache à proximité .Bonne surprise nous avons été les premiers sur quelques caches!!!!Le parcours au bord de la Dordogne est vraiment très sympa. Nous découvrons des caches travaillées et croisons de nombreuses teams sur le circuit. Les indices récoltés nous permettent d'obtenir les coordonnées de la finale.

Un grand merci clavelitos pour ces belles caches.\end{cacheText}

\cacheNumber{1446}\needspace{5\baselineskip}\cacheName{\href{http://coord.info/GC7KRW4}{[GTAQ10] 05 - Sainte-Terre} — \href{http://coord.info/GC7KRW4\Number{}767589813}{1446}}\cacheData{{2018/05/10 clavelitos, Traditional Cache (1.5/1.5)}}\begin{cacheText}Les caches seront loguées \Quoted{team 4} pour les départements 40, 64 \And{} 44 (Dune33/Fleura64/Fabilab).

Nous avons d'abord choisi l'emplacement pour le pique-nique et avons commencé par la cache à proximité .Bonne surprise nous avons été les premiers sur quelques caches!!!!Le parcours au bord de la Dordogne est vraiment très sympa. Nous découvrons des caches travaillées et croisons de nombreuses teams sur le circuit. Les indices récoltés nous permettent d'obtenir les coordonnées de la finale.

Un grand merci clavelitos pour ces belles caches.\end{cacheText}

\cacheNumber{1447}\needspace{5\baselineskip}\cacheName{\href{http://coord.info/GC7KTHV}{[GTAQ10] 06 - Sainte-Terre} — \href{http://coord.info/GC7KTHV\Number{}767589201}{1447}}\cacheData{{2018/05/10 clavelitos, Traditional Cache (1.5/1.5)}}\begin{cacheText}Les caches seront loguées \Quoted{team 4} pour les départements 40, 64 \And{} 44 (Dune33/Fleura64/Fabilab).

Nous avons d'abord choisi l'emplacement pour le pique-nique et avons commencé par la cache à proximité .Bonne surprise nous avons été les premiers sur quelques caches!!!!Le parcours au bord de la Dordogne est vraiment très sympa. Nous découvrons des caches travaillées et croisons de nombreuses teams sur le circuit. Les indices récoltés nous permettent d'obtenir les coordonnées de la finale.

Un grand merci clavelitos pour ces belles caches.\end{cacheText}

\cacheNumber{1448}\needspace{5\baselineskip}\cacheName{\href{http://coord.info/GC7KTHY}{[GTAQ10] 07 - Sainte-Terre} — \href{http://coord.info/GC7KTHY\Number{}767588532}{1448}}\cacheData{{2018/05/10 clavelitos, Traditional Cache (1.5/1.5)}}\begin{cacheText}Les caches seront loguées \Quoted{team 4} pour les départements 40, 64 \And{} 44 (Dune33/Fleura64/Fabilab).

Nous avons d'abord choisi l'emplacement pour le pique-nique et avons commencé par la cache à proximité .Bonne surprise nous avons été les premiers sur quelques caches!!!!Le parcours au bord de la Dordogne est vraiment très sympa. Nous découvrons des caches travaillées et croisons de nombreuses teams sur le circuit. Les indices récoltés nous permettent d'obtenir les coordonnées de la finale.

Un grand merci clavelitos pour ces belles caches.\end{cacheText}

\cacheNumber{1449}\needspace{5\baselineskip}\cacheName{\href{http://coord.info/GC7KTHZ}{[GTAQ10] 08 - Sainte-Terre} — \href{http://coord.info/GC7KTHZ\Number{}767587788}{1449}}\cacheData{{2018/05/10 clavelitos, Traditional Cache (2.5/1.5)}}\begin{cacheText}Les caches seront loguées \Quoted{team 4} pour les départements 40, 64 \And{} 44 (Dune33/Fleura64/Fabilab).

Nous avons d'abord choisi l'emplacement pour le pique-nique et avons commencé par la cache à proximité .Bonne surprise nous avons été les premiers sur quelques caches!!!!Le parcours au bord de la Dordogne est vraiment très sympa. Nous découvrons des caches travaillées et croisons de nombreuses teams sur le circuit. Les indices récoltés nous permettent d'obtenir les coordonnées de la finale.

Un grand merci clavelitos pour ces belles caches.\end{cacheText}

\cacheNumber{1450}\needspace{5\baselineskip}\cacheName{\href{http://coord.info/GC7KTJ2}{[GTAQ10] 09 - Sainte-Terre} — \href{http://coord.info/GC7KTJ2\Number{}767584729}{1450}}\cacheData{{2018/05/10 clavelitos, Traditional Cache (1.5/1.5)}}\begin{cacheText}Les caches seront loguées \Quoted{team 4} pour les départements 40, 64 \And{} 44 (Dune33/Fleura64/Fabilab).

Nous avons d'abord choisi l'emplacement pour le pique-nique et avons commencé par la cache à proximité .Bonne surprise nous avons été les premiers sur quelques caches!!!!Le parcours au bord de la Dordogne est vraiment très sympa. Nous découvrons des caches travaillées et croisons de nombreuses teams sur le circuit. Les indices récoltés nous permettent d'obtenir les coordonnées de la finale.

Un grand merci clavelitos pour ces belles caches.\end{cacheText}

\cacheNumber{1451}\needspace{5\baselineskip}\cacheName{\href{http://coord.info/GC7KTJ4}{[GTAQ10] 10 - Sainte-Terre} — \href{http://coord.info/GC7KTJ4\Number{}767584203}{1451}}\cacheData{{2018/05/10 clavelitos, Traditional Cache (2/2.5)}}\begin{cacheText}Les caches seront loguées \Quoted{team 4} pour les départements 40, 64 \And{} 44 (Dune33/Fleura64/Fabilab).

Nous avons d'abord choisi l'emplacement pour le pique-nique et avons commencé par la cache à proximité .Bonne surprise nous avons été les premiers sur quelques caches!!!!Le parcours au bord de la Dordogne est vraiment très sympa. Nous découvrons des caches travaillées et croisons de nombreuses teams sur le circuit. Les indices récoltés nous permettent d'obtenir les coordonnées de la finale.

Un grand merci clavelitos pour ces belles caches et un PF bien mérité.\end{cacheText}

\cacheNumber{1452}\needspace{5\baselineskip}\cacheName{\href{http://coord.info/GC7KTJA}{[GTAQ10] 11 - Sainte-Terre} — \href{http://coord.info/GC7KTJA\Number{}767583100}{1452}}\cacheData{{2018/05/10 clavelitos, Traditional Cache (1.5/1.5)}}\begin{cacheText}Les caches seront loguées \Quoted{team 4} pour les départements 40, 64 \And{} 44 (Dune33/Fleura64/Fabilab).

Nous avons d'abord choisi l'emplacement pour le pique-nique et avons commencé par la cache à proximité .Bonne surprise nous avons été les premiers sur quelques caches!!!!Le parcours au bord de la Dordogne est vraiment très sympa. Nous découvrons des caches travaillées et croisons de nombreuses teams sur le circuit. Les indices récoltés nous permettent d'obtenir les coordonnées de la finale.

Un grand merci clavelitos pour ces belles caches.\end{cacheText}

\cacheNumber{1453}\needspace{5\baselineskip}\cacheName{\href{http://coord.info/GC7KTJC}{[GTAQ10] 12 - Sainte-Terre} — \href{http://coord.info/GC7KTJC\Number{}767582479}{1453}}\cacheData{{2018/05/10 clavelitos, Traditional Cache (2/1.5)}}\begin{cacheText}Les caches seront loguées \Quoted{team 4} pour les départements 40, 64 \And{} 44 (Dune33/Fleura64/Fabilab).

Nous avons d'abord choisi l'emplacement pour le pique-nique et avons commencé par la cache à proximité .Bonne surprise nous avons été les premiers sur quelques caches!!!!Le parcours au bord de la Dordogne est vraiment très sympa. Nous découvrons des caches travaillées et croisons de nombreuses teams sur le circuit. Les indices récoltés nous permettent d'obtenir les coordonnées de la finale.

Un grand merci clavelitos pour ces belles caches.\end{cacheText}

\cacheNumber{1454}\needspace{5\baselineskip}\cacheName{\href{http://coord.info/GC7KTJG}{[GTAQ10] 13 - Sainte-Terre} — \href{http://coord.info/GC7KTJG\Number{}767581902}{1454}}\cacheData{{2018/05/10 clavelitos, Traditional Cache (3/1.5)}}\begin{cacheText}(TTF)

Les caches seront loguées \Quoted{team 4} pour les départements 40, 64 \And{} 44 (Dune33/Fleura64/Fabilab).

Nous avons d'abord choisi l'emplacement pour le pique-nique et avons commencé par la cache à proximité .Bonne surprise nous avons été les premiers sur quelques caches!!!!Le parcours au bord de la Dordogne est vraiment très sympa. Nous découvrons des caches travaillées et croisons de nombreuses teams sur le circuit. Les indices récoltés nous permettent d'obtenir les coordonnées de la finale.

Un grand merci clavelitos pour ces belles caches.\end{cacheText}

\cacheNumber{1455}\needspace{5\baselineskip}\cacheName{\href{http://coord.info/GC7KTJJ}{[GTAQ10] 14 - Sainte-Terre} — \href{http://coord.info/GC7KTJJ\Number{}767581446}{1455}}\cacheData{{2018/05/10 clavelitos, Traditional Cache (1.5/1.5)}}\begin{cacheText}Les caches seront loguées \Quoted{team 4} pour les départements 40, 64 \And{} 44 (Dune33/Fleura64/Fabilab).

Nous avons d'abord choisi l'emplacement pour le pique-nique et avons commencé par la cache à proximité .Bonne surprise nous avons été les premiers sur quelques caches!!!!Le parcours au bord de la Dordogne est vraiment très sympa. Nous découvrons des caches travaillées et croisons de nombreuses teams sur le circuit. Les indices récoltés nous permettent d'obtenir les coordonnées de la finale.

Un grand merci clavelitos pour ces belles caches.\end{cacheText}

\cacheNumber{1456}\needspace{5\baselineskip}\cacheName{\href{http://coord.info/GC7KTJK}{[GTAQ10] 15 - Sainte-Terre} — \href{http://coord.info/GC7KTJK\Number{}767577764}{1456}}\cacheData{{2018/05/10 clavelitos, Traditional Cache (1.5/1.5)}}\begin{cacheText}Les caches seront loguées \Quoted{team 4} pour les départements 40, 64 \And{} 44 (Dune33/Fleura64/Fabilab).

Nous avons d'abord choisi l'emplacement pour le pique-nique et avons commencé par la cache à proximité .Bonne surprise nous avons été les premiers sur quelques caches!!!!Le parcours au bord de la Dordogne est vraiment très sympa. Nous découvrons des caches travaillées et croisons de nombreuses teams sur le circuit. Les indices récoltés nous permettent d'obtenir les coordonnées de la finale.

Un grand merci clavelitos pour ces belles caches.\end{cacheText}

\cacheNumber{1457}\needspace{5\baselineskip}\cacheName{\href{http://coord.info/GC7KTJM}{[GTAQ10] 16 - Sainte-Terre} — \href{http://coord.info/GC7KTJM\Number{}767577221}{1457}}\cacheData{{2018/05/10 clavelitos, Traditional Cache (1.5/1.5)}}\begin{cacheText}Les caches seront loguées \Quoted{team 4} pour les départements 40, 64 \And{} 44 (Dune33/Fleura64/Fabilab).

Nous avons d'abord choisi l'emplacement pour le pique-nique et avons commencé par la cache à proximité .Bonne surprise nous avons été les premiers sur quelques caches!!!!Le parcours au bord de la Dordogne est vraiment très sympa. Nous découvrons des caches travaillées et croisons de nombreuses teams sur le circuit. Les indices récoltés nous permettent d'obtenir les coordonnées de la finale.

Un grand merci clavelitos pour ces belles caches.\end{cacheText}

\cacheNumber{1458}\needspace{5\baselineskip}\cacheName{\href{http://coord.info/GC7KTJP}{[GTAQ10] 17 - Sainte-Terre} — \href{http://coord.info/GC7KTJP\Number{}767576826}{1458}}\cacheData{{2018/05/10 clavelitos, Traditional Cache (1.5/1.5)}}\begin{cacheText}(STF)

Les caches seront loguées \Quoted{team 4} pour les départements 40, 64 \And{} 44 (Dune33/Fleura64/Fabilab).

Nous avons d'abord choisi l'emplacement pour le pique-nique et avons commencé par la cache à proximité .Bonne surprise nous avons été les premiers sur quelques caches!!!!Le parcours au bord de la Dordogne est vraiment très sympa. Nous découvrons des caches travaillées et croisons de nombreuses teams sur le circuit. Les indices récoltés nous permettent d'obtenir les coordonnées de la finale.

Un grand merci clavelitos pour ces belles caches.\end{cacheText}

\cacheNumber{1459}\needspace{5\baselineskip}\cacheName{\href{http://coord.info/GC7KTJR}{[GTAQ10] 18 - Sainte-Terre} — \href{http://coord.info/GC7KTJR\Number{}767576312}{1459}}\cacheData{{2018/05/10 clavelitos, Traditional Cache (1.5/1.5)}}\begin{cacheText}co(FTF)

Les caches seront loguées \Quoted{team 4} pour les départements 40, 64 \And{} 44 (Dune33/Fleura64/Fabilab).

Nous avons d'abord choisi l'emplacement pour le pique-nique et avons commencé par la cache à proximité .Bonne surprise nous avons été les premiers sur quelques caches!!!!Le parcours au bord de la Dordogne est vraiment très sympa. Nous découvrons des caches travaillées et croisons de nombreuses teams sur le circuit. Les indices récoltés nous permettent d'obtenir les coordonnées de la finale.

Un grand merci clavelitos pour ces belles caches.\end{cacheText}

\cacheNumber{1460}\needspace{5\baselineskip}\cacheName{\href{http://coord.info/GC7KTJT}{[GTAQ10] 19 - Sainte-Terre} — \href{http://coord.info/GC7KTJT\Number{}767575901}{1460}}\cacheData{{2018/05/10 clavelitos, Traditional Cache (1.5/1.5)}}\begin{cacheText}co(FTF)

Les caches seront loguées \Quoted{team 4} pour les départements 40, 64 \And{} 44 (Dune33/Fleura64/Fabilab).

Nous avons d'abord choisi l'emplacement pour le pique-nique et avons commencé par la cache à proximité .Bonne surprise nous avons été les premiers sur quelques caches!!!!Le parcours au bord de la Dordogne est vraiment très sympa. Nous découvrons des caches travaillées et croisons de nombreuses teams sur le circuit. Les indices récoltés nous permettent d'obtenir les coordonnées de la finale.

Un grand merci clavelitos pour ces belles caches.\end{cacheText}

\cacheNumber{1461}\needspace{5\baselineskip}\cacheName{\href{http://coord.info/GC7KTJX}{[GTAQ10] 20 - Sainte-Terre} — \href{http://coord.info/GC7KTJX\Number{}767564734}{1461}}\cacheData{{2018/05/10 clavelitos, Traditional Cache (1.5/1.5)}}\begin{cacheText}co(FTF)

Les caches seront loguées \Quoted{team 4} pour les départements 40, 64 \And{} 44 (Dune33/Fleura64/Fabilab).

Nous avons d'abord choisi l'emplacement pour le pique-nique et avons commencé par la cache à proximité .Bonne surprise nous avons été les premiers sur quelques caches!!!!Le parcours au bord de la Dordogne est vraiment très sympa. Nous découvrons des caches travaillées et croisons de nombreuses teams sur le circuit. Les indices récoltés nous permettent d'obtenir les coordonnées de la finale.

Un grand merci clavelitos pour ces belles caches.\end{cacheText}

\cacheNumber{1462}\needspace{5\baselineskip}\cacheName{\href{http://coord.info/GC7KTJZ}{[GTAQ10] 21 - Sainte-Terre} — \href{http://coord.info/GC7KTJZ\Number{}767655484}{1462}}\cacheData{{2018/05/10 clavelitos, Traditional Cache (1.5/1.5)}}\begin{cacheText}Les caches seront loguées \Quoted{team 4} pour les départements 40, 64 \And{} 44 (Dune33/Fleura64/Fabilab).

Nous avons d'abord choisi l'emplacement pour le pique-nique et avons commencé par la cache à proximité .Bonne surprise nous avons été les premiers sur quelques caches!!!!Le parcours au bord de la Dordogne est vraiment très sympa. Nous découvrons des caches travaillées et croisons de nombreuses teams sur le circuit. Les indices récoltés nous permettent d'obtenir les coordonnées de la finale.

Un grand merci clavelitos pour ces belles caches.\end{cacheText}

\cacheNumber{1463}\needspace{5\baselineskip}\cacheName{\href{http://coord.info/GC7KTK0}{[GTAQ10] 22 - Sainte-Terre} — \href{http://coord.info/GC7KTK0\Number{}767653875}{1463}}\cacheData{{2018/05/10 clavelitos, Traditional Cache (1.5/1.5)}}\begin{cacheText}Les caches seront loguées \Quoted{team 4} pour les départements 40, 64 \And{} 44 (Dune33/Fleura64/Fabilab).

Nous avons d'abord choisi l'emplacement pour le pique-nique et avons commencé par la cache à proximité .Bonne surprise nous avons été les premiers sur quelques caches!!!!Le parcours au bord de la Dordogne est vraiment très sympa. Nous découvrons des caches travaillées et croisons de nombreuses teams sur le circuit. Les indices récoltés nous permettent d'obtenir les coordonnées de la finale.

Un grand merci clavelitos pour ces belles caches.\end{cacheText}

\cacheNumber{1464}\needspace{5\baselineskip}\cacheName{\href{http://coord.info/GC7KTK2}{[GTAQ10] 23 - Sainte-Terre} — \href{http://coord.info/GC7KTK2\Number{}767653343}{1464}}\cacheData{{2018/05/10 clavelitos, Traditional Cache (1.5/1.5)}}\begin{cacheText}Les caches seront loguées \Quoted{team 4} pour les départements 40, 64 \And{} 44 (Dune33/Fleura64/Fabilab).

Nous avons d'abord choisi l'emplacement pour le pique-nique et avons commencé par la cache à proximité .Bonne surprise nous avons été les premiers sur quelques caches!!!!Le parcours au bord de la Dordogne est vraiment très sympa. Nous découvrons des caches travaillées et croisons de nombreuses teams sur le circuit. Les indices récoltés nous permettent d'obtenir les coordonnées de la finale.

Un grand merci clavelitos pour ces belles caches.\end{cacheText}

\cacheNumber{1465}\needspace{5\baselineskip}\cacheName{\href{http://coord.info/GC7KTK5}{[GTAQ10] 24 - Sainte-Terre} — \href{http://coord.info/GC7KTK5\Number{}767652609}{1465}}\cacheData{{2018/05/10 clavelitos, Traditional Cache (1.5/1.5)}}\begin{cacheText}Les caches seront loguées \Quoted{team 4} pour les départements 40, 64 \And{} 44 (Dune33/Fleura64/Fabilab).

Nous avons d'abord choisi l'emplacement pour le pique-nique et avons commencé par la cache à proximité .Bonne surprise nous avons été les premiers sur quelques caches!!!!Le parcours au bord de la Dordogne est vraiment très sympa. Nous découvrons des caches travaillées et croisons de nombreuses teams sur le circuit. Les indices récoltés nous permettent d'obtenir les coordonnées de la finale.

Un grand merci clavelitos pour ces belles caches.\end{cacheText}

\cacheNumber{1466}\needspace{5\baselineskip}\cacheName{\href{http://coord.info/GC7KTKA}{[GTAQ10] 25 - Sainte-Terre} — \href{http://coord.info/GC7KTKA\Number{}767651667}{1466}}\cacheData{{2018/05/10 clavelitos, Traditional Cache (1.5/1.5)}}\begin{cacheText}Les caches seront loguées \Quoted{team 4} pour les départements 40, 64 \And{} 44 (Dune33/Fleura64/Fabilab).

Nous avons d'abord choisi l'emplacement pour le pique-nique et avons commencé par la cache à proximité .Bonne surprise nous avons été les premiers sur quelques caches!!!!Le parcours au bord de la Dordogne est vraiment très sympa. Nous découvrons des caches travaillées et croisons de nombreuses teams sur le circuit. Les indices récoltés nous permettent d'obtenir les coordonnées de la finale.

Un grand merci clavelitos pour ces belles caches.\end{cacheText}

\cacheNumber{1467}\needspace{5\baselineskip}\cacheName{\href{http://coord.info/GC7KTKC}{[GTAQ10] 26 - Sainte-Terre} — \href{http://coord.info/GC7KTKC\Number{}767651081}{1467}}\cacheData{{2018/05/10 clavelitos, Traditional Cache (1.5/1.5)}}\begin{cacheText}Les caches seront loguées \Quoted{team 4} pour les départements 40, 64 \And{} 44 (Dune33/Fleura64/Fabilab).

Nous avons d'abord choisi l'emplacement pour le pique-nique et avons commencé par la cache à proximité .Bonne surprise nous avons été les premiers sur quelques caches!!!!Le parcours au bord de la Dordogne est vraiment très sympa. Nous découvrons des caches travaillées et croisons de nombreuses teams sur le circuit. Les indices récoltés nous permettent d'obtenir les coordonnées de la finale.

Un grand merci clavelitos pour ces belles caches.\end{cacheText}

\cacheNumber{1468}\needspace{5\baselineskip}\cacheName{\href{http://coord.info/GC7KTKF}{[GTAQ10] 27 - Sainte-Terre} — \href{http://coord.info/GC7KTKF\Number{}767650428}{1468}}\cacheData{{2018/05/10 clavelitos, Traditional Cache (2/2)}}\begin{cacheText}Les caches seront loguées \Quoted{team 4} pour les départements 40, 64 \And{} 44 (Dune33/Fleura64/Fabilab).

Nous avons d'abord choisi l'emplacement pour le pique-nique et avons commencé par la cache à proximité .Bonne surprise nous avons été les premiers sur quelques caches!!!!Le parcours au bord de la Dordogne est vraiment très sympa. Nous découvrons des caches travaillées et croisons de nombreuses teams sur le circuit. Les indices récoltés nous permettent d'obtenir les coordonnées de la finale.

Un grand merci clavelitos pour ces belles caches.\end{cacheText}

\cacheNumber{1469}\needspace{5\baselineskip}\cacheName{\href{http://coord.info/GC7KTKM}{[GTAQ10] 28 - Sainte-Terre} — \href{http://coord.info/GC7KTKM\Number{}767649476}{1469}}\cacheData{{2018/05/10 clavelitos, Traditional Cache (1.5/1.5)}}\begin{cacheText}Les caches seront loguées \Quoted{team 4} pour les départements 40, 64 \And{} 44 (Dune33/Fleura64/Fabilab).

Nous avons d'abord choisi l'emplacement pour le pique-nique et avons commencé par la cache à proximité .Bonne surprise nous avons été les premiers sur quelques caches!!!!Le parcours au bord de la Dordogne est vraiment très sympa. Nous découvrons des caches travaillées et croisons de nombreuses teams sur le circuit. Les indices récoltés nous permettent d'obtenir les coordonnées de la finale.

Un grand merci clavelitos pour ces belles caches.\end{cacheText}

\cacheNumber{1470}\needspace{5\baselineskip}\cacheName{\href{http://coord.info/GC7KTKR}{[GTAQ10] 29 - Sainte-Terre} — \href{http://coord.info/GC7KTKR\Number{}767646589}{1470}}\cacheData{{2018/05/10 clavelitos, Traditional Cache (1.5/1.5)}}\begin{cacheText}Les caches seront loguées \Quoted{team 4} pour les départements 40, 64 \And{} 44 (Dune33/Fleura64/Fabilab).

Nous avons d'abord choisi l'emplacement pour le pique-nique et avons commencé par la cache à proximité .Bonne surprise nous avons été les premiers sur quelques caches!!!!Le parcours au bord de la Dordogne est vraiment très sympa. Nous découvrons des caches travaillées et croisons de nombreuses teams sur le circuit. Les indices récoltés nous permettent d'obtenir les coordonnées de la finale.

Un grand merci clavelitos pour ces belles caches.\end{cacheText}

\cacheNumber{1471}\needspace{5\baselineskip}\cacheName{\href{http://coord.info/GC7KTKV}{[GTAQ10] 30 - Sainte-Terre} — \href{http://coord.info/GC7KTKV\Number{}767640023}{1471}}\cacheData{{2018/05/10 clavelitos, Traditional Cache (1.5/1.5)}}\begin{cacheText}Les caches seront loguées \Quoted{team 4} pour les départements 40, 64 \And{} 44 (Dune33/Fleura64/Fabilab).

Nous avons d'abord choisi l'emplacement pour le pique-nique et avons commencé par la cache à proximité .Bonne surprise nous avons été les premiers sur quelques caches!!!!Le parcours au bord de la Dordogne est vraiment très sympa. Nous découvrons des caches travaillées et croisons de nombreuses teams sur le circuit. Les indices récoltés nous permettent d'obtenir les coordonnées de la finale.

Un grand merci clavelitos pour ces belles caches.\end{cacheText}

\cacheNumber{1472}\needspace{5\baselineskip}\cacheName{\href{http://coord.info/GC7KTM5}{[GTAQ10] Bonus - Sainte-Terre} — \href{http://coord.info/GC7KTM5\Number{}767634103}{1472}}\cacheData{{2018/05/10 clavelitos, Unknown Cache (2.5/2)}}\begin{cacheText}co(TTF) 

Les caches seront loguées \Quoted{team 4} pour les départements 40, 64 \And{} 44 (Dune33/Fleura64/Fabilab).

Nous avons d'abord choisi l'emplacement pour le pique-nique et avons commencé par la cache à proximité .Bonne surprise nous avons été les premiers sur quelques caches!!!!Le parcours au bord de la Dordogne est vraiment très sympa. Nous découvrons des caches travaillées et croisons de nombreuses teams sur le circuit. Les indices récoltés nous permettent d'obtenir les coordonnées de la finale.

 Nous arrivons sur les lieux en même temps que la team LBDO(Les Batailleurs De l'Ouest)

Un grand merci clavelitos pour ces belles caches et un PF pour la finale.\end{cacheText}

\cacheNumber{1473}\needspace{5\baselineskip}\cacheName{\href{http://coord.info/GC7M6B6}{[GTAQ10] 01 - Bayat} — \href{http://coord.info/GC7M6B6\Number{}767671500}{1473}}\cacheData{{2018/05/10 spooks753, Traditional Cache (1.5/1.5)}}\begin{cacheText}Apres une longue pause pique nique la\Quoted{team 4}( pour les départements 40(Dune33), 64(Fleura64) \And{} 44(Fabilab) se lance sur ce sympathique circuit.Nous 

rencontrons à la cache 03 les Black\Underscore{}archeon et decidons de faire le circuit ensemble.Toutes les caches sont trouvées et le parcours est fort agréable (nous avons même mangé quelques cerises en passant).

Un grand merci spooks753 pour toutes les caches.\end{cacheText}

\cacheNumber{1474}\needspace{5\baselineskip}\cacheName{\href{http://coord.info/GC7N8WF}{[GTAQ10] 02 - Bayat} — \href{http://coord.info/GC7N8WF\Number{}767673087}{1474}}\cacheData{{2018/05/10 SPooks753, Traditional Cache (1.5/1.5)}}\begin{cacheText}Apres une longue pause pique nique la\Quoted{team 4}( pour les départements 40(Dune33), 64(Fleura64) \And{} 44(Fabilab) se lance sur ce sympathique circuit.Nous 

rencontrons à la cache 03 les Black\Underscore{}archeon et decidons de faire le circuit ensemble.Toutes les caches sont trouvées et le parcours est fort agréable (nous avons même mangé quelques cerises en passant).

Un grand merci spooks753 pour toutes les caches.\end{cacheText}

\cacheNumber{1475}\needspace{5\baselineskip}\cacheName{\href{http://coord.info/GC7N8WN}{[GTAQ10] 03 - Bayat} — \href{http://coord.info/GC7N8WN\Number{}767673409}{1475}}\cacheData{{2018/05/10 SPooks753, Traditional Cache (1.5/1.5)}}\begin{cacheText}Apres une longue pause pique nique la\Quoted{team 4}( pour les départements 40(Dune33), 64(Fleura64) \And{} 44(Fabilab) se lance sur ce sympathique circuit.Nous 

rencontrons à la cache 03 les Black\Underscore{}archeon et decidons de faire le circuit ensemble.Toutes les caches sont trouvées et le parcours est fort agréable (nous avons même mangé quelques cerises en passant).

Un grand merci spooks753 pour toutes les caches.\end{cacheText}

\cacheNumber{1476}\needspace{5\baselineskip}\cacheName{\href{http://coord.info/GC7N8WV}{[GTAQ10] 04 - Bayat} — \href{http://coord.info/GC7N8WV\Number{}767674484}{1476}}\cacheData{{2018/05/10 SPooks753, Traditional Cache (1.5/1.5)}}\begin{cacheText}Apres une longue pause pique nique la\Quoted{team 4}( pour les départements 40(Dune33), 64(Fleura64) \And{} 44(Fabilab) se lance sur ce sympathique circuit.Nous 

rencontrons à la cache 03 les Black\Underscore{}archeon et decidons de faire le circuit ensemble.Toutes les caches sont trouvées et le parcours est fort agréable (nous avons même mangé quelques cerises en passant).

Un grand merci spooks753 pour toutes les caches.\end{cacheText}

\cacheNumber{1477}\needspace{5\baselineskip}\cacheName{\href{http://coord.info/GC7N8X3}{[GTAQ10] 05 - Bayat} — \href{http://coord.info/GC7N8X3\Number{}767676707}{1477}}\cacheData{{2018/05/10 SPooks753, Traditional Cache (1.5/1.5)}}\begin{cacheText}Apres une longue pause pique nique la\Quoted{team 4}( pour les départements 40(Dune33), 64(Fleura64) \And{} 44(Fabilab) se lance sur ce sympathique circuit.Nous 

rencontrons à la cache 03 les Black\Underscore{}archeon et decidons de faire le circuit ensemble.Toutes les caches sont trouvées et le parcours est fort agréable (nous avons même mangé quelques cerises en passant).

Un grand merci spooks753 pour toutes les caches.\end{cacheText}

\cacheNumber{1478}\needspace{5\baselineskip}\cacheName{\href{http://coord.info/GC7N8X5}{[GTAQ10] 06 - Bayat} — \href{http://coord.info/GC7N8X5\Number{}767751631}{1478}}\cacheData{{2018/05/10 SPooks753, Traditional Cache (1.5/1.5)}}\begin{cacheText}Apres une longue pause pique nique la\Quoted{team 4}( pour les départements 40(Dune33), 64(Fleura64) \And{} 44(Fabilab) se lance sur ce sympathique circuit.Nous 

rencontrons à la cache 03 les Black\Underscore{}archeon et decidons de faire le circuit ensemble.Toutes les caches sont trouvées et le parcours est fort agréable (nous avons même mangé quelques cerises en passant).

Un grand merci spooks753 pour toutes les caches.\end{cacheText}

\cacheNumber{1479}\needspace{5\baselineskip}\cacheName{\href{http://coord.info/GC7N8XB}{[GTAQ10] 07 - Bayat} — \href{http://coord.info/GC7N8XB\Number{}767755614}{1479}}\cacheData{{2018/05/10 SPooks753, Traditional Cache (1.5/1.5)}}\begin{cacheText}Apres une longue pause pique nique la\Quoted{team 4}( pour les départements 40(Dune33), 64(Fleura64) \And{} 44(Fabilab) se lance sur ce sympathique circuit.Nous 

rencontrons à la cache 03 les Black\Underscore{}archeon et decidons de faire le circuit ensemble.Toutes les caches sont trouvées et le parcours est fort agréable (nous avons même mangé quelques cerises en passant).

Un grand merci spooks753 pour toutes les caches.\end{cacheText}

\cacheNumber{1480}\needspace{5\baselineskip}\cacheName{\href{http://coord.info/GC7N8XD}{[GTAQ10] 08 - Bayat} — \href{http://coord.info/GC7N8XD\Number{}767756135}{1480}}\cacheData{{2018/05/10 SPooks753, Traditional Cache (1.5/1.5)}}\begin{cacheText}Apres une longue pause pique nique la\Quoted{team 4}( pour les départements 40(Dune33), 64(Fleura64) \And{} 44(Fabilab) se lance sur ce sympathique circuit.Nous 

rencontrons à la cache 03 les Black\Underscore{}archeon et decidons de faire le circuit ensemble.Toutes les caches sont trouvées et le parcours est fort agréable (nous avons même mangé quelques cerises en passant).

Un grand merci spooks753 pour toutes les caches.\end{cacheText}

\cacheNumber{1481}\needspace{5\baselineskip}\cacheName{\href{http://coord.info/GC7N8XJ}{[GTAQ10] 09 - Bayat} — \href{http://coord.info/GC7N8XJ\Number{}767756923}{1481}}\cacheData{{2018/05/10 SPooks753, Traditional Cache (2/1.5)}}\begin{cacheText}Apres une longue pause pique nique la\Quoted{team 4}( pour les départements 40(Dune33), 64(Fleura64) \And{} 44(Fabilab) se lance sur ce sympathique circuit.Nous 

rencontrons à la cache 03 les Black\Underscore{}archeon et decidons de faire le circuit ensemble.Toutes les caches sont trouvées et le parcours est fort agréable (nous avons même mangé quelques cerises en passant).Un PF pour le super camouflage.

Un grand merci spooks753 pour toutes les caches.\end{cacheText}

\cacheNumber{1482}\needspace{5\baselineskip}\cacheName{\href{http://coord.info/GC7N8XR}{[GTAQ10] 10 - Bayat} — \href{http://coord.info/GC7N8XR\Number{}767757123}{1482}}\cacheData{{2018/05/10 SPooks753, Traditional Cache (1.5/1.5)}}\begin{cacheText}Apres une longue pause pique nique la\Quoted{team 4}( pour les départements 40(Dune33), 64(Fleura64) \And{} 44(Fabilab) se lance sur ce sympathique circuit.Nous 

rencontrons à la cache 03 les Black\Underscore{}archeon et decidons de faire le circuit ensemble.Toutes les caches sont trouvées et le parcours est fort agréable (nous avons même mangé quelques cerises en passant).

Un grand merci spooks753 pour toutes les caches.\end{cacheText}

\cacheNumber{1483}\needspace{5\baselineskip}\cacheName{\href{http://coord.info/GC7N8XV}{[GTAQ10] 11 - Bayat} — \href{http://coord.info/GC7N8XV\Number{}767757480}{1483}}\cacheData{{2018/05/10 SPooks753, Traditional Cache (1.5/1.5)}}\begin{cacheText}Apres une longue pause pique nique la\Quoted{team 4}( pour les départements 40(Dune33), 64(Fleura64) \And{} 44(Fabilab) se lance sur ce sympathique circuit.Nous 

rencontrons à la cache 03 les Black\Underscore{}archeon et decidons de faire le circuit ensemble.Toutes les caches sont trouvées et le parcours est fort agréable (nous avons même mangé quelques cerises en passant).

Un grand merci spooks753 pour toutes les caches.\end{cacheText}

\cacheNumber{1484}\needspace{5\baselineskip}\cacheName{\href{http://coord.info/GC7N8Y4}{[GTAQ10] 12 - Bayat} — \href{http://coord.info/GC7N8Y4\Number{}767757946}{1484}}\cacheData{{2018/05/10 SPooks753, Traditional Cache (1.5/1.5)}}\begin{cacheText}Apres une longue pause pique nique la\Quoted{team 4}( pour les départements 40(Dune33), 64(Fleura64) \And{} 44(Fabilab) se lance sur ce sympathique circuit.Nous 

rencontrons à la cache 03 les Black\Underscore{}archeon et decidons de faire le circuit ensemble.Toutes les caches sont trouvées et le parcours est fort agréable (nous avons même mangé quelques cerises en passant).

Un grand merci spooks753 pour toutes les caches.\end{cacheText}

\cacheNumber{1485}\needspace{5\baselineskip}\cacheName{\href{http://coord.info/GC7N8Y8}{[GTAQ10] 13 - Bayat} — \href{http://coord.info/GC7N8Y8\Number{}767758092}{1485}}\cacheData{{2018/05/10 SPooks753, Traditional Cache (1.5/1.5)}}\begin{cacheText}Apres une longue pause pique nique la\Quoted{team 4}( pour les départements 40(Dune33), 64(Fleura64) \And{} 44(Fabilab) se lance sur ce sympathique circuit.Nous 

rencontrons à la cache 03 les Black\Underscore{}archeon et decidons de faire le circuit ensemble.Toutes les caches sont trouvées et le parcours est fort agréable (nous avons même mangé quelques cerises en passant).

Un grand merci spooks753 pour toutes les caches.\end{cacheText}

\cacheNumber{1486}\needspace{5\baselineskip}\cacheName{\href{http://coord.info/GC7N8YC}{[GTAQ10] 14 - Bayat} — \href{http://coord.info/GC7N8YC\Number{}767758194}{1486}}\cacheData{{2018/05/10 SPooks753, Traditional Cache (1.5/1.5)}}\begin{cacheText}Apres une longue pause pique nique la\Quoted{team 4}( pour les départements 40(Dune33), 64(Fleura64) \And{} 44(Fabilab) se lance sur ce sympathique circuit.Nous 

rencontrons à la cache 03 les Black\Underscore{}archeon et decidons de faire le circuit ensemble.Toutes les caches sont trouvées et le parcours est fort agréable (nous avons même mangé quelques cerises en passant).

Un grand merci spooks753 pour toutes les caches.\end{cacheText}

\cacheNumber{1487}\needspace{5\baselineskip}\cacheName{\href{http://coord.info/GC7N8YH}{[GTAQ10] 15 - Bayat} — \href{http://coord.info/GC7N8YH\Number{}767758427}{1487}}\cacheData{{2018/05/10 SPooks753, Traditional Cache (1.5/1.5)}}\begin{cacheText}Apres une longue pause pique nique la\Quoted{team 4}( pour les départements 40(Dune33), 64(Fleura64) \And{} 44(Fabilab) se lance sur ce sympathique circuit.Nous 

rencontrons à la cache 03 les Black\Underscore{}archeon et decidons de faire le circuit ensemble.Toutes les caches sont trouvées et le parcours est fort agréable (nous avons même mangé quelques cerises en passant).

Un grand merci spooks753 pour toutes les caches. Un PF pour l'ensemble du circuit\end{cacheText}

\cacheNumber{1488}\needspace{5\baselineskip}\cacheName{\href{http://coord.info/GC7NVCP}{[GTAQ12] Night 3D} — \href{http://coord.info/GC7NVCP\Number{}767880901}{1488}}\cacheData{{2018/05/10 Calimero33, Unknown Cache (3/1.5)}}\begin{cacheText}Apres avoir parcouru Bayat et Saint Terre la fatigue se fait sentir. ..Nous optons donc avec Fabilab et Dune33 pour la night 3D qui semble moins longue et qui se déroule en ville. Arrivés au PZ il y a foule!!!!Equipés de lampes UV les chercheurs découvrent les indices et Domino50 décode les coordonnées. Direction la finale et la belle Boite qui est cachée dans un lieu quelque peu insolite. Nous loguons en groupe sous le pseudo Team Night. Merci Calimero33 pour la cache et un PF évidemment.\end{cacheText}

\cacheNumber{1489}\needspace{5\baselineskip}\cacheName{\href{http://coord.info/GC6HA70}{[33 TOUR] - SAINT EMILION} — \href{http://coord.info/GC6HA70\Number{}767982201}{1489}}\cacheData{{2018/05/11 Grim's, Traditional Cache (1.5/1.5)}}\begin{cacheText}Après le circuit de la bataille de Castillon, la pause déjeuner est très appréciée. Nous optons pour un petit circuit car la chaleur est étouffante et nous commençons à ressentir la fatigue. Direction Saint-Émilion. Nous commençons par la tradi du 33 tours. Guidés par le GPS nous trouvons la boîte et signons le plus discrètement possible car le village est envahi de moldus . Nous loguons sous le pseudo Team 4.Merci pour la cache.\end{cacheText}

\cacheNumber{1490}\needspace{5\baselineskip}\cacheName{\href{http://coord.info/GC7J92D}{GTAQ - SAISON 4 - Episode 2 : Les Vignobles} — \href{http://coord.info/GC7J92D\Number{}767900090}{1490}}\cacheData{{2018/05/11 GTAQ, Event Cache (1/1)}}\begin{cacheText}Deuxième jour du GTAQ, la team 4 est en pleine forme pour aller sur les chemins. Bataille de  Castillon et les petites histoires sont au programme. Le soleil est au rendez vous et la bonne humeur aussi!!!!Que de beaux paysages, que de belles rencontres!!!Le chou vert et le rougail saucisse clôturent une journée bien remplie. Merci à l'équipe GTAQ et aux poseurs pour cette excellente journée.\end{cacheText}

\cacheNumber{1491}\needspace{5\baselineskip}\cacheName{\href{http://coord.info/GC7M2W9}{[GTAQ-11] 01- Bataille de Castillon} — \href{http://coord.info/GC7M2W9\Number{}767924621}{1491}}\cacheData{{2018/05/11 papyyoyo, Traditional Cache (1.5/1.5)}}\begin{cacheText}Un grand soleil est annoncé :nous choisissons le circuit le plus long pour la matinée. Toutes les caches seront trouvées et nous avons adoré le parcours et les boites travaillées. La team 4 (Dune33 du 40,Fabilab du 44 et moi même du 64) choisit d'abord un emplacement pour le pique nique partagé du déjeuner. Voila pourquoi nous débutons par la 14.

Nous avons bien rigolé et fait de belles rencontres.

Merci SuperPP: il comprendra....

Un grand merci papyyoyo pour ce superbe parcours et pour le travail de pose.\end{cacheText}

\cacheNumber{1492}\needspace{5\baselineskip}\cacheName{\href{http://coord.info/GC7M2WB}{[GTAQ-11] 02 - Bataille de Castillon} — \href{http://coord.info/GC7M2WB\Number{}767924132}{1492}}\cacheData{{2018/05/11 papyyoyo, Traditional Cache (1.5/1.5)}}\begin{cacheText}Un grand soleil est annoncé :nous choisissons le circuit le plus long pour la matinée. Toutes les caches seront trouvées et nous avons adoré le parcours et les boites travaillées. La team 4 (Dune33 du 40,Fabilab du 44 et moi même du 64) choisit d'abord un emplacement pour le pique nique partagé du déjeuner. Voila pourquoi nous débutons par la 14.

Nous avons bien rigolé et fait de belles rencontres.

Un grand merci papyyoyo pour ce superbe parcours et pour le travail de pose.\end{cacheText}

\cacheNumber{1493}\needspace{5\baselineskip}\cacheName{\href{http://coord.info/GC7M2WE}{[GTAQ-11] 03 - Bataille de Castillon} — \href{http://coord.info/GC7M2WE\Number{}767923151}{1493}}\cacheData{{2018/05/11 papyyoyo, Traditional Cache (1.5/1.5)}}\begin{cacheText}Un grand soleil est annoncé :nous choisissons le circuit le plus long pour la matinée. Toutes les caches seront trouvées et nous avons adoré le parcours et les boites travaillées. La team 4 (Dune33 du 40,Fabilab du 44 et moi même du 64) choisit d'abord un emplacement pour le pique nique partagé du déjeuner. Voila pourquoi nous débutons par la 14.

Nous avons bien rigolé et fait de belles rencontres.

Un grand merci papyyoyo pour ce superbe parcours et pour le travail de pose.\end{cacheText}

\cacheNumber{1494}\needspace{5\baselineskip}\cacheName{\href{http://coord.info/GC7M2WK}{[GTAQ-11] 05 - Bataille de Castillon} — \href{http://coord.info/GC7M2WK\Number{}767919280}{1494}}\cacheData{{2018/05/11 papyyoyo, Traditional Cache (1.5/1.5)}}\begin{cacheText}Co(STF)

Un grand soleil est annoncé :nous choisissons le circuit le plus long pour la matinée. Toutes les caches seront trouvées et nous avons adoré le parcours et les boites travaillées. La team 4 (Dune33 du 40,Fabilab du 44 et moi même du 64) choisit d'abord un emplacement pour le pique nique partagé du déjeuner. Voila pourquoi nous débutons par la 14.

Nous avons bien rigolé et fait de belles rencontres.

Un grand merci papyyoyo pour ce superbe parcours et pour le travail de pose.\end{cacheText}

\cacheNumber{1495}\needspace{5\baselineskip}\cacheName{\href{http://coord.info/GC7M2WT}{[GTAQ-11] 06 - Bataille de Castillon} — \href{http://coord.info/GC7M2WT\Number{}767918459}{1495}}\cacheData{{2018/05/11 papyyoyo, Traditional Cache (1.5/1.5)}}\begin{cacheText}Co(FTF)

Un grand soleil est annoncé :nous choisissons le circuit le plus long pour la matinée. Toutes les caches seront trouvées et nous avons adoré le parcours et les boites travaillées. La team 4 (Dune33 du 40,Fabilab du 44 et moi même du 64) choisit d'abord un emplacement pour le pique nique partagé du déjeuner. Voila pourquoi nous débutons par la 14.

Nous avons bien rigolé et fait de belles rencontres.

Un grand merci papyyoyo pour ce superbe parcours et pour le travail de pose.\end{cacheText}

\cacheNumber{1496}\needspace{5\baselineskip}\cacheName{\href{http://coord.info/GC7M2WZ}{[GTAQ-11] 04 - Bataille de Castillon} — \href{http://coord.info/GC7M2WZ\Number{}767922754}{1496}}\cacheData{{2018/05/11 papyyoyo, Traditional Cache (1.5/1.5)}}\begin{cacheText}Co(STF)           :)1500:) 

Un grand soleil est annoncé :nous choisissons le circuit le plus long pour la matinée. Toutes les caches seront trouvées et nous avons adoré le parcours et les boites travaillées. La team 4 (Dune33 du 40,Fabilab du 44 et moi même du 64) choisit d'abord un emplacement pour le pique nique partagé du déjeuner. Voila pourquoi nous débutons par la 14.

Nous avons bien rigolé et fait de belles rencontres.

Un grand merci papyyoyo pour ce superbe parcours et pour le travail de pose. Un PF pour la Belle.\end{cacheText}

\cacheNumber{1497}\needspace{5\baselineskip}\cacheName{\href{http://coord.info/GC7M2X4}{[GTAQ-11] 07 - Bataille de Castillon} — \href{http://coord.info/GC7M2X4\Number{}767918255}{1497}}\cacheData{{2018/05/11 papyyoyo, Traditional Cache (1.5/3)}}\begin{cacheText}Co(FTF)

Un grand soleil est annoncé :nous choisissons le circuit le plus long pour la matinée. Toutes les caches seront trouvées et nous avons adoré le parcours et les boites travaillées. La team 4 (Dune33 du 40,Fabilab du 44 et moi même du 64) choisit d'abord un emplacement pour le pique nique partagé du déjeuner. Voila pourquoi nous débutons par la 14.

Nous avons bien rigolé et fait de belles rencontres.

Un grand merci papyyoyo pour ce superbe parcours et pour le travail de pose.\end{cacheText}

\cacheNumber{1498}\needspace{5\baselineskip}\cacheName{\href{http://coord.info/GC7M2X7}{[GTAQ-11] 08 - Bataille de Castillon} — \href{http://coord.info/GC7M2X7\Number{}767917874}{1498}}\cacheData{{2018/05/11 papyyoyo, Traditional Cache (1.5/1.5)}}\begin{cacheText}Co(FTF)

Un grand soleil est annoncé :nous choisissons le circuit le plus long pour la matinée. Toutes les caches seront trouvées et nous avons adoré le parcours et les boites travaillées. La team 4 (Dune33 du 40,Fabilab du 44 et moi même du 64) choisit d'abord un emplacement pour le pique nique partagé du déjeuner. Voila pourquoi nous débutons par la 14.

Nous avons bien rigolé et fait de belles rencontres.

Un grand merci papyyoyo pour ce superbe parcours et pour le travail de pose.\end{cacheText}

\cacheNumber{1499}\needspace{5\baselineskip}\cacheName{\href{http://coord.info/GC7M2XB}{[GTAQ-11] 09 - Bataille de Castillon} — \href{http://coord.info/GC7M2XB\Number{}767916823}{1499}}\cacheData{{2018/05/11 papyyoyo, Traditional Cache (1.5/3)}}\begin{cacheText}Co(FTF)

Un grand soleil est annoncé :nous choisissons le circuit le plus long pour la matinée. Toutes les caches seront trouvées et nous avons adoré le parcours et les boites travaillées. La team 4 (Dune33 du 40,Fabilab du 44 et moi même du 64) choisit d'abord un emplacement pour le pique nique partagé du déjeuner. Voila pourquoi nous débutons par la 14.

Nous avons bien rigolé et fait de belles rencontres.

Un grand merci papyyoyo pour ce superbe parcours et pour le travail de pose.\end{cacheText}

\cacheNumber{1500}\needspace{5\baselineskip}\cacheName{\href{http://coord.info/GC7M2XD}{[GTAQ-11] 10 - Bataille de Castillon} — \href{http://coord.info/GC7M2XD\Number{}767916453}{1500}}\cacheData{{2018/05/11 papyyoyo, Traditional Cache (1.5/1.5)}}\begin{cacheText}Co(FTF)

Un grand soleil est annoncé :nous choisissons le circuit le plus long pour la matinée. Toutes les caches seront trouvées et nous avons adoré le parcours et les boites travaillées. La team 4 (Dune33 du 40,Fabilab du 44 et moi même du 64) choisit d'abord un emplacement pour le pique nique partagé du déjeuner. Voila pourquoi nous débutons par la 14.

Nous avons bien rigolé et fait de belles rencontres.

Un grand merci papyyoyo pour ce superbe parcours et pour le travail de pose.\end{cacheText}

\cacheNumber{1501}\needspace{5\baselineskip}\cacheName{\href{http://coord.info/GC7M2XH}{[GTAQ-11] 11 - Bataille de Castillon} — \href{http://coord.info/GC7M2XH\Number{}767915792}{1501}}\cacheData{{2018/05/11 papyyoyo, Traditional Cache (1.5/1.5)}}\begin{cacheText}Co(FTF)

Un grand soleil est annoncé :nous choisissons le circuit le plus long pour la matinée. Toutes les caches seront trouvées et nous avons adoré le parcours et les boites travaillées. La team 4 (Dune33 du 40,Fabilab du 44 et moi même du 64) choisit d'abord un emplacement pour le pique nique partagé du déjeuner. Voila pourquoi nous débutons par la 14.

Nous avons bien rigolé et fait de belles rencontres.

Un grand merci papyyoyo pour ce superbe parcours et pour le travail de pose.\end{cacheText}

\cacheNumber{1502}\needspace{5\baselineskip}\cacheName{\href{http://coord.info/GC7M2Y4}{[GTAQ-11] 12 - Bataille de Castillon} — \href{http://coord.info/GC7M2Y4\Number{}767915520}{1502}}\cacheData{{2018/05/11 papyyoyo, Traditional Cache (2.5/1.5)}}\begin{cacheText}Co(FTF)

Un grand soleil est annoncé :nous choisissons le circuit le plus long pour la matinée. Toutes les caches seront trouvées et nous avons adoré le parcours et les boites travaillées. La team 4 (Dune33 du 40,Fabilab du 44 et moi même du 64) choisit d'abord un emplacement pour le pique nique partagé du déjeuner. Voila pourquoi nous débutons par la 14.

Nous avons bien rigolé et fait de belles rencontres.

Un grand merci papyyoyo pour ce superbe parcours et pour le travail de pose. Un PF pour l'intrus\end{cacheText}

\cacheNumber{1503}\needspace{5\baselineskip}\cacheName{\href{http://coord.info/GC7M2YB}{[GTAQ-11] 13 - Bataille de Castillon} — \href{http://coord.info/GC7M2YB\Number{}767914763}{1503}}\cacheData{{2018/05/11 papyyoyo, Traditional Cache (1.5/1.5)}}\begin{cacheText}Co(FTF)

Un grand soleil est annoncé :nous choisissons le circuit le plus long pour la matinée. Toutes les caches seront trouvées et nous avons adoré le parcours et les boites travaillées. La team 4 (Dune33 du 40,Fabilab du 44 et moi même du 64) choisit d'abord un emplacement pour le pique nique partagé du déjeuner. Voila pourquoi nous débutons par la 14.

Nous avons bien rigolé et fait de belles rencontres.

Un grand merci papyyoyo pour ce superbe parcours et pour le travail de pose.\end{cacheText}

\cacheNumber{1504}\needspace{5\baselineskip}\cacheName{\href{http://coord.info/GC7M2YD}{[GTAQ-11] 14 - Bataille de Castillon} — \href{http://coord.info/GC7M2YD\Number{}767913192}{1504}}\cacheData{{2018/05/11 papyyoyo, Traditional Cache (1.5/1.5)}}\begin{cacheText}Co(FTF)

Un grand soleil est annoncé :nous choisissons le circuit le plus long pour la matinée. Toutes les caches seront trouvées et nous avons adoré le parcours et les boites travaillées. La team 4 (Dune33 du 40,Fabilab du 44 et moi même du 64) choisit d'abord un emplacement pour le pique nique partagé du déjeuner. Voila pourquoi nous débutons par la 14.

Nous avons bien rigolé et fait de belles rencontres.

Un grand merci papyyoyo pour ce superbe parcours et pour le travail de pose.\end{cacheText}

\cacheNumber{1505}\needspace{5\baselineskip}\cacheName{\href{http://coord.info/GC7M2YG}{[GTAQ-11] 15 - Bataille de Castillon} — \href{http://coord.info/GC7M2YG\Number{}767938210}{1505}}\cacheData{{2018/05/11 papyyoyo, Traditional Cache (1.5/1.5)}}\begin{cacheText}Un grand soleil est annoncé :nous choisissons le circuit le plus long pour la matinée. Toutes les caches seront trouvées et nous avons adoré le parcours et les boites travaillées. La team 4 (Dune33 du 40,Fabilab du 44 et moi même du 64) choisit d'abord un emplacement pour le pique nique partagé du déjeuner. Voila pourquoi nous débutons par la 14.

Nous avons bien rigolé et fait de belles rencontres.

Un grand merci papyyoyo pour ce superbe parcours et pour le travail de pose. Un PF pour la bébête\end{cacheText}

\cacheNumber{1506}\needspace{5\baselineskip}\cacheName{\href{http://coord.info/GC7M2YJ}{[GTAQ-11] 16 - Bataille de Castillon} — \href{http://coord.info/GC7M2YJ\Number{}767937699}{1506}}\cacheData{{2018/05/11 papyyoyo, Traditional Cache (1.5/1.5)}}\begin{cacheText}Un grand soleil est annoncé :nous choisissons le circuit le plus long pour la matinée. Toutes les caches seront trouvées et nous avons adoré le parcours et les boites travaillées. La team 4 (Dune33 du 40,Fabilab du 44 et moi même du 64) choisit d'abord un emplacement pour le pique nique partagé du déjeuner. Voila pourquoi nous débutons par la 14.

Nous avons bien rigolé et fait de belles rencontres.

Un grand merci papyyoyo pour ce superbe parcours et pour le travail de pose.Un PF pour la découverte.\end{cacheText}

\cacheNumber{1507}\needspace{5\baselineskip}\cacheName{\href{http://coord.info/GC7M2YN}{[GTAQ-11] 17 - Bataille de Castillon} — \href{http://coord.info/GC7M2YN\Number{}767937079}{1507}}\cacheData{{2018/05/11 papyyoyo, Traditional Cache (1.5/1.5)}}\begin{cacheText}Un grand soleil est annoncé :nous choisissons le circuit le plus long pour la matinée. Toutes les caches seront trouvées et nous avons adoré le parcours et les boites travaillées. La team 4 (Dune33 du 40,Fabilab du 44 et moi même du 64) choisit d'abord un emplacement pour le pique nique partagé du déjeuner. Voila pourquoi nous débutons par la 14.

Nous avons bien rigolé et fait de belles rencontres.

Un grand merci papyyoyo pour ce superbe parcours et pour le travail de pose.\end{cacheText}

\cacheNumber{1508}\needspace{5\baselineskip}\cacheName{\href{http://coord.info/GC7M2YV}{[GTAQ-11] 18 - Bataille de Castillon} — \href{http://coord.info/GC7M2YV\Number{}767935580}{1508}}\cacheData{{2018/05/11 papyyoyo, Traditional Cache (3/1.5)}}\begin{cacheText}Un grand soleil est annoncé :nous choisissons le circuit le plus long pour la matinée. Toutes les caches seront trouvées et nous avons adoré le parcours et les boites travaillées. La team 4 (Dune33 du 40,Fabilab du 44 et moi même du 64) choisit d'abord un emplacement pour le pique nique partagé du déjeuner. Voila pourquoi nous débutons par la 14.

Nous avons bien rigolé et fait de belles rencontres.

Un grand merci papyyoyo pour ce superbe parcours et pour le travail de pose.Un PF pour l'intégration parfaite.\end{cacheText}

\cacheNumber{1509}\needspace{5\baselineskip}\cacheName{\href{http://coord.info/GC7M2YX}{[GTAQ-11] 19- Bataille de Castillon} — \href{http://coord.info/GC7M2YX\Number{}767934670}{1509}}\cacheData{{2018/05/11 papyyoyo, Traditional Cache (1.5/1.5)}}\begin{cacheText}Un grand soleil est annoncé :nous choisissons le circuit le plus long pour la matinée. Toutes les caches seront trouvées et nous avons adoré le parcours et les boites travaillées. La team 4 (Dune33 du 40,Fabilab du 44 et moi même du 64) choisit d'abord un emplacement pour le pique nique partagé du déjeuner. Voila pourquoi nous débutons par la 14.

Nous avons bien rigolé et fait de belles rencontres.

Un grand merci papyyoyo pour ce superbe parcours et pour le travail de pose.\end{cacheText}

\cacheNumber{1510}\needspace{5\baselineskip}\cacheName{\href{http://coord.info/GC7M2Z1}{[GTAQ-11] 20 - Bataille de Castillon} — \href{http://coord.info/GC7M2Z1\Number{}767933789}{1510}}\cacheData{{2018/05/11 papyyoyo, Traditional Cache (1.5/1.5)}}\begin{cacheText}Un grand soleil est annoncé :nous choisissons le circuit le plus long pour la matinée. Toutes les caches seront trouvées et nous avons adoré le parcours et les boites travaillées. La team 4 (Dune33 du 40,Fabilab du 44 et moi même du 64) choisit d'abord un emplacement pour le pique nique partagé du déjeuner. Voila pourquoi nous débutons par la 14.

Nous avons bien rigolé et fait de belles rencontres.

Un grand merci papyyoyo pour ce superbe parcours et pour le travail de pose. Et un PF supplémentaire.\end{cacheText}

\cacheNumber{1511}\needspace{5\baselineskip}\cacheName{\href{http://coord.info/GC7M2Z2}{[GTAQ-11] 21 - Bataille de Castillon} — \href{http://coord.info/GC7M2Z2\Number{}767933106}{1511}}\cacheData{{2018/05/11 papyyoyo, Traditional Cache (1.5/1.5)}}\begin{cacheText}Un grand soleil est annoncé :nous choisissons le circuit le plus long pour la matinée. Toutes les caches seront trouvées et nous avons adoré le parcours et les boites travaillées. La team 4 (Dune33 du 40,Fabilab du 44 et moi même du 64) choisit d'abord un emplacement pour le pique nique partagé du déjeuner. Voila pourquoi nous débutons par la 14.

Nous avons bien rigolé et fait de belles rencontres.

Un grand merci papyyoyo pour ce superbe parcours et pour le travail de pose.\end{cacheText}

\cacheNumber{1512}\needspace{5\baselineskip}\cacheName{\href{http://coord.info/GC7M2Z6}{[GTAQ-11] 22 - Bataille de Castillon} — \href{http://coord.info/GC7M2Z6\Number{}767932750}{1512}}\cacheData{{2018/05/11 papyyoyo, Traditional Cache (2.5/1.5)}}\begin{cacheText}Un grand soleil est annoncé :nous choisissons le circuit le plus long pour la matinée. Toutes les caches seront trouvées et nous avons adoré le parcours et les boites travaillées. La team 4 (Dune33 du 40,Fabilab du 44 et moi même du 64) choisit d'abord un emplacement pour le pique nique partagé du déjeuner. Voila pourquoi nous débutons par la 14.

Nous avons bien rigolé et fait de belles rencontres.

Un grand merci papyyoyo pour ce superbe parcours et pour le travail de pose.\end{cacheText}

\cacheNumber{1513}\needspace{5\baselineskip}\cacheName{\href{http://coord.info/GC7M2Z8}{[GTAQ-11] 23 - Bataille de Castillon} — \href{http://coord.info/GC7M2Z8\Number{}767931480}{1513}}\cacheData{{2018/05/11 papyyoyo, Traditional Cache (1.5/1.5)}}\begin{cacheText}Un grand soleil est annoncé :nous choisissons le circuit le plus long pour la matinée. Toutes les caches seront trouvées et nous avons adoré le parcours et les boites travaillées. La team 4 (Dune33 du 40,Fabilab du 44 et moi même du 64) choisit d'abord un emplacement pour le pique nique partagé du déjeuner. Voila pourquoi nous débutons par la 14.

Nous avons bien rigolé et fait de belles rencontres.

Un grand merci papyyoyo pour ce superbe parcours et pour le travail de pose.\end{cacheText}

\cacheNumber{1514}\needspace{5\baselineskip}\cacheName{\href{http://coord.info/GC7M2ZB}{[GTAQ-11] 24 - Bataille de Castillon} — \href{http://coord.info/GC7M2ZB\Number{}767930036}{1514}}\cacheData{{2018/05/11 papyyoyo, Traditional Cache (1.5/1.5)}}\begin{cacheText}Un grand soleil est annoncé :nous choisissons le circuit le plus long pour la matinée. Toutes les caches seront trouvées et nous avons adoré le parcours et les boites travaillées. La team 4 (Dune33 du 40,Fabilab du 44 et moi même du 64) choisit d'abord un emplacement pour le pique nique partagé du déjeuner. Voila pourquoi nous débutons par la 14.

Nous avons bien rigolé et fait de belles rencontres.

Un grand merci papyyoyo pour ce superbe parcours et pour le travail de pose.\end{cacheText}

\cacheNumber{1515}\needspace{5\baselineskip}\cacheName{\href{http://coord.info/GC7M2ZC}{[GTAQ-11] 25 - Bataille de Castillon} — \href{http://coord.info/GC7M2ZC\Number{}767929707}{1515}}\cacheData{{2018/05/11 papyyoyo, Traditional Cache (2.5/1.5)}}\begin{cacheText}Un grand soleil est annoncé :nous choisissons le circuit le plus long pour la matinée. Toutes les caches seront trouvées et nous avons adoré le parcours et les boites travaillées. La team 4 (Dune33 du 40,Fabilab du 44 et moi même du 64) choisit d'abord un emplacement pour le pique nique partagé du déjeuner. Voila pourquoi nous débutons par la 14.

Nous avons bien rigolé et fait de belles rencontres.

Un grand merci papyyoyo pour ce superbe parcours et pour le travail de pose.\end{cacheText}

\cacheNumber{1516}\needspace{5\baselineskip}\cacheName{\href{http://coord.info/GC7M2ZJ}{[GTAQ-11] 26 - Bataille de Castillon} — \href{http://coord.info/GC7M2ZJ\Number{}767928844}{1516}}\cacheData{{2018/05/11 papyyoyo, Traditional Cache (1.5/1.5)}}\begin{cacheText}Un grand soleil est annoncé :nous choisissons le circuit le plus long pour la matinée. Toutes les caches seront trouvées et nous avons adoré le parcours et les boites travaillées. La team 4 (Dune33 du 40,Fabilab du 44 et moi même du 64) choisit d'abord un emplacement pour le pique nique partagé du déjeuner. Voila pourquoi nous débutons par la 14.

Nous avons bien rigolé et fait de belles rencontres.

Un grand merci papyyoyo pour ce superbe parcours et pour le travail de pose. Un PF pour celui qui dore au pied du noyer\end{cacheText}

\cacheNumber{1517}\needspace{5\baselineskip}\cacheName{\href{http://coord.info/GC7M2ZN}{[GTAQ-11] 27 - Bataille de Castillon} — \href{http://coord.info/GC7M2ZN\Number{}767928412}{1517}}\cacheData{{2018/05/11 papyyoyo, Traditional Cache (2.5/1.5)}}\begin{cacheText}Un grand soleil est annoncé :nous choisissons le circuit le plus long pour la matinée. Toutes les caches seront trouvées et nous avons adoré le parcours et les boites travaillées. La team 4 (Dune33 du 40,Fabilab du 44 et moi même du 64) choisit d'abord un emplacement pour le pique nique partagé du déjeuner. Voila pourquoi nous débutons par la 14.

Nous avons bien rigolé et fait de belles rencontres.

Un grand merci papyyoyo pour ce superbe parcours et pour le travail de pose.\end{cacheText}

\cacheNumber{1518}\needspace{5\baselineskip}\cacheName{\href{http://coord.info/GC7M2ZW}{[GTAQ-11] 28 - Bataille de Castillon} — \href{http://coord.info/GC7M2ZW\Number{}767927990}{1518}}\cacheData{{2018/05/11 papyyoyo, Traditional Cache (2/1.5)}}\begin{cacheText}Un grand soleil est annoncé :nous choisissons le circuit le plus long pour la matinée. Toutes les caches seront trouvées et nous avons adoré le parcours et les boites travaillées. La team 4 (Dune33 du 40,Fabilab du 44 et moi même du 64) choisit d'abord un emplacement pour le pique nique partagé du déjeuner. Voila pourquoi nous débutons par la 14.

Nous avons bien rigolé et fait de belles rencontres.

Un grand merci papyyoyo pour ce superbe parcours et pour le travail de pose.\end{cacheText}

\cacheNumber{1519}\needspace{5\baselineskip}\cacheName{\href{http://coord.info/GC7M304}{[GTAQ-11] 29 - Bataille de Castillon} — \href{http://coord.info/GC7M304\Number{}767927591}{1519}}\cacheData{{2018/05/11 papyyoyo, Traditional Cache (1.5/1.5)}}\begin{cacheText}Un grand soleil est annoncé :nous choisissons le circuit le plus long pour la matinée. Toutes les caches seront trouvées et nous avons adoré le parcours et les boites travaillées. La team 4 (Dune33 du 40,Fabilab du 44 et moi même du 64) choisit d'abord un emplacement pour le pique nique partagé du déjeuner. Voila pourquoi nous débutons par la 14.

Nous avons bien rigolé et fait de belles rencontres.

Un grand merci papyyoyo pour ce superbe parcours et pour le travail de pose.\end{cacheText}

\cacheNumber{1520}\needspace{5\baselineskip}\cacheName{\href{http://coord.info/GC7M308}{[GTAQ-11] 30 - Bataille de Castillon} — \href{http://coord.info/GC7M308\Number{}767927296}{1520}}\cacheData{{2018/05/11 papyyoyo, Traditional Cache (1.5/1.5)}}\begin{cacheText}Un grand soleil est annoncé :nous choisissons le circuit le plus long pour la matinée. Toutes les caches seront trouvées et nous avons adoré le parcours et les boites travaillées. La team 4 (Dune33 du 40,Fabilab du 44 et moi même du 64) choisit d'abord un emplacement pour le pique nique partagé du déjeuner. Voila pourquoi nous débutons par la 14.

Nous avons bien rigolé et fait de belles rencontres.

Un grand merci papyyoyo pour ce superbe parcours et pour le travail de pose.Un PF pour la belle découverte.\end{cacheText}

\cacheNumber{1521}\needspace{5\baselineskip}\cacheName{\href{http://coord.info/GC7M30B}{[GTAQ11] 31 -  Bataille de Castillon} — \href{http://coord.info/GC7M30B\Number{}767926818}{1521}}\cacheData{{2018/05/11 papyyoyo, Traditional Cache (2/1.5)}}\begin{cacheText}Un grand soleil est annoncé :nous choisissons le circuit le plus long pour la matinée. Toutes les caches seront trouvées et nous avons adoré le parcours et les boites travaillées. La team 4 (Dune33 du 40,Fabilab du 44 et moi même du 64) choisit d'abord un emplacement pour le pique nique partagé du déjeuner. Voila pourquoi nous débutons par la 14.

Nous avons bien rigolé et fait de belles rencontres.

Un grand merci papyyoyo pour ce superbe parcours et pour le travail de pose.\end{cacheText}

\cacheNumber{1522}\needspace{5\baselineskip}\cacheName{\href{http://coord.info/GC7M30F}{[GTAQ-11] 32 - Bataille de Castillon} — \href{http://coord.info/GC7M30F\Number{}767925744}{1522}}\cacheData{{2018/05/11 papyyoyo, Traditional Cache (3/1.5)}}\begin{cacheText}Un grand soleil est annoncé :nous choisissons le circuit le plus long pour la matinée. Toutes les caches seront trouvées et nous avons adoré le parcours et les boites travaillées. La team 4 (Dune33 du 40,Fabilab du 44 et moi même du 64) choisit d'abord un emplacement pour le pique nique partagé du déjeuner. Voila pourquoi nous débutons par la 14.

Nous avons bien rigolé et fait de belles rencontres.

Un grand merci papyyoyo pour ce superbe parcours et pour le travail de pose.Un PF pour orange et blanc.\end{cacheText}

\cacheNumber{1523}\needspace{5\baselineskip}\cacheName{\href{http://coord.info/GC7M30J}{[GTAQ-11] 33 - Bataille de Castillon} — \href{http://coord.info/GC7M30J\Number{}767925099}{1523}}\cacheData{{2018/05/11 papyyoyo, Traditional Cache (2/1.5)}}\begin{cacheText}Un grand soleil est annoncé :nous choisissons le circuit le plus long pour la matinée. Toutes les caches seront trouvées et nous avons adoré le parcours et les boites travaillées. La team 4 (Dune33 du 40,Fabilab du 44 et moi même du 64) choisit d'abord un emplacement pour le pique nique partagé du déjeuner. Voila pourquoi nous débutons par la 14.

Nous avons bien rigolé et fait de belles rencontres.

Un grand merci papyyoyo pour ce superbe parcours et pour le travail de pose.\end{cacheText}

\cacheNumber{1524}\needspace{5\baselineskip}\cacheName{\href{http://coord.info/GC7M30M}{[GTAQ-11] 34 - Bataille de Castillon} — \href{http://coord.info/GC7M30M\Number{}767924932}{1524}}\cacheData{{2018/05/11 papyyoyo, Traditional Cache (1.5/1.5)}}\begin{cacheText}Un grand soleil est annoncé :nous choisissons le circuit le plus long pour la matinée. Toutes les caches seront trouvées et nous avons adoré le parcours et les boites travaillées. La team 4 (Dune33 du 40,Fabilab du 44 et moi même du 64) choisit d'abord un emplacement pour le pique nique partagé du déjeuner. Voila pourquoi nous débutons par la 14.

Nous avons bien rigolé et fait de belles rencontres.

Un grand merci papyyoyo pour ce superbe parcours et pour le travail de pose.\end{cacheText}

\cacheNumber{1525}\needspace{5\baselineskip}\cacheName{\href{http://coord.info/GC7NP30}{[GTAQ11] 02 - Les petites histoire} — \href{http://coord.info/GC7NP30\Number{}768169668}{1525}}\cacheData{{2018/05/11 tichivi, Traditional Cache (2/2.5)}}\begin{cacheText}Après le circuit de la bataille de Castillon, la pause déjeuner est très appréciée. Nous optons pour un petit circuit car la chaleur est étouffante et nous commençons à ressentir la fatigue. Direction Saint-Émilion...La balade est très agréable malgré la chaleur: des rencontres sympas, de gros fou rire et de très belles réalisations. Que du bonheur !!!!! . Nous loguons sous le pseudo Team 4 (Dune33 du 40,Fabilab du 44 et nous du 64).

Très belle cache parfaitement intégrée dans le décor: un PF.

Merci pour la cache.\end{cacheText}

\cacheNumber{1526}\needspace{5\baselineskip}\cacheName{\href{http://coord.info/GC7NP32}{[GTAQ11] 03 - Les petites histoires} — \href{http://coord.info/GC7NP32\Number{}768169760}{1526}}\cacheData{{2018/05/11 tichivi, Traditional Cache (2/2.5)}}\begin{cacheText}Après le circuit de la bataille de Castillon, la pause déjeuner est très appréciée. Nous optons pour un petit circuit car la chaleur est étouffante et nous commençons à ressentir la fatigue. Direction Saint-Émilion...La balade est très agréable malgré la chaleur: des rencontres sympas, de gros fou rire et de très belles réalisations. Que du bonheur !!!!! . Nous loguons sous le pseudo Team 4 (Dune33 du 40,Fabilab du 44 et nous du 64).

Merci porla cache.\end{cacheText}

\cacheNumber{1527}\needspace{5\baselineskip}\cacheName{\href{http://coord.info/GC7NQ70}{[GTAQ11] 04 - Les petites histoires} — \href{http://coord.info/GC7NQ70\Number{}768170079}{1527}}\cacheData{{2018/05/11 tichivi, Traditional Cache (2/2)}}\begin{cacheText}Après le circuit de la bataille de Castillon, la pause déjeuner est très appréciée. Nous optons pour un petit circuit car la chaleur est étouffante et nous commençons à ressentir la fatigue. Direction Saint-Émilion...La balade est très agréable malgré la chaleur: des rencontres sympas, de gros fou rire et de très belles réalisations. Que du bonheur !!!!! . Nous loguons sous le pseudo Team 4 (Dune33 du 40,Fabilab du 44 et nous du 64).

Un magnifique champ de 💐 se trouve à coté de la cache. Impossible de partir sans faire quelques photos. Merci pour la cache\end{cacheText}

\cacheNumber{1528}\needspace{5\baselineskip}\cacheName{\href{http://coord.info/GC7NQ74}{[GTAQ11] 05 - Les petites histoires} — \href{http://coord.info/GC7NQ74\Number{}768170136}{1528}}\cacheData{{2018/05/11 tichivi, Traditional Cache (2/2)}}\begin{cacheText}Après le circuit de la bataille de Castillon, la pause déjeuner est très appréciée. Nous optons pour un petit circuit car la chaleur est étouffante et nous commençons à ressentir la fatigue. Direction Saint-Émilion...La balade est très agréable malgré la chaleur: des rencontres sympas, de gros fou rire et de très belles réalisations. Que du bonheur !!!!! . Nous loguons sous le pseudo Team 4 (Dune33 du 40,Fabilab du 44 et nous du 64).Merci pour la cache.\end{cacheText}

\cacheNumber{1529}\needspace{5\baselineskip}\cacheName{\href{http://coord.info/GC7NQ79}{[GTAQ11] 06 - Les petites histoires} — \href{http://coord.info/GC7NQ79\Number{}768170237}{1529}}\cacheData{{2018/05/11 tichivi, Traditional Cache (2/2)}}\begin{cacheText}Après le circuit de la bataille de Castillon, la pause déjeuner est très appréciée. Nous optons pour un petit circuit car la chaleur est étouffante et nous commençons à ressentir la fatigue. Direction Saint-Émilion...La balade est très agréable malgré la chaleur: des rencontres sympas, de gros fou rire et de très belles réalisations. Que du bonheur !!!!! . Nous loguons sous le pseudo Team 4 (Dune33 du 40,Fabilab du 44 et nous du 64).Merci pour la cache.\end{cacheText}

\cacheNumber{1530}\needspace{5\baselineskip}\cacheName{\href{http://coord.info/GC7NQ7C}{[GTAQ11] 07 - Les petites histoires} — \href{http://coord.info/GC7NQ7C\Number{}768170276}{1530}}\cacheData{{2018/05/11 tichivi, Traditional Cache (1.5/2)}}\begin{cacheText}Après le circuit de la bataille de Castillon, la pause déjeuner est très appréciée. Nous optons pour un petit circuit car la chaleur est étouffante et nous commençons à ressentir la fatigue. Direction Saint-Émilion...La balade est très agréable malgré la chaleur: des rencontres sympas, de gros fou rire et de très belles réalisations. Que du bonheur !!!!! . Nous loguons sous le pseudo Team 4 (Dune33 du 40,Fabilab du 44 et nous du 64).Merci pour la cache.\end{cacheText}

\cacheNumber{1531}\needspace{5\baselineskip}\cacheName{\href{http://coord.info/GC7NQ7Y}{[GTAQ11] 08 - Les petites histoires} — \href{http://coord.info/GC7NQ7Y\Number{}768170313}{1531}}\cacheData{{2018/05/11 tichivi, Traditional Cache (1.5/2)}}\begin{cacheText}Après le circuit de la bataille de Castillon, la pause déjeuner est très appréciée. Nous optons pour un petit circuit car la chaleur est étouffante et nous commençons à ressentir la fatigue. Direction Saint-Émilion...La balade est très agréable malgré la chaleur: des rencontres sympas, de gros fou rire et de très belles réalisations. Que du bonheur !!!!! . Nous loguons sous le pseudo Team 4 (Dune33 du 40,Fabilab du 44 et nous du 64).Merci pour la cache.\end{cacheText}

\cacheNumber{1532}\needspace{5\baselineskip}\cacheName{\href{http://coord.info/GC7NQ81}{[GTAQ11] 09 - Les petites histoires} — \href{http://coord.info/GC7NQ81\Number{}768170367}{1532}}\cacheData{{2018/05/11 tichivi, Traditional Cache (2/2)}}\begin{cacheText}Après le circuit de la bataille de Castillon, la pause déjeuner est très appréciée. Nous optons pour un petit circuit car la chaleur est étouffante et nous commençons à ressentir la fatigue. Direction Saint-Émilion...La balade est très agréable malgré la chaleur: des rencontres sympas, de gros fou rire et de très belles réalisations. Que du bonheur !!!!! . Nous loguons sous le pseudo Team 4 (Dune33 du 40,Fabilab du 44 et nous du 64).Merci pour la cache.\end{cacheText}

\cacheNumber{1533}\needspace{5\baselineskip}\cacheName{\href{http://coord.info/GC7NQ84}{[GTAQ11] 10 - Les petites histoires} — \href{http://coord.info/GC7NQ84\Number{}768170412}{1533}}\cacheData{{2018/05/11 tichivi, Traditional Cache (2/2.5)}}\begin{cacheText}Après le circuit de la bataille de Castillon, la pause déjeuner est très appréciée. Nous optons pour un petit circuit car la chaleur est étouffante et nous commençons à ressentir la fatigue. Direction Saint-Émilion...La balade est très agréable malgré la chaleur: des rencontres sympas, de gros fou rire et de très belles réalisations. Que du bonheur !!!!! . Nous loguons sous le pseudo Team 4 (Dune33 du 40,Fabilab du 44 et nous du 64).Merci pour la cache.\end{cacheText}

\cacheNumber{1534}\needspace{5\baselineskip}\cacheName{\href{http://coord.info/GC7NQ88}{[GTAQ11] 11 - Les petites histoires} — \href{http://coord.info/GC7NQ88\Number{}768170470}{1534}}\cacheData{{2018/05/11 tichivi, Traditional Cache (1.5/2.5)}}\begin{cacheText}Après le circuit de la bataille de Castillon, la pause déjeuner est très appréciée. Nous optons pour un petit circuit car la chaleur est étouffante et nous commençons à ressentir la fatigue. Direction Saint-Émilion...La balade est très agréable malgré la chaleur: des rencontres sympas, de gros fou rire et de très belles réalisations. Que du bonheur !!!!! . Nous loguons sous le pseudo Team 4 (Dune33 du 40,Fabilab du 44 et nous du 64).Merci pour la cache.\end{cacheText}

\cacheNumber{1535}\needspace{5\baselineskip}\cacheName{\href{http://coord.info/GC7NQ8B}{[GTAQ11] 12 - Les petites histoires} — \href{http://coord.info/GC7NQ8B\Number{}768170506}{1535}}\cacheData{{2018/05/11 tichivi, Traditional Cache (2/2.5)}}\begin{cacheText}Après le circuit de la bataille de Castillon, la pause déjeuner est très appréciée. Nous optons pour un petit circuit car la chaleur est étouffante et nous commençons à ressentir la fatigue. Direction Saint-Émilion...La balade est très agréable malgré la chaleur: des rencontres sympas, de gros fou rire et de très belles réalisations. Que du bonheur !!!!! . Nous loguons sous le pseudo Team 4 (Dune33 du 40,Fabilab du 44 et nous du 64).Merci pour la cache.\end{cacheText}

\cacheNumber{1536}\needspace{5\baselineskip}\cacheName{\href{http://coord.info/GC7NQ8D}{[GTAQ11] 13 - Les petites histoires} — \href{http://coord.info/GC7NQ8D\Number{}768170544}{1536}}\cacheData{{2018/05/11 tichivi, Traditional Cache (2/3)}}\begin{cacheText}Après le circuit de la bataille de Castillon, la pause déjeuner est très appréciée. Nous optons pour un petit circuit car la chaleur est étouffante et nous commençons à ressentir la fatigue. Direction Saint-Émilion...La balade est très agréable malgré la chaleur: des rencontres sympas, de gros fou rire et de très belles réalisations. Que du bonheur !!!!! . Nous loguons sous le pseudo Team 4 (Dune33 du 40,Fabilab du 44 et nous du 64).Merci pour la cache.\end{cacheText}

\cacheNumber{1537}\needspace{5\baselineskip}\cacheName{\href{http://coord.info/GC7NQ8H}{[GTAQ11] 14 - Les petites histoires} — \href{http://coord.info/GC7NQ8H\Number{}768170615}{1537}}\cacheData{{2018/05/11 tichivi, Traditional Cache (1.5/2)}}\begin{cacheText}Après le circuit de la bataille de Castillon, la pause déjeuner est très appréciée. Nous optons pour un petit circuit car la chaleur est étouffante et nous commençons à ressentir la fatigue. Direction Saint-Émilion...La balade est très agréable malgré la chaleur: des rencontres sympas, de gros fou rire et de très belles réalisations. Que du bonheur !!!!! . Nous loguons sous le pseudo Team 4 (Dune33 du 40,Fabilab du 44 et nous du 64).Merci pour la cache.\end{cacheText}

\cacheNumber{1538}\needspace{5\baselineskip}\cacheName{\href{http://coord.info/GC7NQ8M}{[GTAQ11] 15 - Les petites histoires} — \href{http://coord.info/GC7NQ8M\Number{}768170672}{1538}}\cacheData{{2018/05/11 tichivi, Traditional Cache (1.5/2)}}\begin{cacheText}Après le circuit de la bataille de Castillon, la pause déjeuner est très appréciée. Nous optons pour un petit circuit car la chaleur est étouffante et nous commençons à ressentir la fatigue. Direction Saint-Émilion...La balade est très agréable malgré la chaleur: des rencontres sympas, de gros fou rire et de très belles réalisations. Que du bonheur !!!!! . Nous loguons sous le pseudo Team 4 (Dune33 du 40,Fabilab du 44 et nous du 64).Merci pour la cache.\end{cacheText}

\cacheNumber{1539}\needspace{5\baselineskip}\cacheName{\href{http://coord.info/GC7NQ8Q}{[GTAQ11] 16 - Les petites histoires} — \href{http://coord.info/GC7NQ8Q\Number{}768170724}{1539}}\cacheData{{2018/05/11 tichivi, Traditional Cache (1.5/2)}}\begin{cacheText}Après le circuit de la bataille de Castillon, la pause déjeuner est très appréciée. Nous optons pour un petit circuit car la chaleur est étouffante et nous commençons à ressentir la fatigue. Direction Saint-Émilion...La balade est très agréable malgré la chaleur: des rencontres sympas, de gros fou rire et de très belles réalisations. Que du bonheur !!!!! . Nous loguons sous le pseudo Team 4 (Dune33 du 40,Fabilab du 44 et nous du 64).Merci pour la cache.\end{cacheText}

\cacheNumber{1540}\needspace{5\baselineskip}\cacheName{\href{http://coord.info/GC7NQ8V}{[GTAQ11] 17 - Les petites histoires} — \href{http://coord.info/GC7NQ8V\Number{}768170974}{1540}}\cacheData{{2018/05/11 tichivi, Traditional Cache (3/2)}}\begin{cacheText}Après le circuit de la bataille de Castillon, la pause déjeuner est très appréciée. Nous optons pour un petit circuit car la chaleur est étouffante et nous commençons à ressentir la fatigue. Direction Saint-Émilion...La balade est très agréable malgré la chaleur: des rencontres sympas, de gros fou rire et de très belles réalisations. Que du bonheur !!!!! . Nous loguons sous le pseudo Team 4 (Dune33 du 40,Fabilab du 44 et nous du 64).

Un PF pour l'ensemble du super circuit.

Merci pour la cache.\end{cacheText}

\cacheNumber{1541}\needspace{5\baselineskip}\cacheName{\href{http://coord.info/GC7NRZB}{[GTAQ11] 1-Les vins bios : le chateau Canon} — \href{http://coord.info/GC7NRZB\Number{}768170922}{1541}}\cacheData{{2018/05/11 GAVbrioche33, Traditional Cache (1/1.5)}}\begin{cacheText}Après le circuit de la bataille de Castillon, la pause déjeuner est très appréciée. Nous optons pour un petit circuit car la chaleur est étouffante et nous commençons à ressentir la fatigue. Direction Saint-Émilion...La balade est très agréable malgré la chaleur: des rencontres sympas, de gros fou rire et de très belles réalisations. Que du bonheur !!!!! . Nous loguons sous le pseudo Team 4 (Dune33 du 40,Fabilab du 44 et nous du 64).

Nous n'aurons pas le temps de faire le parcours des vins bio mais cette cache est sur le passage....Merci pour la cache.\end{cacheText}

\cacheNumber{1542}\needspace{5\baselineskip}\cacheName{\href{http://coord.info/GC68WC6}{Pont de Tranchard} — \href{http://coord.info/GC68WC6\Number{}768228871}{1542}}\cacheData{{2018/05/12 chabert27, Traditional Cache (1.5/1.5)}}\begin{cacheText}C’est au troisième jour du GTAQ sur le retour, que nous nous arrêtons pour déloger la belle. La Nano est très bien cachée. Un peu de difficulté pour sortir le logbook mais il ne nous résiste pas longtemps. Merci pour la cache.\end{cacheText}

\cacheNumber{1543}\needspace{5\baselineskip}\cacheName{\href{http://coord.info/GC774XY}{Castillon: les quais et le lavoir} — \href{http://coord.info/GC774XY\Number{}768228350}{1543}}\cacheData{{2018/05/12 chabert27, Traditional Cache (1.5/1.5)}}\begin{cacheText}C’est au retour de la troisième journée du GTAQ 4 que nous nous arrêtons chercher la belle. Pas de soucis pour la déloger. Merci pour la cache.\end{cacheText}

\cacheNumber{1544}\needspace{5\baselineskip}\cacheName{\href{http://coord.info/GC774YA}{Castillon: les quais et les fontaines} — \href{http://coord.info/GC774YA\Number{}768228616}{1544}}\cacheData{{2018/05/12 chabert27, Traditional Cache (1.5/1.5)}}\begin{cacheText}C’est toujours sous la pluie que nous continuons notre quête. Au retour du troisième jour du GTAQ 4 on s’arrête pour déloger la belle. Merci pour la cache.\end{cacheText}

\cacheNumber{1545}\needspace{5\baselineskip}\cacheName{\href{http://coord.info/GC7J92E}{GTAQ - SAISON 4 - Episode 3 : Les Patrimoines} — \href{http://coord.info/GC7J92E\Number{}768172209}{1545}}\cacheData{{2018/05/12 GTAQ, Event Cache (1/1)}}\begin{cacheText}Troisième jour du GTAQ4 et la bonne humeur est toujours de la partie. Le temps est à la pluie, nous décidons  donc de visiter la région ,notamment Saint Emilion avec sa wherigo (une première pour la team 4).Comme toujours ,le chou vert et le repas sont parfaits. L'organisation est au top et l'annonce du GTAQ5 fait l'unanimité. Un immense merci à toute l'équipe GTAQ;\end{cacheText}

\cacheNumber{1546}\needspace{5\baselineskip}\cacheName{\href{http://coord.info/GC7NQD1}{[GTAQ9H12] 01 - Initiation} — \href{http://coord.info/GC7NQD1\Number{}768229231}{1546}}\cacheData{{2018/05/12 Calimero33, Traditional Cache (2/1.5)}}\begin{cacheText}C’est lors de la Night cache du troisième soir que nous découvrons la Belle. Merci pour la cache.\end{cacheText}

\cacheNumber{1547}\needspace{5\baselineskip}\cacheName{\href{http://coord.info/GC7P13N}{[GTAQ11] I ♥ Saint-Emilion} — \href{http://coord.info/GC7P13N\Number{}768204780}{1547}}\cacheData{{2018/05/12 Calimero33, Wherigo Cache (3/2)}}\begin{cacheText}Après le petit déjeuner pris au QG du GTAQ direction Saint Emilion pour découvrir le patrimoine grâce à la wherigo. C' est notre première et nous adorons le concept. La découverte de la ville est très ludique et nous admirons de nombreux trésors. Nous loguons Team4 .Un grand merci Calimero33 pour ce grand moment et un PF.\end{cacheText}

\cacheNumber{1548}\needspace{5\baselineskip}\cacheName{\href{http://coord.info/GC4XDTD}{A63 - Aire de repos de Lugos Ouest} — \href{http://coord.info/GC4XDTD\Number{}768231728}{1548}}\cacheData{{2018/05/13 gilles64, Traditional Cache (1.5/1.5)}}\begin{cacheText}C’est sur le retour du GTAQ 4 en gironde que nous nous arrêtons faire une petite halte. Quelle chance ...une cache!!!! . Pas de difficulté pour la déloger la Belle et personne à l'horizon pour loguer.. Merci Gilles.\end{cacheText}

\cacheNumber{1549}\needspace{5\baselineskip}\cacheName{\href{http://coord.info/GC7J92G}{GTAQ - SAISON 4 - Episode 4 : Les Amis} — \href{http://coord.info/GC7J92G\Number{}768230835}{1549}}\cacheData{{2018/05/13 GTAQ, Event Cache (1/1)}}\begin{cacheText}Dernier jour du GTAQ4 ....Nous avons adoré cette édition et nous en redemandons!!!!Tout était parfait. Un grand merci aux organisateurs et aux poseurs pour tout le travail accompli. Vive le GTAQ....\end{cacheText}

\cacheNumber{1550}\needspace{5\baselineskip}\cacheName{\href{http://coord.info/GC7NQDC}{[GTAQ9H12] 02 - Initiation} — \href{http://coord.info/GC7NQDC\Number{}768231375}{1550}}\cacheData{{2018/05/13 Calimero33, Traditional Cache (3/1.5)}}\begin{cacheText}Dernière cache avant de reprendre la route ...La cache est rapidement trouvée. Merci\end{cacheText}

\cacheNumber{1551}\needspace{5\baselineskip}\cacheName{\href{http://coord.info/GC5E6YX}{[BSD] \Number{}076} — \href{http://coord.info/GC5E6YX\Number{}770913940}{1551}}\cacheData{{2018/05/23 GéoLandesTour, Traditional Cache (1.5/1.5)}}\begin{cacheText}C’est la sortie du boulot. J’ai le temps de faire quelques caches avant de rejoindre l’Event Terra Aventura à l’OT de Dax. Je continue donc la boucle que j’avais entamée il y a quelques mois. Grâce à l’indice la Belle est rapidement trouvée Merci pour la cache.\end{cacheText}

\cacheNumber{1552}\needspace{5\baselineskip}\cacheName{\href{http://coord.info/GC65EVH}{Salon de Narosse} — \href{http://coord.info/GC65EVH\Number{}770914599}{1552}}\cacheData{{2018/05/23 lauki3940, Traditional Cache (2.5/2.5)}}\begin{cacheText}Sur Dax pour assister à l’Event Terra Aventura, j’en profite pour faire quelques caches. C’est une petite curiosité ce salon qui est au bord de la route près d’un semblant de marre boueuse...Les coordonnées sont précises et la cache est bien à sa place. Merci pour la boîte et la découverte du site.\end{cacheText}

\cacheNumber{1553}\needspace{5\baselineskip}\cacheName{\href{http://coord.info/GC65YGR}{\Number{}LOL ☺ Cheminé en ville ☺} — \href{http://coord.info/GC65YGR\Number{}770942888}{1553}}\cacheData{{2018/05/23 mizaga, Traditional Cache (2/1.5)}}\begin{cacheText}C’est en partant manger une pizza à la sortie de l’Event de Terra Aventura à l’OT que le groupe s’arrête pour découvrir la cache. L’endroit est très joli merci pour la découverte et pour la boîte.Un PF évidemment.\end{cacheText}

\cacheNumber{1554}\needspace{5\baselineskip}\cacheName{\href{http://coord.info/GC75393}{nouveau parcours DAX} — \href{http://coord.info/GC75393\Number{}770942404}{1554}}\cacheData{{2018/05/23 myrzen, Traditional Cache (1.5/1)}}\begin{cacheText}C’est à la sortie de l’Event de Terra Aventura et en compagnie de Dune33 et de nanard 40 270 que je découvre la cache. Heureusement la rue n’est pas très fréquentée nous pouvons loguer en toute tranquillité. Merci pour la cache.\end{cacheText}

\cacheNumber{1555}\needspace{5\baselineskip}\cacheName{\href{http://coord.info/GC7P454}{Terra Aventura dans les Landes !} — \href{http://coord.info/GC7P454\Number{}770942010}{1555}}\cacheData{{2018/05/23 Terraaventura, Event Cache (1/1.5)}}\begin{cacheText}Un accueil chaleureux nous a été réservé par l'équipe de l'OT de Dax .Terra Aventura? Je ne connaissais que vaguement le nom. Grace à Sophie je découvre un univers parallèle peuplé de charmants personnages .En exclusivité nous recevons la carte qui dévoile les nouveaux parcours 2018 et je pense tenter l'aventure des le 9 juin!!!Un grand merci à tous.\end{cacheText}

\cacheNumber{1556}\needspace{5\baselineskip}\cacheName{\href{http://coord.info/GC7N766}{Capharnaüm N°36} — \href{http://coord.info/GC7N766\Number{}770970283}{1556}}\cacheData{{2018/05/26 Opmb40, Traditional Cache (1.5/1.5)}}\begin{cacheText}(FTF)

Les mails arrivent les uns après les autres ...incroyable plus d'une centaine de caches. Que du bonheur!!! Malheureusement des obligations m'empêchent de partir dès le matin.

C'est donc en début d'après midi entre deux averses que je commence ce presque drive in.

Les grands chercheurs locaux (Dorisbear et Crispol40) et les voisins (Gamboy) ont deja ratissé le terrain ...mais j'arrive à glaner quelques FTF (sur les caches excentrées essentiellement).

Les caches sont plus ou moins faciles .Quel travail ,quel  joyeux capharnaüm pour notre plus grand plaisir!!! Un grand MERCI Opmb40\end{cacheText}

\cacheNumber{1557}\needspace{5\baselineskip}\cacheName{\href{http://coord.info/GC7NB53}{Capharnaüm N°47} — \href{http://coord.info/GC7NB53\Number{}770971890}{1557}}\cacheData{{2018/05/26 Opmb40, Traditional Cache (1.5/1.5)}}\begin{cacheText}(STF) après les Gamboy

Les mails arrivent les uns après les autres ...incroyable plus d'une centaine de caches. Que du bonheur!!! Malheureusement des obligations m'empêchent de partir dès le matin.

C'est donc en début d'après midi entre deux averses que je commence ce presque drive in.

Les grands chercheurs locaux (Dorisbear et Crispol40) et les voisins (Gamboy) ont deja ratissé le terrain ...mais j'arrive à glaner quelques FTF (sur les caches excentrées essentiellement).

Les caches sont plus ou moins faciles .Quel travail ,quel joyeux capharnaüm pour notre plus grand plaisir!!! Un grand MERCI Opmb40\end{cacheText}

\cacheNumber{1558}\needspace{5\baselineskip}\cacheName{\href{http://coord.info/GC7NRM3}{Capharnaüm N°62} — \href{http://coord.info/GC7NRM3\Number{}770987401}{1558}}\cacheData{{2018/05/26 Opmb40, Traditional Cache (1.5/2.5)}}\begin{cacheText}(STF) 

Les mails arrivent les uns après les autres ...incroyable plus d'une centaine de caches. Que du bonheur!!! Malheureusement des obligations m'empêchent de partir dès le matin.

C'est donc en début d'après midi entre deux averses que je commence ce presque drive in.

Les grands chercheurs locaux (Dorisbear et Crispol40) et les voisins (Gamboy) ont deja ratissé le terrain ...mais j'arrive à glaner quelques FTF (sur les caches excentrées essentiellement).

Les caches sont plus ou moins faciles .Quel travail ,quel joyeux capharnaüm pour notre plus grand plaisir!!! Un grand MERCI Opmb40\end{cacheText}

\cacheNumber{1559}\needspace{5\baselineskip}\cacheName{\href{http://coord.info/GC7NRMC}{Capharnaüm N°65} — \href{http://coord.info/GC7NRMC\Number{}770985565}{1559}}\cacheData{{2018/05/26 Opmb40, Traditional Cache (1.5/1.5)}}\begin{cacheText}(STF) 

Les mails arrivent les uns après les autres ...incroyable plus d'une centaine de caches. Que du bonheur!!! Malheureusement des obligations m'empêchent de partir dès le matin.

C'est donc en début d'après midi entre deux averses que je commence ce presque drive in.

Les grands chercheurs locaux (Dorisbear et Crispol40) et les voisins (Gamboy) ont deja ratissé le terrain ...mais j'arrive à glaner quelques FTF (sur les caches excentrées essentiellement).

Les caches sont plus ou moins faciles .Quel travail ,quel joyeux capharnaüm pour notre plus grand plaisir!!! Un grand MERCI Opmb40\end{cacheText}

\cacheNumber{1560}\needspace{5\baselineskip}\cacheName{\href{http://coord.info/GC7NTZB}{Capharnaüm N°67} — \href{http://coord.info/GC7NTZB\Number{}770984306}{1560}}\cacheData{{2018/05/26 Opmb40, Traditional Cache (1.5/1.5)}}\begin{cacheText}(STF) 

Les mails arrivent les uns après les autres ...incroyable plus d'une centaine de caches. Que du bonheur!!! Malheureusement des obligations m'empêchent de partir dès le matin.

C'est donc en début d'après midi entre deux averses que je commence ce presque drive in.

Les grands chercheurs locaux (Dorisbear et Crispol40) et les voisins (Gamboy) ont deja ratissé le terrain ...mais j'arrive à glaner quelques FTF (sur les caches excentrées essentiellement).

Les caches sont plus ou moins faciles .Quel travail ,quel joyeux capharnaüm pour notre plus grand plaisir!!! Un grand MERCI Opmb40\end{cacheText}

\cacheNumber{1561}\needspace{5\baselineskip}\cacheName{\href{http://coord.info/GC7NV0C}{Capharnaüm N°71} — \href{http://coord.info/GC7NV0C\Number{}770985149}{1561}}\cacheData{{2018/05/26 Opmb40, Traditional Cache (1.5/2)}}\begin{cacheText}(STF) 

Les mails arrivent les uns après les autres ...incroyable plus d'une centaine de caches. Que du bonheur!!! Malheureusement des obligations m'empêchent de partir dès le matin.

C'est donc en début d'après midi entre deux averses que je commence ce presque drive in.

Les grands chercheurs locaux (Dorisbear et Crispol40) et les voisins (Gamboy) ont deja ratissé le terrain ...mais j'arrive à glaner quelques FTF (sur les caches excentrées essentiellement).

Les caches sont plus ou moins faciles .Quel travail ,quel joyeux capharnaüm pour notre plus grand plaisir!!! Un grand MERCI Opmb40\end{cacheText}

\cacheNumber{1562}\needspace{5\baselineskip}\cacheName{\href{http://coord.info/GC7NV1A}{Capharnaüm N°74} — \href{http://coord.info/GC7NV1A\Number{}770986778}{1562}}\cacheData{{2018/05/26 Opmb40, Traditional Cache (1.5/2)}}\begin{cacheText}FTF 

Les mails arrivent les uns après les autres ...incroyable plus d'une centaine de caches. Que du bonheur!!! Malheureusement des obligations m'empêchent de partir dès le matin.

C'est donc en début d'après midi entre deux averses que je commence ce presque drive in.

Les grands chercheurs locaux (Dorisbear et Crispol40) et les voisins (Gamboy) ont deja ratissé le terrain ...mais j'arrive à glaner quelques FTF (sur les caches excentrées essentiellement).

Les caches sont plus ou moins faciles .Quel travail ,quel joyeux capharnaüm pour notre plus grand plaisir!!! Un grand MERCI Opmb40\end{cacheText}

\cacheNumber{1563}\needspace{5\baselineskip}\cacheName{\href{http://coord.info/GC7NV1H}{Capharnaüm N°75} — \href{http://coord.info/GC7NV1H\Number{}770979520}{1563}}\cacheData{{2018/05/26 Opmb40, Traditional Cache (1.5/1.5)}}\begin{cacheText}TTF après les ours et Crispol40.

Les mails arrivent les uns après les autres ...incroyable plus d'une centaine de caches. Que du bonheur!!! Malheureusement des obligations m'empêchent de partir dès le matin.

C'est donc en début d'après midi entre deux averses que je commence ce presque drive in.

Les grands chercheurs locaux (Dorisbear et Crispol40) et les voisins (Gamboy) ont deja ratissé le terrain ...mais j'arrive à glaner quelques FTF (sur les caches excentrées essentiellement).

Les caches sont plus ou moins faciles .Quel travail ,quel joyeux capharnaüm pour notre plus grand plaisir!!! Un grand MERCI Opmb40\end{cacheText}

\cacheNumber{1564}\needspace{5\baselineskip}\cacheName{\href{http://coord.info/GC7NV1R}{Capharnaüm N°76} — \href{http://coord.info/GC7NV1R\Number{}770987588}{1564}}\cacheData{{2018/05/26 Opmb40, Traditional Cache (1.5/1.5)}}\begin{cacheText}FTF 

Les mails arrivent les uns après les autres ...incroyable plus d'une centaine de caches. Que du bonheur!!! Malheureusement des obligations m'empêchent de partir dès le matin.

C'est donc en début d'après midi entre deux averses que je commence ce presque drive in.

Les grands chercheurs locaux (Dorisbear et Crispol40) et les voisins (Gamboy) ont deja ratissé le terrain ...mais j'arrive à glaner quelques FTF (sur les caches excentrées essentiellement).

Les caches sont plus ou moins faciles .Quel travail ,quel joyeux capharnaüm pour notre plus grand plaisir!!! Un grand MERCI Opmb40\end{cacheText}

\cacheNumber{1565}\needspace{5\baselineskip}\cacheName{\href{http://coord.info/GC7PBRY}{Capharnaüm N°87} — \href{http://coord.info/GC7PBRY\Number{}770988042}{1565}}\cacheData{{2018/05/26 Opmb40, Traditional Cache (1.5/1.5)}}\begin{cacheText}(STF) 

Les mails arrivent les uns après les autres ...incroyable plus d'une centaine de caches. Que du bonheur!!! Malheureusement des obligations m'empêchent de partir dès le matin.

C'est donc en début d'après midi entre deux averses que je commence ce presque drive in.

Les grands chercheurs locaux (Dorisbear et Crispol40) et les voisins (Gamboy) ont deja ratissé le terrain ...mais j'arrive à glaner quelques FTF (sur les caches excentrées essentiellement).

Les caches sont plus ou moins faciles .Quel travail ,quel joyeux capharnaüm pour notre plus grand plaisir!!! Un grand MERCI Opmb40\end{cacheText}

\cacheNumber{1566}\needspace{5\baselineskip}\cacheName{\href{http://coord.info/GC7PBT1}{Capharnaüm N°88} — \href{http://coord.info/GC7PBT1\Number{}770972931}{1566}}\cacheData{{2018/05/26 Opmb40, Traditional Cache (1.5/1.5)}}\begin{cacheText}(STF) après nos amis les ours.

Les mails arrivent les uns après les autres ...incroyable plus d'une centaine de caches. Que du bonheur!!! Malheureusement des obligations m'empêchent de partir dès le matin.

C'est donc en début d'après midi entre deux averses que je commence ce presque drive in.

Les grands chercheurs locaux (Dorisbear et Crispol40) et les voisins (Gamboy) ont deja ratissé le terrain ...mais j'arrive à glaner quelques FTF (sur les caches excentrées essentiellement).

Les caches sont plus ou moins faciles .Quel travail ,quel joyeux capharnaüm pour notre plus grand plaisir!!! Un grand MERCI Opmb40\end{cacheText}

\cacheNumber{1567}\needspace{5\baselineskip}\cacheName{\href{http://coord.info/GC7PBTD}{Capharnaüm N°90} — \href{http://coord.info/GC7PBTD\Number{}770973778}{1567}}\cacheData{{2018/05/26 Opmb40, Traditional Cache (1.5/1.5)}}\begin{cacheText}(STF) après nos amis les ours.

Les mails arrivent les uns après les autres ...incroyable plus d'une centaine de caches. Que du bonheur!!! Malheureusement des obligations m'empêchent de partir dès le matin.

C'est donc en début d'après midi entre deux averses que je commence ce presque drive in.

Les grands chercheurs locaux (Dorisbear et Crispol40) et les voisins (Gamboy) ont deja ratissé le terrain ...mais j'arrive à glaner quelques FTF (sur les caches excentrées essentiellement).

Les caches sont plus ou moins faciles .Quel travail ,quel joyeux capharnaüm pour notre plus grand plaisir!!! Un grand MERCI Opmb40\end{cacheText}

\cacheNumber{1568}\needspace{5\baselineskip}\cacheName{\href{http://coord.info/GC7PC2W}{Capharnaüm N°95} — \href{http://coord.info/GC7PC2W\Number{}770985794}{1568}}\cacheData{{2018/05/26 Opmb40, Traditional Cache (1.5/1.5)}}\begin{cacheText}(STF) 

Les mails arrivent les uns après les autres ...incroyable plus d'une centaine de caches. Que du bonheur!!! Malheureusement des obligations m'empêchent de partir dès le matin.

C'est donc en début d'après midi entre deux averses que je commence ce presque drive in.

Les grands chercheurs locaux (Dorisbear et Crispol40) et les voisins (Gamboy) ont deja ratissé le terrain ...mais j'arrive à glaner quelques FTF (sur les caches excentrées essentiellement).

Les caches sont plus ou moins faciles .Quel travail ,quel joyeux capharnaüm pour notre plus grand plaisir!!! Un grand MERCI Opmb40\end{cacheText}

\cacheNumber{1569}\needspace{5\baselineskip}\cacheName{\href{http://coord.info/GC7PC30}{Capharnaüm N°96} — \href{http://coord.info/GC7PC30\Number{}770972428}{1569}}\cacheData{{2018/05/26 Opmb40, Traditional Cache (1.5/1.5)}}\begin{cacheText}(STF ) après nos amis les ours.

Les mails arrivent les uns après les autres ...incroyable plus d'une centaine de caches. Que du bonheur!!! Malheureusement des obligations m'empêchent de partir dès le matin.

C'est donc en début d'après midi entre deux averses que je commence ce presque drive in.

Les grands chercheurs locaux (Dorisbear et Crispol40) et les voisins (Gamboy) ont deja ratissé le terrain ...mais j'arrive à glaner quelques FTF (sur les caches excentrées essentiellement).

Les caches sont plus ou moins faciles .Quel travail ,quel joyeux capharnaüm pour notre plus grand plaisir!!! Un grand MERCI Opmb40\end{cacheText}

\cacheNumber{1570}\needspace{5\baselineskip}\cacheName{\href{http://coord.info/GC7PC4F}{Capharnaüm N°107} — \href{http://coord.info/GC7PC4F\Number{}770973230}{1570}}\cacheData{{2018/05/26 Opmb40, Traditional Cache (1.5/1.5)}}\begin{cacheText}(STF) après nos amis les ours.

Les mails arrivent les uns après les autres ...incroyable plus d'une centaine de caches. Que du bonheur!!! Malheureusement des obligations m'empêchent de partir dès le matin.

C'est donc en début d'après midi entre deux averses que je commence ce presque drive in.

Les grands chercheurs locaux (Dorisbear et Crispol40) et les voisins (Gamboy) ont deja ratissé le terrain ...mais j'arrive à glaner quelques FTF (sur les caches excentrées essentiellement).

Les caches sont plus ou moins faciles .Quel travail ,quel joyeux capharnaüm pour notre plus grand plaisir!!! Un grand MERCI Opmb40\end{cacheText}

\cacheNumber{1571}\needspace{5\baselineskip}\cacheName{\href{http://coord.info/GC7PC4K}{Capharnaüm N°108} — \href{http://coord.info/GC7PC4K\Number{}770988560}{1571}}\cacheData{{2018/05/26 Opmb40, Traditional Cache (1.5/1.5)}}\begin{cacheText}(FTF)

Les mails arrivent les uns après les autres ...incroyable plus d'une centaine de caches. Que du bonheur!!! Malheureusement des obligations m'empêchent de partir dès le matin.

C'est donc en début d'après midi entre deux averses que je commence ce presque drive in.

Les grands chercheurs locaux (Dorisbear et Crispol40) et les voisins (Gamboy) ont deja ratissé le terrain ...mais j'arrive à glaner quelques FTF (sur les caches excentrées essentiellement).

Les caches sont plus ou moins faciles .Quel travail ,quel joyeux capharnaüm pour notre plus grand plaisir!!! Un grand MERCI Opmb40\end{cacheText}

\cacheNumber{1572}\needspace{5\baselineskip}\cacheName{\href{http://coord.info/GC7PC73}{Capharnaüm N°119} — \href{http://coord.info/GC7PC73\Number{}770987830}{1572}}\cacheData{{2018/05/26 Opmb40, Traditional Cache (1.5/1.5)}}\begin{cacheText}(STF) 

Les mails arrivent les uns après les autres ...incroyable plus d'une centaine de caches. Que du bonheur!!! Malheureusement des obligations m'empêchent de partir dès le matin.

C'est donc en début d'après midi entre deux averses que je commence ce presque drive in.

Les grands chercheurs locaux (Dorisbear et Crispol40) et les voisins (Gamboy) ont deja ratissé le terrain ...mais j'arrive à glaner quelques FTF (sur les caches excentrées essentiellement).

Les caches sont plus ou moins faciles .Quel travail ,quel joyeux capharnaüm pour notre plus grand plaisir!!! Un grand MERCI Opmb40\end{cacheText}

\cacheNumber{1573}\needspace{5\baselineskip}\cacheName{\href{http://coord.info/GC7PC96}{Capharnaüm N°129} — \href{http://coord.info/GC7PC96\Number{}770975106}{1573}}\cacheData{{2018/05/26 Opmb40, Traditional Cache (1.5/1.5)}}\begin{cacheText}(STF) après nos amis les ours.

Les mails arrivent les uns après les autres ...incroyable plus d'une centaine de caches. Que du bonheur!!! Malheureusement des obligations m'empêchent de partir dès le matin.

C'est donc en début d'après midi entre deux averses que je commence ce presque drive in.

Les grands chercheurs locaux (Dorisbear et Crispol40) et les voisins (Gamboy) ont deja ratissé le terrain ...mais j'arrive à glaner quelques FTF (sur les caches excentrées essentiellement).

Les caches sont plus ou moins faciles .Quel travail ,quel joyeux capharnaüm pour notre plus grand plaisir!!! Un grand MERCI Opmb40\end{cacheText}

\cacheNumber{1574}\needspace{5\baselineskip}\cacheName{\href{http://coord.info/GC7PC9B}{Capharnaüm N°130} — \href{http://coord.info/GC7PC9B\Number{}770988320}{1574}}\cacheData{{2018/05/26 Opmb40, Traditional Cache (1.5/1.5)}}\begin{cacheText}(FTF)

Les mails arrivent les uns après les autres ...incroyable plus d'une centaine de caches. Que du bonheur!!! Malheureusement des obligations m'empêchent de partir dès le matin.

C'est donc en début d'après midi entre deux averses que je commence ce presque drive in.

Les grands chercheurs locaux (Dorisbear et Crispol40) et les voisins (Gamboy) ont deja ratissé le terrain ...mais j'arrive à glaner quelques FTF (sur les caches excentrées essentiellement).

Les caches sont plus ou moins faciles .Quel travail ,quel joyeux capharnaüm pour notre plus grand plaisir!!! Un grand MERCI Opmb40\end{cacheText}

\cacheNumber{1575}\needspace{5\baselineskip}\cacheName{\href{http://coord.info/GC7PCBP}{Capharnaüm N°134} — \href{http://coord.info/GC7PCBP\Number{}770986075}{1575}}\cacheData{{2018/05/26 Opmb40, Traditional Cache (1.5/1.5)}}\begin{cacheText}(STF) 

Les mails arrivent les uns après les autres ...incroyable plus d'une centaine de caches. Que du bonheur!!! Malheureusement des obligations m'empêchent de partir dès le matin.

C'est donc en début d'après midi entre deux averses que je commence ce presque drive in.

Les grands chercheurs locaux (Dorisbear et Crispol40) et les voisins (Gamboy) ont deja ratissé le terrain ...mais j'arrive à glaner quelques FTF (sur les caches excentrées essentiellement).

Les caches sont plus ou moins faciles .Quel travail ,quel joyeux capharnaüm pour notre plus grand plaisir!!! Un grand MERCI Opmb40\end{cacheText}

\cacheNumber{1576}\needspace{5\baselineskip}\cacheName{\href{http://coord.info/GC7KW0X}{64 201 - Pyrénées Atlantique} — \href{http://coord.info/GC7KW0X\Number{}773649260}{1576}}\cacheData{{2018/06/03 gilles64, Unknown Cache (1.5/1.5)}}\begin{cacheText}A peine trouvé la bonus du 4 ,l'orage ,annoncé la veille, se fait entendre....Nous décidons malgré tout de faire le parcours 2 :les premières se feront en voiture et nous verrons pour la suite!!!A peine les trois suivantes délogées qu'un déluge s'abat sur Ustaritz. Impossible de continuer ...Nous décidons de remettre la balade au lendemain. Merci Gilles pour tout ce travail de pose qui nous ravit tant.\end{cacheText}

\cacheNumber{1577}\needspace{5\baselineskip}\cacheName{\href{http://coord.info/GC7KW0Z}{64 202 - Pyrénées Atlantique} — \href{http://coord.info/GC7KW0Z\Number{}773649557}{1577}}\cacheData{{2018/06/03 gilles64, Unknown Cache (2/1.5)}}\begin{cacheText}A peine trouvé la bonus du 4 ,l'orage ,annoncé la veille, se fait entendre....Nous décidons malgré tout de faire le parcours 2 :les premières se feront en voiture et nous verrons pour la suite!!!A peine les trois suivantes délogées qu'un déluge s'abat sur Ustaritz. Impossible de continuer ...Nous décidons de remettre la balade au lendemain. Merci Gilles pour tout ce travail de pose qui nous ravit tant.\end{cacheText}

\cacheNumber{1578}\needspace{5\baselineskip}\cacheName{\href{http://coord.info/GC7KW11}{64 203 - Pyrénées Atlantique} — \href{http://coord.info/GC7KW11\Number{}773650083}{1578}}\cacheData{{2018/06/03 gilles64, Unknown Cache (2/1.5)}}\begin{cacheText}A peine trouvé la bonus du 4 ,l'orage ,annoncé la veille, se fait entendre....Nous décidons malgré tout de faire le parcours 2 :les premières se feront en voiture et nous verrons pour la suite!!!A peine les trois suivantes délogées qu'un déluge s'abat sur Ustaritz. Impossible de continuer ...Nous décidons de remettre la balade au lendemain. Merci Gilles pour tout ce travail de pose qui nous ravit tant.\end{cacheText}

\cacheNumber{1579}\needspace{5\baselineskip}\cacheName{\href{http://coord.info/GC7KW1A}{64 205 - Pyrénées Atlantique} — \href{http://coord.info/GC7KW1A\Number{}773649746}{1579}}\cacheData{{2018/06/03 gilles64, Unknown Cache (2/1.5)}}\begin{cacheText}A peine trouvé la bonus du 4 ,l'orage ,annoncé la veille, se fait entendre....Nous décidons malgré tout de faire le parcours 2 :les premières se feront en voiture et nous verrons pour la suite!!!A peine les trois suivantes délogées qu'un déluge s'abat sur Ustaritz. Impossible de continuer ...Nous décidons de remettre la balade au lendemain. Merci Gilles pour tout ce travail de pose qui nous ravit tant.\end{cacheText}

\cacheNumber{1580}\needspace{5\baselineskip}\cacheName{\href{http://coord.info/GC7KWXP}{64 401 - Pyrénées Atlantique} — \href{http://coord.info/GC7KWXP\Number{}773600866}{1580}}\cacheData{{2018/06/03 gilles64, Unknown Cache (2/3)}}\begin{cacheText}Enfin le 64 tant attendu se dévoile. Après avoir passé la soirée à checker les coordonnées je suis prête pour aller sur le terrain dès le lendemain. Malheureusement l’orage et la grêle sont annoncés : j’hésite… J’y vais? je n’y vais pas?La tentation est trop forte j’y vais ...on verra bien!!!!Je décide d'attaquer le circuit 4 et c’est en descendant de voiture que j’aperçois Domino50 qui sort du Sentier. Nous décidons de faire le parcours ensemble. Trop contente!!! Nous serons STF sur la plupart des caches sauf une, oubliée par le redoutable Crispol40 qui nous a devancé , et qui nous permet d’inscrire un FTF. Toutes les caches sont trouvées sans trop de difficulté grâce aux hints et aux spoilers et les indices relevés pour la bonus et la super-bonus.Nous découvrons lors de la balade de beaux paysages avec de magnifiques arbres centenaires, des troupeaux de brebis (tête rousse), et des quads et quatre-quatre qui ont un peu \Quoted{pollué} le calme environnant!!!

Un grand merci Gilles pour cette super randonnée et pour la bonus qui nous a fait découvrir une vue à couper le souffle.\end{cacheText}

\cacheNumber{1581}\needspace{5\baselineskip}\cacheName{\href{http://coord.info/GC7KWXZ}{64 402 - Pyrénées Atlantique} — \href{http://coord.info/GC7KWXZ\Number{}773600293}{1581}}\cacheData{{2018/06/03 gilles64, Unknown Cache (2/1.5)}}\begin{cacheText}Enfin le 64 tant attendu se dévoile. Après avoir passé la soirée à checker les coordonnées je suis prête pour aller sur le terrain dès le lendemain. Malheureusement l’orage et la grêle sont annoncés : j’hésite… J’y vais? je n’y vais pas?La tentation est trop forte j’y vais ...on verra bien!!!!Je décide d'attaquer le circuit 4 et c’est en descendant de voiture que j’aperçois Domino50 qui sort du Sentier. Nous décidons de faire le parcours ensemble. Trop contente!!! Nous serons STF sur la plupart des caches sauf une, oubliée par le redoutable Crispol40 qui nous a devancé , et qui nous permet d’inscrire un FTF. Toutes les caches sont trouvées sans trop de difficulté grâce aux hints et aux spoilers et les indices relevés pour la bonus et la super-bonus.Nous découvrons lors de la balade de beaux paysages avec de magnifiques arbres centenaires, des troupeaux de brebis (tête rousse), et des quads et quatre-quatre qui ont un peu \Quoted{pollué} le calme environnant!!!

Un grand merci Gilles pour cette super randonnée et pour la bonus qui nous a fait découvrir une vue à couper le souffle.\end{cacheText}

\cacheNumber{1582}\needspace{5\baselineskip}\cacheName{\href{http://coord.info/GC7P3EY}{64 403 - Pyrénées Atlantique} — \href{http://coord.info/GC7P3EY\Number{}773599756}{1582}}\cacheData{{2018/06/03 gilles64, Unknown Cache (2/1.5)}}\begin{cacheText}Co [{STF}] avec Domino50

Enfin le 64 tant attendu se dévoile. Après avoir passé la soirée à checker les coordonnées je suis prête pour aller sur le terrain dès le lendemain. Malheureusement l’orage et la grêle sont annoncés : j’hésite… J’y vais? je n’y vais pas?La tentation est trop forte j’y vais ...on verra bien!!!!Je décide d'attaquer le circuit 4 et c’est en descendant de voiture que j’aperçois Domino50 qui sort du Sentier. Nous décidons de faire le parcours ensemble. Trop contente!!! Nous serons STF sur la plupart des caches sauf une, oubliée par le redoutable Crispol40 qui nous a devancé , et qui nous permet d’inscrire un FTF. Toutes les caches sont trouvées sans trop de difficulté grâce aux hints et aux spoilers et les indices relevés pour la bonus et la super-bonus.Nous découvrons lors de la balade de beaux paysages avec de magnifiques arbres centenaires, des troupeaux de brebis (tête rousse), et des quads et quatre-quatre qui ont un peu \Quoted{pollué} le calme environnant!!!

Un grand merci Gilles pour cette super randonnée et pour la bonus qui nous a fait découvrir une vue à couper le souffle.\end{cacheText}

\cacheNumber{1583}\needspace{5\baselineskip}\cacheName{\href{http://coord.info/GC7P3EZ}{64 404 - Pyrénées Atlantique} — \href{http://coord.info/GC7P3EZ\Number{}773599050}{1583}}\cacheData{{2018/06/03 gilles64, Unknown Cache (2/1.5)}}\begin{cacheText}Co [{STF}] avec Domino50

Enfin le 64 tant attendu se dévoile. Après avoir passé la soirée à checker les coordonnées je suis prête pour aller sur le terrain dès le lendemain. Malheureusement l’orage et la grêle sont annoncés : j’hésite… J’y vais? je n’y vais pas?La tentation est trop forte j’y vais ...on verra bien!!!!Je décide d'attaquer le circuit 4 et c’est en descendant de voiture que j’aperçois Domino50 qui sort du Sentier. Nous décidons de faire le parcours ensemble. Trop contente!!! Nous serons STF sur la plupart des caches sauf une, oubliée par le redoutable Crispol40 qui nous a devancé , et qui nous permet d’inscrire un FTF. Toutes les caches sont trouvées sans trop de difficulté grâce aux hints et aux spoilers et les indices relevés pour la bonus et la super-bonus.Nous découvrons lors de la balade de beaux paysages avec de magnifiques arbres centenaires, des troupeaux de brebis (tête rousse), et des quads et quatre-quatre qui ont un peu \Quoted{pollué} le calme environnant!!!

Un grand merci Gilles pour cette super randonnée et pour la bonus qui nous a fait découvrir une vue à couper le souffle.\end{cacheText}

\cacheNumber{1584}\needspace{5\baselineskip}\cacheName{\href{http://coord.info/GC7P3F0}{64 405 - Pyrénées Atlantique} — \href{http://coord.info/GC7P3F0\Number{}773598607}{1584}}\cacheData{{2018/06/03 gilles64, Unknown Cache (2/1.5)}}\begin{cacheText}Co [{STF}] avec Domino50

Enfin le 64 tant attendu se dévoile. Après avoir passé la soirée à checker les coordonnées je suis prête pour aller sur le terrain dès le lendemain. Malheureusement l’orage et la grêle sont annoncés : j’hésite… J’y vais? je n’y vais pas?La tentation est trop forte j’y vais ...on verra bien!!!!Je décide d'attaquer le circuit 4 et c’est en descendant de voiture que j’aperçois Domino50 qui sort du Sentier. Nous décidons de faire le parcours ensemble. Trop contente!!! Nous serons STF sur la plupart des caches sauf une, oubliée par le redoutable Crispol40 qui nous a devancé , et qui nous permet d’inscrire un FTF. Toutes les caches sont trouvées sans trop de difficulté grâce aux hints et aux spoilers et les indices relevés pour la bonus et la super-bonus.Nous découvrons lors de la balade de beaux paysages avec de magnifiques arbres centenaires, des troupeaux de brebis (tête rousse), et des quads et quatre-quatre qui ont un peu \Quoted{pollué} le calme environnant!!!

Un grand merci Gilles pour cette super randonnée et pour la bonus qui nous a fait découvrir une vue à couper le souffle.\end{cacheText}

\cacheNumber{1585}\needspace{5\baselineskip}\cacheName{\href{http://coord.info/GC7P3F3}{64 406 - Pyrénées Atlantique} — \href{http://coord.info/GC7P3F3\Number{}773598258}{1585}}\cacheData{{2018/06/03 gilles64, Unknown Cache (2/1.5)}}\begin{cacheText}Co [{STF}] avec Domino50

Enfin le 64 tant attendu se dévoile. Après avoir passé la soirée à checker les coordonnées je suis prête pour aller sur le terrain dès le lendemain. Malheureusement l’orage et la grêle sont annoncés : j’hésite… J’y vais? je n’y vais pas?La tentation est trop forte j’y vais ...on verra bien!!!!Je décide d'attaquer le circuit 4 et c’est en descendant de voiture que j’aperçois Domino50 qui sort du Sentier. Nous décidons de faire le parcours ensemble. Trop contente!!! Nous serons STF sur la plupart des caches sauf une, oubliée par le redoutable Crispol40 qui nous a devancé , et qui nous permet d’inscrire un FTF. Toutes les caches sont trouvées sans trop de difficulté grâce aux hints et aux spoilers et les indices relevés pour la bonus et la super-bonus.Nous découvrons lors de la balade de beaux paysages avec de magnifiques arbres centenaires, des troupeaux de brebis (tête rousse), et des quads et quatre-quatre qui ont un peu \Quoted{pollué} le calme environnant!!!

Un grand merci Gilles pour cette super randonnée et pour la bonus qui nous a fait découvrir une vue à couper le souffle.\end{cacheText}

\cacheNumber{1586}\needspace{5\baselineskip}\cacheName{\href{http://coord.info/GC7P3F4}{64 407 - Pyrénées Atlantique} — \href{http://coord.info/GC7P3F4\Number{}773597842}{1586}}\cacheData{{2018/06/03 gilles64, Unknown Cache (2/1.5)}}\begin{cacheText}Co [{STF}] avec Domino50

Enfin le 64 tant attendu se dévoile. Après avoir passé la soirée à checker les coordonnées je suis prête pour aller sur le terrain dès le lendemain. Malheureusement l’orage et la grêle sont annoncés : j’hésite… J’y vais? je n’y vais pas?La tentation est trop forte j’y vais ...on verra bien!!!!Je décide d'attaquer le circuit 4 et c’est en descendant de voiture que j’aperçois Domino50 qui sort du Sentier. Nous décidons de faire le parcours ensemble. Trop contente!!! Nous serons STF sur la plupart des caches sauf une, oubliée par le redoutable Crispol40 qui nous a devancé , et qui nous permet d’inscrire un FTF. Toutes les caches sont trouvées sans trop de difficulté grâce aux hints et aux spoilers et les indices relevés pour la bonus et la super-bonus.Nous découvrons lors de la balade de beaux paysages avec de magnifiques arbres centenaires, des troupeaux de brebis (tête rousse), et des quads et quatre-quatre qui ont un peu \Quoted{pollué} le calme environnant!!!

Un grand merci Gilles pour cette super randonnée et pour la bonus qui nous a fait découvrir une vue à couper le souffle.\end{cacheText}

\cacheNumber{1587}\needspace{5\baselineskip}\cacheName{\href{http://coord.info/GC7P3F7}{64 408 - Pyrénées Atlantique} — \href{http://coord.info/GC7P3F7\Number{}773597407}{1587}}\cacheData{{2018/06/03 gilles64, Unknown Cache (2/1.5)}}\begin{cacheText}Co [{STF}] avec Domino50

Enfin le 64 tant attendu se dévoile. Après avoir passé la soirée à checker les coordonnées je suis prête pour aller sur le terrain dès le lendemain. Malheureusement l’orage et la grêle sont annoncés : j’hésite… J’y vais? je n’y vais pas?La tentation est trop forte j’y vais ...on verra bien!!!!Je décide d'attaquer le circuit 4 et c’est en descendant de voiture que j’aperçois Domino50 qui sort du Sentier. Nous décidons de faire le parcours ensemble. Trop contente!!! Nous serons STF sur la plupart des caches sauf une, oubliée par le redoutable Crispol40 qui nous a devancé , et qui nous permet d’inscrire un FTF. Toutes les caches sont trouvées sans trop de difficulté grâce aux hints et aux spoilers et les indices relevés pour la bonus et la super-bonus.Nous découvrons lors de la balade de beaux paysages avec de magnifiques arbres centenaires, des troupeaux de brebis (tête rousse), et des quads et quatre-quatre qui ont un peu \Quoted{pollué} le calme environnant!!!

Un grand merci Gilles pour cette super randonnée et pour la bonus qui nous a fait découvrir une vue à couper le souffle.\end{cacheText}

\cacheNumber{1588}\needspace{5\baselineskip}\cacheName{\href{http://coord.info/GC7P3F8}{64 409 - Pyrénées Atlantique} — \href{http://coord.info/GC7P3F8\Number{}773597072}{1588}}\cacheData{{2018/06/03 gilles64, Unknown Cache (2/1.5)}}\begin{cacheText}Co [{STF}] avec Domino50

Enfin le 64 tant attendu se dévoile. Après avoir passé la soirée à checker les coordonnées je suis prête pour aller sur le terrain dès le lendemain. Malheureusement l’orage et la grêle sont annoncés : j’hésite… J’y vais? je n’y vais pas?La tentation est trop forte j’y vais ...on verra bien!!!!Je décide d'attaquer le circuit 4 et c’est en descendant de voiture que j’aperçois Domino50 qui sort du Sentier. Nous décidons de faire le parcours ensemble. Trop contente!!! Nous serons STF sur la plupart des caches sauf une, oubliée par le redoutable Crispol40 qui nous a devancé , et qui nous permet d’inscrire un FTF. Toutes les caches sont trouvées sans trop de difficulté grâce aux hints et aux spoilers et les indices relevés pour la bonus et la super-bonus.Nous découvrons lors de la balade de beaux paysages avec de magnifiques arbres centenaires, des troupeaux de brebis (tête rousse), et des quads et quatre-quatre qui ont un peu \Quoted{pollué} le calme environnant!!!

Un grand merci Gilles pour cette super randonnée et pour la bonus qui nous a fait découvrir une vue à couper le souffle.\end{cacheText}

\cacheNumber{1589}\needspace{5\baselineskip}\cacheName{\href{http://coord.info/GC7P3F9}{64 410 - Pyrénées Atlantique} — \href{http://coord.info/GC7P3F9\Number{}773596627}{1589}}\cacheData{{2018/06/03 gilles64, Unknown Cache (2/1.5)}}\begin{cacheText}Co [{STF}] avec Domino50

Enfin le 64 tant attendu se dévoile. Après avoir passé la soirée à checker les coordonnées je suis prête pour aller sur le terrain dès le lendemain. Malheureusement l’orage et la grêle sont annoncés : j’hésite… J’y vais? je n’y vais pas?La tentation est trop forte j’y vais ...on verra bien!!!!Je décide d'attaquer le circuit 4 et c’est en descendant de voiture que j’aperçois Domino50 qui sort du Sentier. Nous décidons de faire le parcours ensemble. Trop contente!!! Nous serons STF sur la plupart des caches sauf une, oubliée par le redoutable Crispol40 qui nous a devancé , et qui nous permet d’inscrire un FTF. Toutes les caches sont trouvées sans trop de difficulté grâce aux hints et aux spoilers et les indices relevés pour la bonus et la super-bonus.Nous découvrons lors de la balade de beaux paysages avec de magnifiques arbres centenaires, des troupeaux de brebis (tête rousse), et des quads et quatre-quatre qui ont un peu \Quoted{pollué} le calme environnant!!!

Un grand merci Gilles pour cette super randonnée et pour la bonus qui nous a fait découvrir une vue à couper le souffle.\end{cacheText}

\cacheNumber{1590}\needspace{5\baselineskip}\cacheName{\href{http://coord.info/GC7P3FK}{64 411 - Pyrénées Atlantique} — \href{http://coord.info/GC7P3FK\Number{}773596016}{1590}}\cacheData{{2018/06/03 gilles64, Unknown Cache (2/2)}}\begin{cacheText}Co [{STF}] avec Domino50

Enfin le 64 tant attendu se dévoile. Après avoir passé la soirée à checker les coordonnées je suis prête pour aller sur le terrain dès le lendemain. Malheureusement l’orage et la grêle sont annoncés : j’hésite… J’y vais? je n’y vais pas?La tentation est trop forte j’y vais ...on verra bien!!!!Je décide d'attaquer le circuit 4 et c’est en descendant de voiture que j’aperçois Domino50 qui sort du Sentier. Nous décidons de faire le parcours ensemble. Trop contente!!! Nous serons STF sur la plupart des caches sauf une, oubliée par le redoutable Crispol40 qui nous a devancé , et qui nous permet d’inscrire un FTF. Toutes les caches sont trouvées sans trop de difficulté grâce aux hints et aux spoilers et les indices relevés pour la bonus et la super-bonus.Nous découvrons lors de la balade de beaux paysages avec de magnifiques arbres centenaires, des troupeaux de brebis (tête rousse), et des quads et quatre-quatre qui ont un peu \Quoted{pollué} le calme environnant!!!

Un grand merci Gilles pour cette super randonnée et pour la bonus qui nous a fait découvrir une vue à couper le souffle.\end{cacheText}

\cacheNumber{1591}\needspace{5\baselineskip}\cacheName{\href{http://coord.info/GC7P3FN}{64 412 - Pyrénées Atlantique} — \href{http://coord.info/GC7P3FN\Number{}773595569}{1591}}\cacheData{{2018/06/03 gilles64, Unknown Cache (2/2)}}\begin{cacheText}Co [{STF}] avec Domino50

Enfin le 64 tant attendu se dévoile. Après avoir passé la soirée à checker les coordonnées je suis prête pour aller sur le terrain dès le lendemain. Malheureusement l’orage et la grêle sont annoncés : j’hésite… J’y vais? je n’y vais pas?La tentation est trop forte j’y vais ...on verra bien!!!!Je décide d'attaquer le circuit 4 et c’est en descendant de voiture que j’aperçois Domino50 qui sort du Sentier. Nous décidons de faire le parcours ensemble. Trop contente!!! Nous serons STF sur la plupart des caches sauf une, oubliée par le redoutable Crispol40 qui nous a devancé , et qui nous permet d’inscrire un FTF. Toutes les caches sont trouvées sans trop de difficulté grâce aux hints et aux spoilers et les indices relevés pour la bonus et la super-bonus.Nous découvrons lors de la balade de beaux paysages avec de magnifiques arbres centenaires, des troupeaux de brebis (tête rousse), et des quads et quatre-quatre qui ont un peu \Quoted{pollué} le calme environnant!!!

Un grand merci Gilles pour cette super randonnée et pour la bonus qui nous a fait découvrir une vue à couper le souffle.\end{cacheText}

\cacheNumber{1592}\needspace{5\baselineskip}\cacheName{\href{http://coord.info/GC7P3FQ}{64 413 - Pyrénées Atlantique} — \href{http://coord.info/GC7P3FQ\Number{}773595149}{1592}}\cacheData{{2018/06/03 gilles64, Unknown Cache (2/2)}}\begin{cacheText}Co [{STF}] avec Domino50

Enfin le 64 tant attendu se dévoile. Après avoir passé la soirée à checker les coordonnées je suis prête pour aller sur le terrain dès le lendemain. Malheureusement l’orage et la grêle sont annoncés : j’hésite… J’y vais? je n’y vais pas?La tentation est trop forte j’y vais ...on verra bien!!!!Je décide d'attaquer le circuit 4 et c’est en descendant de voiture que j’aperçois Domino50 qui sort du Sentier. Nous décidons de faire le parcours ensemble. Trop contente!!! Nous serons STF sur la plupart des caches sauf une, oubliée par le redoutable Crispol40 qui nous a devancé , et qui nous permet d’inscrire un FTF. Toutes les caches sont trouvées sans trop de difficulté grâce aux hints et aux spoilers et les indices relevés pour la bonus et la super-bonus.Nous découvrons lors de la balade de beaux paysages avec de magnifiques arbres centenaires, des troupeaux de brebis (tête rousse), et des quads et quatre-quatre qui ont un peu \Quoted{pollué} le calme environnant!!!

Un grand merci Gilles pour cette super randonnée et pour la bonus qui nous a fait découvrir une vue à couper le souffle.\end{cacheText}

\cacheNumber{1593}\needspace{5\baselineskip}\cacheName{\href{http://coord.info/GC7P3FT}{64 414 - Pyrénées Atlantique} — \href{http://coord.info/GC7P3FT\Number{}773594765}{1593}}\cacheData{{2018/06/03 gilles64, Unknown Cache (2/2)}}\begin{cacheText}Co [{STF}] avec Domino50

Enfin le 64 tant attendu se dévoile. Après avoir passé la soirée à checker les coordonnées je suis prête pour aller sur le terrain dès le lendemain. Malheureusement l’orage et la grêle sont annoncés : j’hésite… J’y vais? je n’y vais pas?La tentation est trop forte j’y vais ...on verra bien!!!!Je décide d'attaquer le circuit 4 et c’est en descendant de voiture que j’aperçois Domino50 qui sort du Sentier. Nous décidons de faire le parcours ensemble. Trop contente!!! Nous serons STF sur la plupart des caches sauf une, oubliée par le redoutable Crispol40 qui nous a devancé , et qui nous permet d’inscrire un FTF. Toutes les caches sont trouvées sans trop de difficulté grâce aux hints et aux spoilers et les indices relevés pour la bonus et la super-bonus.Nous découvrons lors de la balade de beaux paysages avec de magnifiques arbres centenaires, des troupeaux de brebis (tête rousse), et des quads et quatre-quatre qui ont un peu \Quoted{pollué} le calme environnant!!!

Un grand merci Gilles pour cette super randonnée et pour la bonus qui nous a fait découvrir une vue à couper le souffle.\end{cacheText}

\cacheNumber{1594}\needspace{5\baselineskip}\cacheName{\href{http://coord.info/GC7P3FW}{64 415 - Pyrénées Atlantique} — \href{http://coord.info/GC7P3FW\Number{}773592684}{1594}}\cacheData{{2018/06/03 gilles64, Unknown Cache (2/1.5)}}\begin{cacheText}Co [{STF}] avec Domino50

Enfin le 64 tant attendu se dévoile. Après avoir passé la soirée à checker les coordonnées je suis prête pour aller sur le terrain dès le lendemain. Malheureusement l’orage et la grêle sont annoncés : j’hésite… J’y vais? je n’y vais pas?La tentation est trop forte j’y vais ...on verra bien!!!!Je décide d'attaquer le circuit 4 et c’est en descendant de voiture que j’aperçois Domino50 qui sort du Sentier. Nous décidons de faire le parcours ensemble. Trop contente!!! Nous serons STF sur la plupart des caches sauf une, oubliée par le redoutable Crispol40 qui nous a devancé , et qui nous permet d’inscrire un FTF. Toutes les caches sont trouvées sans trop de difficulté grâce aux hints et aux spoilers et les indices relevés pour la bonus et la super-bonus.Nous découvrons lors de la balade de beaux paysages avec de magnifiques arbres centenaires, des troupeaux de brebis (tête rousse), et des quads et quatre-quatre qui ont un peu \Quoted{pollué} le calme environnant!!!

Un grand merci Gilles pour cette super randonnée et pour la bonus qui nous a fait découvrir une vue à couper le souffle.\end{cacheText}

\cacheNumber{1595}\needspace{5\baselineskip}\cacheName{\href{http://coord.info/GC7P3FX}{64 416 - Pyrénées Atlantique} — \href{http://coord.info/GC7P3FX\Number{}773592393}{1595}}\cacheData{{2018/06/03 gilles64, Unknown Cache (2/1.5)}}\begin{cacheText}Co [{STF}] avec Domino50

Enfin le 64 tant attendu se dévoile. Après avoir passé la soirée à checker les coordonnées je suis prête pour aller sur le terrain dès le lendemain. Malheureusement l’orage et la grêle sont annoncés : j’hésite… J’y vais? je n’y vais pas?La tentation est trop forte j’y vais ...on verra bien!!!!Je décide d'attaquer le circuit 4 et c’est en descendant de voiture que j’aperçois Domino50 qui sort du Sentier. Nous décidons de faire le parcours ensemble. Trop contente!!! Nous serons STF sur la plupart des caches sauf une, oubliée par le redoutable Crispol40 qui nous a devancé , et qui nous permet d’inscrire un FTF. Toutes les caches sont trouvées sans trop de difficulté grâce aux hints et aux spoilers et les indices relevés pour la bonus et la super-bonus.Nous découvrons lors de la balade de beaux paysages avec de magnifiques arbres centenaires, des troupeaux de brebis (tête rousse), et des quads et quatre-quatre qui ont un peu \Quoted{pollué} le calme environnant!!!

Un grand merci Gilles pour cette super randonnée et pour la bonus qui nous a fait découvrir une vue à couper le souffle.\end{cacheText}

\cacheNumber{1596}\needspace{5\baselineskip}\cacheName{\href{http://coord.info/GC7P3FZ}{64 417 - Pyrénées Atlantique} — \href{http://coord.info/GC7P3FZ\Number{}773592095}{1596}}\cacheData{{2018/06/03 gilles64, Unknown Cache (2/2)}}\begin{cacheText}Co [{FTF}] avec Domino50

Enfin le 64 tant attendu se dévoile. Après avoir passé la soirée à checker les coordonnées je suis prête pour aller sur le terrain dès le lendemain. Malheureusement l’orage et la grêle sont annoncés : j’hésite… J’y vais? je n’y vais pas?La tentation est trop forte j’y vais ...on verra bien!!!!Je décide d'attaquer le circuit 4 et c’est en descendant de voiture que j’aperçois Domino50 qui sort du Sentier. Nous décidons de faire le parcours ensemble. Trop contente!!! Nous serons STF sur la plupart des caches sauf une, oubliée par le redoutable Crispol40 qui nous a devancé , et qui nous permet d’inscrire un FTF. Toutes les caches sont trouvées sans trop de difficulté grâce aux hints et aux spoilers et les indices relevés pour la bonus et la super-bonus.Nous découvrons lors de la balade de beaux paysages avec de magnifiques arbres centenaires, des troupeaux de brebis (tête rousse), et des quads et quatre-quatre qui ont un peu \Quoted{pollué} le calme environnant!!!

Un grand merci Gilles pour cette super randonnée et pour la bonus qui nous a fait découvrir une vue à couper le souffle.\end{cacheText}

\cacheNumber{1597}\needspace{5\baselineskip}\cacheName{\href{http://coord.info/GC7P3G1}{64 418 - Pyrénées Atlantique} — \href{http://coord.info/GC7P3G1\Number{}773603887}{1597}}\cacheData{{2018/06/03 gilles64, Unknown Cache (2/3)}}\begin{cacheText}Co [{STF}] avec Domino50

Enfin le 64 tant attendu se dévoile. Après avoir passé la soirée à checker les coordonnées je suis prête pour aller sur le terrain dès le lendemain. Malheureusement l’orage et la grêle sont annoncés : j’hésite… J’y vais? je n’y vais pas?La tentation est trop forte j’y vais ...on verra bien!!!!Je décide d'attaquer le circuit 4 et c’est en descendant de voiture que j’aperçois Domino50 qui sort du Sentier. Nous décidons de faire le parcours ensemble. Trop contente!!! Nous serons STF sur la plupart des caches sauf une, oubliée par le redoutable Crispol40 qui nous a devancé , et qui nous permet d’inscrire un FTF. Toutes les caches sont trouvées sans trop de difficulté grâce aux hints et aux spoilers et les indices relevés pour la bonus et la super-bonus.Nous découvrons lors de la balade de beaux paysages avec de magnifiques arbres centenaires, des troupeaux de brebis (tête rousse), et des quads et quatre-quatre qui ont un peu \Quoted{pollué} le calme environnant!!!

Un grand merci Gilles pour cette super randonnée et pour la bonus qui nous a fait découvrir une vue à couper le souffle.\end{cacheText}

\cacheNumber{1598}\needspace{5\baselineskip}\cacheName{\href{http://coord.info/GC7P3G2}{64 419 - Pyrénées Atlantique} — \href{http://coord.info/GC7P3G2\Number{}773591313}{1598}}\cacheData{{2018/06/03 gilles64, Unknown Cache (2.5/3)}}\begin{cacheText}Co [{STF}] avec Domino50

Enfin le 64 tant attendu se dévoile. Après avoir passé la soirée à checker les coordonnées je suis prête pour aller sur le terrain dès le lendemain. Malheureusement l’orage et la grêle sont annoncés : j’hésite… J’y vais? je n’y vais pas?La tentation est trop forte j’y vais ...on verra bien!!!!Je décide d'attaquer le circuit 4 et c’est en descendant de voiture que j’aperçois Domino50 qui sort du Sentier. Nous décidons de faire le parcours ensemble. Trop contente!!! Nous serons STF sur la plupart des caches sauf une, oubliée par le redoutable Crispol40 qui nous a devancé , et qui nous permet d’inscrire un FTF. Toutes les caches sont trouvées sans trop de difficulté grâce aux hints et aux spoilers et les indices relevés pour la bonus et la super-bonus.Nous découvrons lors de la balade de beaux paysages avec de magnifiques arbres centenaires, des troupeaux de brebis (tête rousse), et des quads et quatre-quatre qui ont un peu \Quoted{pollué} le calme environnant!!!

Un grand merci Gilles pour cette super randonnée et pour la bonus qui nous a fait découvrir une vue à couper le souffle.\end{cacheText}

\cacheNumber{1599}\needspace{5\baselineskip}\cacheName{\href{http://coord.info/GC7P3G3}{64 420 - Pyrénées Atlantique} — \href{http://coord.info/GC7P3G3\Number{}773589511}{1599}}\cacheData{{2018/06/03 gilles64, Unknown Cache (2/2.5)}}\begin{cacheText}Co [{STF}] avec Domino50

Enfin le 64 tant attendu se dévoile. Après avoir passé la soirée à checker les coordonnées je suis prête pour aller sur le terrain dès le lendemain. Malheureusement l’orage et la grêle sont annoncés : j’hésite… J’y vais? je n’y vais pas?La tentation est trop forte j’y vais ...on verra bien!!!!Je décide d'attaquer le circuit 4 et c’est en descendant de voiture que j’aperçois Domino50 qui sort du Sentier. Nous décidons de faire le parcours ensemble. Trop contente!!! Nous serons STF sur la plupart des caches sauf une, oubliée par le redoutable Crispol40 qui nous a devancé , et qui nous permet d’inscrire un FTF. Toutes les caches sont trouvées sans trop de difficulté grâce aux hints et aux spoilers et les indices relevés pour la bonus et la super-bonus.Nous découvrons lors de la balade de beaux paysages avec de magnifiques arbres centenaires, des troupeaux de brebis (tête rousse), et des quads et quatre-quatre qui ont un peu \Quoted{pollué} le calme environnant!!!

Un grand merci Gilles pour cette super randonnée et pour la bonus qui nous a fait découvrir une vue à couper le souffle.\end{cacheText}

\cacheNumber{1600}\needspace{5\baselineskip}\cacheName{\href{http://coord.info/GC7P3G5}{64 421 - Pyrénées Atlantique} — \href{http://coord.info/GC7P3G5\Number{}773589131}{1600}}\cacheData{{2018/06/03 gilles64, Unknown Cache (2/2.5)}}\begin{cacheText}Co [{STF}] avec Domino50

Enfin le 64 tant attendu se dévoile. Après avoir passé la soirée à checker les coordonnées je suis prête pour aller sur le terrain dès le lendemain. Malheureusement l’orage et la grêle sont annoncés : j’hésite… J’y vais? je n’y vais pas?La tentation est trop forte j’y vais ...on verra bien!!!!Je décide d'attaquer le circuit 4 et c’est en descendant de voiture que j’aperçois Domino50 qui sort du Sentier. Nous décidons de faire le parcours ensemble. Trop contente!!! Nous serons STF sur la plupart des caches sauf une, oubliée par le redoutable Crispol40 qui nous a devancé , et qui nous permet d’inscrire un FTF. Toutes les caches sont trouvées sans trop de difficulté grâce aux hints et aux spoilers et les indices relevés pour la bonus et la super-bonus.Nous découvrons lors de la balade de beaux paysages avec de magnifiques arbres centenaires, des troupeaux de brebis (tête rousse), et des quads et quatre-quatre qui ont un peu \Quoted{pollué} le calme environnant!!!

Un grand merci Gilles pour cette super randonnée et pour la bonus qui nous a fait découvrir une vue à couper le souffle.\end{cacheText}

\cacheNumber{1601}\needspace{5\baselineskip}\cacheName{\href{http://coord.info/GC7P3G7}{64 422 - Pyrénées Atlantique} — \href{http://coord.info/GC7P3G7\Number{}773588850}{1601}}\cacheData{{2018/06/03 gilles64, Unknown Cache (2/2.5)}}\begin{cacheText}Co [{STF}] avec Domino50

Enfin le 64 tant attendu se dévoile. Après avoir passé la soirée à checker les coordonnées je suis prête pour aller sur le terrain dès le lendemain. Malheureusement l’orage et la grêle sont annoncés : j’hésite… J’y vais? je n’y vais pas?La tentation est trop forte j’y vais ...on verra bien!!!!Je décide d'attaquer le circuit 4 et c’est en descendant de voiture que j’aperçois Domino50 qui sort du Sentier. Nous décidons de faire le parcours ensemble. Trop contente!!! Nous serons STF sur la plupart des caches sauf une, oubliée par le redoutable Crispol40 qui nous a devancé , et qui nous permet d’inscrire un FTF. Toutes les caches sont trouvées sans trop de difficulté grâce aux hints et aux spoilers et les indices relevés pour la bonus et la super-bonus.Nous découvrons lors de la balade de beaux paysages avec de magnifiques arbres centenaires, des troupeaux de brebis (tête rousse), et des quads et quatre-quatre qui ont un peu \Quoted{pollué} le calme environnant!!!

Un grand merci Gilles pour cette super randonnée et pour la bonus qui nous a fait découvrir une vue à couper le souffle.\end{cacheText}

\cacheNumber{1602}\needspace{5\baselineskip}\cacheName{\href{http://coord.info/GC7P3G8}{64 423 - Pyrénées Atlantique} — \href{http://coord.info/GC7P3G8\Number{}773588573}{1602}}\cacheData{{2018/06/03 gilles64, Unknown Cache (2/2)}}\begin{cacheText}Co [{STF}] avec Domino50

Enfin le 64 tant attendu se dévoile. Après avoir passé la soirée à checker les coordonnées je suis prête pour aller sur le terrain dès le lendemain. Malheureusement l’orage et la grêle sont annoncés : j’hésite… J’y vais? je n’y vais pas?La tentation est trop forte j’y vais ...on verra bien!!!!Je décide d'attaquer le circuit 4 et c’est en descendant de voiture que j’aperçois Domino50 qui sort du Sentier. Nous décidons de faire le parcours ensemble. Trop contente!!! Nous serons STF sur la plupart des caches sauf une, oubliée par le redoutable Crispol40 qui nous a devancé , et qui nous permet d’inscrire un FTF. Toutes les caches sont trouvées sans trop de difficulté grâce aux hints et aux spoilers et les indices relevés pour la bonus et la super-bonus.Nous découvrons lors de la balade de beaux paysages avec de magnifiques arbres centenaires, des troupeaux de brebis (tête rousse), et des quads et quatre-quatre qui ont un peu \Quoted{pollué} le calme environnant!!!

Un grand merci Gilles pour cette super randonnée et pour la bonus qui nous a fait découvrir une vue à couper le souffle.\end{cacheText}

\cacheNumber{1603}\needspace{5\baselineskip}\cacheName{\href{http://coord.info/GC7P3GB}{64 424 - Pyrénées Atlantique} — \href{http://coord.info/GC7P3GB\Number{}773587878}{1603}}\cacheData{{2018/06/03 gilles64, Unknown Cache (2/2)}}\begin{cacheText}Co [{STF}] avec Domino50

Enfin le 64 tant attendu se dévoile. Après avoir passé la soirée à checker les coordonnées je suis prête pour aller sur le terrain dès le lendemain. Malheureusement l’orage et la grêle sont annoncés : j’hésite… J’y vais? je n’y vais pas?La tentation est trop forte j’y vais ...on verra bien!!!!Je décide d'attaquer le circuit 4 et c’est en descendant de voiture que j’aperçois Domino50 qui sort du Sentier. Nous décidons de faire le parcours ensemble. Trop contente!!! Nous serons STF sur la plupart des caches sauf une, oubliée par le redoutable Crispol40 qui nous a devancé , et qui nous permet d’inscrire un FTF. Toutes les caches sont trouvées sans trop de difficulté grâce aux hints et aux spoilers et les indices relevés pour la bonus et la super-bonus.Nous découvrons lors de la balade de beaux paysages avec de magnifiques arbres centenaires, des troupeaux de brebis (tête rousse), et des quads et quatre-quatre qui ont un peu \Quoted{pollué} le calme environnant!!!

Un grand merci Gilles pour cette super randonnée et pour la bonus qui nous a fait découvrir une vue à couper le souffle.\end{cacheText}

\cacheNumber{1604}\needspace{5\baselineskip}\cacheName{\href{http://coord.info/GC7P3GD}{64 425 - Pyrénées Atlantique} — \href{http://coord.info/GC7P3GD\Number{}773587489}{1604}}\cacheData{{2018/06/03 gilles64, Unknown Cache (2/2)}}\begin{cacheText}Co [{STF}] avec Domino50

Enfin le 64 tant attendu se dévoile. Après avoir passé la soirée à checker les coordonnées je suis prête pour aller sur le terrain dès le lendemain. Malheureusement l’orage et la grêle sont annoncés : j’hésite… J’y vais? je n’y vais pas?La tentation est trop forte j’y vais ...on verra bien!!!!Je décide d'attaquer le circuit 4 et c’est en descendant de voiture que j’aperçois Domino50 qui sort du Sentier. Nous décidons de faire le parcours ensemble. Trop contente!!! Nous serons STF sur la plupart des caches sauf une, oubliée par le redoutable Crispol40 qui nous a devancé , et qui nous permet d’inscrire un FTF. Toutes les caches sont trouvées sans trop de difficulté grâce aux hints et aux spoilers et les indices relevés pour la bonus et la super-bonus.Nous découvrons lors de la balade de beaux paysages avec de magnifiques arbres centenaires, des troupeaux de brebis (tête rousse), et des quads et quatre-quatre qui ont un peu \Quoted{pollué} le calme environnant!!!

Un grand merci Gilles pour cette super randonnée et pour la bonus qui nous a fait découvrir une vue à couper le souffle.\end{cacheText}

\cacheNumber{1605}\needspace{5\baselineskip}\cacheName{\href{http://coord.info/GC7P3GE}{64 426 - Pyrénées Atlantique} — \href{http://coord.info/GC7P3GE\Number{}773603432}{1605}}\cacheData{{2018/06/03 gilles64, Unknown Cache (2/2)}}\begin{cacheText}Co [{STF}] avec Domino50

Enfin le 64 tant attendu se dévoile. Après avoir passé la soirée à checker les coordonnées je suis prête pour aller sur le terrain dès le lendemain. Malheureusement l’orage et la grêle sont annoncés : j’hésite… J’y vais? je n’y vais pas?La tentation est trop forte j’y vais ...on verra bien!!!!Je décide d'attaquer le circuit 4 et c’est en descendant de voiture que j’aperçois Domino50 qui sort du Sentier. Nous décidons de faire le parcours ensemble. Trop contente!!! Nous serons STF sur la plupart des caches sauf une, oubliée par le redoutable Crispol40 qui nous a devancé , et qui nous permet d’inscrire un FTF. Toutes les caches sont trouvées sans trop de difficulté grâce aux hints et aux spoilers et les indices relevés pour la bonus et la super-bonus.Nous découvrons lors de la balade de beaux paysages avec de magnifiques arbres centenaires, des troupeaux de brebis (tête rousse), et des quads et quatre-quatre qui ont un peu \Quoted{pollué} le calme environnant!!!

Un grand merci Gilles pour cette super randonnée et pour la bonus qui nous a fait découvrir une vue à couper le souffle.\end{cacheText}

\cacheNumber{1606}\needspace{5\baselineskip}\cacheName{\href{http://coord.info/GC7P3GG}{64 427 - Pyrénées Atlantique} — \href{http://coord.info/GC7P3GG\Number{}773587253}{1606}}\cacheData{{2018/06/03 gilles64, Unknown Cache (2/2)}}\begin{cacheText}Co [{STF}] avec Domino50

Enfin le 64 tant attendu se dévoile. Après avoir passé la soirée à checker les coordonnées je suis prête pour aller sur le terrain dès le lendemain. Malheureusement l’orage et la grêle sont annoncés : j’hésite… J’y vais? je n’y vais pas?La tentation est trop forte j’y vais ...on verra bien!!!!Je décide d'attaquer le circuit 4 et c’est en descendant de voiture que j’aperçois Domino50 qui sort du Sentier. Nous décidons de faire le parcours ensemble. Trop contente!!! Nous serons STF sur la plupart des caches sauf une, oubliée par le redoutable Crispol40 qui nous a devancé , et qui nous permet d’inscrire un FTF. Toutes les caches sont trouvées sans trop de difficulté grâce aux hints et aux spoilers et les indices relevés pour la bonus et la super-bonus.Nous découvrons lors de la balade de beaux paysages avec de magnifiques arbres centenaires, des troupeaux de brebis (tête rousse), et des quads et quatre-quatre qui ont un peu \Quoted{pollué} le calme environnant!!!

Un grand merci Gilles pour cette super randonnée et pour la bonus qui nous a fait découvrir une vue à couper le souffle.\end{cacheText}

\cacheNumber{1607}\needspace{5\baselineskip}\cacheName{\href{http://coord.info/GC7P3GH}{64 428 - Pyrénées Atlantique} — \href{http://coord.info/GC7P3GH\Number{}773587029}{1607}}\cacheData{{2018/06/03 gilles64, Unknown Cache (2/2)}}\begin{cacheText}Co [{STF}] avec Domino50

Enfin le 64 tant attendu se dévoile. Après avoir passé la soirée à checker les coordonnées je suis prête pour aller sur le terrain dès le lendemain. Malheureusement l’orage et la grêle sont annoncés : j’hésite… J’y vais? je n’y vais pas?La tentation est trop forte j’y vais ...on verra bien!!!!Je décide d'attaquer le circuit 4 et c’est en descendant de voiture que j’aperçois Domino50 qui sort du Sentier. Nous décidons de faire le parcours ensemble. Trop contente!!! Nous serons STF sur la plupart des caches sauf une, oubliée par le redoutable Crispol40 qui nous a devancé , et qui nous permet d’inscrire un FTF. Toutes les caches sont trouvées sans trop de difficulté grâce aux hints et aux spoilers et les indices relevés pour la bonus et la super-bonus.Nous découvrons lors de la balade de beaux paysages avec de magnifiques arbres centenaires, des troupeaux de brebis (tête rousse), et des quads et quatre-quatre qui ont un peu \Quoted{pollué} le calme environnant!!!

Un grand merci Gilles pour cette super randonnée et pour la bonus qui nous a fait découvrir une vue à couper le souffle.\end{cacheText}

\cacheNumber{1608}\needspace{5\baselineskip}\cacheName{\href{http://coord.info/GC7P3GJ}{64 429 - Pyrénées Atlantique} — \href{http://coord.info/GC7P3GJ\Number{}773586768}{1608}}\cacheData{{2018/06/03 gilles64, Unknown Cache (2/2)}}\begin{cacheText}Co [{STF}] avec Domino50

Enfin le 64 tant attendu se dévoile. Après avoir passé la soirée à checker les coordonnées je suis prête pour aller sur le terrain dès le lendemain. Malheureusement l’orage et la grêle sont annoncés : j’hésite… J’y vais? je n’y vais pas?La tentation est trop forte j’y vais ...on verra bien!!!!Je décide d'attaquer le circuit 4 et c’est en descendant de voiture que j’aperçois Domino50 qui sort du Sentier. Nous décidons de faire le parcours ensemble. Trop contente!!! Nous serons STF sur la plupart des caches sauf une, oubliée par le redoutable Crispol40 qui nous a devancé , et qui nous permet d’inscrire un FTF. Toutes les caches sont trouvées sans trop de difficulté grâce aux hints et aux spoilers et les indices relevés pour la bonus et la super-bonus.Nous découvrons lors de la balade de beaux paysages avec de magnifiques arbres centenaires, des troupeaux de brebis (tête rousse), et des quads et quatre-quatre qui ont un peu \Quoted{pollué} le calme environnant!!!

Un grand merci Gilles pour cette super randonnée et pour la bonus qui nous a fait découvrir une vue à couper le souffle.\end{cacheText}

\cacheNumber{1609}\needspace{5\baselineskip}\cacheName{\href{http://coord.info/GC7P3GK}{64 430 - Pyrénées Atlantique} — \href{http://coord.info/GC7P3GK\Number{}773586603}{1609}}\cacheData{{2018/06/03 gilles64, Unknown Cache (2/2)}}\begin{cacheText}Co [{STF}] avec Domino50

Enfin le 64 tant attendu se dévoile. Après avoir passé la soirée à checker les coordonnées je suis prête pour aller sur le terrain dès le lendemain. Malheureusement l’orage et la grêle sont annoncés : j’hésite… J’y vais? je n’y vais pas?La tentation est trop forte j’y vais ...on verra bien!!!!Je décide d'attaquer le circuit 4 et c’est en descendant de voiture que j’aperçois Domino50 qui sort du Sentier. Nous décidons de faire le parcours ensemble. Trop contente!!! Nous serons STF sur la plupart des caches sauf une, oubliée par le redoutable Crispol40 qui nous a devancé , et qui nous permet d’inscrire un FTF. Toutes les caches sont trouvées sans trop de difficulté grâce aux hints et aux spoilers et les indices relevés pour la bonus et la super-bonus.Nous découvrons lors de la balade de beaux paysages avec de magnifiques arbres centenaires, des troupeaux de brebis (tête rousse), et des quads et quatre-quatre qui ont un peu \Quoted{pollué} le calme environnant!!!

Un grand merci Gilles pour cette super randonnée et pour la bonus qui nous a fait découvrir une vue à couper le souffle.\end{cacheText}

\cacheNumber{1610}\needspace{5\baselineskip}\cacheName{\href{http://coord.info/GC7P3GM}{64 431 - Pyrénées Atlantique} — \href{http://coord.info/GC7P3GM\Number{}773586451}{1610}}\cacheData{{2018/06/03 gilles64, Unknown Cache (2/2)}}\begin{cacheText}Co [{STF}] avec Domino50

Enfin le 64 tant attendu se dévoile. Après avoir passé la soirée à checker les coordonnées je suis prête pour aller sur le terrain dès le lendemain. Malheureusement l’orage et la grêle sont annoncés : j’hésite… J’y vais? je n’y vais pas?La tentation est trop forte j’y vais ...on verra bien!!!!Je décide d'attaquer le circuit 4 et c’est en descendant de voiture que j’aperçois Domino50 qui sort du Sentier. Nous décidons de faire le parcours ensemble. Trop contente!!! Nous serons STF sur la plupart des caches sauf une, oubliée par le redoutable Crispol40 qui nous a devancé , et qui nous permet d’inscrire un FTF. Toutes les caches sont trouvées sans trop de difficulté grâce aux hints et aux spoilers et les indices relevés pour la bonus et la super-bonus.Nous découvrons lors de la balade de beaux paysages avec de magnifiques arbres centenaires, des troupeaux de brebis (tête rousse), et des quads et quatre-quatre qui ont un peu \Quoted{pollué} le calme environnant!!!

Un grand merci Gilles pour cette super randonnée et pour la bonus qui nous a fait découvrir une vue à couper le souffle.\end{cacheText}

\cacheNumber{1611}\needspace{5\baselineskip}\cacheName{\href{http://coord.info/GC7P3GN}{64 432 - Pyrénées Atlantique} — \href{http://coord.info/GC7P3GN\Number{}773586100}{1611}}\cacheData{{2018/06/03 gilles64, Unknown Cache (2/2)}}\begin{cacheText}Co [{STF}] avec Domino50

Enfin le 64 tant attendu se dévoile. Après avoir passé la soirée à checker les coordonnées je suis prête pour aller sur le terrain dès le lendemain. Malheureusement l’orage et la grêle sont annoncés : j’hésite… J’y vais? je n’y vais pas?La tentation est trop forte j’y vais ...on verra bien!!!!Je décide d'attaquer le circuit 4 et c’est en descendant de voiture que j’aperçois Domino50 qui sort du Sentier. Nous décidons de faire le parcours ensemble. Trop contente!!! Nous serons STF sur la plupart des caches sauf une, oubliée par le redoutable Crispol40 qui nous a devancé , et qui nous permet d’inscrire un FTF. Toutes les caches sont trouvées sans trop de difficulté grâce aux hints et aux spoilers et les indices relevés pour la bonus et la super-bonus.Nous découvrons lors de la balade de beaux paysages avec de magnifiques arbres centenaires, des troupeaux de brebis (tête rousse), et des quads et quatre-quatre qui ont un peu \Quoted{pollué} le calme environnant!!!

Un grand merci Gilles pour cette super randonnée et pour la bonus qui nous a fait découvrir une vue à couper le souffle.\end{cacheText}

\cacheNumber{1612}\needspace{5\baselineskip}\cacheName{\href{http://coord.info/GC7P3GP}{64 433 - Pyrénées Atlantique} — \href{http://coord.info/GC7P3GP\Number{}773585931}{1612}}\cacheData{{2018/06/03 gilles64, Unknown Cache (2/2)}}\begin{cacheText}Co [{STF}] avec Domino50

Enfin le 64 tant attendu se dévoile. Après avoir passé la soirée à checker les coordonnées je suis prête pour aller sur le terrain dès le lendemain. Malheureusement l’orage et la grêle sont annoncés : j’hésite… J’y vais? je n’y vais pas?La tentation est trop forte j’y vais ...on verra bien!!!!Je décide d'attaquer le circuit 4 et c’est en descendant de voiture que j’aperçois Domino50 qui sort du Sentier. Nous décidons de faire le parcours ensemble. Trop contente!!! Nous serons STF sur la plupart des caches sauf une, oubliée par le redoutable Crispol40 qui nous a devancé , et qui nous permet d’inscrire un FTF. Toutes les caches sont trouvées sans trop de difficulté grâce aux hints et aux spoilers et les indices relevés pour la bonus et la super-bonus.Nous découvrons lors de la balade de beaux paysages avec de magnifiques arbres centenaires, des troupeaux de brebis (tête rousse), et des quads et quatre-quatre qui ont un peu \Quoted{pollué} le calme environnant!!!

Un grand merci Gilles pour cette super randonnée et pour la bonus qui nous a fait découvrir une vue à couper le souffle.\end{cacheText}

\cacheNumber{1613}\needspace{5\baselineskip}\cacheName{\href{http://coord.info/GC7P3GQ}{64 434 - Pyrénées Atlantique} — \href{http://coord.info/GC7P3GQ\Number{}773585739}{1613}}\cacheData{{2018/06/03 gilles64, Unknown Cache (2/2)}}\begin{cacheText}Co [{STF}] avec Domino50

Enfin le 64 tant attendu se dévoile. Après avoir passé la soirée à checker les coordonnées je suis prête pour aller sur le terrain dès le lendemain. Malheureusement l’orage et la grêle sont annoncés : j’hésite… J’y vais? je n’y vais pas?La tentation est trop forte j’y vais ...on verra bien!!!!Je décide d'attaquer le circuit 4 et c’est en descendant de voiture que j’aperçois Domino50 qui sort du Sentier. Nous décidons de faire le parcours ensemble. Trop contente!!! Nous serons STF sur la plupart des caches sauf une, oubliée par le redoutable Crispol40 qui nous a devancé , et qui nous permet d’inscrire un FTF. Toutes les caches sont trouvées sans trop de difficulté grâce aux hints et aux spoilers et les indices relevés pour la bonus et la super-bonus.Nous découvrons lors de la balade de beaux paysages avec de magnifiques arbres centenaires, des troupeaux de brebis (tête rousse), et des quads et quatre-quatre qui ont un peu \Quoted{pollué} le calme environnant!!!

Un grand merci Gilles pour cette super randonnée et pour la bonus qui nous a fait découvrir une vue à couper le souffle.\end{cacheText}

\cacheNumber{1614}\needspace{5\baselineskip}\cacheName{\href{http://coord.info/GC7P3GT}{64 435 -Pyrénées Atlantique} — \href{http://coord.info/GC7P3GT\Number{}773585526}{1614}}\cacheData{{2018/06/03 gilles64, Unknown Cache (2/2)}}\begin{cacheText}Co [{STF}] avec Domino50

Enfin le 64 tant attendu se dévoile. Après avoir passé la soirée à checker les coordonnées je suis prête pour aller sur le terrain dès le lendemain. Malheureusement l’orage et la grêle sont annoncés : j’hésite… J’y vais? je n’y vais pas?La tentation est trop forte j’y vais ...on verra bien!!!!Je décide d'attaquer le circuit 4 et c’est en descendant de voiture que j’aperçois Domino50 qui sort du Sentier. Nous décidons de faire le parcours ensemble. Trop contente!!! Nous serons STF sur la plupart des caches sauf une, oubliée par le redoutable Crispol40 qui nous a devancé , et qui nous permet d’inscrire un FTF. Toutes les caches sont trouvées sans trop de difficulté grâce aux hints et aux spoilers et les indices relevés pour la bonus et la super-bonus.Nous découvrons lors de la balade de beaux paysages avec de magnifiques arbres centenaires, des troupeaux de brebis (tête rousse), et des quads et quatre-quatre qui ont un peu \Quoted{pollué} le calme environnant!!!

Un grand merci Gilles pour cette super randonnée et pour la bonus qui nous a fait découvrir une vue à couper le souffle.\end{cacheText}

\cacheNumber{1615}\needspace{5\baselineskip}\cacheName{\href{http://coord.info/GC7P3GV}{64 436 - Pyrénées Atlantique} — \href{http://coord.info/GC7P3GV\Number{}773604538}{1615}}\cacheData{{2018/06/03 gilles64, Unknown Cache (2/2)}}\begin{cacheText}Enfin le 64 tant attendu se dévoile. Après avoir passé la soirée à checker les coordonnées je suis prête pour aller sur le terrain dès le lendemain. Malheureusement l’orage et la grêle sont annoncés : j’hésite… J’y vais? je n’y vais pas?La tentation est trop forte j’y vais ...on verra bien!!!!Je décide d'attaquer le circuit 4 et c’est en descendant de voiture que j’aperçois Domino50 qui sort du Sentier. Nous décidons de faire le parcours ensemble. Trop contente!!! Nous serons STF sur la plupart des caches sauf une, oubliée par le redoutable Crispol40 qui nous a devancé , et qui nous permet d’inscrire un FTF. Toutes les caches sont trouvées sans trop de difficulté grâce aux hints et aux spoilers et les indices relevés pour la bonus et la super-bonus.Nous découvrons lors de la balade de beaux paysages avec de magnifiques arbres centenaires, des troupeaux de brebis (tête rousse), et des quads et quatre-quatre qui ont un peu \Quoted{pollué} le calme environnant!!!

Un grand merci Gilles pour cette super randonnée et pour la bonus qui nous a fait découvrir une vue à couper le souffle.\end{cacheText}

\cacheNumber{1616}\needspace{5\baselineskip}\cacheName{\href{http://coord.info/GC7P3GW}{64 437 - Pyrénées Atlantique} — \href{http://coord.info/GC7P3GW\Number{}773647548}{1616}}\cacheData{{2018/06/03 gilles64, Unknown Cache (2/2.5)}}\begin{cacheText}Enfin le 64 tant attendu se dévoile. Après avoir passé la soirée à checker les coordonnées je suis prête pour aller sur le terrain dès le lendemain. Malheureusement l’orage et la grêle sont annoncés : j’hésite… J’y vais? je n’y vais pas?La tentation est trop forte j’y vais ...on verra bien!!!!Je décide d'attaquer le circuit 4 et c’est en descendant de voiture que j’aperçois Domino50 qui sort du Sentier. Nous décidons de faire le parcours ensemble. Trop contente!!! Nous serons STF sur la plupart des caches sauf une, oubliée par le redoutable Crispol40 qui nous a devancé , et qui nous permet d’inscrire un FTF. Toutes les caches sont trouvées sans trop de difficulté grâce aux hints et aux spoilers et les indices relevés pour la bonus et la super-bonus.Nous découvrons lors de la balade de beaux paysages avec de magnifiques arbres centenaires, des troupeaux de brebis (tête rousse), et des quads et quatre-quatre qui ont un peu \Quoted{pollué} le calme environnant!!!



Un grand merci Gilles pour cette super randonnée et pour la bonus qui nous a fait découvrir une vue à couper le souffle.\end{cacheText}

\cacheNumber{1617}\needspace{5\baselineskip}\cacheName{\href{http://coord.info/GC7P3GY}{64 438 bonus - Pyrénées Atlantique} — \href{http://coord.info/GC7P3GY\Number{}773601196}{1617}}\cacheData{{2018/06/03 gilles64, Unknown Cache (3.5/2)}}\begin{cacheText}Co [{STF}] avec Domino50

Enfin le 64 tant attendu se dévoile. Après avoir passé la soirée à checker les coordonnées je suis prête pour aller sur le terrain dès le lendemain. Malheureusement l’orage et la grêle sont annoncés : j’hésite… J’y vais? je n’y vais pas?La tentation est trop forte j’y vais ...on verra bien!!!!Je décide d'attaquer le circuit 4 et c’est en descendant de voiture que j’aperçois Domino50 qui sort du Sentier. Nous décidons de faire le parcours ensemble. Trop contente!!! Nous serons STF sur la plupart des caches sauf une, oubliée par le redoutable Crispol40 qui nous a devancé , et qui nous permet d’inscrire un FTF. Toutes les caches sont trouvées sans trop de difficulté grâce aux hints et aux spoilers et les indices relevés pour la bonus et la super-bonus.Nous découvrons lors de la balade de beaux paysages avec de magnifiques arbres centenaires, des troupeaux de brebis (tête rousse), et des quads et quatre-quatre qui ont un peu \Quoted{pollué} le calme environnant!!!

Un grand merci Gilles pour cette super randonnée et pour la bonus qui nous a fait découvrir une vue à couper le souffle. Un PF pour l'ensemble de la boucle.\end{cacheText}

\cacheNumber{1618}\needspace{5\baselineskip}\cacheName{\href{http://coord.info/GC3G3KM}{GR8\Number{}80} — \href{http://coord.info/GC3G3KM\Number{}773780389}{1618}}\cacheData{{2018/06/04 Peyo64, Traditional Cache (2/1.5)}}\begin{cacheText}De retour du 64 , en compagnie de Dune33, nous nous arrêtons effacer ce vilain point bleu sur la carte. Avec ce temps il n'y a aucun moldu à l'horizon et nous pouvons chercher tranquille. Merci Peyo.\end{cacheText}

\cacheNumber{1619}\needspace{5\baselineskip}\cacheName{\href{http://coord.info/GC7KW19}{64 204 - Pyrénées Atlantique} — \href{http://coord.info/GC7KW19\Number{}773772046}{1619}}\cacheData{{2018/06/04 gilles64, Unknown Cache (2/1.5)}}\begin{cacheText}Rendez-vous est pris avec Dune33 pour faire le parcours 2 du 64. Le temps est à la pluie mais nous ne nous décourageons pas. Nous attaquons le circuit à l’envers .Il semble facile mais par la suite les choses se compliquent : une vraie patinoire!!!Quelques glissades ont poussé Dédé à nous équiper de jolis bâtons de noisetier pour éviter les chutes. Merci Dédé. Nous aurons le plaisir de croiser Crispol 40 qui nous a soufflé le FTF de la 224 sous le nez (grrrr ) car nous étions sur le mauvais côté de la rive!!!Domino nous a également croisé ( arrivée plus tard sur le circuit et gros problème de réseau).Malgré le mauvais temps nous avons passé une excellente journée:2 caches nous ont résisté (217 et 214 car la végétation a pris le dessus).Merci Gilles pour tout ce travail ,nous avons adoré .\end{cacheText}

\cacheNumber{1620}\needspace{5\baselineskip}\cacheName{\href{http://coord.info/GC7KW1D}{64 206 - Pyrénées Atlantique} — \href{http://coord.info/GC7KW1D\Number{}773771844}{1620}}\cacheData{{2018/06/04 gilles64, Unknown Cache (2/1.5)}}\begin{cacheText}Rendez-vous est pris avec Dune33 pour faire le parcours 2 du 64. Le temps est à la pluie mais nous ne nous décourageons pas. Nous attaquons le circuit à l’envers .Il semble facile mais par la suite les choses se compliquent : une vraie patinoire!!!Quelques glissades ont poussé Dédé à nous équiper de jolis bâtons de noisetier pour éviter les chutes. Merci Dédé. Nous aurons le plaisir de croiser Crispol 40 qui nous a soufflé le FTF de la 224 sous le nez (grrrr ) car nous étions sur le mauvais côté de la rive!!!Domino nous a également croisé ( arrivée plus tard sur le circuit et gros problème de réseau).Malgré le mauvais temps nous avons passé une excellente journée:2 caches nous ont résisté (217 et 214 car la végétation a pris le dessus).Merci Gilles pour tout ce travail ,nous avons adoré .\end{cacheText}

\cacheNumber{1621}\needspace{5\baselineskip}\cacheName{\href{http://coord.info/GC7KW1G}{64 207 - Pyrénées Atlantique} — \href{http://coord.info/GC7KW1G\Number{}773771668}{1621}}\cacheData{{2018/06/04 gilles64, Unknown Cache (2/1.5)}}\begin{cacheText}Rendez-vous est pris avec Dune33 pour faire le parcours 2 du 64. Le temps est à la pluie mais nous ne nous décourageons pas. Nous attaquons le circuit à l’envers .Il semble facile mais par la suite les choses se compliquent : une vraie patinoire!!!Quelques glissades ont poussé Dédé à nous équiper de jolis bâtons de noisetier pour éviter les chutes. Merci Dédé. Nous aurons le plaisir de croiser Crispol 40 qui nous a soufflé le FTF de la 224 sous le nez (grrrr ) car nous étions sur le mauvais côté de la rive!!!Domino nous a également croisé ( arrivée plus tard sur le circuit et gros problème de réseau).Malgré le mauvais temps nous avons passé une excellente journée:2 caches nous ont résisté (217 et 214 car la végétation a pris le dessus).Merci Gilles pour tout ce travail ,nous avons adoré .\end{cacheText}

\cacheNumber{1622}\needspace{5\baselineskip}\cacheName{\href{http://coord.info/GC7KW1K}{64 208 - Pyrénées Atlantique} — \href{http://coord.info/GC7KW1K\Number{}773771353}{1622}}\cacheData{{2018/06/04 gilles64, Unknown Cache (2/1.5)}}\begin{cacheText}Rendez-vous est pris avec Dune33 pour faire le parcours 2 du 64. Le temps est à la pluie mais nous ne nous décourageons pas. Nous attaquons le circuit à l’envers .Il semble facile mais par la suite les choses se compliquent : une vraie patinoire!!!Quelques glissades ont poussé Dédé à nous équiper de jolis bâtons de noisetier pour éviter les chutes. Merci Dédé. Nous aurons le plaisir de croiser Crispol 40 qui nous a soufflé le FTF de la 224 sous le nez (grrrr ) car nous étions sur le mauvais côté de la rive!!!Domino nous a également croisé ( arrivée plus tard sur le circuit et gros problème de réseau).Malgré le mauvais temps nous avons passé une excellente journée:2 caches nous ont résisté (217 et 214 car la végétation a pris le dessus).Merci Gilles pour tout ce travail ,nous avons adoré .\end{cacheText}

\cacheNumber{1623}\needspace{5\baselineskip}\cacheName{\href{http://coord.info/GC7KW1T}{64 209 - Pyrénées Atlantique} — \href{http://coord.info/GC7KW1T\Number{}773771208}{1623}}\cacheData{{2018/06/04 gilles64, Unknown Cache (2/1.5)}}\begin{cacheText}Rendez-vous est pris avec Dune33 pour faire le parcours 2 du 64. Le temps est à la pluie mais nous ne nous décourageons pas. Nous attaquons le circuit à l’envers .Il semble facile mais par la suite les choses se compliquent : une vraie patinoire!!!Quelques glissades ont poussé Dédé à nous équiper de jolis bâtons de noisetier pour éviter les chutes. Merci Dédé. Nous aurons le plaisir de croiser Crispol 40 qui nous a soufflé le FTF de la 224 sous le nez (grrrr ) car nous étions sur le mauvais côté de la rive!!!Domino nous a également croisé ( arrivée plus tard sur le circuit et gros problème de réseau).Malgré le mauvais temps nous avons passé une excellente journée:2 caches nous ont résisté (217 et 214 car la végétation a pris le dessus).Merci Gilles pour tout ce travail ,nous avons adoré .\end{cacheText}

\cacheNumber{1624}\needspace{5\baselineskip}\cacheName{\href{http://coord.info/GC7KW1V}{64 210 - Pyrénées Atlantique} — \href{http://coord.info/GC7KW1V\Number{}773770963}{1624}}\cacheData{{2018/06/04 gilles64, Unknown Cache (2/1.5)}}\begin{cacheText}Rendez-vous est pris avec Dune33 pour faire le parcours 2 du 64. Le temps est à la pluie mais nous ne nous décourageons pas. Nous attaquons le circuit à l’envers .Il semble facile mais par la suite les choses se compliquent : une vraie patinoire!!!Quelques glissades ont poussé Dédé à nous équiper de jolis bâtons de noisetier pour éviter les chutes. Merci Dédé. Nous aurons le plaisir de croiser Crispol 40 qui nous a soufflé le FTF de la 224 sous le nez (grrrr ) car nous étions sur le mauvais côté de la rive!!!Domino nous a également croisé ( arrivée plus tard sur le circuit et gros problème de réseau).Malgré le mauvais temps nous avons passé une excellente journée:2 caches nous ont résisté (217 et 214 car la végétation a pris le dessus).Merci Gilles pour tout ce travail ,nous avons adoré .\end{cacheText}

\cacheNumber{1625}\needspace{5\baselineskip}\cacheName{\href{http://coord.info/GC7KW1X}{64 211 - Pyrénées Atlantique} — \href{http://coord.info/GC7KW1X\Number{}773770776}{1625}}\cacheData{{2018/06/04 gilles64, Unknown Cache (2/1.5)}}\begin{cacheText}Rendez-vous est pris avec Dune33 pour faire le parcours 2 du 64. Le temps est à la pluie mais nous ne nous décourageons pas. Nous attaquons le circuit à l’envers .Il semble facile mais par la suite les choses se compliquent : une vraie patinoire!!!Quelques glissades ont poussé Dédé à nous équiper de jolis bâtons de noisetier pour éviter les chutes. Merci Dédé. Nous aurons le plaisir de croiser Crispol 40 qui nous a soufflé le FTF de la 224 sous le nez (grrrr ) car nous étions sur le mauvais côté de la rive!!!Domino nous a également croisé ( arrivée plus tard sur le circuit et gros problème de réseau).Malgré le mauvais temps nous avons passé une excellente journée:2 caches nous ont résisté (217 et 214 car la végétation a pris le dessus).Merci Gilles pour tout ce travail ,nous avons adoré .\end{cacheText}

\cacheNumber{1626}\needspace{5\baselineskip}\cacheName{\href{http://coord.info/GC7KW1Y}{64 212 - Pyrénées Atlantique} — \href{http://coord.info/GC7KW1Y\Number{}773770637}{1626}}\cacheData{{2018/06/04 gilles64, Unknown Cache (2/1.5)}}\begin{cacheText}Rendez-vous est pris avec Dune33 pour faire le parcours 2 du 64. Le temps est à la pluie mais nous ne nous décourageons pas. Nous attaquons le circuit à l’envers .Il semble facile mais par la suite les choses se compliquent : une vraie patinoire!!!Quelques glissades ont poussé Dédé à nous équiper de jolis bâtons de noisetier pour éviter les chutes. Merci Dédé. Nous aurons le plaisir de croiser Crispol 40 qui nous a soufflé le FTF de la 224 sous le nez (grrrr ) car nous étions sur le mauvais côté de la rive!!!Domino nous a également croisé ( arrivée plus tard sur le circuit et gros problème de réseau).Malgré le mauvais temps nous avons passé une excellente journée:2 caches nous ont résisté (217 et 214 car la végétation a pris le dessus).Merci Gilles pour tout ce travail ,nous avons adoré .\end{cacheText}

\cacheNumber{1627}\needspace{5\baselineskip}\cacheName{\href{http://coord.info/GC7KW20}{64 213 - Pyrénées Atlantique} — \href{http://coord.info/GC7KW20\Number{}773762635}{1627}}\cacheData{{2018/06/04 gilles64, Unknown Cache (2/2)}}\begin{cacheText}Rendez-vous est pris avec Dune33 pour faire le parcours 2 du 64. Le temps est à la pluie mais nous ne nous décourageons pas. Nous attaquons le circuit à l’envers .Il semble facile mais par la suite les choses se compliquent : une vraie patinoire!!!Quelques glissades ont poussé Dédé à nous équiper de jolis bâtons de noisetier pour éviter les chutes. Merci Dédé. Nous aurons le plaisir de croiser Crispol 40 qui nous a soufflé le FTF de la 224 sous le nez (grrrr ) car nous étions sur le mauvais côté de la rive!!!Domino nous a également croisé ( arrivée plus tard sur le circuit et gros problème de réseau).Malgré le mauvais temps nous avons passé une excellente journée:2 caches nous ont résisté (217 et 214 car la végétation a pris le dessus).Merci Gilles pour tout ce travail ,nous avons adoré .\end{cacheText}

\cacheNumber{1628}\needspace{5\baselineskip}\cacheName{\href{http://coord.info/GC7KW23}{64 215 - Pyrénées Atlantique} — \href{http://coord.info/GC7KW23\Number{}773770294}{1628}}\cacheData{{2018/06/04 gilles64, Unknown Cache (2/2)}}\begin{cacheText}Rendez-vous est pris avec Dune33 pour faire le parcours 2 du 64. Le temps est à la pluie mais nous ne nous décourageons pas. Nous attaquons le circuit à l’envers .Il semble facile mais par la suite les choses se compliquent : une vraie patinoire!!!Quelques glissades ont poussé Dédé à nous équiper de jolis bâtons de noisetier pour éviter les chutes. Merci Dédé. Nous aurons le plaisir de croiser Crispol 40 qui nous a soufflé le FTF de la 224 sous le nez (grrrr ) car nous étions sur le mauvais côté de la rive!!!Domino nous a également croisé ( arrivée plus tard sur le circuit et gros problème de réseau).Malgré le mauvais temps nous avons passé une excellente journée:2 caches nous ont résisté (217 et 214 car la végétation a pris le dessus).Merci Gilles pour tout ce travail ,nous avons adoré .\end{cacheText}

\cacheNumber{1629}\needspace{5\baselineskip}\cacheName{\href{http://coord.info/GC7KW25}{64 216 - Pyrénées Atlantique} — \href{http://coord.info/GC7KW25\Number{}773769901}{1629}}\cacheData{{2018/06/04 gilles64, Unknown Cache (2/2.5)}}\begin{cacheText}Rendez-vous est pris avec Dune33 pour faire le parcours 2 du 64. Le temps est à la pluie mais nous ne nous décourageons pas. Nous attaquons le circuit à l’envers .Il semble facile mais par la suite les choses se compliquent : une vraie patinoire!!!Quelques glissades ont poussé Dédé à nous équiper de jolis bâtons de noisetier pour éviter les chutes. Merci Dédé. Nous aurons le plaisir de croiser Crispol 40 qui nous a soufflé le FTF de la 224 sous le nez (grrrr ) car nous étions sur le mauvais côté de la rive!!!Domino nous a également croisé ( arrivée plus tard sur le circuit et gros problème de réseau).Malgré le mauvais temps nous avons passé une excellente journée:2 caches nous ont résisté (217 et 214 car la végétation a pris le dessus).Merci Gilles pour tout ce travail ,nous avons adoré .\end{cacheText}

\cacheNumber{1630}\needspace{5\baselineskip}\cacheName{\href{http://coord.info/GC7KW27}{64 218 - Pyrénées Atlantique} — \href{http://coord.info/GC7KW27\Number{}773767726}{1630}}\cacheData{{2018/06/04 gilles64, Unknown Cache (2/1.5)}}\begin{cacheText}Rendez-vous est pris avec Dune33 pour faire le parcours 2 du 64. Le temps est à la pluie mais nous ne nous décourageons pas. Nous attaquons le circuit à l’envers .Il semble facile mais par la suite les choses se compliquent : une vraie patinoire!!!Quelques glissades ont poussé Dédé à nous équiper de jolis bâtons de noisetier pour éviter les chutes. Merci Dédé. Nous aurons le plaisir de croiser Crispol 40 qui nous a soufflé le FTF de la 224 sous le nez (grrrr ) car nous étions sur le mauvais côté de la rive!!!Domino nous a également croisé ( arrivée plus tard sur le circuit et gros problème de réseau).Malgré le mauvais temps nous avons passé une excellente journée:2 caches nous ont résisté (217 et 214 car la végétation a pris le dessus).Merci Gilles pour tout ce travail ,nous avons adoré .\end{cacheText}

\cacheNumber{1631}\needspace{5\baselineskip}\cacheName{\href{http://coord.info/GC7KW28}{64 219 - Pyrénées Atlantique} — \href{http://coord.info/GC7KW28\Number{}773767603}{1631}}\cacheData{{2018/06/04 gilles64, Unknown Cache (2/2)}}\begin{cacheText}Rendez-vous est pris avec Dune33 pour faire le parcours 2 du 64. Le temps est à la pluie mais nous ne nous décourageons pas. Nous attaquons le circuit à l’envers .Il semble facile mais par la suite les choses se compliquent : une vraie patinoire!!!Quelques glissades ont poussé Dédé à nous équiper de jolis bâtons de noisetier pour éviter les chutes. Merci Dédé. Nous aurons le plaisir de croiser Crispol 40 qui nous a soufflé le FTF de la 224 sous le nez (grrrr ) car nous étions sur le mauvais côté de la rive!!!Domino nous a également croisé ( arrivée plus tard sur le circuit et gros problème de réseau).Malgré le mauvais temps nous avons passé une excellente journée:2 caches nous ont résisté (217 et 214 car la végétation a pris le dessus).Merci Gilles pour tout ce travail ,nous avons adoré .\end{cacheText}

\cacheNumber{1632}\needspace{5\baselineskip}\cacheName{\href{http://coord.info/GC7KW2A}{64 220 - Pyrénées Atlantique} — \href{http://coord.info/GC7KW2A\Number{}773767307}{1632}}\cacheData{{2018/06/04 gilles64, Unknown Cache (2/2)}}\begin{cacheText}Rendez-vous est pris avec Dune33 pour faire le parcours 2 du 64. Le temps est à la pluie mais nous ne nous décourageons pas. Nous attaquons le circuit à l’envers .Il semble facile mais par la suite les choses se compliquent : une vraie patinoire!!!Quelques glissades ont poussé Dédé à nous équiper de jolis bâtons de noisetier pour éviter les chutes. Merci Dédé. Nous aurons le plaisir de croiser Crispol 40 qui nous a soufflé le FTF de la 224 sous le nez (grrrr ) car nous étions sur le mauvais côté de la rive!!!Domino nous a également croisé ( arrivée plus tard sur le circuit et gros problème de réseau).Malgré le mauvais temps nous avons passé une excellente journée:2 caches nous ont résisté (217 et 214 car la végétation a pris le dessus).Merci Gilles pour tout ce travail ,nous avons adoré .\end{cacheText}

\cacheNumber{1633}\needspace{5\baselineskip}\cacheName{\href{http://coord.info/GC7KW3M}{64 221 - Pyrénées Atlantique} — \href{http://coord.info/GC7KW3M\Number{}773766538}{1633}}\cacheData{{2018/06/04 gilles64, Unknown Cache (2/2)}}\begin{cacheText}Rendez-vous est pris avec Dune33 pour faire le parcours 2 du 64. Le temps est à la pluie mais nous ne nous décourageons pas. Nous attaquons le circuit à l’envers .Il semble facile mais par la suite les choses se compliquent : une vraie patinoire!!!Quelques glissades ont poussé Dédé à nous équiper de jolis bâtons de noisetier pour éviter les chutes. Merci Dédé. Nous aurons le plaisir de croiser Crispol 40 qui nous a soufflé le FTF de la 224 sous le nez (grrrr ) car nous étions sur le mauvais côté de la rive!!!Domino nous a également croisé ( arrivée plus tard sur le circuit et gros problème de réseau).Malgré le mauvais temps nous avons passé une excellente journée:2 caches nous ont résisté (217 et 214 car la végétation a pris le dessus).Merci Gilles pour tout ce travail ,nous avons adoré .\end{cacheText}

\cacheNumber{1634}\needspace{5\baselineskip}\cacheName{\href{http://coord.info/GC7KW3T}{64 222 - Pyrénées Atlantique} — \href{http://coord.info/GC7KW3T\Number{}773766205}{1634}}\cacheData{{2018/06/04 gilles64, Unknown Cache (2/2)}}\begin{cacheText}Rendez-vous est pris avec Dune33 pour faire le parcours 2 du 64. Le temps est à la pluie mais nous ne nous décourageons pas. Nous attaquons le circuit à l’envers .Il semble facile mais par la suite les choses se compliquent : une vraie patinoire!!!Quelques glissades ont poussé Dédé à nous équiper de jolis bâtons de noisetier pour éviter les chutes. Merci Dédé. Nous aurons le plaisir de croiser Crispol 40 qui nous a soufflé le FTF de la 224 sous le nez (grrrr ) car nous étions sur le mauvais côté de la rive!!!Domino nous a également croisé ( arrivée plus tard sur le circuit et gros problème de réseau).Malgré le mauvais temps nous avons passé une excellente journée:2 caches nous ont résisté (217 et 214 car la végétation a pris le dessus).Merci Gilles pour tout ce travail ,nous avons adoré .\end{cacheText}

\cacheNumber{1635}\needspace{5\baselineskip}\cacheName{\href{http://coord.info/GC7KW3W}{64 223 - Pyrénées Atlantique} — \href{http://coord.info/GC7KW3W\Number{}773764400}{1635}}\cacheData{{2018/06/04 gilles64, Unknown Cache (2/2)}}\begin{cacheText}Rendez-vous est pris avec Dune33 pour faire le parcours 2 du 64. Le temps est à la pluie mais nous ne nous décourageons pas. Nous attaquons le circuit à l’envers .Il semble facile mais par la suite les choses se compliquent : une vraie patinoire!!!Quelques glissades ont poussé Dédé à nous équiper de jolis bâtons de noisetier pour éviter les chutes. Merci Dédé. Nous aurons le plaisir de croiser Crispol 40 qui nous a soufflé le FTF de la 224 sous le nez (grrrr ) car nous étions sur le mauvais côté de la rive!!!Domino nous a également croisé ( arrivée plus tard sur le circuit et gros problème de réseau).Malgré le mauvais temps nous avons passé une excellente journée:2 caches nous ont résisté (217 et 214 car la végétation a pris le dessus).Merci Gilles pour tout ce travail ,nous avons adoré .\end{cacheText}

\cacheNumber{1636}\needspace{5\baselineskip}\cacheName{\href{http://coord.info/GC7KW3Z}{64 224 - Pyrénées Atlantique} — \href{http://coord.info/GC7KW3Z\Number{}773763742}{1636}}\cacheData{{2018/06/04 gilles64, Unknown Cache (2/2)}}\begin{cacheText}Co (STF) avec Dune33

Rendez-vous est pris avec Dune33 pour faire le parcours 2 du 64. Le temps est à la pluie mais nous ne nous décourageons pas. Nous attaquons le circuit à l’envers .Il semble facile mais par la suite les choses se compliquent : une vraie patinoire!!!Quelques glissades ont poussé Dédé à nous équiper de jolis bâtons de noisetier pour éviter les chutes. Merci Dédé. Nous aurons le plaisir de croiser Crispol 40 qui nous a soufflé le FTF de la 224 sous le nez (grrrr ) car nous étions sur le mauvais côté de la rive!!!Domino nous a également croisé ( arrivée plus tard sur le circuit et gros problème de réseau).Malgré le mauvais temps nous avons passé une excellente journée:2 caches nous ont résisté (217 et 214 car la végétation a pris le dessus).Merci Gilles pour tout ce travail ,nous avons adoré .\end{cacheText}

\cacheNumber{1637}\needspace{5\baselineskip}\cacheName{\href{http://coord.info/GC7KW41}{64 225 - Pyrénées Atlantique} — \href{http://coord.info/GC7KW41\Number{}773761384}{1637}}\cacheData{{2018/06/04 gilles64, Unknown Cache (2/2)}}\begin{cacheText}Rendez-vous est pris avec Dune33 pour faire le parcours 2 du 64. Le temps est à la pluie mais nous ne nous décourageons pas. Nous attaquons le circuit à l’envers .Il semble facile mais par la suite les choses se compliquent : une vraie patinoire!!!Quelques glissades ont poussé Dédé à nous équiper de jolis bâtons de noisetier pour éviter les chutes. Merci Dédé. Nous aurons le plaisir de croiser Crispol 40 qui nous a soufflé le FTF de la 224 sous le nez (grrrr ) car nous étions sur le mauvais côté de la rive!!!Domino nous a également croisé ( arrivée plus tard sur le circuit et gros problème de réseau).Malgré le mauvais temps nous avons passé une excellente journée:2 caches nous ont résisté (217 et 214 car la végétation a pris le dessus).Merci Gilles pour tout ce travail ,nous avons adoré .\end{cacheText}

\cacheNumber{1638}\needspace{5\baselineskip}\cacheName{\href{http://coord.info/GC7KW44}{64 226 - Pyrénées Atlantique} — \href{http://coord.info/GC7KW44\Number{}773654109}{1638}}\cacheData{{2018/06/04 gilles64, Unknown Cache (2/2)}}\begin{cacheText}Rendez-vous est pris avec Dune33 pour faire le parcours 2 du 64. Le temps est à la pluie mais nous ne nous décourageons pas. Nous attaquons le circuit à l’envers .Il semble facile mais par la suite les choses se compliquent : une vraie patinoire!!!Quelques glissades ont poussé Dédé à nous équiper de jolis bâtons de noisetier pour éviter les chutes. Merci Dédé. Nous aurons le plaisir de croiser Crispol 40 qui nous a soufflé le FTF de la 224 sous le nez (grrrr ) car nous étions sur le mauvais côté de la rive!!!Domino nous a également croisé ( arrivée plus tard sur le circuit et gros problème de réseau).Malgré le mauvais temps nous avons passé une excellente journée:2 caches nous ont résisté (217 et 214 car la végétation a pris le dessus).Merci Gilles pour tout ce travail ,nous avons adoré .\end{cacheText}

\cacheNumber{1639}\needspace{5\baselineskip}\cacheName{\href{http://coord.info/GC7KWRM}{64 227 - Pyrénées Atlantique} — \href{http://coord.info/GC7KWRM\Number{}773653908}{1639}}\cacheData{{2018/06/04 gilles64, Unknown Cache (2/2)}}\begin{cacheText}Co (STF) avec Dune33

Rendez-vous est pris avec Dune33 pour faire le parcours 2 du 64. Le temps est à la pluie mais nous ne nous décourageons pas. Nous attaquons le circuit à l’envers .Il semble facile mais par la suite les choses se compliquent : une vraie patinoire!!!Quelques glissades ont poussé Dédé à nous équiper de jolis bâtons de noisetier pour éviter les chutes. Merci Dédé. Nous aurons le plaisir de croiser Crispol 40 qui nous a soufflé le FTF de la 224 sous le nez (grrrr ) car nous étions sur le mauvais côté de la rive!!!Domino nous a également croisé ( arrivée plus tard sur le circuit et gros problème de réseau).Malgré le mauvais temps nous avons passé une excellente journée:2 caches nous ont résisté (217 et 214 car la végétation a pris le dessus).Merci Gilles pour tout ce travail ,nous avons adoré .\end{cacheText}

\cacheNumber{1640}\needspace{5\baselineskip}\cacheName{\href{http://coord.info/GC7KWRP}{64 228 - Pyrénées Atlantique} — \href{http://coord.info/GC7KWRP\Number{}773653737}{1640}}\cacheData{{2018/06/04 gilles64, Unknown Cache (2/2)}}\begin{cacheText}Rendez-vous est pris avec Dune33 pour faire le parcours 2 du 64. Le temps est à la pluie mais nous ne nous décourageons pas. Nous attaquons le circuit à l’envers .Il semble facile mais par la suite les choses se compliquent : une vraie patinoire!!!Quelques glissades ont poussé Dédé à nous équiper de jolis bâtons de noisetier pour éviter les chutes. Merci Dédé. Nous aurons le plaisir de croiser Crispol 40 qui nous a soufflé le FTF de la 224 sous le nez (grrrr ) car nous étions sur le mauvais côté de la rive!!!Domino nous a également croisé ( arrivée plus tard sur le circuit et gros problème de réseau).Malgré le mauvais temps nous avons passé une excellente journée:2 caches nous ont résisté (217 et 214 car la végétation a pris le dessus).Merci Gilles pour tout ce travail ,nous avons adoré .\end{cacheText}

\cacheNumber{1641}\needspace{5\baselineskip}\cacheName{\href{http://coord.info/GC7KWRW}{64229 - Pyrénées Atlantique} — \href{http://coord.info/GC7KWRW\Number{}773653623}{1641}}\cacheData{{2018/06/04 gilles64, Unknown Cache (2/2)}}\begin{cacheText}Rendez-vous est pris avec Dune33 pour faire le parcours 2 du 64. Le temps est à la pluie mais nous ne nous décourageons pas. Nous attaquons le circuit à l’envers .Il semble facile mais par la suite les choses se compliquent : une vraie patinoire!!!Quelques glissades ont poussé Dédé à nous équiper de jolis bâtons de noisetier pour éviter les chutes. Merci Dédé. Nous aurons le plaisir de croiser Crispol 40 qui nous a soufflé le FTF de la 224 sous le nez (grrrr ) car nous étions sur le mauvais côté de la rive!!!Domino nous a également croisé ( arrivée plus tard sur le circuit et gros problème de réseau).Malgré le mauvais temps nous avons passé une excellente journée:2 caches nous ont résisté (217 et 214 car la végétation a pris le dessus).Merci Gilles pour tout ce travail ,nous avons adoré .\end{cacheText}

\cacheNumber{1642}\needspace{5\baselineskip}\cacheName{\href{http://coord.info/GC7KWRY}{64 230 - Pyrénées Atlantique} — \href{http://coord.info/GC7KWRY\Number{}773653533}{1642}}\cacheData{{2018/06/04 gilles64, Unknown Cache (2/2)}}\begin{cacheText}Rendez-vous est pris avec Dune33 pour faire le parcours 2 du 64. Le temps est à la pluie mais nous ne nous décourageons pas. Nous attaquons le circuit à l’envers .Il semble facile mais par la suite les choses se compliquent : une vraie patinoire!!!Quelques glissades ont poussé Dédé à nous équiper de jolis bâtons de noisetier pour éviter les chutes. Merci Dédé. Nous aurons le plaisir de croiser Crispol 40 qui nous a soufflé le FTF de la 224 sous le nez (grrrr ) car nous étions sur le mauvais côté de la rive!!!Domino nous a également croisé ( arrivée plus tard sur le circuit et gros problème de réseau).Malgré le mauvais temps nous avons passé une excellente journée:2 caches nous ont résisté (217 et 214 car la végétation a pris le dessus).Merci Gilles pour tout ce travail ,nous avons adoré .\end{cacheText}

\cacheNumber{1643}\needspace{5\baselineskip}\cacheName{\href{http://coord.info/GC7KWW5}{64 231 - Pyrénées Atlantique} — \href{http://coord.info/GC7KWW5\Number{}773653397}{1643}}\cacheData{{2018/06/04 gilles64, Unknown Cache (2/2)}}\begin{cacheText}Rendez-vous est pris avec Dune33 pour faire le parcours 2 du 64. Le temps est à la pluie mais nous ne nous décourageons pas. Nous attaquons le circuit à l’envers .Il semble facile mais par la suite les choses se compliquent : une vraie patinoire!!!Quelques glissades ont poussé Dédé à nous équiper de jolis bâtons de noisetier pour éviter les chutes. Merci Dédé. Nous aurons le plaisir de croiser Crispol 40 qui nous a soufflé le FTF de la 224 sous le nez (grrrr ) car nous étions sur le mauvais côté de la rive!!!Domino nous a également croisé ( arrivée plus tard sur le circuit et gros problème de réseau).Malgré le mauvais temps nous avons passé une excellente journée:2 caches nous ont résisté (217 et 214 car la végétation a pris le dessus).Merci Gilles pour tout ce travail ,nous avons adoré .\end{cacheText}

\cacheNumber{1644}\needspace{5\baselineskip}\cacheName{\href{http://coord.info/GC7KWWB}{64 232 - Pyrénées Atlantique} — \href{http://coord.info/GC7KWWB\Number{}773653225}{1644}}\cacheData{{2018/06/04 gilles64, Unknown Cache (2/2)}}\begin{cacheText}Co (STF)avec Dune33

Rendez-vous est pris avec Dune33 pour faire le parcours 2 du 64. Le temps est à la pluie mais nous ne nous décourageons pas. Nous attaquons le circuit à l’envers .Il semble facile mais par la suite les choses se compliquent : une vraie patinoire!!!Quelques glissades ont poussé Dédé à nous équiper de jolis bâtons de noisetier pour éviter les chutes. Merci Dédé. Nous aurons le plaisir de croiser Crispol 40 qui nous a soufflé le FTF de la 224 sous le nez (grrrr ) car nous étions sur le mauvais côté de la rive!!!Domino nous a également croisé ( arrivée plus tard sur le circuit et gros problème de réseau).Malgré le mauvais temps nous avons passé une excellente journée:2 caches nous ont résisté (217 et 214 car la végétation a pris le dessus).Merci Gilles pour tout ce travail ,nous avons adoré .\end{cacheText}

\cacheNumber{1645}\needspace{5\baselineskip}\cacheName{\href{http://coord.info/GC7KWWD}{64233 - Pyrénées Atlantique} — \href{http://coord.info/GC7KWWD\Number{}773652988}{1645}}\cacheData{{2018/06/04 gilles64, Unknown Cache (2/2)}}\begin{cacheText}Rendez-vous est pris avec Dune33 pour faire le parcours 2 du 64. Le temps est à la pluie mais nous ne nous décourageons pas. Nous attaquons le circuit à l’envers .Il semble facile mais par la suite les choses se compliquent : une vraie patinoire!!!Quelques glissades ont poussé Dédé à nous équiper de jolis bâtons de noisetier pour éviter les chutes. Merci Dédé. Nous aurons le plaisir de croiser Crispol 40 qui nous a soufflé le FTF de la 224 sous le nez (grrrr ) car nous étions sur le mauvais côté de la rive!!!Domino nous a également croisé ( arrivée plus tard sur le circuit et gros problème de réseau).Malgré le mauvais temps nous avons passé une excellente journée:2 caches nous ont résisté (217 et 214 car la végétation a pris le dessus).Merci Gilles pour tout ce travail ,nous avons adoré .\end{cacheText}

\cacheNumber{1646}\needspace{5\baselineskip}\cacheName{\href{http://coord.info/GC7KWWF}{64 234 - Pyrénées Atlantique} — \href{http://coord.info/GC7KWWF\Number{}773652878}{1646}}\cacheData{{2018/06/04 gilles64, Unknown Cache (2/2)}}\begin{cacheText}Rendez-vous est pris avec Dune33 pour faire le parcours 2 du 64. Le temps est à la pluie mais nous ne nous décourageons pas. Nous attaquons le circuit à l’envers .Il semble facile mais par la suite les choses se compliquent : une vraie patinoire!!!Quelques glissades ont poussé Dédé à nous équiper de jolis bâtons de noisetier pour éviter les chutes. Merci Dédé. Nous aurons le plaisir de croiser Crispol 40 qui nous a soufflé le FTF de la 224 sous le nez (grrrr ) car nous étions sur le mauvais côté de la rive!!!Domino nous a également croisé ( arrivée plus tard sur le circuit et gros problème de réseau).Malgré le mauvais temps nous avons passé une excellente journée:2 caches nous ont résisté (217 et 214 car la végétation a pris le dessus).Merci Gilles pour tout ce travail ,nous avons adoré .\end{cacheText}

\cacheNumber{1647}\needspace{5\baselineskip}\cacheName{\href{http://coord.info/GC7KWXJ}{64 235 - Pyrénées Atlantique} — \href{http://coord.info/GC7KWXJ\Number{}773652714}{1647}}\cacheData{{2018/06/04 gilles64, Unknown Cache (2/2)}}\begin{cacheText}Rendez-vous est pris avec Dune33 pour faire le parcours 2 du 64. Le temps est à la pluie mais nous ne nous décourageons pas. Nous attaquons le circuit à l’envers .Il semble facile mais par la suite les choses se compliquent : une vraie patinoire!!!Quelques glissades ont poussé Dédé à nous équiper de jolis bâtons de noisetier pour éviter les chutes. Merci Dédé. Nous aurons le plaisir de croiser Crispol 40 qui nous a soufflé le FTF de la 224 sous le nez (grrrr ) car nous étions sur le mauvais côté de la rive!!!Domino nous a également croisé ( arrivée plus tard sur le circuit et gros problème de réseau).Malgré le mauvais temps nous avons passé une excellente journée:2 caches nous ont résisté (217 et 214 car la végétation a pris le dessus).Merci Gilles pour tout ce travail ,nous avons adoré .\end{cacheText}

\cacheNumber{1648}\needspace{5\baselineskip}\cacheName{\href{http://coord.info/GC7KWXM}{64 236 - Pyrénées Atlantique} — \href{http://coord.info/GC7KWXM\Number{}773652355}{1648}}\cacheData{{2018/06/04 gilles64, Unknown Cache (2/1.5)}}\begin{cacheText}Rendez-vous est pris avec Dune33 pour faire le parcours 2 du 64. Le temps est à la pluie mais nous ne nous décourageons pas. Nous attaquons le circuit à l’envers .Il semble facile mais par la suite les choses se compliquent : une vraie patinoire!!!Quelques glissades ont poussé Dédé à nous équiper de jolis bâtons de noisetier pour éviter les chutes. Merci Dédé. Nous aurons le plaisir de croiser Crispol 40 qui nous a soufflé le FTF de la 224 sous le nez (grrrr ) car nous étions sur le mauvais côté de la rive!!!Domino nous a également croisé ( arrivée plus tard sur le circuit et gros problème de réseau).Malgré le mauvais temps nous avons passé une excellente journée:2 caches nous ont résisté (217 et 214 car la végétation a pris le dessus).Merci Gilles pour tout ce travail ,nous avons adoré .\end{cacheText}

\cacheNumber{1649}\needspace{5\baselineskip}\cacheName{\href{http://coord.info/GC7KWY6}{64 237 Bonus - Pyrénées Atlantique} — \href{http://coord.info/GC7KWY6\Number{}773773191}{1649}}\cacheData{{2018/06/04 gilles64, Unknown Cache (3.5/2)}}\begin{cacheText}Co (TTF) avec Dune33

Rendez-vous est pris avec Dune33 pour faire le parcours 2 du 64. Le temps est à la pluie mais nous ne nous décourageons pas. Nous attaquons le circuit à l’envers .Il semble facile mais par la suite les choses se compliquent : une vraie patinoire!!!Quelques glissades ont poussé Dédé à nous équiper de jolis bâtons de noisetier pour éviter les chutes. Merci Dédé. Nous aurons le plaisir de croiser Crispol 40 qui nous a soufflé le FTF de la 224 sous le nez (grrrr ) car nous étions sur le mauvais côté de la rive!!!Domino nous a également croisé ( arrivée plus tard sur le circuit et gros problème de réseau).Malgré le mauvais temps nous avons passé une excellente journée:2 caches nous ont résisté (217 et 214 car la végétation a pris le dessus).Merci Gilles pour tout ce travail ,nous avons adoré .

Un PF amplement mérité pour la série.\end{cacheText}

\cacheNumber{1650}\needspace{5\baselineskip}\cacheName{\href{http://coord.info/GC5CYWV}{[BSD] \Number{}041} — \href{http://coord.info/GC5CYWV\Number{}773784488}{1650}}\cacheData{{2018/06/06 GéoLandesTour, Traditional Cache (1.5/1.5)}}\begin{cacheText}Ici Je ne dirais pas qu’elle est suspendue mais plutôt coincée dans la liane. Elle est bien là ! Je change le logbook qui est trempé. Merci pour la cache.\end{cacheText}

\cacheNumber{1651}\needspace{5\baselineskip}\cacheName{\href{http://coord.info/GC5CZ2B}{[BSD] \Number{}042} — \href{http://coord.info/GC5CZ2B\Number{}773784082}{1651}}\cacheData{{2018/06/06 GéoLandesTour, Traditional Cache (1.5/1.5)}}\begin{cacheText}La Belle est toujours en place. Je remplace le logbook qui est détrempé. Je continue la route et merci pour ce super parcours\end{cacheText}

\cacheNumber{1652}\needspace{5\baselineskip}\cacheName{\href{http://coord.info/GC5CZ2Y}{[BSD] \Number{}043} — \href{http://coord.info/GC5CZ2Y\Number{}773783836}{1652}}\cacheData{{2018/06/06 GéoLandesTour, Traditional Cache (2/1.5)}}\begin{cacheText}Ici les coordonnées GPS sont précises .Le hint,et le spoiler ne laissent aucun doute sur l'endroit. La cache est vite découverte. Je change le logbook qui est détrempé. Merci pour la cache\end{cacheText}

\cacheNumber{1653}\needspace{5\baselineskip}\cacheName{\href{http://coord.info/GC5CZ4K}{[BSD] \Number{}044} — \href{http://coord.info/GC5CZ4K\Number{}773781434}{1653}}\cacheData{{2018/06/06 GéoLandesTour, Traditional Cache (1.5/1.5)}}\begin{cacheText}Effectivement arrivée au PZ il n’y a plus de logbook. J’assure la maintenance avec un logbook tout neuf et un contenant de fortune.Je découvre ici une superbe petite église.Merci pour la cache.\end{cacheText}

\cacheNumber{1654}\needspace{5\baselineskip}\cacheName{\href{http://coord.info/GC5D0EP}{[BSD] \Number{}045} — \href{http://coord.info/GC5D0EP\Number{}773781930}{1654}}\cacheData{{2018/06/06 GéoLandesTour, Traditional Cache (1.5/1.5)}}\begin{cacheText}Arrivée au PZ, le paysage n’a pas trop changé. La cache est débusquée sans aucun problème. Elle nous attend au creux. Merci pour la cache.\end{cacheText}

\cacheNumber{1655}\needspace{5\baselineskip}\cacheName{\href{http://coord.info/GC5D0F4}{[BSD] \Number{}048} — \href{http://coord.info/GC5D0F4\Number{}773782154}{1655}}\cacheData{{2018/06/06 GéoLandesTour, Traditional Cache (1.5/1.5)}}\begin{cacheText}C’est mon deuxième passage sur ce PZ. La cache a bel et bien disparu. Nouvelle boîte fiez-vous à la photo. Merci pour la cache\end{cacheText}

\cacheNumber{1656}\needspace{5\baselineskip}\cacheName{\href{http://coord.info/GC7JYKR}{parc pour titiger} — \href{http://coord.info/GC7JYKR\Number{}773780958}{1656}}\cacheData{{2018/06/06 lauki3940, Traditional Cache (1.5/1.5)}}\begin{cacheText}Sur Dax pour un rendez vous, j’en profite pour faire quelques caches. Je découvre ici un vrai petit paradis. Un parc perdu au milieu d’un lotissement avec ses colverts et ses énormes poissons rouge. Une bien belle trouvaille. Merci lauki3940 pour cette belle découverte. Un PF\end{cacheText}

\cacheNumber{1657}\needspace{5\baselineskip}\cacheName{\href{http://coord.info/GC7JXKW}{64 117 - Pyrénées Atlantique} — \href{http://coord.info/GC7JXKW\Number{}774206920}{1657}}\cacheData{{2018/06/07 gilles64, Unknown Cache (2/1.5)}}\begin{cacheText}Enfin une journée ensoleillée est annoncée!!!Impossible de résister à la tentation de ce beau Géoart. Équipée de bottes, mieux vaut prévenir que guérir,je rejoins Domino50 pour réaliser ce parcours et glaner tous les indices. C’est un parcours sans faute qui est réalisé : toutes les caches sont découvertes et les paysages sont à la hauteur de nos espérances.Malgré un terrain très boueux et des passages difficiles (et non je ne regrette pas d’avoir les bottes) nous réalisons la balade assez rapidement. Nous concluons le circuit avec la bonus qui nous attend au pied du tordu:TTF pour celle ci. Un grand merci Gilles pour cette série qui nous fait découvrir la beauté et la richesse du Pays Basque.\end{cacheText}

\cacheNumber{1658}\needspace{5\baselineskip}\cacheName{\href{http://coord.info/GC7JXKY}{64 118 - Pyrénées Atlantique} — \href{http://coord.info/GC7JXKY\Number{}774428023}{1658}}\cacheData{{2018/06/07 gilles64, Unknown Cache (2/1.5)}}\begin{cacheText}Enfin une journée ensoleillée est annoncée!!!Impossible de résister à la tentation de ce beau Géoart. Équipée de bottes, mieux vaut prévenir que guérir,je rejoins Domino50 pour réaliser ce parcours et glaner tous les indices. C’est un parcours sans faute qui est réalisé : toutes les caches sont découvertes et les paysages sont à la hauteur de nos espérances.Malgré un terrain très boueux et des passages difficiles (et non je ne regrette pas d’avoir les bottes) nous réalisons la balade assez rapidement. Nous concluons le circuit avec la bonus qui nous attend au pied du tordu:TTF pour celle ci. Un grand merci Gilles pour cette série qui nous fait découvrir la beauté et la richesse du Pays Basque.\end{cacheText}

\cacheNumber{1659}\needspace{5\baselineskip}\cacheName{\href{http://coord.info/GC67EMP}{Lavoir de La Bastide Clairence} — \href{http://coord.info/GC67EMP\Number{}773945367}{1659}}\cacheData{{2018/06/07 gilles64, Traditional Cache (2/1.5)}}\begin{cacheText}En partance pour le 64 ou j’ai rendez-vous à 10h avec Domino50 je m’arrête faire quelques caches sur le trajet. Je découvre ici un très joli lavoir ainsi qu'un moulin très joliment rénové. La cacha est  parfaitement intégrée : bonne idée. Merci pour la cache.\end{cacheText}

\cacheNumber{1660}\needspace{5\baselineskip}\cacheName{\href{http://coord.info/GC7JKRB}{64 101 - Pyrénées Atlantique} — \href{http://coord.info/GC7JKRB\Number{}774175841}{1660}}\cacheData{{2018/06/07 gilles64, Unknown Cache (2/1.5)}}\begin{cacheText}Enfin une journée ensoleillée est annoncée!!!Impossible de résister à la tentation de ce beau Géoart. Équipée de bottes, mieux vaut prévenir que guérir,je rejoins Domino50 pour réaliser ce parcours et glaner tous les indices. C’est un parcours sans faute qui est réalisé : toutes les caches sont découvertes et les paysages sont à la hauteur de nos espérances.Malgré un terrain très boueux et des passages difficiles (et non je ne regrette pas d’avoir les bottes) nous réalisons la balade assez rapidement. Nous concluons le circuit avec la bonus qui nous attend au pied du tordu:TTF pour celle ci. Un grand merci Gilles pour cette série qui nous fait découvrir la beauté et la richesse du Pays Basque.\end{cacheText}

\cacheNumber{1661}\needspace{5\baselineskip}\cacheName{\href{http://coord.info/GC7JXD3}{64 103 - Pyrénées Atlantique} — \href{http://coord.info/GC7JXD3\Number{}774175993}{1661}}\cacheData{{2018/06/07 gilles64, Unknown Cache (2/1.5)}}\begin{cacheText}Enfin une journée ensoleillée est annoncée!!!Impossible de résister à la tentation de ce beau Géoart. Équipée de bottes, mieux vaut prévenir que guérir,je rejoins Domino50 pour réaliser ce parcours et glaner tous les indices. C’est un parcours sans faute qui est réalisé : toutes les caches sont découvertes et les paysages sont à la hauteur de nos espérances.Malgré un terrain très boueux et des passages difficiles (et non je ne regrette pas d’avoir les bottes) nous réalisons la balade assez rapidement. Nous concluons le circuit avec la bonus qui nous attend au pied du tordu:TTF pour celle ci. Un grand merci Gilles pour cette série qui nous fait découvrir la beauté et la richesse du Pays Basque.\end{cacheText}

\cacheNumber{1662}\needspace{5\baselineskip}\cacheName{\href{http://coord.info/GC7JXDC}{64 102 - Pyrénées Atlantique} — \href{http://coord.info/GC7JXDC\Number{}774175897}{1662}}\cacheData{{2018/06/07 gilles64, Unknown Cache (2/1.5)}}\begin{cacheText}Enfin une journée ensoleillée est annoncée!!!Impossible de résister à la tentation de ce beau Géoart. Équipée de bottes, mieux vaut prévenir que guérir,je rejoins Domino50 pour réaliser ce parcours et glaner tous les indices. C’est un parcours sans faute qui est réalisé : toutes les caches sont découvertes et les paysages sont à la hauteur de nos espérances.Malgré un terrain très boueux et des passages difficiles (et non je ne regrette pas d’avoir les bottes) nous réalisons la balade assez rapidement. Nous concluons le circuit avec la bonus qui nous attend au pied du tordu:TTF pour celle ci. Un grand merci Gilles pour cette série qui nous fait découvrir la beauté et la richesse du Pays Basque.\end{cacheText}

\cacheNumber{1663}\needspace{5\baselineskip}\cacheName{\href{http://coord.info/GC7JXDT}{64 104 - Pyrénées Atlantique} — \href{http://coord.info/GC7JXDT\Number{}774176064}{1663}}\cacheData{{2018/06/07 gilles64, Unknown Cache (2/1.5)}}\begin{cacheText}Enfin une journée ensoleillée est annoncée!!!Impossible de résister à la tentation de ce beau Géoart. Équipée de bottes, mieux vaut prévenir que guérir,je rejoins Domino50 pour réaliser ce parcours et glaner tous les indices. C’est un parcours sans faute qui est réalisé : toutes les caches sont découvertes et les paysages sont à la hauteur de nos espérances.Malgré un terrain très boueux et des passages difficiles (et non je ne regrette pas d’avoir les bottes) nous réalisons la balade assez rapidement. Nous concluons le circuit avec la bonus qui nous attend au pied du tordu:TTF pour celle ci. Un grand merci Gilles pour cette série qui nous fait découvrir la beauté et la richesse du Pays Basque.\end{cacheText}

\cacheNumber{1664}\needspace{5\baselineskip}\cacheName{\href{http://coord.info/GC7JXHP}{64 105 - Pyrénées Atlantique} — \href{http://coord.info/GC7JXHP\Number{}774197341}{1664}}\cacheData{{2018/06/07 gilles64, Unknown Cache (2/1.5)}}\begin{cacheText}Enfin une journée ensoleillée est annoncée!!!Impossible de résister à la tentation de ce beau Géoart. Équipée de bottes, mieux vaut prévenir que guérir,je rejoins Domino50 pour réaliser ce parcours et glaner tous les indices. C’est un parcours sans faute qui est réalisé : toutes les caches sont découvertes et les paysages sont à la hauteur de nos espérances.Malgré un terrain très boueux et des passages difficiles (et non je ne regrette pas d’avoir les bottes) nous réalisons la balade assez rapidement. Nous concluons le circuit avec la bonus qui nous attend au pied du tordu:TTF pour celle ci. Un grand merci Gilles pour cette série qui nous fait découvrir la beauté et la richesse du Pays Basque.\end{cacheText}

\cacheNumber{1665}\needspace{5\baselineskip}\cacheName{\href{http://coord.info/GC7JXJ3}{64 106 - Pyrénées Atlantique} — \href{http://coord.info/GC7JXJ3\Number{}774197614}{1665}}\cacheData{{2018/06/07 gilles64, Unknown Cache (2/1.5)}}\begin{cacheText}Enfin une journée ensoleillée est annoncée!!!Impossible de résister à la tentation de ce beau Géoart. Équipée de bottes, mieux vaut prévenir que guérir,je rejoins Domino50 pour réaliser ce parcours et glaner tous les indices. C’est un parcours sans faute qui est réalisé : toutes les caches sont découvertes et les paysages sont à la hauteur de nos espérances.Malgré un terrain très boueux et des passages difficiles (et non je ne regrette pas d’avoir les bottes) nous réalisons la balade assez rapidement. Nous concluons le circuit avec la bonus qui nous attend au pied du tordu:TTF pour celle ci. Un grand merci Gilles pour cette série qui nous fait découvrir la beauté et la richesse du Pays Basque.\end{cacheText}

\cacheNumber{1666}\needspace{5\baselineskip}\cacheName{\href{http://coord.info/GC7JXJ6}{64 107 - Pyrénées Atlantique} — \href{http://coord.info/GC7JXJ6\Number{}774197799}{1666}}\cacheData{{2018/06/07 gilles64, Unknown Cache (2/1.5)}}\begin{cacheText}Enfin une journée ensoleillée est annoncée!!!Impossible de résister à la tentation de ce beau Géoart. Équipée de bottes, mieux vaut prévenir que guérir,je rejoins Domino50 pour réaliser ce parcours et glaner tous les indices. C’est un parcours sans faute qui est réalisé : toutes les caches sont découvertes et les paysages sont à la hauteur de nos espérances.Malgré un terrain très boueux et des passages difficiles (et non je ne regrette pas d’avoir les bottes) nous réalisons la balade assez rapidement. Nous concluons le circuit avec la bonus qui nous attend au pied du tordu:TTF pour celle ci. Un grand merci Gilles pour cette série qui nous fait découvrir la beauté et la richesse du Pays Basque.\end{cacheText}

\cacheNumber{1667}\needspace{5\baselineskip}\cacheName{\href{http://coord.info/GC7JXJD}{64 108 - Pyrénées Atlantique} — \href{http://coord.info/GC7JXJD\Number{}774198035}{1667}}\cacheData{{2018/06/07 gilles64, Unknown Cache (2/1.5)}}\begin{cacheText}Enfin une journée ensoleillée est annoncée!!!Impossible de résister à la tentation de ce beau Géoart. Équipée de bottes, mieux vaut prévenir que guérir,je rejoins Domino50 pour réaliser ce parcours et glaner tous les indices. C’est un parcours sans faute qui est réalisé : toutes les caches sont découvertes et les paysages sont à la hauteur de nos espérances.Malgré un terrain très boueux et des passages difficiles (et non je ne regrette pas d’avoir les bottes) nous réalisons la balade assez rapidement. Nous concluons le circuit avec la bonus qui nous attend au pied du tordu:TTF pour celle ci. Un grand merci Gilles pour cette série qui nous fait découvrir la beauté et la richesse du Pays Basque.\end{cacheText}

\cacheNumber{1668}\needspace{5\baselineskip}\cacheName{\href{http://coord.info/GC7JXJH}{64 109 - Pyrénées Atlantique} — \href{http://coord.info/GC7JXJH\Number{}774198279}{1668}}\cacheData{{2018/06/07 gilles64, Unknown Cache (2/1.5)}}\begin{cacheText}Enfin une journée ensoleillée est annoncée!!!Impossible de résister à la tentation de ce beau Géoart. Équipée de bottes, mieux vaut prévenir que guérir,je rejoins Domino50 pour réaliser ce parcours et glaner tous les indices. C’est un parcours sans faute qui est réalisé : toutes les caches sont découvertes et les paysages sont à la hauteur de nos espérances.Malgré un terrain très boueux et des passages difficiles (et non je ne regrette pas d’avoir les bottes) nous réalisons la balade assez rapidement. Nous concluons le circuit avec la bonus qui nous attend au pied du tordu:TTF pour celle ci. Un grand merci Gilles pour cette série qui nous fait découvrir la beauté et la richesse du Pays Basque.\end{cacheText}

\cacheNumber{1669}\needspace{5\baselineskip}\cacheName{\href{http://coord.info/GC7JXJW}{64 110 - Pyrénées Atlantique} — \href{http://coord.info/GC7JXJW\Number{}774198612}{1669}}\cacheData{{2018/06/07 gilles64, Unknown Cache (2/3)}}\begin{cacheText}Enfin une journée ensoleillée est annoncée!!!Impossible de résister à la tentation de ce beau Géoart. Équipée de bottes, mieux vaut prévenir que guérir,je rejoins Domino50 pour réaliser ce parcours et glaner tous les indices. C’est un parcours sans faute qui est réalisé : toutes les caches sont découvertes et les paysages sont à la hauteur de nos espérances.Malgré un terrain très boueux et des passages difficiles (et non je ne regrette pas d’avoir les bottes) nous réalisons la balade assez rapidement. Nous concluons le circuit avec la bonus qui nous attend au pied du tordu:TTF pour celle ci. Un grand merci Gilles pour cette série qui nous fait découvrir la beauté et la richesse du Pays Basque.\end{cacheText}

\cacheNumber{1670}\needspace{5\baselineskip}\cacheName{\href{http://coord.info/GC7JXK2}{64 111 - Pyrénées Atlantique} — \href{http://coord.info/GC7JXK2\Number{}774198714}{1670}}\cacheData{{2018/06/07 gilles64, Unknown Cache (2/2.5)}}\begin{cacheText}Enfin une journée ensoleillée est annoncée!!!Impossible de résister à la tentation de ce beau Géoart. Équipée de bottes, mieux vaut prévenir que guérir,je rejoins Domino50 pour réaliser ce parcours et glaner tous les indices. C’est un parcours sans faute qui est réalisé : toutes les caches sont découvertes et les paysages sont à la hauteur de nos espérances.Malgré un terrain très boueux et des passages difficiles (et non je ne regrette pas d’avoir les bottes) nous réalisons la balade assez rapidement. Nous concluons le circuit avec la bonus qui nous attend au pied du tordu:TTF pour celle ci. Un grand merci Gilles pour cette série qui nous fait découvrir la beauté et la richesse du Pays Basque.\end{cacheText}

\cacheNumber{1671}\needspace{5\baselineskip}\cacheName{\href{http://coord.info/GC7JXKB}{64 112 - Pyrénées Atlantique} — \href{http://coord.info/GC7JXKB\Number{}774198829}{1671}}\cacheData{{2018/06/07 gilles64, Unknown Cache (2/1.5)}}\begin{cacheText}Enfin une journée ensoleillée est annoncée!!!Impossible de résister à la tentation de ce beau Géoart. Équipée de bottes, mieux vaut prévenir que guérir,je rejoins Domino50 pour réaliser ce parcours et glaner tous les indices. C’est un parcours sans faute qui est réalisé : toutes les caches sont découvertes et les paysages sont à la hauteur de nos espérances.Malgré un terrain très boueux et des passages difficiles (et non je ne regrette pas d’avoir les bottes) nous réalisons la balade assez rapidement. Nous concluons le circuit avec la bonus qui nous attend au pied du tordu:TTF pour celle ci. Un grand merci Gilles pour cette série qui nous fait découvrir la beauté et la richesse du Pays Basque.\end{cacheText}

\cacheNumber{1672}\needspace{5\baselineskip}\cacheName{\href{http://coord.info/GC7JXKH}{64 113 - Pyrénées Atlantique} — \href{http://coord.info/GC7JXKH\Number{}774199024}{1672}}\cacheData{{2018/06/07 gilles64, Unknown Cache (2/1.5)}}\begin{cacheText}Enfin une journée ensoleillée est annoncée!!!Impossible de résister à la tentation de ce beau Géoart. Équipée de bottes, mieux vaut prévenir que guérir,je rejoins Domino50 pour réaliser ce parcours et glaner tous les indices. C’est un parcours sans faute qui est réalisé : toutes les caches sont découvertes et les paysages sont à la hauteur de nos espérances.Malgré un terrain très boueux et des passages difficiles (et non je ne regrette pas d’avoir les bottes) nous réalisons la balade assez rapidement. Nous concluons le circuit avec la bonus qui nous attend au pied du tordu:TTF pour celle ci. Un grand merci Gilles pour cette série qui nous fait découvrir la beauté et la richesse du Pays Basque.\end{cacheText}

\cacheNumber{1673}\needspace{5\baselineskip}\cacheName{\href{http://coord.info/GC7JXKM}{64 114 - Pyrénées Atlantique} — \href{http://coord.info/GC7JXKM\Number{}774199188}{1673}}\cacheData{{2018/06/07 gilles64, Unknown Cache (2/1.5)}}\begin{cacheText}Enfin une journée ensoleillée est annoncée!!!Impossible de résister à la tentation de ce beau Géoart. Équipée de bottes, mieux vaut prévenir que guérir,je rejoins Domino50 pour réaliser ce parcours et glaner tous les indices. C’est un parcours sans faute qui est réalisé : toutes les caches sont découvertes et les paysages sont à la hauteur de nos espérances.Malgré un terrain très boueux et des passages difficiles (et non je ne regrette pas d’avoir les bottes) nous réalisons la balade assez rapidement. Nous concluons le circuit avec la bonus qui nous attend au pied du tordu:TTF pour celle ci. Un grand merci Gilles pour cette série qui nous fait découvrir la beauté et la richesse du Pays Basque.\end{cacheText}

\cacheNumber{1674}\needspace{5\baselineskip}\cacheName{\href{http://coord.info/GC7JXKR}{64 115 - Pyrénées Atlantique} — \href{http://coord.info/GC7JXKR\Number{}774199480}{1674}}\cacheData{{2018/06/07 gilles64, Unknown Cache (2/1.5)}}\begin{cacheText}Enfin une journée ensoleillée est annoncée!!!Impossible de résister à la tentation de ce beau Géoart. Équipée de bottes, mieux vaut prévenir que guérir,je rejoins Domino50 pour réaliser ce parcours et glaner tous les indices. C’est un parcours sans faute qui est réalisé : toutes les caches sont découvertes et les paysages sont à la hauteur de nos espérances.Malgré un terrain très boueux et des passages difficiles (et non je ne regrette pas d’avoir les bottes) nous réalisons la balade assez rapidement. Nous concluons le circuit avec la bonus qui nous attend au pied du tordu:TTF pour celle ci. Un grand merci Gilles pour cette série qui nous fait découvrir la beauté et la richesse du Pays Basque.\end{cacheText}

\cacheNumber{1675}\needspace{5\baselineskip}\cacheName{\href{http://coord.info/GC7JXKV}{64 116 - Pyrénées Atlantique} — \href{http://coord.info/GC7JXKV\Number{}774199614}{1675}}\cacheData{{2018/06/07 gilles64, Unknown Cache (2/1.5)}}\begin{cacheText}Enfin une journée ensoleillée est annoncée!!!Impossible de résister à la tentation de ce beau Géoart. Équipée de bottes, mieux vaut prévenir que guérir,je rejoins Domino50 pour réaliser ce parcours et glaner tous les indices. C’est un parcours sans faute qui est réalisé : toutes les caches sont découvertes et les paysages sont à la hauteur de nos espérances.Malgré un terrain très boueux et des passages difficiles (et non je ne regrette pas d’avoir les bottes) nous réalisons la balade assez rapidement. Nous concluons le circuit avec la bonus qui nous attend au pied du tordu:TTF pour celle ci. Un grand merci Gilles pour cette série qui nous fait découvrir la beauté et la richesse du Pays Basque.\end{cacheText}

\cacheNumber{1676}\needspace{5\baselineskip}\cacheName{\href{http://coord.info/GC7JXM4}{64 120 - Pyrénées Atlantique} — \href{http://coord.info/GC7JXM4\Number{}774436575}{1676}}\cacheData{{2018/06/07 gilles64, Unknown Cache (2/3)}}\begin{cacheText}Enfin une journée ensoleillée est annoncée!!!Impossible de résister à la tentation de ce beau Géoart. Équipée de bottes, mieux vaut prévenir que guérir,je rejoins Domino50 pour réaliser ce parcours et glaner tous les indices. C’est un parcours sans faute qui est réalisé : toutes les caches sont découvertes et les paysages sont à la hauteur de nos espérances.Malgré un terrain très boueux et des passages difficiles (et non je ne regrette pas d’avoir les bottes) nous réalisons la balade assez rapidement. Nous concluons le circuit avec la bonus qui nous attend au pied du tordu:TTF pour celle ci. Un grand merci Gilles pour cette série qui nous fait découvrir la beauté et la richesse du Pays Basque.\end{cacheText}

\cacheNumber{1677}\needspace{5\baselineskip}\cacheName{\href{http://coord.info/GC7JXM8}{64 121 - Pyrénées Atlantique} — \href{http://coord.info/GC7JXM8\Number{}774436805}{1677}}\cacheData{{2018/06/07 gilles64, Unknown Cache (2/2.5)}}\begin{cacheText}Enfin une journée ensoleillée est annoncée!!!Impossible de résister à la tentation de ce beau Géoart. Équipée de bottes, mieux vaut prévenir que guérir,je rejoins Domino50 pour réaliser ce parcours et glaner tous les indices. C’est un parcours sans faute qui est réalisé : toutes les caches sont découvertes et les paysages sont à la hauteur de nos espérances.Malgré un terrain très boueux et des passages difficiles (et non je ne regrette pas d’avoir les bottes) nous réalisons la balade assez rapidement. Nous concluons le circuit avec la bonus qui nous attend au pied du tordu:TTF pour celle ci. Un grand merci Gilles pour cette série qui nous fait découvrir la beauté et la richesse du Pays Basque.\end{cacheText}

\cacheNumber{1678}\needspace{5\baselineskip}\cacheName{\href{http://coord.info/GC7JXW0}{64 122 - Pyrénées Atlantique} — \href{http://coord.info/GC7JXW0\Number{}774436977}{1678}}\cacheData{{2018/06/07 gilles64, Unknown Cache (2/1.5)}}\begin{cacheText}Enfin une journée ensoleillée est annoncée!!!Impossible de résister à la tentation de ce beau Géoart. Équipée de bottes, mieux vaut prévenir que guérir,je rejoins Domino50 pour réaliser ce parcours et glaner tous les indices. C’est un parcours sans faute qui est réalisé : toutes les caches sont découvertes et les paysages sont à la hauteur de nos espérances.Malgré un terrain très boueux et des passages difficiles (et non je ne regrette pas d’avoir les bottes) nous réalisons la balade assez rapidement. Nous concluons le circuit avec la bonus qui nous attend au pied du tordu:TTF pour celle ci. Un grand merci Gilles pour cette série qui nous fait découvrir la beauté et la richesse du Pays Basque.\end{cacheText}

\cacheNumber{1679}\needspace{5\baselineskip}\cacheName{\href{http://coord.info/GC7JXZ0}{64 123 - Pyrénées Atlantique} — \href{http://coord.info/GC7JXZ0\Number{}774436957}{1679}}\cacheData{{2018/06/07 gilles64, Unknown Cache (2/1.5)}}\begin{cacheText}Enfin une journée ensoleillée est annoncée!!!Impossible de résister à la tentation de ce beau Géoart. Équipée de bottes, mieux vaut prévenir que guérir,je rejoins Domino50 pour réaliser ce parcours et glaner tous les indices. C’est un parcours sans faute qui est réalisé : toutes les caches sont découvertes et les paysages sont à la hauteur de nos espérances.Malgré un terrain très boueux et des passages difficiles (et non je ne regrette pas d’avoir les bottes) nous réalisons la balade assez rapidement. Nous concluons le circuit avec la bonus qui nous attend au pied du tordu:TTF pour celle ci. Un grand merci Gilles pour cette série qui nous fait découvrir la beauté et la richesse du Pays Basque.\end{cacheText}

\cacheNumber{1680}\needspace{5\baselineskip}\cacheName{\href{http://coord.info/GC7JXZ3}{64 124 - Pyrénées Atlantique} — \href{http://coord.info/GC7JXZ3\Number{}774437844}{1680}}\cacheData{{2018/06/07 gilles64, Unknown Cache (2/2)}}\begin{cacheText}Enfin une journée ensoleillée est annoncée!!!Impossible de résister à la tentation de ce beau Géoart. Équipée de bottes, mieux vaut prévenir que guérir,je rejoins Domino50 pour réaliser ce parcours et glaner tous les indices. C’est un parcours sans faute qui est réalisé : toutes les caches sont découvertes et les paysages sont à la hauteur de nos espérances.Malgré un terrain très boueux et des passages difficiles (et non je ne regrette pas d’avoir les bottes) nous réalisons la balade assez rapidement. Nous concluons le circuit avec la bonus qui nous attend au pied du tordu:TTF pour celle ci. Un grand merci Gilles pour cette série qui nous fait découvrir la beauté et la richesse du Pays Basque.\end{cacheText}

\cacheNumber{1681}\needspace{5\baselineskip}\cacheName{\href{http://coord.info/GC7JYFE}{64 325 - Pyrénées Atlantique} — \href{http://coord.info/GC7JYFE\Number{}774504651}{1681}}\cacheData{{2018/06/07 gilles64, Unknown Cache (2/2)}}\begin{cacheText}Afin de récolter les indices nécessaires à la Super Bonus, j’enchaîne sur le parcours 3 accompagnée de mes fidèles bottes. Si je veux une place sur le podium , bien méritée, c'est maintenant ou jamais!!!  Ici aussi, je découvre de très beaux endroits très boueux ! Toutes les caches, exceptée la 319 qui n’a été trouvée par personne, sont découvertes malgré l’absence des spoilers(vraiment pas très au point avec la nouvelle technologie!!!). C’est en sortant du sentier que je retrouve Domino50 qui s’apprête à rentrer après avoir loguer la multi en FTF. Elle décide de m’accompagner sur la bonus à fin de blablater sur la petite Multi que je n’aurai pas l’occasion de faire cette fois ci car l'heure tourne (ayant bloqué mon téléphone ,je suis coupée de la civilisation :on doit s'inquiéter!!!) .Trop contente...je suis également TTF sur la Bonus3.Merci Gilles pour toutes ces caches placées dans de somptueux  décors. j'en redemande....\end{cacheText}

\cacheNumber{1682}\needspace{5\baselineskip}\cacheName{\href{http://coord.info/GC7JYFH}{64 125 - Pyrénées Atlantique} — \href{http://coord.info/GC7JYFH\Number{}774437380}{1682}}\cacheData{{2018/06/07 gilles64, Unknown Cache (2/2)}}\begin{cacheText}Enfin une journée ensoleillée est annoncée!!!Impossible de résister à la tentation de ce beau Géoart. Équipée de bottes, mieux vaut prévenir que guérir,je rejoins Domino50 pour réaliser ce parcours et glaner tous les indices. C’est un parcours sans faute qui est réalisé : toutes les caches sont découvertes et les paysages sont à la hauteur de nos espérances.Malgré un terrain très boueux et des passages difficiles (et non je ne regrette pas d’avoir les bottes) nous réalisons la balade assez rapidement. Nous concluons le circuit avec la bonus qui nous attend au pied du tordu:TTF pour celle ci. Un grand merci Gilles pour cette série qui nous fait découvrir la beauté et la richesse du Pays Basque.\end{cacheText}

\cacheNumber{1683}\needspace{5\baselineskip}\cacheName{\href{http://coord.info/GC7JYFK}{64 126 - Pyrénées Atlantique} — \href{http://coord.info/GC7JYFK\Number{}774438340}{1683}}\cacheData{{2018/06/07 gilles64, Unknown Cache (2/1.5)}}\begin{cacheText}Enfin une journée ensoleillée est annoncée!!!Impossible de résister à la tentation de ce beau Géoart. Équipée de bottes, mieux vaut prévenir que guérir,je rejoins Domino50 pour réaliser ce parcours et glaner tous les indices. C’est un parcours sans faute qui est réalisé : toutes les caches sont découvertes et les paysages sont à la hauteur de nos espérances.Malgré un terrain très boueux et des passages difficiles (et non je ne regrette pas d’avoir les bottes) nous réalisons la balade assez rapidement. Nous concluons le circuit avec la bonus qui nous attend au pied du tordu:TTF pour celle ci. Un grand merci Gilles pour cette série qui nous fait découvrir la beauté et la richesse du Pays Basque.\end{cacheText}

\cacheNumber{1684}\needspace{5\baselineskip}\cacheName{\href{http://coord.info/GC7JYFP}{64 127 - Pyrénées Atlantique} — \href{http://coord.info/GC7JYFP\Number{}774438479}{1684}}\cacheData{{2018/06/07 gilles64, Unknown Cache (2/2)}}\begin{cacheText}Enfin une journée ensoleillée est annoncée!!!Impossible de résister à la tentation de ce beau Géoart. Équipée de bottes, mieux vaut prévenir que guérir,je rejoins Domino50 pour réaliser ce parcours et glaner tous les indices. C’est un parcours sans faute qui est réalisé : toutes les caches sont découvertes et les paysages sont à la hauteur de nos espérances.Malgré un terrain très boueux et des passages difficiles (et non je ne regrette pas d’avoir les bottes) nous réalisons la balade assez rapidement. Nous concluons le circuit avec la bonus qui nous attend au pied du tordu:TTF pour celle ci. Un grand merci Gilles pour cette série qui nous fait découvrir la beauté et la richesse du Pays Basque.\end{cacheText}

\cacheNumber{1685}\needspace{5\baselineskip}\cacheName{\href{http://coord.info/GC7JYFR}{64 301 - Pyrénées Atlantique} — \href{http://coord.info/GC7JYFR\Number{}774442554}{1685}}\cacheData{{2018/06/07 gilles64, Unknown Cache (2/1.5)}}\begin{cacheText}Afin de récolter les indices nécessaires à la Super Bonus, j’enchaîne sur le parcours 3 accompagnée de mes fidèles bottes. Si je veux une place sur le podium , bien méritée, c'est maintenant ou jamais!!!  Ici aussi, je découvre de très beaux endroits très boueux ! Toutes les caches, exceptée la 319 qui n’a été trouvée par personne, sont découvertes malgré l’absence des spoilers(vraiment pas très au point avec la nouvelle technologie!!!). C’est en sortant du sentier que je retrouve Domino50 qui s’apprête à rentrer après avoir loguer la multi en FTF. Elle décide de m’accompagner sur la bonus à fin de blablater sur la petite Multi que je n’aurai pas l’occasion de faire cette fois ci car l'heure tourne (ayant bloqué mon téléphone ,je suis coupée de la civilisation :on doit s'inquiéter!!!) .Trop contente...je suis également TTF sur la Bonus3.Merci Gilles pour toutes ces caches placées dans de somptueux  décors. j'en redemande....\end{cacheText}

\cacheNumber{1686}\needspace{5\baselineskip}\cacheName{\href{http://coord.info/GC7JYFV}{64 302 - Pyrénées Atlantique} — \href{http://coord.info/GC7JYFV\Number{}774444634}{1686}}\cacheData{{2018/06/07 gilles64, Unknown Cache (2/1.5)}}\begin{cacheText}Afin de récolter les indices nécessaires à la Super Bonus, j’enchaîne sur le parcours 3 accompagnée de mes fidèles bottes. Si je veux une place sur le podium , bien méritée, c'est maintenant ou jamais!!!  Ici aussi, je découvre de très beaux endroits très boueux ! Toutes les caches, exceptée la 319 qui n’a été trouvée par personne, sont découvertes malgré l’absence des spoilers(vraiment pas très au point avec la nouvelle technologie!!!). C’est en sortant du sentier que je retrouve Domino50 qui s’apprête à rentrer après avoir loguer la multi en FTF. Elle décide de m’accompagner sur la bonus à fin de blablater sur la petite Multi que je n’aurai pas l’occasion de faire cette fois ci car l'heure tourne (ayant bloqué mon téléphone ,je suis coupée de la civilisation :on doit s'inquiéter!!!) .Trop contente...je suis également TTF sur la Bonus3.Merci Gilles pour toutes ces caches placées dans de somptueux  décors. j'en redemande....\end{cacheText}

\cacheNumber{1687}\needspace{5\baselineskip}\cacheName{\href{http://coord.info/GC7JYG2}{64 128 Bonus - Pyrénées Atlantique} — \href{http://coord.info/GC7JYG2\Number{}774439077}{1687}}\cacheData{{2018/06/07 gilles64/patricia64, Unknown Cache (3.5/2)}}\begin{cacheText}Co [{TTF}] avec Domino 50

Enfin une journée ensoleillée est annoncée!!!Impossible de résister à la tentation de ce beau Géoart. Équipée de bottes, mieux vaut prévenir que guérir,je rejoins Domino50 pour réaliser ce parcours et glaner tous les indices. C’est un parcours sans faute qui est réalisé : toutes les caches sont découvertes et les paysages sont à la hauteur de nos espérances.Malgré un terrain très boueux et des passages difficiles (et non je ne regrette pas d’avoir les bottes) nous réalisons la balade assez rapidement. Nous concluons le circuit avec la bonus qui nous attend au pied du tordu:TTF pour celle ci. Un grand merci Gilles pour cette série qui nous fait découvrir la beauté et la richesse du Pays Basque. Un PF évidemment pour cette belle randonnée.\end{cacheText}

\cacheNumber{1688}\needspace{5\baselineskip}\cacheName{\href{http://coord.info/GC7KV4G}{64 303 - Pyrénées Atlantique} — \href{http://coord.info/GC7KV4G\Number{}774446012}{1688}}\cacheData{{2018/06/07 gilles64, Unknown Cache (2/1.5)}}\begin{cacheText}Afin de récolter les indices nécessaires à la Super Bonus, j’enchaîne sur le parcours 3 accompagnée de mes fidèles bottes. Si je veux une place sur le podium , bien méritée, c'est maintenant ou jamais!!!  Ici aussi, je découvre de très beaux endroits très boueux ! Toutes les caches, exceptée la 319 qui n’a été trouvée par personne, sont découvertes malgré l’absence des spoilers(vraiment pas très au point avec la nouvelle technologie!!!). C’est en sortant du sentier que je retrouve Domino50 qui s’apprête à rentrer après avoir loguer la multi en FTF. Elle décide de m’accompagner sur la bonus à fin de blablater sur la petite Multi que je n’aurai pas l’occasion de faire cette fois ci car l'heure tourne (ayant bloqué mon téléphone ,je suis coupée de la civilisation :on doit s'inquiéter!!!) .Trop contente...je suis également TTF sur la Bonus3.Merci Gilles pour toutes ces caches placées dans de somptueux  décors. j'en redemande....\end{cacheText}

\cacheNumber{1689}\needspace{5\baselineskip}\cacheName{\href{http://coord.info/GC7KV4J}{64 304 - Pyrénées Atlantique} — \href{http://coord.info/GC7KV4J\Number{}774446053}{1689}}\cacheData{{2018/06/07 gilles64, Unknown Cache (2/1.5)}}\begin{cacheText}Afin de récolter les indices nécessaires à la Super Bonus, j’enchaîne sur le parcours 3 accompagnée de mes fidèles bottes. Si je veux une place sur le podium , bien méritée, c'est maintenant ou jamais!!!  Ici aussi, je découvre de très beaux endroits très boueux ! Toutes les caches, exceptée la 319 qui n’a été trouvée par personne, sont découvertes malgré l’absence des spoilers(vraiment pas très au point avec la nouvelle technologie!!!). C’est en sortant du sentier que je retrouve Domino50 qui s’apprête à rentrer après avoir loguer la multi en FTF. Elle décide de m’accompagner sur la bonus à fin de blablater sur la petite Multi que je n’aurai pas l’occasion de faire cette fois ci car l'heure tourne (ayant bloqué mon téléphone ,je suis coupée de la civilisation :on doit s'inquiéter!!!) .Trop contente...je suis également TTF sur la Bonus3.Merci Gilles pour toutes ces caches placées dans de somptueux  décors. j'en redemande....\end{cacheText}

\cacheNumber{1690}\needspace{5\baselineskip}\cacheName{\href{http://coord.info/GC7KV4N}{64 305 - Pyrénées Atlantique} — \href{http://coord.info/GC7KV4N\Number{}774446628}{1690}}\cacheData{{2018/06/07 gilles64, Unknown Cache (2/1.5)}}\begin{cacheText}Afin de récolter les indices nécessaires à la Super Bonus, j’enchaîne sur le parcours 3 accompagnée de mes fidèles bottes. Si je veux une place sur le podium , bien méritée, c'est maintenant ou jamais!!!  Ici aussi, je découvre de très beaux endroits très boueux ! Toutes les caches, exceptée la 319 qui n’a été trouvée par personne, sont découvertes malgré l’absence des spoilers(vraiment pas très au point avec la nouvelle technologie!!!). C’est en sortant du sentier que je retrouve Domino50 qui s’apprête à rentrer après avoir loguer la multi en FTF. Elle décide de m’accompagner sur la bonus à fin de blablater sur la petite Multi que je n’aurai pas l’occasion de faire cette fois ci car l'heure tourne (ayant bloqué mon téléphone ,je suis coupée de la civilisation :on doit s'inquiéter!!!) .Trop contente...je suis également TTF sur la Bonus3.Merci Gilles pour toutes ces caches placées dans de somptueux  décors. j'en redemande....\end{cacheText}

\cacheNumber{1691}\needspace{5\baselineskip}\cacheName{\href{http://coord.info/GC7KV7H}{64 306 - Pyrénées Atlantique} — \href{http://coord.info/GC7KV7H\Number{}774446904}{1691}}\cacheData{{2018/06/07 gilles64, Unknown Cache (2/1.5)}}\begin{cacheText}Afin de récolter les indices nécessaires à la Super Bonus, j’enchaîne sur le parcours 3 accompagnée de mes fidèles bottes. Si je veux une place sur le podium , bien méritée, c'est maintenant ou jamais!!!  Ici aussi, je découvre de très beaux endroits très boueux ! Toutes les caches, exceptée la 319 qui n’a été trouvée par personne, sont découvertes malgré l’absence des spoilers(vraiment pas très au point avec la nouvelle technologie!!!). C’est en sortant du sentier que je retrouve Domino50 qui s’apprête à rentrer après avoir loguer la multi en FTF. Elle décide de m’accompagner sur la bonus à fin de blablater sur la petite Multi que je n’aurai pas l’occasion de faire cette fois ci car l'heure tourne (ayant bloqué mon téléphone ,je suis coupée de la civilisation :on doit s'inquiéter!!!) .Trop contente...je suis également TTF sur la Bonus3.Merci Gilles pour toutes ces caches placées dans de somptueux  décors. j'en redemande....\end{cacheText}

\cacheNumber{1692}\needspace{5\baselineskip}\cacheName{\href{http://coord.info/GC7KV7N}{64 307 - Pyrénées Atlantique} — \href{http://coord.info/GC7KV7N\Number{}774447155}{1692}}\cacheData{{2018/06/07 gilles64, Unknown Cache (2/1.5)}}\begin{cacheText}Afin de récolter les indices nécessaires à la Super Bonus, j’enchaîne sur le parcours 3 accompagnée de mes fidèles bottes. Si je veux une place sur le podium , bien méritée, c'est maintenant ou jamais!!!  Ici aussi, je découvre de très beaux endroits très boueux ! Toutes les caches, exceptée la 319 qui n’a été trouvée par personne, sont découvertes malgré l’absence des spoilers(vraiment pas très au point avec la nouvelle technologie!!!). C’est en sortant du sentier que je retrouve Domino50 qui s’apprête à rentrer après avoir loguer la multi en FTF. Elle décide de m’accompagner sur la bonus à fin de blablater sur la petite Multi que je n’aurai pas l’occasion de faire cette fois ci car l'heure tourne (ayant bloqué mon téléphone ,je suis coupée de la civilisation :on doit s'inquiéter!!!) .Trop contente...je suis également TTF sur la Bonus3.Merci Gilles pour toutes ces caches placées dans de somptueux  décors. j'en redemande....\end{cacheText}

\cacheNumber{1693}\needspace{5\baselineskip}\cacheName{\href{http://coord.info/GC7KV7R}{64 308 - Pyrénées Atlantique} — \href{http://coord.info/GC7KV7R\Number{}774453406}{1693}}\cacheData{{2018/06/07 gilles64, Unknown Cache (2/2)}}\begin{cacheText}Afin de récolter les indices nécessaires à la Super Bonus, j’enchaîne sur le parcours 3 accompagnée de mes fidèles bottes. Si je veux une place sur le podium , bien méritée, c'est maintenant ou jamais!!!  Ici aussi, je découvre de très beaux endroits très boueux ! Toutes les caches, exceptée la 319 qui n’a été trouvée par personne, sont découvertes malgré l’absence des spoilers(vraiment pas très au point avec la nouvelle technologie!!!). C’est en sortant du sentier que je retrouve Domino50 qui s’apprête à rentrer après avoir loguer la multi en FTF. Elle décide de m’accompagner sur la bonus à fin de blablater sur la petite Multi que je n’aurai pas l’occasion de faire cette fois ci car l'heure tourne (ayant bloqué mon téléphone ,je suis coupée de la civilisation :on doit s'inquiéter!!!) .Trop contente...je suis également TTF sur la Bonus3.Merci Gilles pour toutes ces caches placées dans de somptueux  décors. j'en redemande....\end{cacheText}

\cacheNumber{1694}\needspace{5\baselineskip}\cacheName{\href{http://coord.info/GC7KV7Y}{64 309 - Pyrénées Atlantique} — \href{http://coord.info/GC7KV7Y\Number{}774453458}{1694}}\cacheData{{2018/06/07 gilles64, Unknown Cache (2/2.5)}}\begin{cacheText}Afin de récolter les indices nécessaires à la Super Bonus, j’enchaîne sur le parcours 3 accompagnée de mes fidèles bottes. Si je veux une place sur le podium , bien méritée, c'est maintenant ou jamais!!!  Ici aussi, je découvre de très beaux endroits très boueux ! Toutes les caches, exceptée la 319 qui n’a été trouvée par personne, sont découvertes malgré l’absence des spoilers(vraiment pas très au point avec la nouvelle technologie!!!). C’est en sortant du sentier que je retrouve Domino50 qui s’apprête à rentrer après avoir loguer la multi en FTF. Elle décide de m’accompagner sur la bonus à fin de blablater sur la petite Multi que je n’aurai pas l’occasion de faire cette fois ci car l'heure tourne (ayant bloqué mon téléphone ,je suis coupée de la civilisation :on doit s'inquiéter!!!) .Trop contente...je suis également TTF sur la Bonus3.Merci Gilles pour toutes ces caches placées dans de somptueux  décors. j'en redemande....\end{cacheText}

\cacheNumber{1695}\needspace{5\baselineskip}\cacheName{\href{http://coord.info/GC7KV82}{64 310 - Pyrénées Atlantique} — \href{http://coord.info/GC7KV82\Number{}774454218}{1695}}\cacheData{{2018/06/07 gilles64, Unknown Cache (2/2)}}\begin{cacheText}Afin de récolter les indices nécessaires à la Super Bonus, j’enchaîne sur le parcours 3 accompagnée de mes fidèles bottes. Si je veux une place sur le podium , bien méritée, c'est maintenant ou jamais!!!  Ici aussi, je découvre de très beaux endroits très boueux ! Toutes les caches, exceptée la 319 qui n’a été trouvée par personne, sont découvertes malgré l’absence des spoilers(vraiment pas très au point avec la nouvelle technologie!!!). C’est en sortant du sentier que je retrouve Domino50 qui s’apprête à rentrer après avoir loguer la multi en FTF. Elle décide de m’accompagner sur la bonus à fin de blablater sur la petite Multi que je n’aurai pas l’occasion de faire cette fois ci car l'heure tourne (ayant bloqué mon téléphone ,je suis coupée de la civilisation :on doit s'inquiéter!!!) .Trop contente...je suis également TTF sur la Bonus3.Merci Gilles pour toutes ces caches placées dans de somptueux  décors. j'en redemande....\end{cacheText}

\cacheNumber{1696}\needspace{5\baselineskip}\cacheName{\href{http://coord.info/GC7KV87}{64 311 - Pyrénées Atlantique} — \href{http://coord.info/GC7KV87\Number{}774454503}{1696}}\cacheData{{2018/06/07 gilles64, Unknown Cache (2/2.5)}}\begin{cacheText}Afin de récolter les indices nécessaires à la Super Bonus, j’enchaîne sur le parcours 3 accompagnée de mes fidèles bottes. Si je veux une place sur le podium , bien méritée, c'est maintenant ou jamais!!!  Ici aussi, je découvre de très beaux endroits très boueux ! Toutes les caches, exceptée la 319 qui n’a été trouvée par personne, sont découvertes malgré l’absence des spoilers(vraiment pas très au point avec la nouvelle technologie!!!). C’est en sortant du sentier que je retrouve Domino50 qui s’apprête à rentrer après avoir loguer la multi en FTF. Elle décide de m’accompagner sur la bonus à fin de blablater sur la petite Multi que je n’aurai pas l’occasion de faire cette fois ci car l'heure tourne (ayant bloqué mon téléphone ,je suis coupée de la civilisation :on doit s'inquiéter!!!) .Trop contente...je suis également TTF sur la Bonus3.Merci Gilles pour toutes ces caches placées dans de somptueux  décors. j'en redemande....\end{cacheText}

\cacheNumber{1697}\needspace{5\baselineskip}\cacheName{\href{http://coord.info/GC7KV8A}{64 312 - Pyrénées Atlantique} — \href{http://coord.info/GC7KV8A\Number{}774454698}{1697}}\cacheData{{2018/06/07 gilles64, Unknown Cache (2/2)}}\begin{cacheText}Afin de récolter les indices nécessaires à la Super Bonus, j’enchaîne sur le parcours 3 accompagnée de mes fidèles bottes. Si je veux une place sur le podium , bien méritée, c'est maintenant ou jamais!!!  Ici aussi, je découvre de très beaux endroits très boueux ! Toutes les caches, exceptée la 319 qui n’a été trouvée par personne, sont découvertes malgré l’absence des spoilers(vraiment pas très au point avec la nouvelle technologie!!!). C’est en sortant du sentier que je retrouve Domino50 qui s’apprête à rentrer après avoir loguer la multi en FTF. Elle décide de m’accompagner sur la bonus à fin de blablater sur la petite Multi que je n’aurai pas l’occasion de faire cette fois ci car l'heure tourne (ayant bloqué mon téléphone ,je suis coupée de la civilisation :on doit s'inquiéter!!!) .Trop contente...je suis également TTF sur la Bonus3.Merci Gilles pour toutes ces caches placées dans de somptueux  décors. j'en redemande....\end{cacheText}

\cacheNumber{1698}\needspace{5\baselineskip}\cacheName{\href{http://coord.info/GC7KV8C}{64 313 - Pyrénées Atlantique} — \href{http://coord.info/GC7KV8C\Number{}774455058}{1698}}\cacheData{{2018/06/07 gilles64, Unknown Cache (2/2)}}\begin{cacheText}Afin de récolter les indices nécessaires à la Super Bonus, j’enchaîne sur le parcours 3 accompagnée de mes fidèles bottes. Si je veux une place sur le podium , bien méritée, c'est maintenant ou jamais!!!  Ici aussi, je découvre de très beaux endroits très boueux ! Toutes les caches, exceptée la 319 qui n’a été trouvée par personne, sont découvertes malgré l’absence des spoilers(vraiment pas très au point avec la nouvelle technologie!!!). C’est en sortant du sentier que je retrouve Domino50 qui s’apprête à rentrer après avoir loguer la multi en FTF. Elle décide de m’accompagner sur la bonus à fin de blablater sur la petite Multi que je n’aurai pas l’occasion de faire cette fois ci car l'heure tourne (ayant bloqué mon téléphone ,je suis coupée de la civilisation :on doit s'inquiéter!!!) .Trop contente...je suis également TTF sur la Bonus3.Merci Gilles pour toutes ces caches placées dans de somptueux  décors. j'en redemande....\end{cacheText}

\cacheNumber{1699}\needspace{5\baselineskip}\cacheName{\href{http://coord.info/GC7KV8G}{64 314 - Pyrénées Atlantique} — \href{http://coord.info/GC7KV8G\Number{}774455275}{1699}}\cacheData{{2018/06/07 gilles64, Unknown Cache (2/2)}}\begin{cacheText}Afin de récolter les indices nécessaires à la Super Bonus, j’enchaîne sur le parcours 3 accompagnée de mes fidèles bottes. Si je veux une place sur le podium , bien méritée, c'est maintenant ou jamais!!!  Ici aussi, je découvre de très beaux endroits très boueux ! Toutes les caches, exceptée la 319 qui n’a été trouvée par personne, sont découvertes malgré l’absence des spoilers(vraiment pas très au point avec la nouvelle technologie!!!). C’est en sortant du sentier que je retrouve Domino50 qui s’apprête à rentrer après avoir loguer la multi en FTF. Elle décide de m’accompagner sur la bonus à fin de blablater sur la petite Multi que je n’aurai pas l’occasion de faire cette fois ci car l'heure tourne (ayant bloqué mon téléphone ,je suis coupée de la civilisation :on doit s'inquiéter!!!) .Trop contente...je suis également TTF sur la Bonus3.Merci Gilles pour toutes ces caches placées dans de somptueux  décors. j'en redemande....\end{cacheText}

\cacheNumber{1700}\needspace{5\baselineskip}\cacheName{\href{http://coord.info/GC7KVZ6}{64 315 - Pyrénées Atlantique} — \href{http://coord.info/GC7KVZ6\Number{}774455882}{1700}}\cacheData{{2018/06/07 gilles64, Unknown Cache (2/2)}}\begin{cacheText}Afin de récolter les indices nécessaires à la Super Bonus, j’enchaîne sur le parcours 3 accompagnée de mes fidèles bottes. Si je veux une place sur le podium , bien méritée, c'est maintenant ou jamais!!!  Ici aussi, je découvre de très beaux endroits très boueux ! Toutes les caches, exceptée la 319 qui n’a été trouvée par personne, sont découvertes malgré l’absence des spoilers(vraiment pas très au point avec la nouvelle technologie!!!). C’est en sortant du sentier que je retrouve Domino50 qui s’apprête à rentrer après avoir loguer la multi en FTF. Elle décide de m’accompagner sur la bonus à fin de blablater sur la petite Multi que je n’aurai pas l’occasion de faire cette fois ci car l'heure tourne (ayant bloqué mon téléphone ,je suis coupée de la civilisation :on doit s'inquiéter!!!) .Trop contente...je suis également TTF sur la Bonus3.Merci Gilles pour toutes ces caches placées dans de somptueux  décors. j'en redemande....\end{cacheText}

\cacheNumber{1701}\needspace{5\baselineskip}\cacheName{\href{http://coord.info/GC7KVZD}{64 316 - Pyrénées Atlantique} — \href{http://coord.info/GC7KVZD\Number{}774455914}{1701}}\cacheData{{2018/06/07 gilles64, Unknown Cache (2/3)}}\begin{cacheText}Afin de récolter les indices nécessaires à la Super Bonus, j’enchaîne sur le parcours 3 accompagnée de mes fidèles bottes. Si je veux une place sur le podium , bien méritée, c'est maintenant ou jamais!!!  Ici aussi, je découvre de très beaux endroits très boueux ! Toutes les caches, exceptée la 319 qui n’a été trouvée par personne, sont découvertes malgré l’absence des spoilers(vraiment pas très au point avec la nouvelle technologie!!!). C’est en sortant du sentier que je retrouve Domino50 qui s’apprête à rentrer après avoir loguer la multi en FTF. Elle décide de m’accompagner sur la bonus à fin de blablater sur la petite Multi que je n’aurai pas l’occasion de faire cette fois ci car l'heure tourne (ayant bloqué mon téléphone ,je suis coupée de la civilisation :on doit s'inquiéter!!!) .Trop contente...je suis également TTF sur la Bonus3.Merci Gilles pour toutes ces caches placées dans de somptueux  décors. j'en redemande....\end{cacheText}

\cacheNumber{1702}\needspace{5\baselineskip}\cacheName{\href{http://coord.info/GC7KVZE}{64 317 - Pyrénées Atlantique} — \href{http://coord.info/GC7KVZE\Number{}774456916}{1702}}\cacheData{{2018/06/07 gilles64, Unknown Cache (2/2.5)}}\begin{cacheText}Afin de récolter les indices nécessaires à la Super Bonus, j’enchaîne sur le parcours 3 accompagnée de mes fidèles bottes. Si je veux une place sur le podium , bien méritée, c'est maintenant ou jamais!!!  Ici aussi, je découvre de très beaux endroits très boueux ! Toutes les caches, exceptée la 319 qui n’a été trouvée par personne, sont découvertes malgré l’absence des spoilers(vraiment pas très au point avec la nouvelle technologie!!!). C’est en sortant du sentier que je retrouve Domino50 qui s’apprête à rentrer après avoir loguer la multi en FTF. Elle décide de m’accompagner sur la bonus à fin de blablater sur la petite Multi que je n’aurai pas l’occasion de faire cette fois ci car l'heure tourne (ayant bloqué mon téléphone ,je suis coupée de la civilisation :on doit s'inquiéter!!!) .Trop contente...je suis également TTF sur la Bonus3.Merci Gilles pour toutes ces caches placées dans de somptueux  décors. j'en redemande....\end{cacheText}

\cacheNumber{1703}\needspace{5\baselineskip}\cacheName{\href{http://coord.info/GC7KVZF}{64 318 - Pyrénées Atlantique} — \href{http://coord.info/GC7KVZF\Number{}774456993}{1703}}\cacheData{{2018/06/07 gilles64, Unknown Cache (2/2.5)}}\begin{cacheText}Afin de récolter les indices nécessaires à la Super Bonus, j’enchaîne sur le parcours 3 accompagnée de mes fidèles bottes. Si je veux une place sur le podium , bien méritée, c'est maintenant ou jamais!!!  Ici aussi, je découvre de très beaux endroits très boueux ! Toutes les caches, exceptée la 319 qui n’a été trouvée par personne, sont découvertes malgré l’absence des spoilers(vraiment pas très au point avec la nouvelle technologie!!!). C’est en sortant du sentier que je retrouve Domino50 qui s’apprête à rentrer après avoir loguer la multi en FTF. Elle décide de m’accompagner sur la bonus à fin de blablater sur la petite Multi que je n’aurai pas l’occasion de faire cette fois ci car l'heure tourne (ayant bloqué mon téléphone ,je suis coupée de la civilisation :on doit s'inquiéter!!!) .Trop contente...je suis également TTF sur la Bonus3.Merci Gilles pour toutes ces caches placées dans de somptueux  décors. j'en redemande....\end{cacheText}

\cacheNumber{1704}\needspace{5\baselineskip}\cacheName{\href{http://coord.info/GC7KVZP}{64 320 - Pyrénées Atlantique} — \href{http://coord.info/GC7KVZP\Number{}774457822}{1704}}\cacheData{{2018/06/07 gilles64, Unknown Cache (2/2)}}\begin{cacheText}Afin de récolter les indices nécessaires à la Super Bonus, j’enchaîne sur le parcours 3 accompagnée de mes fidèles bottes. Si je veux une place sur le podium , bien méritée, c'est maintenant ou jamais!!!  Ici aussi, je découvre de très beaux endroits très boueux ! Toutes les caches, exceptée la 319 qui n’a été trouvée par personne, sont découvertes malgré l’absence des spoilers(vraiment pas très au point avec la nouvelle technologie!!!). C’est en sortant du sentier que je retrouve Domino50 qui s’apprête à rentrer après avoir loguer la multi en FTF. Elle décide de m’accompagner sur la bonus à fin de blablater sur la petite Multi que je n’aurai pas l’occasion de faire cette fois ci car l'heure tourne (ayant bloqué mon téléphone ,je suis coupée de la civilisation :on doit s'inquiéter!!!) .Trop contente...je suis également TTF sur la Bonus3.Merci Gilles pour toutes ces caches placées dans de somptueux  décors. j'en redemande....\end{cacheText}

\cacheNumber{1705}\needspace{5\baselineskip}\cacheName{\href{http://coord.info/GC7KVZT}{64 321 - Pyrénées Atlantique} — \href{http://coord.info/GC7KVZT\Number{}774481296}{1705}}\cacheData{{2018/06/07 gilles64, Unknown Cache (2/2)}}\begin{cacheText}{[STF]}

Afin de récolter les indices nécessaires à la Super Bonus, j’enchaîne sur le parcours 3 accompagnée de mes fidèles bottes. Si je veux une place sur le podium , bien méritée, c'est maintenant ou jamais!!!  Ici aussi, je découvre de très beaux endroits très boueux ! Toutes les caches, exceptée la 319 qui n’a été trouvée par personne, sont découvertes malgré l’absence des spoilers(vraiment pas très au point avec la nouvelle technologie!!!). C’est en sortant du sentier que je retrouve Domino50 qui s’apprête à rentrer après avoir loguer la multi en FTF. Elle décide de m’accompagner sur la bonus à fin de blablater sur la petite Multi que je n’aurai pas l’occasion de faire cette fois ci car l'heure tourne (ayant bloqué mon téléphone ,je suis coupée de la civilisation :on doit s'inquiéter!!!) .Trop contente...je suis également TTF sur la Bonus3.Merci Gilles pour toutes ces caches placées dans de somptueux  décors. j'en redemande....\end{cacheText}

\cacheNumber{1706}\needspace{5\baselineskip}\cacheName{\href{http://coord.info/GC7KVZV}{64 322 - Pyrénées Atlantique} — \href{http://coord.info/GC7KVZV\Number{}774482128}{1706}}\cacheData{{2018/06/07 gilles64, Unknown Cache (2/2)}}\begin{cacheText}[{STF}]

Afin de récolter les indices nécessaires à la Super Bonus, j’enchaîne sur le parcours 3 accompagnée de mes fidèles bottes. Si je veux une place sur le podium , bien méritée, c'est maintenant ou jamais!!!  Ici aussi, je découvre de très beaux endroits très boueux ! Toutes les caches, exceptée la 319 qui n’a été trouvée par personne, sont découvertes malgré l’absence des spoilers(vraiment pas très au point avec la nouvelle technologie!!!). C’est en sortant du sentier que je retrouve Domino50 qui s’apprête à rentrer après avoir loguer la multi en FTF. Elle décide de m’accompagner sur la bonus à fin de blablater sur la petite Multi que je n’aurai pas l’occasion de faire cette fois ci car l'heure tourne (ayant bloqué mon téléphone ,je suis coupée de la civilisation :on doit s'inquiéter!!!) .Trop contente...je suis également TTF sur la Bonus3.Merci Gilles pour toutes ces caches placées dans de somptueux  décors. j'en redemande....\end{cacheText}

\cacheNumber{1707}\needspace{5\baselineskip}\cacheName{\href{http://coord.info/GC7KVZY}{64 323 - Pyrénées Atlantique} — \href{http://coord.info/GC7KVZY\Number{}774484053}{1707}}\cacheData{{2018/06/07 gilles64, Unknown Cache (2/2)}}\begin{cacheText}Afin de récolter les indices nécessaires à la Super Bonus, j’enchaîne sur le parcours 3 accompagnée de mes fidèles bottes. Si je veux une place sur le podium , bien méritée, c'est maintenant ou jamais!!!  Ici aussi, je découvre de très beaux endroits très boueux ! Toutes les caches, exceptée la 319 qui n’a été trouvée par personne, sont découvertes malgré l’absence des spoilers(vraiment pas très au point avec la nouvelle technologie!!!). C’est en sortant du sentier que je retrouve Domino50 qui s’apprête à rentrer après avoir loguer la multi en FTF. Elle décide de m’accompagner sur la bonus à fin de blablater sur la petite Multi que je n’aurai pas l’occasion de faire cette fois ci car l'heure tourne (ayant bloqué mon téléphone ,je suis coupée de la civilisation :on doit s'inquiéter!!!) .Trop contente...je suis également TTF sur la Bonus3.Merci Gilles pour toutes ces caches placées dans de somptueux  décors. j'en redemande....\end{cacheText}

\cacheNumber{1708}\needspace{5\baselineskip}\cacheName{\href{http://coord.info/GC7KW02}{64 324 - Pyrénées Atlantique} — \href{http://coord.info/GC7KW02\Number{}774504334}{1708}}\cacheData{{2018/06/07 gilles64, Unknown Cache (2/2)}}\begin{cacheText}Afin de récolter les indices nécessaires à la Super Bonus, j’enchaîne sur le parcours 3 accompagnée de mes fidèles bottes. Si je veux une place sur le podium , bien méritée, c'est maintenant ou jamais!!!  Ici aussi, je découvre de très beaux endroits très boueux ! Toutes les caches, exceptée la 319 qui n’a été trouvée par personne, sont découvertes malgré l’absence des spoilers(vraiment pas très au point avec la nouvelle technologie!!!). C’est en sortant du sentier que je retrouve Domino50 qui s’apprête à rentrer après avoir loguer la multi en FTF. Elle décide de m’accompagner sur la bonus à fin de blablater sur la petite Multi que je n’aurai pas l’occasion de faire cette fois ci car l'heure tourne (ayant bloqué mon téléphone ,je suis coupée de la civilisation :on doit s'inquiéter!!!) .Trop contente...je suis également TTF sur la Bonus3.Merci Gilles pour toutes ces caches placées dans de somptueux  décors. j'en redemande....\end{cacheText}

\cacheNumber{1709}\needspace{5\baselineskip}\cacheName{\href{http://coord.info/GC7KW05}{64326 Bonus - Pyrénées Atlantique} — \href{http://coord.info/GC7KW05\Number{}774507217}{1709}}\cacheData{{2018/06/07 gilles64, Unknown Cache (3.5/2)}}\begin{cacheText}TTF

Afin de récolter les indices nécessaires à la Super Bonus, j’enchaîne sur le parcours 3 accompagnée de mes fidèles bottes. Si je veux une place sur le podium , bien méritée, c'est maintenant ou jamais!!!  Ici aussi, je découvre de très beaux endroits très boueux ! Toutes les caches, exceptée la 319 qui n’a été trouvée par personne, sont découvertes malgré l’absence des spoilers(vraiment pas très au point avec la nouvelle technologie!!!). C’est en sortant du sentier que je retrouve Domino50 qui s’apprête à rentrer après avoir loguer la multi en FTF. Elle décide de m’accompagner sur la bonus à fin de blablater sur la petite Multi que je n’aurai pas l’occasion de faire cette fois ci car l'heure tourne (ayant bloqué mon téléphone ,je suis coupée de la civilisation :on doit s'inquiéter!!!) .Trop contente...je suis également TTF sur la Bonus3.Merci Gilles pour toutes ces caches placées dans de somptueux  décors. j'en redemande....et un PF de plus.\end{cacheText}

\cacheNumber{1710}\needspace{5\baselineskip}\cacheName{\href{http://coord.info/GC7Q5B6}{64 Super Bonus - Pyrénées Atlantique} — \href{http://coord.info/GC7Q5B6\Number{}774534138}{1710}}\cacheData{{2018/06/07 gilles64, Unknown Cache (4.5/2)}}\begin{cacheText}Coordonnées validées par le checker ,nous partons avec Domino50 vers le trésor.Enfin le Saint Graal!!!!Arrivées au PZ ,nous découvrons une vue magnifique et le soleil, qui est aujourd'hui au rendez vous ,nous permet d'apprécier le paysage. Après avoir tourné un moment autour de la boite, nous finissons par mettre la main dessus .Champagne!!!!Nous récupérons nos récompenses respectives et tamponnons le logbook, fières de nous. Nous nous attardons un peu (le repos du guerrier!!!) pour profiter du soleil et du panorama et nous reprenons la route un peu triste que l'aventure se termine. Je reviendrai effacer ces trois points bleu dés que possible .Un immense MERCI à Gilles et Patricia pour ces trois jours formidables: beaucoup de travail pour beaucoup de plaisir. Un PF pour l'ensemble du 64.\end{cacheText}

\cacheNumber{1711}\needspace{5\baselineskip}\cacheName{\href{http://coord.info/GC7JXM0}{64 119 - Pyrénées Atlantique} — \href{http://coord.info/GC7JXM0\Number{}774435366}{1711}}\cacheData{{2018/06/07 gilles64, Unknown Cache (2/2.5)}}\begin{cacheText}Enfin une journée ensoleillée est annoncée!!!Impossible de résister à la tentation de ce beau Géoart. Équipée de bottes, mieux vaut prévenir que guérir,je rejoins Domino50 pour réaliser ce parcours et glaner tous les indices. C’est un parcours sans faute qui est réalisé : toutes les caches sont découvertes et les paysages sont à la hauteur de nos espérances.Malgré un terrain très boueux et des passages difficiles (et non je ne regrette pas d’avoir les bottes) nous réalisons la balade assez rapidement. Nous concluons le circuit avec la bonus qui nous attend au pied du tordu:TTF pour celle ci. Un grand merci Gilles pour cette série qui nous fait découvrir la beauté et la richesse du Pays Basque.\end{cacheText}

\cacheNumber{1712}\needspace{5\baselineskip}\cacheName{\href{http://coord.info/GC7MTFQ}{Capharnaüm N°2} — \href{http://coord.info/GC7MTFQ\Number{}776454450}{1712}}\cacheData{{2018/06/18 Opmb40, Traditional Cache (1.5/1.5)}}\begin{cacheText}La cache est vite délogée. Merci pour ce petit tour de Mimbaste.\end{cacheText}

\cacheNumber{1713}\needspace{5\baselineskip}\cacheName{\href{http://coord.info/GC7MTG0}{Capharnaüm N°3} — \href{http://coord.info/GC7MTG0\Number{}776451980}{1713}}\cacheData{{2018/06/18 Opmb40, Traditional Cache (1.5/1.5)}}\begin{cacheText}Ici aussi, aucun doute possible… La belle est facilement trouvée. Merci pour la cache.\end{cacheText}

\cacheNumber{1714}\needspace{5\baselineskip}\cacheName{\href{http://coord.info/GC7MTGC}{Capharnaüm N°5} — \href{http://coord.info/GC7MTGC\Number{}776440096}{1714}}\cacheData{{2018/06/18 Opmb40, Traditional Cache (1.5/1.5)}}\begin{cacheText}Rue du Pouy… Je trouve sans difficulté la Belle grâce a l’indice . Merci.\end{cacheText}

\cacheNumber{1715}\needspace{5\baselineskip}\cacheName{\href{http://coord.info/GC7MTQE}{Capharnaüm N°11} — \href{http://coord.info/GC7MTQE\Number{}776449701}{1715}}\cacheData{{2018/06/18 Opmb40, Traditional Cache (1.5/1.5)}}\begin{cacheText}Il n’y a pas grande circulation: cela me permet de chercher et de signer tranquillement. Merci pour la cache.\end{cacheText}

\cacheNumber{1716}\needspace{5\baselineskip}\cacheName{\href{http://coord.info/GC7MTRB}{Capharnaüm N°13} — \href{http://coord.info/GC7MTRB\Number{}776451687}{1716}}\cacheData{{2018/06/18 Opmb40, Traditional Cache (1.5/1.5)}}\begin{cacheText}Aucun doute possible sur le lieu où se trouve la cache ! Vite trouvée ,vite loguée. Merci pour la cache\end{cacheText}

\cacheNumber{1717}\needspace{5\baselineskip}\cacheName{\href{http://coord.info/GC7MTVG}{Capharnaüm N°17} — \href{http://coord.info/GC7MTVG\Number{}776451417}{1717}}\cacheData{{2018/06/18 Opmb40, Traditional Cache (2/1.5)}}\begin{cacheText}La coquine est bien cachée….mais à genoux je la déloge. Merci pour la cache.\end{cacheText}

\cacheNumber{1718}\needspace{5\baselineskip}\cacheName{\href{http://coord.info/GC7N3QZ}{Capharnaüm N°31} — \href{http://coord.info/GC7N3QZ\Number{}776456170}{1718}}\cacheData{{2018/06/18 Opmb40, Traditional Cache (1.5/1.5)}}\begin{cacheText}Quel calme!!!! Je peux chercher sereine la cache. En route pour la suite. ..Merci pour la cache.\end{cacheText}

\cacheNumber{1719}\needspace{5\baselineskip}\cacheName{\href{http://coord.info/GC7N8MM}{Capharnaüm N°38} — \href{http://coord.info/GC7N8MM\Number{}776454791}{1719}}\cacheData{{2018/06/18 Opmb40, Traditional Cache (1.5/1.5)}}\begin{cacheText}Les récentes inondations ont laissé beaucoup de traces dans la région. Mais la belle est bien là. Merci pour la cache.\end{cacheText}

\cacheNumber{1720}\needspace{5\baselineskip}\cacheName{\href{http://coord.info/GC7N8MW}{Capharnaüm N°40} — \href{http://coord.info/GC7N8MW\Number{}776455068}{1720}}\cacheData{{2018/06/18 Opmb40, Traditional Cache (2.5/2.5)}}\begin{cacheText}J’avais quelques doutes en arrivant près du PZ : avec toutes ces inondations j’avais peur qu’elle soit partie. Eh bien non, elle est toujours bien en place!!!!Merci pour la cache\end{cacheText}

\cacheNumber{1721}\needspace{5\baselineskip}\cacheName{\href{http://coord.info/GC7NTZ6}{Capharnaüm N°66} — \href{http://coord.info/GC7NTZ6\Number{}776455909}{1721}}\cacheData{{2018/06/18 Opmb40, Traditional Cache (1.5/1.5)}}\begin{cacheText}Ces petites routes de campagne sont peu fréquentée. Il est aisé de loguer facilement. Merci pour la cache\end{cacheText}

\cacheNumber{1722}\needspace{5\baselineskip}\cacheName{\href{http://coord.info/GC7PC3F}{Capharnaüm N°99} — \href{http://coord.info/GC7PC3F\Number{}776455323}{1722}}\cacheData{{2018/06/18 Opmb40, Traditional Cache (1.5/1.5)}}\begin{cacheText}Les indices donnés ne laissent pas grand endroit où chercher la cache. Elle est bien là effectivement ! Merci pour la cache\end{cacheText}

\cacheNumber{1723}\needspace{5\baselineskip}\cacheName{\href{http://coord.info/GC7PC3T}{Capharnaüm N°102} — \href{http://coord.info/GC7PC3T\Number{}776452354}{1723}}\cacheData{{2018/06/18 Opmb40, Traditional Cache (1.5/2.5)}}\begin{cacheText}Après l’ascension …. la cache est vite découverte sous son caillou : les fourmis ont élu domicile ! Je les déloge pour loguer. Merci pour la cache.\end{cacheText}

\cacheNumber{1724}\needspace{5\baselineskip}\cacheName{\href{http://coord.info/GC7PC8C}{Capharnaüm N°124} — \href{http://coord.info/GC7PC8C\Number{}776439858}{1724}}\cacheData{{2018/06/18 Opmb40, Traditional Cache (1.5/1.5)}}\begin{cacheText}L'indice me mène tout droit à la cache. Un grand merci.\end{cacheText}

\cacheNumber{1725}\needspace{5\baselineskip}\cacheName{\href{http://coord.info/GC7PCBW}{Capharnaüm N°135} — \href{http://coord.info/GC7PCBW\Number{}776455479}{1725}}\cacheData{{2018/06/18 Opmb40, Traditional Cache (1.5/1.5)}}\begin{cacheText}Et hop encore une. Celle-ci est cachée devant une superbe bâtisse. Merci pour la découverte.\end{cacheText}

\cacheNumber{1726}\needspace{5\baselineskip}\cacheName{\href{http://coord.info/GC621GZ}{La grotte de la vierge et son lavoir - Bonloc} — \href{http://coord.info/GC621GZ\Number{}777556697}{1726}}\cacheData{{2018/06/24 gilles64, Traditional Cache (2/1.5)}}\begin{cacheText}Ce matin je suis de corvée pour aller chercher les enfants qui ont passé la nuit aux fêtes d'Hasparren. J’en profite pour partir de bonne heure et faire quelques caches tout autour. C’est une très belle découverte que je fais ici entre le lavoir et la grotte qui est très surprenante. Merci Gilles pour cette belle découverte.\end{cacheText}

\cacheNumber{1727}\needspace{5\baselineskip}\cacheName{\href{http://coord.info/GC621K4}{\Number{}L3-02 Bonloc - Le Comté du Labourd} — \href{http://coord.info/GC621K4\Number{}777557484}{1727}}\cacheData{{2018/06/24 gilles64, Multi-cache (2.5/2)}}\begin{cacheText}Grâce à cette Multi je découvre un charmant petit village chargé d’histoire. Je ne faisais que passer et je ne me doutais pas qu’il y avait tant à voir ! Arrivée au PZ  je cherche quelque peu mais l'œil est attiré…..  Bingo !Merci Gilles pour cette belle découverte\end{cacheText}

\cacheNumber{1728}\needspace{5\baselineskip}\cacheName{\href{http://coord.info/GC62ZEG}{\Number{}CC01 - Le Calvaire des Missionnaires !} — \href{http://coord.info/GC62ZEG\Number{}777561015}{1728}}\cacheData{{2018/06/24 gilles64, Multi-cache (1.5/1.5)}}\begin{cacheText}L'indice relevé ,je me dirige vers le PZ mais l'entrée n'est pas facile à repérer!!!!Je finis par déloger la Belle . Et c’est le début du chemin de croix....Merci pour la cache\end{cacheText}

\cacheNumber{1729}\needspace{5\baselineskip}\cacheName{\href{http://coord.info/GC67BRH}{Lavoir Labiry - Hasparen} — \href{http://coord.info/GC67BRH\Number{}777714001}{1729}}\cacheData{{2018/06/24 gilles64, Traditional Cache (3/1.5)}}\begin{cacheText}Enfin une de trouvée...Je désespérais...La cache est vite découverte car les coordonnées sont ultra précises.. Le lavoir est bien entretenu et le parcours est rigolo .Aucun moldu à l'horizon car c'est l'heure du repas!!!Merci Gilles\end{cacheText}

\cacheNumber{1730}\needspace{5\baselineskip}\cacheName{\href{http://coord.info/GC3TZAE}{La Fontaine Royale  Lupin} — \href{http://coord.info/GC3TZAE\Number{}780517211}{1730}}\cacheData{{2018/06/29 taris17, Traditional Cache (1.5/1.5)}}\begin{cacheText}En week-end avec le GTAQ pour visiter l’ile Madame, nous en profitons pour faire quelques caches aux alentours.Nous arrivons à la tombée de la nuit et profitons du superbe coucher de soleil. La cache est vite débusquée par les fins limiers . Merci pour la découverte de cette superbe Fontaine.\end{cacheText}

\cacheNumber{1731}\needspace{5\baselineskip}\cacheName{\href{http://coord.info/GC6W14C}{Abreuvoire Communal} — \href{http://coord.info/GC6W14C\Number{}780780892}{1731}}\cacheData{{2018/06/29 mrateau, Traditional Cache (1.5/1.5)}}\begin{cacheText}En week-end avec le GTAQ pour visiter l’ile Madame, nous en profitons pour faire quelques caches aux alentours après le repas .C'est à la lampe de poche que nous délogeons la Belle. Merci pour la découverte de ce superbe abreuvoir.\end{cacheText}

\cacheNumber{1732}\needspace{5\baselineskip}\cacheName{\href{http://coord.info/GC1RVR4}{Ile Madame} — \href{http://coord.info/GC1RVR4\Number{}781693972}{1732}}\cacheData{{2018/06/30 Tabbie's Staff, Traditional Cache (1.5/1.5)}}\begin{cacheText}Apres l'Event ,la troupe, baptisée Team GTAQ , s'élance à l'assaut de l'ile et de ses caches. Elles sont toutes dénichées grâce aux fins limiers de l'équipe. L'ile est superbe et nous prenons grand plaisir à découvrir son histoire malgré un thermomètre qui ne cesse de monter!!!Un grand merci .\end{cacheText}

\cacheNumber{1733}\needspace{5\baselineskip}\cacheName{\href{http://coord.info/GC2AYCT}{MOEZE - Croix Hosannière} — \href{http://coord.info/GC2AYCT\Number{}781708916}{1733}}\cacheData{{2018/06/30 Goupil30, Traditional Cache (2/1.5)}}\begin{cacheText}Petite halte sur le retour de la citadelle de Brouage. La chaleur est insupportable mais la team GTAQ ne se démonte pas et part à la recherche de la belle. C’est Dune33 qui la découvre. Merci pour la cache.\end{cacheText}

\cacheNumber{1734}\needspace{5\baselineskip}\cacheName{\href{http://coord.info/GC66HXM}{Le tombolo de l'Île Madame} — \href{http://coord.info/GC66HXM\Number{}781729506}{1734}}\cacheData{{2018/06/30 CMJN, Earthcache (1.5/1)}}\begin{cacheText}C'est lors de l'Event GTAQ fait le tour de l'ile Madame que je découvre le tombolo et sa création. Les réponses sont envoyées.Merci pour toutes ces informations et un PF.\end{cacheText}

\cacheNumber{1735}\needspace{5\baselineskip}\cacheName{\href{http://coord.info/GC6EXCC}{Le port souterrain} — \href{http://coord.info/GC6EXCC\Number{}781702586}{1735}}\cacheData{{2018/06/30 Flibustier17, Traditional Cache (1/1.5)}}\begin{cacheText}C'est pour réaliser la cache Terra Aventura que nous nous rendons à Brouage. Aucun regret, la ville est magnifique. Malgré la fournaise ,la Team GTAQ débusque sans difficulté toutes les caches. Un grand merci et un PF pour la découverte du port souterrain.\end{cacheText}

\cacheNumber{1736}\needspace{5\baselineskip}\cacheName{\href{http://coord.info/GC6FEHD}{Un coup de froid} — \href{http://coord.info/GC6FEHD\Number{}781703522}{1736}}\cacheData{{2018/06/30 Flibustier17, Traditional Cache (1.5/1.5)}}\begin{cacheText}C'est pour réaliser la cache Terra Aventura que nous nous rendons à Brouage. Aucun regret, la ville est magnifique. Malgré la fournaise ,la Team GTAQ débusque sans difficulté toutes les caches. Un grand merci et un PF pour cette glacière qui nous permet de nous rafraichir.\end{cacheText}

\cacheNumber{1737}\needspace{5\baselineskip}\cacheName{\href{http://coord.info/GC6FRVE}{Square Parat} — \href{http://coord.info/GC6FRVE\Number{}781712283}{1737}}\cacheData{{2018/06/30 Flibustier17, Traditional Cache (1.5/1.5)}}\begin{cacheText}Après une journée bien remplie ,(visite de l’île Madame et de la citadelle de Brouage) sous un soleil de plomb, la team GTAQ fait une pause restau sur Rochefort. Avant de rejoindre le camping l’équipe débusque quelques caches supplémentaires. Addiction oblige....À la lumière des lampes torches nous débusquons la Belle. Merci pour la découverte de ce square.\end{cacheText}

\cacheNumber{1738}\needspace{5\baselineskip}\cacheName{\href{http://coord.info/GC6G07V}{Rochefort - La Veille Paroisse} — \href{http://coord.info/GC6G07V\Number{}781713414}{1738}}\cacheData{{2018/06/30 CMJN, Traditional Cache (2/1.5)}}\begin{cacheText}Après une journée bien remplie ,(visite de l’île Madame et de la citadelle de Brouage) sous un soleil de plomb, la team GTAQ fait une pause restau sur Rochefort. Avant de rejoindre le camping l’équipe débusque quelques caches supplémentaires. Addiction oblige....À la lumière des lampes torches nous débusquons la Belle. Merci pour la découverte de cette eglise.\end{cacheText}

\cacheNumber{1739}\needspace{5\baselineskip}\cacheName{\href{http://coord.info/GC736JW}{\Number{}01 Ile Madame - Bienvenue} — \href{http://coord.info/GC736JW\Number{}781693698}{1739}}\cacheData{{2018/06/30 ChrisEtEmi, Traditional Cache (2/2)}}\begin{cacheText}Apres l'Event ,la troupe, baptisée Team GTAQ , s'élance à l'assaut de l'ile et de ses caches. Elles sont toutes dénichées grâce aux fins limiers de l'équipe. L'ile est superbe et nous prenons grand plaisir à découvrir son histoire malgré un thermomètre qui ne cesse de monter!!!Un grand merci et un PF pour l'ensemble de la série .\end{cacheText}

\cacheNumber{1740}\needspace{5\baselineskip}\cacheName{\href{http://coord.info/GC736R4}{\Number{}02 Ile Madame - La croix de galets} — \href{http://coord.info/GC736R4\Number{}781697050}{1740}}\cacheData{{2018/06/30 ChrisEtEmi, Traditional Cache (2/2)}}\begin{cacheText}Apres l'Event ,la troupe, baptisée Team GTAQ , s'élance à l'assaut de l'ile et de ses caches. Elles sont toutes dénichées grâce aux fins limiers de l'équipe. L'ile est superbe et nous prenons grand plaisir à découvrir son histoire malgré un thermomètre qui ne cesse de monter!!!Un grand merci .\end{cacheText}

\cacheNumber{1741}\needspace{5\baselineskip}\cacheName{\href{http://coord.info/GC736V3}{\Number{}03 Ile Madame - Vue sur la redoute} — \href{http://coord.info/GC736V3\Number{}781696620}{1741}}\cacheData{{2018/06/30 ChrisEtEmi, Traditional Cache (1.5/1.5)}}\begin{cacheText}Apres l'Event ,la troupe, baptisée Team GTAQ , s'élance à l'assaut de l'ile et de ses caches. Elles sont toutes dénichées grâce aux fins limiers de l'équipe. L'ile est superbe et nous prenons grand plaisir à découvrir son histoire malgré un thermomètre qui ne cesse de monter!!!Un grand merci .\end{cacheText}

\cacheNumber{1742}\needspace{5\baselineskip}\cacheName{\href{http://coord.info/GC736WH}{\Number{}04 Ile Madame - 2e vue sur la redoute} — \href{http://coord.info/GC736WH\Number{}781696307}{1742}}\cacheData{{2018/06/30 ChrisEtEmi, Traditional Cache (1.5/1.5)}}\begin{cacheText}Apres l'Event ,la troupe, baptisée Team GTAQ , s'élance à l'assaut de l'ile et de ses caches. Elles sont toutes dénichées grâce aux fins limiers de l'équipe. L'ile est superbe et nous prenons grand plaisir à découvrir son histoire malgré un thermomètre qui ne cesse de monter!!!Un grand merci .\end{cacheText}

\cacheNumber{1743}\needspace{5\baselineskip}\cacheName{\href{http://coord.info/GC736YD}{\Number{}05 Ile Madame - Vue sur Fouras} — \href{http://coord.info/GC736YD\Number{}781696065}{1743}}\cacheData{{2018/06/30 ChrisEtEmi, Traditional Cache (2/2)}}\begin{cacheText}Apres l'Event ,la troupe, baptisée Team GTAQ , s'élance à l'assaut de l'ile et de ses caches. Elles sont toutes dénichées grâce aux fins limiers de l'équipe. L'ile est superbe et nous prenons grand plaisir à découvrir son histoire malgré un thermomètre qui ne cesse de monter!!!Un grand merci .\end{cacheText}

\cacheNumber{1744}\needspace{5\baselineskip}\cacheName{\href{http://coord.info/GC7370Q}{\Number{}06 Ile Madame - Le Puits des Insurgés} — \href{http://coord.info/GC7370Q\Number{}781695796}{1744}}\cacheData{{2018/06/30 ChrisEtEmi, Traditional Cache (2/2)}}\begin{cacheText}Apres l'Event ,la troupe, baptisée Team GTAQ , s'élance à l'assaut de l'ile et de ses caches. Elles sont toutes dénichées grâce aux fins limiers de l'équipe. L'ile est superbe et nous prenons grand plaisir à découvrir son histoire malgré un thermomètre qui ne cesse de monter!!!Un grand merci .\end{cacheText}

\cacheNumber{1745}\needspace{5\baselineskip}\cacheName{\href{http://coord.info/GC73722}{\Number{}07 Ile Madame - Au pied de la redoute} — \href{http://coord.info/GC73722\Number{}781695570}{1745}}\cacheData{{2018/06/30 ChrisEtEmi, Traditional Cache (2/1.5)}}\begin{cacheText}Apres l'Event ,la troupe, baptisée Team GTAQ , s'élance à l'assaut de l'ile et de ses caches. Elles sont toutes dénichées grâce aux fins limiers de l'équipe. L'ile est superbe et nous prenons grand plaisir à découvrir son histoire malgré un thermomètre qui ne cesse de monter!!!Un grand merci .\end{cacheText}

\cacheNumber{1746}\needspace{5\baselineskip}\cacheName{\href{http://coord.info/GC73741}{\Number{}08 Ile Madame - Le plateau des Palles} — \href{http://coord.info/GC73741\Number{}781695276}{1746}}\cacheData{{2018/06/30 ChrisEtEmi, Traditional Cache (2/2)}}\begin{cacheText}Apres l'Event ,la troupe, baptisée Team GTAQ , s'élance à l'assaut de l'ile et de ses caches. Elles sont toutes dénichées grâce aux fins limiers de l'équipe. L'ile est superbe et nous prenons grand plaisir à découvrir son histoire malgré un thermomètre qui ne cesse de monter!!!Un grand merci .\end{cacheText}

\cacheNumber{1747}\needspace{5\baselineskip}\cacheName{\href{http://coord.info/GC73754}{\Number{}09 Ile Madame - Les carrelets} — \href{http://coord.info/GC73754\Number{}781695051}{1747}}\cacheData{{2018/06/30 ChrisEtEmi, Traditional Cache (1.5/1.5)}}\begin{cacheText}Apres l'Event ,la troupe, baptisée Team GTAQ , s'élance à l'assaut de l'ile et de ses caches. Elles sont toutes dénichées grâce aux fins limiers de l'équipe. L'ile est superbe et nous prenons grand plaisir à découvrir son histoire malgré un thermomètre qui ne cesse de monter!!!Un grand merci .\end{cacheText}

\cacheNumber{1748}\needspace{5\baselineskip}\cacheName{\href{http://coord.info/GC7376B}{\Number{}10 Ile Madame - La ferme aquacole} — \href{http://coord.info/GC7376B\Number{}781694816}{1748}}\cacheData{{2018/06/30 ChrisEtEmi, Traditional Cache (2/1.5)}}\begin{cacheText}Apres l'Event ,la troupe, baptisée Team GTAQ , s'élance à l'assaut de l'ile et de ses caches. Elles sont toutes dénichées grâce aux fins limiers de l'équipe. L'ile est superbe et nous prenons grand plaisir à découvrir son histoire malgré un thermomètre qui ne cesse de monter!!!Un grand merci .\end{cacheText}

\cacheNumber{1749}\needspace{5\baselineskip}\cacheName{\href{http://coord.info/GC73789}{\Number{}11 Ile Madame - La réserve naturelle} — \href{http://coord.info/GC73789\Number{}781694626}{1749}}\cacheData{{2018/06/30 ChrisEtEmi, Traditional Cache (2/1.5)}}\begin{cacheText}Apres l'Event ,la troupe, baptisée Team GTAQ , s'élance à l'assaut de l'ile et de ses caches. Elles sont toutes dénichées grâce aux fins limiers de l'équipe. L'ile est superbe et nous prenons grand plaisir à découvrir son histoire malgré un thermomètre qui ne cesse de monter!!!Un grand merci .\end{cacheText}

\cacheNumber{1750}\needspace{5\baselineskip}\cacheName{\href{http://coord.info/GC737AP}{\Number{}12 Ile Madame - Entre terre et mer} — \href{http://coord.info/GC737AP\Number{}781694429}{1750}}\cacheData{{2018/06/30 ChrisEtEmi, Traditional Cache (2/1.5)}}\begin{cacheText}Apres l'Event ,la troupe, baptisée Team GTAQ , s'élance à l'assaut de l'ile et de ses caches. Elles sont toutes dénichées grâce aux fins limiers de l'équipe. L'ile est superbe et nous prenons grand plaisir à découvrir son histoire malgré un thermomètre qui ne cesse de monter!!!Un grand merci .\end{cacheText}

\cacheNumber{1751}\needspace{5\baselineskip}\cacheName{\href{http://coord.info/GC737BB}{\Number{}13 Ile Madame - Vue sur Port-des-Barques} — \href{http://coord.info/GC737BB\Number{}781694152}{1751}}\cacheData{{2018/06/30 ChrisEtEmi, Traditional Cache (2/1.5)}}\begin{cacheText}Apres l'Event ,la troupe, baptisée Team GTAQ , s'élance à l'assaut de l'ile et de ses caches. Elles sont toutes dénichées grâce aux fins limiers de l'équipe. L'ile est superbe et nous prenons grand plaisir à découvrir son histoire malgré un thermomètre qui ne cesse de monter!!!Un grand merci .\end{cacheText}

\cacheNumber{1752}\needspace{5\baselineskip}\cacheName{\href{http://coord.info/GC785YV}{Le Chemin de Ronde des Remparts de Brouage} — \href{http://coord.info/GC785YV\Number{}781706089}{1752}}\cacheData{{2018/06/30 Flibustier17, Traditional Cache (1.5/2)}}\begin{cacheText}C'est pour réaliser la cache Terra Aventura que nous nous rendons à Brouage. Aucun regret, la ville est magnifique. Malgré la fournaise ,la Team GTAQ débusque sans difficulté toutes les caches. Un grand merci et un PF\end{cacheText}

\cacheNumber{1753}\needspace{5\baselineskip}\cacheName{\href{http://coord.info/GC7B884}{Brouage} — \href{http://coord.info/GC7B884\Number{}781704701}{1753}}\cacheData{{2018/06/30 Faloegal, Virtual Cache (1.5/2)}}\begin{cacheText}C'est pour réaliser la cache Terra Aventura et surtout la virtuelle que nous nous rendons à Brouage. Aucun regret, la ville est magnifique. Malgré la fournaise ,la Team GTAQ débusque sans difficulté toutes les caches. Un grand merci et un PF pour la découverte de cette splendide citadelle.\end{cacheText}

\cacheNumber{1754}\needspace{5\baselineskip}\cacheName{\href{http://coord.info/GC7R18C}{GTAQ fait le tour de l’île Madame} — \href{http://coord.info/GC7R18C\Number{}781675063}{1754}}\cacheData{{2018/06/30 GTAQ, Event Cache (1/1.5)}}\begin{cacheText}Enfin le week-end tant attendu sur l’île Madame est là !!!! Et je ne suis pas déçue ...Accueil chaleureux des gentils et efficaces organisateurs, nouvelles connaissances et superbes paysages maritimes font de cet Évent  une réussite. Un grand merci à tous.\end{cacheText}

\cacheNumber{1755}\needspace{5\baselineskip}\cacheName{\href{http://coord.info/GC5KVY2}{A10 - Aire de repos de Chermignac Ouest} — \href{http://coord.info/GC5KVY2\Number{}781820823}{1755}}\cacheData{{2018/07/01 gilles64, Traditional Cache (2/2)}}\begin{cacheText}C’est sur le chemin du retour et en compagnie de Dune33 que je découvre la cache. Pas de moldus dans cet endroit, nous sommes bien tranquilles. Merci Gilles.\end{cacheText}

\cacheNumber{1756}\needspace{5\baselineskip}\cacheName{\href{http://coord.info/GC5KXW6}{A10 - Saint-Ciers-du-Taillon Ouest} — \href{http://coord.info/GC5KXW6\Number{}781826603}{1756}}\cacheData{{2018/07/01 gilles64, Traditional Cache (2/2)}}\begin{cacheText}C’est sur le chemin du retour de l’île Madame ,en compagnie de Dune33 que je découvre la belle. La zone est envahie de moustiques ….c'est une horreur mais que ne ferions nous pas pour une cache? Merci Gilles pour cette cache.\end{cacheText}

\cacheNumber{1757}\needspace{5\baselineskip}\cacheName{\href{http://coord.info/GC5WWX9}{A10 - Aire de Saint Leger} — \href{http://coord.info/GC5WWX9\Number{}781823043}{1757}}\cacheData{{2018/07/01 Grim's, Traditional Cache (1.5/2)}}\begin{cacheText}C’est sur le chemin du retour de l’île Madame ,en compagnie de Dune33 que je découvre la belle. Merci Grims pour cette cache.\end{cacheText}

\cacheNumber{1758}\needspace{5\baselineskip}\cacheName{\href{http://coord.info/GC6B66W}{L'écluse de Monportail} — \href{http://coord.info/GC6B66W\Number{}781780635}{1758}}\cacheData{{2018/07/01 CMJN, Traditional Cache (1/1.5)}}\begin{cacheText}Après la pause déjeuner au camping ,une partie du groupe décide de prendre la route. Mais un groupe d’irréductible décide de continuer le sabre pour enfin admirer ce géoart. C’est donc en compagnie de Dune33,de DorisBear et de Crispol40 (Team 5 GTAQ) que je déniche les trésors. Encore de tres belles découvertes.Un grand merci CM JN pour tout ce travail .\end{cacheText}

\cacheNumber{1759}\needspace{5\baselineskip}\cacheName{\href{http://coord.info/GC6MMHZ}{11 - À l'abordage - Codes de pirates (1)} — \href{http://coord.info/GC6MMHZ\Number{}781741865}{1759}}\cacheData{{2018/07/01 CMJN, Unknown Cache (2.5/2.5)}}\begin{cacheText}Après une bonne nuit de sommeil et un copieux petit déjeuner la team GTAQ défie le sabre. Les énigmes, décodées depuis un moment, laissent place à plein de petits trésors. Toutes les caches sont débusquées grâce à la force du groupe. De très belles découvertes tant au niveau des caches que des paysages. Un grand merci CM JN pour tout ce travail.\end{cacheText}

\cacheNumber{1760}\needspace{5\baselineskip}\cacheName{\href{http://coord.info/GC6MMJC}{12 - À l'abordage - Codes de pirates (2)} — \href{http://coord.info/GC6MMJC\Number{}781741321}{1760}}\cacheData{{2018/07/01 CMJN, Unknown Cache (2/1.5)}}\begin{cacheText}Après une bonne nuit de sommeil et un copieux petit déjeuner la team GTAQ défie le sabre. Les énigmes, décodées depuis un moment, laissent place à plein de petits trésors. Toutes les caches sont débusquées grâce à la force du groupe. De très belles découvertes tant au niveau des caches que des paysages. Un grand merci CM JN pour tout ce travail.\end{cacheText}

\cacheNumber{1761}\needspace{5\baselineskip}\cacheName{\href{http://coord.info/GC6MMJH}{13 - À l'abordage - Codes de pirates (3)} — \href{http://coord.info/GC6MMJH\Number{}781740867}{1761}}\cacheData{{2018/07/01 CMJN, Unknown Cache (3/1.5)}}\begin{cacheText}Après une bonne nuit de sommeil et un copieux petit déjeuner la team GTAQ défie le sabre. Les énigmes, décodées depuis un moment, laissent place à plein de petits trésors. Toutes les caches sont débusquées grâce à la force du groupe. De très belles découvertes tant au niveau des caches que des paysages. Un grand merci CM JN pour tout ce travail.\end{cacheText}

\cacheNumber{1762}\needspace{5\baselineskip}\cacheName{\href{http://coord.info/GC6MMJQ}{14 - À l'abordage - Codes de pirates (4)} — \href{http://coord.info/GC6MMJQ\Number{}781740328}{1762}}\cacheData{{2018/07/01 CMJN, Unknown Cache (2/1.5)}}\begin{cacheText}Après une bonne nuit de sommeil et un copieux petit déjeuner la team GTAQ défie le sabre. Les énigmes, décodées depuis un moment, laissent place à plein de petits trésors. Toutes les caches sont débusquées grâce à la force du groupe. De très belles découvertes tant au niveau des caches que des paysages. Un grand merci CM JN pour tout ce travail.\end{cacheText}

\cacheNumber{1763}\needspace{5\baselineskip}\cacheName{\href{http://coord.info/GC6MMK0}{15 - À l'abordage - Codes de pirates (B)} — \href{http://coord.info/GC6MMK0\Number{}781739452}{1763}}\cacheData{{2018/07/01 CMJN, Unknown Cache (1.5/1.5)}}\begin{cacheText}Après une bonne nuit de sommeil et un copieux petit déjeuner la team GTAQ défie le sabre. Les énigmes, décodées depuis un moment, laissent place à plein de petits trésors. Toutes les caches sont débusquées grâce à la force du groupe. De très belles découvertes tant au niveau des caches que des paysages. Un grand merci CM JN pour tout ce travail.\end{cacheText}

\cacheNumber{1764}\needspace{5\baselineskip}\cacheName{\href{http://coord.info/GC6MMMC}{6 - Jeux pour petits pirates (1)} — \href{http://coord.info/GC6MMMC\Number{}781747485}{1764}}\cacheData{{2018/07/01 CMJN, Unknown Cache (2/1.5)}}\begin{cacheText}Après une bonne nuit de sommeil et un copieux petit déjeuner la team GTAQ défie le sabre. Les énigmes, décodées depuis un moment, laissent place à plein de petits trésors. Toutes les caches sont débusquées grâce à la force du groupe. De très belles découvertes tant au niveau des caches que des paysages. Un grand merci CM JN pour tout ce travail.\end{cacheText}

\cacheNumber{1765}\needspace{5\baselineskip}\cacheName{\href{http://coord.info/GC6MMMT}{7 - Jeux pour petits pirates (2)} — \href{http://coord.info/GC6MMMT\Number{}781746928}{1765}}\cacheData{{2018/07/01 CMJN, Unknown Cache (2/1.5)}}\begin{cacheText}Après une bonne nuit de sommeil et un copieux petit déjeuner la team GTAQ défie le sabre. Les énigmes, décodées depuis un moment, laissent place à plein de petits trésors. Toutes les caches sont débusquées grâce à la force du groupe. De très belles découvertes tant au niveau des caches que des paysages. Un grand merci CM JN pour tout ce travail.\end{cacheText}

\cacheNumber{1766}\needspace{5\baselineskip}\cacheName{\href{http://coord.info/GC6MMMY}{8 - Jeux pour petits pirates (3)} — \href{http://coord.info/GC6MMMY\Number{}781746467}{1766}}\cacheData{{2018/07/01 CMJN, Unknown Cache (1/3)}}\begin{cacheText}Après une bonne nuit de sommeil et un copieux petit déjeuner la team GTAQ défie le sabre. Les énigmes, décodées depuis un moment, laissent place à plein de petits trésors. Toutes les caches sont débusquées grâce à la force du groupe. De très belles découvertes tant au niveau des caches que des paysages. Un grand merci CM JN pour tout ce travail.\end{cacheText}

\cacheNumber{1767}\needspace{5\baselineskip}\cacheName{\href{http://coord.info/GC6MMN1}{9 - Jeux pour petits pirates (4)} — \href{http://coord.info/GC6MMN1\Number{}781745965}{1767}}\cacheData{{2018/07/01 CMJN, Unknown Cache (1.5/1.5)}}\begin{cacheText}Après une bonne nuit de sommeil et un copieux petit déjeuner la team GTAQ défie le sabre. Les énigmes, décodées depuis un moment, laissent place à plein de petits trésors. Toutes les caches sont débusquées grâce à la force du groupe. De très belles découvertes tant au niveau des caches que des paysages. Un grand merci CM JN pour tout ce travail.\end{cacheText}

\cacheNumber{1768}\needspace{5\baselineskip}\cacheName{\href{http://coord.info/GC6MMN4}{10 - Jeux pour petits pirates (B)} — \href{http://coord.info/GC6MMN4\Number{}781742672}{1768}}\cacheData{{2018/07/01 CMJN, Unknown Cache (1.5/2)}}\begin{cacheText}Après une bonne nuit de sommeil et un copieux petit déjeuner la team GTAQ défie le sabre. Les énigmes, décodées depuis un moment, laissent place à plein de petits trésors. Toutes les caches sont débusquées grâce à la force du groupe. De très belles découvertes tant au niveau des caches que des paysages. Un grand merci CM JN pour tout ce travail.\end{cacheText}

\cacheNumber{1769}\needspace{5\baselineskip}\cacheName{\href{http://coord.info/GC6MVCZ}{16 - Messages de pirates (1)} — \href{http://coord.info/GC6MVCZ\Number{}781742249}{1769}}\cacheData{{2018/07/01 CMJN, Unknown Cache (1/1.5)}}\begin{cacheText}Après une bonne nuit de sommeil et un copieux petit déjeuner la team GTAQ défie le sabre. Les énigmes, décodées depuis un moment, laissent place à plein de petits trésors. Toutes les caches sont débusquées grâce à la force du groupe. De très belles découvertes tant au niveau des caches que des paysages. Un grand merci CM JN pour tout ce travail.\end{cacheText}

\cacheNumber{1770}\needspace{5\baselineskip}\cacheName{\href{http://coord.info/GC6MVD9}{17 - Messages de pirates (2)} — \href{http://coord.info/GC6MVD9\Number{}781743230}{1770}}\cacheData{{2018/07/01 CMJN, Unknown Cache (1.5/1.5)}}\begin{cacheText}Après une bonne nuit de sommeil et un copieux petit déjeuner la team GTAQ défie le sabre. Les énigmes, décodées depuis un moment, laissent place à plein de petits trésors. Toutes les caches sont débusquées grâce à la force du groupe. De très belles découvertes tant au niveau des caches que des paysages. Un grand merci CM JN pour tout ce travail.\end{cacheText}

\cacheNumber{1771}\needspace{5\baselineskip}\cacheName{\href{http://coord.info/GC6MVDG}{18 - Messages de pirates (3)} — \href{http://coord.info/GC6MVDG\Number{}781744502}{1771}}\cacheData{{2018/07/01 CMJN, Unknown Cache (2/4)}}\begin{cacheText}Après une bonne nuit de sommeil et un copieux petit déjeuner la team GTAQ défie le sabre. Les énigmes, décodées depuis un moment, laissent place à plein de petits trésors. Toutes les caches sont débusquées grâce à la force du groupe. De très belles découvertes tant au niveau des caches que des paysages. Un grand merci CM JN pour tout ce travail.\end{cacheText}

\cacheNumber{1772}\needspace{5\baselineskip}\cacheName{\href{http://coord.info/GC6MVDK}{19 - Messages de pirates (4)} — \href{http://coord.info/GC6MVDK\Number{}781745309}{1772}}\cacheData{{2018/07/01 CMJN, Unknown Cache (2.5/1.5)}}\begin{cacheText}Après une bonne nuit de sommeil et un copieux petit déjeuner la team GTAQ défie le sabre. Les énigmes, décodées depuis un moment, laissent place à plein de petits trésors. Toutes les caches sont débusquées grâce à la force du groupe. De très belles découvertes tant au niveau des caches que des paysages. Un grand merci CM JN pour tout ce travail.\end{cacheText}

\cacheNumber{1773}\needspace{5\baselineskip}\cacheName{\href{http://coord.info/GC6MVDQ}{20 - Histoires de pirates (B)} — \href{http://coord.info/GC6MVDQ\Number{}781780960}{1773}}\cacheData{{2018/07/01 CMJN, Unknown Cache (1.5/1.5)}}\begin{cacheText}Après la pause déjeuner au camping ,une partie du groupe décide de prendre la route. Mais un groupe d’irréductible décide de continuer le sabre pour enfin admirer ce géoart. C’est donc en compagnie de Dune33,de DorisBear et de Crispol40 (Team 5 GTAQ) que je déniche les trésors. Encore de tres belles découvertes.Un grand merci CM JN pour tout ce travail et un PF.\end{cacheText}

\cacheNumber{1774}\needspace{5\baselineskip}\cacheName{\href{http://coord.info/GC6NA41}{21 - Mettons les voiles !} — \href{http://coord.info/GC6NA41\Number{}781768176}{1774}}\cacheData{{2018/07/01 CMJN, Unknown Cache (2.5/1.5)}}\begin{cacheText}Après la pause déjeuner au camping ,une partie du groupe décide de prendre la route. Mais un groupe d’irréductible décide de continuer le sabre pour enfin admirer ce géoart. C’est donc en compagnie de Dune33,de DorisBear et de Crispol40  (Team 5 GTAQ) que je déniche les trésors. Encore de tres belles découvertes.Un grand merci CM JN pour tout ce travail.\end{cacheText}

\cacheNumber{1775}\needspace{5\baselineskip}\cacheName{\href{http://coord.info/GC6NA4T}{22 - Besoin d'une carte ?} — \href{http://coord.info/GC6NA4T\Number{}781765434}{1775}}\cacheData{{2018/07/01 CMJN, Unknown Cache (3/1)}}\begin{cacheText}Après la pause déjeuner au camping ,une partie du groupe décide de prendre la route. Mais un groupe d’irréductible décide de continuer le sabre pour enfin admirer ce géoart. C’est donc en compagnie de Dune33,de DorisBear et de Crispol40  (Team 5 GTAQ) que je déniche les trésors. Encore de tres belles découvertes.Un grand merci CM JN pour tout ce travail.\end{cacheText}

\cacheNumber{1776}\needspace{5\baselineskip}\cacheName{\href{http://coord.info/GC6NA56}{23 - Il est beau mon navire} — \href{http://coord.info/GC6NA56\Number{}781765858}{1776}}\cacheData{{2018/07/01 CMJN, Unknown Cache (3.5/1.5)}}\begin{cacheText}Après la pause déjeuner au camping ,une partie du groupe décide de prendre la route. Mais un groupe d’irréductible décide de continuer le sabre pour enfin admirer ce géoart. C’est donc en compagnie de Dune33,de DorisBear et de Crispol40  (Team 5 GTAQ) que je déniche les trésors. Encore de tres belles découvertes.Un grand merci CM JN pour tout ce travail.\end{cacheText}

\cacheNumber{1777}\needspace{5\baselineskip}\cacheName{\href{http://coord.info/GC6NA5M}{24 - À l'abordage !} — \href{http://coord.info/GC6NA5M\Number{}781766916}{1777}}\cacheData{{2018/07/01 CMJN, Unknown Cache (1.5/1.5)}}\begin{cacheText}Après la pause déjeuner au camping ,une partie du groupe décide de prendre la route. Mais un groupe d’irréductible décide de continuer le sabre pour enfin admirer ce géoart. C’est donc en compagnie de Dune33,de DorisBear et de Crispol40  (Team 5 GTAQ) que je déniche les trésors. Encore de tres belles découvertes.Un grand merci CM JN pour tout ce travail.\end{cacheText}

\cacheNumber{1778}\needspace{5\baselineskip}\cacheName{\href{http://coord.info/GC6NA6T}{25 - Drôles de tours} — \href{http://coord.info/GC6NA6T\Number{}781764414}{1778}}\cacheData{{2018/07/01 CMJN, Unknown Cache (1.5/1.5)}}\begin{cacheText}Après la pause déjeuner au camping ,une partie du groupe décide de prendre la route. Mais un groupe d’irréductible décide de continuer le sabre pour enfin admirer ce géoart. C’est donc en compagnie de Dune33,de DorisBear et de Crispol40  (Team 5 GTAQ) que je déniche les trésors. Encore de tres belles découvertes.Un grand merci CM JN pour tout ce travail.\end{cacheText}

\cacheNumber{1779}\needspace{5\baselineskip}\cacheName{\href{http://coord.info/GC6NAB3}{26 - Hissez le pavillon noir !} — \href{http://coord.info/GC6NAB3\Number{}781769186}{1779}}\cacheData{{2018/07/01 CMJN, Unknown Cache (2/1.5)}}\begin{cacheText}Après la pause déjeuner au camping ,une partie du groupe décide de prendre la route. Mais un groupe d’irréductible décide de continuer le sabre pour enfin admirer ce géoart. C’est donc en compagnie de Dune33,de DorisBear et de Crispol40  (Team 5 GTAQ) que je déniche les trésors. Encore de tres belles découvertes.Un grand merci CM JN pour tout ce travail et un autre PF.\end{cacheText}

\cacheNumber{1780}\needspace{5\baselineskip}\cacheName{\href{http://coord.info/GC6NABH}{1 - Des pirates et des bulles (1)} — \href{http://coord.info/GC6NABH\Number{}781776548}{1780}}\cacheData{{2018/07/01 CMJN, Unknown Cache (1.5/1.5)}}\begin{cacheText}Après la pause déjeuner au camping ,une partie du groupe décide de prendre la route. Mais un groupe d’irréductible décide de continuer le sabre pour enfin admirer ce géoart. C’est donc en compagnie de Dune33,de DorisBear et de Crispol40 (Team 5 GTAQ) que je déniche les trésors. Encore de tres belles découvertes.Un grand merci CM JN pour tout ce travail .\end{cacheText}

\cacheNumber{1781}\needspace{5\baselineskip}\cacheName{\href{http://coord.info/GC6NABK}{2 - Des pirates et des bulles (2)} — \href{http://coord.info/GC6NABK\Number{}781778094}{1781}}\cacheData{{2018/07/01 CMJN, Unknown Cache (1.5/1.5)}}\begin{cacheText}Après la pause déjeuner au camping ,une partie du groupe décide de prendre la route. Mais un groupe d’irréductible décide de continuer le sabre pour enfin admirer ce géoart. C’est donc en compagnie de Dune33,de DorisBear et de Crispol40 (Team 5 GTAQ) que je déniche les trésors. Encore de tres belles découvertes.Un grand merci CM JN pour tout ce travail .\end{cacheText}

\cacheNumber{1782}\needspace{5\baselineskip}\cacheName{\href{http://coord.info/GC6NABW}{3 - Des pirates et des bulles (3)} — \href{http://coord.info/GC6NABW\Number{}781776044}{1782}}\cacheData{{2018/07/01 CMJN, Unknown Cache (1.5/1.5)}}\begin{cacheText}Après la pause déjeuner au camping ,une partie du groupe décide de prendre la route. Mais un groupe d’irréductible décide de continuer le sabre pour enfin admirer ce géoart. C’est donc en compagnie de Dune33,de DorisBear et de Crispol40 (Team 5 GTAQ) que je déniche les trésors. Encore de tres belles découvertes.Un grand merci CM JN pour tout ce travail .\end{cacheText}

\cacheNumber{1783}\needspace{5\baselineskip}\cacheName{\href{http://coord.info/GC6NAC4}{4 - Des pirates et des bulles (4)} — \href{http://coord.info/GC6NAC4\Number{}781779790}{1783}}\cacheData{{2018/07/01 CMJN, Unknown Cache (1.5/1.5)}}\begin{cacheText}Après la pause déjeuner au camping ,une partie du groupe décide de prendre la route. Mais un groupe d’irréductible décide de continuer le sabre pour enfin admirer ce géoart. C’est donc en compagnie de Dune33,de DorisBear et de Crispol40 (Team 5 GTAQ) que je déniche les trésors. Encore de tres belles découvertes.Un grand merci CM JN pour tout ce travail .\end{cacheText}

\cacheNumber{1784}\needspace{5\baselineskip}\cacheName{\href{http://coord.info/GC6NAC9}{5 - Des pirates et des bulles (B)} — \href{http://coord.info/GC6NAC9\Number{}781778598}{1784}}\cacheData{{2018/07/01 CMJN, Unknown Cache (1.5/1.5)}}\begin{cacheText}Après la pause déjeuner au camping ,une partie du groupe décide de prendre la route. Mais un groupe d’irréductible décide de continuer le sabre pour enfin admirer ce géoart. C’est donc en compagnie de Dune33,de DorisBear et de Crispol40 (Team 5 GTAQ) que je déniche les trésors. Encore de tres belles découvertes.Un grand merci CM JN pour tout ce travail .\end{cacheText}

\cacheNumber{1785}\needspace{5\baselineskip}\cacheName{\href{http://coord.info/GC6NZYJ}{27 - Le trésor des pirates (B)} — \href{http://coord.info/GC6NZYJ\Number{}781767769}{1785}}\cacheData{{2018/07/01 CMJN, Unknown Cache (2/2.5)}}\begin{cacheText}Après la pause déjeuner au camping ,une partie du groupe décide de prendre la route. Mais un groupe d’irréductible décide de continuer le sabre pour enfin admirer ce géoart. C’est donc en compagnie de Dune33,de DorisBear et de Crispol40  (Team 5 GTAQ) que je déniche les trésors. Encore de tres belles découvertes.Un grand merci CM JN pour tout ce travail et un PF.\end{cacheText}

\cacheNumber{1786}\needspace{5\baselineskip}\cacheName{\href{http://coord.info/GC6W4EG}{Moulin de Moëze} — \href{http://coord.info/GC6W4EG\Number{}781774924}{1786}}\cacheData{{2018/07/01 mrateau, Traditional Cache (1.5/1.5)}}\begin{cacheText}Après la pause déjeuner au camping ,une partie du groupe décide de prendre la route. Mais un groupe d’irréductible décide de continuer le sabre pour enfin admirer ce géoart. C’est donc en compagnie de Dune33,de DorisBear et de Crispol40  (Team 5 GTAQ) que je déniche les trésors. C'est notre troisième passage ( Un le soir de notre arrivée mais avec la torche ce n'est pas évident!!!,le second la veille en fin d'après midi après la visite de l'ile Madame et de la citadelle de Brouage avec un petit 34°) et sommes très heureux de mettre la main dessus. Merci Stiper40.Encore une belle découverte: le moulin est superbe.Un grand merci mrateau.\end{cacheText}

\cacheNumber{1787}\needspace{5\baselineskip}\cacheName{\href{http://coord.info/GC6JDY1}{REALLY SideTracked - Mios by bob} — \href{http://coord.info/GC6JDY1\Number{}781830778}{1787}}\cacheData{{2018/07/02 2433, Traditional Cache (1.5/1.5)}}\begin{cacheText}Dernier jour de notre escapade:après l’île Madame et une bonne nuit à l’hôtel nous attaquons la journée par cette cache avant de partir sur les berges de la Leyre. Nous découvrons une jolie petite gare et la cache est vite trouvée.Merci.\end{cacheText}

\cacheNumber{1788}\needspace{5\baselineskip}\cacheName{\href{http://coord.info/GC74FHB}{A la claire Fontaine} — \href{http://coord.info/GC74FHB\Number{}781832283}{1788}}\cacheData{{2018/07/02 Miosmel, Traditional Cache (1.5/1.5)}}\begin{cacheText}Dernier jour de notre escapade: après l’île Madame et une bonne nuit à l’hôtel nous attaquons la journée par cette cache avant de partir sur les berges de la Leyre. Nous découvrons une jolie fontaine et la cache est vite trouvée. Merci Miosmel\end{cacheText}

\cacheNumber{1789}\needspace{5\baselineskip}\cacheName{\href{http://coord.info/GC7AA65}{A-01 Les berges de la Leyre} — \href{http://coord.info/GC7AA65\Number{}781926258}{1789}}\cacheData{{2018/07/02 Elsadodo49, Traditional Cache (1.5/1.5)}}\begin{cacheText}Après l’escapade avec le GTAQ sur l'ile Madame nous décidons avec Dune33 de prolonger le week-end. Notre première idée était de faire le héron mais cette grosse chaleur nous dissuade. Nous optons pour le circuit les berges de la Leyre qui nous semble bien plus adapté. Nous n’avions pas du tout, mais alors du tout, pensé aux moustiques. Ils nous ont accompagné tout le long du circuit. C’était terrible. Nous avons découvert toutes les caches sauf la 20( peut-être a-t-elle été avalée par les sangliers qui semblent très présents ici). Le circuit est très agréable et les caches vraiment très sympa .Je le recommande mais en dehors des fortes chaleurs :peaux sensibles s'abstenir!!!. Merci Elsadodo49 pour ce super parcours .Un PF pour l’ensemble de la série posé sur la 25 .\end{cacheText}

\cacheNumber{1790}\needspace{5\baselineskip}\cacheName{\href{http://coord.info/GC7AA8K}{A-02 Les berges de la Leyre} — \href{http://coord.info/GC7AA8K\Number{}781925534}{1790}}\cacheData{{2018/07/02 Elsadodo49, Traditional Cache (1.5/1.5)}}\begin{cacheText}Après l’escapade avec le GTAQ sur l'ile Madame nous décidons avec Dune33 de prolonger le week-end. Notre première idée était de faire le héron mais cette grosse chaleur nous dissuade. Nous optons pour le circuit les berges de la Leyre qui nous semble bien plus adapté. Nous n’avions pas du tout, mais alors du tout, pensé aux moustiques. Ils nous ont accompagné tout le long du circuit. C’était terrible. Nous avons découvert toutes les caches sauf la 20( peut-être a-t-elle été avalée par les sangliers qui semblent très présents ici). Le circuit est très agréable et les caches vraiment très sympa .Je le recommande mais en dehors des fortes chaleurs :peaux sensibles s'abstenir!!!. Merci Elsadodo49 pour ce super parcours .Un PF pour l’ensemble de la série posé sur la 25 .\end{cacheText}

\cacheNumber{1791}\needspace{5\baselineskip}\cacheName{\href{http://coord.info/GC7AA92}{A-03 Les berges de la Leyre} — \href{http://coord.info/GC7AA92\Number{}781910033}{1791}}\cacheData{{2018/07/02 Elsadodo49, Traditional Cache (1.5/1.5)}}\begin{cacheText}Après l’escapade avec le GTAQ sur l'ile Madame nous décidons avec Dune33 de prolonger le week-end. Notre première idée était de faire le héron mais cette grosse chaleur nous dissuade. Nous optons pour le circuit les berges de la Leyre qui nous semble bien plus adapté. Nous n’avions pas du tout, mais alors du tout, pensé aux moustiques. Ils nous ont accompagné tout le long du circuit. C’était terrible. Nous avons découvert toutes les caches sauf la 20( peut-être a-t-elle été avalée par les sangliers qui semblent très présents ici). Le circuit est très agréable et les caches vraiment très sympa .Je le recommande mais en dehors des fortes chaleurs :peaux sensibles s'abstenir!!!. Merci Elsadodo49 pour ce super parcours .Un PF pour l’ensemble de la série posé sur la 25 .\end{cacheText}

\cacheNumber{1792}\needspace{5\baselineskip}\cacheName{\href{http://coord.info/GC7AA9F}{A-25 Les berges de la Leyre} — \href{http://coord.info/GC7AA9F\Number{}781928012}{1792}}\cacheData{{2018/07/02 Elsadodo49, Traditional Cache (3/2.5)}}\begin{cacheText}Après l’escapade avec le GTAQ sur l'ile Madame nous décidons avec Dune33 de prolonger le week-end. Notre première idée était de faire le héron mais cette grosse chaleur nous dissuade. Nous optons pour le circuit les berges de la Leyre qui nous semble bien plus adapté. Nous n’avions pas du tout, mais alors du tout, pensé aux moustiques. Ils nous ont accompagné tout le long du circuit. C’était terrible. Nous avons découvert toutes les caches sauf la 20( peut-être a-t-elle été avalée par les sangliers qui semblent très présents ici). Le circuit est très agréable et les caches vraiment très sympa .Je le recommande mais en dehors des fortes chaleurs :peaux sensibles s'abstenir!!!. Merci Elsadodo49 pour ce super parcours .Un PF pour l’ensemble de la série posé sur la 25 .\end{cacheText}

\cacheNumber{1793}\needspace{5\baselineskip}\cacheName{\href{http://coord.info/GC7AA9G}{A-04 Les berges de la Leyre} — \href{http://coord.info/GC7AA9G\Number{}781909277}{1793}}\cacheData{{2018/07/02 Elsadodo49, Traditional Cache (1.5/1.5)}}\begin{cacheText}Après l’escapade avec le GTAQ sur l'ile Madame nous décidons avec Dune33 de prolonger le week-end. Notre première idée était de faire le héron mais cette grosse chaleur nous dissuade. Nous optons pour le circuit les berges de la Leyre qui nous semble bien plus adapté. Nous n’avions pas du tout, mais alors du tout, pensé aux moustiques. Ils nous ont accompagné tout le long du circuit. C’était terrible. Nous avons découvert toutes les caches sauf la 20( peut-être a-t-elle été avalée par les sangliers qui semblent très présents ici). Le circuit est très agréable et les caches vraiment très sympa .Je le recommande mais en dehors des fortes chaleurs :peaux sensibles s'abstenir!!!. Merci Elsadodo49 pour ce super parcours .Un PF pour l’ensemble de la série posé sur la 25 .\end{cacheText}

\cacheNumber{1794}\needspace{5\baselineskip}\cacheName{\href{http://coord.info/GC7AA9K}{A-05 Les berges de la Leyre} — \href{http://coord.info/GC7AA9K\Number{}781908825}{1794}}\cacheData{{2018/07/02 Elsadodo49, Traditional Cache (2.5/2)}}\begin{cacheText}Après l’escapade avec le GTAQ sur l'ile Madame nous décidons avec Dune33 de prolonger le week-end. Notre première idée était de faire le héron mais cette grosse chaleur nous dissuade. Nous optons pour le circuit les berges de la Leyre qui nous semble bien plus adapté. Nous n’avions pas du tout, mais alors du tout, pensé aux moustiques. Ils nous ont accompagné tout le long du circuit. C’était terrible. Nous avons découvert toutes les caches sauf la 20( peut-être a-t-elle été avalée par les sangliers qui semblent très présents ici). Le circuit est très agréable et les caches vraiment très sympa .Je le recommande mais en dehors des fortes chaleurs :peaux sensibles s'abstenir!!!. Merci Elsadodo49 pour ce super parcours .Un PF pour l’ensemble de la série posé sur la 25 et sur celle ci. .\end{cacheText}

\cacheNumber{1795}\needspace{5\baselineskip}\cacheName{\href{http://coord.info/GC7AA9V}{A-06 Les berges de la Leyre} — \href{http://coord.info/GC7AA9V\Number{}781905402}{1795}}\cacheData{{2018/07/02 Elsadodo49, Traditional Cache (3/2)}}\begin{cacheText}Après l’escapade avec le GTAQ sur l'ile Madame nous décidons avec Dune33 de prolonger le week-end. Notre première idée était de faire le héron mais cette grosse chaleur nous dissuade. Nous optons pour le circuit les berges de la Leyre qui nous semble bien plus adapté. Nous n’avions pas du tout, mais alors du tout, pensé aux moustiques. Ils nous ont accompagné tout le long du circuit. C’était terrible. Nous avons découvert toutes les caches sauf la 20( peut-être a-t-elle été avalée par les sangliers qui semblent très présents ici). Le circuit est très agréable et les caches vraiment très sympa .Je le recommande mais en dehors des fortes chaleurs :peaux sensibles s'abstenir!!!. Merci Elsadodo49 pour ce super parcours .Un PF pour l’ensemble de la série posé sur la 25 .\end{cacheText}

\cacheNumber{1796}\needspace{5\baselineskip}\cacheName{\href{http://coord.info/GC7AAA0}{A-07 Les berges de la Leyre} — \href{http://coord.info/GC7AAA0\Number{}781902129}{1796}}\cacheData{{2018/07/02 Elsadodo49, Traditional Cache (2.5/1.5)}}\begin{cacheText}Après l’escapade avec le GTAQ sur l'ile Madame nous décidons avec Dune33 de prolonger le week-end. Notre première idée était de faire le héron mais cette grosse chaleur nous dissuade. Nous optons pour le circuit les berges de la Leyre qui nous semble bien plus adapté. Nous n’avions pas du tout, mais alors du tout, pensé aux moustiques. Ils nous ont accompagné tout le long du circuit. C’était terrible. Nous avons découvert toutes les caches sauf la 20( peut-être a-t-elle été avalée par les sangliers qui semblent très présents ici). Le circuit est très agréable et les caches vraiment très sympa .Je le recommande mais en dehors des fortes chaleurs :peaux sensibles s'abstenir!!!. Merci Elsadodo49 pour ce super parcours .Un PF pour l’ensemble de la série posé sur la 25 .\end{cacheText}

\cacheNumber{1797}\needspace{5\baselineskip}\cacheName{\href{http://coord.info/GC7AAA6}{A-08 Les berges de la Leyre} — \href{http://coord.info/GC7AAA6\Number{}781891426}{1797}}\cacheData{{2018/07/02 Elsadodo49, Traditional Cache (2.5/2.5)}}\begin{cacheText}Après l’escapade avec le GTAQ sur l'ile Madame nous décidons avec Dune33 de prolonger le week-end. Notre première idée était de faire le héron mais cette grosse chaleur nous dissuade. Nous optons pour le circuit les berges de la Leyre qui nous semble bien plus adapté. Nous n’avions pas du tout, mais alors du tout, pensé aux moustiques. Ils nous ont accompagné tout le long du circuit. C’était terrible. Nous avons découvert toutes les caches sauf la 20( peut-être a-t-elle été avalée par les sangliers qui semblent très présents ici). Le circuit est très agréable et les caches vraiment très sympa .Je le recommande mais en dehors des fortes chaleurs :peaux sensibles s'abstenir!!!. Merci Elsadodo49 pour ce super parcours .Un PF pour l’ensemble de la série posé sur la 25 .\end{cacheText}

\cacheNumber{1798}\needspace{5\baselineskip}\cacheName{\href{http://coord.info/GC7AAA8}{A-24 Les berges de la Leyre} — \href{http://coord.info/GC7AAA8\Number{}781930122}{1798}}\cacheData{{2018/07/02 Elsadodo49, Traditional Cache (2.5/2.5)}}\begin{cacheText}Après l’escapade avec le GTAQ sur l'ile Madame nous décidons avec Dune33 de prolonger le week-end. Notre première idée était de faire le héron mais cette grosse chaleur nous dissuade. Nous optons pour le circuit les berges de la Leyre qui nous semble bien plus adapté. Nous n’avions pas du tout, mais alors du tout, pensé aux moustiques. Ils nous ont accompagné tout le long du circuit. C’était terrible. Nous avons découvert toutes les caches sauf la 20( peut-être a-t-elle été avalée par les sangliers qui semblent très présents ici). Le circuit est très agréable et les caches vraiment très sympa .Je le recommande mais en dehors des fortes chaleurs :peaux sensibles s'abstenir!!!. Merci Elsadodo49 pour ce super parcours .Un PF pour l’ensemble de la série posé sur la 25 .\end{cacheText}

\cacheNumber{1799}\needspace{5\baselineskip}\cacheName{\href{http://coord.info/GC7AAAD}{A-09 Les berges de la Leyre} — \href{http://coord.info/GC7AAAD\Number{}781890633}{1799}}\cacheData{{2018/07/02 Elsadodo49, Traditional Cache (2/2)}}\begin{cacheText}Après l’escapade avec le GTAQ sur l'ile Madame nous décidons avec Dune33 de prolonger le week-end. Notre première idée était de faire le héron mais cette grosse chaleur nous dissuade. Nous optons pour le circuit les berges de la Leyre qui nous semble bien plus adapté. Nous n’avions pas du tout, mais alors du tout, pensé aux moustiques. Ils nous ont accompagné tout le long du circuit. C’était terrible. Nous avons découvert toutes les caches sauf la 20( peut-être a-t-elle été avalée par les sangliers qui semblent très présents ici). Le circuit est très agréable et les caches vraiment très sympa .Je le recommande mais en dehors des fortes chaleurs :peaux sensibles s'abstenir!!!. Merci Elsadodo49 pour ce super parcours .Un PF pour l’ensemble de la série posé sur la 25 .\end{cacheText}

\cacheNumber{1800}\needspace{5\baselineskip}\cacheName{\href{http://coord.info/GC7AAAK}{A-23 Les berges de la Leyre} — \href{http://coord.info/GC7AAAK\Number{}781931084}{1800}}\cacheData{{2018/07/02 Elsadodo49, Traditional Cache (2.5/2.5)}}\begin{cacheText}Après l’escapade avec le GTAQ sur l'ile Madame nous décidons avec Dune33 de prolonger le week-end. Notre première idée était de faire le héron mais cette grosse chaleur nous dissuade. Nous optons pour le circuit les berges de la Leyre qui nous semble bien plus adapté. Nous n’avions pas du tout, mais alors du tout, pensé aux moustiques. Ils nous ont accompagné tout le long du circuit. C’était terrible. Nous avons découvert toutes les caches sauf la 20( peut-être a-t-elle été avalée par les sangliers qui semblent très présents ici). Le circuit est très agréable et les caches vraiment très sympa .Je le recommande mais en dehors des fortes chaleurs :peaux sensibles s'abstenir!!!. Merci Elsadodo49 pour ce super parcours .Un PF pour l’ensemble de la série posé sur la 25 .\end{cacheText}

\cacheNumber{1801}\needspace{5\baselineskip}\cacheName{\href{http://coord.info/GC7AAAT}{A-10 Les berges de la Leyre} — \href{http://coord.info/GC7AAAT\Number{}781890131}{1801}}\cacheData{{2018/07/02 Elsadodo49, Traditional Cache (2/2)}}\begin{cacheText}Après l’escapade avec le GTAQ sur l'ile Madame nous décidons avec Dune33 de prolonger le week-end. Notre première idée était de faire le héron mais cette grosse chaleur nous dissuade. Nous optons pour le circuit les berges de la Leyre qui nous semble bien plus adapté. Nous n’avions pas du tout, mais alors du tout, pensé aux moustiques. Ils nous ont accompagné tout le long du circuit. C’était terrible. Nous avons découvert toutes les caches sauf la 20( peut-être a-t-elle été avalée par les sangliers qui semblent très présents ici). Le circuit est très agréable et les caches vraiment très sympa .Je le recommande mais en dehors des fortes chaleurs :peaux sensibles s'abstenir!!!. Merci Elsadodo49 pour ce super parcours .Un PF pour l’ensemble de la série posé sur la 25 .\end{cacheText}

\cacheNumber{1802}\needspace{5\baselineskip}\cacheName{\href{http://coord.info/GC7AAB9}{A-22 les berges de la Leyre} — \href{http://coord.info/GC7AAB9\Number{}781932555}{1802}}\cacheData{{2018/07/02 Elsadodo49, Traditional Cache (3/3)}}\begin{cacheText}Après l’escapade avec le GTAQ sur l'ile Madame nous décidons avec Dune33 de prolonger le week-end. Notre première idée était de faire le héron mais cette grosse chaleur nous dissuade. Nous optons pour le circuit les berges de la Leyre qui nous semble bien plus adapté. Nous n’avions pas du tout, mais alors du tout, pensé aux moustiques. Ils nous ont accompagné tout le long du circuit. C’était terrible. Nous avons découvert toutes les caches sauf la 20( peut-être a-t-elle été avalée par les sangliers qui semblent très présents ici). Le circuit est très agréable et les caches vraiment très sympa .Je le recommande mais en dehors des fortes chaleurs :peaux sensibles s'abstenir!!!. Merci Elsadodo49 pour ce super parcours .Un PF pour l’ensemble de la série posé sur la 25 .\end{cacheText}

\cacheNumber{1803}\needspace{5\baselineskip}\cacheName{\href{http://coord.info/GC7AABF}{A-11 Les berges de la Leyre} — \href{http://coord.info/GC7AABF\Number{}781887604}{1803}}\cacheData{{2018/07/02 Elsadodo49, Traditional Cache (2/2)}}\begin{cacheText}Après l’escapade avec le GTAQ sur l'ile Madame nous décidons avec Dune33 de prolonger le week-end. Notre première idée était de faire le héron mais cette grosse chaleur nous dissuade. Nous optons pour le circuit les berges de la Leyre qui nous semble bien plus adapté. Nous n’avions pas du tout, mais alors du tout, pensé aux moustiques. Ils nous ont accompagné tout le long du circuit. C’était terrible. Nous avons découvert toutes les caches sauf la 20( peut-être a-t-elle été avalée par les sangliers qui semblent très présents ici). Le circuit est très agréable et les caches vraiment très sympa .Je le recommande mais en dehors des fortes chaleurs :peaux sensibles s'abstenir!!!. Merci Elsadodo49 pour ce super parcours .Un PF pour l’ensemble de la série posé sur la 25 .\end{cacheText}

\cacheNumber{1804}\needspace{5\baselineskip}\cacheName{\href{http://coord.info/GC7AABJ}{A-21 Les berges de la Leyre} — \href{http://coord.info/GC7AABJ\Number{}781936534}{1804}}\cacheData{{2018/07/02 Elsadodo49, Traditional Cache (2/2)}}\begin{cacheText}Après l’escapade avec le GTAQ sur l'ile Madame nous décidons avec Dune33 de prolonger le week-end. Notre première idée était de faire le héron mais cette grosse chaleur nous dissuade. Nous optons pour le circuit les berges de la Leyre qui nous semble bien plus adapté. Nous n’avions pas du tout, mais alors du tout, pensé aux moustiques. Ils nous ont accompagné tout le long du circuit. C’était terrible. Nous avons découvert toutes les caches sauf la 20( peut-être a-t-elle été avalée par les sangliers qui semblent très présents ici). Le circuit est très agréable et les caches vraiment très sympa .Je le recommande mais en dehors des fortes chaleurs :peaux sensibles s'abstenir!!!. Merci Elsadodo49 pour ce super parcours .Un PF pour l’ensemble de la série posé sur la 25 .\end{cacheText}

\cacheNumber{1805}\needspace{5\baselineskip}\cacheName{\href{http://coord.info/GC7AABM}{A-12 Les berges de la Leyre} — \href{http://coord.info/GC7AABM\Number{}781886839}{1805}}\cacheData{{2018/07/02 Elsadodo49, Traditional Cache (2/2)}}\begin{cacheText}Après l’escapade avec le GTAQ sur l'ile Madame nous décidons avec Dune33 de prolonger le week-end. Notre première idée était de faire le héron mais cette grosse chaleur nous dissuade. Nous optons pour le circuit les berges de la Leyre qui nous semble bien plus adapté. Nous n’avions pas du tout, mais alors du tout, pensé aux moustiques. Ils nous ont accompagné tout le long du circuit. C’était terrible. Nous avons découvert toutes les caches sauf la 20( peut-être a-t-elle été avalée par les sangliers qui semblent très présents ici). Le circuit est très agréable et les caches vraiment très sympa .Je le recommande mais en dehors des fortes chaleurs :peaux sensibles s'abstenir!!!. Merci Elsadodo49 pour ce super parcours .Un PF pour l’ensemble de la série posé sur la 25 .\end{cacheText}

\cacheNumber{1806}\needspace{5\baselineskip}\cacheName{\href{http://coord.info/GC7AABT}{A-13 Les berges de la Leyre} — \href{http://coord.info/GC7AABT\Number{}781881074}{1806}}\cacheData{{2018/07/02 Elsadodo49, Traditional Cache (2/2)}}\begin{cacheText}Après l’escapade avec le GTAQ sur l'ile Madame nous décidons avec Dune33 de prolonger le week-end. Notre première idée était de faire le héron mais cette grosse chaleur nous dissuade. Nous optons pour le circuit les berges de la Leyre qui nous semble bien plus adapté. Nous n’avions pas du tout, mais alors du tout, pensé aux moustiques. Ils nous ont accompagné tout le long du circuit. C’était terrible. Nous avons découvert toutes les caches sauf la 20( peut-être a-t-elle été avalée par les sangliers qui semblent très présents ici). Le circuit est très agréable et les caches vraiment très sympa .Je le recommande mais en dehors des fortes chaleurs :peaux sensibles s'abstenir!!!. Merci Elsadodo49 pour ce super parcours .Un PF pour l’ensemble de la série posé sur la 25 .\end{cacheText}

\cacheNumber{1807}\needspace{5\baselineskip}\cacheName{\href{http://coord.info/GC7AABW}{A-19 Les berges de la Leyre} — \href{http://coord.info/GC7AABW\Number{}781940493}{1807}}\cacheData{{2018/07/02 Elsadodo49, Traditional Cache (2/2)}}\begin{cacheText}Après l’escapade avec le GTAQ sur l'ile Madame nous décidons avec Dune33 de prolonger le week-end. Notre première idée était de faire le héron mais cette grosse chaleur nous dissuade. Nous optons pour le circuit les berges de la Leyre qui nous semble bien plus adapté. Nous n’avions pas du tout, mais alors du tout, pensé aux moustiques. Ils nous ont accompagné tout le long du circuit. C’était terrible. Nous avons découvert toutes les caches sauf la 20( peut-être a-t-elle été avalée par les sangliers qui semblent très présents ici). Le circuit est très agréable et les caches vraiment très sympa .Je le recommande mais en dehors des fortes chaleurs :peaux sensibles s'abstenir!!!. Merci Elsadodo49 pour ce super parcours .Un PF pour l’ensemble de la série posé sur la 25 .\end{cacheText}

\cacheNumber{1808}\needspace{5\baselineskip}\cacheName{\href{http://coord.info/GC7AABZ}{A-14 Les berges de la Leyre} — \href{http://coord.info/GC7AABZ\Number{}781845708}{1808}}\cacheData{{2018/07/02 Elsadodo49, Traditional Cache (2/2)}}\begin{cacheText}Après l’escapade avec le GTAQ sur l'ile Madame nous décidons avec Dune33 de prolonger le week-end. Notre première idée était de faire le héron mais cette grosse chaleur nous dissuade. Nous optons pour le circuit les berges de la Leyre qui nous semble bien plus adapté. Nous n’avions pas du tout, mais alors du tout, pensé aux moustiques. Ils nous ont accompagné tout le long du circuit. C’était terrible. Nous avons découvert toutes les caches sauf la 20( peut-être a-t-elle été avalée par les sangliers qui semblent très présents ici). Le circuit est très agréable et les caches vraiment très sympa .Je le recommande mais en dehors des fortes chaleurs :peaux sensibles s'abstenir!!!. Merci Elsadodo49 pour ce super parcours .Un PF pour l’ensemble de la série posé sur la 25 .\end{cacheText}

\cacheNumber{1809}\needspace{5\baselineskip}\cacheName{\href{http://coord.info/GC7AAC2}{A-18 Les berges de la Leyre} — \href{http://coord.info/GC7AAC2\Number{}781941298}{1809}}\cacheData{{2018/07/02 Elsadodo49, Traditional Cache (2.5/2.5)}}\begin{cacheText}Après l’escapade avec le GTAQ sur l'ile Madame nous décidons avec Dune33 de prolonger le week-end. Notre première idée était de faire le héron mais cette grosse chaleur nous dissuade. Nous optons pour le circuit les berges de la Leyre qui nous semble bien plus adapté. Nous n’avions pas du tout, mais alors du tout, pensé aux moustiques. Ils nous ont accompagné tout le long du circuit. C’était terrible. Nous avons découvert toutes les caches sauf la 20( peut-être a-t-elle été avalée par les sangliers qui semblent très présents ici). Le circuit est très agréable et les caches vraiment très sympa .Je le recommande mais en dehors des fortes chaleurs :peaux sensibles s'abstenir!!!. Merci Elsadodo49 pour ce super parcours .Un PF pour l’ensemble de la série posé sur la 25 .\end{cacheText}

\cacheNumber{1810}\needspace{5\baselineskip}\cacheName{\href{http://coord.info/GC7AAC3}{A-15 Les berges de la Leyre} — \href{http://coord.info/GC7AAC3\Number{}781845253}{1810}}\cacheData{{2018/07/02 Elsadodo49, Traditional Cache (2/2.5)}}\begin{cacheText}Après l’escapade avec le GTAQ sur l'ile Madame nous décidons avec Dune33 de prolonger le week-end. Notre première idée était de faire le héron mais cette grosse chaleur nous dissuade. Nous optons pour le circuit les berges de la Leyre qui nous semble bien plus adapté. Nous n’avions pas du tout, mais alors du tout, pensé aux moustiques. Ils nous ont accompagné tout le long du circuit. C’était terrible. Nous avons découvert toutes les caches sauf la 20( peut-être a-t-elle été avalée par les sangliers qui semblent très présents ici). Le circuit est très agréable et les caches vraiment très sympa .Je le recommande mais en dehors des fortes chaleurs :peaux sensibles s'abstenir!!!. Merci Elsadodo49 pour ce super parcours .Un PF pour l’ensemble de la série posé sur la 25 .\end{cacheText}

\cacheNumber{1811}\needspace{5\baselineskip}\cacheName{\href{http://coord.info/GC7AAC6}{A-17 Les berges de la Leyre} — \href{http://coord.info/GC7AAC6\Number{}781942190}{1811}}\cacheData{{2018/07/02 Elsadodo49, Traditional Cache (2.5/2.5)}}\begin{cacheText}Après l’escapade avec le GTAQ sur l'ile Madame nous décidons avec Dune33 de prolonger le week-end. Notre première idée était de faire le héron mais cette grosse chaleur nous dissuade. Nous optons pour le circuit les berges de la Leyre qui nous semble bien plus adapté. Nous n’avions pas du tout, mais alors du tout, pensé aux moustiques. Ils nous ont accompagné tout le long du circuit. C’était terrible. Nous avons découvert toutes les caches sauf la 20( peut-être a-t-elle été avalée par les sangliers qui semblent très présents ici). Le circuit est très agréable et les caches vraiment très sympa .Je le recommande mais en dehors des fortes chaleurs :peaux sensibles s'abstenir!!!. Merci Elsadodo49 pour ce super parcours .Un PF pour l’ensemble de la série posé sur la 25 .\end{cacheText}

\cacheNumber{1812}\needspace{5\baselineskip}\cacheName{\href{http://coord.info/GC7AAC8}{A-16 Les berges de la Leyre} — \href{http://coord.info/GC7AAC8\Number{}781844934}{1812}}\cacheData{{2018/07/02 Elsadodo49, Traditional Cache (2/2)}}\begin{cacheText}Après l’escapade avec le GTAQ sur l'ile Madame nous décidons avec Dune33 de prolonger le week-end. Notre première idée était de faire le héron mais cette grosse chaleur nous dissuade. Nous optons pour le circuit les berges de la Leyre qui nous semble bien plus adapté. Nous n’avions pas du tout, mais alors du tout, pensé aux moustiques. Ils nous ont accompagné tout le long du circuit. C’était terrible. Nous avons découvert toutes les caches sauf la 20( peut-être a-t-elle été avalée par les sangliers qui semblent très présents ici). Le circuit est très agréable et les caches vraiment très sympa .Je le recommande mais en dehors des fortes chaleurs :peaux sensibles s'abstenir!!!. Merci Elsadodo49 pour ce super parcours .Un PF pour l’ensemble de la série posé sur la 25 .\end{cacheText}

\cacheNumber{1813}\needspace{5\baselineskip}\cacheName{\href{http://coord.info/GC7NM46}{La cache de Jules} — \href{http://coord.info/GC7NM46\Number{}781914158}{1813}}\cacheData{{2018/07/02 Maxikass, Traditional Cache (2.5/3)}}\begin{cacheText}Après l’escapade avec le GTAQ sur l'ile Madame nous décidons avec Dune33 de prolonger le week-end. Notre première idée était de faire le héron mais cette grosse chaleur nous dissuade. Nous optons pour le circuit les berges de la Leyre qui nous semble bien plus adapté. Nous n’avions pas du tout, mais alors du tout, pensé aux moustiques. Ils nous ont accompagné tout le long du circuit!!!!

Nous faisons un petit détour pour découvrir cette fontaine sans eau oubliée de tous ou presque. Un PF pour l'endroit qui est charmant. Merci pour la cache.\end{cacheText}

\cacheNumber{1814}\needspace{5\baselineskip}\cacheName{\href{http://coord.info/GC7NW7Z}{[33 TOUR] MIOS - V2 by bob} — \href{http://coord.info/GC7NW7Z\Number{}781834633}{1814}}\cacheData{{2018/07/02 2433 sur une idée de MKL33210, Traditional Cache (1.5/1.5)}}\begin{cacheText}Dernier jour de notre escapade: après l’île Madame et une bonne nuit à l’hôtel nous attaquons la journée par cette cache avant de partir sur les berges de la Leyre.La Belle est trouvée en compagnie de Dune33.Merci pour la cache.\end{cacheText}

\cacheNumber{1815}\needspace{5\baselineskip}\cacheName{\href{http://coord.info/GC5JV2E}{Bilan Complet pour Sujiva-03} — \href{http://coord.info/GC5JV2E\Number{}781954245}{1815}}\cacheData{{2018/07/07 teamlilas, Traditional Cache (1.5/1.5)}}\begin{cacheText}La cache est trouvée lors de mon passage pour aller à l'Event de Dune33 à Miramont Sensacq. La vue sur le lac est très sympa. Merci pour la cache.\end{cacheText}

\cacheNumber{1816}\needspace{5\baselineskip}\cacheName{\href{http://coord.info/GC7F65J}{Rosalie en vadrouille *NINA*} — \href{http://coord.info/GC7F65J\Number{}784380382}{1816}}\cacheData{{2018/07/07 dune33, Traditional Cache (1.5/2)}}\begin{cacheText}C’est lors de l’Event ,Lac de Miramont-Sensacq, et en compagnie de EMOA 06 et de Flapouille que je découvre ce magnifique nid fabriqué par notre cher Dédé.Pour y arriver il faut traverser beaucoup de houx mais cela vaut vraiment le coup. Merci Dune33 et un PF évidemment...\end{cacheText}

\cacheNumber{1817}\needspace{5\baselineskip}\cacheName{\href{http://coord.info/GC7Q7JY}{EVENT Lac de Miramont-Sensacq} — \href{http://coord.info/GC7Q7JY\Number{}784397011}{1817}}\cacheData{{2018/07/07 dune33, Event Cache (1/1.5)}}\begin{cacheText}Une vrai réussite ce premier Évent.Un Event haut en couleur, gai et chaleureux à l’image des propriétaires.Un grand bravo pour le choix du lieu ( très sympa la maison au bord du lac), pour l’organisation ( Un logbook admirable, des badges ,des caches parfaites) et pour le repas ( tout était délicieux).. J’ai eu plaisir à revoir les géocacheurs locaux et à rencontrer de nouvelles têtes qui nous ont réservé une petite surprise avec la whérigo que nous sommes allés chercher de nuit. Tout était parfait….MERCI. A quand le prochain?\end{cacheText}

\cacheNumber{1818}\needspace{5\baselineskip}\cacheName{\href{http://coord.info/GC7RXDQ}{1-Vue sur le lac} — \href{http://coord.info/GC7RXDQ\Number{}784418531}{1818}}\cacheData{{2018/07/07 dune33, Traditional Cache (1.5/1.5)}}\begin{cacheText}Co (FTF) avec la Team Dune33

C'est lors de l'Event Lac de Miramont Sensacq que les caches sont découvertes et loguées sous le pseudo Team Dune33.

Les propriétaires nous ont bien aidé pour certaines d'entre elles car trés trés bien camouflées!!!

Du trés bon travail les amis et un grand merci.\end{cacheText}

\cacheNumber{1819}\needspace{5\baselineskip}\cacheName{\href{http://coord.info/GC7RXE6}{2-Vue sur le lac} — \href{http://coord.info/GC7RXE6\Number{}784419557}{1819}}\cacheData{{2018/07/07 dune33, Traditional Cache (1.5/1.5)}}\begin{cacheText}Co (FTF) avec la Team Dune33

C'est lors de l'Event Lac de Miramont Sensacq que les caches sont découvertes et loguées sous le pseudo Team Dune33.

Les propriétaires nous ont bien aidé pour certaines d'entre elles car trés trés bien camouflées!!!

Du trés bon travail les amis et un grand merci.\end{cacheText}

\cacheNumber{1820}\needspace{5\baselineskip}\cacheName{\href{http://coord.info/GC7RXE8}{3-Vue sur le lac} — \href{http://coord.info/GC7RXE8\Number{}784420494}{1820}}\cacheData{{2018/07/07 dune33, Traditional Cache (1.5/1.5)}}\begin{cacheText}Co (FTF) avec la Team Dune33

C'est lors de l'Event Lac de Miramont Sensacq que les caches sont découvertes et loguées sous le pseudo Team Dune33.

Les propriétaires nous ont bien aidé pour certaines d'entre elles car trés trés bien camouflées!!!

Du trés bon travail les amis et un grand merci.\end{cacheText}

\cacheNumber{1821}\needspace{5\baselineskip}\cacheName{\href{http://coord.info/GC7RXEZ}{4-Vue sur le lac} — \href{http://coord.info/GC7RXEZ\Number{}784421401}{1821}}\cacheData{{2018/07/07 dune33, Traditional Cache (1.5/1.5)}}\begin{cacheText}Co (FTF) avec la Team Dune33

C'est lors de l'Event Lac de Miramont Sensacq que les caches sont découvertes et loguées sous le pseudo Team Dune33.

Les propriétaires nous ont bien aidé pour certaines d'entre elles car trés trés bien camouflées!!!

Du trés bon travail les amis et un grand merci.\end{cacheText}

\cacheNumber{1822}\needspace{5\baselineskip}\cacheName{\href{http://coord.info/GC7RXF3}{5-Vue sur le lac} — \href{http://coord.info/GC7RXF3\Number{}784421881}{1822}}\cacheData{{2018/07/07 dune33, Traditional Cache (1.5/1.5)}}\begin{cacheText}Co (FTF) avec la Team Dune33

C'est lors de l'Event Lac de Miramont Sensacq que les caches sont découvertes et loguées sous le pseudo Team Dune33.

Les propriétaires nous ont bien aidé pour certaines d'entre elles car trés trés bien camouflées!!!

Du trés bon travail les amis et un grand merci.\end{cacheText}

\cacheNumber{1823}\needspace{5\baselineskip}\cacheName{\href{http://coord.info/GC7RXF8}{6-Vue sur le lac} — \href{http://coord.info/GC7RXF8\Number{}784423225}{1823}}\cacheData{{2018/07/07 dune33, Traditional Cache (1.5/1.5)}}\begin{cacheText}Co (FTF) avec la Team Dune33

C'est lors de l'Event Lac de Miramont Sensacq que les caches sont découvertes et loguées sous le pseudo Team Dune33.

Les propriétaires nous ont bien aidé pour certaines d'entre elles car trés trés bien camouflées!!!

Du trés bon travail les amis et un grand merci.\end{cacheText}

\cacheNumber{1824}\needspace{5\baselineskip}\cacheName{\href{http://coord.info/GC7RXFA}{7-Vue sur le lac} — \href{http://coord.info/GC7RXFA\Number{}784426666}{1824}}\cacheData{{2018/07/07 dune33, Traditional Cache (1.5/1.5)}}\begin{cacheText}Co (FTF) avec la Team Dune33

C'est lors de l'Event Lac de Miramont Sensacq que les caches sont découvertes et loguées sous le pseudo Team Dune33.

Les propriétaires nous ont bien aidé pour certaines d'entre elles car trés trés bien camouflées!!!

Du trés bon travail les amis et un grand merci.\end{cacheText}

\cacheNumber{1825}\needspace{5\baselineskip}\cacheName{\href{http://coord.info/GC7RXFH}{8-Vue sur le lac} — \href{http://coord.info/GC7RXFH\Number{}784426280}{1825}}\cacheData{{2018/07/07 dune33, Traditional Cache (1.5/1.5)}}\begin{cacheText}Co (FTF) avec la Team Dune33

C'est lors de l'Event Lac de Miramont Sensacq que les caches sont découvertes et loguées sous le pseudo Team Dune33.

Les propriétaires nous ont bien aidé pour certaines d'entre elles car trés trés bien camouflées!!!

Du trés bon travail les amis et un grand merci.\end{cacheText}

\cacheNumber{1826}\needspace{5\baselineskip}\cacheName{\href{http://coord.info/GC7RXFM}{9-Vue sur le lac} — \href{http://coord.info/GC7RXFM\Number{}784427507}{1826}}\cacheData{{2018/07/07 dune33, Traditional Cache (1.5/1.5)}}\begin{cacheText}Co (FTF) avec la Team Dune33

C'est lors de l'Event Lac de Miramont Sensacq que les caches sont découvertes et loguées sous le pseudo Team Dune33.

Les propriétaires nous ont bien aidé pour certaines d'entre elles car trés trés bien camouflées!!!

Du trés bon travail les amis et un grand merci.\end{cacheText}

\cacheNumber{1827}\needspace{5\baselineskip}\cacheName{\href{http://coord.info/GC7RXFQ}{10-Vue sur le lac} — \href{http://coord.info/GC7RXFQ\Number{}784425531}{1827}}\cacheData{{2018/07/07 dune33, Traditional Cache (1.5/1.5)}}\begin{cacheText}Co (FTF) avec la Team Dune33

C'est lors de l'Event Lac de Miramont Sensacq que les caches sont découvertes et loguées sous le pseudo Team Dune33.

Les propriétaires nous ont bien aidé pour certaines d'entre elles car trés trés bien camouflées!!!

Du trés bon travail les amis et un grand merci.\end{cacheText}

\cacheNumber{1828}\needspace{5\baselineskip}\cacheName{\href{http://coord.info/GC7T89M}{Quelle star cette Dune33!} — \href{http://coord.info/GC7T89M\Number{}784435184}{1828}}\cacheData{{2018/07/07 flapemoa \And{} domino, Wherigo Cache (2.5/2)}}\begin{cacheText}Co (FTF) avec la Team Dune33

C'est lors de l'Event Lac de Miramont Sensacq que la cache est découverte et loguées sous le pseudo Team Dune33 de nuit.La cache est ingénieuse...Bravo Dédé!!!

C'est ma seconde wherigo et c'est grace à l'aide de Flapouille et de EMOA06 que je réussit à aller au bout du jeu et obtenir les coordonnées finales. Un grand merci pour ce bon moment et un PF.\end{cacheText}

\cacheNumber{1829}\needspace{5\baselineskip}\cacheName{\href{http://coord.info/GC14T9H}{IRATI ~ Salbatore Kapera ~ Nun da Basa Jauna ?} — \href{http://coord.info/GC14T9H\Number{}784441130}{1829}}\cacheData{{2018/07/10 laminak, Traditional Cache (1.5/1.5)}}\begin{cacheText}C'est sur le retour d'Iraty et de la découverte de la Terra Aventura, que nous trouvons en compagnie de Doudou 91 cette superbe chapelle fréquentée par les moutons!!!La cache est toujours impeccable malgré le temps.Merci\end{cacheText}

\cacheNumber{1830}\needspace{5\baselineskip}\cacheName{\href{http://coord.info/GC5F5F9}{Tuilerie de la Rochebardière / The Tilery} — \href{http://coord.info/GC5F5F9\Number{}784447108}{1830}}\cacheData{{2018/07/15 dorisbear, Traditional Cache (1.5/1.5)}}\begin{cacheText}De retour de la Terra Aventura j’en profite pour faire quelques caches supplémentaires sur le retour . Celle-ci est trouvée assez rapidement bien intégrée.... Au moment où je signe j’ entends des klaxons et des cris de joie… Je suppose que la France vient de marquer un but ...youpi!!! Merci pour la cache.\end{cacheText}

\cacheNumber{1831}\needspace{5\baselineskip}\cacheName{\href{http://coord.info/GC5WYNF}{Lavoir de quotidies} — \href{http://coord.info/GC5WYNF\Number{}784442395}{1831}}\cacheData{{2018/07/15 mizaga, Traditional Cache (1.5/1.5)}}\begin{cacheText}C’est en allant faire la Terra Aventura à Saint-Vincent de Paul que je m’arrête pour découvrir ce joli petit lavoir laissé à l’abandon. Quel dommage ! Après avoir fait le tour je finis par mettre la main sur la cache : sous le toit.!!!Merci pour cette découverte.\end{cacheText}

\cacheNumber{1832}\needspace{5\baselineskip}\cacheName{\href{http://coord.info/GC5XV7M}{LK Tennis de Saint Paul les Dax} — \href{http://coord.info/GC5XV7M\Number{}784448709}{1832}}\cacheData{{2018/07/15 lauki3940, Traditional Cache (2.5/1.5)}}\begin{cacheText}Au retour de la cache Terra Aventura de Saint-Vincent de Paul j’en profite pour effacer ce petit point bleu. Et cette fois ci… Bingo je mets la main sur la Belle. Merci de Lauki  pour cette cache.\end{cacheText}

\cacheNumber{1833}\needspace{5\baselineskip}\cacheName{\href{http://coord.info/GC64YAB}{Un Mammouth pour Agnès} — \href{http://coord.info/GC64YAB\Number{}784445731}{1833}}\cacheData{{2018/07/15 dorisbear, Traditional Cache (2/1.5)}}\begin{cacheText}De retour de la Terra Aventura de Saint-Vincent de Paul je m’arrête pour faire quelques cacheS supplémentaireS. Le parking est désert en ce jour de finale du mondial. Quelle chance je peux déloger la belle tranquillement. Armée de mon aimant, je retire sans difficulté la cache. Encore du super travail réalisé par nos ours favoris.Un grand merci.\end{cacheText}

\cacheNumber{1834}\needspace{5\baselineskip}\cacheName{\href{http://coord.info/GC65GKH}{bienvenue 4} — \href{http://coord.info/GC65GKH\Number{}784443469}{1834}}\cacheData{{2018/07/15 lauki3940, Traditional Cache (2/1)}}\begin{cacheText}Sur le retour de la Terra Aventura de Saint Vincent de Paul je m’arrête faire cette cache en passant. Bonne cachette ...merci.\end{cacheText}

\cacheNumber{1835}\needspace{5\baselineskip}\cacheName{\href{http://coord.info/GC6MEWT}{Dax, ville thermal} — \href{http://coord.info/GC6MEWT\Number{}784600973}{1835}}\cacheData{{2018/07/15 mizaga, Earthcache (1.5/1)}}\begin{cacheText}En promenade sur Dax, j'en profite pour faire quelques caches. Merci pour cette earth qui m'en apprend un peu plus sur les eaux thermales et minérales.\end{cacheText}

\cacheNumber{1836}\needspace{5\baselineskip}\cacheName{\href{http://coord.info/GC6MHHG}{Fontaine d'eau chaude} — \href{http://coord.info/GC6MHHG\Number{}784610800}{1836}}\cacheData{{2018/07/15 mizaga, Earthcache (1.5/1)}}\begin{cacheText}Superbe fontaine très fréquentée en ce jour de finale!!!!Merci pour cette Earth\end{cacheText}

\cacheNumber{1837}\needspace{5\baselineskip}\cacheName{\href{http://coord.info/GC7QB3J}{Au début du chemin} — \href{http://coord.info/GC7QB3J\Number{}784450716}{1837}}\cacheData{{2018/07/15 belachao, Traditional Cache (2/1.5)}}\begin{cacheText}Sur le retour de la Terra Aventura de Saint-Vincent de Paul, j’en profite pour faire quelques caches. Celle-ci est trouvée assez rapidement. C’est une superbe réalisation ,parfaitement intégrée. Un PF pour ce travail. Merci pour la cache.\end{cacheText}

\cacheNumber{1838}\needspace{5\baselineskip}\cacheName{\href{http://coord.info/GC7QG9P}{jungle} — \href{http://coord.info/GC7QG9P\Number{}784453141}{1838}}\cacheData{{2018/07/15 belachao, Traditional Cache (4/1)}}\begin{cacheText}Je continue ma quête et j'arrive près des bambous. Ma hantise !!! Au bout d’un certain temps je finis par mettre la main dessus… Je déroule le log book ...j’en pleure il s’agit d'un leurre ! Mais très vite je finis par déloger la Belle sous les acclamations des supporters de l’équipe de France qui vient de marquer un but. Que du bonheur. ..Une très belle réalisation ...J’en redemande. Un PF et un grand merci.\end{cacheText}

\cacheNumber{1839}\needspace{5\baselineskip}\cacheName{\href{http://coord.info/GC7QGAJ}{la souche} — \href{http://coord.info/GC7QGAJ\Number{}784449798}{1839}}\cacheData{{2018/07/15 belachao, Traditional Cache (1.5/1)}}\begin{cacheText}C'est sur le retour de la Terra aventura de Saint-Vincent de Paul que j’en profite pour faire quelques caches supplémentaires. Celle-ci me tente depuis un moment… Arrivée sur le PZ , aucun doute sur le lieu de la cache… Bingo elle est bien là .Merci pour la cache.\end{cacheText}

\cacheNumber{1840}\needspace{5\baselineskip}\cacheName{\href{http://coord.info/GC4VTGR}{Terdax} — \href{http://coord.info/GC4VTGR\Number{}787773739}{1840}}\cacheData{{2018/07/22 kar@melos40, Traditional Cache (1.5/1.5)}}\begin{cacheText}En route pour la Terra Aventura, je décide de faire quelques caches au passage. Le GPS  n’est pas très précis et il me faut lire les commentaires pour arriver à la dénicher.L’indice prend tout son sens arrivé au PZ. Merci pour la découverte de la régie municipale des eaux de Dax.\end{cacheText}

\cacheNumber{1841}\needspace{5\baselineskip}\cacheName{\href{http://coord.info/GC5CVYG}{[BSD] \Number{}131} — \href{http://coord.info/GC5CVYG\Number{}787774359}{1841}}\cacheData{{2018/07/22 GéoLandesTour, Traditional Cache (1.5/1.5)}}\begin{cacheText}Sur Dax pour faire la Terra Aventura j’en profite pour compléter mes caches. Arrivée sur le PZ la photo ne correspond plus : l’arbre a été abattu mais la cache est toujours bien en place. Merci\end{cacheText}

\cacheNumber{1842}\needspace{5\baselineskip}\cacheName{\href{http://coord.info/GC7T0RJ}{Meet and Greet : les Bourguignons à la plage...} — \href{http://coord.info/GC7T0RJ\Number{}787925492}{1842}}\cacheData{{2018/07/25 othello21, Event Cache (1/1.5)}}\begin{cacheText}Un Évent super chouette 🦉 : rien d’étonnant car organisé par un bourguignon !!! Merci à tous pour cet excellent moment plein de bonne humeur et au plaisir de vous croiser sur les chemins .\end{cacheText}

\cacheNumber{1843}\needspace{5\baselineskip}\cacheName{\href{http://coord.info/GC7T89B}{Géoblabla-Café à Bayonne} — \href{http://coord.info/GC7T89B\Number{}789422694}{1843}}\cacheData{{2018/07/27 Gamboy, Event Cache (1/3)}}\begin{cacheText}C’est en tenue de festayre , étrennée la veille pour l’ouverture et préconisée par l’owner, que je rejoins le PZ. Après les\Quoted{géoblabla} habituels  autour de la copieuse collation apportée par les organisateurs ,la joyeuse troupe rejoint la place de la mairie pour réveiller le roi Léon. Puis ,après avoir déambulé dans les rues très animées de Bayonne avec quelques arrêts dans des lieux chargés de souvenirs, nous avons terminé au QG des géonails qui nous avaient concocté un festin pour l'occasion!!!L'heure de se séparer est arrivée bien trop vite.... Un immense merci pour cette formidable journée.\end{cacheText}

\cacheNumber{1844}\needspace{5\baselineskip}\cacheName{\href{http://coord.info/GC67VH8}{SAINT GOIN} — \href{http://coord.info/GC67VH8\Number{}790285456}{1844}}\cacheData{{2018/08/02 chantou 64, Traditional Cache (1/1.5)}}\begin{cacheText}C’est sur le retour des deux caches de Préchacq Navarrenx que je fais un petit détour pour déloger la Belle. Personne à l’horizon… Je peux chercher et loguer tranquillement. Merci Chantou 64 pour la cache.\end{cacheText}

\cacheNumber{1845}\needspace{5\baselineskip}\cacheName{\href{http://coord.info/GC7T9KA}{Enigme béarn n°1} — \href{http://coord.info/GC7T9KA\Number{}790288760}{1845}}\cacheData{{2018/08/02 notsag1, Traditional Cache (1.5/1.5)}}\begin{cacheText}C’est sur le retour des deux caches de Préchacq Navarrenx que je fais un petit détour pour déloger la Belle.Je ne croise personne pour arriver à la cache mais à l'approche du PZ j’aperçois deux pèlerins !!! Je patiente pour déloger la belle et signer le log book et Je relève l’indice.Merci pour la cache.\end{cacheText}

\cacheNumber{1846}\needspace{5\baselineskip}\cacheName{\href{http://coord.info/GC7VDQK}{Ancien Lavoir Prechacq-Navarrenx} — \href{http://coord.info/GC7VDQK\Number{}790278535}{1846}}\cacheData{{2018/08/02 Natdelamontagne09, Traditional Cache (2/2)}}\begin{cacheText}[{FTF}]

Ding Ding Ding l’alerte retentit ...   2 nouvelles caches viennent de paraître. Malheureusement il faut patienter car l’heure de la débauche est encore loin!!!Enfin il est temps de partir Toujours pas de FTF... je file vers le PZ.

Arrivée sur les lieux je découvre un endroit charmant au bord du gave qui cache un superbe lavoir en parfait état. Il ne me faut pas longtemps pour déloger la Belle grâce à l'indice.

Merci pour la cache et la découverte de ce site .\end{cacheText}

\cacheNumber{1847}\needspace{5\baselineskip}\cacheName{\href{http://coord.info/GC7VG7T}{Le puit de Prechacq-Navarrenx} — \href{http://coord.info/GC7VG7T\Number{}790283371}{1847}}\cacheData{{2018/08/02 Natdelamontagne09, Traditional Cache (2/1.5)}}\begin{cacheText}[{FTF}]

Ding Ding Ding l’alerte retentit ...   2 nouvelles caches viennent de paraître. Malheureusement il faut patienter car l’heure de la débauche est encore loin!!!Enfin il est temps de partir Toujours pas de FTF... je file vers le PZ.

Préchacq Navarrenx est un charmant village béarnais au centre duquel je découvre un magnifique puit donné à la commune. Et, surprise...Il abrite une bibliotheque bien garnie en tout genre .La Belle esi bien camoufflée mais je finis par mettre la main dessus.Merci Natdelamontagne 09 pour cette belle cache.\end{cacheText}

\cacheNumber{1848}\needspace{5\baselineskip}\cacheName{\href{http://coord.info/GC28474}{La Madeleine} — \href{http://coord.info/GC28474\Number{}792613154}{1848}}\cacheData{{2018/08/05 Peyo64, Traditional Cache (1.5/1.5)}}\begin{cacheText}En ces jours de canicule, je suis à la recherche de fraicheur….direction les caches d'altitude . Je n'étais jamais montée à la Madeleine et j'ai découvert un panorama à couper le souffle. La cache est bien dissimulée mais facile à trouver. Un grand merci pour cette belle promenade.\end{cacheText}

\cacheNumber{1849}\needspace{5\baselineskip}\cacheName{\href{http://coord.info/GC5J8DV}{Le mur des Champions} — \href{http://coord.info/GC5J8DV\Number{}792758294}{1849}}\cacheData{{2018/08/05 LeBearnais64, Traditional Cache (1/1.5)}}\begin{cacheText}C’est grâce aux différents commentaires que j’ai fini par mettre la main sur la belle. Merci pour la découverte de ces champions.\end{cacheText}

\cacheNumber{1850}\needspace{5\baselineskip}\cacheName{\href{http://coord.info/GC5J9JX}{Détruit par un tremblement de terre.} — \href{http://coord.info/GC5J9JX\Number{}792757488}{1850}}\cacheData{{2018/08/05 LeBearnais64, Traditional Cache (1.5/1.5)}}\begin{cacheText}Nous avons d'abord visité l'église et lu les coupures concernant le tremblement de terre puis nous sommes allées en quête de la Belle. Bien à l'abris des regards elle nous attend. Merci  LeBearnais64 .\end{cacheText}

\cacheNumber{1851}\needspace{5\baselineskip}\cacheName{\href{http://coord.info/GC5PXXD}{Le Mousquetaire : l'histoire rejoint la fiction !} — \href{http://coord.info/GC5PXXD\Number{}792754241}{1851}}\cacheData{{2018/08/05 LeBearnais64, Traditional Cache (1.5/1)}}\begin{cacheText}Arrivées au PZ, nous découvrons un superbe porche. Après avoir lu les explications concernant Aramits, nous nous mettons en quête de la cache. Les commentaires précédents parlent d'eau… Effectivement il faut bien tendre l’oreille pour l’entendre.Et l’on entend que ça une fois que la cache est découverte. Merci pour la cache et le bon moment passé.\end{cacheText}

\cacheNumber{1852}\needspace{5\baselineskip}\cacheName{\href{http://coord.info/GC719VV}{Petite pause sur la route d'Issarbe} — \href{http://coord.info/GC719VV\Number{}792750586}{1852}}\cacheData{{2018/08/05 yvain64, Traditional Cache (1/1.5)}}\begin{cacheText}Avec cette canicule nous recherchons la fraîcheur... voilà pourquoi nous nous retrouvons dans cette très belle région. Arrivées au PZ ,de charmants campeurs siègent au pied de la cache. Ils ont déjà repéré la boîte qui est apparente, sans savoir ce qu’est le géocaching. Après leur avoir expliqué et passé un bon moment nous reprenons la route. Merci pour la cache et cette belle rencontre.\end{cacheText}

\cacheNumber{1853}\needspace{5\baselineskip}\cacheName{\href{http://coord.info/GC5DZQ3}{Le lavoir de Lamontjoie} — \href{http://coord.info/GC5DZQ3\Number{}792765881}{1853}}\cacheData{{2018/08/07 janokinou, Traditional Cache (1.5/1.5)}}\begin{cacheText}Je découvre ici un superbe lavoir mais malheureusement il n’y a pas beaucoup d’eau. La cache est rapidement trouvée grâce à l’indice. MPLC\end{cacheText}

\cacheNumber{1854}\needspace{5\baselineskip}\cacheName{\href{http://coord.info/GC4H2FM}{Nérac : La fontaine de Fleurette} — \href{http://coord.info/GC4H2FM\Number{}792774033}{1854}}\cacheData{{2018/08/07 nmns26, Traditional Cache (2/2)}}\begin{cacheText}⚠️ Attention le PZ est envahi de moustiques…Peaux sensibles s’abstenir.... en photo les produits indispensables pour loguer la Belle. Nous n'avons pas eu de difficulté pour dénicher la Belle dans la grotte. L'enfer a été de remettre tout en place :les moustiques nous couvraient le corps!!!

Nous avons eu le temps d'admirer à l'aller la superbe fontaine de Fleurette.Merci\end{cacheText}

\cacheNumber{1855}\needspace{5\baselineskip}\cacheName{\href{http://coord.info/GC56MJ7}{alphabet lot et garonnais E comme....ESPIENS} — \href{http://coord.info/GC56MJ7\Number{}792775860}{1855}}\cacheData{{2018/08/07 titous titous, Traditional Cache (1/1.5)}}\begin{cacheText}C'est mon second passage sur cette cache et aujourd'hui elle me saute aux yeux. C'est une évidence!!!!Le village est superbe et très joliment fleuri. Le panorama est à couper le souffle, merci pour la découverte de ce magnifique village.\end{cacheText}

\cacheNumber{1856}\needspace{5\baselineskip}\cacheName{\href{http://coord.info/GC5J66Y}{L'église Saint Amand de Bruch} — \href{http://coord.info/GC5J66Y\Number{}792780780}{1856}}\cacheData{{2018/08/07 janokinou, Traditional Cache (2/1)}}\begin{cacheText}En vacances dans le 47 pour quelques jours, nous en profitons pour compléter la carte des points jaunes. Ici nous découvrons une superbe église qui témoigne du riche passé du village.La cache est rapidement localisée grâce à l’indice. Aucun moldu pour nous déranger. Merci pour la cache.\end{cacheText}

\cacheNumber{1857}\needspace{5\baselineskip}\cacheName{\href{http://coord.info/GC5J67F}{La fontaine Saint Amand et le calvaire de Bruch} — \href{http://coord.info/GC5J67F\Number{}792782344}{1857}}\cacheData{{2018/08/07 janokinou, Traditional Cache (1.5/1.5)}}\begin{cacheText}C’est en faisant la cache terraaventura que j’en profite pour chercher la Belle. Elle est bien à l’abri des regards. Merci pour la découverte de cette superbe croiX.\end{cacheText}

\cacheNumber{1858}\needspace{5\baselineskip}\cacheName{\href{http://coord.info/GC7CN11}{Les reliques de Saint-Louis} — \href{http://coord.info/GC7CN11\Number{}792763808}{1858}}\cacheData{{2018/08/07 Terraaventura, Multi-cache (1/1.5)}}\begin{cacheText}La cache a été trouvée grâce à l'application Terraaventura. J'adore le concept !!!!Superbe parcours qui nous fait visiter un  village chargé d’histoire. Nous avons fait de belles rencontres(Entre autre M Saint Mézard qui nous a longuement parlé des nombreux célèbres habitants du village et fait découvrir la fresque galloromaine cachée au fond à droite dans l'église) .Nous n'avons malheureusement pas eu la chance de voir les reliques.Merci pour cet excellent moment .\end{cacheText}

\cacheNumber{1859}\needspace{5\baselineskip}\cacheName{\href{http://coord.info/GCVNEB}{La cache du site d’Esclaux} — \href{http://coord.info/GCVNEB\Number{}792769852}{1859}}\cacheData{{2018/08/07 Serge Robert, Traditional Cache (1/1.5)}}\begin{cacheText}Pour quelques jours dans le 47,nous faisons un petit détour pour valider le Gers. Et nous ne sommes pas déçus de la découverte. Le site est superbe et l'église et le presbytère sont disproportionnés par rapport à l'endroit. La cache est vite trouvée grâce à l'indice mais le logbook est trempé. Malheureusement  je n'ai rien pour le changer. Merci pour la cache.\end{cacheText}

\cacheNumber{1860}\needspace{5\baselineskip}\cacheName{\href{http://coord.info/GC5H9W8}{Tuquo , Tuquouret} — \href{http://coord.info/GC5H9W8\Number{}792783418}{1860}}\cacheData{{2018/08/08 jomatoju 47, Traditional Cache (1.5/1.5)}}\begin{cacheText}C’est mon deuxième passage sur cette cache… Mais aujourd’hui elle ne me résiste pas et c’est en deux temps trois mouvements que je la déloge. Le panorama est superbe merci pour cette belle découverte.\end{cacheText}

\cacheNumber{1861}\needspace{5\baselineskip}\cacheName{\href{http://coord.info/GC5QHT4}{AMOU la 2} — \href{http://coord.info/GC5QHT4\Number{}793177233}{1861}}\cacheData{{2018/08/09 crispol40, Traditional Cache (1.5/1.5)}}\begin{cacheText}C'est aprés les caches CK du Luy que nous partons à la conquéte des caches d'Amou.Le gouter nous a revigoré et nous trouvons facilement la Belle grace à l'indice.Quelques sportifs passent devant le PZ mais nous loguons discretement.Merci pour la découverte de cette passerelle.\end{cacheText}

\cacheNumber{1862}\needspace{5\baselineskip}\cacheName{\href{http://coord.info/GC7TVKC}{EVENT LUY DE BEARN} — \href{http://coord.info/GC7TVKC\Number{}792966528}{1862}}\cacheData{{2018/08/09 crispol40, Event Cache (1/1.5)}}\begin{cacheText}C’était avec une grande impatience que j’attendais l’événement, et je n’ai pas été déçue ! Pour mon baptême en canoë mon fiston a tenu à m’accompagner ( sait on jamais !!!). Arrivés sur les lieux sans faire d’escale, nous retrouvons les joyeux drilles en train de ranger le pique nique..Présentation (même des estoniens ont fait le déplacement ), geobla-bla et il est temps d’aller à l’office du tourisme pour chercher les tickets de la descente. La souriante Céline a faillit en perdre la tête ...1 ticket ? non 2...monsieur et madame? Non madame est avec monsieur et monsieur est seul mais 2 tickets…. C'est dans la bonne humeur et une demi heure plus tard que nous rejoignons l'Event. Notre ami, Crispol40 ,nous attend avec sa ribambelle de boissons et de gâteaux ( excellentes les bouchées à la noix de coco!!!).Apres avoir signé la super pagaie qui fait office de logbook (j'adore!!!),nous récupérons les gilets de sauvetage et montons dans le bus pour rejoindre le départ du circuit. Luy ….nous voila!!!!Sauf qu'à l'arrivée c'est le déluge!!!Pas très courageux, nous attendons, à l'abris dans le bus,l'accalmie avant de nous jeter à l'eau!!!!La promenade est mémorable…échouage sur les galets, télescopage, chavirage dans\Quoted{les rapides}, perte momentanée de GPS dans les eaux profondes, chute dans l'eau lors d'un sauvetage,  sans oublier la recherche de nos chères caches. Que de fou rire!!!!Et même le soleil a fini par remplacé la pluie. A notre arrivée un grand gouter nous attend pour reprendre des forces avant de faire quelques caches terrestres. La journée va se conclure chez nos amis les Ours qui nous ont concocté, comme ils savent si bien le faire, un grand banquet .

Un grand merci à Stiper 40 qui a eu pitié des débutants et qui nous a permis d'apprécier la balade .

Un immense merci à Crispol40 pour l'organisation de cette journée exeptionnelle et aux Dorisbear pour cette super soirée.\end{cacheText}

\cacheNumber{1863}\needspace{5\baselineskip}\cacheName{\href{http://coord.info/GC7TVNK}{CK01} — \href{http://coord.info/GC7TVNK\Number{}793125122}{1863}}\cacheData{{2018/08/09 crispol40, Traditional Cache (2/4.5)}}\begin{cacheText}Co [{FTF}] avec la Team Évent Luy  Canoë 🛶 en compagnie de l’owner en personne.

Et c’est parti... les plus habiles s’empressent d’aller loguer la cache pendant que les débutants s’exercent à manier les rames !!!! Merci Paul pour ce super Event.\end{cacheText}

\cacheNumber{1864}\needspace{5\baselineskip}\cacheName{\href{http://coord.info/GC7TVRJ}{CK02} — \href{http://coord.info/GC7TVRJ\Number{}793158858}{1864}}\cacheData{{2018/08/09 crispol40, Traditional Cache (1.5/4.5)}}\begin{cacheText}Co [{FTF}] avec la Team Évent Luy Canoë 🛶 en compagnie de l’owner en personne.

Les caches se succèdent le long du Luy. Toutes sont trouvées dans la bonne humeur et la franche rigolade. Merci Paul pour ce super Event.\end{cacheText}

\cacheNumber{1865}\needspace{5\baselineskip}\cacheName{\href{http://coord.info/GC7TVZR}{CK03} — \href{http://coord.info/GC7TVZR\Number{}793159471}{1865}}\cacheData{{2018/08/09 crispol40, Traditional Cache (4.5/5)}}\begin{cacheText}Co [{FTF}] avec la Team Évent Luy Canoë 🛶 en compagnie de l’owner en personne.

Les caches se succèdent le long du Luy. Toutes sont trouvées dans la bonne humeur et la franche rigolade. Merci Paul pour ce super Event.\end{cacheText}

\cacheNumber{1866}\needspace{5\baselineskip}\cacheName{\href{http://coord.info/GC7TW0C}{CK04} — \href{http://coord.info/GC7TW0C\Number{}793159919}{1866}}\cacheData{{2018/08/09 crispol40, Traditional Cache (3.5/4.5)}}\begin{cacheText}Co [{FTF}] avec la Team Évent Luy Canoë 🛶 en compagnie de l’owner en personne.

Les caches se succèdent le long du Luy. Toutes sont trouvées dans la bonne humeur et la franche rigolade. Merci Paul pour ce super Event.\end{cacheText}

\cacheNumber{1867}\needspace{5\baselineskip}\cacheName{\href{http://coord.info/GC7TW0Q}{CK05} — \href{http://coord.info/GC7TW0Q\Number{}793160455}{1867}}\cacheData{{2018/08/09 crispol40, Traditional Cache (2/5)}}\begin{cacheText}Co [{FTF}] avec la Team Évent Luy Canoë 🛶 en compagnie de l’owner en personne.

Les caches se succèdent le long du Luy. Toutes sont trouvées dans la bonne humeur et la franche rigolade. Merci Paul pour ce super Event.\end{cacheText}

\cacheNumber{1868}\needspace{5\baselineskip}\cacheName{\href{http://coord.info/GC7TW19}{CK06} — \href{http://coord.info/GC7TW19\Number{}793160961}{1868}}\cacheData{{2018/08/09 crispol40, Traditional Cache (4/5)}}\begin{cacheText}Co [{FTF}] avec la Team Évent Luy Canoë 🛶 en compagnie de l’owner en personne.

Les caches se succèdent le long du Luy. Toutes sont trouvées dans la bonne humeur et la franche rigolade. Merci Paul pour ce super Event.\end{cacheText}

\cacheNumber{1869}\needspace{5\baselineskip}\cacheName{\href{http://coord.info/GC7TW1K}{CK07} — \href{http://coord.info/GC7TW1K\Number{}793161213}{1869}}\cacheData{{2018/08/09 crispol40, Traditional Cache (3.5/5)}}\begin{cacheText}Co [{FTF}] avec la Team Évent Luy Canoë 🛶 en compagnie de l’owner en personne.

Les caches se succèdent le long du Luy. Toutes sont trouvées dans la bonne humeur et la franche rigolade. Merci Paul pour ce super Event.\end{cacheText}

\cacheNumber{1870}\needspace{5\baselineskip}\cacheName{\href{http://coord.info/GC7TW2B}{CK08} — \href{http://coord.info/GC7TW2B\Number{}793161710}{1870}}\cacheData{{2018/08/09 crispol40, Traditional Cache (2.5/5)}}\begin{cacheText}Co [{FTF}] avec la Team Évent Luy Canoë 🛶 en compagnie de l’owner en personne.

Les caches se succèdent le long du Luy. Toutes sont trouvées dans la bonne humeur et la franche rigolade. Merci Paul pour ce super Event.\end{cacheText}

\cacheNumber{1871}\needspace{5\baselineskip}\cacheName{\href{http://coord.info/GC5QHVQ}{AMOU la 5} — \href{http://coord.info/GC5QHVQ\Number{}793401054}{1871}}\cacheData{{2018/08/10 crispol40, Unknown Cache (1.5/1.5)}}\begin{cacheText}L’heure tourne , il faut être raisonnable pour ne pas arriver en retard au banquet des DorisBear. Cela sera la dernière de la journée !!! Nanard40270, Peyo64 et moi même partons à la recherche de la Belle. C’est Peyo qui découvre la boîte avec un joli TB à l’intérieur. Merci Paul pour toutes ces caches sympathiques.\end{cacheText}

\cacheNumber{1872}\needspace{5\baselineskip}\cacheName{\href{http://coord.info/GC5QJYC}{AMOU la 4} — \href{http://coord.info/GC5QJYC\Number{}793397053}{1872}}\cacheData{{2018/08/10 crispol40, Traditional Cache (3.5/1.5)}}\begin{cacheText}C’est en compagnie de Nanard40270 que je pars déloger la Belle près de l’écluse. L’indice m’a été très précieux. Merci pour la cache.\end{cacheText}

\cacheNumber{1873}\needspace{5\baselineskip}\cacheName{\href{http://coord.info/GC5QHV3}{AMOU la 3} — \href{http://coord.info/GC5QHV3\Number{}793395633}{1873}}\cacheData{{2018/08/10 crispol40, Multi-cache (1.5/1.5)}}\begin{cacheText}Je continue ma quête et l’attelle à cette multi. Elle est assez facile à résoudre. Par chance un troupeau est passé avant l’évent et a nettoyé le terrain. Oufff!!! Au dessus de ma tête une jolie mue de serpent!!! Pas vraiment rassurée je logue et je repars  en vitesse. Merci Paul\end{cacheText}

\cacheNumber{1874}\needspace{5\baselineskip}\cacheName{\href{http://coord.info/GC7D0A1}{L'OEUF VOYAGEUR DE ROSALIE} — \href{http://coord.info/GC7D0A1\Number{}793406135}{1874}}\cacheData{{2018/08/10 crispol40, Letterbox Hybrid (1/1)}}\begin{cacheText}Aujourd’hui c’est le grand jour: initiation aux T5 arboricoles (après les T5 nautiques de la veille)pour ceux qui le souhaitent !!! Le rendez-vous est chez notre ami Crispol40 , le grand Maître de l’escalade.Rosalie ayant fait son nid pas très loin , j’en profite pour aller jeter un œil avant l’arrivée de tous les participants et c’est sous la houlette du maître que je déloge la Belle tourterelle. Merci Paul pour ce beau poème que tu acceptes de partager. Un PF évidemment.\end{cacheText}

\cacheNumber{1875}\needspace{5\baselineskip}\cacheName{\href{http://coord.info/GC7EV8C}{Les Berges de l'OURSEAU\Number{}01} — \href{http://coord.info/GC7EV8C\Number{}793489114}{1875}}\cacheData{{2018/08/10 crispol40, Unknown Cache (1/5)}}\begin{cacheText}Après les T5 aquatiques, la joyeuse troupe se retrouve pour s’initier aux T5 arboricoles. Mais le moment venu, il y a peu de volontaire!!! Seul le courageux Puma qui grogne se propose pour aller voir les visiteurs verts. Le dispositif de sécurité est à son comble: Crispol40, grand maître de l’escalade, installe l’équipement qu’il vérifie un à un,Bernard et Dédé s’affairent pour tendre la corde de secours et DEBDX mouline et range au fur et à mesure le précieux matériel. Et voilà le Puma qui grogne qui monte avec une telle facilité... nous sommes tous admiratifs. Tous les indices en poche , nous finissons par un très bel arbre . Merci Crispol40 pour cette super journée\end{cacheText}

\cacheNumber{1876}\needspace{5\baselineskip}\cacheName{\href{http://coord.info/GC7EVD6}{Les Berges de l'OURSEAU\Number{}02} — \href{http://coord.info/GC7EVD6\Number{}793490018}{1876}}\cacheData{{2018/08/10 crispol40, Unknown Cache (1.5/5)}}\begin{cacheText}Après les T5 aquatiques, la joyeuse troupe se retrouve pour s’initier aux T5 arboricoles. Mais le moment venu, il y a peu de volontaire!!! Seul le courageux Puma qui grogne se propose pour aller voir les visiteurs verts. Le dispositif de sécurité est à son comble: Crispol40, grand maître de l’escalade, installe l’équipement qu’il vérifie un à un,Bernard et Dédé s’affairent pour tendre la corde de secours et DEBDX mouline et range au fur et à mesure le précieux matériel. Et voilà le Puma qui grogne qui monte avec une telle facilité... nous sommes tous admiratifs. Tous les indices en poche , nous finissons par un très bel arbre . Merci Crispol40 pour cette super journée\end{cacheText}

\cacheNumber{1877}\needspace{5\baselineskip}\cacheName{\href{http://coord.info/GC7EVE4}{Les Berges de l'OURSEAU\Number{}03} — \href{http://coord.info/GC7EVE4\Number{}793490657}{1877}}\cacheData{{2018/08/10 crispol40, Unknown Cache (3/5)}}\begin{cacheText}Après les T5 aquatiques, la joyeuse troupe se retrouve pour s’initier aux T5 arboricoles. Mais le moment venu, il y a peu de volontaire!!! Seul le courageux Puma qui grogne se propose pour aller voir les visiteurs verts. Le dispositif de sécurité est à son comble: Crispol40, grand maître de l’escalade, installe l’équipement qu’il vérifie un à un,Bernard et Dédé s’affairent pour tendre la corde de secours et DEBDX mouline et range au fur et à mesure le précieux matériel. Et voilà le Puma qui grogne qui monte avec une telle facilité... nous sommes tous admiratifs. Tous les indices en poche , nous finissons par un très bel arbre . Merci Crispol40 pour cette super journée\end{cacheText}

\cacheNumber{1878}\needspace{5\baselineskip}\cacheName{\href{http://coord.info/GC7EVEK}{Les Berges de l'OURSEAU\Number{}04} — \href{http://coord.info/GC7EVEK\Number{}793491305}{1878}}\cacheData{{2018/08/10 crispol40, Unknown Cache (3.5/5)}}\begin{cacheText}Après les T5 aquatiques, la joyeuse troupe se retrouve pour s’initier aux T5 arboricoles. Mais le moment venu, il y a peu de volontaire!!! Seul le courageux Puma qui grogne se propose pour aller voir les visiteurs verts. Le dispositif de sécurité est à son comble: Crispol40, grand maître de l’escalade, installe l’équipement qu’il vérifie un à un,Bernard et Dédé s’affairent pour tendre la corde de secours et DEBDX mouline et range au fur et à mesure le précieux matériel. Et voilà le Puma qui grogne qui monte avec une telle facilité... nous sommes tous admiratifs. Tous les indices en poche , nous finissons par un très bel arbre . Merci Crispol40 pour cette super journée\end{cacheText}

\cacheNumber{1879}\needspace{5\baselineskip}\cacheName{\href{http://coord.info/GC7EVF3}{Les Berges de l'OURSEAU\Number{}05} — \href{http://coord.info/GC7EVF3\Number{}793491655}{1879}}\cacheData{{2018/08/10 crispol40, Unknown Cache (4/5)}}\begin{cacheText}Après les T5 aquatiques, la joyeuse troupe se retrouve pour s’initier aux T5 arboricoles. Mais le moment venu, il y a peu de volontaire!!! Seul le courageux Puma qui grogne se propose pour aller voir les visiteurs verts. Le dispositif de sécurité est à son comble: Crispol40, grand maître de l’escalade, installe l’équipement qu’il vérifie un à un,Bernard et Dédé s’affairent pour tendre la corde de secours et DEBDX mouline et range au fur et à mesure le précieux matériel. Et voilà le Puma qui grogne qui monte avec une telle facilité... nous sommes tous admiratifs. Tous les indices en poche , nous finissons par un très bel arbre . Merci Crispol40 pour cette super journée 

Un PF pour toute la série.\end{cacheText}

\cacheNumber{1880}\needspace{5\baselineskip}\cacheName{\href{http://coord.info/GC2FM59}{L'Amphithéatre Romain} — \href{http://coord.info/GC2FM59\Number{}793492957}{1880}}\cacheData{{2018/08/13 lulu\Underscore{}et\Underscore{}compagnie, Traditional Cache (1.5/1.5)}}\begin{cacheText}En route pour passer une semaine de vacances en Loire-Atlantique et notamment pour participer à l’Atlantic Event nous faisons une escale pour dormir à Saintes. Après le repas ,nous décidons de faire quelques caches et nous choisissons l’amphithéâtre , un très beau lieu chargé d’histoire. La nuit tombe et nous avons un peu de mal à trouver la belle mais grâce à l’indice c’est chose faite. Merci beaucoup pour cette belle découverte.\end{cacheText}

\cacheNumber{1881}\needspace{5\baselineskip}\cacheName{\href{http://coord.info/GC637BF}{église saint Eutrope} — \href{http://coord.info/GC637BF\Number{}793493533}{1881}}\cacheData{{2018/08/13 cyrillesteph17, Traditional Cache (1/1)}}\begin{cacheText}Nous sommes sur Saintes pour passer la nuit. Après le repas ,nous décidons de faire quelques caches et nous découvrons cette sublime église. Après avoir lu les différents commentaires nous finissons par mettre la main sur le PZ. Malheureusement le logbook  s’est enfoncé et on ne peut plus l’attraper. J’assure la maintenance avec une nouvelle boîte. Merci pour la découverte de ce haut-lieu religieux.\end{cacheText}

\cacheNumber{1882}\needspace{5\baselineskip}\cacheName{\href{http://coord.info/GC6386V}{monument aux morts} — \href{http://coord.info/GC6386V\Number{}793494595}{1882}}\cacheData{{2018/08/13 cyrillesteph17, Traditional Cache (1.5/1.5)}}\begin{cacheText}Nous décidons de faire quelques caches après le repas car nous passons la nuit sur Saintes.Celle-ci nous donne un peu de mal :chercher dans une haie la nuit ce n’est pas évident mais finalement avec un peu de chance nous mettons la main dessus. Merci pour la cache.\end{cacheText}

\cacheNumber{1883}\needspace{5\baselineskip}\cacheName{\href{http://coord.info/GC63D8K}{caserne de pompiers de Saintes} — \href{http://coord.info/GC63D8K\Number{}793494037}{1883}}\cacheData{{2018/08/13 cyrillesteph17, Traditional Cache (1.5/1.5)}}\begin{cacheText}Ce soir nous dormons à Saintes et après le repas une petite balade digestive s’impose. Nous découvrons la ville de nuit qui est très calme .Nous sommes tranquilles pour loguer la belle qui est rapidement découverte. Merci pour la cache.\end{cacheText}

\cacheNumber{1884}\needspace{5\baselineskip}\cacheName{\href{http://coord.info/GC4BD19}{Colimaçon} — \href{http://coord.info/GC4BD19\Number{}794051796}{1884}}\cacheData{{2018/08/14 crackel, Traditional Cache (2/2)}}\begin{cacheText}Accueillis par les Fabilab pour participer à l’Atlantic Évent en fin de semaine , nous décidons de filer vers le Croisic pour faire un des nombreux circuit du PTTTN. Et nous n’avons pas été déçus : de belles caches, de somptueux paysages et des rencontres fort sympathiques.

Merci crackel pour cette cache en hauteur!!!\end{cacheText}

\cacheNumber{1885}\needspace{5\baselineskip}\cacheName{\href{http://coord.info/GC4BD4P}{Le cimetière des coquillages} — \href{http://coord.info/GC4BD4P\Number{}794052578}{1885}}\cacheData{{2018/08/14 crackel, Traditional Cache (2.5/2.5)}}\begin{cacheText}Accueillis par les Fabilab pour participer à l’Atlantic Évent en fin de semaine , nous décidons de filer vers le Croisic pour faire un des nombreux circuit du PTTTN. Et nous n’avons pas été déçus : de belles caches, de somptueux paysages et des rencontres fort sympathiques.

Nous remplaçons la boite abimée.Merci pour la cache.\end{cacheText}

\cacheNumber{1886}\needspace{5\baselineskip}\cacheName{\href{http://coord.info/GC5GAFB}{TDBB \Number{}05 - Chemin de la Meunière} — \href{http://coord.info/GC5GAFB\Number{}794061130}{1886}}\cacheData{{2018/08/14 lesdecouvreurs, Traditional Cache (2/3)}}\begin{cacheText}En vacances chez les Fabilab nous avons fait plusieurs caches sur le Croisic. C'est au retour que nous loguons cette Belle .Nous tenions à en valider une sur Batz sur mer car la commune est jumelée avec Salies de Béarn notre village d'origine!!!!Trop sympa!!! Merci depenser à ceux qui ne sont pas experts en escalade.\end{cacheText}

\cacheNumber{1887}\needspace{5\baselineskip}\cacheName{\href{http://coord.info/GC61EAC}{Fort de la Barrière / Pen Castel} — \href{http://coord.info/GC61EAC\Number{}794057138}{1887}}\cacheData{{2018/08/14 piotrgeo, Traditional Cache (1.5/2.5)}}\begin{cacheText}Accueillis par les Fabilab pour participer à l’Atlantic Évent en fin de semaine , nous décidons de filer vers le Croisic pour faire un des nombreux circuit du PTTTN. Et nous n’avons pas été déçus : de belles caches, de somptueux paysages et des rencontres fort sympathiques.

Merci pour la cache.\end{cacheText}

\cacheNumber{1888}\needspace{5\baselineskip}\cacheName{\href{http://coord.info/GC711DC}{PTTTN - A1} — \href{http://coord.info/GC711DC\Number{}794056317}{1888}}\cacheData{{2018/08/14 ptitloup, Traditional Cache (1.5/1.5)}}\begin{cacheText}Log général 

Accueillis par les Fabilab pour participer à l’Atlantic Évent en fin de semaine , nous décidons de filer vers le Croisic pour faire un des nombreux circuit du PTTTN. Et nous n’avons pas été déçus : de belles caches, de somptueux paysages et des rencontres fort sympathiques.

Merci ptitloup\end{cacheText}

\cacheNumber{1889}\needspace{5\baselineskip}\cacheName{\href{http://coord.info/GC711DF}{PTTTN - A2} — \href{http://coord.info/GC711DF\Number{}794056164}{1889}}\cacheData{{2018/08/14 ptitloup, Traditional Cache (1.5/2)}}\begin{cacheText}Log général 

Accueillis par les Fabilab pour participer à l’Atlantic Évent en fin de semaine , nous décidons de filer vers le Croisic pour faire un des nombreux circuit du PTTTN. Et nous n’avons pas été déçus : de belles caches, de somptueux paysages et des rencontres fort sympathiques.

Merci ptitloup\end{cacheText}

\cacheNumber{1890}\needspace{5\baselineskip}\cacheName{\href{http://coord.info/GC711DR}{PTTTN - A4} — \href{http://coord.info/GC711DR\Number{}794054999}{1890}}\cacheData{{2018/08/14 ptitloup, Traditional Cache (1.5/1.5)}}\begin{cacheText}Log général 

Accueillis par les Fabilab pour participer à l’Atlantic Évent en fin de semaine , nous décidons de filer vers le Croisic pour faire un des nombreux circuit du PTTTN. Et nous n’avons pas été déçus : de belles caches, de somptueux paysages et des rencontres fort sympathiques.

Merci ptitloup\end{cacheText}

\cacheNumber{1891}\needspace{5\baselineskip}\cacheName{\href{http://coord.info/GC711E7}{PTTTN - A5} — \href{http://coord.info/GC711E7\Number{}794054827}{1891}}\cacheData{{2018/08/14 ptitloup, Traditional Cache (1.5/2)}}\begin{cacheText}Log général 

Accueillis par les Fabilab pour participer à l’Atlantic Évent en fin de semaine , nous décidons de filer vers le Croisic pour faire un des nombreux circuit du PTTTN. Et nous n’avons pas été déçus : de belles caches, de somptueux paysages et des rencontres fort sympathiques.

Merci ptitloup\end{cacheText}

\cacheNumber{1892}\needspace{5\baselineskip}\cacheName{\href{http://coord.info/GC711E8}{PTTTN - A6} — \href{http://coord.info/GC711E8\Number{}794054409}{1892}}\cacheData{{2018/08/14 ptitloup, Traditional Cache (1.5/2)}}\begin{cacheText}Log général 

Accueillis par les Fabilab pour participer à l’Atlantic Évent en fin de semaine , nous décidons de filer vers le Croisic pour faire un des nombreux circuit du PTTTN. Et nous n’avons pas été déçus : de belles caches, de somptueux paysages et des rencontres fort sympathiques.

Merci ptitloup\end{cacheText}

\cacheNumber{1893}\needspace{5\baselineskip}\cacheName{\href{http://coord.info/GC711E9}{PTTTN - A7} — \href{http://coord.info/GC711E9\Number{}794054253}{1893}}\cacheData{{2018/08/14 ptitloup, Traditional Cache (1.5/1.5)}}\begin{cacheText}Log général 

Accueillis par les Fabilab pour participer à l’Atlantic Évent en fin de semaine , nous décidons de filer vers le Croisic pour faire un des nombreux circuit du PTTTN. Et nous n’avons pas été déçus : de belles caches, de somptueux paysages et des rencontres fort sympathiques.

Merci ptitloup\end{cacheText}

\cacheNumber{1894}\needspace{5\baselineskip}\cacheName{\href{http://coord.info/GC711EA}{PTTTN - A8} — \href{http://coord.info/GC711EA\Number{}794052706}{1894}}\cacheData{{2018/08/14 ptitloup, Traditional Cache (1.5/1.5)}}\begin{cacheText}Log général 

Accueillis par les Fabilab pour participer à l’Atlantic Évent en fin de semaine , nous décidons de filer vers le Croisic pour faire un des nombreux circuit du PTTTN. Et nous n’avons pas été déçus : de belles caches, de somptueux paysages et des rencontres fort sympathiques.

Merci ptitloup\end{cacheText}

\cacheNumber{1895}\needspace{5\baselineskip}\cacheName{\href{http://coord.info/GC711ED}{PTTTN - A9} — \href{http://coord.info/GC711ED\Number{}794051951}{1895}}\cacheData{{2018/08/14 ptitloup, Traditional Cache (1.5/1.5)}}\begin{cacheText}Log général 

Accueillis par les Fabilab pour participer à l’Atlantic Évent en fin de semaine , nous décidons de filer vers le Croisic pour faire un des nombreux circuit du PTTTN. Et nous n’avons pas été déçus : de belles caches, de somptueux paysages et des rencontres fort sympathiques.

Merci ptitloup\end{cacheText}

\cacheNumber{1896}\needspace{5\baselineskip}\cacheName{\href{http://coord.info/GC711EG}{PTTTN - A10} — \href{http://coord.info/GC711EG\Number{}794051428}{1896}}\cacheData{{2018/08/14 ptitloup, Traditional Cache (1.5/1.5)}}\begin{cacheText}Log général 

Accueillis par les Fabilab pour participer à l’Atlantic Évent en fin de semaine , nous décidons de filer vers le Croisic pour faire un des nombreux circuit du PTTTN. Et nous n’avons pas été déçus : de belles caches, de somptueux paysages et des rencontres fort sympathiques.

Merci ptitloup\end{cacheText}

\cacheNumber{1897}\needspace{5\baselineskip}\cacheName{\href{http://coord.info/GC711EQ}{PTTTN - A14} — \href{http://coord.info/GC711EQ\Number{}794048660}{1897}}\cacheData{{2018/08/14 ptitloup, Traditional Cache (1.5/1.5)}}\begin{cacheText}Log général 

Accueillis par les Fabilab pour participer à l’Atlantic Évent en fin de semaine , nous décidons de filer vers le Croisic pour faire un des nombreux circuit du PTTTN. Et nous n’avons pas été déçus : de belles caches, de somptueux paysages et des rencontres fort sympathiques.

Merci ptitloup\end{cacheText}

\cacheNumber{1898}\needspace{5\baselineskip}\cacheName{\href{http://coord.info/GC711ET}{PTTTN - A15} — \href{http://coord.info/GC711ET\Number{}794048346}{1898}}\cacheData{{2018/08/14 ptitloup, Traditional Cache (1.5/2)}}\begin{cacheText}Log général 

Accueillis par les Fabilab pour participer à l’Atlantic Évent en fin de semaine , nous décidons de filer vers le Croisic pour faire un des nombreux circuit du PTTTN. Et nous n’avons pas été déçus : de belles caches, de somptueux paysages et des rencontres fort sympathiques.

Merci ptitloup\end{cacheText}

\cacheNumber{1899}\needspace{5\baselineskip}\cacheName{\href{http://coord.info/GC711F1}{PTTTN - A17} — \href{http://coord.info/GC711F1\Number{}794047365}{1899}}\cacheData{{2018/08/14 ptitloup, Traditional Cache (2.5/1.5)}}\begin{cacheText}Log général 

Accueillis par les Fabilab pour participer à l’Atlantic Évent en fin de semaine , nous décidons de filer vers le Croisic pour faire un des nombreux circuit du PTTTN. Et nous n’avons pas été déçus : de belles caches, de somptueux paysages et des rencontres fort sympathiques.

Merci ptitloup et un PF pour cette super boite.Bravo\end{cacheText}

\cacheNumber{1900}\needspace{5\baselineskip}\cacheName{\href{http://coord.info/GC711F3}{PTTTN - A16} — \href{http://coord.info/GC711F3\Number{}794047997}{1900}}\cacheData{{2018/08/14 ptitloup, Traditional Cache (1.5/1.5)}}\begin{cacheText}Log général 

Accueillis par les Fabilab pour participer à l’Atlantic Évent en fin de semaine , nous décidons de filer vers le Croisic pour faire un des nombreux circuit du PTTTN. Et nous n’avons pas été déçus : de belles caches, de somptueux paysages et des rencontres fort sympathiques.

Merci ptitloup\end{cacheText}

\cacheNumber{1901}\needspace{5\baselineskip}\cacheName{\href{http://coord.info/GC711F6}{PTTTN - A19} — \href{http://coord.info/GC711F6\Number{}794046832}{1901}}\cacheData{{2018/08/14 ptitloup, Traditional Cache (1.5/1.5)}}\begin{cacheText}Log général 

Accueillis par les Fabilab pour participer à l’Atlantic Évent en fin de semaine , nous décidons de filer vers le Croisic pour faire un des nombreux circuit du PTTTN. Et nous n’avons pas été déçus : de belles caches, de somptueux paysages et des rencontres fort sympathiques.

Merci ptitloup\end{cacheText}

\cacheNumber{1902}\needspace{5\baselineskip}\cacheName{\href{http://coord.info/GC711FG}{PTTTN - A20} — \href{http://coord.info/GC711FG\Number{}794046697}{1902}}\cacheData{{2018/08/14 ptitloup, Traditional Cache (1.5/1.5)}}\begin{cacheText}Log général 

Accueillis par les Fabilab pour participer à l’Atlantic Évent en fin de semaine , nous décidons de filer vers le Croisic pour faire un des nombreux circuit du PTTTN. Et nous n’avons pas été déçus : de belles caches, de somptueux paysages et des rencontres fort sympathiques.

Merci ptitloup\end{cacheText}

\cacheNumber{1903}\needspace{5\baselineskip}\cacheName{\href{http://coord.info/GC711FN}{PTTTN - A22} — \href{http://coord.info/GC711FN\Number{}794046262}{1903}}\cacheData{{2018/08/14 ptitloup, Traditional Cache (1.5/1.5)}}\begin{cacheText}Log général 

Accueillis par les Fabilab pour participer à l’Atlantic Évent en fin de semaine , nous décidons de filer vers le Croisic pour faire un des nombreux circuit du PTTTN. Et nous n’avons pas été déçus : de belles caches, de somptueux paysages et des rencontres fort sympathiques.

Merci ptitloup\end{cacheText}

\cacheNumber{1904}\needspace{5\baselineskip}\cacheName{\href{http://coord.info/GC711FV}{PTTTN - A23} — \href{http://coord.info/GC711FV\Number{}794046056}{1904}}\cacheData{{2018/08/14 ptitloup, Traditional Cache (1.5/1.5)}}\begin{cacheText}Log général 

Accueillis par les Fabilab pour participer à l’Atlantic Évent en fin de semaine , nous décidons de filer vers le Croisic pour faire un des nombreux circuit du PTTTN. Et nous n’avons pas été déçus : de belles caches, de somptueux paysages et des rencontres fort sympathiques.

Merci ptitloup\end{cacheText}

\cacheNumber{1905}\needspace{5\baselineskip}\cacheName{\href{http://coord.info/GC711FW}{PTTTN - A24} — \href{http://coord.info/GC711FW\Number{}794045900}{1905}}\cacheData{{2018/08/14 ptitloup, Traditional Cache (1.5/1.5)}}\begin{cacheText}Log général 

Accueillis par les Fabilab pour participer à l’Atlantic Évent en fin de semaine , nous décidons de filer vers le Croisic pour faire un des nombreux circuit du PTTTN. Et nous n’avons pas été déçus : de belles caches, de somptueux paysages et des rencontres fort sympathiques.

Merci ptitloup\end{cacheText}

\cacheNumber{1906}\needspace{5\baselineskip}\cacheName{\href{http://coord.info/GC711FZ}{PTTTN - A25} — \href{http://coord.info/GC711FZ\Number{}794045742}{1906}}\cacheData{{2018/08/14 ptitloup, Traditional Cache (1.5/1.5)}}\begin{cacheText}Log général 

Accueillis par les Fabilab pour participer à l’Atlantic Évent en fin de semaine , nous décidons de filer vers le Croisic pour faire un des nombreux circuit du PTTTN. Et nous n’avons pas été déçus : de belles caches, de somptueux paysages et des rencontres fort sympathiques.

Merci ptitloup\end{cacheText}

\cacheNumber{1907}\needspace{5\baselineskip}\cacheName{\href{http://coord.info/GC711G1}{PTTTN - A26} — \href{http://coord.info/GC711G1\Number{}794045533}{1907}}\cacheData{{2018/08/14 ptitloup, Traditional Cache (2/1.5)}}\begin{cacheText}Log général 

Accueillis par les Fabilab pour participer à l’Atlantic Évent en fin de semaine , nous décidons de filer vers le Croisic pour faire un des nombreux circuit du PTTTN. Et nous n’avons pas été déçus : de belles caches, de somptueux paysages et des rencontres fort sympathiques.

Merci ptitloup\end{cacheText}

\cacheNumber{1908}\needspace{5\baselineskip}\cacheName{\href{http://coord.info/GC711G4}{PTTTN - A27} — \href{http://coord.info/GC711G4\Number{}794045024}{1908}}\cacheData{{2018/08/14 ptitloup, Traditional Cache (1.5/1.5)}}\begin{cacheText}Log général 

Accueillis par les Fabilab pour participer à l’Atlantic Évent en fin de semaine , nous décidons de filer vers le Croisic pour faire un des nombreux circuit du PTTTN. Et nous n’avons pas été déçus : de belles caches, de somptueux paysages et des rencontres fort sympathiques.

Merci ptitloup\end{cacheText}

\cacheNumber{1909}\needspace{5\baselineskip}\cacheName{\href{http://coord.info/GC711G9}{PTTTN - A28} — \href{http://coord.info/GC711G9\Number{}794044790}{1909}}\cacheData{{2018/08/14 ptitloup, Traditional Cache (1.5/1.5)}}\begin{cacheText}Log général 

Accueillis par les Fabilab pour participer à l’Atlantic Évent en fin de semaine , nous décidons de filer vers le Croisic pour faire un des nombreux circuit du PTTTN. Et nous n’avons pas été déçus : de belles caches, de somptueux paysages et des rencontres fort sympathiques.

Merci ptitloup\end{cacheText}

\cacheNumber{1910}\needspace{5\baselineskip}\cacheName{\href{http://coord.info/GC711GB}{PTTTN - A29} — \href{http://coord.info/GC711GB\Number{}794044643}{1910}}\cacheData{{2018/08/14 ptitloup, Traditional Cache (1.5/1.5)}}\begin{cacheText}Log général 

Accueillis par les Fabilab pour participer à l’Atlantic Évent en fin de semaine , nous décidons de filer vers le Croisic pour faire un des nombreux circuit du PTTTN. Et nous n’avons pas été déçus : de belles caches, de somptueux paysages et des rencontres fort sympathiques.

Merci ptitloup\end{cacheText}

\cacheNumber{1911}\needspace{5\baselineskip}\cacheName{\href{http://coord.info/GC711GF}{PTTTN - A30} — \href{http://coord.info/GC711GF\Number{}794044503}{1911}}\cacheData{{2018/08/14 ptitloup, Traditional Cache (1.5/2)}}\begin{cacheText}Log général 

Accueillis par les Fabilab pour participer à l’Atlantic Évent en fin de semaine , nous décidons de filer vers le Croisic pour faire un des nombreux circuit du PTTTN. Et nous n’avons pas été déçus : de belles caches, de somptueux paysages et des rencontres fort sympathiques.

Merci ptitloup\end{cacheText}

\cacheNumber{1912}\needspace{5\baselineskip}\cacheName{\href{http://coord.info/GC711GK}{PTTTN - A31} — \href{http://coord.info/GC711GK\Number{}794044363}{1912}}\cacheData{{2018/08/14 ptitloup, Traditional Cache (1.5/1.5)}}\begin{cacheText}Log général 

Accueillis par les Fabilab pour participer à l’Atlantic Évent en fin de semaine , nous décidons de filer vers le Croisic pour faire un des nombreux circuit du PTTTN. Et nous n’avons pas été déçus : de belles caches, de somptueux paysages et des rencontres fort sympathiques.

Merci ptitloup\end{cacheText}

\cacheNumber{1913}\needspace{5\baselineskip}\cacheName{\href{http://coord.info/GC711GT}{PTTTN - A33} — \href{http://coord.info/GC711GT\Number{}794042550}{1913}}\cacheData{{2018/08/14 ptitloup, Traditional Cache (2/2)}}\begin{cacheText}Log général 

Accueillis par les Fabilab pour participer à l’Atlantic Évent en fin de semaine , nous décidons de filer vers le Croisic pour faire un des nombreux circuit du PTTTN. Et nous n’avons pas été déçus : de belles caches, de somptueux paysages et des rencontres fort sympathiques.

Merci ptitloup et un PF\end{cacheText}

\cacheNumber{1914}\needspace{5\baselineskip}\cacheName{\href{http://coord.info/GC711H0}{PTTTN - A34} — \href{http://coord.info/GC711H0\Number{}794042287}{1914}}\cacheData{{2018/08/14 ptitloup, Traditional Cache (1.5/1.5)}}\begin{cacheText}Log général 

Accueillis par les Fabilab pour participer à l’Atlantic Évent en fin de semaine , nous décidons de filer vers le Croisic pour faire un des nombreux circuit du PTTTN. Et nous n’avons pas été déçus : de belles caches, de somptueux paysages et des rencontres fort sympathiques.

Merci ptitloup\end{cacheText}

\cacheNumber{1915}\needspace{5\baselineskip}\cacheName{\href{http://coord.info/GC711H4}{PTTTN - A35} — \href{http://coord.info/GC711H4\Number{}794042093}{1915}}\cacheData{{2018/08/14 ptitloup, Traditional Cache (1.5/2)}}\begin{cacheText}Log général 

Accueillis par les Fabilab pour participer à l’Atlantic Évent en fin de semaine , nous décidons de filer vers le Croisic pour faire un des nombreux circuit du PTTTN. Et nous n’avons pas été déçus : de belles caches, de somptueux paysages et des rencontres fort sympathiques.

Merci ptitloup\end{cacheText}

\cacheNumber{1916}\needspace{5\baselineskip}\cacheName{\href{http://coord.info/GC711HA}{PTTTN - A36} — \href{http://coord.info/GC711HA\Number{}794041924}{1916}}\cacheData{{2018/08/14 ptitloup, Traditional Cache (2/1.5)}}\begin{cacheText}Log général 

Accueillis par les Fabilab pour participer à l’Atlantic Évent en fin de semaine , nous décidons de filer vers le Croisic pour faire un des nombreux circuit du PTTTN. Et nous n’avons pas été déçus : de belles caches, de somptueux paysages et des rencontres fort sympathiques.

Merci ptitloup\end{cacheText}

\cacheNumber{1917}\needspace{5\baselineskip}\cacheName{\href{http://coord.info/GC711HG}{PTTTN - A38} — \href{http://coord.info/GC711HG\Number{}794041508}{1917}}\cacheData{{2018/08/14 ptitloup, Traditional Cache (1.5/1.5)}}\begin{cacheText}Log général 

Accueillis par les Fabilab pour participer à l’Atlantic Évent en fin de semaine , nous décidons de filer vers le Croisic pour faire un des nombreux circuit du PTTTN. Et nous n’avons pas été déçus : de belles caches, de somptueux paysages et des rencontres fort sympathiques.

Merci ptitloup\end{cacheText}

\cacheNumber{1918}\needspace{5\baselineskip}\cacheName{\href{http://coord.info/GC711HP}{PTTTN - A39} — \href{http://coord.info/GC711HP\Number{}794041290}{1918}}\cacheData{{2018/08/14 ptitloup, Traditional Cache (2.5/1.5)}}\begin{cacheText}Log général 

Accueillis par les Fabilab pour participer à l’Atlantic Évent en fin de semaine , nous décidons de filer vers le Croisic pour faire un des nombreux circuit du PTTTN. Et nous n’avons pas été déçus : de belles caches, de somptueux paysages et des rencontres fort sympathiques.

Merci ptitloup\end{cacheText}

\cacheNumber{1919}\needspace{5\baselineskip}\cacheName{\href{http://coord.info/GC711HT}{PTTTN - A40} — \href{http://coord.info/GC711HT\Number{}794041069}{1919}}\cacheData{{2018/08/14 ptitloup, Traditional Cache (1.5/1.5)}}\begin{cacheText}Log général 

Accueillis par les Fabilab pour participer à l’Atlantic Évent en fin de semaine , nous décidons de filer vers le Croisic pour faire un des nombreux circuit du PTTTN. Et nous n’avons pas été déçus : de belles caches, de somptueux paysages et des rencontres fort sympathiques.

Merci ptitloup\end{cacheText}

\cacheNumber{1920}\needspace{5\baselineskip}\cacheName{\href{http://coord.info/GC711J0}{PTTTN - A12} — \href{http://coord.info/GC711J0\Number{}794050977}{1920}}\cacheData{{2018/08/14 ptitloup, Traditional Cache (1.5/1.5)}}\begin{cacheText}Log général 

Accueillis par les Fabilab pour participer à l’Atlantic Évent en fin de semaine , nous décidons de filer vers le Croisic pour faire un des nombreux circuit du PTTTN. Et nous n’avons pas été déçus : de belles caches, de somptueux paysages et des rencontres fort sympathiques.

Merci ptitloup  et un PF pour la superbe goutière\end{cacheText}

\cacheNumber{1921}\needspace{5\baselineskip}\cacheName{\href{http://coord.info/GC711V4}{PTTTN - A43} — \href{http://coord.info/GC711V4\Number{}794040212}{1921}}\cacheData{{2018/08/14 ptitloup, Traditional Cache (1.5/1.5)}}\begin{cacheText}Log général 

Accueillis par les Fabilab pour participer à l’Atlantic Évent en fin de semaine , nous décidons de filer vers le Croisic pour faire un des nombreux circuit du PTTTN. Et nous n’avons pas été déçus : de belles caches, de somptueux paysages et des rencontres fort sympathiques.

Merci ptitloup\end{cacheText}

\cacheNumber{1922}\needspace{5\baselineskip}\cacheName{\href{http://coord.info/GC711VW}{PTTTN - B20} — \href{http://coord.info/GC711VW\Number{}794056541}{1922}}\cacheData{{2018/08/14 ptitloup, Traditional Cache (3/1.5)}}\begin{cacheText}Log général 

Accueillis par les Fabilab pour participer à l’Atlantic Évent en fin de semaine , nous décidons de filer vers le Croisic pour faire un des nombreux circuit du PTTTN. Et nous n’avons pas été déçus : de belles caches, de somptueux paysages et des rencontres fort sympathiques.

Merci ptitloup\end{cacheText}

\cacheNumber{1923}\needspace{5\baselineskip}\cacheName{\href{http://coord.info/GC73GEF}{PTTTN - A13} — \href{http://coord.info/GC73GEF\Number{}794049168}{1923}}\cacheData{{2018/08/14 ptitloup, Traditional Cache (1.5/1.5)}}\begin{cacheText}Log général 

Accueillis par les Fabilab pour participer à l’Atlantic Évent en fin de semaine , nous décidons de filer vers le Croisic pour faire un des nombreux circuit du PTTTN. Et nous n’avons pas été déçus : de belles caches, de somptueux paysages et des rencontres fort sympathiques.

Merci ptitloup et un PF pour la cache parfaitement intégrée.\end{cacheText}

\cacheNumber{1924}\needspace{5\baselineskip}\cacheName{\href{http://coord.info/GC759BB}{PTTTN - A18} — \href{http://coord.info/GC759BB\Number{}794046976}{1924}}\cacheData{{2018/08/14 ptitloup, Traditional Cache (1.5/1.5)}}\begin{cacheText}Log général 

Accueillis par les Fabilab pour participer à l’Atlantic Évent en fin de semaine , nous décidons de filer vers le Croisic pour faire un des nombreux circuit du PTTTN. Et nous n’avons pas été déçus : de belles caches, de somptueux paysages et des rencontres fort sympathiques.

Merci ptitloup\end{cacheText}

\cacheNumber{1925}\needspace{5\baselineskip}\cacheName{\href{http://coord.info/GC75K65}{PTTTN - A41} — \href{http://coord.info/GC75K65\Number{}794040877}{1925}}\cacheData{{2018/08/14 ptitloup, Traditional Cache (1.5/1.5)}}\begin{cacheText}Log général 

Accueillis par les Fabilab pour participer à l’Atlantic Évent en fin de semaine , nous décidons de filer vers le Croisic pour faire un des nombreux circuit du PTTTN. Et nous n’avons pas été déçus : de belles caches, de somptueux paysages et des rencontres fort sympathiques.

Merci ptitloup\end{cacheText}

\cacheNumber{1926}\needspace{5\baselineskip}\cacheName{\href{http://coord.info/GC75K81}{PTTTN - A46} — \href{http://coord.info/GC75K81\Number{}794056880}{1926}}\cacheData{{2018/08/14 ptitloup, Traditional Cache (1.5/1.5)}}\begin{cacheText}Log général 

Accueillis par les Fabilab pour participer à l’Atlantic Évent en fin de semaine , nous décidons de filer vers le Croisic pour faire un des nombreux circuit du PTTTN. Et nous n’avons pas été déçus : de belles caches, de somptueux paysages et des rencontres fort sympathiques.

Merci ptitloup\end{cacheText}

\cacheNumber{1927}\needspace{5\baselineskip}\cacheName{\href{http://coord.info/GC75K8A}{PTTTN - A47} — \href{http://coord.info/GC75K8A\Number{}794056735}{1927}}\cacheData{{2018/08/14 ptitloup, Traditional Cache (1.5/1.5)}}\begin{cacheText}Log général 

Accueillis par les Fabilab pour participer à l’Atlantic Évent en fin de semaine , nous décidons de filer vers le Croisic pour faire un des nombreux circuit du PTTTN. Et nous n’avons pas été déçus : de belles caches, de somptueux paysages et des rencontres fort sympathiques.

Merci ptitloup\end{cacheText}

\cacheNumber{1928}\needspace{5\baselineskip}\cacheName{\href{http://coord.info/GC48M12}{croix de brandu} — \href{http://coord.info/GC48M12\Number{}794226435}{1928}}\cacheData{{2018/08/15 le turballais, Traditional Cache (1.5/1.5)}}\begin{cacheText}Arrivés en Loire Atlantique la veille pour assister à l’Atlantic Évent, nous décidons avec Fabilab de faire quelques caches sur La Turballe et Piriac sur mer. Aidés de nos vélos nous faisons le tour des caches en ville et sur la côte. Un grand merci pour ce travail de pose\end{cacheText}

\cacheNumber{1929}\needspace{5\baselineskip}\cacheName{\href{http://coord.info/GC48M35}{les chemins de brandu} — \href{http://coord.info/GC48M35\Number{}794227331}{1929}}\cacheData{{2018/08/15 Corto-Maltese, Traditional Cache (2/2)}}\begin{cacheText}Arrivés en Loire-Atlantique pour participer à l’Atlantic Évent avec nos amis les Fabilab nous profitons de ces quelques jours pour denicher les caches aux alentours. Celle-ci est trouvée rapidement. Merci pour la cache.\end{cacheText}

\cacheNumber{1930}\needspace{5\baselineskip}\cacheName{\href{http://coord.info/GC42ECV}{sentier côtier du castelli \Number{}1: port de Lerat} — \href{http://coord.info/GC42ECV\Number{}794251376}{1930}}\cacheData{{2018/08/15 Corto-Maltese, Traditional Cache (2/2)}}\begin{cacheText}Pas trop de moldu dans la zone, nous descendons de vélo et fouillons la zone.La nano est vite découverte. La vue sur les bateaux est vraiment chouette. Merci pour la cache.\end{cacheText}

\cacheNumber{1931}\needspace{5\baselineskip}\cacheName{\href{http://coord.info/GC42H26}{sentier côtier du castelli \Number{}4: sea line} — \href{http://coord.info/GC42H26\Number{}794254767}{1931}}\cacheData{{2018/08/15 Corto-Maltese, Traditional Cache (1.5/1.5)}}\begin{cacheText}Encore une très jolie cache fort sympathique. Elle a été rapidement localisée grâce à l’indice. Merci pour ce bon moment.\end{cacheText}

\cacheNumber{1932}\needspace{5\baselineskip}\cacheName{\href{http://coord.info/GC47DHG}{\Number{}04 Circuit de la duchesse (A la pêche)} — \href{http://coord.info/GC47DHG\Number{}794312465}{1932}}\cacheData{{2018/08/15 Corto-Maltese, Traditional Cache (2/2.5)}}\begin{cacheText}C' est en compagnie de Fabilab que nous partons à la pêche. Encore une cache super originale. Merci\end{cacheText}

\cacheNumber{1933}\needspace{5\baselineskip}\cacheName{\href{http://coord.info/GC48KTN}{Sentier du Castelli \Number{}25: tennis} — \href{http://coord.info/GC48KTN\Number{}794251131}{1933}}\cacheData{{2018/08/15 Corto-Maltese, Traditional Cache (2/1.5)}}\begin{cacheText}Arrivés sur le PZ , nous fouillons tous les lierres de la zone. La persévérance finit par payer et nous finissons par deloger la Belle. Chouette le log book est vierge de toute signature… Merci Corto Maltesse pour cette jolie cache.\end{cacheText}

\cacheNumber{1934}\needspace{5\baselineskip}\cacheName{\href{http://coord.info/GC48KVN}{les chemins piriacais} — \href{http://coord.info/GC48KVN\Number{}794250710}{1934}}\cacheData{{2018/08/15 Corto-Maltese, Traditional Cache (2/2.5)}}\begin{cacheText}Ici le GPS n’est pas très précis :nous regardons les photos, les logs précédents .Nous continuons les recherches ,nous élargissons la zone et nous finissons par mettre la main dessus .Ouffff. Merci pour la cache\end{cacheText}

\cacheNumber{1935}\needspace{5\baselineskip}\cacheName{\href{http://coord.info/GC4QAA5}{Sentier côtier du Castelli \Number{}2: Port Creux} — \href{http://coord.info/GC4QAA5\Number{}794253718}{1935}}\cacheData{{2018/08/15 Corto-Maltese, Traditional Cache (2.5/2.5)}}\begin{cacheText}C’est sous l’œil amusé des Fabilab (ils l'ont trouvée quelques jours plutôt et ils s’en amusent),que nous cherchons la Belle. Quelle super cache!!! Parfaitement intégrée :nous adorons. Merci et un PF\end{cacheText}

\cacheNumber{1936}\needspace{5\baselineskip}\cacheName{\href{http://coord.info/GC4QC0R}{Sentier du Castelli \Number{}3 (jouons aux billes)} — \href{http://coord.info/GC4QC0R\Number{}794254193}{1936}}\cacheData{{2018/08/15 Corto-Maltese, Traditional Cache (2/2)}}\begin{cacheText}Après avoir fouillé méticuleusement tous les buissons nous finissons par déloger la belle. Le tout est de comprendre le système… Superbe j’adore. Un PF pour la cache.\end{cacheText}

\cacheNumber{1937}\needspace{5\baselineskip}\cacheName{\href{http://coord.info/GC4QC1T}{sentier côtier du castelli \Number{}5: Les dents de Madame} — \href{http://coord.info/GC4QC1T\Number{}794259530}{1937}}\cacheData{{2018/08/15 Corto-Maltese, Traditional Cache (2/2)}}\begin{cacheText}Pas de difficultés pour trouver la cache. Nous tombons dessus très rapidement. Quelle chance.!!!Merci pour la cache.\end{cacheText}

\cacheNumber{1938}\needspace{5\baselineskip}\cacheName{\href{http://coord.info/GC58H3T}{\Number{}2 Circuit de la duchesse (hôtel a TB insolite)} — \href{http://coord.info/GC58H3T\Number{}794303072}{1938}}\cacheData{{2018/08/15 Corto-Maltese, Letterbox Hybrid (3/2)}}\begin{cacheText}Superbe ...génial mais un vrai casse tête pour ouvrir tout ça. En tandem de choc avec Fabilab , nous y parvenons au bout d'un certain temps et avec de grands fou rire. Du très beau travail...Félicitations et un PF.Merci\end{cacheText}

\cacheNumber{1939}\needspace{5\baselineskip}\cacheName{\href{http://coord.info/GC5XGH2}{Ferel - le four à pain -} — \href{http://coord.info/GC5XGH2\Number{}794314998}{1939}}\cacheData{{2018/08/15 tesoro72, Traditional Cache (1.5/2)}}\begin{cacheText}Arrivés au PZ nous découvrons un très joli jardin et un superbe four à pain  . Nous faisons plusieurs fois le tour et finissons par mettre l’œil dessus. Merci pour la découverte de ce lieu très sympa.\end{cacheText}

\cacheNumber{1940}\needspace{5\baselineskip}\cacheName{\href{http://coord.info/GC68JB6}{\Number{}03 Circuit de la Duchesse (La geoposte)} — \href{http://coord.info/GC68JB6\Number{}794290472}{1940}}\cacheData{{2018/08/15 Corto-Maltese, Letterbox Hybrid (1.5/2)}}\begin{cacheText}Après le repas nous prenons la direction de Saint Molf pour découvrir la Letter box et l'Hôtel à TB .Les Fabilab nous en parlent depuis fort longtemps et nous comprenons pourquoi : super!!!! Une petite carte postale, on récupere le TB et en route pour l'Hotel.Un grand merci.\end{cacheText}

\cacheNumber{1941}\needspace{5\baselineskip}\cacheName{\href{http://coord.info/GC6J9JT}{Filon de quartz dans du granite à Lérat} — \href{http://coord.info/GC6J9JT\Number{}794296098}{1941}}\cacheData{{2018/08/15 Allierla, Earthcache (1.5/2.5)}}\begin{cacheText}En vacances en Loire Atlantique pour assister à l' Atlantic Event nous en profitons pour faire quelques caches dans la région. La vue est superbe et la cote n'a rien à voir avec celle de la Cote Basque. Les réponses sont envoyées via la messagerie géocaching. Merci pour la cache\end{cacheText}

\cacheNumber{1942}\needspace{5\baselineskip}\cacheName{\href{http://coord.info/GC6V02J}{foot ferel} — \href{http://coord.info/GC6V02J\Number{}794315688}{1942}}\cacheData{{2018/08/15 tytoutiti, Traditional Cache (1/1.5)}}\begin{cacheText}En vacances en Loire-Atlantique pour assister à l’Atlantic Event prévu à Saint  Brévin les pins ,nous en profitons pour faire un détour avec Fabilab pour loguer le département 56. Nous faisons  cette cache en passant. Elle est toujours bien à sa place .Merci tytoutiti\end{cacheText}

\cacheNumber{1943}\needspace{5\baselineskip}\cacheName{\href{http://coord.info/GC6W973}{Entre mer et campagne .083} — \href{http://coord.info/GC6W973\Number{}795045848}{1943}}\cacheData{{2018/08/15 locleva, Traditional Cache (1.5/1.5)}}\begin{cacheText}La cache est rapidement trouvée grâce à l’indice. Merci\end{cacheText}

\cacheNumber{1944}\needspace{5\baselineskip}\cacheName{\href{http://coord.info/GC6ZBQZ}{Entre mer et campagne .084} — \href{http://coord.info/GC6ZBQZ\Number{}795046176}{1944}}\cacheData{{2018/08/15 locleva, Traditional Cache (1.5/1.5)}}\begin{cacheText}Nous retrouvons ce petit chemin ombragé qui nous fait bien plaisir par cette chaleur. La cache est trouvée rapidement grâce au spoiler. Nous reprenons la route merci\end{cacheText}

\cacheNumber{1945}\needspace{5\baselineskip}\cacheName{\href{http://coord.info/GC6ZBR4}{Entre mer et campagne .085} — \href{http://coord.info/GC6ZBR4\Number{}795046403}{1945}}\cacheData{{2018/08/15 locleva, Traditional Cache (1.5/2.5)}}\begin{cacheText}Le PZ est vite repéré et l’animal aussi ! Malheureusement il est un peu cassé et la boîte a disparu .Nous la remplaçons avec un petit bout de papier. Merci pour la cache et un PF\end{cacheText}

\cacheNumber{1946}\needspace{5\baselineskip}\cacheName{\href{http://coord.info/GC6ZBR6}{Entre mer et campagne .086} — \href{http://coord.info/GC6ZBR6\Number{}795050924}{1946}}\cacheData{{2018/08/15 locleva, Traditional Cache (2/2.5)}}\begin{cacheText}La promenade se poursuit ,toujours aussi agréable. Grâce au spoiler et à l’indice nous arrivons à trouver la belle. Merci\end{cacheText}

\cacheNumber{1947}\needspace{5\baselineskip}\cacheName{\href{http://coord.info/GC6ZBRA}{Entre mer et campagne .087} — \href{http://coord.info/GC6ZBRA\Number{}795051252}{1947}}\cacheData{{2018/08/15 locleva, Traditional Cache (1.5/3)}}\begin{cacheText}Ici nous découvrons une cache super sympa grâce au spoiler. Le bouchon a disparu : le log a déjà été changé mais il risque d’être mouillé très rapidement. Quel dommage !Merci.\end{cacheText}

\cacheNumber{1948}\needspace{5\baselineskip}\cacheName{\href{http://coord.info/GC6ZBRF}{Entre mer et campagne .089} — \href{http://coord.info/GC6ZBRF\Number{}795052787}{1948}}\cacheData{{2018/08/15 locleva, Traditional Cache (2.5/2.5)}}\begin{cacheText}La promenade se poursuit toujours aussi agréable et la cache est loguée au milieu du lierre. Merci\end{cacheText}

\cacheNumber{1949}\needspace{5\baselineskip}\cacheName{\href{http://coord.info/GC6ZBRG}{Entre mer et campagne .090} — \href{http://coord.info/GC6ZBRG\Number{}795053202}{1949}}\cacheData{{2018/08/15 locleva, Traditional Cache (2/1.5)}}\begin{cacheText}Nous découvrons une super jolie croix perdue au milieu des bois. La cache est rapidement localisée grâce à l’indice.Merci pour la découverte de ce petit monument.\end{cacheText}

\cacheNumber{1950}\needspace{5\baselineskip}\cacheName{\href{http://coord.info/GC7BGG4}{EMEC .091} — \href{http://coord.info/GC7BGG4\Number{}795053571}{1950}}\cacheData{{2018/08/15 locleva, Traditional Cache (1.5/2.5)}}\begin{cacheText}Nous découvrons un joli petit ...perdu au milieu des bois. Le log book est en vrac à l’intérieur . Il risque de ne pas durer longtemps. Quel dommage ...Merci pour la balade.\end{cacheText}

\cacheNumber{1951}\needspace{5\baselineskip}\cacheName{\href{http://coord.info/GC7BGGB}{EMEC .092} — \href{http://coord.info/GC7BGGB\Number{}795053879}{1951}}\cacheData{{2018/08/15 locleva, Traditional Cache (1.5/2)}}\begin{cacheText}Pour la dernière  nous avons eu un peu de mal :un coup à droite, un coup à gauche ,un coup en haut ,un coup en bas et Fabienne finit par mettre la main dessus :elle avait glissé du talus!Log signé , il est temps de rentrer. Merci pour ce super parcours .\end{cacheText}

\cacheNumber{1952}\needspace{5\baselineskip}\cacheName{\href{http://coord.info/GC7BGGF}{EMEC .093} — \href{http://coord.info/GC7BGGF\Number{}794312798}{1952}}\cacheData{{2018/08/15 locleva, Traditional Cache (1.5/2.5)}}\begin{cacheText}En vacances dans le département de Loire Atlantique pour assister à l' Atlantic Event , nous décidons de faire quelques kilomètres pour faire quelques caches dans le 56 afin de valider la carte. Dans les bois le GPS n’est pas très précis mais nous finissons par mettre la main sur la Belle. Trop jolie vraiment trop jolie. Merci pour la cache et un PF évidemment\end{cacheText}

\cacheNumber{1953}\needspace{5\baselineskip}\cacheName{\href{http://coord.info/GC7BGGW}{EMEC .094} — \href{http://coord.info/GC7BGGW\Number{}794313061}{1953}}\cacheData{{2018/08/15 locleva, Traditional Cache (1.5/1.5)}}\begin{cacheText}Aussitôt descendus de voiture nous balayons la zone jusqu’à trouver la Belle. Merci pour la cache.\end{cacheText}

\cacheNumber{1954}\needspace{5\baselineskip}\cacheName{\href{http://coord.info/GC7BGGY}{EMEC .095} — \href{http://coord.info/GC7BGGY\Number{}794313233}{1954}}\cacheData{{2018/08/15 locleva, Traditional Cache (2.5/2.5)}}\begin{cacheText}Pendant que nous déchargons les vélos, Fabienne inspecte les alentours et trouve la Belle. Merci pour la cache.\end{cacheText}

\cacheNumber{1955}\needspace{5\baselineskip}\cacheName{\href{http://coord.info/GC7BGH6}{EMEC .097} — \href{http://coord.info/GC7BGH6\Number{}794313579}{1955}}\cacheData{{2018/08/15 locleva, Traditional Cache (2.5/2)}}\begin{cacheText}La cache est vite repérée .Merci\end{cacheText}

\cacheNumber{1956}\needspace{5\baselineskip}\cacheName{\href{http://coord.info/GC7BGHA}{EMEC .098} — \href{http://coord.info/GC7BGHA\Number{}794313722}{1956}}\cacheData{{2018/08/15 locleva, Traditional Cache (2/1.5)}}\begin{cacheText}Le GPS nous promène un coup à droite un coup à gauche .On avance, on revient sur nos pas, heureusement que nous sommes trois pour les recherches. Nous finissons par la trouver au centre ! Merci pour la cache.\end{cacheText}

\cacheNumber{1957}\needspace{5\baselineskip}\cacheName{\href{http://coord.info/GC7BGHV}{EMEC .100} — \href{http://coord.info/GC7BGHV\Number{}794313838}{1957}}\cacheData{{2018/08/15 locleva, Traditional Cache (1.5/1.5)}}\begin{cacheText}L'indice ne nous laisse pas beaucoup de choix pour l’endroit de la cache. Merci.\end{cacheText}

\cacheNumber{1958}\needspace{5\baselineskip}\cacheName{\href{http://coord.info/GC7BGJ1}{EMEC .101} — \href{http://coord.info/GC7BGJ1\Number{}794313961}{1958}}\cacheData{{2018/08/15 locleva, Traditional Cache (1.5/1.5)}}\begin{cacheText}Ici aussi la cache est rapidement trouvée par Pierre. Trop facile. Merci pour la cache\end{cacheText}

\cacheNumber{1959}\needspace{5\baselineskip}\cacheName{\href{http://coord.info/GC7BGJ4}{EMEC .102} — \href{http://coord.info/GC7BGJ4\Number{}794314223}{1959}}\cacheData{{2018/08/15 locleva, Traditional Cache (2/2.5)}}\begin{cacheText}Et ici la cache demande un petit peu de recherche car le GPS nous promène .On fini par la trouver car elle fait un peu intrus. Merci pour la cache.\end{cacheText}

\cacheNumber{1960}\needspace{5\baselineskip}\cacheName{\href{http://coord.info/GC7BGJ7}{EMEC .103} — \href{http://coord.info/GC7BGJ7\Number{}794314605}{1960}}\cacheData{{2018/08/15 locleva, Traditional Cache (2.5/2)}}\begin{cacheText}La cache nécessite un peu de recherche. C’est Fabienne qui trouve où cela se croise. On approche du PZ et nous trouvons la Belle parterre. Nous la remettons à sa place. Merci pour la cache.\end{cacheText}

\cacheNumber{1961}\needspace{5\baselineskip}\cacheName{\href{http://coord.info/GC7BGK0}{EMEC .105} — \href{http://coord.info/GC7BGK0\Number{}794315284}{1961}}\cacheData{{2018/08/15 locleva, Traditional Cache (2.5/2.5)}}\begin{cacheText}Ici la photo nous mène tout droit à la cache qui est délogée par Fabienne. Nous continuons la balade. Merci pour la cache.\end{cacheText}

\cacheNumber{1962}\needspace{5\baselineskip}\cacheName{\href{http://coord.info/GC7VVC5}{Remember AE\Number{}8 : Paimboeuf 2016} — \href{http://coord.info/GC7VVC5\Number{}793991517}{1962}}\cacheData{{2018/08/16 Les GéotrouveTout44, Traditional Cache (1.5/1.5)}}\begin{cacheText}[{FTF}]

Paimboeuf 20 heures 06  

Accueillis par nos amis les Fabilab pour assister à l’Atlantic Évent , nous en profitons pour découvrir la région. Ce matin nous avons découvert La Turballe , Piriac sur mer et cet après-midi nous continuons sur Saint Molf et dans le département du Morbihan pour le valider . C’est sur la fin de notre promenade à vélo sur Ferel que l’alerte retentit mais il faut prendre son mal en patience !!! Vélos chargés nous prenons la direction de la nouvelle cache.Suspense dans la voiture ,le temps de trouver la Belle, de dérouler le log book… Nous sommes les premiers!!!Quel plaisir. Merci pour la cache.\end{cacheText}

\cacheNumber{1963}\needspace{5\baselineskip}\cacheName{\href{http://coord.info/GC3G822}{[Bac/ch/Ro/Bac] \Number{}04 pas de moteur!} — \href{http://coord.info/GC3G822\Number{}795145047}{1963}}\cacheData{{2018/08/16 bob le bricoleur, Traditional Cache (1.5/1.5)}}\begin{cacheText}En vacances en Loire Atlantique pour assister à l'Atlantic Event, nous en profitons pour faire des caches aux alentours. Aujourd'hui ,c'est Le Pellerin et ces circuits et nous faisons cette cache en passant.Merci\end{cacheText}

\cacheNumber{1964}\needspace{5\baselineskip}\cacheName{\href{http://coord.info/GC54Y87}{\Number{}DA VINCI CODE 01 Jacques Saunière} — \href{http://coord.info/GC54Y87\Number{}795542241}{1964}}\cacheData{{2018/08/16 Les GéotrouveTout44, Traditional Cache (1.5/1.5)}}\begin{cacheText}La première est la bonne!!! Personne en vu ,nous continuons. Merci pour la cache\end{cacheText}

\cacheNumber{1965}\needspace{5\baselineskip}\cacheName{\href{http://coord.info/GC55031}{\Number{}DA VINCI CODE 05 Bézu Fache} — \href{http://coord.info/GC55031\Number{}795542519}{1965}}\cacheData{{2018/08/16 Les GéotrouveTout44, Traditional Cache (1.5/1.5)}}\begin{cacheText}Le PZ est envahi de ronces, mais rien n'arrête monsieur. La cache est trouvée et loguée. Merci.\end{cacheText}

\cacheNumber{1966}\needspace{5\baselineskip}\cacheName{\href{http://coord.info/GC554D8}{\Number{}DA VINCI CODE 08 Le nombre d'Or} — \href{http://coord.info/GC554D8\Number{}795544042}{1966}}\cacheData{{2018/08/16 Les GéotrouveTout44, Traditional Cache (1.5/1.5)}}\begin{cacheText}Et non… On n' est pas tombé dedans ! Merci pour la cache.\end{cacheText}

\cacheNumber{1967}\needspace{5\baselineskip}\cacheName{\href{http://coord.info/GC55VT9}{\Number{}DA VINCI CODE 21 La Banque Zurichoise} — \href{http://coord.info/GC55VT9\Number{}795131512}{1967}}\cacheData{{2018/08/16 Les GéotrouveTout44, Traditional Cache (1.5/1.5)}}\begin{cacheText}Après un peu de recherches dans le trou ,nous finissons par trouver la belle. Merci pour la cache\end{cacheText}

\cacheNumber{1968}\needspace{5\baselineskip}\cacheName{\href{http://coord.info/GC55VVE}{\Number{}DA VINCI CODE 20 Marie-Madeleine} — \href{http://coord.info/GC55VVE\Number{}795113001}{1968}}\cacheData{{2018/08/16 Les GéotrouveTout44, Traditional Cache (1.5/1.5)}}\begin{cacheText}En vacances en Loire-Atlantique pour assister au 10è Atlantic Event, on nous a chaudement recommandé les circuits de Tintin et du Da Vinci code. Nous venons de terminer le premier à vélo et pendant que Monsieur range les vélos Madame en profite pour aller loguer la Belle. Très belle initiative que ce Da Vinci code j’adore l’idée. Merci pour la cache.\end{cacheText}

\cacheNumber{1969}\needspace{5\baselineskip}\cacheName{\href{http://coord.info/GC55WYQ}{\Number{}DA VINCI CODE 19 Le Sang Real} — \href{http://coord.info/GC55WYQ\Number{}795114612}{1969}}\cacheData{{2018/08/16 Les GéotrouveTout44, Traditional Cache (1.5/1.5)}}\begin{cacheText}Ici le GPS est bien précis et la cache nous attend bien sagement.Merci pour la cache\end{cacheText}

\cacheNumber{1970}\needspace{5\baselineskip}\cacheName{\href{http://coord.info/GC55X0P}{\Number{}DA VINCI CODE 18 Les Templiers} — \href{http://coord.info/GC55X0P\Number{}795115059}{1970}}\cacheData{{2018/08/16 Les GéotrouveTout44, Traditional Cache (1.5/1.5)}}\begin{cacheText}Aucun moldu à l’horizon c’est bien pratique pour signer. Merci pour la cache.\end{cacheText}

\cacheNumber{1971}\needspace{5\baselineskip}\cacheName{\href{http://coord.info/GC56BV9}{\Number{}DA VINCI CODE 09 Le Prieuré de Sion} — \href{http://coord.info/GC56BV9\Number{}795543733}{1971}}\cacheData{{2018/08/16 Les GéotrouveTout44, Traditional Cache (1.5/1.5)}}\begin{cacheText}Pas de soucis pour trouver la belle, je récupère le TB. Merci pour la cache.\end{cacheText}

\cacheNumber{1972}\needspace{5\baselineskip}\cacheName{\href{http://coord.info/GC56DTW}{\Number{}DA VINCI CODE 10 Sophie Neveu} — \href{http://coord.info/GC56DTW\Number{}795543410}{1972}}\cacheData{{2018/08/16 Les GéotrouveTout44, Traditional Cache (1.5/1.5)}}\begin{cacheText}Dur dur de localiser la belle mais nous finissons par y arriver .Entre deux ou trois mures nous signons le logbook. Merci pour la cache.\end{cacheText}

\cacheNumber{1973}\needspace{5\baselineskip}\cacheName{\href{http://coord.info/GC56F90}{\Number{}DA VINCI CODE 15 Le Louvre} — \href{http://coord.info/GC56F90\Number{}795116019}{1973}}\cacheData{{2018/08/16 Les GéotrouveTout44, Traditional Cache (2.5/1.5)}}\begin{cacheText}Le DNF annoncé ne nous décourage pas;Nous fouillons la zone méticuleusement et finissons par trouver la pierre qui cache… Merci pour ce parcours sympathique.\end{cacheText}

\cacheNumber{1974}\needspace{5\baselineskip}\cacheName{\href{http://coord.info/GC56F9M}{\Number{}DA VINCI CODE 16 Mona Lisa} — \href{http://coord.info/GC56F9M\Number{}795115571}{1974}}\cacheData{{2018/08/16 Les GéotrouveTout44, Traditional Cache (2.5/1.5)}}\begin{cacheText}Parfaites… Les coordonnées sont parfaites: nous loguons et reprenons  la route. Merci pour la cache.\end{cacheText}

\cacheNumber{1975}\needspace{5\baselineskip}\cacheName{\href{http://coord.info/GC56J2F}{\Number{}DA VINCI CODE 24 Rémy Legaludec} — \href{http://coord.info/GC56J2F\Number{}795141367}{1975}}\cacheData{{2018/08/16 Les GéotrouveTout44, Traditional Cache (1.5/1.5)}}\begin{cacheText}La promenade se poursuit :la cache est bien en place.Merci\end{cacheText}

\cacheNumber{1976}\needspace{5\baselineskip}\cacheName{\href{http://coord.info/GC56J3N}{\Number{}DA VINCI CODE 25 Temple Church} — \href{http://coord.info/GC56J3N\Number{}795142022}{1976}}\cacheData{{2018/08/16 Les GéotrouveTout44, Traditional Cache (1.5/1.5)}}\begin{cacheText}Très bel arbre!!! Le plus dur est de passer les barbelés, le reste n’est que formalité. Merci pour la cache\end{cacheText}

\cacheNumber{1977}\needspace{5\baselineskip}\cacheName{\href{http://coord.info/GC56J5R}{\Number{}DA VINCI CODE 26 Sarah} — \href{http://coord.info/GC56J5R\Number{}795541563}{1977}}\cacheData{{2018/08/16 Les GéotrouveTout44, Traditional Cache (1.5/1.5)}}\begin{cacheText}Pas de réseau et au pif la Belle est délogée. A la suivante. Merci\end{cacheText}

\cacheNumber{1978}\needspace{5\baselineskip}\cacheName{\href{http://coord.info/GC56J7R}{\Number{}DA VINCI CODE 27 Westminster Abbey} — \href{http://coord.info/GC56J7R\Number{}795542052}{1978}}\cacheData{{2018/08/16 Les GéotrouveTout44, Traditional Cache (1.5/1.5)}}\begin{cacheText}Arrivéz au PZ, nous repérons vite fait l'arbre concerné…? le tout est de choisir le bon !C’est Monsieur qui déniche la Belle. Merci pour la cache.\end{cacheText}

\cacheNumber{1979}\needspace{5\baselineskip}\cacheName{\href{http://coord.info/GC56JAC}{\Number{}DA VINCI CODE 28 Rosslyn Chapel} — \href{http://coord.info/GC56JAC\Number{}795147953}{1979}}\cacheData{{2018/08/16 Les GéotrouveTout44, Traditional Cache (1.5/1.5)}}\begin{cacheText}Ici l'indice n’est plus d’actualité: la cache n’est plus dans la liane mais au sol sous une pierre. Merci\end{cacheText}

\cacheNumber{1980}\needspace{5\baselineskip}\cacheName{\href{http://coord.info/GC56JC2}{\Number{}DA VINCI CODE 29 Le Cryptex} — \href{http://coord.info/GC56JC2\Number{}795146410}{1980}}\cacheData{{2018/08/16 Les GéotrouveTout44, Traditional Cache (3/1.5)}}\begin{cacheText}Sésame ….ouvre toi !!!Ayant adoré le Da Vinci Code il me reste quelques souvenirs heureusement…1er essai et le cryptex  s’ouvre. Merci pour cette série très sympa. Un PF.\end{cacheText}

\cacheNumber{1981}\needspace{5\baselineskip}\cacheName{\href{http://coord.info/GC5JV5K}{La geocache challenge des 2000 found it} — \href{http://coord.info/GC5JV5K\Number{}795138455}{1981}}\cacheData{{2018/08/16 Corto-Maltese, Unknown Cache (1.5/2)}}\begin{cacheText}Ça y est nous avons la 2000 sur la GC 55VT9 DA VINCI CODE 21 La Banque Zurichoise !!!! Merci Corto\Underscore{} Maltese pour cette cache challenge, une idée très originale\end{cacheText}

\cacheNumber{1982}\needspace{5\baselineskip}\cacheName{\href{http://coord.info/GC5VWM7}{TINTIN 1/20 L'OREILLE CASSEE} — \href{http://coord.info/GC5VWM7\Number{}795081954}{1982}}\cacheData{{2018/08/16 bokostar, Traditional Cache (1.5/2)}}\begin{cacheText}En  Loire-Atlantique pour assister à l’Atlantic Event, nous sommes accueillis par les Fabilab qui nous ont chaudement recommandé ce parcours. Arrivés au PZ l'indice nous guide tout droit jusqu’à la boite. Superbe travail,nous adorons. Merci pour la cache.\end{cacheText}

\cacheNumber{1983}\needspace{5\baselineskip}\cacheName{\href{http://coord.info/GC5VWN3}{TINTIN 3/20 LA CASTAFIORE} — \href{http://coord.info/GC5VWN3\Number{}795085619}{1983}}\cacheData{{2018/08/16 bokostar, Traditional Cache (1.5/1.5)}}\begin{cacheText}Arrivés sur les lieux, nous repérons facilement la belle. La castafiore nous attend !Nous repartons en sifflant!!! Merci pour cette belle réalisation.\end{cacheText}

\cacheNumber{1984}\needspace{5\baselineskip}\cacheName{\href{http://coord.info/GC5VWNA}{TINTIN 4/20 AU PAYS DE L'OR NOIR} — \href{http://coord.info/GC5VWNA\Number{}795094987}{1984}}\cacheData{{2018/08/16 bokostar, Traditional Cache (1.5/2.5)}}\begin{cacheText}Arrivés au stop, nous inspectons tous les panneaux et finissons par découvrir la Belle au pied du piquet.Merci pour ce travail.\end{cacheText}

\cacheNumber{1985}\needspace{5\baselineskip}\cacheName{\href{http://coord.info/GC5VWNY}{TINTIN 5/20 MOULINSART} — \href{http://coord.info/GC5VWNY\Number{}795095564}{1985}}\cacheData{{2018/08/16 bokostar, Traditional Cache (1.5/2.5)}}\begin{cacheText}Malgré le manque de réseau, nous arrivons à mettre la main sur la Belle. Encore une jolie réalisation. Merci pour la cache.\end{cacheText}

\cacheNumber{1986}\needspace{5\baselineskip}\cacheName{\href{http://coord.info/GC5VX1V}{TINTIN 7/20 LE SCEPTRE D'OTTOKAR} — \href{http://coord.info/GC5VX1V\Number{}795097774}{1986}}\cacheData{{2018/08/16 bokostar, Traditional Cache (1.5/2)}}\begin{cacheText}Magnifique. Arrivés près PZ nous repérons facilement la boîte mais lors ce que l’on découvre ….quelle surprise!!! Très belle réalisation ,un PF .Merci pour la cache.\end{cacheText}

\cacheNumber{1987}\needspace{5\baselineskip}\cacheName{\href{http://coord.info/GC5VX23}{TINTIN 8/20 LE TRESOR DE RACKAM LE ROUGE} — \href{http://coord.info/GC5VX23\Number{}795100973}{1987}}\cacheData{{2018/08/16 bokostar, Multi-cache (2.5/2.5)}}\begin{cacheText}Nous n'avons pas eu de problème pour trouver la première boite, Cela s'est compliqué par la suite ….Faut il regarder à droite ou à gauche?Dans le doute nous avons tout exploré!!! Enfin monsieur a finit par mettre la main sur le trésor!!!!Et oui un vrai trésor!!!Merci pour la cache.\end{cacheText}

\cacheNumber{1988}\needspace{5\baselineskip}\cacheName{\href{http://coord.info/GC5VX38}{TINTIN 11/20 L'ILE NOIRE} — \href{http://coord.info/GC5VX38\Number{}795103174}{1988}}\cacheData{{2018/08/16 bokostar, Letterbox Hybrid (2/2)}}\begin{cacheText}Ici pas trop de difficultés pour trouver la boîte. Monsieur se faufile sous les barbelés et déloge la belle. Nous avons oublié nos cartes postales à la voiture dommage .Le log book est très original.Merci pour la cache\end{cacheText}

\cacheNumber{1989}\needspace{5\baselineskip}\cacheName{\href{http://coord.info/GC5VX4H}{TINTIN 13/20 LA BOUCHERIE SANZOT} — \href{http://coord.info/GC5VX4H\Number{}795104082}{1989}}\cacheData{{2018/08/16 bokostar, Traditional Cache (2/2.5)}}\begin{cacheText}Cette cache nous a demandé beaucoup ,beaucoup de temps mais grâce au spoiler nous finissons par mettre la main dessus. Le log book est très humide et nous avons du mal à signer : photo pour preuve. Encore une idée  originale. Merci pour la cache.\end{cacheText}

\cacheNumber{1990}\needspace{5\baselineskip}\cacheName{\href{http://coord.info/GC5W0D4}{TINTIN 15/20 LES CIGARES DU PHARAON} — \href{http://coord.info/GC5W0D4\Number{}795105553}{1990}}\cacheData{{2018/08/16 bokostar, Traditional Cache (1.5/2)}}\begin{cacheText}La végétation a beaucoup changé depuis la photo. C’est le log précédent qui nous parle d’orties ,de ronces ,de moustiques etc. qui nous met sur la voie. La cache est encore très travaillée et toujours dans le thème.Félicitations. Et merci\end{cacheText}

\cacheNumber{1991}\needspace{5\baselineskip}\cacheName{\href{http://coord.info/GC5W0DC}{TINTIN 16/20 L'ETOILE MYSTERIEUSE} — \href{http://coord.info/GC5W0DC\Number{}795105995}{1991}}\cacheData{{2018/08/16 bokostar, Traditional Cache (1.5/2)}}\begin{cacheText}Le GPS est de plus en plus imprécis. Nous mettons beaucoup de temps et devons élargir les recherches mais la patience finit toujours par payer ;Nous trouvons la super cache. Encore du bon boulot et un grand merci.\end{cacheText}

\cacheNumber{1992}\needspace{5\baselineskip}\cacheName{\href{http://coord.info/GC5WYXD}{TINTIN 17/20 LE LOTUS BLEU} — \href{http://coord.info/GC5WYXD\Number{}795106675}{1992}}\cacheData{{2018/08/16 bokostar, Traditional Cache (2/2.5)}}\begin{cacheText}Il nous a fallu encore du temps pour trouver la belle mais  quelle belle surprise!!!Effectivement, les fourmis ont colonisé la boîte que nous prenons soin de bien nettoyer .Nous signons et repartons. Merci pour la belle.\end{cacheText}

\cacheNumber{1993}\needspace{5\baselineskip}\cacheName{\href{http://coord.info/GC5WYXK}{TINTIN 18/20 MILOU} — \href{http://coord.info/GC5WYXK\Number{}795109691}{1993}}\cacheData{{2018/08/16 bokostar, Traditional Cache (2/2)}}\begin{cacheText}Ici l’indice très précis et le spoiler nous mènent tout droit à la cache : pas besoin du GPS. Encore une cache originale parfaitement dans le thème… Félicitations.Oufff nous gagnons un peu de temps.L' indice relevé, nous continuons. Merci pour la cache\end{cacheText}

\cacheNumber{1994}\needspace{5\baselineskip}\cacheName{\href{http://coord.info/GC5WYXW}{TINTIN 14/20 LES 7 BOULES DE CRISTAL} — \href{http://coord.info/GC5WYXW\Number{}795104602}{1994}}\cacheData{{2018/08/16 bokostar, Traditional Cache (2/2.5)}}\begin{cacheText}Comme nos prédécesseurs, le GPS nous envoi assez loin de la cache. Mais nous finissons par la trouver car elle est assez remarquable. C’est une très belle réalisation vraiment j’adore. Un PF en plus. Merci\end{cacheText}

\cacheNumber{1995}\needspace{5\baselineskip}\cacheName{\href{http://coord.info/GC6EEFZ}{Coin coin [AE2016]} — \href{http://coord.info/GC6EEFZ\Number{}795055618}{1995}}\cacheData{{2018/08/16 lesdecouvreurs, Traditional Cache (2.5/4.5)}}\begin{cacheText}Ici nous découvrons un joli petit parc avec une mare aux canards ( coin coin) nous repérons rapidement la cache et partons à l’assaut… Merci .\end{cacheText}

\cacheNumber{1996}\needspace{5\baselineskip}\cacheName{\href{http://coord.info/GC6EFVG}{Le centre commercial - AE2016} — \href{http://coord.info/GC6EFVG\Number{}795056979}{1996}}\cacheData{{2018/08/16 tiphtiph, Traditional Cache (2/1)}}\begin{cacheText}Et oui la cache est toujours là, bien camouflée .Après avoir tourné en rond nous finissons par mettre la main dessus. Merci pour la cache.\end{cacheText}

\cacheNumber{1997}\needspace{5\baselineskip}\cacheName{\href{http://coord.info/GC6EFVY}{La mare - AE2016} — \href{http://coord.info/GC6EFVY\Number{}795059139}{1997}}\cacheData{{2018/08/16 tiphtiph, Traditional Cache (3/2.5)}}\begin{cacheText}Arrivés près du PZ nous mettons quelques minutes pour localiser la belle. Effectivement elle n’est pas loin de l’eau mais bien attachée. Merci pour cette cache.\end{cacheText}

\cacheNumber{1998}\needspace{5\baselineskip}\cacheName{\href{http://coord.info/GC6FH0V}{Cale du fer à cheval [AE2016]} — \href{http://coord.info/GC6FH0V\Number{}795060370}{1998}}\cacheData{{2018/08/16 KST44, Traditional Cache (1.5/2.5)}}\begin{cacheText}L’indice nous guide bien et nous recherchons donc dans la zone indiquée.Monsieur est en bas Madame en haut : c’est l’œil averti de Madame qui finit par la dénicher. Idée originale ! Merci pour la cache\end{cacheText}

\cacheNumber{1999}\needspace{5\baselineskip}\cacheName{\href{http://coord.info/GC6FH10}{L'Ancre [AE2016]} — \href{http://coord.info/GC6FH10\Number{}795059658}{1999}}\cacheData{{2018/08/16 KST44, Traditional Cache (2.5/1.5)}}\begin{cacheText}Arrivés près de l’ancre nous découvrons la façade d'un café qui est couverte d’objets hétéroclites. Assez original ! La cache est trouvée comme l’indice le dit dans le buisson. Merci pour la cache.\end{cacheText}

\cacheNumber{2000}\needspace{5\baselineskip}\cacheName{\href{http://coord.info/GC6KYB4}{L'aquatique [AE2016]} — \href{http://coord.info/GC6KYB4\Number{}795054925}{2000}}\cacheData{{2018/08/16 Les GéotrouveTout44, Traditional Cache (1/3)}}\begin{cacheText}Ce matin nous décidons de faire quelques caches sur  Paimbœuf avant de rejoindre Le Pellerin pour les caches de tintin. Grâce au flair de Pierre la Belle est vite délogée. Merci pour la cache.\end{cacheText}

\cacheNumber{2001}\needspace{5\baselineskip}\cacheName{\href{http://coord.info/GC6MCYZ}{Ou est le piège ? [AE2016]} — \href{http://coord.info/GC6MCYZ\Number{}795055253}{2001}}\cacheData{{2018/08/16 Les GéotrouveTout44, Traditional Cache (3.5/2)}}\begin{cacheText}Arrivés près du PZ nous tournons ,virons et finissons par mettre la main sur la cache sans camouflage. Le log book est détrempé: j’en  met un sec dans un petit sachet. Merci pour la cache.\end{cacheText}

\cacheNumber{2002}\needspace{5\baselineskip}\cacheName{\href{http://coord.info/GC7VAFT}{Remember Atlantic Event : St Viaud 2017} — \href{http://coord.info/GC7VAFT\Number{}795080280}{2002}}\cacheData{{2018/08/16 fabilab, Traditional Cache (1.5/1.5)}}\begin{cacheText}[{STF}]

Alors que nous sommes sur Paimbœuf pour quelques caches, l’alerte retentit.... une nouvelle cache vient de sortir !!! Une cache de nos amis les Fabilab qui plus est!!!! Nous perdons un peu de temps pour chercher la cache ( qui nous échappe) puis prenons la direction de Saint Viaud. Arrivés au PZ , nous découvrons une très jolie cache parfaitement adaptée au lieu! Bravo ! Nous déroulons le logbook  et…. une team vient de loguer sous notre nez!!!Quel dommage !!!Un grand merci les Fabilab et un PF pour la réalisation.\end{cacheText}

\cacheNumber{2003}\needspace{5\baselineskip}\cacheName{\href{http://coord.info/GC2DC2R}{le pont bossu} — \href{http://coord.info/GC2DC2R\Number{}795706773}{2003}}\cacheData{{2018/08/17 bob le bricoleur, Traditional Cache (2/3)}}\begin{cacheText}Dans la région pour assister à l’Atlantic Event , nous en profitons pour faire quelques caches aux alentours.Il y a un an, nous sommes déjà venu mais malheureusement la cache avait disparu. Aujourd’hui nous pouvons effacer ce vilain petit point bleu sur la carte. Elle est bien dans l’abri. Merci pour la découverte de ce joli pont et pour la cache.\end{cacheText}

\cacheNumber{2004}\needspace{5\baselineskip}\cacheName{\href{http://coord.info/GC4GCXM}{Le moulin de Beaulieu [Atlantic Event 2013]} — \href{http://coord.info/GC4GCXM\Number{}795688504}{2004}}\cacheData{{2018/08/17 KST44, Traditional Cache (2.5/1.5)}}\begin{cacheText}Dans la région pour assister à l’Atlantic Event , nous en profitons pour faire quelques caches aux alentours. Alors que nous entamons les recherches, le propriétaire du moulin arrive pour jeter des déchets verts. Vous cherchez le trésor? Heuuu oui!!!!!!Et nous voila à discuter géocaching . La cache est vite découverte grâce aux coordonnées précises. Merci pour ce bon moment passé.\end{cacheText}

\cacheNumber{2005}\needspace{5\baselineskip}\cacheName{\href{http://coord.info/GC4GD2Q}{La Croix du Moulin des Landes [Atlantic Event 2013} — \href{http://coord.info/GC4GD2Q\Number{}795699535}{2005}}\cacheData{{2018/08/17 KST44, Traditional Cache (2/2)}}\begin{cacheText}Arrivés sur le PZ, nous constatons qu'il y a effectivement  beaucoup d’orties mais nous tenons à loguer la belle donc nous y allons. Après plusieurs piqures on arrive à atteindre le Saint Graal. Merci pour la cache et la découverte de ce moulin.\end{cacheText}

\cacheNumber{2006}\needspace{5\baselineskip}\cacheName{\href{http://coord.info/GC4GD31}{Le Moulin de la Proutière [Atlantic Event 2013]} — \href{http://coord.info/GC4GD31\Number{}795700138}{2006}}\cacheData{{2018/08/17 KST44, Traditional Cache (2/2.5)}}\begin{cacheText}Ici nous découvrons un autre joli moulin qui semble à l’abandon. Quel dommage ! La cache est rapidement repérée grâce à l’indice. Merci pour le circuit.\end{cacheText}

\cacheNumber{2007}\needspace{5\baselineskip}\cacheName{\href{http://coord.info/GC4GD3A}{L'éolienne 1 [Atlantic Event 2013]} — \href{http://coord.info/GC4GD3A\Number{}795702780}{2007}}\cacheData{{2018/08/17 KST44, Traditional Cache (2.5/2)}}\begin{cacheText}Nous n'avons jamais approché une éolienne d'aussi près. C'est impressionnant!!!Apres quelques recherches nous finissons par mettre la main dessus: vraiment du beau travail. Merci pour la cache\end{cacheText}

\cacheNumber{2008}\needspace{5\baselineskip}\cacheName{\href{http://coord.info/GC4GDA0}{Le Caillou Blanc [Atlantic Event 2013]} — \href{http://coord.info/GC4GDA0\Number{}795705299}{2008}}\cacheData{{2018/08/17 KST44, Letterbox Hybrid (2.5/1.5)}}\begin{cacheText}Arrivés près du PZ, nous cherchons et Monsieur repère l'ancre ouffff. Tout est trempé à l’intérieur mais effectivement le logbook est bien à l’abri. Jolie fabrication. Un PF.\end{cacheText}

\cacheNumber{2009}\needspace{5\baselineskip}\cacheName{\href{http://coord.info/GC4GRZD}{Le bourg de St-Michel – Atlantic Event 2013 by Tib} — \href{http://coord.info/GC4GRZD\Number{}795673999}{2009}}\cacheData{{2018/08/17 Team, Multi-cache (1.5/1.5)}}\begin{cacheText}Aujourd’hui direction Saint-Michel chef chef pour faire quelques caches. Nous commençons par cette Multi qui nous fait de découvrir l’église et surtout le magasin des galettes, miam miam,. Promenade très sympa dans le bourg .Merci pour cette Multi\end{cacheText}

\cacheNumber{2010}\needspace{5\baselineskip}\cacheName{\href{http://coord.info/GC4GT7M}{Les tennis de Comberge – Atlantic Event 2013 by Tib} — \href{http://coord.info/GC4GT7M\Number{}795676268}{2010}}\cacheData{{2018/08/17 Team, Traditional Cache (1.5/1.5)}}\begin{cacheText}Dans la région pour assister à l’Atlantic Event , nous en profitons pour faire quelques caches aux alentours. Celle-ci est vite découverte grâce aux coordonnées précises.Quelle bonne idée!!!Merci pour la cache.\end{cacheText}

\cacheNumber{2011}\needspace{5\baselineskip}\cacheName{\href{http://coord.info/GC4GTD7}{La Viauderie \Number{}3 – Atlantic Event 2013 by Tib} — \href{http://coord.info/GC4GTD7\Number{}795679387}{2011}}\cacheData{{2018/08/17 Team, Unknown Cache (3/1.5)}}\begin{cacheText}C’est avec l’aide des Fabilab que nous obtenons les coordonnées de cette mystère.Le final se trouve près d’un lieu consacré aux acteurs de l'enigme… Une bonne idée .Après un bon moment de recherche, nous yeux finissent par découvrir la petite. Merci pour cette énigme et pour la cache.\end{cacheText}

\cacheNumber{2012}\needspace{5\baselineskip}\cacheName{\href{http://coord.info/GC4GTEN}{Le Bois Roy \Number{}4 – Atlantic Event 2013 by Tib} — \href{http://coord.info/GC4GTEN\Number{}795682373}{2012}}\cacheData{{2018/08/17 Team, Wherigo Cache (2/2)}}\begin{cacheText}Elle nous en a donné du mal cette Wherigo ! Nous avons dû mettre la demi-journée pour y arriver .Nous avons recommencé quatre fois le parcours car l’application plantait et  trois fois le reste car nous n’étions pas assez rapide pour noter les coordonnées. Mais nous l’avons eu ….ouffff .Quel soulagement. Un grand merci et un PF pour l'excellent moment passé.\end{cacheText}

\cacheNumber{2013}\needspace{5\baselineskip}\cacheName{\href{http://coord.info/GC557H2}{Gneiss à Saint Brevin} — \href{http://coord.info/GC557H2\Number{}795715592}{2013}}\cacheData{{2018/08/17 FREDAGNES, Earthcache (1/1)}}\begin{cacheText}Dans la région pour assister à l’Atlantic Event , nous en profitons pour faire quelques caches aux alentours. Nous nous arrêtons un moment pour admirer l'ancre et faire cette earth très instructive. Merci\end{cacheText}

\cacheNumber{2014}\needspace{5\baselineskip}\cacheName{\href{http://coord.info/GC6FH7M}{Atlantic Event 2016 - Cache de Nuit [AE2016]} — \href{http://coord.info/GC6FH7M\Number{}795722793}{2014}}\cacheData{{2018/08/17 KST44, Unknown Cache (2.5/5)}}\begin{cacheText}Après le Meet\And{}Greet d'accueil - 10ème AE 2018 de amadeusmozart ,la Team Meet and Greet composée de Fabilab,Pascaldi44 et Fleura64, part à la conquête de la night. Equipés de torches nous nous engageons le long de la Loire à la recherche d'indices. Nous finissons par mettre la main sur les coordonnées finales. L'échelle ,toute neuve, est dépliée pour permettre la signature du logbook. Merci pour cette super soirée.\end{cacheText}

\cacheNumber{2015}\needspace{5\baselineskip}\cacheName{\href{http://coord.info/GC7CQY3}{Boisoucis} — \href{http://coord.info/GC7CQY3\Number{}795708737}{2015}}\cacheData{{2018/08/17 karkarou, Traditional Cache (1.5/1.5)}}\begin{cacheText}Dans la région pour assister à l’Atlantic Event , nous en profitons pour faire quelques caches aux alentours.Arrivés dans le sous-bois nous nous laissons guider par le GPS qui nous mène à la cache. Non loin de là pousse un cyclamen des bois:c’est charmant!!! Merci pour la cache.\end{cacheText}

\cacheNumber{2016}\needspace{5\baselineskip}\cacheName{\href{http://coord.info/GC7MCCV}{Le Pin de Pâques} — \href{http://coord.info/GC7MCCV\Number{}795721096}{2016}}\cacheData{{2018/08/17 PyPa49, Traditional Cache (1.5/1.5)}}\begin{cacheText}Au PZ, c’est un va-et-vient incessant de moldus mais nous arrivons ,malgré tout ,à déloger la Belle. Merci pour la cache\end{cacheText}

\cacheNumber{2017}\needspace{5\baselineskip}\cacheName{\href{http://coord.info/GC7MEA6}{Le canon du Queen Mary's Revenge} — \href{http://coord.info/GC7MEA6\Number{}795720535}{2017}}\cacheData{{2018/08/17 Les Piratacaches, Traditional Cache (2/1.5)}}\begin{cacheText}Nous continuons notre périple : arrivés au PZ une charmante dame est installée sur le banc .Au bout d’un moment nous décidons de lui expliquer notre jeu pour pouvoir loguer la cache. Elle semble très intéressée… Peut-être une nouvelle adepte ? Merci pour la cache\end{cacheText}

\cacheNumber{2018}\needspace{5\baselineskip}\cacheName{\href{http://coord.info/GC7MEAR}{Alerte à Brévinlibu} — \href{http://coord.info/GC7MEAR\Number{}795719945}{2018}}\cacheData{{2018/08/17 Les Piratacaches, Traditional Cache (1.5/1.5)}}\begin{cacheText}Beaucoup de moldus affairés à regarder la mer, à jouer, à manger des glaces…. Nous repérons facilement la cache et repartons. Merci.\end{cacheText}

\cacheNumber{2019}\needspace{5\baselineskip}\cacheName{\href{http://coord.info/GC7MEAT}{Larguez les amarres !!} — \href{http://coord.info/GC7MEAT\Number{}795719526}{2019}}\cacheData{{2018/08/17 Les Piratacaches, Traditional Cache (1.5/1)}}\begin{cacheText}Malgré les moldus aux alentours la Belle est dénichée discrètement à la vue de tous. Merci pour la cache.\end{cacheText}

\cacheNumber{2020}\needspace{5\baselineskip}\cacheName{\href{http://coord.info/GC7NYHP}{La poudre de Perle à Pin Pin} — \href{http://coord.info/GC7NYHP\Number{}795712528}{2020}}\cacheData{{2018/08/17 Les Piratacaches, Traditional Cache (2/1.5)}}\begin{cacheText}Arrivés au PZ nous découvrons tout d’abord un superbe point de vue sur la Loire. Après avoir admiré le paysage , nous nous consacrons à la cache. Grâce à l’indice bien précis ,nous arrivons à la trouver et, quelle surprise unejolie cache. Très bon travail et un PF.Merci pour la cache.\end{cacheText}

\cacheNumber{2021}\needspace{5\baselineskip}\cacheName{\href{http://coord.info/GC7PFEQ}{Dans l'antre d'un blockhaus} — \href{http://coord.info/GC7PFEQ\Number{}795709378}{2021}}\cacheData{{2018/08/17 titi44250, Traditional Cache (2/3.5)}}\begin{cacheText}Trop génial cette idée!!! Encore faut-il oser!!! Heureusement Monsieur est courageux et il part à la conquête de la Belle. Logbook signé il retrouve la lumière du jour. Merci pour la cache e               t un PF pour l’endroit.\end{cacheText}

\cacheNumber{2022}\needspace{5\baselineskip}\cacheName{\href{http://coord.info/GC7QPXE}{Meet\And{}Greet d'accueil - 10ème AE 2018} — \href{http://coord.info/GC7QPXE\Number{}795724748}{2022}}\cacheData{{2018/08/17 amadeusmozart, Event Cache (1/1)}}\begin{cacheText}Un grand merci à amadeusmozart pour cette super soirée. Un accueil chaleureux, de nouvelles rencontres et de très bonnes spécialités.\end{cacheText}

\cacheNumber{2023}\needspace{5\baselineskip}\cacheName{\href{http://coord.info/GC7RCR5}{Où sont Bobine de Cuivre \And{} Bob le Bricoleur \Number{}4} — \href{http://coord.info/GC7RCR5\Number{}795711729}{2023}}\cacheData{{2018/08/17 bob le bricoleur, Traditional Cache (1.5/1.5)}}\begin{cacheText}Dans la région pour assister à l’Atlantic Event , nous en profitons pour faire quelques caches aux alentours. La Belle est vite trouvée. Merci pour la cache.\end{cacheText}

\cacheNumber{2024}\needspace{5\baselineskip}\cacheName{\href{http://coord.info/GC7VDFJ}{Remember AE\Number{}5 : St Michel-Chef-Chef 2013} — \href{http://coord.info/GC7VDFJ\Number{}795678457}{2024}}\cacheData{{2018/08/17 lesdecouvreurs, Traditional Cache (2.5/3.5)}}\begin{cacheText}Aujourd’hui nous jetons notre dévolu sur la ville de Saint-Michel chef chef pour faire quelques cache supplémentaires. Nous ne sommes pas trop adeptes de l’escalade mais tentons notre chance pour celle-ci .Nous avons bien fait car le Graal est atteint. Merci pour la cache.\end{cacheText}

\cacheNumber{2025}\needspace{5\baselineskip}\cacheName{\href{http://coord.info/GC7B8AZ}{[AE18] Virtual Reward Serpent de Mer} — \href{http://coord.info/GC7B8AZ\Number{}799652171}{2025}}\cacheData{{2018/08/18 FREDAGNES, Virtual Cache (2/3.5)}}\begin{cacheText}Un véritable festival ce 10ème AE... et pour couronner le tout une Virtuelle !!!! Nous avions déjà admiré ce superbe serpent de mer l’an dernier mais nous n’avions pas compté les poteaux !!! Un grand merci FREDAGNES pour ce beau cadeau 💝. Un PF évidemment.\end{cacheText}

\cacheNumber{2026}\needspace{5\baselineskip}\cacheName{\href{http://coord.info/GC7EYB3}{Le Dolmen de la Briordais} — \href{http://coord.info/GC7EYB3\Number{}796894851}{2026}}\cacheData{{2018/08/18 fabilab, Traditional Cache (2.5/1.5)}}\begin{cacheText}Un petit détour nous permet de découvrir ce beau dolmen. L'équipe se déploit et les recherches commencent autour de tout ce qui pique!!!C' est Didier qui finit par découvrir le superbe camouflage. Merci pour la cache et un PF évidemment.\end{cacheText}

\cacheNumber{2027}\needspace{5\baselineskip}\cacheName{\href{http://coord.info/GC7NWVK}{Croquetout} — \href{http://coord.info/GC7NWVK\Number{}795870402}{2027}}\cacheData{{2018/08/18 Les Piratacaches, Traditional Cache (2/1.5)}}\begin{cacheText}Enfin le jour de l’Atlantic Event tant attendu est là. Arrivée de bonne heure avec les organisateurs, je m’échappe pour faire les quelques caches aux alentours. Arrivée au PZ ,grâce à l’indice je finis par mettre la main sur  le petit requin. Ouffff. Merci Pour la cache et la superbe vue sur le pont de Saint-Nazaire.\end{cacheText}

\cacheNumber{2028}\needspace{5\baselineskip}\cacheName{\href{http://coord.info/GC7Q4G7}{AE2018 - Estuaire 17} — \href{http://coord.info/GC7Q4G7\Number{}799653400}{2028}}\cacheData{{2018/08/18 lesdecouvreurs, Letterbox Hybrid (1.5/1.5)}}\begin{cacheText}Pour mon Challenge personnel (Challenge des icônes) ,il me fallait une Letterbox. C’est à la fin du repas que nous nous échappons en compagnie de Fabienne des Fabilab pour découvrir cette Belle bien camouflée . Nous n’avons pas perdu de temps, heureusement, et sommes à l’heure pour la Night Cache.Merci pour la cache.\end{cacheText}

\cacheNumber{2029}\needspace{5\baselineskip}\cacheName{\href{http://coord.info/GC7QHWB}{1 Pinus} — \href{http://coord.info/GC7QHWB\Number{}796886357}{2029}}\cacheData{{2018/08/18 fabilab, Traditional Cache (2/2)}}\begin{cacheText}Co [{FTF}] avec les Pascaldi 44

Après la distribution des \og{}Road book\fg{},nous prenons la direction du parcours des Fabilab. Nous attendions avec impatience de découvrir les caches imaginées par nos amis.Stationnés au centre commercial, nous enfourchons les vélos pour nous rendre sur le PZ.Les coordonnées précises nous mènent tout droit à la cache. Parfaitement dans le thème, et bien bricolée , elle mérite son PF. Merci pour la cache.\end{cacheText}

\cacheNumber{2030}\needspace{5\baselineskip}\cacheName{\href{http://coord.info/GC7QMTK}{2 Crataegus} — \href{http://coord.info/GC7QMTK\Number{}796887495}{2030}}\cacheData{{2018/08/18 fabilab, Traditional Cache (2/2)}}\begin{cacheText}Co [{FTF}] avec les Pascaldi 44

Il nous en a fallu du temps pour la dénicher celle-là! Obsédés par l’ancêtre, nous l’avons entièrement dévêtu de son lierre!!!!

Hé bien non... la coquine était ailleurs !!! Merci pour la cache.\end{cacheText}

\cacheNumber{2031}\needspace{5\baselineskip}\cacheName{\href{http://coord.info/GC7QMTQ}{3 Fraxinus} — \href{http://coord.info/GC7QMTQ\Number{}796888247}{2031}}\cacheData{{2018/08/18 fabilab, Traditional Cache (2/2)}}\begin{cacheText}Co [{FTF}] avec les Pascaldi 44

Arrivées les premières ,nous commençons les recherches mais ce sont les garçons qui fermaient la marche. qui mettent la main sur le trésor. Du joli travail ! Merci pour la cache.\end{cacheText}

\cacheNumber{2032}\needspace{5\baselineskip}\cacheName{\href{http://coord.info/GC7QMTW}{4 Tamarix} — \href{http://coord.info/GC7QMTW\Number{}796889904}{2032}}\cacheData{{2018/08/18 fabilab, Traditional Cache (2/2)}}\begin{cacheText}Co [{FTF}] avec les Pascaldi 44

Lorsque nous arrivons près du PZ, une jeune femme patiente à côté de sa voiture. Nous la rassurons quant à nos intentions et commençons nos recherches. La cache repérée,nous jetons un coup d’œil vers la personne qui est très occupée à discuter avec l’agriculteur dans le champ. Tant mieux nous pouvons loguer et ranger la belle en toute tranquillité. Encore une réalisation très sympa. Bravo et merci pour la cache.\end{cacheText}

\cacheNumber{2033}\needspace{5\baselineskip}\cacheName{\href{http://coord.info/GC7QMV6}{5 Fagaceae} — \href{http://coord.info/GC7QMV6\Number{}796890547}{2033}}\cacheData{{2018/08/18 fabilab, Traditional Cache (2/2)}}\begin{cacheText}Co [{FTF}] avec les Pascaldi 44 

C’est pascale qui arrive en premier sur le PZ: \og{}tiens on dirait un serpent dans le tronc.\fg{}. Je lui répond \og{} c’est sûr,ce doit être la cache !!!\fg{}Pendant ce temps Pierre s’exclame \og{}c’est bon j’ai la cache!!!\fg{} et extirpe de son trou la boîte. Très intrigués nous nous approchons de la bestiole : Didier lance un caillou et le serpent disparaît ! Ouf nous avons vraiment eu chaud.Remis de nos émotions nous nous consacrons à la Belle qui finit par nous livrer son mystère. Un air de déjà-vu : au GTAQ 4? Un grand merci pour la cache et un PF évidemment.\end{cacheText}

\cacheNumber{2034}\needspace{5\baselineskip}\cacheName{\href{http://coord.info/GC7QMVW}{6 Salicaceae} — \href{http://coord.info/GC7QMVW\Number{}797640087}{2034}}\cacheData{{2018/08/18 fabilab, Traditional Cache (2/2)}}\begin{cacheText}Co [{FTF}] avec les Pascaldi 44 

Elle nous en a donné du mal celle-là!!!À droite de la route, à gauche de la route ,au pied de tous les arbres mais le GPS tends toujours vers le même endroit!!!Prenant le courage à deux mains nous affrontons les ronces et le taillis pour enfin dégager la Belle. Beaucoup de mal mais le résultat est là ! Merci pour la cache.\end{cacheText}

\cacheNumber{2035}\needspace{5\baselineskip}\cacheName{\href{http://coord.info/GC7QMW0}{7 Ruscus Aculeatus} — \href{http://coord.info/GC7QMW0\Number{}797646697}{2035}}\cacheData{{2018/08/18 fabilab, Traditional Cache (3/2)}}\begin{cacheText}Co[{FTF}] avec LesChons49 et Pascaldi44 

Arrivés au PZ , Didier, Pierre et moi même cherchons dans les différents bosquets de fragonnette mais rien !!! Nous examinons toute la zone mais toujours rien!!!Le nez dans les buissons nous entendons des voix….La Team LesChons49 vient de rejoindre le PZ mais les recherches s'éternisent!!!!!Au bout d'un certain temps et d'un temps certain la cache se révèle loin des coordonnées .L'indice en poche nous reprenons les vélos en direction de la prochaine. Merci pour la cache\end{cacheText}

\cacheNumber{2036}\needspace{5\baselineskip}\cacheName{\href{http://coord.info/GC7QMW5}{9 Laurus Nobilis} — \href{http://coord.info/GC7QMW5\Number{}797654415}{2036}}\cacheData{{2018/08/18 fabilab, Traditional Cache (2/2)}}\begin{cacheText}Oh la la… Arrivés sur le PZ,  le sol est jonché de saletés. Il faut regarder où l’on met les pieds!!!Ici la cache est rapidement trouvée heureusement mais hélas nous ne sommes plus les premiers .Nous avons perdu trop de temps sur les précédentes. Cache originale: un PF.Merci pour la cache.\end{cacheText}

\cacheNumber{2037}\needspace{5\baselineskip}\cacheName{\href{http://coord.info/GC7QMWF}{10 Ulex} — \href{http://coord.info/GC7QMWF\Number{}797661521}{2037}}\cacheData{{2018/08/18 fabilab, Traditional Cache (2/2)}}\begin{cacheText}Lorsque nous arrivons sur le PZ, nous n’avons pas de difficultés pour déloger la belle. Nous continuons. Merci pour la cache.\end{cacheText}

\cacheNumber{2038}\needspace{5\baselineskip}\cacheName{\href{http://coord.info/GC7QMWM}{11 Myrtaceae} — \href{http://coord.info/GC7QMWM\Number{}797662086}{2038}}\cacheData{{2018/08/18 fabilab, Traditional Cache (2/2)}}\begin{cacheText}Toujours en compagnie de la Team Leschons49 nous arrivons près des eucalyptus. C’est alors que nous voyons l’owner sur son vélo venir voir le déroulé des opérations. Après quelques recherches nous finissons par mettre la main sur la Belle. Merci pour la cache.\end{cacheText}

\cacheNumber{2039}\needspace{5\baselineskip}\cacheName{\href{http://coord.info/GC7QMWZ}{12 Vitis} — \href{http://coord.info/GC7QMWZ\Number{}798680353}{2039}}\cacheData{{2018/08/18 fabilab, Traditional Cache (2/2)}}\begin{cacheText}Co [{FTF}] avec les Pascaldi 44 et les Chons 44

Ici aussi la végétation a bien poussé et malgré nos recherches, la Belle nous résiste.Sa découverte ne tient qu’à un fil ! Grâce à l’aide de l’owner nous mettons enfin la main dessus. Encore une idée originale de cache. Merci et un PF évidemment.\end{cacheText}

\cacheNumber{2040}\needspace{5\baselineskip}\cacheName{\href{http://coord.info/GC7QMXX}{La boite à livres \Quoted{Mindin}} — \href{http://coord.info/GC7QMXX\Number{}799648149}{2040}}\cacheData{{2018/08/18 fabilab, Multi-cache (2/3)}}\begin{cacheText}Il me manque pour mon challenge personnel une Multi. Malheureusement le temps est compté…Alors que les Pascaldi44 et Pierre se rafraîchissent, je collecte dans la voiture les informations nécessaires pour résoudre cette cache.C’est au retour de la Wherigo que je viens recueillir l’indice H qu’il me manque pour avoir les coordonnées finales. Sous l’œil bienfaisant de l’owner , je viens dénicher la coquine. Oufff direction l’Aperombola.,.Merci pour cette énigme fort sympathique.\end{cacheText}

\cacheNumber{2041}\needspace{5\baselineskip}\cacheName{\href{http://coord.info/GC7R2ZG}{8 Populus} — \href{http://coord.info/GC7R2ZG\Number{}797649643}{2041}}\cacheData{{2018/08/18 fabilab, Traditional Cache (2/2)}}\begin{cacheText}Co [{FTF}] avec LesChons49 et Pascaldi44

La cache ne résiste pas longtemps aux six paires d’yeux qui sont sur le PZ. D’autant plus que l’indice est très explicite. Par chance aucun moldu dans les parages ,nous pouvons loguer et repartir. Merci pour la cache.\end{cacheText}

\cacheNumber{2042}\needspace{5\baselineskip}\cacheName{\href{http://coord.info/GC7R2ZP}{13 Corylus} — \href{http://coord.info/GC7R2ZP\Number{}798680584}{2042}}\cacheData{{2018/08/18 fabilab, Traditional Cache (2/2)}}\begin{cacheText}Trouvée en compagnie de Pascaldi 44

Arrivés sur le Pz, aucune difficulté pour voir la petite. Elle attend sagement notre passage. Merci pour la cache.\end{cacheText}

\cacheNumber{2043}\needspace{5\baselineskip}\cacheName{\href{http://coord.info/GC7R2ZW}{14 Cerasus} — \href{http://coord.info/GC7R2ZW\Number{}798690043}{2043}}\cacheData{{2018/08/18 fabilab, Traditional Cache (2/2)}}\begin{cacheText}Trouvée en compagnie de Pascaldi 44 et de LesChons 49 . 

Et oui comme tous nous avons inspecté les cerisiers !!! Et il a fallu du temps pour la déloger ! Logbook signé nous reprenons la route. Merci pour la cache\end{cacheText}

\cacheNumber{2044}\needspace{5\baselineskip}\cacheName{\href{http://coord.info/GC7R2ZY}{15 Platanus} — \href{http://coord.info/GC7R2ZY\Number{}798692142}{2044}}\cacheData{{2018/08/18 fabilab, Traditional Cache (2/2)}}\begin{cacheText}Co [{FTF}] avec les Pascaldi 44 et LesChons49

Nous en avons mis du temps pour la découvrir!!! Les coordonnées n’étant pas très justes, nous avons du inspecter tous les platanes ! ! ! Lorsqu’une partie de l’équipe reprenait les vélos, enfin la belle a été trouvée. Elle est pourtant bien visible...Encore une très belle cache .Merci\end{cacheText}

\cacheNumber{2045}\needspace{5\baselineskip}\cacheName{\href{http://coord.info/GC7R306}{Letter box \Quoted{arboretum}} — \href{http://coord.info/GC7R306\Number{}798703408}{2045}}\cacheData{{2018/08/18 fabilab, Unknown Cache (2.5/2)}}\begin{cacheText}Co [{FTF}] avec les Pascaldi 44 et Les Chons49. 

Le code secret en poche(je n'avais pas remarqué cette croix alors que cela fait plusieurs jours que j'y passe devant!!!), nous nous dirigeons vers le final de cette superbe série. C’est l'apothéose: une superbe boîte nous attend. Du très grand travail qui nous a procuré un immense plaisir. Un grand merci pour cette excellente journée. Je ne peux attribuer qu’un PF c’est bien dommage....\end{cacheText}

\cacheNumber{2046}\needspace{5\baselineskip}\cacheName{\href{http://coord.info/GC7VWJJ}{[AE 2018] Circuit du Bodon n°3 - Orc} — \href{http://coord.info/GC7VWJJ\Number{}798686617}{2046}}\cacheData{{2018/08/18 Les GéotrouveTout44, Traditional Cache (1.5/1.5)}}\begin{cacheText}C'est en compagnie de Pascali44 que nous découvrons la Belle.C'est une réalisation très sympa.Du bon travail .Merci pour la cache.\end{cacheText}

\cacheNumber{2047}\needspace{5\baselineskip}\cacheName{\href{http://coord.info/GC7VX12}{[AE2018] - Canal de Bodon - La Licorne} — \href{http://coord.info/GC7VX12\Number{}798683601}{2047}}\cacheData{{2018/08/18 Miss découvreuse, Traditional Cache (1.5/1.5)}}\begin{cacheText}Trouvée en compagnie de Pascaldi 44 lors de l’AE.

Arrivés au PZ , les coordonnées GPS bien précises nous mènent tout droit à la cache . Nous découvrons une superbe réalisation. Du très bon travail et merci pour la cache. Un PF évidemment.\end{cacheText}

\cacheNumber{2048}\needspace{5\baselineskip}\cacheName{\href{http://coord.info/GC7W6TN}{[AE 2018] Patrimoine 7 : LA VILLA N°9} — \href{http://coord.info/GC7W6TN\Number{}799645295}{2048}}\cacheData{{2018/08/18 Les GéotrouveTout44, Wherigo Cache (1.5/1.5)}}\begin{cacheText}Co [{FTF}] avec les Fabilab et les Pascaldi 44 . 

Ayant terminés le parcours Arboretum des Fabilab avec un peu d’avance il nous reste un peu de temps avant l’Apérombola. Nous partons donc à la découverte de la villa. Une promenade très enrichissante qui se clôturera par FTF : quelle bonne surprise!!! Merci pour la cache.\end{cacheText}

\cacheNumber{2049}\needspace{5\baselineskip}\cacheName{\href{http://coord.info/GC7M248}{Atlantic Event 2018} — \href{http://coord.info/GC7M248\Number{}794965589}{2049}}\cacheData{{2018/08/19 duj\Underscore{}family, Event Cache (1/1)}}\begin{cacheText}Enfin le jour tant attendu est arrivé. Que du bonheur!!! Lab caches, virtuelle , multi, tradi,earthcache,Wherigo, letterbox, mystery... un véritable festival. Une organisation parfaite et une super bonne humeur. Un grand merci à toute l’équipe.\end{cacheText}

\cacheNumber{2050}\needspace{5\baselineskip}\cacheName{\href{http://coord.info/GC7W560}{Pause ou pêche...????} — \href{http://coord.info/GC7W560\Number{}799699829}{2050}}\cacheData{{2018/08/22 yvain64, Traditional Cache (1.5/1.5)}}\begin{cacheText}Grrrr ....C’est toujours lorsque l’on n’ est pas là que les nouvelles caches paraissent !!!!! C’est donc au retour de l’Atlantic Évent que nous allons déloger la belle. L’endroit ,paisible, est charmant. Pour nous cela sera pause…. Merci pour cette belle découverte.\end{cacheText}

\cacheNumber{2051}\needspace{5\baselineskip}\cacheName{\href{http://coord.info/GC62PRM}{TAG-FERIA} — \href{http://coord.info/GC62PRM\Number{}799705057}{2051}}\cacheData{{2018/09/04 mizaga, Traditional Cache (1.5/1.5)}}\begin{cacheText}Sur Pomarez pour continuer la landaise de crispol40 j’en profite pour faire çette cache en passant. Dans mes bras tu trouveras me dit l’indice: j’ai eu beau chercher je n’ai rien trouvé .Je dirais plutôt au sol tu me trouveras. Merci pour la découverte de cette fresque et pour la cache.\end{cacheText}

\cacheNumber{2052}\needspace{5\baselineskip}\cacheName{\href{http://coord.info/GC6N7DY}{Alphabet Landais inverse.... z comme Pomarez.} — \href{http://coord.info/GC6N7DY\Number{}799702629}{2052}}\cacheData{{2018/09/04 crispol40, Multi-cache (1.5/1.5)}}\begin{cacheText}Enfin une journée de repos tranquille ! Je peux faire un peu de mon activité favorite. Direction Pomarez pour continuer la Landaise. Je commence par cette Multi qui me fait découvrir cette jolie ville tournée vers les courses landaises. Après avoir perdu un peu de temps à discuter avec l’un des jardiniers, je finis par avoir tous les indices en poche et par calculer les coordonnées finales. Bingo, la petite m'attend dans les tentacules. J’emporte les 3TB qui sont coincés là depuis un long moment . j'ai finis par découvrir la photo de fond sur le parcours!!Merci crispol40 pour cette super Multi.\end{cacheText}

\cacheNumber{2053}\needspace{5\baselineskip}\cacheName{\href{http://coord.info/GC7CE19}{LA LANDAISE \Number{} 01} — \href{http://coord.info/GC7CE19\Number{}799706537}{2053}}\cacheData{{2018/09/04 crispol40, Unknown Cache (2.5/1.5)}}\begin{cacheText}Ici la coquine m’a donné bien du mal ! Mon GPS  n’est pas très précis et à force de persévérance je finis par la déloger: on arrête de jouer au chat et à la souris!!!Merci pour les énigmes et pour la cache.\end{cacheText}

\cacheNumber{2054}\needspace{5\baselineskip}\cacheName{\href{http://coord.info/GC7CYGR}{LA LANDAISE \Number{} 04} — \href{http://coord.info/GC7CYGR\Number{}799707081}{2054}}\cacheData{{2018/09/04 crispol40, Unknown Cache (3/1.5)}}\begin{cacheText}Je n’avais pas trouvé la cache d’origine lors de mon premier passage : elle avait disparu !!!Aujourd’hui c’est sans difficulté que je découvre la belle. Merci pour la cache\end{cacheText}

\cacheNumber{2055}\needspace{5\baselineskip}\cacheName{\href{http://coord.info/GC7CYHA}{LA LANDAISE \Number{} 05} — \href{http://coord.info/GC7CYHA\Number{}799708555}{2055}}\cacheData{{2018/09/04 crispol40, Unknown Cache (3/1.5)}}\begin{cacheText}Ici l'indice,très explicite, me mène tout droit à la cache. Tout est en place , je signe et continue ma route. Merci pour la cache.\end{cacheText}

\cacheNumber{2056}\needspace{5\baselineskip}\cacheName{\href{http://coord.info/GC7CYHN}{LA LANDAISE \Number{} 06} — \href{http://coord.info/GC7CYHN\Number{}799709494}{2056}}\cacheData{{2018/09/04 crispol40, Unknown Cache (2/1.5)}}\begin{cacheText}Ici aussi pas de difficultés pour déloger la Belle, il faut juste faire attention à la clôture électrique pendant les recherches. La belle a un peu souffert ;le log book est humide et le scotch qui tient l'aimant n’est plus efficace. Je remets l'aimant dans la cache au détriment du log book. Désolée mais  je n’ai rien pour faire la maintenance. Merci pour la cache\end{cacheText}

\cacheNumber{2057}\needspace{5\baselineskip}\cacheName{\href{http://coord.info/GC7CYJ4}{LA LANDAISE \Number{} 07} — \href{http://coord.info/GC7CYJ4\Number{}799710257}{2057}}\cacheData{{2018/09/04 crispol40, Unknown Cache (4/3)}}\begin{cacheText}Ici le PZ est envahi de ronces je comprends le niveau de difficulté!!! Après réflexion je trouve un passage pour accéder et découvrir la belle. Merci pour la cache.\end{cacheText}

\cacheNumber{2058}\needspace{5\baselineskip}\cacheName{\href{http://coord.info/GC7CYJH}{LA LANDAISE \Number{} 08} — \href{http://coord.info/GC7CYJH\Number{}799710917}{2058}}\cacheData{{2018/09/04 crispol40, Unknown Cache (1.5/1.5)}}\begin{cacheText}La belle est dénichée entre les bras. Je confirme qu’il n’y a plus de bouchon. Et je n’en ai pas pour faire la maintenance. Merci pour la cache.\end{cacheText}

\cacheNumber{2059}\needspace{5\baselineskip}\cacheName{\href{http://coord.info/GC7CYJW}{LA LANDAISE \Number{} 09} — \href{http://coord.info/GC7CYJW\Number{}799711390}{2059}}\cacheData{{2018/09/04 crispol40, Unknown Cache (4/1.5)}}\begin{cacheText}Arrivée sur le PZ  il n’y a pas beaucoup d’endroits où chercher ! Heureusement l’indice est là pour m’aiguiller : la belle est délogée. Merci pour la cache.\end{cacheText}

\cacheNumber{2060}\needspace{5\baselineskip}\cacheName{\href{http://coord.info/GC7CYK8}{LA LANDAISE \Number{} 10} — \href{http://coord.info/GC7CYK8\Number{}799712178}{2060}}\cacheData{{2018/09/04 crispol40, Unknown Cache (3.5/1.5)}}\begin{cacheText}Le GPS ,précis ,me mène tout droit à la cache : pas de perte de temps pour la déloger. J'essuie la boite car elle a pris l'eau, heureusement ,le log book est tout sec dans son sachet hermétique. Merci pour la cache.\end{cacheText}

\cacheNumber{2061}\needspace{5\baselineskip}\cacheName{\href{http://coord.info/GC7CZ31}{LA LANDAISE \Number{} 12} — \href{http://coord.info/GC7CZ31\Number{}799713602}{2061}}\cacheData{{2018/09/04 crispol40, Unknown Cache (3.5/1.5)}}\begin{cacheText}Alors là je dis bravo ! On ne comprend l'indice que lorsqu’on trouve la cache ! Il fallait avoir l’idée…. Paul l'a fait!!!. Merci pour la cache et un PF évidemment.\end{cacheText}

\cacheNumber{2062}\needspace{5\baselineskip}\cacheName{\href{http://coord.info/GC7CZ36}{LA LANDAISE \Number{} 13} — \href{http://coord.info/GC7CZ36\Number{}799716066}{2062}}\cacheData{{2018/09/04 crispol40, Unknown Cache (3.5/3.5)}}\begin{cacheText}Trop fière d’avoir réussi à la loguer celle-là!!! Merci pour la cache.\end{cacheText}

\cacheNumber{2063}\needspace{5\baselineskip}\cacheName{\href{http://coord.info/GC7CZ3A}{LA LANDAISE \Number{} 14} — \href{http://coord.info/GC7CZ3A\Number{}799719542}{2063}}\cacheData{{2018/09/04 crispol40, Unknown Cache (2/1.5)}}\begin{cacheText}Et une de plus qui sera trouvée sans difficulté au pieds. Je continue .Merci Crispol40\end{cacheText}

\cacheNumber{2064}\needspace{5\baselineskip}\cacheName{\href{http://coord.info/GC7CZ46}{LA LANDAISE \Number{} 15} — \href{http://coord.info/GC7CZ46\Number{}799720072}{2064}}\cacheData{{2018/09/04 crispol40, Unknown Cache (5/2.5)}}\begin{cacheText}Effectivement ici c'est avec un peu d’équilibre quel'on accède à la Belle. C’est génial ….cela permet de compléter ma matrice. Merci Polo\end{cacheText}

\cacheNumber{2065}\needspace{5\baselineskip}\cacheName{\href{http://coord.info/GC7CZ4E}{LA LANDAISE \Number{} 16} — \href{http://coord.info/GC7CZ4E\Number{}799721293}{2065}}\cacheData{{2018/09/04 crispol40, Unknown Cache (3/1.5)}}\begin{cacheText}Oulala le Pz est envahi d’herbes et de ronces ! Réfléchissons…. il n’a pas pu poser la cache au milieu de rien!!!Maudit GPS!!! cherchons ! Tiens du grillage et Bingo!!! Encore une cache originale .Merci crispol pour la cache.\end{cacheText}

\cacheNumber{2066}\needspace{5\baselineskip}\cacheName{\href{http://coord.info/GC7CZ4Q}{LA LANDAISE \Number{} 17} — \href{http://coord.info/GC7CZ4Q\Number{}799723277}{2066}}\cacheData{{2018/09/04 crispol40, Unknown Cache (3.5/1.5)}}\begin{cacheText}Waouh encore une super cache :du bricolage comme je l’aime. L'indice et le GPS sont fort utiles pour localiser la cache. Merci pour la cache\end{cacheText}

\cacheNumber{2067}\needspace{5\baselineskip}\cacheName{\href{http://coord.info/GC5CVRM}{[BSD] \Number{}032} — \href{http://coord.info/GC5CVRM\Number{}802507735}{2067}}\cacheData{{2018/09/09 GéoLandesTour, Traditional Cache (1.5/1.5)}}\begin{cacheText}Encore une cache qui a disparu. J’assure la maintenance avec une nouvelle boîte et un logbook tout neuf. Le nouvel indice est au pied ! Merci\end{cacheText}

\cacheNumber{2068}\needspace{5\baselineskip}\cacheName{\href{http://coord.info/GC5CVRW}{[BSD] \Number{}033} — \href{http://coord.info/GC5CVRW\Number{}802505666}{2068}}\cacheData{{2018/09/09 GéoLandesTour, Traditional Cache (1.5/1.5)}}\begin{cacheText}Sur Dax pour un rendez-vous, j’en profite pour aller faire quelques cache qui me manque. Arrivée au pays aide j’ai beau fouiller je ne trouve rien. Je fais donc une maintenance avec un nouveau log book. Lundi ça n’a plus de sens : la boîte se trouve au pied. Merci.\end{cacheText}

\cacheNumber{2069}\needspace{5\baselineskip}\cacheName{\href{http://coord.info/GC5EMND}{Un lavoir reculé mais comme neuf} — \href{http://coord.info/GC5EMND\Number{}802518875}{2069}}\cacheData{{2018/09/09 kar@melos40, Traditional Cache (2.5/2.5)}}\begin{cacheText}Arrivée  au PZ , je découvre un magnifique lavoir restauré. En regardant bien, la boîte est visible d’en bas. Merci pour la découverte de ce lieu.\end{cacheText}

\cacheNumber{2070}\needspace{5\baselineskip}\cacheName{\href{http://coord.info/GC6VYK7}{I002	 - Géodyssée 40 64} — \href{http://coord.info/GC6VYK7\Number{}802499760}{2070}}\cacheData{{2018/09/09 gilles64, Traditional Cache (1.5/1.5)}}\begin{cacheText}C’est mon second passage sur cette cache. En effet il y a quelques temps je ne l’avais pas trouvée et c’est une amie qui m’avait fait remarquer que sur la photo… En effet !!!!Aujourd’hui sur Capbreton j’en profite pour effacer ce vilain petit point bleu. Il ne faut pas être trop petit !!!!Merci Gilles pour cette cache.\end{cacheText}

\cacheNumber{2071}\needspace{5\baselineskip}\cacheName{\href{http://coord.info/GCW6QG}{STATION BALNEAIRE DE CAPBRETON} — \href{http://coord.info/GCW6QG\Number{}802500436}{2071}}\cacheData{{2018/09/09 f8pkp, Multi-cache (1.5/1)}}\begin{cacheText}Sur Capbreton pour passer la journée j’en profite pour faire cette petite Multi qui me tente depuis un certain temps. Elle nous fait découvrir le bord de plage et le petit port.Coordonnees en poche je me rend sur le PZ et l’indice me guide tout droit à la cache. Nous avons une superbe vue sur le port et je libère l’objet voyageur. Merci pour cette cache.\end{cacheText}

\cacheNumber{2072}\needspace{5\baselineskip}\cacheName{\href{http://coord.info/GC3V8CN}{Aristide BRIAND : Le Pèlerin de la Paix} — \href{http://coord.info/GC3V8CN\Number{}802645108}{2072}}\cacheData{{2018/09/14 Nathalain, Traditional Cache (3/1.5)}}\begin{cacheText}C’est au retour de la Wherigo que nous nous arrêtons faire cette tradi parfaitement intégrée dans le paysage. Du très bon travail. Nous signons Team Atlantic ( avec les Fabilab)et repartons vers la Virtuelle . Merci pour la cache et un PF\end{cacheText}

\cacheNumber{2073}\needspace{5\baselineskip}\cacheName{\href{http://coord.info/GC3Y453}{Musée Dobrée} — \href{http://coord.info/GC3Y453\Number{}802637400}{2073}}\cacheData{{2018/09/14 Ava Dahmer, Traditional Cache (1.5/2)}}\begin{cacheText}Nous découvrons , toujours en compagnie de Fabilab,un magnifique palais Dobrée .L’architecture est superbe. La cache est délogée sans difficulté mais il n’y a aucun TB dans la boîte. Nous signons sous le pseudo Team Atlantic.Merci pour la cache.\end{cacheText}

\cacheNumber{2074}\needspace{5\baselineskip}\cacheName{\href{http://coord.info/GC4D85A}{Basilique Saint-Nicolas} — \href{http://coord.info/GC4D85A\Number{}802641192}{2074}}\cacheData{{2018/09/14 fafahakkai, Traditional Cache (1.5/2)}}\begin{cacheText}C’est toujours accompagnés des Fablab que nous nous dirigeons vers la maison de Saint Nicolas. Arrivés au PZ, le lieu est déjà occupé par un chercheur anglais, s’il vous plaît ! Nous échangeons quelques mots et signons le logbook sous le pseudo Team Atlantic .Nous repartons vers la wherigo. Merci pour la cache.\end{cacheText}

\cacheNumber{2075}\needspace{5\baselineskip}\cacheName{\href{http://coord.info/GC4D86Y}{Chapelle de l’Immaculée} — \href{http://coord.info/GC4D86Y\Number{}802994043}{2075}}\cacheData{{2018/09/14 fafahakkai, Traditional Cache (1.5/1.5)}}\begin{cacheText}Sur Nantes pour participer au Mega Géocoinfest accompagnés des Fabilab, nous en profitons pour faire quelques caches dans la ville.

Cet après-midi nous avons rendez-vous avec la ligue des gentlemen pour mener l’enquête. Malgré notre concentration, nous repérons cette tradi. Nous en profitons pour loguer cette cache sous le pseudo Team Atlantic (Fleura 64, Fabilab, Socolau et abers29). Merci Pour la découverte de cette jolie chapelle.\end{cacheText}

\cacheNumber{2076}\needspace{5\baselineskip}\cacheName{\href{http://coord.info/GC5JX5A}{Cours Saint-André} — \href{http://coord.info/GC5JX5A\Number{}802992651}{2076}}\cacheData{{2018/09/14 Gokyo44, Traditional Cache (1.5/1.5)}}\begin{cacheText}Arrivés près du PZ, nous voyons que la zone est occupée par les forains de la fête foraine. Tout se complique : car ils sont dans la caravane à côté de la cache. Malgré nos à priori l’appel de la cache est plus fort .Nous nous approchons ...Les forains, intrigués par le balai incessant des géocacheurs ,sortent de la caravane. Après leur avoir expliqué le principe du géocaching nous allons loguer tranquillement la belle sous le pseudo Team Atlantic. Merci pour la cache.\end{cacheText}

\cacheNumber{2077}\needspace{5\baselineskip}\cacheName{\href{http://coord.info/GC63JHT}{Le château des Ducs de Bretagne} — \href{http://coord.info/GC63JHT\Number{}802994250}{2077}}\cacheData{{2018/09/14 Gokyo44, Traditional Cache (1.5/1.5)}}\begin{cacheText}Sur Nantes pour participer au Mega Géocoinfest accompagnés des Fabilab, nous en profitons pour faire quelques caches dans la ville.

Cet après-midi nous avons rendez-vous avec la ligue des gentlemen pour mener l’enquête. Malgré notre concentration, nous repérons cette tradi. Nous en profitons pour loguer cette cache.Nous découvrons ce superbe château et la cache qui est au pied. Merci pour la belle.\end{cacheText}

\cacheNumber{2078}\needspace{5\baselineskip}\cacheName{\href{http://coord.info/GC682ZC}{Le beffroi du Bouffay} — \href{http://coord.info/GC682ZC\Number{}802993608}{2078}}\cacheData{{2018/09/14 Tom\Underscore{}44, Traditional Cache (1.5/1.5)}}\begin{cacheText}Sur Nantes pour participer au Mega Géocoinfest accompagnés des Fabilab, nous en profitons pour faire quelques caches dans la ville.

Cet après-midi nous avons rendez-vous avec la ligue des gentlemen pour mener l’enquête. C’est l’heure de notre passage sur la place du Bouffay et malgré notre concentration, nous repérons cette tradi. Nous en profitons pour loguer cette cache sous le pseudo Team Atlantic (Fleura 64, Fabilab, Socolau, abers29). Merci Pour la découverte de cette superbe façade.\end{cacheText}

\cacheNumber{2079}\needspace{5\baselineskip}\cacheName{\href{http://coord.info/GC6WB9Z}{Moustache !} — \href{http://coord.info/GC6WB9Z\Number{}803026494}{2079}}\cacheData{{2018/09/14 ironAss, Unknown Cache (2/1.5)}}\begin{cacheText}Sur Nantes pour participer au géocoinfest nous en profitons pour loguer quelques caches. Cette énigme est résolue de longue date… depuis que nous avons pris la décision de venir!!!!C’est au retour de la rencontre avec le capitaine Nemo que nous venons déloger la Belle. Une super cache parfaitement intégrée. Merci\end{cacheText}

\cacheNumber{2080}\needspace{5\baselineskip}\cacheName{\href{http://coord.info/GC6ZRY2}{Square Maurice Schwob} — \href{http://coord.info/GC6ZRY2\Number{}803021129}{2080}}\cacheData{{2018/09/14 ironAss, Traditional Cache (1.5/1)}}\begin{cacheText}Sur Nantes pour participer au Mega Géocoinfest avec les Fabilab, nous en profitons pour faire quelques caches dans la ville.

Cet après-midi nous avons rendez-vous avec la ligue des gentlemen pour mener l’enquête. C’est à l'issue de notre quête du parchemin que nous nous rendons au GCF18 et en profitons pour trouver d'autres caches. Le PZ est envahi de géocacheurs...nous signons Team Atlantic et repartons. Merci pour la cache\end{cacheText}

\cacheNumber{2081}\needspace{5\baselineskip}\cacheName{\href{http://coord.info/GC70N16}{Église Sainte-Anne ⛪} — \href{http://coord.info/GC70N16\Number{}803020603}{2081}}\cacheData{{2018/09/14 ironAss, Traditional Cache (1/1)}}\begin{cacheText}Sur Nantes pour participer au Mega Géocoinfest avec les Fabilab, nous en profitons pour faire quelques caches dans la ville.

Cet après-midi nous avons rendez-vous avec la ligue des gentlemen pour mener l’enquête. C’est à l'issue de notre quête du parchemin que nous nous rendons au GCF18 et en profitons pour trouver d'autres caches. Le PZ est envahi de géocacheurs...nous signons Team Atlantic et repartons. Merci pour la cache\end{cacheText}

\cacheNumber{2082}\needspace{5\baselineskip}\cacheName{\href{http://coord.info/GC746N3}{[Nantes] cours Cambronne} — \href{http://coord.info/GC746N3\Number{}802509767}{2082}}\cacheData{{2018/09/14 Kitou\And{}Laulo44, Multi-cache (2/1.5)}}\begin{cacheText}Sur Nantes pour assister au méga, nous passons la journée avec Fabilab pour loguer quelques caches et mettre quelques points jaunes sur la carte. Nous commençons par cette Multi qui nous fait découvrir le cour Cambronne : magnifique parc. La Belle est délogée discrètement. Merci pour la cache.\end{cacheText}

\cacheNumber{2083}\needspace{5\baselineskip}\cacheName{\href{http://coord.info/GC764EN}{Entre le Quai des Tanneurs et le Marchix} — \href{http://coord.info/GC764EN\Number{}802991418}{2083}}\cacheData{{2018/09/14 Magimax, Traditional Cache (1/1)}}\begin{cacheText}Toujours en compagnie des Fabilab nous continuons la visite de Nantes au travers des caches.Celle ci nous permet de découvrir un petit passage caché et une superbe façade de maison ancienne. Le logo ok est signé sous le pseudo Team Atlantic.Merci pour la cache.\end{cacheText}

\cacheNumber{2084}\needspace{5\baselineskip}\cacheName{\href{http://coord.info/GC7711H}{Hôtel de Préfecture} — \href{http://coord.info/GC7711H\Number{}802992427}{2084}}\cacheData{{2018/09/14 ironAss, Traditional Cache (1.5/1.5)}}\begin{cacheText}La Team Atlantic ( Fabilab et Fleura64) se rapproche de la cathédrale afin d’aller manger pour être à l’heure pour participer à La Ligue des Gentlemen . Nous faisons cette petite en passant : pas de difficulté :l´indice est très explicite. Merci pour la cache.\end{cacheText}

\cacheNumber{2085}\needspace{5\baselineskip}\cacheName{\href{http://coord.info/GC771M7}{La maison de St Nicolas} — \href{http://coord.info/GC771M7\Number{}802644621}{2085}}\cacheData{{2018/09/14 ironAss, Wherigo Cache (2/1.5)}}\begin{cacheText}La Wherigo a été résolue à la maison avant de prendre la route pour Nantes. Nous avons bien rigolé !!!C’est en compagnie de nos amis les Fabilab que nous découvrons la cache et signons sous le pseudo Team Atlantic. Merci pour la cache.\end{cacheText}

\cacheNumber{2086}\needspace{5\baselineskip}\cacheName{\href{http://coord.info/GC7B9VV}{Le Nid 🥚} — \href{http://coord.info/GC7B9VV\Number{}802995476}{2086}}\cacheData{{2018/09/14 Ethnos, Virtual Cache (1.5/1)}}\begin{cacheText}Sur Nantes pour participer au Mega Géocoinfest avec les Fabilab, nous en profitons pour faire quelques caches dans la ville.

Cet après-midi nous avons rendez-vous avec la ligue des gentlemen pour mener l’enquête. C’est à l'issue de notre quête du parchemin que nous nous rendons à la tour de Bretagne et j'avoue que sans la Virtuelle je n'y serais jamais monté. La haut, la vue sur Nantes est exceptionnelle et la Team Atlantic (Fleura 64, Fabilab, Socolau et abers29) en profite pour se désaltérer .Merci pour la découverte de ce superbe panorama et un PF évidemment.\end{cacheText}

\cacheNumber{2087}\needspace{5\baselineskip}\cacheName{\href{http://coord.info/GC7BFRP}{GCF18 - Rencontre avec le Capitaine Nemo} — \href{http://coord.info/GC7BFRP\Number{}803025623}{2087}}\cacheData{{2018/09/14 OrgaGCF18, Event Cache (1/1.5)}}\begin{cacheText}Cet après-midi, avec les Fabilab nous avons rendez-vous avec la Ligue des Gentlemen pour mener l’enquête  sur la disparition du parchemin de Gilles de Rais. Aidés des sympathiques abers29 et socolau nous réussissons à résoudre l'énigme avec brio dans une super ambiance.C’est à l'issue de notre quête et ensemble que nous nous rendons au GCF18 et en profitons pour trouver d'autres caches au passage.

Beaucoup de monde pour ce deuxième Event, le temps de dire bonjour et il est l'heure de faire la photo avant de partir se restaurer avec d'excellents foués et de se désaltérer avec une excellente bière artisanale.

Merci à la Team Organisatrice pour ce super moment\end{cacheText}

\cacheNumber{2088}\needspace{5\baselineskip}\cacheName{\href{http://coord.info/GC7HB7B}{Bon Anniversaire Jules Verne !} — \href{http://coord.info/GC7HB7B\Number{}803019933}{2088}}\cacheData{{2018/09/14 fafahakkai, Traditional Cache (2.5/1)}}\begin{cacheText}Sur Nantes pour participer au Mega Géocoinfest avec les Fabilab, nous en profitons pour faire quelques caches dans la ville.

Cet après-midi nous avons rendez-vous avec la ligue des gentlemen pour mener l’enquête. C’est à l'issue de notre quête du parchemin que nous nous dirigeons vers GCF18 Rencontre avec le capitaine Némo .Arrivés au PZ, la Team Atlantic (Fabilab ,Fleura64) rejoint un grand nombre de géocacheurs qui se désole….la cache est vandalisée!!!!Quelle honte!!!Aucun respect pour ce beau travail. Merci\end{cacheText}

\cacheNumber{2089}\needspace{5\baselineskip}\cacheName{\href{http://coord.info/GC3YKN5}{Grue Grise} — \href{http://coord.info/GC3YKN5\Number{}803075034}{2089}}\cacheData{{2018/09/15 lomobéré, Traditional Cache (1.5/1.5)}}\begin{cacheText}La cache ,située à côté d’une Labcache, attire tous les géocacheurs du méga. La Belle est trouvée sans difficulté et est loguée Team Atlantic.Merci pour la cache\end{cacheText}

\cacheNumber{2090}\needspace{5\baselineskip}\cacheName{\href{http://coord.info/GC3YKP2}{Anneaux de Buren} — \href{http://coord.info/GC3YKP2\Number{}803074700}{2090}}\cacheData{{2018/09/15 Lomobéré, Traditional Cache (1.5/1.5)}}\begin{cacheText}C’est en sortant de la cantine que nous arrêtons en compagnie des Fabilab pour loguer la Belle :pas de difficulté. Direction les Labcaches du méga....Merci pour la cache\end{cacheText}

\cacheNumber{2091}\needspace{5\baselineskip}\cacheName{\href{http://coord.info/GC3YKPE}{Hangar 32} — \href{http://coord.info/GC3YKPE\Number{}803073488}{2091}}\cacheData{{2018/09/15 Lomobéré, Traditional Cache (1.5/1.5)}}\begin{cacheText}Après avoir testé une bonne partie des labcaches du méga geocoinfest, nous nous dirigeons en compagnie de Fabilab vers la cantine. C’est en passant que nous loguons la belle sous le pseudo Team Atlantic. Merci pour la cache\end{cacheText}

\cacheNumber{2092}\needspace{5\baselineskip}\cacheName{\href{http://coord.info/GC52RA5}{L'Eléphant} — \href{http://coord.info/GC52RA5\Number{}803030094}{2092}}\cacheData{{2018/09/15 fabeva44, Traditional Cache (1/1)}}\begin{cacheText}Aujourd’hui c’est le grand jour : le Méga géocoinfest!!! Mais avant de démarrer les hostilités ,nous avons pris rendez-vous avec un Greeter pour continuer la visite de Nantes sous un autre angle. Nous en profitons avec les Fabilab pour faire quelques caches supplémentaires. L'indice nous mène tout droit à la cache et signons sous le pseudo Team Atlantic.Merci pour la cache.\end{cacheText}

\cacheNumber{2093}\needspace{5\baselineskip}\cacheName{\href{http://coord.info/GC6321A}{Blanc sur blanc} — \href{http://coord.info/GC6321A\Number{}803105971}{2093}}\cacheData{{2018/09/15 ironAss, Unknown Cache (1.5/1.5)}}\begin{cacheText}L’énigme est décodée depuis le jour où nous avons décidé de venir participer au méga geocoinfest de Nantes. Et c’est avec un grand plaisir que nous venons déloger la belle.Il n’y a pas trop de piétons ce qui nous permet de chercher tranquillement. Merci pour la découverte de ce quartier.\end{cacheText}

\cacheNumber{2094}\needspace{5\baselineskip}\cacheName{\href{http://coord.info/GC69NA4}{Le marché de Talensac} — \href{http://coord.info/GC69NA4\Number{}803027080}{2094}}\cacheData{{2018/09/15 Gokyo44, Traditional Cache (1.5/1.5)}}\begin{cacheText}Aujourd’hui c’est le grand jour : le Méga géocoinfest!!! Mais avant de démarrer les hostilités ,nous avons pris rendez-vous avec un Greeter pour continuer la visite de Nantes sous un autre angle. Stéphane qui nous a donné rendez vous à la Cigale, nous guide à travers la ville et nous en profitons pour faire quelques caches en passant. Aujourd’hui c’est le jour du marché mais nous arrivons tout de même à déloger la belle discrètement derrière le commerçant. Merci pour la découverte de ce haut-lieu de Nantes.\end{cacheText}

\cacheNumber{2095}\needspace{5\baselineskip}\cacheName{\href{http://coord.info/GC6Z2ZZ}{Les Machines de L'île} — \href{http://coord.info/GC6Z2ZZ\Number{}803027623}{2095}}\cacheData{{2018/09/15 Amasix, Traditional Cache (1.5/1)}}\begin{cacheText}Aujourd’hui c’est le grand jour : le Méga géocoinfest!!! Mais avant de démarrer les hostilités ,nous avons pris rendez-vous avec un Greeter pour continuer la visite de Nantes sous un autre angle. Ici la Belle est vite délogée.Nous signons avec les Fabilab Team Atlantic.Merci pour la cache\end{cacheText}

\cacheNumber{2096}\needspace{5\baselineskip}\cacheName{\href{http://coord.info/GC6Z9H8}{La cache du GAMER} — \href{http://coord.info/GC6Z9H8\Number{}803086513}{2096}}\cacheData{{2018/09/15 Amasix, Traditional Cache (1.5/1.5)}}\begin{cacheText}C’est lors du méga Géocoinfest de Nantes que nous découvrons la petite bien camouflée. C’est un ballet incessant de géocacheurs… Merci pour la cache.\end{cacheText}

\cacheNumber{2097}\needspace{5\baselineskip}\cacheName{\href{http://coord.info/GC7B8CB}{Miroir, mon beau Miroir...} — \href{http://coord.info/GC7B8CB\Number{}803095362}{2097}}\cacheData{{2018/09/15 fab\Underscore{}seeker, Virtual Cache (1/1)}}\begin{cacheText}Comme beaucoup, nous profitons du méga Géocoinfest de Nantes pour loguer des caches supplémentaires. Avec cette virtuelle nous découvrons ici un magnifique miroir d’eau qui est très fréquenté en ce jour de forte chaleur. Il est aussi chouette que celui de Bordeaux. Le château se reflète dans l'eau, c'est très beau. Le palindrome est envoyé en MP. Merci pour la cache.\end{cacheText}

\cacheNumber{2098}\needspace{5\baselineskip}\cacheName{\href{http://coord.info/GC7BFRB}{GeoCoinFest Europe 2018 - Nantes} — \href{http://coord.info/GC7BFRB\Number{}803035304}{2098}}\cacheData{{2018/09/15 OrgaGCF18, Mega-Event Cache (1/1)}}\begin{cacheText}Cela fait des mois que nous attendons cette date avec grande impatience. Enfin le jour J est arrivé.... Le lieu, magique, se prête parfaitement au géocoinfest. Les labcaches ,géniales,sont toutes résolues plus ou moins facilement en compagnie de nos amis Fabilab. Nous admirons des géocoins magnifiques et n’oublions pas de signer avec notre ruban la superbe montgolfière !!!! Pour immortaliser le moment , nous nous arrêtons au stand du photographe et posons avec les accessoires d’époque Jules Verne !!! Et que dire de toutes ces rencontres??? Un pur bonheur !!! Cerise sur le gâteau..... la rencontre avec Signal notre mascotte préférée !!!

Un million de merci à l'équipe du Géocoinfest et aux nombreux bénévoles qui ont fait que cette journée restera longtemps gravée dans nos esprits.\end{cacheText}

\cacheNumber{2099}\needspace{5\baselineskip}\cacheName{\href{http://coord.info/GC7CZGY}{20 000 lieux sous la Loire} — \href{http://coord.info/GC7CZGY\Number{}803081396}{2099}}\cacheData{{2018/09/15 fafahakkai, Traditional Cache (3.5/1.5)}}\begin{cacheText}Arrivés avec les Fabilab près du PZ ,nous découvrons une team en pleine maintenance. Le logbook a disparu!!! Grace à leur aide nous loguons Team Atlanric. Merci pour la cache.\end{cacheText}

\cacheNumber{2100}\needspace{5\baselineskip}\cacheName{\href{http://coord.info/GC7D7A1}{L'Art Urbain dans les yeux de Mathilde} — \href{http://coord.info/GC7D7A1\Number{}803079314}{2100}}\cacheData{{2018/09/15 botrésor, Traditional Cache (2/1.5)}}\begin{cacheText}Nous commençons à peine les recherches lorsqu'arrive un flot de géocacheurs!!!Dans une super ambiance la quête va bon train et la Belle ne nous résiste pas. Nous signons .Team Évent 1530;Merci pour la découverte de ce beau graff.\end{cacheText}

\cacheNumber{2101}\needspace{5\baselineskip}\cacheName{\href{http://coord.info/GC7D7GJ}{L'Amour de l'Art dans les yeux de Mathilde} — \href{http://coord.info/GC7D7GJ\Number{}803111789}{2101}}\cacheData{{2018/09/15 botrésor, Traditional Cache (1.5/1.5)}}\begin{cacheText}C'est la dernière avant le rendez vous des addicts de Facebook. Pendant que monsieur attend dans la voiture, je fonce vers le paillasson et je découvre la Belle. Encore une belle fresque.Merci pour la cache.\end{cacheText}

\cacheNumber{2102}\needspace{5\baselineskip}\cacheName{\href{http://coord.info/GC7D7K6}{L'Art et les animaux vu des yeux de Mathilde} — \href{http://coord.info/GC7D7K6\Number{}803110388}{2102}}\cacheData{{2018/09/15 botrésor, Traditional Cache (2/1.5)}}\begin{cacheText}Il est bientôt l'heure de rejoindre le groupe Facebook des addicts mais je ne peux pas faire sans m'arrêter!!!L'endroit est un peu craignos et vraiment pas très propre. Dommage car le graff est vraiment sympa. Malgré la présence des manouches à proximité, je cherche et je trouve.Merci pour la cache.\end{cacheText}

\cacheNumber{2103}\needspace{5\baselineskip}\cacheName{\href{http://coord.info/GC7E8EF}{Manny - immeuble Coupechoux} — \href{http://coord.info/GC7E8EF\Number{}803083359}{2103}}\cacheData{{2018/09/15 djgoub, Traditional Cache (2/1.5)}}\begin{cacheText}C'est lors du Méga Géocoinfest que nous découvrons la Belle. Contre toute attente il n'y a pas de géocacheurs qui rodent!!!Juste quelques moldus qui ne nous prêtent pas attention. Le spoiler nous oriente vers la cache qui ne contient pas de TB.Merci pour la cache.\end{cacheText}

\cacheNumber{2104}\needspace{5\baselineskip}\cacheName{\href{http://coord.info/GC42NH5}{D091 - Géodyssée 33 - La libération} — \href{http://coord.info/GC42NH5\Number{}803325472}{2104}}\cacheData{{2018/09/17 Calimero33, Traditional Cache (2/2.5)}}\begin{cacheText}C’est sur le retour du héron de Certes que nous nous arrêtons pour chercher la boite. Grâce à l’indice ,la team 640 (Fleura 64, Dune33, et Isa Asia) logue la belle. Merci pour la cache.\end{cacheText}

\cacheNumber{2105}\needspace{5\baselineskip}\cacheName{\href{http://coord.info/GC65V75}{REALLY SideTracked - Lanton by bob} — \href{http://coord.info/GC65V75\Number{}803324528}{2105}}\cacheData{{2018/09/17 2433, Traditional Cache (2/1.5)}}\begin{cacheText}C’est sur la fin du circuit du héron que la team 640 (Fleura 64 Isa Asia et Dune33 ) fait un détour pour loguer la belle. Les coordonnées et l'indice nous mènent tout droit à la cache. Merci pour la découverte de l’ancienne gare.\end{cacheText}

\cacheNumber{2106}\needspace{5\baselineskip}\cacheName{\href{http://coord.info/GC66HYN}{Covoiturage Castets/Carpooling Castets} — \href{http://coord.info/GC66HYN\Number{}803119447}{2106}}\cacheData{{2018/09/17 DorisBear, Traditional Cache (2/1.5)}}\begin{cacheText}Aujourd’hui c’est le héron qui nous attend !!! Le rendez-vous est pris avec Dune33 à l’aire de covoiturage de Castets: quelle chance une cache nous attend!!!Et pas des moindres , une de nos ours préférés.Nous trouvons ici encore du très beau travail ,quelle belle réussite!!!Un grand merci pour cette cache. Et un PF évidemment.\end{cacheText}

\cacheNumber{2107}\needspace{5\baselineskip}\cacheName{\href{http://coord.info/GC7BJH5}{Eglise de Lanton by bob} — \href{http://coord.info/GC7BJH5\Number{}803323570}{2107}}\cacheData{{2018/09/17 2433, Traditional Cache (1.5/1.5)}}\begin{cacheText}La cache est découverte pendant le circuit du héron. Nous loguons Team 640.Ici nous découvrons une superbe petite église. Merci pour la cache.\end{cacheText}

\cacheNumber{2108}\needspace{5\baselineskip}\cacheName{\href{http://coord.info/GC7JRK2}{1- Le héron de Certes\Number{}A l'origine} — \href{http://coord.info/GC7JRK2\Number{}803311488}{2108}}\cacheData{{2018/09/17 Elsadodo49, Unknown Cache (1.5/1.5)}}\begin{cacheText}Il y a longtemps que le héron me tente (je n’avais pas pu, malheureusement , participer à l’ Évent). Il m’a fallu du temps mais avec de la persévérance et un peu d’aide, j’ai obtenu toutes les coordonnées!!! Aujourd’hui, en compagnie des Dune33 et Isa Asia,nous prenons la direction d’Audenge pour découvrir ce magnifique domaine de Certes. Les paysages sont superbes et le soleil est de la partie  (un peu trop même!!!) . Nous signons Team 640 (64 pour Isa asia et moi et 40 pour Dune33) lorsque nous mettons la main sur la Belle. Un grand merci pour ce superbe circuit et toutes ces énigmes enrichissantes.\end{cacheText}

\cacheNumber{2109}\needspace{5\baselineskip}\cacheName{\href{http://coord.info/GC7JRM7}{2- Le héron de Certes\Number{}Les propriétaires} — \href{http://coord.info/GC7JRM7\Number{}803311706}{2109}}\cacheData{{2018/09/17 Elsadodo49, Unknown Cache (3.5/1.5)}}\begin{cacheText}Il y a longtemps que le héron me tente (je n’avais pas pu, malheureusement , participer à l’ Évent). Il m’a fallu du temps mais avec de la persévérance et un peu d’aide, j’ai obtenu toutes les coordonnées!!! Aujourd’hui, en compagnie des Dune33 et Isa Asia,nous prenons la direction d’Audenge pour découvrir ce magnifique domaine de Certes. Les paysages sont superbes et le soleil est de la partie  (un peu trop même!!!) . Nous signons Team 640 (64 pour Isa asia et moi et 40 pour Dune33) lorsque nous mettons la main sur la Belle. Un grand merci pour ce superbe circuit et toutes ces énigmes enrichissantes.\end{cacheText}

\cacheNumber{2110}\needspace{5\baselineskip}\cacheName{\href{http://coord.info/GC7JRN0}{3- Le héron de Certes\Number{}Carte postale} — \href{http://coord.info/GC7JRN0\Number{}803312065}{2110}}\cacheData{{2018/09/17 Elsadodo49, Unknown Cache (1.5/1.5)}}\begin{cacheText}Il y a longtemps que le héron me tente (je n’avais pas pu, malheureusement , participer à l’ Évent). Il m’a fallu du temps mais avec de la persévérance et un peu d’aide, j’ai obtenu toutes les coordonnées!!! Aujourd’hui, en compagnie des Dune33 et Isa Asia,nous prenons la direction d’Audenge pour découvrir ce magnifique domaine de Certes. Les paysages sont superbes et le soleil est de la partie  (un peu trop même!!!) . Nous signons Team 640 (64 pour Isa asia et moi et 40 pour Dune33) lorsque nous mettons la main sur la Belle. Un grand merci pour ce superbe circuit et toutes ces énigmes enrichissantes.\end{cacheText}

\cacheNumber{2111}\needspace{5\baselineskip}\cacheName{\href{http://coord.info/GC7KENP}{4- Le héron de Certes\Number{}Protections} — \href{http://coord.info/GC7KENP\Number{}803312684}{2111}}\cacheData{{2018/09/17 Elsadodo49, Unknown Cache (2.5/1.5)}}\begin{cacheText}Il y a longtemps que le héron me tente (je n’avais pas pu, malheureusement , participer à l’ Évent). Il m’a fallu du temps mais avec de la persévérance et un peu d’aide, j’ai obtenu toutes les coordonnées!!! Aujourd’hui, en compagnie des Dune33 et Isa Asia,nous prenons la direction d’Audenge pour découvrir ce magnifique domaine de Certes. Les paysages sont superbes et le soleil est de la partie  (un peu trop même!!!) . Nous signons Team 640 (64 pour Isa asia et moi et 40 pour Dune33) lorsque nous mettons la main sur la Belle. Un grand merci pour ce superbe circuit et toutes ces énigmes enrichissantes.\end{cacheText}

\cacheNumber{2112}\needspace{5\baselineskip}\cacheName{\href{http://coord.info/GC7KEQ8}{6-Le héron de Certes\Number{}Carte de Masse} — \href{http://coord.info/GC7KEQ8\Number{}803313297}{2112}}\cacheData{{2018/09/17 Elsadodo49, Unknown Cache (2/1.5)}}\begin{cacheText}Il y a longtemps que le héron me tente (je n’avais pas pu, malheureusement , participer à l’ Évent). Il m’a fallu du temps mais avec de la persévérance et un peu d’aide, j’ai obtenu toutes les coordonnées!!! Aujourd’hui, en compagnie des Dune33 et Isa Asia,nous prenons la direction d’Audenge pour découvrir ce magnifique domaine de Certes. Les paysages sont superbes et le soleil est de la partie  (un peu trop même!!!) . Nous signons Team 640 (64 pour Isa asia et moi et 40 pour Dune33) lorsque nous mettons la main sur la Belle. Un grand merci pour ce superbe circuit et toutes ces énigmes enrichissantes.\end{cacheText}

\cacheNumber{2113}\needspace{5\baselineskip}\cacheName{\href{http://coord.info/GC7KEQF}{7-Le héron de Certes\Number{}Signification} — \href{http://coord.info/GC7KEQF\Number{}803313771}{2113}}\cacheData{{2018/09/17 Elsadodo49, Unknown Cache (2.5/1.5)}}\begin{cacheText}Il y a longtemps que le héron me tente (je n’avais pas pu, malheureusement , participer à l’ Évent). Il m’a fallu du temps mais avec de la persévérance et un peu d’aide, j’ai obtenu toutes les coordonnées!!! Aujourd’hui, en compagnie des Dune33 et Isa Asia,nous prenons la direction d’Audenge pour découvrir ce magnifique domaine de Certes. Les paysages sont superbes et le soleil est de la partie  (un peu trop même!!!) . Nous signons Team 640 (64 pour Isa asia et moi et 40 pour Dune33) lorsque nous mettons la main sur la Belle. Un grand merci pour ce superbe circuit et toutes ces énigmes enrichissantes.\end{cacheText}

\cacheNumber{2114}\needspace{5\baselineskip}\cacheName{\href{http://coord.info/GC7KEQM}{8-Le héron de Certes\Number{}Le château} — \href{http://coord.info/GC7KEQM\Number{}803316853}{2114}}\cacheData{{2018/09/17 Elsadodo49, Unknown Cache (2/2)}}\begin{cacheText}Il y a longtemps que le héron me tente (je n’avais pas pu, malheureusement , participer à l’ Évent). Il m’a fallu du temps mais avec de la persévérance et un peu d’aide, j’ai obtenu toutes les coordonnées!!! Aujourd’hui, en compagnie des Dune33 et Isa Asia,nous prenons la direction d’Audenge pour découvrir ce magnifique domaine de Certes. Les paysages sont superbes et le soleil est de la partie  (un peu trop même!!!) . Nous signons Team 640 (64 pour Isa asia et moi et 40 pour Dune33) lorsque nous mettons la main sur la Belle. Un grand merci pour ce superbe circuit et toutes ces énigmes enrichissantes.\end{cacheText}

\cacheNumber{2115}\needspace{5\baselineskip}\cacheName{\href{http://coord.info/GC7KERH}{9-Le héron de Certes\Number{}Les salines} — \href{http://coord.info/GC7KERH\Number{}803317015}{2115}}\cacheData{{2018/09/17 Elsadodo49, Unknown Cache (2.5/2.5)}}\begin{cacheText}Il y a longtemps que le héron me tente (je n’avais pas pu, malheureusement , participer à l’ Évent). Il m’a fallu du temps mais avec de la persévérance et un peu d’aide, j’ai obtenu toutes les coordonnées!!! Aujourd’hui, en compagnie des Dune33 et Isa Asia,nous prenons la direction d’Audenge pour découvrir ce magnifique domaine de Certes. Les paysages sont superbes et le soleil est de la partie  (un peu trop même!!!) . Nous signons Team 640 (64 pour Isa asia et moi et 40 pour Dune33) lorsque nous mettons la main sur la Belle. Un grand merci pour ce superbe circuit et toutes ces énigmes enrichissantes.\end{cacheText}

\cacheNumber{2116}\needspace{5\baselineskip}\cacheName{\href{http://coord.info/GC7KG7B}{10-Le héron de Certes\Number{}Composition} — \href{http://coord.info/GC7KG7B\Number{}803317307}{2116}}\cacheData{{2018/09/17 Elsadodo49, Unknown Cache (2/1.5)}}\begin{cacheText}Il y a longtemps que le héron me tente (je n’avais pas pu, malheureusement , participer à l’ Évent). Il m’a fallu du temps mais avec de la persévérance et un peu d’aide, j’ai obtenu toutes les coordonnées!!! Aujourd’hui, en compagnie des Dune33 et Isa Asia,nous prenons la direction d’Audenge pour découvrir ce magnifique domaine de Certes. Les paysages sont superbes et le soleil est de la partie  (un peu trop même!!!) . Nous signons Team 640 (64 pour Isa asia et moi et 40 pour Dune33) lorsque nous mettons la main sur la Belle. Un grand merci pour ce superbe circuit et toutes ces énigmes enrichissantes.\end{cacheText}

\cacheNumber{2117}\needspace{5\baselineskip}\cacheName{\href{http://coord.info/GC7KG8A}{11-Le héron de Certes\Number{}Les oiseaux} — \href{http://coord.info/GC7KG8A\Number{}803317562}{2117}}\cacheData{{2018/09/17 Elsadodo49, Unknown Cache (2.5/2)}}\begin{cacheText}Il y a longtemps que le héron me tente (je n’avais pas pu, malheureusement , participer à l’ Évent). Il m’a fallu du temps mais avec de la persévérance et un peu d’aide, j’ai obtenu toutes les coordonnées!!! Aujourd’hui, en compagnie des Dune33 et Isa Asia,nous prenons la direction d’Audenge pour découvrir ce magnifique domaine de Certes. Les paysages sont superbes et le soleil est de la partie  (un peu trop même!!!) . Nous signons Team 640 (64 pour Isa asia et moi et 40 pour Dune33) lorsque nous mettons la main sur la Belle. Un grand merci pour ce superbe circuit et toutes ces énigmes enrichissantes.\end{cacheText}

\cacheNumber{2118}\needspace{5\baselineskip}\cacheName{\href{http://coord.info/GC7KG8F}{12-Le héron de Certes\Number{}L'oiseau blanc} — \href{http://coord.info/GC7KG8F\Number{}803317772}{2118}}\cacheData{{2018/09/17 Elsadodo49, Unknown Cache (3.5/1.5)}}\begin{cacheText}Il y a longtemps que le héron me tente (je n’avais pas pu, malheureusement , participer à l’ Évent). Il m’a fallu du temps mais avec de la persévérance et un peu d’aide, j’ai obtenu toutes les coordonnées!!! Aujourd’hui, en compagnie des Dune33 et Isa Asia,nous prenons la direction d’Audenge pour découvrir ce magnifique domaine de Certes. Les paysages sont superbes et le soleil est de la partie  (un peu trop même!!!) . Nous signons Team 640 (64 pour Isa asia et moi et 40 pour Dune33) lorsque nous mettons la main sur la Belle. Un grand merci pour ce superbe circuit et toutes ces énigmes enrichissantes.\end{cacheText}

\cacheNumber{2119}\needspace{5\baselineskip}\cacheName{\href{http://coord.info/GC7KG94}{13-Le héron de Certes\Number{}La flore} — \href{http://coord.info/GC7KG94\Number{}803317917}{2119}}\cacheData{{2018/09/17 Elsadodo49, Unknown Cache (2.5/2)}}\begin{cacheText}Il y a longtemps que le héron me tente (je n’avais pas pu, malheureusement , participer à l’ Évent). Il m’a fallu du temps mais avec de la persévérance et un peu d’aide, j’ai obtenu toutes les coordonnées!!! Aujourd’hui, en compagnie des Dune33 et Isa Asia,nous prenons la direction d’Audenge pour découvrir ce magnifique domaine de Certes. Les paysages sont superbes et le soleil est de la partie  (un peu trop même!!!) . Nous signons Team 640 (64 pour Isa asia et moi et 40 pour Dune33) lorsque nous mettons la main sur la Belle. Un grand merci pour ce superbe circuit et toutes ces énigmes enrichissantes.\end{cacheText}

\cacheNumber{2120}\needspace{5\baselineskip}\cacheName{\href{http://coord.info/GC7KG9B}{14-Le héron de Certes\Number{}Charade} — \href{http://coord.info/GC7KG9B\Number{}803318078}{2120}}\cacheData{{2018/09/17 Elsadodo49, Unknown Cache (2/1.5)}}\begin{cacheText}Il y a longtemps que le héron me tente (je n’avais pas pu, malheureusement , participer à l’ Évent). Il m’a fallu du temps mais avec de la persévérance et un peu d’aide, j’ai obtenu toutes les coordonnées!!! Aujourd’hui, en compagnie des Dune33 et Isa Asia,nous prenons la direction d’Audenge pour découvrir ce magnifique domaine de Certes. Les paysages sont superbes et le soleil est de la partie  (un peu trop même!!!) . Nous signons Team 640 (64 pour Isa asia et moi et 40 pour Dune33) lorsque nous mettons la main sur la Belle. Un grand merci pour ce superbe circuit et toutes ces énigmes enrichissantes.\end{cacheText}

\cacheNumber{2121}\needspace{5\baselineskip}\cacheName{\href{http://coord.info/GC7KG9H}{15-Le héron de Certes\Number{}L'eyre} — \href{http://coord.info/GC7KG9H\Number{}803318281}{2121}}\cacheData{{2018/09/17 Elsadodo49, Unknown Cache (3.5/1.5)}}\begin{cacheText}Il y a longtemps que le héron me tente (je n’avais pas pu, malheureusement , participer à l’ Évent). Il m’a fallu du temps mais avec de la persévérance et un peu d’aide, j’ai obtenu toutes les coordonnées!!! Aujourd’hui, en compagnie des Dune33 et Isa Asia,nous prenons la direction d’Audenge pour découvrir ce magnifique domaine de Certes. Les paysages sont superbes et le soleil est de la partie  (un peu trop même!!!) . Nous signons Team 640 (64 pour Isa asia et moi et 40 pour Dune33) lorsque nous mettons la main sur la Belle. Un grand merci pour ce superbe circuit et toutes ces énigmes enrichissantes.\end{cacheText}

\cacheNumber{2122}\needspace{5\baselineskip}\cacheName{\href{http://coord.info/GC7KG9R}{16-Le héron de Certes\Number{}Inventaire} — \href{http://coord.info/GC7KG9R\Number{}803318454}{2122}}\cacheData{{2018/09/17 Elsadodo49, Unknown Cache (2/1.5)}}\begin{cacheText}Il y a longtemps que le héron me tente (je n’avais pas pu, malheureusement , participer à l’ Évent). Il m’a fallu du temps mais avec de la persévérance et un peu d’aide, j’ai obtenu toutes les coordonnées!!! Aujourd’hui, en compagnie des Dune33 et Isa Asia,nous prenons la direction d’Audenge pour découvrir ce magnifique domaine de Certes. Les paysages sont superbes et le soleil est de la partie  (un peu trop même!!!) . Nous signons Team 640 (64 pour Isa asia et moi et 40 pour Dune33) lorsque nous mettons la main sur la Belle. Un grand merci pour ce superbe circuit et toutes ces énigmes enrichissantes.\end{cacheText}

\cacheNumber{2123}\needspace{5\baselineskip}\cacheName{\href{http://coord.info/GC7KH0Q}{17-Le héron de Certes\Number{}Navigation} — \href{http://coord.info/GC7KH0Q\Number{}803318741}{2123}}\cacheData{{2018/09/17 Elsadodo49, Unknown Cache (2/1.5)}}\begin{cacheText}Il y a longtemps que le héron me tente (je n’avais pas pu, malheureusement , participer à l’ Évent). Il m’a fallu du temps mais avec de la persévérance et un peu d’aide, j’ai obtenu toutes les coordonnées!!! Aujourd’hui, en compagnie des Dune33 et Isa Asia,nous prenons la direction d’Audenge pour découvrir ce magnifique domaine de Certes. Les paysages sont superbes et le soleil est de la partie  (un peu trop même!!!) . Nous signons Team 640 (64 pour Isa asia et moi et 40 pour Dune33) lorsque nous mettons la main sur la Belle. Un grand merci pour ce superbe circuit et toutes ces énigmes enrichissantes.\end{cacheText}

\cacheNumber{2124}\needspace{5\baselineskip}\cacheName{\href{http://coord.info/GC7KH0X}{18-Le héron de Certes\Number{}\Quoted{Lieux-dits}} — \href{http://coord.info/GC7KH0X\Number{}803318949}{2124}}\cacheData{{2018/09/17 Elsadodo49, Unknown Cache (3/1.5)}}\begin{cacheText}Il y a longtemps que le héron me tente (je n’avais pas pu, malheureusement , participer à l’ Évent). Il m’a fallu du temps mais avec de la persévérance et un peu d’aide, j’ai obtenu toutes les coordonnées!!! Aujourd’hui, en compagnie des Dune33 et Isa Asia,nous prenons la direction d’Audenge pour découvrir ce magnifique domaine de Certes. Les paysages sont superbes et le soleil est de la partie  (un peu trop même!!!) . Nous signons Team 640 (64 pour Isa asia et moi et 40 pour Dune33) lorsque nous mettons la main sur la Belle. Un grand merci pour ce superbe circuit et toutes ces énigmes enrichissantes.\end{cacheText}

\cacheNumber{2125}\needspace{5\baselineskip}\cacheName{\href{http://coord.info/GC7KH0Y}{19-Le héron de Certes\Number{}La brèche} — \href{http://coord.info/GC7KH0Y\Number{}803319149}{2125}}\cacheData{{2018/09/17 Elsadodo49, Unknown Cache (3.5/1.5)}}\begin{cacheText}Il y a longtemps que le héron me tente (je n’avais pas pu, malheureusement , participer à l’ Évent). Il m’a fallu du temps mais avec de la persévérance et un peu d’aide, j’ai obtenu toutes les coordonnées!!! Aujourd’hui, en compagnie des Dune33 et Isa Asia,nous prenons la direction d’Audenge pour découvrir ce magnifique domaine de Certes. Les paysages sont superbes et le soleil est de la partie  (un peu trop même!!!) . Nous signons Team 640 (64 pour Isa asia et moi et 40 pour Dune33) lorsque nous mettons la main sur la Belle. Un grand merci pour ce superbe circuit et toutes ces énigmes enrichissantes.\end{cacheText}

\cacheNumber{2126}\needspace{5\baselineskip}\cacheName{\href{http://coord.info/GC7KH1A}{20-Le héron de Certes\Number{}Ruisseaux} — \href{http://coord.info/GC7KH1A\Number{}803319349}{2126}}\cacheData{{2018/09/17 Elsadodo49, Unknown Cache (2.5/1.5)}}\begin{cacheText}Il y a longtemps que le héron me tente (je n’avais pas pu, malheureusement , participer à l’ Évent). Il m’a fallu du temps mais avec de la persévérance et un peu d’aide, j’ai obtenu toutes les coordonnées!!! Aujourd’hui, en compagnie des Dune33 et Isa Asia,nous prenons la direction d’Audenge pour découvrir ce magnifique domaine de Certes. Les paysages sont superbes et le soleil est de la partie  (un peu trop même!!!) . Nous signons Team 640 (64 pour Isa asia et moi et 40 pour Dune33) lorsque nous mettons la main sur la Belle. Un grand merci pour ce superbe circuit et toutes ces énigmes enrichissantes.\end{cacheText}

\cacheNumber{2127}\needspace{5\baselineskip}\cacheName{\href{http://coord.info/GC7KH1Y}{21-Le héron de Certes\Number{}Toponymie} — \href{http://coord.info/GC7KH1Y\Number{}803319505}{2127}}\cacheData{{2018/09/17 Elsadodo49, Unknown Cache (2/1.5)}}\begin{cacheText}Il y a longtemps que le héron me tente (je n’avais pas pu, malheureusement , participer à l’ Évent). Il m’a fallu du temps mais avec de la persévérance et un peu d’aide, j’ai obtenu toutes les coordonnées!!! Aujourd’hui, en compagnie des Dune33 et Isa Asia,nous prenons la direction d’Audenge pour découvrir ce magnifique domaine de Certes. Les paysages sont superbes et le soleil est de la partie  (un peu trop même!!!) . Nous signons Team 640 (64 pour Isa asia et moi et 40 pour Dune33) lorsque nous mettons la main sur la Belle. Un grand merci pour ce superbe circuit et toutes ces énigmes enrichissantes.\end{cacheText}

\cacheNumber{2128}\needspace{5\baselineskip}\cacheName{\href{http://coord.info/GC7KH21}{22-Le héron de Certes\Number{}abréviation} — \href{http://coord.info/GC7KH21\Number{}803319655}{2128}}\cacheData{{2018/09/17 Elsadodo49, Unknown Cache (2.5/1.5)}}\begin{cacheText}Il y a longtemps que le héron me tente (je n’avais pas pu, malheureusement , participer à l’ Évent). Il m’a fallu du temps mais avec de la persévérance et un peu d’aide, j’ai obtenu toutes les coordonnées!!! Aujourd’hui, en compagnie des Dune33 et Isa Asia,nous prenons la direction d’Audenge pour découvrir ce magnifique domaine de Certes. Les paysages sont superbes et le soleil est de la partie  (un peu trop même!!!) . Nous signons Team 640 (64 pour Isa asia et moi et 40 pour Dune33) lorsque nous mettons la main sur la Belle. Un grand merci pour ce superbe circuit et toutes ces énigmes enrichissantes.\end{cacheText}

\cacheNumber{2129}\needspace{5\baselineskip}\cacheName{\href{http://coord.info/GC7KH25}{23-Le héron de Certes\Number{}Politique} — \href{http://coord.info/GC7KH25\Number{}803319848}{2129}}\cacheData{{2018/09/17 Elsadodo49, Unknown Cache (1.5/1.5)}}\begin{cacheText}Il y a longtemps que le héron me tente (je n’avais pas pu, malheureusement , participer à l’ Évent). Il m’a fallu du temps mais avec de la persévérance et un peu d’aide, j’ai obtenu toutes les coordonnées!!! Aujourd’hui, en compagnie des Dune33 et Isa Asia,nous prenons la direction d’Audenge pour découvrir ce magnifique domaine de Certes. Les paysages sont superbes et le soleil est de la partie  (un peu trop même!!!) . Nous signons Team 640 (64 pour Isa asia et moi et 40 pour Dune33) lorsque nous mettons la main sur la Belle. Un grand merci pour ce superbe circuit et toutes ces énigmes enrichissantes.\end{cacheText}

\cacheNumber{2130}\needspace{5\baselineskip}\cacheName{\href{http://coord.info/GC7KH2N}{24-Le héron de Certes\Number{}Qui suis-je?} — \href{http://coord.info/GC7KH2N\Number{}803319995}{2130}}\cacheData{{2018/09/17 Elsadodo49, Unknown Cache (2/1.5)}}\begin{cacheText}Il y a longtemps que le héron me tente (je n’avais pas pu, malheureusement , participer à l’ Évent). Il m’a fallu du temps mais avec de la persévérance et un peu d’aide, j’ai obtenu toutes les coordonnées!!! Aujourd’hui, en compagnie des Dune33 et Isa Asia,nous prenons la direction d’Audenge pour découvrir ce magnifique domaine de Certes. Les paysages sont superbes et le soleil est de la partie  (un peu trop même!!!) . Nous signons Team 640 (64 pour Isa asia et moi et 40 pour Dune33) lorsque nous mettons la main sur la Belle. Un grand merci pour ce superbe circuit et toutes ces énigmes enrichissantes.\end{cacheText}

\cacheNumber{2131}\needspace{5\baselineskip}\cacheName{\href{http://coord.info/GC7KH2X}{25-Le héron de Certes\Number{}Qui suis-je2 ?} — \href{http://coord.info/GC7KH2X\Number{}803321407}{2131}}\cacheData{{2018/09/17 Elsadodo49, Unknown Cache (1.5/1.5)}}\begin{cacheText}Il y a longtemps que le héron me tente (je n’avais pas pu, malheureusement , participer à l’ Évent). Il m’a fallu du temps mais avec de la persévérance et un peu d’aide, j’ai obtenu toutes les coordonnées!!! Aujourd’hui, en compagnie des Dune33 et Isa Asia,nous prenons la direction d’Audenge pour découvrir ce magnifique domaine de Certes. Les paysages sont superbes et le soleil est de la partie  (un peu trop même!!!) . Nous signons Team 640 (64 pour Isa asia et moi et 40 pour Dune33) lorsque nous mettons la main sur la Belle. Un grand merci pour ce superbe circuit et toutes ces énigmes enrichissantes.\end{cacheText}

\cacheNumber{2132}\needspace{5\baselineskip}\cacheName{\href{http://coord.info/GC7KH31}{26-Le héron de Certes\Number{}Superficie} — \href{http://coord.info/GC7KH31\Number{}803321548}{2132}}\cacheData{{2018/09/17 Elsadodo49, Unknown Cache (1.5/1.5)}}\begin{cacheText}Il y a longtemps que le héron me tente (je n’avais pas pu, malheureusement , participer à l’ Évent). Il m’a fallu du temps mais avec de la persévérance et un peu d’aide, j’ai obtenu toutes les coordonnées!!! Aujourd’hui, en compagnie des Dune33 et Isa Asia,nous prenons la direction d’Audenge pour découvrir ce magnifique domaine de Certes. Les paysages sont superbes et le soleil est de la partie  (un peu trop même!!!) . Nous signons Team 640 (64 pour Isa asia et moi et 40 pour Dune33) lorsque nous mettons la main sur la Belle. Un grand merci pour ce superbe circuit et toutes ces énigmes enrichissantes.\end{cacheText}

\cacheNumber{2133}\needspace{5\baselineskip}\cacheName{\href{http://coord.info/GC7KH3K}{27-Le héron de Certes\Number{}Critère} — \href{http://coord.info/GC7KH3K\Number{}803321763}{2133}}\cacheData{{2018/09/17 Elsadodo49, Unknown Cache (2/1.5)}}\begin{cacheText}Il y a longtemps que le héron me tente (je n’avais pas pu, malheureusement , participer à l’ Évent). Il m’a fallu du temps mais avec de la persévérance et un peu d’aide, j’ai obtenu toutes les coordonnées!!! Aujourd’hui, en compagnie des Dune33 et Isa Asia,nous prenons la direction d’Audenge pour découvrir ce magnifique domaine de Certes. Les paysages sont superbes et le soleil est de la partie  (un peu trop même!!!) . Nous signons Team 640 (64 pour Isa asia et moi et 40 pour Dune33) lorsque nous mettons la main sur la Belle. Un grand merci pour ce superbe circuit et toutes ces énigmes enrichissantes.\end{cacheText}

\cacheNumber{2134}\needspace{5\baselineskip}\cacheName{\href{http://coord.info/GC7KH3Q}{28-Le héron de Certes\Number{}Europe} — \href{http://coord.info/GC7KH3Q\Number{}803322005}{2134}}\cacheData{{2018/09/17 Elsadodo49, Unknown Cache (2/1.5)}}\begin{cacheText}Il y a longtemps que le héron me tente (je n’avais pas pu, malheureusement , participer à l’ Évent). Il m’a fallu du temps mais avec de la persévérance et un peu d’aide, j’ai obtenu toutes les coordonnées!!! Aujourd’hui, en compagnie des Dune33 et Isa Asia,nous prenons la direction d’Audenge pour découvrir ce magnifique domaine de Certes. Les paysages sont superbes et le soleil est de la partie  (un peu trop même!!!) . Nous signons Team 640 (64 pour Isa asia et moi et 40 pour Dune33) lorsque nous mettons la main sur la Belle. Un grand merci pour ce superbe circuit et toutes ces énigmes enrichissantes.\end{cacheText}

\cacheNumber{2135}\needspace{5\baselineskip}\cacheName{\href{http://coord.info/GC7KH8X}{30-Le héron de Certes\Number{}Production} — \href{http://coord.info/GC7KH8X\Number{}803322466}{2135}}\cacheData{{2018/09/17 Elsadodo49, Unknown Cache (2/1.5)}}\begin{cacheText}Il y a longtemps que le héron me tente (je n’avais pas pu, malheureusement , participer à l’ Évent). Il m’a fallu du temps mais avec de la persévérance et un peu d’aide, j’ai obtenu toutes les coordonnées!!! Aujourd’hui, en compagnie des Dune33 et Isa Asia,nous prenons la direction d’Audenge pour découvrir ce magnifique domaine de Certes. Les paysages sont superbes et le soleil est de la partie  (un peu trop même!!!) . Nous signons Team 640 (64 pour Isa asia et moi et 40 pour Dune33) lorsque nous mettons la main sur la Belle. Un grand merci pour ce superbe circuit et toutes ces énigmes enrichissantes.\end{cacheText}

\cacheNumber{2136}\needspace{5\baselineskip}\cacheName{\href{http://coord.info/GC7KH99}{31-Le héron de Certes\Number{}Pisciculture} — \href{http://coord.info/GC7KH99\Number{}803322613}{2136}}\cacheData{{2018/09/17 Elsadodo49, Unknown Cache (2/1.5)}}\begin{cacheText}Il y a longtemps que le héron me tente (je n’avais pas pu, malheureusement , participer à l’ Évent). Il m’a fallu du temps mais avec de la persévérance et un peu d’aide, j’ai obtenu toutes les coordonnées!!! Aujourd’hui, en compagnie des Dune33 et Isa Asia,nous prenons la direction d’Audenge pour découvrir ce magnifique domaine de Certes. Les paysages sont superbes et le soleil est de la partie  (un peu trop même!!!) . Nous signons Team 640 (64 pour Isa asia et moi et 40 pour Dune33) lorsque nous mettons la main sur la Belle. Un grand merci pour ce superbe circuit et toutes ces énigmes enrichissantes.\end{cacheText}

\cacheNumber{2137}\needspace{5\baselineskip}\cacheName{\href{http://coord.info/GC7KH9H}{32-Le héron de Certes\Number{}Agriculture} — \href{http://coord.info/GC7KH9H\Number{}803322862}{2137}}\cacheData{{2018/09/17 Elsadodo49, Unknown Cache (2/1.5)}}\begin{cacheText}Il y a longtemps que le héron me tente (je n’avais pas pu, malheureusement , participer à l’ Évent). Il m’a fallu du temps mais avec de la persévérance et un peu d’aide, j’ai obtenu toutes les coordonnées!!! Aujourd’hui, en compagnie des Dune33 et Isa Asia,nous prenons la direction d’Audenge pour découvrir ce magnifique domaine de Certes. Les paysages sont superbes et le soleil est de la partie  (un peu trop même!!!) . Nous signons Team 640 (64 pour Isa asia et moi et 40 pour Dune33) lorsque nous mettons la main sur la Belle. Un grand merci pour ce superbe circuit et toutes ces énigmes enrichissantes.\end{cacheText}

\cacheNumber{2138}\needspace{5\baselineskip}\cacheName{\href{http://coord.info/GC7KHC3}{34-Le héron de Certes\Number{}La chasse} — \href{http://coord.info/GC7KHC3\Number{}803323900}{2138}}\cacheData{{2018/09/17 Elsadodo49, Unknown Cache (1.5/1.5)}}\begin{cacheText}Il y a longtemps que le héron me tente (je n’avais pas pu, malheureusement , participer à l’ Évent). Il m’a fallu du temps mais avec de la persévérance et un peu d’aide, j’ai obtenu toutes les coordonnées!!! Aujourd’hui, en compagnie des Dune33 et Isa Asia,nous prenons la direction d’Audenge pour découvrir ce magnifique domaine de Certes. Les paysages sont superbes et le soleil est de la partie  (un peu trop même!!!) . Nous signons Team 640 (64 pour Isa asia et moi et 40 pour Dune33) lorsque nous mettons la main sur la Belle. Un grand merci pour ce superbe circuit et toutes ces énigmes enrichissantes.\end{cacheText}

\cacheNumber{2139}\needspace{5\baselineskip}\cacheName{\href{http://coord.info/GC7KHC7}{35-Le héron de Certes\Number{}Ecluses} — \href{http://coord.info/GC7KHC7\Number{}803324778}{2139}}\cacheData{{2018/09/17 Elsadodo49, Unknown Cache (2/1.5)}}\begin{cacheText}Il y a longtemps que le héron me tente (je n’avais pas pu, malheureusement , participer à l’ Évent). Il m’a fallu du temps mais avec de la persévérance et un peu d’aide, j’ai obtenu toutes les coordonnées!!! Aujourd’hui, en compagnie des Dune33 et Isa Asia,nous prenons la direction d’Audenge pour découvrir ce magnifique domaine de Certes. Les paysages sont superbes et le soleil est de la partie  (un peu trop même!!!) . Nous signons Team 640 (64 pour Isa asia et moi et 40 pour Dune33) lorsque nous mettons la main sur la Belle. Un grand merci pour ce superbe circuit et toutes ces énigmes enrichissantes.\end{cacheText}

\cacheNumber{2140}\needspace{5\baselineskip}\cacheName{\href{http://coord.info/GC7KHCC}{36-Le héron de Certes\Number{}Ecluse} — \href{http://coord.info/GC7KHCC\Number{}803325974}{2140}}\cacheData{{2018/09/17 Elsadodo49, Unknown Cache (3.5/1.5)}}\begin{cacheText}Il y a longtemps que le héron me tente (je n’avais pas pu, malheureusement , participer à l’ Évent). Il m’a fallu du temps mais avec de la persévérance et un peu d’aide, j’ai obtenu toutes les coordonnées!!! Aujourd’hui, en compagnie des Dune33 et Isa Asia,nous prenons la direction d’Audenge pour découvrir ce magnifique domaine de Certes. Les paysages sont superbes et le soleil est de la partie  (un peu trop même!!!) . Nous signons Team 640 (64 pour Isa asia et moi et 40 pour Dune33) lorsque nous mettons la main sur la Belle. Un grand merci pour ce superbe circuit et toutes ces énigmes enrichissantes.

Découverte d'un TB.

Un PF amplement mérité.\end{cacheText}

\cacheNumber{2141}\needspace{5\baselineskip}\cacheName{\href{http://coord.info/GC7MK23}{Château de Certes} — \href{http://coord.info/GC7MK23\Number{}803328224}{2141}}\cacheData{{2018/09/17 Elsadodo49, Multi-cache (2/1.5)}}\begin{cacheText}C'est au retour du héron que nous nous lançons sur cette petite multi. Toutes les informations sont sur le panneau et nous n'avons pas de soucis pour découvrir la Belle . La Team 640 est ravie de ce bon moment. Merci pour la cache.\end{cacheText}

\cacheNumber{2142}\needspace{5\baselineskip}\cacheName{\href{http://coord.info/GC4VR94}{A63 - Aire de repos de Labouheyre Est} — \href{http://coord.info/GC4VR94\Number{}806989711}{2142}}\cacheData{{2018/10/05 gilles64, Traditional Cache (1.5/1.5)}}\begin{cacheText}En route pour le GTAQ fait de la barque à Coulon, nous nous arrêtons faire une pause petit déjeuner . Dune33 et Isaasia ont pensé aux chocolatines : il serait dommage de ne pas en profiter !!! La cache est vite délogée sur cette aire de repos déserte à cette heure ci. Merci pour la cache.\end{cacheText}

\cacheNumber{2143}\needspace{5\baselineskip}\cacheName{\href{http://coord.info/GC5EB2X}{La Croix Comtesse} — \href{http://coord.info/GC5EB2X\Number{}806990356}{2143}}\cacheData{{2018/10/05 Karoller, Traditional Cache (1.5/1.5)}}\begin{cacheText}En route pour le GTAQ fait de la barque à Coulon ,la Team 640 (Isaasia, Dune33 et moi même) en profite pour faire quelques caches en chemin . Rejoint par Domino50 nous découvrons ce superbe département grâce aux différents poseurs.Les 10 yeux de lynx permettent de déloger la Belle en deux temps trois mouvements et nous laissent le loisir d admirer cette superbe église. Merci pour la cache.\end{cacheText}

\cacheNumber{2144}\needspace{5\baselineskip}\cacheName{\href{http://coord.info/GC7EJVN}{La Motte Féodale} — \href{http://coord.info/GC7EJVN\Number{}806990480}{2144}}\cacheData{{2018/10/05 Guillaume017, Traditional Cache (1.5/2)}}\begin{cacheText}En route pour le GTAQ fait de la barque à Coulon ,la Team 640 (Isaasia, Dune33 et moi même) en profite pour faire quelques caches en chemin . Rejoint par Domino50 nous découvrons ce superbe département grâce aux différents poseurs.Les 10 yeux de lynx permettent de déloger la Belle en deux temps trois mouvements et nous laissent le loisir découvrir cette motte féodale et les ruines aux alentours . Merci pour la cache.\end{cacheText}

\cacheNumber{2145}\needspace{5\baselineskip}\cacheName{\href{http://coord.info/GC6Z5DC}{Le coupe-gorge} — \href{http://coord.info/GC6Z5DC\Number{}806990731}{2145}}\cacheData{{2018/10/05 LaComtesse et Léatoon, Traditional Cache (1/1.5)}}\begin{cacheText}En route pour le GTAQ fait de la barque à Coulon ,la Team 640 (Isaasia, Dune33 et moi même) en profite pour faire quelques caches en chemin . Rejoint par Domino50 nous découvrons ce superbe département grâce aux différents poseurs.La venelle est parfaitement entretenue et il nous a fallu explorer pas mal de pierres avant de trouver la bonne!!!Au passage nous prenons une délicieuse pomme qui nous fait de l’œil.Merci pour la cache.\end{cacheText}

\cacheNumber{2146}\needspace{5\baselineskip}\cacheName{\href{http://coord.info/GC58B74}{Villeneuve La Comtesse: lettre V, alphabet du 17} — \href{http://coord.info/GC58B74\Number{}806990751}{2146}}\cacheData{{2018/10/05 lulu\Underscore{}et\Underscore{}compagnie, Traditional Cache (2/1.5)}}\begin{cacheText}En route pour le GTAQ fait de la barque à Coulon ,la Team 640 (Isaasia, Dune33 et moi même) en profite pour faire quelques caches en chemin . Rejoint par Domino50 nous découvrons ce superbe département grâce aux différents poseurs.Les 10 yeux de lynx permettent de déloger la Belle en deux temps trois mouvements et nous laissent le loisir d admirer cette superbe église. Merci pour la cache.\end{cacheText}

\cacheNumber{2147}\needspace{5\baselineskip}\cacheName{\href{http://coord.info/GC58FTK}{Commune Deux Sèvres: Eglise St-Etienne-la-Cigogne} — \href{http://coord.info/GC58FTK\Number{}806990981}{2147}}\cacheData{{2018/10/05 Karoller, Traditional Cache (1.5/2)}}\begin{cacheText}En route pour le GTAQ fait de la barque à Coulon ,la Team 640 (Isaasia, Dune33 et moi même) en profite pour faire quelques caches en chemin . Rejoint par Domino50 nous découvrons ce superbe département grâce aux différents poseurs.Notre GPS nous joue des tours et l’indice nous perturbe mais à force de persévérance nous finissons par mettre la main sur la Belle. Encore une superbe église. Merci pour la cache.\end{cacheText}

\cacheNumber{2148}\needspace{5\baselineskip}\cacheName{\href{http://coord.info/GC5FA34}{Commune des Deux-Sévres: Belleville} — \href{http://coord.info/GC5FA34\Number{}806991012}{2148}}\cacheData{{2018/10/05 sara et gege, Traditional Cache (2/1.5)}}\begin{cacheText}En route pour le GTAQ fait de la barque à Coulon ,la Team 640 (Isaasia, Dune33 et moi même) en profite pour faire quelques caches en chemin . Rejoint par Domino50 nous découvrons ce superbe département grâce aux différents poseurs.Les 10 yeux de lynx permettent de déloger la Belle en deux temps trois mouvements et nous laissent le loisir d admirer cette superbe église. Merci pour la cache.\end{cacheText}

\cacheNumber{2149}\needspace{5\baselineskip}\cacheName{\href{http://coord.info/GC72JVK}{Le pont sur \Quoted{les Alleuds}} — \href{http://coord.info/GC72JVK\Number{}806991463}{2149}}\cacheData{{2018/10/05 lulu\Underscore{}et\Underscore{}compagnie, Multi-cache (2.5/3)}}\begin{cacheText}En route pour le GTAQ fait de la barque à Coulon ,la Team 640 (Isaasia, Dune33 et moi même) en profite pour faire quelques caches en chemin . Rejoint par Domino50 nous découvrons ce superbe département grâce aux différents poseurs.Apres avoir exploré le dessous du pont ,Isaasia découvre la plaque et note les indices. Les coordonnées en poche nous découvrons la Belle dans le lit du ruisseau. Pour récompense nous mangeons quelques noix qui sont tombées.Merci pour la cache.\end{cacheText}

\cacheNumber{2150}\needspace{5\baselineskip}\cacheName{\href{http://coord.info/GC5E5AW}{Commune des deux-Sèvres: Prissé-la-Charrière} — \href{http://coord.info/GC5E5AW\Number{}806991652}{2150}}\cacheData{{2018/10/05 Karoller, Traditional Cache (1.5/2)}}\begin{cacheText}En route pour le GTAQ fait de la barque à Coulon ,la Team 640 (Isaasia, Dune33 et moi même) en profite pour faire quelques caches en chemin . Rejoint par Domino50 nous découvrons ce superbe département grâce aux différents poseurs . Nous découvrons un superbe parc agrémenté d’un très joli pont. Le spolier d’un log précédent nous aide bien à déloger la Belle.Merci pour la cache.\end{cacheText}

\cacheNumber{2151}\needspace{5\baselineskip}\cacheName{\href{http://coord.info/GC5BJZ7}{Commune des Deux-Sévres: Beauvoir sur Niort} — \href{http://coord.info/GC5BJZ7\Number{}806991836}{2151}}\cacheData{{2018/10/05 lulu\Underscore{}et\Underscore{}compagnie, Traditional Cache (1.5/1.5)}}\begin{cacheText}En route pour le GTAQ fait de la barque à Coulon ,la Team 640 (Isaasia, Dune33 et moi même) en profite pour faire quelques caches en chemin . Rejoint par Domino50 nous découvrons ce superbe département grâce aux différents poseurs.Les 10 yeux de lynx permettent de déloger la Belle en deux temps trois mouvements et nous laissent le loisir d admirer cette superbe église. Merci pour la cache.\end{cacheText}

\cacheNumber{2152}\needspace{5\baselineskip}\cacheName{\href{http://coord.info/GC1P9N4}{Moulin de Rimbault} — \href{http://coord.info/GC1P9N4\Number{}806991946}{2152}}\cacheData{{2018/10/05 Lulu\Underscore{}et\Underscore{}compagnie, Traditional Cache (2/1.5)}}\begin{cacheText}En route pour le GTAQ fait de la barque à Coulon ,la Team 640 (Isaasia, Dune33 et moi même) en profite pour faire quelques caches en chemin . Rejoint par Domino50 nous découvrons ce superbe département grâce aux différents poseurs.Les 10 yeux de lynx permettent de déloger la Belle en deux temps trois mouvements et nous laissent le loisir d admirer ce superbe moulin en parfait état . Merci pour la cache.\end{cacheText}

\cacheNumber{2153}\needspace{5\baselineskip}\cacheName{\href{http://coord.info/GC5XGX9}{(BLV) 15 Balade à Rimbault :  Sur le retour} — \href{http://coord.info/GC5XGX9\Number{}806992330}{2153}}\cacheData{{2018/10/05 lulu\Underscore{}et\Underscore{}compagnie, Traditional Cache (2/2)}}\begin{cacheText}En route pour le GTAQ fait de la barque à Coulon ,la Team 640 (Isaasia, Dune33 et moi même) en profite pour faire quelques caches en chemin .La forêt domaniale est parfaitement entretenue et la promenade fort agréable. Nous découvrons des caches travaillées mais qui sont hélas quelque peu dégradées par des chercheurs peu délicats . Du joli travail,merci pour la cache.\end{cacheText}

\cacheNumber{2154}\needspace{5\baselineskip}\cacheName{\href{http://coord.info/GC5X2ER}{(BLV) 08 Balade à Rimbault : Vers le GR} — \href{http://coord.info/GC5X2ER\Number{}806992353}{2154}}\cacheData{{2018/10/05 lulu\Underscore{}et\Underscore{}compagnie, Traditional Cache (2.5/2)}}\begin{cacheText}En route pour le GTAQ fait de la barque à Coulon ,la Team 640 (Isaasia, Dune33 et moi même) en profite pour faire quelques caches en chemin .La forêt domaniale est parfaitement entretenue et la promenade fort agréable. Nous découvrons des caches travaillées mais qui sont hélas quelque peu dégradées par des chercheurs peu délicats . Du joli travail,merci pour la cache.\end{cacheText}

\cacheNumber{2155}\needspace{5\baselineskip}\cacheName{\href{http://coord.info/GC5X2FA}{(BLV) 09 Balade à Rimbault : GR 36} — \href{http://coord.info/GC5X2FA\Number{}806992371}{2155}}\cacheData{{2018/10/06 lulu\Underscore{}et\Underscore{}compagnie, Traditional Cache (2.5/2)}}\begin{cacheText}En route pour le GTAQ fait de la barque à Coulon ,la Team 640 (Isaasia, Dune33 et moi même) en profite pour faire quelques caches en chemin .La forêt domaniale est parfaitement entretenue et la promenade fort agréable. Nous découvrons des caches travaillées mais qui sont hélas quelque peu dégradées par des chercheurs peu délicats . Du joli travail,merci pour la cache.\end{cacheText}

\cacheNumber{2156}\needspace{5\baselineskip}\cacheName{\href{http://coord.info/GC5XE89}{(BLV) 10 Balade à Rimbault :  Sur le GR 36} — \href{http://coord.info/GC5XE89\Number{}806992392}{2156}}\cacheData{{2018/10/06 lulu\Underscore{}et\Underscore{}compagnie, Traditional Cache (2/2)}}\begin{cacheText}En route pour le GTAQ fait de la barque à Coulon ,la Team 640 (Isaasia, Dune33 et moi même) en profite pour faire quelques caches en chemin .La forêt domaniale est parfaitement entretenue et la promenade fort agréable. Nous découvrons des caches travaillées mais qui sont hélas quelque peu dégradées par des chercheurs peu délicats . Du joli travail,merci pour la cache.\end{cacheText}

\cacheNumber{2157}\needspace{5\baselineskip}\cacheName{\href{http://coord.info/GC5XE8D}{(BLV) 11 Balade à Rimbault :  Garde-barrière} — \href{http://coord.info/GC5XE8D\Number{}806992421}{2157}}\cacheData{{2018/10/06 lulu\Underscore{}et\Underscore{}compagnie, Traditional Cache (2.5/2)}}\begin{cacheText}En route pour le GTAQ fait de la barque à Coulon ,la Team 640 (Isaasia, Dune33 et moi même) en profite pour faire quelques caches en chemin .La forêt domaniale est parfaitement entretenue et la promenade fort agréable. Nous découvrons des caches travaillées mais qui sont hélas quelque peu dégradées par des chercheurs peu délicats . Du joli travail,merci pour la cache.\end{cacheText}

\cacheNumber{2158}\needspace{5\baselineskip}\cacheName{\href{http://coord.info/GC5XE8M}{(BLV) 12 Balade à Rimbault :  Le majestueux} — \href{http://coord.info/GC5XE8M\Number{}806992674}{2158}}\cacheData{{2018/10/06 lulu\Underscore{}et\Underscore{}compagnie, Letterbox Hybrid (2/2)}}\begin{cacheText}En route pour le GTAQ fait de la barque à Coulon ,la Team 640 (Isaasia, Dune33 et moi même) en profite pour faire quelques caches en chemin .La forêt domaniale est parfaitement entretenue et la promenade fort agréable. Nous découvrons des caches travaillées mais qui sont hélas quelque peu dégradées par des chercheurs peu délicats . Le tampon a disparu !!! J’ai oublié les cartes postales mais je laisse un petit mot sur une feuille libre. L’arbre est juste magnifique.Du joli travail,merci pour la cache.\end{cacheText}

\cacheNumber{2159}\needspace{5\baselineskip}\cacheName{\href{http://coord.info/GC5XGWH}{(BLV) 14 Balade à Rimbault :  Houx le fragon !} — \href{http://coord.info/GC5XGWH\Number{}806992676}{2159}}\cacheData{{2018/10/06 lulu\Underscore{}et\Underscore{}compagnie, Traditional Cache (2/2)}}\begin{cacheText}En route pour le GTAQ fait de la barque à Coulon ,la Team 640 (Isaasia, Dune33 et moi même) en profite pour faire quelques caches en chemin .La forêt domaniale est parfaitement entretenue et la promenade fort agréable. Nous découvrons des caches travaillées mais qui sont hélas quelque peu dégradées par des chercheurs peu délicats . Du joli travail,merci pour la cache.\end{cacheText}

\cacheNumber{2160}\needspace{5\baselineskip}\cacheName{\href{http://coord.info/GC2WFGH}{Event Niortais: maison forestiere de Rimbault} — \href{http://coord.info/GC2WFGH\Number{}806992677}{2160}}\cacheData{{2018/10/06 lulu\Underscore{}et\Underscore{}compagnie, Traditional Cache (2.5/2)}}\begin{cacheText}En route pour le GTAQ fait de la barque à Coulon ,la Team 640 (Isaasia, Dune33 et moi même) en profite pour faire quelques caches en chemin .La forêt domaniale est parfaitement entretenue et la promenade fort agréable. Nous découvrons des caches travaillées mais qui sont hélas quelque peu dégradées par des chercheurs peu délicats . Du joli travail,merci pour la cache.\end{cacheText}

\cacheNumber{2161}\needspace{5\baselineskip}\cacheName{\href{http://coord.info/GC1879E}{La venise verte} — \href{http://coord.info/GC1879E\Number{}809200379}{2161}}\cacheData{{2018/10/06 lulu\Underscore{}et\Underscore{}compagnie and PianoPoppets, Traditional Cache (2.5/1.5)}}\begin{cacheText}Cache loguée sous le pseudo Team 640 ( Fleura64 et isaasia du 64 , Dune33 et DorisBear du 40). Il nous reste encore un peu de temps avant l’apéro avec le chou vert. Nous nous échappons  pour trouver la Belle. Elle est dénichée rapidement et partons faire à la suivante. Merci pour la cache.\end{cacheText}

\cacheNumber{2162}\needspace{5\baselineskip}\cacheName{\href{http://coord.info/GC3K6R5}{Débarcadère, lettre D de l'alphabet deux-sèvrien} — \href{http://coord.info/GC3K6R5\Number{}809103189}{2162}}\cacheData{{2018/10/06 le.gatinais, Traditional Cache (5/1.5)}}\begin{cacheText}C’est lors de l’Event Le GTAQ fait de la barque que nous nous approchons en groupe du débarcadère . Le temps est maussade et il n’y a pas un chat à l’horizon : la cache est délogée sans aucune difficulté.Merci pour la découverte de cette superbe place.Logbook signé Team640\end{cacheText}

\cacheNumber{2163}\needspace{5\baselineskip}\cacheName{\href{http://coord.info/GC50P1E}{(MM) Barque} — \href{http://coord.info/GC50P1E\Number{}809203696}{2163}}\cacheData{{2018/10/06 lulu\Underscore{}et\Underscore{}compagnie, Traditional Cache (2/2)}}\begin{cacheText}Cache validée pour la Team 640 ( Fleura64 et isaasia du 64 , Dune33 et DorisBear du 40) avec la permission de l’owner. Malgré nos longues recherches nous n’avons pas trouvé la cache. L’indice, très explicite ,ne laisse aucun doute sur l’endroit ou chercher.Elle a encore disparu !Nous prenons le temps d'admirer le superbe bouton de nénuphar.Merci Lulu et Compagnie pour ce très joli circuit.\end{cacheText}

\cacheNumber{2164}\needspace{5\baselineskip}\cacheName{\href{http://coord.info/GC50P5M}{(MM) Quai Louis Tardy} — \href{http://coord.info/GC50P5M\Number{}809121134}{2164}}\cacheData{{2018/10/06 lulu\Underscore{}et\Underscore{}compagnie, Traditional Cache (2/2)}}\begin{cacheText}Parmi la Team Pigouille ,il y en a qui connaisse bien Coulon et ses caches!! Nous signons en passant celle ci qui est toujours en place .Merci lulu et compagnie pour nous avoir autorisé à la loguer malgré l'archivage.\end{cacheText}

\cacheNumber{2165}\needspace{5\baselineskip}\cacheName{\href{http://coord.info/GC50P64}{(MM) La grange à Camille} — \href{http://coord.info/GC50P64\Number{}809103550}{2165}}\cacheData{{2018/10/06 lulu\Underscore{}et\Underscore{}compagnie, Traditional Cache (1.5/1)}}\begin{cacheText}C’est après l’Event ,GTAQ fait de la barque ,et avant le traditionnel apéro au Chou Vert que la Team Pigouille part à la recherche des quelques caches terrestres. Avec tous ces yeux, la Belle ne peut pas nous résister. Merci pour la cache.\end{cacheText}

\cacheNumber{2166}\needspace{5\baselineskip}\cacheName{\href{http://coord.info/GC50P6K}{(MM) Sévre Niortaise} — \href{http://coord.info/GC50P6K\Number{}809120417}{2166}}\cacheData{{2018/10/06 lulu\Underscore{}et\Underscore{}compagnie, Traditional Cache (2/2)}}\begin{cacheText}Parmi la Team Pigouille ,il y en a qui connaissent bien Coulon et ses caches!! Nous signons en passant celle ci qui est toujours en place .Merci lulu et compagnie pour nous avoir autorisé à la loguer malgré l'archivage.\end{cacheText}

\cacheNumber{2167}\needspace{5\baselineskip}\cacheName{\href{http://coord.info/GC50P8E}{(MM) Ecluse} — \href{http://coord.info/GC50P8E\Number{}809104663}{2167}}\cacheData{{2018/10/06 lulu\Underscore{}et\Underscore{}compagnie, Traditional Cache (2/2)}}\begin{cacheText}LA Team PIgouille , plus qu'efficace ,déniche la Belle très rapidement. Heureusement….l'heure tourne et elle ne veut pas arriver en retard pour le Chou Vert. Merci pour la cache.\end{cacheText}

\cacheNumber{2168}\needspace{5\baselineskip}\cacheName{\href{http://coord.info/GC50P8N}{(MM) Pause} — \href{http://coord.info/GC50P8N\Number{}809103782}{2168}}\cacheData{{2018/10/06 lulu\Underscore{}et\Underscore{}compagnie, Traditional Cache (1.5/1.5)}}\begin{cacheText}Nous continuons en troupeau la découverte du paysage Poitevin. Après avoir exploré le banc nous mettons la main sur la Belle. La Team Pigouille vous remercie.\end{cacheText}

\cacheNumber{2169}\needspace{5\baselineskip}\cacheName{\href{http://coord.info/GC5ZE5N}{A la pigouille \Number{}02} — \href{http://coord.info/GC5ZE5N\Number{}809706957}{2169}}\cacheData{{2018/10/06 fina18, Traditional Cache (2.5/5)}}\begin{cacheText}Et c’est parti pour un long, long périple dans les canaux du Marais poitevin. La team 640 composée de Dune33, des DorisBear, d’Isaasia et de Fleura64 se répartit les tâches : Dédé et Isa rament, Bernard tient la barre, Diotima lit la carte, Denise attrape les boîtes et je prends les photos ! ! ! Au fil de l’eau, nous sommes émerveillés par les paysages et leurs divers habitants. Toutes les caches sont découvertes sans trop de difficulté et nous n’avons pas vu passer le temps ( sauf les rameurs ! ! !). Que de fou rire le long de cette promenade bucolique ! ! ! Mémorable ! ! ! L’averse de la fin de la journée n’a pas entamé notre bonne humeur. Un grand merci fina 18 pour ce travail de pause dans ce cadre fantastique.\end{cacheText}

\cacheNumber{2170}\needspace{5\baselineskip}\cacheName{\href{http://coord.info/GC7AM7B}{- 01 Aqu@cache- {SCS 45 - La garette-🚣}} — \href{http://coord.info/GC7AM7B\Number{}809710405}{2170}}\cacheData{{2018/10/06 SCS Team, Traditional Cache (1.5/5)}}\begin{cacheText}Et c’est parti pour un long, long périple dans les canaux du Marais poitevin. La team 640 composée de Dune33, des DorisBear, d’Isaasia et de Fleura64 se répartit les tâches : Dédé et Isa rament, Bernard tient la barre, Diotima lit la carte, Denise attrape les boîtes et je prends les photos ! ! ! Au fil de l’eau, nous sommes émerveillés par les paysages et leurs divers habitants. Toutes les caches sont découvertes sans trop de difficulté et nous n’avons pas vu passer le temps ( sauf les rameurs ! ! !). Que de fou rire le long de cette promenade bucolique ! ! ! Mémorable ! ! ! L’averse de la fin de la journée n’a pas entamé notre bonne humeur. Un grand merci à la SCS team pour ce travail de pause dans ce cadre fantastique.\end{cacheText}

\cacheNumber{2171}\needspace{5\baselineskip}\cacheName{\href{http://coord.info/GC7AM7E}{- 02 Aqu@cache- {SCS 45 - La garette-🚣}} — \href{http://coord.info/GC7AM7E\Number{}809710107}{2171}}\cacheData{{2018/10/06 SCS Team, Traditional Cache (1.5/5)}}\begin{cacheText}Et c’est parti pour un long, long périple dans les canaux du Marais poitevin. La team 640 composée de Dune33, des DorisBear, d’Isaasia et de Fleura64 se répartit les tâches : Dédé et Isa rament, Bernard tient la barre, Diotima lit la carte, Denise attrape les boîtes et je prends les photos ! ! ! Au fil de l’eau, nous sommes émerveillés par les paysages et leurs divers habitants. Toutes les caches sont découvertes sans trop de difficulté et nous n’avons pas vu passer le temps ( sauf les rameurs ! ! !). Que de fou rire le long de cette promenade bucolique ! ! ! Mémorable ! ! ! L’averse de la fin de la journée n’a pas entamé notre bonne humeur. Un grand merci à la SCS team pour ce travail de pause dans ce cadre fantastique.\end{cacheText}

\cacheNumber{2172}\needspace{5\baselineskip}\cacheName{\href{http://coord.info/GC7AM7F}{- 03 Aqu@cache- {SCS 45 - La garette-🚣}} — \href{http://coord.info/GC7AM7F\Number{}809707372}{2172}}\cacheData{{2018/10/06 SCS Team, Traditional Cache (1.5/5)}}\begin{cacheText}Et c’est parti pour un long, long périple dans les canaux du Marais poitevin. La team 640 composée de Dune33, des DorisBear, d’Isaasia et de Fleura64 se répartit les tâches : Dédé et Isa rament, Bernard tient la barre, Diotima lit la carte, Denise attrape les boîtes et je prends les photos ! ! ! Au fil de l’eau, nous sommes émerveillés par les paysages et leurs divers habitants. Toutes les caches sont découvertes sans trop de difficulté et nous n’avons pas vu passer le temps ( sauf les rameurs ! ! !). Que de fou rire le long de cette promenade bucolique ! ! ! Mémorable ! ! ! L’averse de la fin de la journée n’a pas entamé notre bonne humeur. Un grand merci à la SCS team pour ce travail de pause dans ce cadre fantastique.\end{cacheText}

\cacheNumber{2173}\needspace{5\baselineskip}\cacheName{\href{http://coord.info/GC7AM7K}{- 04 Aqu@cache- {SCS 45 - La garette-🚣}} — \href{http://coord.info/GC7AM7K\Number{}809709042}{2173}}\cacheData{{2018/10/06 SCS Team, Traditional Cache (2.5/5)}}\begin{cacheText}Et c’est parti pour un long, long périple dans les canaux du Marais poitevin. La team 640 composée de Dune33, des DorisBear, d’Isaasia et de Fleura64 se répartit les tâches : Dédé et Isa rament, Bernard tient la barre, Diotima lit la carte, Denise attrape les boîtes et je prends les photos ! ! ! Au fil de l’eau, nous sommes émerveillés par les paysages et leurs divers habitants. Toutes les caches sont découvertes sans trop de difficulté et nous n’avons pas vu passer le temps ( sauf les rameurs ! ! !). Que de fou rire le long de cette promenade bucolique ! ! ! Mémorable ! ! ! L’averse de la fin de la journée n’a pas entamé notre bonne humeur. Un grand merci à la SCS team pour ce travail de pause dans ce cadre fantastique.\end{cacheText}

\cacheNumber{2174}\needspace{5\baselineskip}\cacheName{\href{http://coord.info/GC7AM7T}{- 05 Aqu@cache- {SCS 45 - La garette-🚣}} — \href{http://coord.info/GC7AM7T\Number{}809705341}{2174}}\cacheData{{2018/10/06 SCS Team, Traditional Cache (1.5/5)}}\begin{cacheText}Et c’est parti pour un long, long périple dans les canaux du Marais poitevin. La team 640 composée de Dune33, des DorisBear, d’Isaasia et de Fleura64 se répartit les tâches : Dédé et Isa rament, Bernard tient la barre, Diotima lit la carte, Denise attrape les boîtes et je prends les photos ! ! ! Au fil de l’eau, nous sommes émerveillés par les paysages et leurs divers habitants. Toutes les caches sont découvertes sans trop de difficulté et nous n’avons pas vu passer le temps ( sauf les rameurs ! ! !). Que de fou rire le long de cette promenade bucolique ! ! ! Mémorable ! ! ! L’averse de la fin de la journée n’a pas entamé notre bonne humeur. Un grand merci à la SCS team pour ce travail de pause dans ce cadre fantastique.\end{cacheText}

\cacheNumber{2175}\needspace{5\baselineskip}\cacheName{\href{http://coord.info/GC7AM80}{- 06 Aqu@cache- {SCS 45 - La garette-🚣}} — \href{http://coord.info/GC7AM80\Number{}809705106}{2175}}\cacheData{{2018/10/06 SCS Team, Traditional Cache (1.5/5)}}\begin{cacheText}Et c’est parti pour un long, long périple dans les canaux du Marais poitevin. La team 640 composée de Dune33, des DorisBear, d’Isaasia et de Fleura64 se répartit les tâches : Dédé et Isa rament, Bernard tient la barre, Diotima lit la carte, Denise attrape les boîtes et je prends les photos ! ! ! Au fil de l’eau, nous sommes émerveillés par les paysages et leurs divers habitants. Toutes les caches sont découvertes sans trop de difficulté et nous n’avons pas vu passer le temps ( sauf les rameurs ! ! !). Que de fou rire le long de cette promenade bucolique ! ! ! Mémorable ! ! ! L’averse de la fin de la journée n’a pas entamé notre bonne humeur. Un grand merci à la SCS team pour ce travail de pause dans ce cadre fantastique.\end{cacheText}

\cacheNumber{2176}\needspace{5\baselineskip}\cacheName{\href{http://coord.info/GC7AM81}{- 07 Aqu@cache- {SCS 45 - La garette-🚣}} — \href{http://coord.info/GC7AM81\Number{}809704836}{2176}}\cacheData{{2018/10/06 SCS Team, Traditional Cache (1.5/5)}}\begin{cacheText}Et c’est parti pour un long, long périple dans les canaux du Marais poitevin. La team 640 composée de Dune33, des DorisBear, d’Isaasia et de Fleura64 se répartit les tâches : Dédé et Isa rament, Bernard tient la barre, Diotima lit la carte, Denise attrape les boîtes et je prends les photos ! ! ! Au fil de l’eau, nous sommes émerveillés par les paysages et leurs divers habitants. Toutes les caches sont découvertes sans trop de difficulté et nous n’avons pas vu passer le temps ( sauf les rameurs ! ! !). Que de fou rire le long de cette promenade bucolique ! ! ! Mémorable ! ! ! L’averse de la fin de la journée n’a pas entamé notre bonne humeur. Un grand merci à la SCS team pour ce travail de pause dans ce cadre fantastique.\end{cacheText}

\cacheNumber{2177}\needspace{5\baselineskip}\cacheName{\href{http://coord.info/GC7AM87}{- 08 Aqu@cache- {SCS 45 - La garette-🚣}} — \href{http://coord.info/GC7AM87\Number{}809467411}{2177}}\cacheData{{2018/10/06 SCS Team, Traditional Cache (1.5/5)}}\begin{cacheText}Et c'est parti pour un long , long périple dans les canaux du marais Poitevin. La Team 640 composée des Dune33, des Dorisbear, de Isaasia  et de Fleura64 se repartit les taches :Dédé et Isa rament, Bernard tient la barre, Diotima lit la carte ,Denise attrape les boites et je prend les photos!!!!Au fil de l'eau nous sommes émerveillés par les paysages et leurs divers habitants. Toutes les caches sont découvertes sans trop de difficulté et nous n'avons pas vu passer le temps (sauf les rameurs!!!!).Que de fou rire le long de cette promenade bucolique!!!Mémorable!!!L'averse de la fin de journée n'a pas entamé notre bonne humeur. Un grand merci à la SCS Team pour ce travail de pose dans ce cadre fantastique.\end{cacheText}

\cacheNumber{2178}\needspace{5\baselineskip}\cacheName{\href{http://coord.info/GC7AM8B}{- 09 Aqu@cache- {SCS 45 - La garette-🚣}} — \href{http://coord.info/GC7AM8B\Number{}809467001}{2178}}\cacheData{{2018/10/06 SCS Team, Traditional Cache (1.5/5)}}\begin{cacheText}Et c'est parti pour un long , long périple dans les canaux du marais Poitevin. La Team 640 composée des Dune33, des Dorisbear, de Isaasia  et de Fleura64 se repartit les taches :Dédé et Isa rament, Bernard tient la barre, Diotima lit la carte ,Denise attrape les boites et je prend les photos!!!!Au fil de l'eau nous sommes émerveillés par les paysages et leurs divers habitants. Toutes les caches sont découvertes sans trop de difficulté et nous n'avons pas vu passer le temps (sauf les rameurs!!!!).Que de fou rire le long de cette promenade bucolique!!!Mémorable!!!L'averse de la fin de journée n'a pas entamé notre bonne humeur. Un grand merci à la SCS Team pour ce travail de pose dans ce cadre fantastique.\end{cacheText}

\cacheNumber{2179}\needspace{5\baselineskip}\cacheName{\href{http://coord.info/GC7AM8G}{- 10 Aqu@cache- {SCS 45 - La garette-🚣}} — \href{http://coord.info/GC7AM8G\Number{}809466242}{2179}}\cacheData{{2018/10/06 SCS Team, Traditional Cache (1.5/5)}}\begin{cacheText}Et c'est parti pour un long , long périple dans les canaux du marais Poitevin. La Team 640 composée des Dune33, des Dorisbear, de Isaasia  et de Fleura64 se repartit les taches :Dédé et Isa rament, Bernard tient la barre, Diotima lit la carte ,Denise attrape les boites et je prend les photos!!!!Au fil de l'eau nous sommes émerveillés par les paysages et leurs divers habitants. Toutes les caches sont découvertes sans trop de difficulté et nous n'avons pas vu passer le temps (sauf les rameurs!!!!).Que de fou rire le long de cette promenade bucolique!!!Mémorable!!!L'averse de la fin de journée n'a pas entamé notre bonne humeur. Un grand merci à la SCS Team pour ce travail de pose dans ce cadre fantastique.\end{cacheText}

\cacheNumber{2180}\needspace{5\baselineskip}\cacheName{\href{http://coord.info/GC7AM8M}{- 11 Aqu@cache- {SCS 45 - La garette-🚣}} — \href{http://coord.info/GC7AM8M\Number{}809711778}{2180}}\cacheData{{2018/10/06 SCS Team, Traditional Cache (1.5/5)}}\begin{cacheText}Et c’est parti pour un long, long périple dans les canaux du Marais poitevin. La team 640 composée de Dune33, des DorisBear, d’Isaasia et de Fleura64 se répartit les tâches : Dédé et Isa rament, Bernard tient la barre, Diotima lit la carte, Denise attrape les boîtes et je prends les photos ! ! ! Au fil de l’eau, nous sommes émerveillés par les paysages et leurs divers habitants. Toutes les caches sont découvertes sans trop de difficulté et nous n’avons pas vu passer le temps ( sauf les rameurs ! ! !). Que de fou rire le long de cette promenade bucolique ! ! ! Mémorable ! ! ! L’averse de la fin de la journée n’a pas entamé notre bonne humeur. Un grand merci à la SCS team pour ce travail de pause dans ce cadre fantastique.\end{cacheText}

\cacheNumber{2181}\needspace{5\baselineskip}\cacheName{\href{http://coord.info/GC7AM8W}{- 12 Aqu@cache- {SCS 45 - La garette-🚣}} — \href{http://coord.info/GC7AM8W\Number{}809711515}{2181}}\cacheData{{2018/10/06 SCS Team, Traditional Cache (1.5/5)}}\begin{cacheText}Et c’est parti pour un long, long périple dans les canaux du Marais poitevin. La team 640 composée de Dune33, des DorisBear, d’Isaasia et de Fleura64 se répartit les tâches : Dédé et Isa rament, Bernard tient la barre, Diotima lit la carte, Denise attrape les boîtes et je prends les photos ! ! ! Au fil de l’eau, nous sommes émerveillés par les paysages et leurs divers habitants. Toutes les caches sont découvertes sans trop de difficulté et nous n’avons pas vu passer le temps ( sauf les rameurs ! ! !). Que de fou rire le long de cette promenade bucolique ! ! ! Mémorable ! ! ! L’averse de la fin de la journée n’a pas entamé notre bonne humeur. Un grand merci à la SCS team pour ce travail de pause dans ce cadre fantastique.\end{cacheText}

\cacheNumber{2182}\needspace{5\baselineskip}\cacheName{\href{http://coord.info/GC7AM90}{- 13 Aqu@cache- {SCS 45 - La garette-🚣}} — \href{http://coord.info/GC7AM90\Number{}809711201}{2182}}\cacheData{{2018/10/06 SCS Team, Traditional Cache (1.5/5)}}\begin{cacheText}Et c’est parti pour un long, long périple dans les canaux du Marais poitevin. La team 640 composée de Dune33, des DorisBear, d’Isaasia et de Fleura64 se répartit les tâches : Dédé et Isa rament, Bernard tient la barre, Diotima lit la carte, Denise attrape les boîtes et je prends les photos ! ! ! Au fil de l’eau, nous sommes émerveillés par les paysages et leurs divers habitants. Toutes les caches sont découvertes sans trop de difficulté et nous n’avons pas vu passer le temps ( sauf les rameurs ! ! !). Que de fou rire le long de cette promenade bucolique ! ! ! Mémorable ! ! ! L’averse de la fin de la journée n’a pas entamé notre bonne humeur. Un grand merci à la SCS team pour ce travail de pause dans ce cadre fantastique.\end{cacheText}

\cacheNumber{2183}\needspace{5\baselineskip}\cacheName{\href{http://coord.info/GC7AM9N}{- 14 Aqu@cache- {SCS 45 - La garette-🚣}} — \href{http://coord.info/GC7AM9N\Number{}809711011}{2183}}\cacheData{{2018/10/06 SCS Team, Traditional Cache (1.5/5)}}\begin{cacheText}Et c’est parti pour un long, long périple dans les canaux du Marais poitevin. La team 640 composée de Dune33, des DorisBear, d’Isaasia et de Fleura64 se répartit les tâches : Dédé et Isa rament, Bernard tient la barre, Diotima lit la carte, Denise attrape les boîtes et je prends les photos ! ! ! Au fil de l’eau, nous sommes émerveillés par les paysages et leurs divers habitants. Toutes les caches sont découvertes sans trop de difficulté et nous n’avons pas vu passer le temps ( sauf les rameurs ! ! !). Que de fou rire le long de cette promenade bucolique ! ! ! Mémorable ! ! ! L’averse de la fin de la journée n’a pas entamé notre bonne humeur. Un grand merci à la SCS team pour ce travail de pause dans ce cadre fantastique.\end{cacheText}

\cacheNumber{2184}\needspace{5\baselineskip}\cacheName{\href{http://coord.info/GC7AM9P}{- 15 Aqu@cache- {SCS 45 - La garette-🚣}} — \href{http://coord.info/GC7AM9P\Number{}809710776}{2184}}\cacheData{{2018/10/06 SCS Team, Traditional Cache (1.5/5)}}\begin{cacheText}Et c’est parti pour un long, long périple dans les canaux du Marais poitevin. La team 640 composée de Dune33, des DorisBear, d’Isaasia et de Fleura64 se répartit les tâches : Dédé et Isa rament, Bernard tient la barre, Diotima lit la carte, Denise attrape les boîtes et je prends les photos ! ! ! Au fil de l’eau, nous sommes émerveillés par les paysages et leurs divers habitants. Toutes les caches sont découvertes sans trop de difficulté et nous n’avons pas vu passer le temps ( sauf les rameurs ! ! !). Que de fou rire le long de cette promenade bucolique ! ! ! Mémorable ! ! ! L’averse de la fin de la journée n’a pas entamé notre bonne humeur. Un grand merci à la SCS team pour ce travail de pause dans ce cadre fantastique.\end{cacheText}

\cacheNumber{2185}\needspace{5\baselineskip}\cacheName{\href{http://coord.info/GC7Y03C}{GTAQ fait de la barque} — \href{http://coord.info/GC7Y03C\Number{}809368819}{2185}}\cacheData{{2018/10/06 GTAQ, Event Cache (1/1)}}\begin{cacheText}Enfin le jour J ... Après une courte nuit , nous ( Dune33, Isaasia et Fleura64) arrivons au PZ. Nous rencontrons les géocacheurs locaux et d’ailleurs ( toujours sympa de mettre un visage sur un pseudo) et échangeons quelques TB.Décision est prise de faire les caches terrestres ,en groupe sous le pseudo Team Pigouille ,en attendant l’apéro Chou Vert et le repas.Après une promenade sympa ,nous rejoignons le rendez-vous(zut on a loupé la photo de groupe) pour nous restaurer et nous abreuver ou plutôt nous abreuver et nous restaurer ( 1 PF pour le gâteau au chocolat d’Emile qui était excellent). L’heure d’embarquer est là ...nous prenons la direction du départ des caches aquatiques.Encore un super Event en charmante compagnie ,organisé à la perfection.Un grand merci aux organisateurs pour ce très bon moment. Merci Auguste.\end{cacheText}

\cacheNumber{2186}\needspace{5\baselineskip}\cacheName{\href{http://coord.info/GC17MT0}{à la pigouille....} — \href{http://coord.info/GC17MT0\Number{}809709778}{2186}}\cacheData{{2018/10/06 paleolog, Traditional Cache (3/5)}}\begin{cacheText}Et c’est parti pour un long, long périple dans les canaux du Marais poitevin. La team 640 composée de Dune33, des DorisBear, d’Isaasia et de Fleura64 se répartit les tâches : Dédé et Isa rament, Bernard tient la barre, Diotima lit la carte, Denise attrape les boîtes et je prends les photos ! ! ! Au fil de l’eau, nous sommes émerveillés par les paysages et leurs divers habitants. Toutes les caches sont découvertes sans trop de difficulté et nous n’avons pas vu passer le temps ( sauf les rameurs ! ! !). Que de fou rire le long de cette promenade bucolique ! ! ! Mémorable ! ! ! L’averse de la fin de la journée n’a pas entamé notre bonne humeur. Un grand merci paleolog pour ce travail de pause dans ce cadre fantastique.\end{cacheText}

\cacheNumber{2187}\needspace{5\baselineskip}\cacheName{\href{http://coord.info/GC2731A}{Deauville dans le marais} — \href{http://coord.info/GC2731A\Number{}809737893}{2187}}\cacheData{{2018/10/07 lulu\Underscore{}et\Underscore{}compagnie, Traditional Cache (1.5/1.5)}}\begin{cacheText}Venus dans le Marais Poitevin pour participer à l’évent le GTAQ fait de la barque, la team 640 composée de Dune33, Isaasia et moi-même décide de prolonger le séjour en compagnie de Domino50 et de Crispol40. Toutes les caches seront trouvées avec plus ou moins de difficultés dans un décor de rêve. Merci pour la cache.\end{cacheText}

\cacheNumber{2188}\needspace{5\baselineskip}\cacheName{\href{http://coord.info/GC28PAD}{L'écluse de la Sotterie} — \href{http://coord.info/GC28PAD\Number{}810201274}{2188}}\cacheData{{2018/10/07 lulu\Underscore{}et\Underscore{}compagnie, Traditional Cache (2/1.5)}}\begin{cacheText}C’est de nuit que la team 640 vient déloger la Belle avant le retour au camping. Les coordonnées précises et l’indice nous mènent tout droit au PZ. Nous reviendrons demain matin pour faire quelques photos de ce bel endroit. Merci pour la cache.\end{cacheText}

\cacheNumber{2189}\needspace{5\baselineskip}\cacheName{\href{http://coord.info/GC3QY9N}{Marais Poitevin,lettre M de l'alphabet Deux-Sévrie} — \href{http://coord.info/GC3QY9N\Number{}809736055}{2189}}\cacheData{{2018/10/07 lulu\Underscore{}et\Underscore{}compagnie, Traditional Cache (1.5/1.5)}}\begin{cacheText}Venus dans le Marais Poitevin pour participer à l’évent le GTAQ fait de la barque, la team 640 composée de Dune33, Isaasia et moi-même décide de prolonger le séjour en compagnie de Domino50 et de Crispol40. Toutes les caches seront trouvées avec plus ou moins de difficultés dans un décor de rêve. Merci pour la cache.\end{cacheText}

\cacheNumber{2190}\needspace{5\baselineskip}\cacheName{\href{http://coord.info/GC3VB2P}{[EMP] La cabane} — \href{http://coord.info/GC3VB2P\Number{}809720029}{2190}}\cacheData{{2018/10/07 lulu\Underscore{}et\Underscore{}compagnie, Traditional Cache (2.5/1.5)}}\begin{cacheText}Venus dans le Marais Poitevin pour participer à l’évent le GTAQ fait de la barque, la team 640 composée de Dune33, Isaasia et moi-même décide de prolonger le séjour en compagnie de Domino50 et de Crispol40. Toutes les caches seront trouvées avec plus ou moins de difficultés dans un décor de rêve. Merci pour la cache.\end{cacheText}

\cacheNumber{2191}\needspace{5\baselineskip}\cacheName{\href{http://coord.info/GC3VB3E}{[EMP] Passerelle} — \href{http://coord.info/GC3VB3E\Number{}809720541}{2191}}\cacheData{{2018/10/07 lulu\Underscore{}et\Underscore{}compagnie, Traditional Cache (2/2)}}\begin{cacheText}Venus dans le Marais Poitevin pour participer à l’évent le GTAQ fait de la barque, la team 640 composée de Dune33, Isaasia et moi-même décide de prolonger le séjour en compagnie de Domino50 et de Crispol40. Toutes les caches seront trouvées avec plus ou moins de difficultés dans un décor de rêve. Merci pour la cache.\end{cacheText}

\cacheNumber{2192}\needspace{5\baselineskip}\cacheName{\href{http://coord.info/GC3VB40}{[EMP] Les Planches} — \href{http://coord.info/GC3VB40\Number{}809737583}{2192}}\cacheData{{2018/10/07 lulu\Underscore{}et\Underscore{}compagnie, Traditional Cache (2/2.5)}}\begin{cacheText}Venus dans le Marais Poitevin pour participer à l’évent le GTAQ fait de la barque, la team 640 composée de Dune33, Isaasia et moi-même décide de prolonger le séjour en compagnie de Domino50 et de Crispol40. Toutes les caches seront trouvées avec plus ou moins de difficultés dans un décor de rêve. Merci pour la cache.\end{cacheText}

\cacheNumber{2193}\needspace{5\baselineskip}\cacheName{\href{http://coord.info/GC3VB5B}{[EMP] Intersection} — \href{http://coord.info/GC3VB5B\Number{}809738195}{2193}}\cacheData{{2018/10/07 lulu\Underscore{}et\Underscore{}compagnie, Traditional Cache (2/2)}}\begin{cacheText}Venus dans le Marais Poitevin pour participer à l’évent le GTAQ fait de la barque, la team 640 composée de Dune33, Isaasia et moi-même décide de prolonger le séjour en compagnie de Domino50 et de Crispol40. Toutes les caches seront trouvées avec plus ou moins de difficultés dans un décor de rêve. Merci pour la cache.\end{cacheText}

\cacheNumber{2194}\needspace{5\baselineskip}\cacheName{\href{http://coord.info/GC3VCN2}{[EMP]La maison abandonnée} — \href{http://coord.info/GC3VCN2\Number{}809738711}{2194}}\cacheData{{2018/10/07 lulu\Underscore{}et\Underscore{}compagnie, Traditional Cache (2/1.5)}}\begin{cacheText}Venus dans le Marais Poitevin pour participer à l’évent le GTAQ fait de la barque, la team 640 composée de Dune33, Isaasia et moi-même décide de prolonger le séjour en compagnie de Domino50 et de Crispol40. Toutes les caches seront trouvées avec plus ou moins de difficultés dans un décor de rêve. Merci pour la cache.\end{cacheText}

\cacheNumber{2195}\needspace{5\baselineskip}\cacheName{\href{http://coord.info/GC3VCND}{[EMP] Maison du marais} — \href{http://coord.info/GC3VCND\Number{}809738461}{2195}}\cacheData{{2018/10/07 lulu\Underscore{}et\Underscore{}compagnie, Traditional Cache (2/2)}}\begin{cacheText}Venus dans le Marais Poitevin pour participer à l’évent le GTAQ fait de la barque, la team 640 composée de Dune33, Isaasia et moi-même décide de prolonger le séjour en compagnie de Domino50 et de Crispol40. Toutes les caches seront trouvées avec plus ou moins de difficultés dans un décor de rêve. Merci pour la cache.\end{cacheText}

\cacheNumber{2196}\needspace{5\baselineskip}\cacheName{\href{http://coord.info/GC3VD29}{[EMP] Cale et barques} — \href{http://coord.info/GC3VD29\Number{}809718722}{2196}}\cacheData{{2018/10/07 lulu\Underscore{}et\Underscore{}compagnie, Traditional Cache (1.5/1.5)}}\begin{cacheText}Venus dans le Marais Poitevin pour participer à l’évent le GTAQ fait de la barque, la team 640 composée de Dune33, Isaasia et moi-même décide de prolonger le séjour en compagnie de Domino50 et de Crispol40. Toutes les caches seront trouvées avec plus ou moins de difficultés dans un décor de rêve. Merci pour la cache.\end{cacheText}

\cacheNumber{2197}\needspace{5\baselineskip}\cacheName{\href{http://coord.info/GC3VQCC}{[EMP] le Tertre} — \href{http://coord.info/GC3VQCC\Number{}809721711}{2197}}\cacheData{{2018/10/07 lulu\Underscore{}et\Underscore{}compagnie, Traditional Cache (2/2)}}\begin{cacheText}Venus dans le Marais Poitevin pour participer à l’évent le GTAQ fait de la barque, la team 640 composée de Dune33, Isaasia et moi-même décide de prolonger le séjour en compagnie de Domino50 et de Crispol40. Toutes les caches seront trouvées avec plus ou moins de difficultés dans un décor de rêve. Merci pour la cache.\end{cacheText}

\cacheNumber{2198}\needspace{5\baselineskip}\cacheName{\href{http://coord.info/GC3VQD2}{[EMP] En route vers le sommet !} — \href{http://coord.info/GC3VQD2\Number{}809722495}{2198}}\cacheData{{2018/10/07 lulu\Underscore{}et\Underscore{}compagnie, Traditional Cache (2/2)}}\begin{cacheText}Venus dans le Marais Poitevin pour participer à l’évent le GTAQ fait de la barque, la team 640 composée de Dune33, Isaasia et moi-même décide de prolonger le séjour en compagnie de Domino50 et de Crispol40. Toutes les caches seront trouvées avec plus ou moins de difficultés dans un décor de rêve. Merci pour la cache.\end{cacheText}

\cacheNumber{2199}\needspace{5\baselineskip}\cacheName{\href{http://coord.info/GC3VV35}{[EMP] Rue des Gravées} — \href{http://coord.info/GC3VV35\Number{}809737268}{2199}}\cacheData{{2018/10/07 lulu\Underscore{}et\Underscore{}compagnie, Traditional Cache (1.5/1.5)}}\begin{cacheText}Venus dans le Marais Poitevin pour participer à l’évent le GTAQ fait de la barque, la team 640 composée de Dune33, Isaasia et moi-même décide de prolonger le séjour en compagnie de Domino50 et de Crispol40. Toutes les caches seront trouvées avec plus ou moins de difficultés dans un décor de rêve. Merci pour la cache.\end{cacheText}

\cacheNumber{2200}\needspace{5\baselineskip}\cacheName{\href{http://coord.info/GC4FMMV}{2 / Chemin du marais / Balade} — \href{http://coord.info/GC4FMMV\Number{}809723385}{2200}}\cacheData{{2018/10/07 lulu\Underscore{}et\Underscore{}compagnie, Traditional Cache (2/1.5)}}\begin{cacheText}Venus dans le Marais Poitevin pour participer à l’évent le GTAQ fait de la barque, la team 640 composée de Dune33, Isaasia et moi-même décide de prolonger le séjour en compagnie de Domino50 et de Crispol40. Toutes les caches seront trouvées avec plus ou moins de difficultés dans un décor de rêve. Merci pour la cache.\end{cacheText}

\cacheNumber{2201}\needspace{5\baselineskip}\cacheName{\href{http://coord.info/GC4FMN0}{3 / Chemin du marais  / Cycliste} — \href{http://coord.info/GC4FMN0\Number{}809723810}{2201}}\cacheData{{2018/10/07 lulu\Underscore{}et\Underscore{}compagnie, Traditional Cache (1.5/1.5)}}\begin{cacheText}Venus dans le Marais Poitevin pour participer à l’évent le GTAQ fait de la barque, la team 640 composée de Dune33, Isaasia et moi-même décide de prolonger le séjour en compagnie de Domino50 et de Crispol40. Toutes les caches seront trouvées avec plus ou moins de difficultés dans un décor de rêve. Merci pour la cache.\end{cacheText}

\cacheNumber{2202}\needspace{5\baselineskip}\cacheName{\href{http://coord.info/GC4FMN6}{4 / Chemin du marais / Heron} — \href{http://coord.info/GC4FMN6\Number{}809724134}{2202}}\cacheData{{2018/10/07 lulu\Underscore{}et\Underscore{}compagnie, Traditional Cache (2.5/2)}}\begin{cacheText}Venus dans le Marais Poitevin pour participer à l’évent le GTAQ fait de la barque, la team 640 composée de Dune33, Isaasia et moi-même décide de prolonger le séjour en compagnie de Domino50 et de Crispol40. Toutes les caches seront trouvées avec plus ou moins de difficultés dans un décor de rêve. Merci pour la cache.\end{cacheText}

\cacheNumber{2203}\needspace{5\baselineskip}\cacheName{\href{http://coord.info/GC4FMN9}{5 / Chemin du marais  / Ecluse} — \href{http://coord.info/GC4FMN9\Number{}809724368}{2203}}\cacheData{{2018/10/07 lulu\Underscore{}et\Underscore{}compagnie, Traditional Cache (2/3)}}\begin{cacheText}Venus dans le Marais Poitevin pour participer à l’évent le GTAQ fait de la barque, la team 640 composée de Dune33, Isaasia et moi-même décide de prolonger le séjour en compagnie de Domino50 et de Crispol40. Toutes les caches seront trouvées avec plus ou moins de difficultés dans un décor de rêve. Merci pour la cache.\end{cacheText}

\cacheNumber{2204}\needspace{5\baselineskip}\cacheName{\href{http://coord.info/GC4FMNP}{7 / Chemin du marais / passerelle bois} — \href{http://coord.info/GC4FMNP\Number{}809724871}{2204}}\cacheData{{2018/10/07 lulu\Underscore{}et\Underscore{}compagnie, Traditional Cache (2/2)}}\begin{cacheText}Venus dans le Marais Poitevin pour participer à l’évent le GTAQ fait de la barque, la team 640 composée de Dune33, Isaasia et moi-même décide de prolonger le séjour en compagnie de Domino50 et de Crispol40. Toutes les caches seront trouvées avec plus ou moins de difficultés dans un décor de rêve. Merci pour la cache.\end{cacheText}

\cacheNumber{2205}\needspace{5\baselineskip}\cacheName{\href{http://coord.info/GC4FMNW}{8/ Chemin du marais / Conche} — \href{http://coord.info/GC4FMNW\Number{}809725424}{2205}}\cacheData{{2018/10/07 lulu\Underscore{}et\Underscore{}compagnie, Traditional Cache (2/2)}}\begin{cacheText}Venus dans le Marais Poitevin pour participer à l’évent le GTAQ fait de la barque, la team 640 composée de Dune33, Isaasia et moi-même décide de prolonger le séjour en compagnie de Domino50 et de Crispol40. Toutes les caches seront trouvées avec plus ou moins de difficultés dans un décor de rêve. Merci pour la cache.\end{cacheText}

\cacheNumber{2206}\needspace{5\baselineskip}\cacheName{\href{http://coord.info/GC4FMP0}{9 / Chemin du marais / Rigole} — \href{http://coord.info/GC4FMP0\Number{}809731640}{2206}}\cacheData{{2018/10/07 lulu\Underscore{}et\Underscore{}compagnie, Traditional Cache (2/2)}}\begin{cacheText}Venus dans le Marais Poitevin pour participer à l’évent le GTAQ fait de la barque, la team 640 composée de Dune33, Isaasia et moi-même décide de prolonger le séjour en compagnie de Domino50 et de Crispol40. Toutes les caches seront trouvées avec plus ou moins de difficultés dans un décor de rêve. Merci pour la cache.\end{cacheText}

\cacheNumber{2207}\needspace{5\baselineskip}\cacheName{\href{http://coord.info/GC4FMP2}{10 / Chemin du marais / letter} — \href{http://coord.info/GC4FMP2\Number{}810361466}{2207}}\cacheData{{2018/10/07 lulu\Underscore{}et\Underscore{}compagnie, Letterbox Hybrid (2/2)}}\begin{cacheText}Venus dans le Marais Poitevin pour participer à l’évent le GTAQ fait de la barque, la team 640 composée de Dune33, Isaasia et moi-même décide de prolonger le séjour en compagnie de Domino50 et de Crispol40. Toutes les caches seront trouvées avec plus ou moins de difficultés dans un décor de rêve. Merci pour la cache.\end{cacheText}

\cacheNumber{2208}\needspace{5\baselineskip}\cacheName{\href{http://coord.info/GC4FMP7}{11 /Chemin du marais / passerelle métalique} — \href{http://coord.info/GC4FMP7\Number{}809736667}{2208}}\cacheData{{2018/10/07 lulu\Underscore{}et\Underscore{}compagnie, Traditional Cache (2/2.5)}}\begin{cacheText}Venus dans le Marais Poitevin pour participer à l’évent le GTAQ fait de la barque, la team 640 composée de Dune33, Isaasia et moi-même décide de prolonger le séjour en compagnie de Domino50 et de Crispol40. Toutes les caches seront trouvées avec plus ou moins de difficultés dans un décor de rêve. Merci pour la cache.\end{cacheText}

\cacheNumber{2209}\needspace{5\baselineskip}\cacheName{\href{http://coord.info/GC4FMP8}{12 / Chemin du marais / retour} — \href{http://coord.info/GC4FMP8\Number{}809736952}{2209}}\cacheData{{2018/10/07 lulu\Underscore{}et\Underscore{}compagnie, Traditional Cache (2/2)}}\begin{cacheText}Venus dans le Marais Poitevin pour participer à l’évent le GTAQ fait de la barque, la team 640 composée de Dune33, Isaasia et moi-même décide de prolonger le séjour en compagnie de Domino50 et de Crispol40. Toutes les caches seront trouvées avec plus ou moins de difficultés dans un décor de rêve. Merci pour la cache.\end{cacheText}

\cacheNumber{2210}\needspace{5\baselineskip}\cacheName{\href{http://coord.info/GC5BX3Q}{Commune des Deux-Sévres: Coulon} — \href{http://coord.info/GC5BX3Q\Number{}809721036}{2210}}\cacheData{{2018/10/07 lulu\Underscore{}et\Underscore{}compagnie, Traditional Cache (2/1.5)}}\begin{cacheText}Venus dans le Marais Poitevin pour participer à l’évent le GTAQ fait de la barque, la team 640 composée de Dune33, Isaasia et moi-même décide de prolonger le séjour en compagnie de Domino50 et de Crispol40. Toutes les caches seront trouvées avec plus ou moins de difficultés dans un décor de rêve. Merci pour la cache.\end{cacheText}

\cacheNumber{2211}\needspace{5\baselineskip}\cacheName{\href{http://coord.info/GC5K2EN}{Les rigoles du Marais} — \href{http://coord.info/GC5K2EN\Number{}810765085}{2211}}\cacheData{{2018/10/07 lulu\Underscore{}et\Underscore{}compagnie, Earthcache (1.5/1.5)}}\begin{cacheText}La cache est validée par l’owner. Merci lulu et compagnie pour ce superbe parcours dans le marais .Un PF pour l’ensemble du travail.\end{cacheText}

\cacheNumber{2212}\needspace{5\baselineskip}\cacheName{\href{http://coord.info/GC6HP2E}{6 / Chemin du marais V2 / Vache} — \href{http://coord.info/GC6HP2E\Number{}809724628}{2212}}\cacheData{{2018/10/07 lulu\Underscore{}et\Underscore{}compagnie, Traditional Cache (2/2)}}\begin{cacheText}Venus dans le Marais Poitevin pour participer à l’évent le GTAQ fait de la barque, la team 640 composée de Dune33, Isaasia et moi-même décide de prolonger le séjour en compagnie de Domino50 et de Crispol40. Toutes les caches seront trouvées avec plus ou moins de difficultés dans un décor de rêve. Merci pour la cache.\end{cacheText}

\cacheNumber{2213}\needspace{5\baselineskip}\cacheName{\href{http://coord.info/GC7BN07}{13 / chemin du marais / ragondin} — \href{http://coord.info/GC7BN07\Number{}809733550}{2213}}\cacheData{{2018/10/07 lulu\Underscore{}et\Underscore{}compagnie, Traditional Cache (2/2.5)}}\begin{cacheText}Venus dans le Marais Poitevin pour participer à l’évent le GTAQ fait de la barque, la team 640 composée de Dune33, Isaasia et moi-même décide de prolonger le séjour en compagnie de Domino50 et de Crispol40. Toutes les caches seront trouvées avec plus ou moins de difficultés dans un décor de rêve. Je récupère le TB.Merci pour la cache.\end{cacheText}

\cacheNumber{2214}\needspace{5\baselineskip}\cacheName{\href{http://coord.info/GC7BN0E}{14 / chemin du marais / ecrevisse} — \href{http://coord.info/GC7BN0E\Number{}809733843}{2214}}\cacheData{{2018/10/07 lulu\Underscore{}et\Underscore{}compagnie, Traditional Cache (2/2.5)}}\begin{cacheText}Venus dans le Marais Poitevin pour participer à l’évent le GTAQ fait de la barque, la team 640 composée de Dune33, Isaasia et moi-même décide de prolonger le séjour en compagnie de Domino50 et de Crispol40. Toutes les caches seront trouvées avec plus ou moins de difficultés dans un décor de rêve. Merci pour la cache.\end{cacheText}

\cacheNumber{2215}\needspace{5\baselineskip}\cacheName{\href{http://coord.info/GC7N1HK}{15/ Chemin du marais/ le pont} — \href{http://coord.info/GC7N1HK\Number{}809734052}{2215}}\cacheData{{2018/10/07 lulu\Underscore{}et\Underscore{}compagnie, Traditional Cache (3/3)}}\begin{cacheText}Venus dans le Marais Poitevin pour participer à l’évent le GTAQ fait de la barque, la team 640 composée de Dune33, Isaasia et moi-même décide de prolonger le séjour en compagnie de Domino50 et de Crispol40. Toutes les caches seront trouvées avec plus ou moins de difficultés dans un décor de rêve. Merci pour la cache.\end{cacheText}

\cacheNumber{2216}\needspace{5\baselineskip}\cacheName{\href{http://coord.info/GC7N782}{16/ Chemin du marais/ à la croisée des chemin} — \href{http://coord.info/GC7N782\Number{}809734427}{2216}}\cacheData{{2018/10/07 lulu\Underscore{}et\Underscore{}compagnie, Traditional Cache (2/2)}}\begin{cacheText}Venus dans le Marais Poitevin pour participer à l’évent le GTAQ fait de la barque, la team 640 composée de Dune33, Isaasia et moi-même décide de prolonger le séjour en compagnie de Domino50 et de Crispol40. Toutes les caches seront trouvées avec plus ou moins de difficultés dans un décor de rêve. Merci pour la cache.\end{cacheText}

\cacheNumber{2217}\needspace{5\baselineskip}\cacheName{\href{http://coord.info/GC7ED9M}{Stade de football de Coulon} — \href{http://coord.info/GC7ED9M\Number{}809722284}{2217}}\cacheData{{2018/10/07 Grap88, Traditional Cache (1.5/1.5)}}\begin{cacheText}Venus dans le Marais Poitevin pour participer à l’évent le GTAQ fait de la barque, la team 640 composée de Dune33, Isaasia et moi-même décide de prolonger le séjour en compagnie de Domino50 et de Crispol40. Toutes les caches seront trouvées avec plus ou moins de difficultés dans un décor de rêve. Merci pour la cache.\end{cacheText}

\cacheNumber{2218}\needspace{5\baselineskip}\cacheName{\href{http://coord.info/GC66JVQ}{0 / Chemin du marais / la petite passerelle} — \href{http://coord.info/GC66JVQ\Number{}809722784}{2218}}\cacheData{{2018/10/07 lulu\Underscore{}et\Underscore{}compagnie, Traditional Cache (2/1.5)}}\begin{cacheText}Venus dans le Marais Poitevin pour participer à l’évent le GTAQ fait de la barque, la team 640 composée de Dune33, Isaasia et moi-même décide de prolonger le séjour en compagnie de Domino50 et de Crispol40. Toutes les caches seront trouvées avec plus ou moins de difficultés dans un décor de rêve. Merci pour la cache.\end{cacheText}

\cacheNumber{2219}\needspace{5\baselineskip}\cacheName{\href{http://coord.info/GC4FMMJ}{1/Chemin du marais 1/ c'est parti.} — \href{http://coord.info/GC4FMMJ\Number{}809733216}{2219}}\cacheData{{2018/10/07 lulu\Underscore{}et\Underscore{}compagnie, Traditional Cache (2/2)}}\begin{cacheText}Venus dans le Marais Poitevin pour participer à l’évent le GTAQ fait de la barque, la team 640 composée de Dune33, Isaasia et moi-même décide de prolonger le séjour en compagnie de Domino50 et de Crispol40. Toutes les caches seront trouvées avec plus ou moins de difficultés dans un décor de rêve. Merci pour la cache.\end{cacheText}

\cacheNumber{2220}\needspace{5\baselineskip}\cacheName{\href{http://coord.info/GC3VV50}{[EMP] Herbier} — \href{http://coord.info/GC3VV50\Number{}807565358}{2220}}\cacheData{{2018/10/08 lulu\Underscore{}et\Underscore{}compagnie, Multi-cache (2/1.5)}}\begin{cacheText}C’est en compagnie de Dune33 ,Isa Asia Crispol40 et Domino50 que nous nous attaquons à cette petite Multi. Très intéressante ,nous en apprenons  beaucoup sur les divers végétaux. Coordonnées en poche nous filons au PZ et grâce à notre ami Dédé nous finissons par mettre la main sur la belle. Un grand merci pour cette cache.\end{cacheText}

\cacheNumber{2221}\needspace{5\baselineskip}\cacheName{\href{http://coord.info/GC3VCPM}{[EMP] Barriere du marais} — \href{http://coord.info/GC3VCPM\Number{}810305892}{2221}}\cacheData{{2018/10/08 lulu\Underscore{}et\Underscore{}compagnie, Multi-cache (2/2)}}\begin{cacheText}Nous avons été surpris par la nuit hier au soir : impossible de trouver l’ancien panneau!!!! Ce matin la team640 y voit beaucoup plus clair!!! Tous les indices en poche nous mettons le cap sur le PZ final. Et quel final !!! Merci pour cette super multi.\end{cacheText}

\cacheNumber{2222}\needspace{5\baselineskip}\cacheName{\href{http://coord.info/GC4PKWJ}{Lavoir de la Courance} — \href{http://coord.info/GC4PKWJ\Number{}810364874}{2222}}\cacheData{{2018/10/08 lulu\Underscore{}et\Underscore{}compagnie, Traditional Cache (2/2.5)}}\begin{cacheText}Venus dans le Marais Poitevin pour participer à l’évent le GTAQ fait de la barque, la team 640 composée de Dune33, Isaasia et moi-même décide de prolonger le séjour en compagnie de Domino50.Nous découvrons ici un charmant petit lavoir. C’est Isaasia, très souple, qui finit par mettre la main dessus. Merci pour la cache.\end{cacheText}

\cacheNumber{2223}\needspace{5\baselineskip}\cacheName{\href{http://coord.info/GC58KHW}{Commune des Deux-Sévres: Epannes} — \href{http://coord.info/GC58KHW\Number{}810365191}{2223}}\cacheData{{2018/10/08 lulu\Underscore{}et\Underscore{}compagnie, Traditional Cache (1.5/1.5)}}\begin{cacheText}Venus dans le Marais Poitevin pour participer à l’évent le GTAQ fait de la barque, la team 640 composée de Dune33, Isaasia et moi-même décide de prolonger le séjour en compagnie de Domino50.Nous découvrons ici une charmante petite église et la cache ne nous résiste pas. Il est temps de se quitter avec Domino50 : chacun reprend sa route. Merci pour la cache.\end{cacheText}

\cacheNumber{2224}\needspace{5\baselineskip}\cacheName{\href{http://coord.info/GC5BX3N}{Commune des Deux-Sévres: Sansais-La Garette} — \href{http://coord.info/GC5BX3N\Number{}810364242}{2224}}\cacheData{{2018/10/08 lulu\Underscore{}et\Underscore{}compagnie, Traditional Cache (1.5/1.5)}}\begin{cacheText}Venus dans le Marais Poitevin pour participer à l’évent le GTAQ fait de la barque, la team 640 composée de Dune33, Isaasia et moi-même décide de prolonger le séjour en compagnie de Domino50 .Malgré les  DNF précédents nous ne nous décourageons pas. Les huit yeux dénichent enfin la Belle à côté d’une jolie Ciste. Merci pour la cache.\end{cacheText}

\cacheNumber{2225}\needspace{5\baselineskip}\cacheName{\href{http://coord.info/GC5FF6P}{Auto-Stop A10 : Aire de Saugon Ouest N-->S} — \href{http://coord.info/GC5FF6P\Number{}810365333}{2225}}\cacheData{{2018/10/08 \$24.RC, Traditional Cache (1.5/1.5)}}\begin{cacheText}C’est sur le retour du GTAQ fait de la barque à Coulon que nous loguons sous le pseudo Team 640 en compagnie de Dune33 et Isaasia. Cette cache nous oblige à faire une pause. Le PZ est désert et la Belle est vite délogée.Merci pour la cache.\end{cacheText}

\cacheNumber{2226}\needspace{5\baselineskip}\cacheName{\href{http://coord.info/GC5HCPC}{A10 - Aire de FENIOUX Ouest} — \href{http://coord.info/GC5HCPC\Number{}810365262}{2226}}\cacheData{{2018/10/08 eureka17, Traditional Cache (1.5/1.5)}}\begin{cacheText}Sur le retour du GTAQ fait de la barque à Coulon nous loguons sous le pseudo Team 640 en compagnie de Dune33 et Isaasia. Cette cache nous oblige à faire une pause. Merci pour la cache.\end{cacheText}

\cacheNumber{2227}\needspace{5\baselineskip}\cacheName{\href{http://coord.info/GC6RBXK}{Dolmens d'Amuré} — \href{http://coord.info/GC6RBXK\Number{}810364162}{2227}}\cacheData{{2018/10/08 Trofortish, Traditional Cache (1/1)}}\begin{cacheText}Venus dans le Marais Poitevin pour participer à l’évent le GTAQ fait de la barque, la team 640 ,composée de Dune33, Isaasia et moi-même ,décide de prolonger le séjour en compagnie de Domino50. Il est très surprenant de trouver ces deux dolmens plantés dans le cimetière. Nous respectons les lieux : la cache est effectivement trouvée très rapidement. Merci Trofortish pour cette découverte.\end{cacheText}

\cacheNumber{2228}\needspace{5\baselineskip}\cacheName{\href{http://coord.info/GC6RJKK}{Le Port Rexois} — \href{http://coord.info/GC6RJKK\Number{}810361669}{2228}}\cacheData{{2018/10/08 Trofortish, Traditional Cache (1.5/1.5)}}\begin{cacheText}Venus dans le Marais Poitevin pour participer à l’évent le GTAQ fait de la barque, la team 640 composée de Dune33, Isaasia et moi-même décide de prolonger le séjour en compagnie de Domino50 .Sur le village pour trouver la terra aventura, nous en profitons au passage pour loguer cette cache. Nous découvrons un port superbe et la cache est judicieusement camouflée.Merci pour cette belle découverte.\end{cacheText}

\cacheNumber{2229}\needspace{5\baselineskip}\cacheName{\href{http://coord.info/GC6RKEG}{Le Pigeonnier Rexois} — \href{http://coord.info/GC6RKEG\Number{}810361784}{2229}}\cacheData{{2018/10/08 Trofortish, Traditional Cache (1.5/1)}}\begin{cacheText}Venus dans le Marais Poitevin pour participer à l’évent le GTAQ fait de la barque, la team 640 composée de Dune33, Isaasia et moi-même décide de prolonger le séjour en compagnie de Domino50 . Sur le village pour trouver la terra aventura, nous en profitons pour découvrir ce superbe pigeonnier et ses boulins. Merci pour la cache.\end{cacheText}

\cacheNumber{2230}\needspace{5\baselineskip}\cacheName{\href{http://coord.info/GC7CG7W}{Les gardiens du marais} — \href{http://coord.info/GC7CG7W\Number{}810364068}{2230}}\cacheData{{2018/10/08 Terraaventura, Multi-cache (3/2)}}\begin{cacheText}Venus dans le Marais Poitevin pour participer à l’évent le GTAQ fait de la barque, la team 640 composée de Dune33, Isaasia et moi-même décide de prolonger le séjour en compagnie de Domino50 .Pas question de rentrer sans faire la Terra Aventura!!! Après avoir découvert ce charmant village et ses marais nous mettons la main sur la cache. Un poitz en souvenir et hop nous prenons la direction du retour. Merci pour cet excellent moment.\end{cacheText}

\cacheNumber{2231}\needspace{5\baselineskip}\cacheName{\href{http://coord.info/GC7MKGE}{Toi qui passes sans me voir !!!!} — \href{http://coord.info/GC7MKGE\Number{}810364749}{2231}}\cacheData{{2018/10/08 lulu\Underscore{}et\Underscore{}compagnie, Traditional Cache (4/2)}}\begin{cacheText}Venus dans le Marais Poitevin pour participer à l’évent le GTAQ fait de la barque, la team 640 composée de Dune33, Isaasia et moi-même décide de prolonger le séjour en compagnie de Domino50.Elle nous en a donné du mal celle-là ! Mais à force de persévérance nous finissons par la déloger. Merci pour la cache.\end{cacheText}

\cacheNumber{2232}\needspace{5\baselineskip}\cacheName{\href{http://coord.info/GC7WPKJ}{petit chemin} — \href{http://coord.info/GC7WPKJ\Number{}810365083}{2232}}\cacheData{{2018/10/08 FamilyBerth, Traditional Cache (1.5/1)}}\begin{cacheText}Venus dans le Marais Poitevin pour participer à l’évent le GTAQ fait de la barque, la team 640 composée de Dune33, Isaasia et moi-même décide de prolonger le séjour en compagnie de Domino50.Aucun moldu à l’horizon, nous pouvons chercher tranquille. La belle est vite délogée grâce a nos 8 yeux. Merci pour la cache.\end{cacheText}

\cacheNumber{2233}\needspace{5\baselineskip}\cacheName{\href{http://coord.info/GC7WVKN}{Les amandiers} — \href{http://coord.info/GC7WVKN\Number{}810364960}{2233}}\cacheData{{2018/10/08 FamilyBerth, Traditional Cache (1.5/1)}}\begin{cacheText}Venus dans le Marais Poitevin pour participer à l’évent le GTAQ fait de la barque, la team 640 composée de Dune33, Isaasia et moi-même décide de prolonger le séjour en compagnie de Domino50.Cette cache se trouve sur notre route du retour et nous nous arrêtons pour la déloger. Merci pour la cache.\end{cacheText}

\cacheNumber{2234}\needspace{5\baselineskip}\cacheName{\href{http://coord.info/GC5XE0M}{Hippodrome DAX} — \href{http://coord.info/GC5XE0M\Number{}810365848}{2234}}\cacheData{{2018/10/13 mizaga, Traditional Cache (1.5/1.5)}}\begin{cacheText}En route pour Castets, j’en profite pour faire quelques caches. Arrivée au PZ, je cherche au pied des lauriers. Le joli camouflage est enfin débusqué  .Merci les Mizaga pour cette super cache.\end{cacheText}

\cacheNumber{2235}\needspace{5\baselineskip}\cacheName{\href{http://coord.info/GC65KVK}{bienvenue 6} — \href{http://coord.info/GC65KVK\Number{}810365593}{2235}}\cacheData{{2018/10/13 lauki3940, Traditional Cache (1.5/1.5)}}\begin{cacheText}Sur la route de Castets , une petite pause pour valider cette cache qui est trouvée sans difficulté. Merci pour la cache.\end{cacheText}

\cacheNumber{2236}\needspace{5\baselineskip}\cacheName{\href{http://coord.info/GC66HYR}{Castets : le Sequoia} — \href{http://coord.info/GC66HYR\Number{}810470115}{2236}}\cacheData{{2018/10/13 DorisBear, Traditional Cache (1/1)}}\begin{cacheText}C’est pour l’initiation géocaching que je suis sur Castets . Un petit repérage du parcours s’impose. Je découvre un magnifique Séquoia et une cache bien dissimulée. Merci les Dorisbear.\end{cacheText}

\cacheNumber{2237}\needspace{5\baselineskip}\cacheName{\href{http://coord.info/GC66HYX}{Castets : les Forges} — \href{http://coord.info/GC66HYX\Number{}810470437}{2237}}\cacheData{{2018/10/13 DorisBear, Traditional Cache (2/2)}}\begin{cacheText}Suite du parcours ,accompagnée d’Isaasia et de Domino50. Je découvre une magnifique forge qui malheureusement est un peu à l’abandon. Quel dommage ! Après avoir repéré les lieux la cache est trouvée.Encore du superbe travail digne des Dorisbear. Un PF évidemment. Merci pour la cache.\end{cacheText}

\cacheNumber{2238}\needspace{5\baselineskip}\cacheName{\href{http://coord.info/GC69Y1M}{Castets : de l'arbre à la cloche} — \href{http://coord.info/GC69Y1M\Number{}810470765}{2238}}\cacheData{{2018/10/13 DorisBear, Multi-cache (2/1.5)}}\begin{cacheText}Cette petite Multi est réalisée en compagnie d’Isaasia et de Domino50. L’arbre est reconnaissable… Mais heureusement que l’ami Google est là pour me donner son nom latin. Les coordonnées sont résolues et la cache vite trouvée lors de l’initiation géocaching. Merci pour la cache.\end{cacheText}

\cacheNumber{2239}\needspace{5\baselineskip}\cacheName{\href{http://coord.info/GC6DZE9}{Poisson d'Avril} — \href{http://coord.info/GC6DZE9\Number{}810761270}{2239}}\cacheData{{2018/10/13 DorisBear, Unknown Cache (2/2)}}\begin{cacheText}Après la chasse aux œufs de tourterelles, Isaasia ,Crispol40 et moi-même optons pour une partie de pêche à la cache. Arrivés au PZ, nous découvrons un petit poisson qui nous livre son secret. Nous enchaînons sur la pêche au gros!!! La forêt est magnifique avec ses couleurs d’automne et le poisson se fait désirer mais à près un certain temps nous finissons par ferrer la Belle. Quelle prise,quelle œuvre !!! Encore un PF bien mérité. Merci les amis pour ce bon moment passé.\end{cacheText}

\cacheNumber{2240}\needspace{5\baselineskip}\cacheName{\href{http://coord.info/GC6DZFH}{Casse-tête à Castets} — \href{http://coord.info/GC6DZFH\Number{}810479458}{2240}}\cacheData{{2018/10/13 DorisBear, Unknown Cache (4/5)}}\begin{cacheText}C’est à la fin de l’initiation geocaching qu’Isaasia, Dune33, Crispol 40 et moi-même partons vers le casse-tête. La cache porte bien son nom : casse-tête pour résoudre l’énigme, casse-tête pour attraper la cache, casse-tête pour l’ouvrir et enfin un casse-tête pour tout remettre à sa place . Arrivé au PZ, l’arbre est rapidement repéré.C’est l’agile Isaasia qui décide de grimper à l’arbre et nous faire passer la cache.Une fois au sol, le reste de la troupe s’évertue à ouvrir la Belle. Après un long moment de résistance, enfin le Saint Graal. Voici encore du grand Art, une cache comme on aimerait en trouver plus souvent. Un gros PF et un grand merci à mes ours préférés.\end{cacheText}

\cacheNumber{2241}\needspace{5\baselineskip}\cacheName{\href{http://coord.info/GC6DZG0}{Castets : la salle des fêtes} — \href{http://coord.info/GC6DZG0\Number{}810470923}{2241}}\cacheData{{2018/10/13 DorisBear, Traditional Cache (1.5/1.5)}}\begin{cacheText}Toujours en compagnie de Isaasia et de Domino50, je découvre  une super idée de camouflage. La façade de l’ancienne salle des fêtes est très sympa.Merci pour la cache.\end{cacheText}

\cacheNumber{2242}\needspace{5\baselineskip}\cacheName{\href{http://coord.info/GC6GC0C}{Castets : le moustique} — \href{http://coord.info/GC6GC0C\Number{}810761345}{2242}}\cacheData{{2018/10/13 DorisBear, Traditional Cache (1.5/1.5)}}\begin{cacheText}Il y a des activités dont on ne se lasse pas… le géocaching en fait partie.Avant de prendre la route, petit arrêt avec Isaasia pour dénicher la Belle. A cette heure ci, il n’y a pas grand moldu, nous pouvons donc explorer le rond-point tranquille.Apres quelques recherches, Isa met la main sur le moustique. Nous simulons une séance photo pour distraire les automobilistes et loguer la cache. Merci les Dorisbear .\end{cacheText}

\cacheNumber{2243}\needspace{5\baselineskip}\cacheName{\href{http://coord.info/GC6TYXR}{Le ruisseau des Forges} — \href{http://coord.info/GC6TYXR\Number{}810470246}{2243}}\cacheData{{2018/10/13 tichivi, Traditional Cache (1.5/1.5)}}\begin{cacheText}C’est en compagnie d’Isaasia et de Domino50 que je procède au repérage du parcours pour l’initiation géocaching. La cache est délogée assez rapidement . Merci pour la cache.\end{cacheText}

\cacheNumber{2244}\needspace{5\baselineskip}\cacheName{\href{http://coord.info/GC6TYY1}{Le chemin de Barrats} — \href{http://coord.info/GC6TYY1\Number{}810471601}{2244}}\cacheData{{2018/10/13 tichivi, Traditional Cache (1.5/1.5)}}\begin{cacheText}Aujourd’hui c’est journée Bidouille à Castets. Invitée à participer à l’initiation  géocaching, je fais un repérage en compagnie d’Isa Asia et de Domino50. La cache est vite repérée et elle nous fait découvrir un très joli parc. Merci pour la cache.\end{cacheText}

\cacheNumber{2245}\needspace{5\baselineskip}\cacheName{\href{http://coord.info/GC74PMR}{Demis de mélée} — \href{http://coord.info/GC74PMR\Number{}810720079}{2245}}\cacheData{{2018/10/13 DorisBear, Unknown Cache (2.5/2)}}\begin{cacheText}Après le casse tète tant cérébral que physique, quoi de mieux qu'une partie de rugby pour se dégourdir les jambes. L'énigme ,décodée de longue date , nous permet d'atteindre le terrain de jeu que nous explorons en long, en large et en travers. Le coup de sifflet de l'ami Paul retentit et c'est grâce au placage au sol que nous marquons l'essai .Et quel essai!!!Un vrai coup de coeur pour cette superbe cache et donc un PF comme trophé pour mon équipe préférée.\end{cacheText}

\cacheNumber{2246}\needspace{5\baselineskip}\cacheName{\href{http://coord.info/GC75HGG}{*Rosalie*} — \href{http://coord.info/GC75HGG\Number{}810761141}{2246}}\cacheData{{2018/10/13 DorisBear, Traditional Cache (3/2.5)}}\begin{cacheText}Et nous voilà partis (Isaasia, Crispol40 et moi même) découvrir le nid de la Belle Rosalie. Elle est bien cachée dans cette immense forêt mais nous finissons par trouver l’arbre qui l’abrite. Et ici encore , nous pouvons observer un travail d’artiste avec un super mécanisme. Nous prenons la direction de la couvée qui est bien à l’abris pour récupérer des rejetons. Il se pourrait qu’un oeuf vienne à éclore dans le 64.... Un PF pour cette super idée .\end{cacheText}

\cacheNumber{2247}\needspace{5\baselineskip}\cacheName{\href{http://coord.info/GC6JMJ5}{RDV-1} — \href{http://coord.info/GC6JMJ5\Number{}810761414}{2247}}\cacheData{{2018/10/14 mizaga, Traditional Cache (1.5/1.5)}}\begin{cacheText}C’est sur le retour de l’initiation j’ai au caching à casse-tête que nous délogeons en compagnie d’Isa Asia cette cache. La nuit tombe il est temps d’arrêter et de rentrer à la maison. Merci pour la cache.\end{cacheText}

\cacheNumber{2248}\needspace{5\baselineskip}\cacheName{\href{http://coord.info/GC62QV3}{La fontaine d'Amaniou} — \href{http://coord.info/GC62QV3\Number{}809102457}{2248}}\cacheData{{2018/10/14 lauki3940, Earthcache (1.5/1.5)}}\begin{cacheText}De retour de Capbreton,un petit arrêt s'impose pour découvrir la Fontaine .C'est un véritable chef d'œuvre réalisé par les compagnons du devoir.Merci Lauki 3940 pour cette earthcache très instructive .\end{cacheText}

\cacheNumber{2249}\needspace{5\baselineskip}\cacheName{\href{http://coord.info/GC65D07}{Covoiturage Aire de Begaar / Carpooling Begaar} — \href{http://coord.info/GC65D07\Number{}812997660}{2249}}\cacheData{{2018/10/21 dorisbear, Traditional Cache (1.5/1.5)}}\begin{cacheText}De retour de la XL Centre, petit arrêt sur l’aire de covoiturage en compagnie de l’owner lui-même. Il faut être discret ,des moldus stationnent à côté du PZ. À force de diversion je finis par loguer la Belle. Du beau travail. Merci pour la cache.\end{cacheText}

\cacheNumber{2250}\needspace{5\baselineskip}\cacheName{\href{http://coord.info/GC6RQ7R}{L’église Saint Martin d'Estampon} — \href{http://coord.info/GC6RQ7R\Number{}812840549}{2250}}\cacheData{{2018/10/21 Duodenfer, Multi-cache (2.5/1.5)}}\begin{cacheText}C’est sur le retour de la XL Est que nous découvrons en compagnie de Dune33, DorisBear et puma qui grogne ,ce superbe petit village qui semble oublié du monde. Nous découvrons une superbe charpente avec ses indices et une magnifique église qui mériterait une rénovation. La cache est vite délogée et nous repartons en admirant le magnifique platane. Merci pour la cache.\end{cacheText}

\cacheNumber{2251}\needspace{5\baselineskip}\cacheName{\href{http://coord.info/GC6T3J4}{Cimetière d'Estampon} — \href{http://coord.info/GC6T3J4\Number{}812995954}{2251}}\cacheData{{2018/10/21 Duodenfer, Traditional Cache (2.5/1.5)}}\begin{cacheText}Arrivés au cimetière, nous découvrons un magnifique ancien corbillard qui semble avoir été oublié. Après avoir inspecté les différentes tombes,nous finissons par découvrir Madeleine et Lucy. Direction l’Est ... et nous mettons la main sur la Belle. Merci pour la cache.\end{cacheText}

\cacheNumber{2252}\needspace{5\baselineskip}\cacheName{\href{http://coord.info/GC7X2H0}{XL Est 01} — \href{http://coord.info/GC7X2H0\Number{}811622784}{2252}}\cacheData{{2018/10/21 crispol40, Unknown Cache (3.5/1)}}\begin{cacheText}Aujourd’hui c’est le grand jour.Nous prenons la direction ,en compagnie de Madame Dorisbear ,de la XL Est. Dune33 et puma qui grogne sont aussi de la partie. Le brouillard nous a accompagné une bonne partie de la matinée mais il n’a pas entamé notre bonne humeur.

À peine sortis des voitures, Madame Dorisbear déloge la Belle derrière sa liane pendant que nous préparons les sacs à dos. Merci pour la cache.\end{cacheText}

\cacheNumber{2253}\needspace{5\baselineskip}\cacheName{\href{http://coord.info/GC7X2K0}{XL Est 02} — \href{http://coord.info/GC7X2K0\Number{}811622970}{2253}}\cacheData{{2018/10/21 crispol40, Unknown Cache (4/3.5)}}\begin{cacheText}Arrivés sur le PZ, les GPS nous promènent un peu mais nous finissons par repérer la belle en haut de l’arbre !

Pas simple d’y monter ! Puma qui grogne se dévoue et en deux temps trois mouvements attrape la cache. Merci  Paul\end{cacheText}

\cacheNumber{2254}\needspace{5\baselineskip}\cacheName{\href{http://coord.info/GC7X32K}{XL Est 03} — \href{http://coord.info/GC7X32K\Number{}811623131}{2254}}\cacheData{{2018/10/21 crispol40, Unknown Cache (3.5/1.5)}}\begin{cacheText}Rongé, rongé.... voilà l’indice ! Nous cherchons donc quelque chose de rongé et enfin nous trouvons un arbre rongé … mais RIEN.C’est Madame Bear qui finit par mettre le pied dessus et ..l’indice prend tout son sens. Excellent , un PF. Merci pour la cache.\end{cacheText}

\cacheNumber{2255}\needspace{5\baselineskip}\cacheName{\href{http://coord.info/GC7X32W}{XL Est 04} — \href{http://coord.info/GC7X32W\Number{}811623333}{2255}}\cacheData{{2018/10/21 crispol40, Unknown Cache (1/2.5)}}\begin{cacheText}Nous poursuivons la promenade en admirant les Coulemelles.Dès que nous approchons du PZ, Dune33 ,avec son œil de lynx ,repère la cache. Merci pour la cache.\end{cacheText}

\cacheNumber{2256}\needspace{5\baselineskip}\cacheName{\href{http://coord.info/GC7X335}{XL Est 05} — \href{http://coord.info/GC7X335\Number{}811623474}{2256}}\cacheData{{2018/10/21 crispol40, Unknown Cache (3.5/1.5)}}\begin{cacheText}Nous continuons le chemin toujours dans la bonne humeur et nous admirons le spectacle de la nature automnale. Pas de soucis pour la cache:les 8 yeux sont bien ouverts. Merci pour la cache.\end{cacheText}

\cacheNumber{2257}\needspace{5\baselineskip}\cacheName{\href{http://coord.info/GC7X33D}{XL Est 06} — \href{http://coord.info/GC7X33D\Number{}811801131}{2257}}\cacheData{{2018/10/21 crispol40, Unknown Cache (4/1.5)}}\begin{cacheText}Ici l’indice nous perturbe un peu : nous inspectons donc le peu de petits trous qu'il y a dans le coin. À force de recherches, c’est puma qui grogne qui finit par la découvrir bien camouflée… Pourtant un classique ! Merci pour la cache.\end{cacheText}

\cacheNumber{2258}\needspace{5\baselineskip}\cacheName{\href{http://coord.info/GC7X33J}{XL Est 07} — \href{http://coord.info/GC7X33J\Number{}811803723}{2258}}\cacheData{{2018/10/21 crispol40, Unknown Cache (3/3.5)}}\begin{cacheText}Nous continuons le long du chemin et nous sommes dérangés par un 4*4 de chasseurs.Arrivés au PZ, un oufff de soulagement : la cache n’est pas trop haute.Merci pour la cache.\end{cacheText}

\cacheNumber{2259}\needspace{5\baselineskip}\cacheName{\href{http://coord.info/GC7X33P}{XL Est 08} — \href{http://coord.info/GC7X33P\Number{}811803997}{2259}}\cacheData{{2018/10/21 crispol40, Unknown Cache (4.5/1.5)}}\begin{cacheText}Le GPS nous promène ...un coup à droite ,un coup à gauche...C’est l’œil avisé de Diotima qui repère la belle, très bien cachée. Merci.\end{cacheText}

\cacheNumber{2260}\needspace{5\baselineskip}\cacheName{\href{http://coord.info/GC7X345}{XL Est 09} — \href{http://coord.info/GC7X345\Number{}812660772}{2260}}\cacheData{{2018/10/21 crispol40, Unknown Cache (5/1.5)}}\begin{cacheText}Nous arrivons dans le No man’s land :Les pins ont été coupés et il n’y a pas grand endroit pour mettre la cache.À force de recherches, nous découvrons la branchette bien camouflée . Merci pour la cache.\end{cacheText}

\cacheNumber{2261}\needspace{5\baselineskip}\cacheName{\href{http://coord.info/GC7X349}{XL Est 10} — \href{http://coord.info/GC7X349\Number{}812661158}{2261}}\cacheData{{2018/10/21 crispol40, Unknown Cache (3/1.5)}}\begin{cacheText}Ici, c’est puma qui grogne qui déloge la Belle en passant.Elle est bien à l’abri ! ! ! Merci pour la cache.\end{cacheText}

\cacheNumber{2262}\needspace{5\baselineskip}\cacheName{\href{http://coord.info/GC7X34C}{XL Est 11} — \href{http://coord.info/GC7X34C\Number{}812661485}{2262}}\cacheData{{2018/10/21 crispol40, Unknown Cache (4.5/1.5)}}\begin{cacheText}Nous longeons le grillage jusqu’au PZ et découvrons la Belle bien enfilée . Nous continuons le chemin avec nos bavardages.... merci Paul\end{cacheText}

\cacheNumber{2263}\needspace{5\baselineskip}\cacheName{\href{http://coord.info/GC7X34Q}{XL Est 12} — \href{http://coord.info/GC7X34Q\Number{}812661769}{2263}}\cacheData{{2018/10/21 crispol40, Unknown Cache (4.5/1.5)}}\begin{cacheText}Nous avons inspecté tous les pins de la zone indiquée par le GPS. Sur le point d’abandonner, puma qui grogne retrouve un bouchon et un bout de plastique. Alors que nous allions mettre un nouveau tube nous  apercevons l’original un peu plus loin. Assez voraces les bestioles!!!! Merci pour la cache.\end{cacheText}

\cacheNumber{2264}\needspace{5\baselineskip}\cacheName{\href{http://coord.info/GC7X34V}{XL Est 13} — \href{http://coord.info/GC7X34V\Number{}812662055}{2264}}\cacheData{{2018/10/21 crispol40, Unknown Cache (3.5/4)}}\begin{cacheText}Et c’est reparti pour l’escalade ! Heureusement que Jean-Luc est là pour s’atteler à la tâche. Nous le soutenons moralement et lui prodiguons nos bons conseils…Merci Paul.\end{cacheText}

\cacheNumber{2265}\needspace{5\baselineskip}\cacheName{\href{http://coord.info/GC7X351}{XL Est 14} — \href{http://coord.info/GC7X351\Number{}812662801}{2265}}\cacheData{{2018/10/21 crispol40, Unknown Cache (4/1.5)}}\begin{cacheText}Nous avons repris des forces lors de notre pique-nique improvisé sur le bord de la route. Arrivés au PZ nous délogeons la Belle et nous ne nous attardons pas car des frôlons asiatiques butinent les fleurs du lierre. Merci pour la cache.\end{cacheText}

\cacheNumber{2266}\needspace{5\baselineskip}\cacheName{\href{http://coord.info/GC7X355}{XL Est 16} — \href{http://coord.info/GC7X355\Number{}812666204}{2266}}\cacheData{{2018/10/21 crispol40, Unknown Cache (4/1.5)}}\begin{cacheText}Nous avons beau chercher la Belle est bien camouflée.C’est Denise qui finit par mettre le pied dessus. Belle intégration ! Merci pour la cache.\end{cacheText}

\cacheNumber{2267}\needspace{5\baselineskip}\cacheName{\href{http://coord.info/GC7X35C}{XL Est 15} — \href{http://coord.info/GC7X35C\Number{}812663083}{2267}}\cacheData{{2018/10/21 crispol40, Unknown Cache (1/2)}}\begin{cacheText}L’indice est assez clair...Et c’est Denise qui finit par mettre l’œil dessus. Pas facile de tout remettre en place , heureusement que Jean-Luc est là !!!Merci pour la cache.\end{cacheText}

\cacheNumber{2268}\needspace{5\baselineskip}\cacheName{\href{http://coord.info/GC7XAYZ}{XL Centre} — \href{http://coord.info/GC7XAYZ\Number{}812996383}{2268}}\cacheData{{2018/10/21 crispol40, Unknown Cache (4/1.5)}}\begin{cacheText}Les coordonnées en poche, grâce à puma qui grogne, la team 640 se dirige vers le centre.Le groupe se divise : deux devant et deux derrière. Un vrai chantier ! Le Graal est enfin repéré. Nous libérons le TB et partons satisfait de notre trouvaille. Merci Paul pour ce grand travail. Un PF pour l’ensemble.\end{cacheText}

\cacheNumber{2269}\needspace{5\baselineskip}\cacheName{\href{http://coord.info/GC42KNW}{Event SdG GR653 L\Underscore{}L-07} — \href{http://coord.info/GC42KNW\Number{}814185169}{2269}}\cacheData{{2018/11/03 Charnègues, Traditional Cache (1.5/1.5)}}\begin{cacheText}En route pour le Meet \And{} Greet à Lons d’Eloflomar, j’en profite pour faire quelques caches avant l’heure du rendez-vous. Je jette mon dévolu sur celle ci qui se trouve sur ma route. Ici, la découverte ne tenait qu’à un fil : sur le point de renoncer j’aperçois…. Merci Charnegues pour cette cache.\end{cacheText}

\cacheNumber{2270}\needspace{5\baselineskip}\cacheName{\href{http://coord.info/GC70Y93}{VVLT \Number{}19} — \href{http://coord.info/GC70Y93\Number{}814243022}{2270}}\cacheData{{2018/11/03 l.a.f.l.e.u.r, Traditional Cache (1.5/1.5)}}\begin{cacheText}En route pour le Meet \And{} Greet à Lons d'Eloflomar ,il me reste quelques minutes.J'en profite pour découvrir l’endroit qui est très sympa .Très fréquenté par les moldus  je furette et déloge la belle en toute discrétion. Merci pour cette découverte.\end{cacheText}

\cacheNumber{2271}\needspace{5\baselineskip}\cacheName{\href{http://coord.info/GC7Z06N}{Meet \And{} Greet à Lons} — \href{http://coord.info/GC7Z06N\Number{}814265611}{2271}}\cacheData{{2018/11/03 Eloflomar, Event Cache (1/1.5)}}\begin{cacheText}Et voilà c’est déjà terminé !!!Le temps passe toujours trop vite en bonne compagnie.Une grande réussite ce Meet \And{}Greet : de super participants ( nouveaux et anciens géo) qui ont géoblablaté toute la soirée,un excellent repas et pour finir un petit tour sur la piste de danse grâce à MadameGeo et coxigypaete.Un grand merci Eloflomar pour cet Évent\end{cacheText}

\cacheNumber{2272}\needspace{5\baselineskip}\cacheName{\href{http://coord.info/GC70TZY}{Ancienne Gare de Tarsacq} — \href{http://coord.info/GC70TZY\Number{}814182886}{2272}}\cacheData{{2018/11/03 l.a.f.l.e.u.r, Traditional Cache (1.5/2)}}\begin{cacheText}En route pour le Meet \And{} Greet à Lons d’Eloflomar, il me reste un peu de temps avant l’heure du rendez-vous.J’en profite donc pour jaunir la carte sur ce côté du département. Ici je découvre un charmant petit village et son ancienne gare très bien rénovée. La cache est efficace et rapidement trouvée grâce au spoiler. Merci.\end{cacheText}

\cacheNumber{2273}\needspace{5\baselineskip}\cacheName{\href{http://coord.info/GC4YQ7C}{Hiriburu : Ah, Quoi ? T'es là !} — \href{http://coord.info/GC4YQ7C\Number{}814605457}{2273}}\cacheData{{2018/11/06 Gamboy, Traditional Cache (2/1)}}\begin{cacheText}Arrivée sur le PZ je ne me fie qu’à l’indice et je fonce tout droit vers le transfo… Après un bon quart d'heure de recherches ,2 voitures se garent à proximité ….oups je dérange!!! Je décide d’abandonner lorsque je regarde le GPS …..qui me montre la cache!!! Je n'oublie pas de noter le nom de la maison pour la Bonus .Merci pour la cache.\end{cacheText}

\cacheNumber{2274}\needspace{5\baselineskip}\cacheName{\href{http://coord.info/GC68QRT}{\Number{}L2-09 Urcuit - Le Comté du Labourd} — \href{http://coord.info/GC68QRT\Number{}814601687}{2274}}\cacheData{{2018/11/06 gilles64, Multi-cache (2/1.5)}}\begin{cacheText}C’est mon troisième passage sur cette multi... et aujourd’hui BINGO grâce à la maintenance. Très joli point de vue avec les belles couleurs automnales . Indice relevé pour la Bonus. Merci Gilles.\end{cacheText}

\cacheNumber{2275}\needspace{5\baselineskip}\cacheName{\href{http://coord.info/GC7M0QM}{OUROUSPOURE} — \href{http://coord.info/GC7M0QM\Number{}814558529}{2275}}\cacheData{{2018/11/06 graftitouan64, Traditional Cache (1.5/1)}}\begin{cacheText}Sur Bayonne pour un rendez-vous J’en profite pour jaunir la carte aux alentours.Arrivée au PZ aucun \og{} Pétanqueur \fg{} sur le terrain. Quelle aubaine ! La cache estrapidement repérée. Magnifique saule pleureur.... Merci pour la cache.\end{cacheText}

\cacheNumber{2276}\needspace{5\baselineskip}\cacheName{\href{http://coord.info/GC53AY4}{En route vers l'Event} — \href{http://coord.info/GC53AY4\Number{}815024677}{2276}}\cacheData{{2018/11/09 Charnègues, Traditional Cache (1.5/1.5)}}\begin{cacheText}Sur le retour de la Multi d'Hastingues, je passe le portail pour aller chercher la Belle sur l'aire d'autoroute. C'est joliment aménagé et il n'y a pas de moldu à l'horizon. La cache est vite repérée.Merci\end{cacheText}

\cacheNumber{2277}\needspace{5\baselineskip}\cacheName{\href{http://coord.info/GC67DY4}{\Number{}L2-07 Mouguerre - Le Comté du Labourd} — \href{http://coord.info/GC67DY4\Number{}815015570}{2277}}\cacheData{{2018/11/09 gilles64, Multi-cache (2/1.5)}}\begin{cacheText}Les indices se laissent récolter sans difficulté. Je découvre un très joli village notamment le panorama de l’Obélisque qui est superbe. L’église et son vitrail sont tout à fait charmant. Calcul effectué, checker vert ...direction la cache qui est vite dénichée . L’indice est relevé pour la super bonus. Merci Gilles pour ce bon moment.\end{cacheText}

\cacheNumber{2278}\needspace{5\baselineskip}\cacheName{\href{http://coord.info/GC7P4H5}{La multi d'Hastingues} — \href{http://coord.info/GC7P4H5\Number{}815023858}{2278}}\cacheData{{2018/11/09 lauki3940 \And{} titiger39, Multi-cache (2/1.5)}}\begin{cacheText}(STF)

Elle m'en a donné du mal cette petite multi !!! Des sa sortie, j'étais sur le terrain pour récolter les indices dans ce très beau village médiéval. Le calcul des coordonnées a été plus que chaotique et quelques ajustements de l'owner ont été nécessaires pour atteindre le PZ final. Sur le terrain, la fouille a été minutieuse mais RIEN. J'ai donc abandonné!!!!Mais ce point orange sur la carte m'agaçait jusqu'au jour ou le FTF est tombé, remporté par les lililej. C'est eux qui m'ont expliqué que le PZ est à environ 15 m !!!Aujourd'hui petit détour pour tenter ma chance et BINGO. Merci pour la cache.\end{cacheText}

\cacheNumber{2279}\needspace{5\baselineskip}\cacheName{\href{http://coord.info/GC3G264}{GR8\Number{}52} — \href{http://coord.info/GC3G264\Number{}815004647}{2279}}\cacheData{{2018/11/09 Peyo64, Traditional Cache (1.5/1.5)}}\begin{cacheText}Un petit rayon de soleil ... j’en profite pour aller me promener. Direction Mouguerre pour la multi du comté du Labourd. Sur place je viens vérifier, à tout hasard les caches toujours en bleu. Ici le point se transforme en soleil et la vue est magnifique . Merci Peyo.\end{cacheText}

\cacheNumber{2280}\needspace{5\baselineskip}\cacheName{\href{http://coord.info/GC3FZTA}{GR8\Number{}48} — \href{http://coord.info/GC3FZTA\Number{}815004876}{2280}}\cacheData{{2018/11/09 Peyo64, Traditional Cache (2/1.5)}}\begin{cacheText}Arrivée au PZ, aucun moldu à l’horizon. Je peux donc chercher en toute tranquillité. Enfin j’arrive à jaunir ce vilain petit point bleu car la cache a été remplacée. Merci Peyo.\end{cacheText}

\cacheNumber{2281}\needspace{5\baselineskip}\cacheName{\href{http://coord.info/GC3FZRY}{GR8\Number{}46} — \href{http://coord.info/GC3FZRY\Number{}815006561}{2281}}\cacheData{{2018/11/09 Peyo64, Traditional Cache (2/1.5)}}\begin{cacheText}Malgré la maintenance de Mazra, la cache a encore disparu. Vraiment pas de chance !!! Le lieu est très fréquenté et l’on comprend pourquoi lorsque l’on découvre la beauté du site. Je dépose un nouveau tube en espérant qu’elle dure un peu plus longtemps. Merci pour la cache.\end{cacheText}

\cacheNumber{2282}\needspace{5\baselineskip}\cacheName{\href{http://coord.info/GC2G5JT}{Fontaine de la Catalote} — \href{http://coord.info/GC2G5JT\Number{}815218377}{2282}}\cacheData{{2018/11/10 dune33, Traditional Cache (1/1.5)}}\begin{cacheText}Oh la la quelle belle découverte ! Cette fontaine est magnifique, récemment rénovée il n’y a plus le fameux noisetier. Dommage ! Entourée de vignes, La Fontaine inspire au repos.Le lieu est magique. La Belle est rapidement trouvée : il n’y a pas grand endroit où la cacher! MerciLes Dune33 pour ce superbe endroit.\end{cacheText}

\cacheNumber{2283}\needspace{5\baselineskip}\cacheName{\href{http://coord.info/GC2G5TY}{Fontaine Ste-ROSE} — \href{http://coord.info/GC2G5TY\Number{}815211447}{2283}}\cacheData{{2018/11/10 dune33, Traditional Cache (1/2)}}\begin{cacheText}Quelle surprise en arrivant à la fontaine de Sainte Rose. C’est magnifique, cette chapelle perdue dans la campagne. La cache est trouvée sans difficulté : les coordonnées sont justes. Merci les Dune33 pour cette belle découverte. Un PF bien mérité.\end{cacheText}

\cacheNumber{2284}\needspace{5\baselineskip}\cacheName{\href{http://coord.info/GC2PGPA}{Lavoir Vielle Tursan} — \href{http://coord.info/GC2PGPA\Number{}815220154}{2284}}\cacheData{{2018/11/10 lokateo64, Traditional Cache (1.5/1.5)}}\begin{cacheText}En route pour l'Event des HelPat à Eugénie Les Bains j'en profite pour faire quelques caches aux alentours.Pas simple d’atteindre la cache!!! L'endroit est superbe et semble hors du temps. L'indice me guide car le GPS est capricieux dans ce bas fond. Je finis par dénicher la Belle. Merci lokateo64.\end{cacheText}

\cacheNumber{2285}\needspace{5\baselineskip}\cacheName{\href{http://coord.info/GC4NZB3}{Alphabet Landais ... B pour Bélair à BATS} — \href{http://coord.info/GC4NZB3\Number{}815217812}{2285}}\cacheData{{2018/11/10 MLF40, Traditional Cache (1.5/1.5)}}\begin{cacheText}En route pour l'Event des HelPat  à Eugénie les bains, j’en profite pour faire quelques caches aux alentours. Celle-là fait partie d’un challenge, je m’arrête donc et persiste pour la trouver. Grace à la photo ,je fonce sur l'arbre concerné et découvre la Belle. Elle est vraiment trop mimi. Merci MLF 40 pour la cache.\end{cacheText}

\cacheNumber{2286}\needspace{5\baselineskip}\cacheName{\href{http://coord.info/GC5DE09}{Alphabet Landais ... K pour kiwis} — \href{http://coord.info/GC5DE09\Number{}815206519}{2286}}\cacheData{{2018/11/10 dune33, Traditional Cache (2.5/1.5)}}\begin{cacheText}En route pour l'Event des HelPat à Eugénie les bains, j’en profite pour faire quelques caches sur la route. Arrivée au PZ, je cherche dans les coins possibles… Rien! Après avoir relu les logs précédents, je prends un peu de recul et me pose la question de savoir où je l'aurais cachée.Je m’approche et bingo elle est là parterre ! Je la camoufle et repart ,Merci les Dune33 pour cette cache  sympathique.\end{cacheText}

\cacheNumber{2287}\needspace{5\baselineskip}\cacheName{\href{http://coord.info/GC5K786}{Alphabet Landais ... R comme Radar météo} — \href{http://coord.info/GC5K786\Number{}815205141}{2287}}\cacheData{{2018/11/10 fdcdm, Traditional Cache (1.5/1.5)}}\begin{cacheText}Aujourd’hui, c’est Géocaching avec au final l'Event des HelPat à Eugénie les bains. J’étais déjà passée sur cette cache mais mes recherches étaient infructueuses. Je retente ma chance aujourd’hui malgré un DNF précédent. Je fouille minutieusement l' endroit mais toujours rien. Sur les conseils de domino dans un log précédent je m’intéresse d'un peu plus près à l’indice,je prends un peu de recul et tout d’un coup l’illumination. La belle est bien là ! Merci Éric pour ce bon moment.\end{cacheText}

\cacheNumber{2288}\needspace{5\baselineskip}\cacheName{\href{http://coord.info/GC73Y9Y}{Double αβ Landais... CC pour  Centre Culturel} — \href{http://coord.info/GC73Y9Y\Number{}815214345}{2288}}\cacheData{{2018/11/10 fdcdm, Traditional Cache (1.5/1.5)}}\begin{cacheText}De passage pour aller à l'Event des HelPat à Eugénie les bains, j’en profite pour faire les caches aux alentours. Grâce à l'indice, je n’ai aucun doute sur le lieu où est camouflée la belle. Elle est découverte en deux temps trois mouvements. Merci Éric pour la cache\end{cacheText}

\cacheNumber{2289}\needspace{5\baselineskip}\cacheName{\href{http://coord.info/GC7VGTJ}{EGLISE SAINT -JEAN de SAMADET} — \href{http://coord.info/GC7VGTJ\Number{}815215036}{2289}}\cacheData{{2018/11/10 dune33, Traditional Cache (2/1.5)}}\begin{cacheText}Encore une église qui tombe dans l’oubli! Propriété privée, je pense donc qu'elle ne sera jamais rénovée. Quel dommage ! Je me demande si le verger n’est pas implanté sur le cimetière ! La cache est rapidement trouvée : les coordonnées GPS sont pile poil. Merci les dunes pour m’avoir fait découvrir cette église.\end{cacheText}

\cacheNumber{2290}\needspace{5\baselineskip}\cacheName{\href{http://coord.info/GC7VHFX}{MANTvenez} — \href{http://coord.info/GC7VHFX\Number{}815211023}{2290}}\cacheData{{2018/11/10 dune33, Traditional Cache (1.5/1.5)}}\begin{cacheText}Pas de difficulté  sur ce P Z : un classique du genre. Le ciel se charge en nuages noirs, j’espère finir avant la bourrasque.Merci pour la cache.\end{cacheText}

\cacheNumber{2291}\needspace{5\baselineskip}\cacheName{\href{http://coord.info/GC7VHG2}{Le Pont} — \href{http://coord.info/GC7VHG2\Number{}815210696}{2291}}\cacheData{{2018/11/10 dune33, Traditional Cache (1.5/1.5)}}\begin{cacheText}Je stationne à 30 m du PZ juste avant le pont pour ne pas gêner l'éventuelle circulation. La zone est envahie d'orties et de houx. Effectivement ça pique dur !!!Je finis par mettre la main sur la Belle. Merci les Dune33 pour la cache.\end{cacheText}

\cacheNumber{2292}\needspace{5\baselineskip}\cacheName{\href{http://coord.info/GC7VHG9}{L' Ilex} — \href{http://coord.info/GC7VHG9\Number{}815209850}{2292}}\cacheData{{2018/11/10 dune33, Traditional Cache (1.5/1.5)}}\begin{cacheText}Pas facile ,facile celle la!!!L'ilex est vite repéré mais la cache se fait attendre!!!A force de recherches elle finit par sortir de la tourbe!!!Merci pour la cache.\end{cacheText}

\cacheNumber{2293}\needspace{5\baselineskip}\cacheName{\href{http://coord.info/GC7VPYN}{CAMP ROMAIN} — \href{http://coord.info/GC7VPYN\Number{}815216759}{2293}}\cacheData{{2018/11/10 Disney33     dune33, Traditional Cache (2/2.5)}}\begin{cacheText}Le bois est très sympa. Le long du sentier je dérange les palombes qui sont posées sur les arbres. Le sol est jonché de plumes!!!Arrivée au PZ, je fouille la zone car le GPS n'est pas très fiable et finis par déloger la Belle bien camouflée. La boite est remplie de joujou!!!Merci pour la cache\end{cacheText}

\cacheNumber{2294}\needspace{5\baselineskip}\cacheName{\href{http://coord.info/GC7VPZG}{Lavoir de Samadet 1} — \href{http://coord.info/GC7VPZG\Number{}815213548}{2294}}\cacheData{{2018/11/10 dune33, Traditional Cache (1.5/1.5)}}\begin{cacheText}Encore un super endroit ! Les logs précédents nous conseillent de se fier aux GPS…c'est ce que je fais. La cache est bien la, camouflée. Merci.\end{cacheText}

\cacheNumber{2295}\needspace{5\baselineskip}\cacheName{\href{http://coord.info/GC7VQ04}{Lavoir de Samadet 2} — \href{http://coord.info/GC7VQ04\Number{}815212974}{2295}}\cacheData{{2018/11/10 dune33, Traditional Cache (1.5/1.5)}}\begin{cacheText}Joli petit lavoir qui mériterait une bonne rénovation. Après avoir franchi le petit cours d’eau les recherches s' annoncent difficiles. Le GPS m’envoie de l’autre côté de la barrière !!!Apres avoir bien observé les lieux un seul endroit semble possible et BINGO ,la Belle est bien camouflée. Bravo les amis ,c'est du bon travail.Merci.\end{cacheText}

\cacheNumber{2296}\needspace{5\baselineskip}\cacheName{\href{http://coord.info/GC7ZG06}{Retour à Eugénie} — \href{http://coord.info/GC7ZG06\Number{}815224715}{2296}}\cacheData{{2018/11/10 HelPat - 32/44, Event Cache (1/1.5)}}\begin{cacheText}J'ai toujours autant de plaisir à rencontrer les amis géocacheurs. Le kiosque, point de rendez vous est très sympa et les géoblablas  vont bon train comme d'habitude. Les owners ,très chaleureux, ont prévu le nécessaire pour nous accueillir. Après l'échange des TB, la nuit nous pousse à nous séparer. Merci les HelPat pour ce sympathique Event.\end{cacheText}

\cacheNumber{2297}\needspace{5\baselineskip}\cacheName{\href{http://coord.info/GC6623Z}{Covoiturage / Auto partekatze gunea Ametzondo} — \href{http://coord.info/GC6623Z\Number{}815453007}{2297}}\cacheData{{2018/11/11 DorisBear, Traditional Cache (2/1.5)}}\begin{cacheText}De passage pour me rendre sur Bayonne, je m’arrête pour la seconde fois et aujourd’hui c’est carton plein !!!La cache ne m’a pas résisté longtemps!!! En effet, il reste des traces des recherches des geocacheurs précédents. Cela aide bien.Encore du super bricolage et elle est parfaitement intégrée. Merci les DorisBear.\end{cacheText}

\cacheNumber{2298}\needspace{5\baselineskip}\cacheName{\href{http://coord.info/GC3YMZY}{Fontaine-Lavoir Laborde} — \href{http://coord.info/GC3YMZY\Number{}816189098}{2298}}\cacheData{{2018/11/16 gilles64, Traditional Cache (1.5/1.5)}}\begin{cacheText}Sur Anglet pour trouver la cache des gamboy ,Reading the clogs, j’en profite pour faire un petit détour et découvrir le lavoir. On ne voit plus rien: il est enseveli sous les ronces. Quel dommage ! La cache est trouvée assez difficilement car le GPS me joue des tours. C’est grâce aux commentaires précédents que j’arrive à la déloger. Merci Gilles .\end{cacheText}

\cacheNumber{2299}\needspace{5\baselineskip}\cacheName{\href{http://coord.info/GC6921P}{Ferme Camiade - Anglet} — \href{http://coord.info/GC6921P\Number{}816189166}{2299}}\cacheData{{2018/11/16 gilles64, Traditional Cache (1.5/1.5)}}\begin{cacheText}Un vrai plaisir de voir une ferme traditionnelle basque au cœur d’Anglet. La demeure est majestueuse. Peu de passage à cette heure ci. La cache attend sagement au pied. Merci Gilles pour cette découverte sympa.\end{cacheText}

\cacheNumber{2300}\needspace{5\baselineskip}\cacheName{\href{http://coord.info/GC7W4DV}{Reading the clogs} — \href{http://coord.info/GC7W4DV\Number{}816188861}{2300}}\cacheData{{2018/11/16 Gamboy, Unknown Cache (3.5/1.5)}}\begin{cacheText}((TTF))

Lors de la parution de l’❓des 🐌, nous étions à l’ AE  et j’avoue qu’après ,elle m’est sortie de la \includegraphics{./emoji_images/hires/1F467-1F3FC.pdf}!C’est en regardant la 🗺 pour une prochaine sortie que je suis tombée dessus .En avant la 🎶, il faut relire tout les clogs et il y a du boulot!!! Heureusement que je suis en convalescence et que j’ai du temps à tuer!!!Les personnages me sont familiers : 👑, 🌍, 🛵🏉, 🎩👒, 😇, 🐻🐧, 🕷\includegraphics{./emoji_images/hires/1F578-FE0F.pdf}, 🍹, 🦆✈️, 🏎, ⛺️, \includegraphics{./emoji_images/hires/1F487-1F3FC-2640.pdf}\includegraphics{./emoji_images/hires/1F54A-FE0F.pdf}, 🐦, 🚁, 🍄, ⚠️, 🐗 ,et après quelques essais le chekeur passe au vert..L’équipage est une autre affaire !!! Les 🚘,🚲,🎁,🎒,🙌🏻,🏳️‍🌈,🏅,📍,🔦,📕🖊📧,🔪,🚧 sont facilement reconnaissables mais c’est bien plus compliqué pour démêler les 📱,💻,🖱,\includegraphics{./emoji_images/hires/1F933-1F3FB.pdf}et les🗣 . Le petit indice rajouté dans certitude me permet de récolter le dernier mot pour le final.Après une bonne 🌑 ,je prend la ➡️ d’Anglet ,avec ma titine 🚗 . Arrivée au PZ , je n’ai vraiment pas de 🍀: 2 👨🏻‍🎨 s’affairent sur la 🏠 voisine . Il me faut être vigilante pour ne pas éveiller les soupçons !!! Oufff personne 🙈, je tend  la\includegraphics{./emoji_images/hires/1F590.pdf} et trouve la 🏆 .Bravo les 🐌 pour cette🦉énigme : quel plaisir de vous lire!!! Un ❤️ évidemment .\end{cacheText}

\cacheNumber{2301}\needspace{5\baselineskip}\cacheName{\href{http://coord.info/GC6927C}{Arene Rion des landes} — \href{http://coord.info/GC6927C\Number{}816682880}{2301}}\cacheData{{2018/11/18 E.L.40, Traditional Cache (1.5/1.5)}}\begin{cacheText}Sur Rion pour assister à un match de rugby 🏉 , nous en profitons pour faire quelques caches sur le village. En s'approchant du PZ nous n'avons aucun doute sur le lieu de pose de la magnétique. Pas de moldu, nous signons tranquillement. Merci pour la découverte de ces belles arènes .\end{cacheText}

\cacheNumber{2302}\needspace{5\baselineskip}\cacheName{\href{http://coord.info/GC6928M}{Eglise Rion des landes} — \href{http://coord.info/GC6928M\Number{}816679210}{2302}}\cacheData{{2018/11/18 E.L.40, Traditional Cache (1.5/1)}}\begin{cacheText}Sur Rion pour assister à un match de rugby 🏉 , nous en profitons pour faire quelques caches. Nous découvrons une magnifique église et une superbe mairie. Quel patrimoine !!! Arrivées au PZ nous devons patienter car 2 moldus prennent des photos . La cache est retirée du trou: Le logbook est inutilisable, nous changeons donc l’ensemble car il n’y a plus de bouchon. Merci pour la cache.\end{cacheText}

\cacheNumber{2303}\needspace{5\baselineskip}\cacheName{\href{http://coord.info/GC6TH1Y}{Fronton de Rion} — \href{http://coord.info/GC6TH1Y\Number{}816685805}{2303}}\cacheData{{2018/11/18 fdcdm, Traditional Cache (1.5/1.5)}}\begin{cacheText}Sur Rion pour assister à un match de rugby 🏉 , nous en profitons pour faire quelques caches sur le village. Nous suivons scrupuleusement l'indice et allons nous asseoir….bingo la cache est découverte en deux temps trois mouvements. Merci Eric.\end{cacheText}

\cacheNumber{2304}\needspace{5\baselineskip}\cacheName{\href{http://coord.info/GC7BNJR}{ToutencarMont \Number{}3} — \href{http://coord.info/GC7BNJR\Number{}818145212}{2304}}\cacheData{{2018/11/25 ayous, Traditional Cache (1.5/1.5)}}\begin{cacheText}Petit détour pour vérifier cette cache. Mais comment ai-je fait pour ne pas la trouver la première fois ? ? ? Un effet d’optique très certainement : noir c’est noir… Dernier indice en poche direction la bonus. Merci ayous pour la cache.\end{cacheText}

\cacheNumber{2305}\needspace{5\baselineskip}\cacheName{\href{http://coord.info/GC7BNPH}{Le trésor de ToutencarMont} — \href{http://coord.info/GC7BNPH\Number{}818145449}{2305}}\cacheData{{2018/11/25 ayous, Unknown Cache (2.5/2)}}\begin{cacheText}Apres avoir relevé le dernier indice manquant, je termine mes calculs. Direction le PZ pour débusquer la Belle. D'après les commentaires précédents, les coordonnées GPS semblaient correctes mais mon smatphone m'a joué des tours!!! J'ai tournée en rond 3/4 d'heure!!!Sur le point d'abandonner j'ai fini par mettre la main dessus. Merci ayous pour cet excellent moment. Un PF pour l'ensemble du travail (jeux de mots, énigmes et caches).\end{cacheText}

\cacheNumber{2306}\needspace{5\baselineskip}\cacheName{\href{http://coord.info/GC7Z5DF}{L'Ousse des bois et son boulevard - 1} — \href{http://coord.info/GC7Z5DF\Number{}818190362}{2306}}\cacheData{{2018/11/30 92zelda64, Traditional Cache (1.5/1.5)}}\begin{cacheText}Rien de mieux qu'une journée de géocaching avec de joyeux lurons pour oublier quelques heures le chagrin. Le rendez vous est à 9h00 au parking du ZENITH et j'arrive avec quelques minutes de retard (le temps d'écouter les gilets jaunes aux péages).Les Dunes 33 m'attendent ,impatients de découvrir le circuit. Le temps est à la pluie et nous décidons donc de faire le tour en voiture.

Arrivés au PZ ,nous inspectons le pont et ses alentours. Nous finissons par découvrir la Belle bien à l'abris, grâce au sachet plastique qui dépasse. Merci pour la cache.\end{cacheText}

\cacheNumber{2307}\needspace{5\baselineskip}\cacheName{\href{http://coord.info/GC7Z711}{L'Ousse des bois et son boulevard - 2} — \href{http://coord.info/GC7Z711\Number{}818200284}{2307}}\cacheData{{2018/11/30 92zelda64, Traditional Cache (1.5/1.5)}}\begin{cacheText}Quelle chance, la pluie cesse!!! Nous stationnons la voiture dans le lotissement et partons à pied rejoindre le PZ. Les GPS nous baladent mais Denise ,avec son œil de lynx déniche la jolie bébête bien camouflée .Il était temps… la pluie est de retour et les parapluies sont bien au sec dans la voiture ! Merci pour cette jolie cache.\end{cacheText}

\cacheNumber{2308}\needspace{5\baselineskip}\cacheName{\href{http://coord.info/GC7Z718}{L'Ousse des bois et son boulevard - 3} — \href{http://coord.info/GC7Z718\Number{}818200383}{2308}}\cacheData{{2018/11/30 92zelda64, Traditional Cache (1.5/1.5)}}\begin{cacheText}Ou la la ,elle nous en a donné du mal celle-là. Nous avons bloqué sur l’indice mais n’avons pas eu le déclic. Les moldus passent, indifférents à nos recherches effrénées et c’est Dédé qui finit par découvrir le pot aux roses.Merci pour la cache.\end{cacheText}

\cacheNumber{2309}\needspace{5\baselineskip}\cacheName{\href{http://coord.info/GC7Z72H}{L'Ousse des bois et son boulevard - 4} — \href{http://coord.info/GC7Z72H\Number{}818200480}{2309}}\cacheData{{2018/11/30 92zelda64, Traditional Cache (1.5/1.5)}}\begin{cacheText}Quelle belle découverte ce petit parc perdu au milieu du lotissement ! Nous aidant du spoiler, nous fouillons la zone. Rien!!! Et c’est grâce au coup de fil à un ami ,que Denise finit par la déloger.  Intégration parfaite!!!! La cache est toute simple mais hyper efficace.Bravo et merci.\end{cacheText}

\cacheNumber{2310}\needspace{5\baselineskip}\cacheName{\href{http://coord.info/GC7Z72Q}{L'Ousse des bois et son boulevard - 5} — \href{http://coord.info/GC7Z72Q\Number{}818200594}{2310}}\cacheData{{2018/11/30 92zelda64, Traditional Cache (1.5/1.5)}}\begin{cacheText}Nous continuons notre périple en essayant d’éviter les gouttes.

Les coordonnées parfaites nous mènent tout droit à la jolie cache. Encore une belle réalisation. Merci pour la cache.\end{cacheText}

\cacheNumber{2311}\needspace{5\baselineskip}\cacheName{\href{http://coord.info/GC7Z72X}{L'Ousse des bois et son boulevard - 6} — \href{http://coord.info/GC7Z72X\Number{}818210979}{2311}}\cacheData{{2018/11/30 92zelda64, Traditional Cache (1.5/1.5)}}\begin{cacheText}La pluie semble nous fausser compagnie. Tant mieux ! Arrivés sur le PZ , nous suivons l’indice et trouvons la belle suspendue. Merci pour la cache\end{cacheText}

\cacheNumber{2312}\needspace{5\baselineskip}\cacheName{\href{http://coord.info/GC7Z734}{L'Ousse des bois et son boulevard - 7} — \href{http://coord.info/GC7Z734\Number{}818211376}{2312}}\cacheData{{2018/11/30 92zelda64, Traditional Cache (1.5/1.5)}}\begin{cacheText}Après avoir interprété l’indice, nous nous répartissons les taches. Dédé à droite et je pars à gauche mais rien de mon côté. Je rejoins mon acolyte et je lui fais remarquer un petit détail. Fausse joie, il l’a déjà inspecté et n’a rien trouvé. Nous élargissons les recherches autour du PZ. Denise qui est revenue sur le détail , approfondit la recherche et finit par mettre la main sur la Belle. Cache vraiment sympa. Merci\end{cacheText}

\cacheNumber{2313}\needspace{5\baselineskip}\cacheName{\href{http://coord.info/GC7Z737}{L'Ousse des bois et son boulevard - 8} — \href{http://coord.info/GC7Z737\Number{}818211542}{2313}}\cacheData{{2018/11/30 92zelda64, Traditional Cache (1.5/1.5)}}\begin{cacheText}Nos GPS ne sont pas très précis mais grâce à la photo nous délimitons la zone de recherches. Un détail intrigue Denise… Bingo la cache est bien là. Pas facile ,facile. Merci pour la cache.\end{cacheText}

\cacheNumber{2314}\needspace{5\baselineskip}\cacheName{\href{http://coord.info/GC7Z739}{L'Ousse des bois et son boulevard - 9} — \href{http://coord.info/GC7Z739\Number{}818212199}{2314}}\cacheData{{2018/11/30 92zelda64, Traditional Cache (1.5/1.5)}}\begin{cacheText}La découverte du quartier se fait au rythme des caches. Un joli border collie nous accueille dans la maison voisine lorsque l’on s’approche du PZ. Après avoir fait le tour, nous finissons par découvrir la belle. Merci pour la cache.\end{cacheText}

\cacheNumber{2315}\needspace{5\baselineskip}\cacheName{\href{http://coord.info/GC7Z73C}{L'Ousse des bois et son boulevard - 10} — \href{http://coord.info/GC7Z73C\Number{}818212714}{2315}}\cacheData{{2018/11/30 92zelda64, Traditional Cache (1.5/1.5)}}\begin{cacheText}L’indice nous mène dans une mauvaise direction... nous avons beau chercher dans tous les endroits possible et inimaginables, nous ne trouvons rien .Désespérés ,nous passons un petit coup de fil à Agan64 qui nous remet dans la bonne direction. Ouf l’honneur est sauf. Ici aussi nous trouvons une intégration parfaite. Merci pour la cache.\end{cacheText}

\cacheNumber{2316}\needspace{5\baselineskip}\cacheName{\href{http://coord.info/GC7Z73E}{L'Ousse des bois et son boulevard - 11} — \href{http://coord.info/GC7Z73E\Number{}818277805}{2316}}\cacheData{{2018/11/30 92zelda64, Traditional Cache (1.5/1.5)}}\begin{cacheText}L’indice prend tout son sens en arrivant sur les lieux !Aucun moldu à l’horizon :nous sommes tranquilles pour faire le tour. La belle nous attend sous le tapis… Merci pour la cache.\end{cacheText}

\cacheNumber{2317}\needspace{5\baselineskip}\cacheName{\href{http://coord.info/GC7Z73H}{L'Ousse des bois et son boulevard - 12} — \href{http://coord.info/GC7Z73H\Number{}818277703}{2317}}\cacheData{{2018/11/30 92zelda64, Traditional Cache (1.5/1.5)}}\begin{cacheText}Après une halte au chinois où nous avons fêté la Saint André comme il se doit nous reprenons la série. Alors que Dédé commence sa sieste dans la voiture, Denise et moi inspectons le carrefour. Nous trouvons la Belle suspendue…Merci pour la cache.\end{cacheText}

\cacheNumber{2318}\needspace{5\baselineskip}\cacheName{\href{http://coord.info/GC7Z73T}{L'Ousse des bois et son boulevard - 13} — \href{http://coord.info/GC7Z73T\Number{}818277960}{2318}}\cacheData{{2018/11/30 92zelda64, Traditional Cache (1.5/1.5)}}\begin{cacheText}Après avoir fait le tour de tous les coffrets et de toutes les boites, un petit détail attire l’attention. Bingo c’est la boîte . Merci pour la cache\end{cacheText}

\cacheNumber{2319}\needspace{5\baselineskip}\cacheName{\href{http://coord.info/GC7Z743}{L'Ousse des bois et son boulevard - BONUS} — \href{http://coord.info/GC7Z743\Number{}818278228}{2319}}\cacheData{{2018/11/30 92zelda64, Unknown Cache (1.5/1.5)}}\begin{cacheText}Direction le PZ final pour déloger la belle.Elle a trouvé refuge dans un arbre superbe. Après avoir logué discrètement, nous remettons tout en place en la camouflant bien car l’endroit est assez fréquenté. Merci 92Zelda64 pour ce super circuit. Un PF pour récompenser l’ensemble du travail.\end{cacheText}

\cacheNumber{2320}\needspace{5\baselineskip}\cacheName{\href{http://coord.info/GC7Z91W}{L'Ousse des bois et son boulevard - 14} — \href{http://coord.info/GC7Z91W\Number{}818278065}{2320}}\cacheData{{2018/11/30 92zelda64, Traditional Cache (1.5/1.5)}}\begin{cacheText}Ultime étape et pas la plus facile !!! Ici aussi la cache est parfaitement intégrée au décor. Il nous en a fallu du temps pour la découvrir ! Le log book est très humide.Merci pour la cache.\end{cacheText}

\cacheNumber{2321}\needspace{5\baselineskip}\cacheName{\href{http://coord.info/GC7NV0W}{Capharnaüm N°72} — \href{http://coord.info/GC7NV0W\Number{}818320984}{2321}}\cacheData{{2018/12/01 Opmb40, Traditional Cache (1.5/1.5)}}\begin{cacheText}La pluie a cessé: pas question de se morfondre devant la télé! Direction le capharnaüm et ses caches en drive.Les coordonnées sont précises et les caches sont facilement trouvées.Merci Opmb40 pour tout le travail de pose.\end{cacheText}

\cacheNumber{2322}\needspace{5\baselineskip}\cacheName{\href{http://coord.info/GC7NV2G}{Capharnaüm N°79} — \href{http://coord.info/GC7NV2G\Number{}818321120}{2322}}\cacheData{{2018/12/01 Opmb40, Traditional Cache (1.5/1.5)}}\begin{cacheText}La pluie a cessé: pas question de se morfondre devant la télé! Direction le capharnaüm et ses caches en drive.Les coordonnées sont précises et les caches sont facilement trouvées.Merci Opmb40 pour tout le travail de pose.\end{cacheText}

\cacheNumber{2323}\needspace{5\baselineskip}\cacheName{\href{http://coord.info/GC7PBRE}{Capharnaüm N°84} — \href{http://coord.info/GC7PBRE\Number{}818319144}{2323}}\cacheData{{2018/12/01 Opmb40, Traditional Cache (1.5/1.5)}}\begin{cacheText}La pluie a cessé: pas question de se morfondre devant la télé! Direction le capharnaüm et ses caches en drive.Les coordonnées sont précises et les caches sont facilement trouvées.Merci Opmb40 pour tout le travail de pose.\end{cacheText}

\cacheNumber{2324}\needspace{5\baselineskip}\cacheName{\href{http://coord.info/GC7PBT6}{Capharnaüm N°89} — \href{http://coord.info/GC7PBT6\Number{}818319413}{2324}}\cacheData{{2018/12/01 Opmb40, Traditional Cache (1.5/1.5)}}\begin{cacheText}La pluie a cessé: pas question de se morfondre devant la télé! Direction le capharnaüm et ses caches en drive.Les coordonnées sont précises et les caches sont facilement trouvées.Merci Opmb40 pour tout le travail de pose.\end{cacheText}

\cacheNumber{2325}\needspace{5\baselineskip}\cacheName{\href{http://coord.info/GC7PBTN}{Capharnaüm N°92} — \href{http://coord.info/GC7PBTN\Number{}818316053}{2325}}\cacheData{{2018/12/01 Opmb40, Traditional Cache (1.5/1.5)}}\begin{cacheText}La pluie a cessé: pas question de se morfondre devant la télé! Direction le capharnaüm  et ses caches en drive.Les coordonnées sont précises et les caches sont facilement trouvées.Merci Opmb40 pour tout le travail de pose.\end{cacheText}

\cacheNumber{2326}\needspace{5\baselineskip}\cacheName{\href{http://coord.info/GC7PC47}{Capharnaüm N°105} — \href{http://coord.info/GC7PC47\Number{}818317859}{2326}}\cacheData{{2018/12/01 Opmb40, Traditional Cache (1.5/1.5)}}\begin{cacheText}La pluie a cessé: pas question de se morfondre devant la télé! Direction le capharnaüm et ses caches en drive.Les coordonnées sont précises et les caches sont facilement trouvées.Merci Opmb40 pour tout le travail de pose.\end{cacheText}

\cacheNumber{2327}\needspace{5\baselineskip}\cacheName{\href{http://coord.info/GC7PC57}{Capharnaüm N°111} — \href{http://coord.info/GC7PC57\Number{}818320782}{2327}}\cacheData{{2018/12/01 Opmb40, Traditional Cache (1.5/1.5)}}\begin{cacheText}La pluie a cessé: pas question de se morfondre devant la télé! Direction le capharnaüm et ses caches en drive.Les coordonnées sont précises et les caches sont facilement trouvées.Merci Opmb40 pour tout le travail de pose.\end{cacheText}

\cacheNumber{2328}\needspace{5\baselineskip}\cacheName{\href{http://coord.info/GC7PC5E}{Capharnaüm N°112} — \href{http://coord.info/GC7PC5E\Number{}818316333}{2328}}\cacheData{{2018/12/01 Opmb40, Traditional Cache (1.5/1.5)}}\begin{cacheText}La pluie a cessé: pas question de se morfondre devant la télé! Direction le capharnaüm et ses caches en drive.Les coordonnées sont précises et les caches sont facilement trouvées.Merci Opmb40 pour tout le travail de pose.\end{cacheText}

\cacheNumber{2329}\needspace{5\baselineskip}\cacheName{\href{http://coord.info/GC7PC5T}{Capharnaüm N°114} — \href{http://coord.info/GC7PC5T\Number{}818316598}{2329}}\cacheData{{2018/12/01 Opmb40, Traditional Cache (1.5/1.5)}}\begin{cacheText}La pluie a cessé: pas question de se morfondre devant la télé! Direction le capharnaüm et ses caches en drive.Les coordonnées sont précises et les caches sont facilement trouvées.Merci Opmb40 pour tout le travail de pose.\end{cacheText}

\cacheNumber{2330}\needspace{5\baselineskip}\cacheName{\href{http://coord.info/GC7PC6Y}{Capharnaüm N°118} — \href{http://coord.info/GC7PC6Y\Number{}818318016}{2330}}\cacheData{{2018/12/01 Opmb40, Traditional Cache (1.5/1.5)}}\begin{cacheText}La pluie a cessé: pas question de se morfondre devant la télé! Direction le capharnaüm et ses caches en drive.Les coordonnées sont précises et les caches sont facilement trouvées.Merci Opmb40 pour tout le travail de pose.\end{cacheText}

\cacheNumber{2331}\needspace{5\baselineskip}\cacheName{\href{http://coord.info/GC7PC87}{Capharnaüm N°123} — \href{http://coord.info/GC7PC87\Number{}818316461}{2331}}\cacheData{{2018/12/01 Opmb40, Traditional Cache (1.5/1.5)}}\begin{cacheText}La pluie a cessé: pas question de se morfondre devant la télé! Direction le capharnaüm et ses caches en drive.Les coordonnées sont précises et les caches sont facilement trouvées.Merci Opmb40 pour tout le travail de pose.\end{cacheText}

\cacheNumber{2332}\needspace{5\baselineskip}\cacheName{\href{http://coord.info/GC7PC8F}{Capharnaüm N°125} — \href{http://coord.info/GC7PC8F\Number{}818318243}{2332}}\cacheData{{2018/12/01 Opmb40, Traditional Cache (1.5/1.5)}}\begin{cacheText}La pluie a cessé: pas question de se morfondre devant la télé! Direction le capharnaüm et ses caches en drive.Les coordonnées sont précises et les caches sont facilement trouvées.Merci Opmb40 pour tout le travail de pose.\end{cacheText}

\cacheNumber{2333}\needspace{5\baselineskip}\cacheName{\href{http://coord.info/GC7PC8P}{Capharnaüm N°127} — \href{http://coord.info/GC7PC8P\Number{}818316738}{2333}}\cacheData{{2018/12/01 Opmb40, Traditional Cache (1.5/2)}}\begin{cacheText}La pluie a cessé: pas question de se morfondre devant la télé! Direction le capharnaüm et ses caches en drive.Les coordonnées sont précises et les caches sont facilement trouvées.Merci Opmb40 pour tout le travail de pose.\end{cacheText}

\cacheNumber{2334}\needspace{5\baselineskip}\cacheName{\href{http://coord.info/GC7PCB8}{Capharnaüm N°132} — \href{http://coord.info/GC7PCB8\Number{}818319572}{2334}}\cacheData{{2018/12/01 Opmb40, Traditional Cache (1.5/2)}}\begin{cacheText}La pluie a cessé: pas question de se morfondre devant la télé! Direction le capharnaüm et ses caches en drive.Les coordonnées sont précises et les caches sont facilement trouvées.Merci Opmb40 pour tout le travail de pose.\end{cacheText}

\cacheNumber{2335}\needspace{5\baselineskip}\cacheName{\href{http://coord.info/GC5XWEH}{\Number{}L2-02 Briscous - Le Comté du Labourd} — \href{http://coord.info/GC5XWEH\Number{}818490773}{2335}}\cacheData{{2018/12/02 gilles64, Multi-cache (2/1.5)}}\begin{cacheText}Aujourd'hui direction Briscous pour continuer la série du Comté du Labourd. Je découvre un très joli village bien entretenu avec des maisons typiques du Labourd. Les indices sont rapidement relevés avec un doute sur la date du B. Heureusement les PZ ne sont pas loin!!!En arrivant sur site, je n'ai aucun doute grâce au spoiler. La vigne vierge a juste perdu toutes ses feuilles. Les recherches ont été assez longues et alors que j'allais renoncer….BINGO.Elle était bien cachée sous un cailloux enseveli sous les feuilles. Comme d'habitude sur cette série, la vue est superbe. Merci Gilles pour cet agréable moment.\end{cacheText}

\cacheNumber{2336}\needspace{5\baselineskip}\cacheName{\href{http://coord.info/GC7MTNB}{Capharnaüm N°6} — \href{http://coord.info/GC7MTNB\Number{}819209217}{2336}}\cacheData{{2018/12/04 Opmb40, Traditional Cache (1.5/1.5)}}\begin{cacheText}Cache vite trouvée: peu d’endroits où la camoufler . Merci.\end{cacheText}

\cacheNumber{2337}\needspace{5\baselineskip}\cacheName{\href{http://coord.info/GC7NRK1}{Capharnaüm N°56} — \href{http://coord.info/GC7NRK1\Number{}819209013}{2337}}\cacheData{{2018/12/04 Opmb40, Traditional Cache (1.5/1.5)}}\begin{cacheText}Ici le GPS n’est pas très fiable mais à force de recherches je finis par la trouver. Il faut ouvrir l’œil et le bon!Merci pour la cache.\end{cacheText}

\cacheNumber{2338}\needspace{5\baselineskip}\cacheName{\href{http://coord.info/GC7NRKW}{Capharnaüm N°61} — \href{http://coord.info/GC7NRKW\Number{}818734019}{2338}}\cacheData{{2018/12/04 Opmb40, Traditional Cache (1.5/2.5)}}\begin{cacheText}La cache est un peu éloignée de la route, et il faut traverser un champ plein de boue. Heureusement que j’ai les bottes ! La cache est vite trouvée dans les bras. Merci.\end{cacheText}

\cacheNumber{2339}\needspace{5\baselineskip}\cacheName{\href{http://coord.info/GC7NRM5}{Capharnaüm N°63} — \href{http://coord.info/GC7NRM5\Number{}818733607}{2339}}\cacheData{{2018/12/04 Opmb40, Traditional Cache (1.5/1.5)}}\begin{cacheText}Pierre qui roule n’amasse pas mousse. Pour ma part, j’ai une cache en plus. Merci.\end{cacheText}

\cacheNumber{2340}\needspace{5\baselineskip}\cacheName{\href{http://coord.info/GC7NV06}{Capharnaüm N°70} — \href{http://coord.info/GC7NV06\Number{}818733410}{2340}}\cacheData{{2018/12/04 Opmb40, Traditional Cache (1.5/1.5)}}\begin{cacheText}C’est mon second passage sur cette cache. La première fois , lors de la sortie,elle avait disparu . Aujourd’hui,Elle ne m’échappe pas. Merci pour la cache.\end{cacheText}

\cacheNumber{2341}\needspace{5\baselineskip}\cacheName{\href{http://coord.info/GC7PBRG}{Capharnaüm N°85} — \href{http://coord.info/GC7PBRG\Number{}818726781}{2341}}\cacheData{{2018/12/04 Opmb40, Traditional Cache (2/1.5)}}\begin{cacheText}Je continue le chemin et j’entends au loin une meute de chiens qui aboient . Décidément, pas moyen de rester au calme !Je trouve une boîte qui n’est plus aimantée, celle d’origine a dû disparaître ! Merci pour la cache.\end{cacheText}

\cacheNumber{2342}\needspace{5\baselineskip}\cacheName{\href{http://coord.info/GC7PBRR}{Capharnaüm N°86} — \href{http://coord.info/GC7PBRR\Number{}819210422}{2342}}\cacheData{{2018/12/04 Opmb40, Traditional Cache (1.5/1.5)}}\begin{cacheText}La coquine m’a donné du fil à retordre!!! La photo ne laisse aucun doute sur l’endroit où chercher alors je persiste !Bien m’en a pris, mes yeux ont fini par tomber dessus. Merci pour la cache.\end{cacheText}

\cacheNumber{2343}\needspace{5\baselineskip}\cacheName{\href{http://coord.info/GC7PBTQ}{Capharnaüm N°94} — \href{http://coord.info/GC7PBTQ\Number{}818726499}{2343}}\cacheData{{2018/12/04 Opmb40, Traditional Cache (1.5/1.5)}}\begin{cacheText}Je laisse la voiture au bout du chemin et pars me dégourdir les jambes par ce bel après-midi. Les arbres sont superbes avec leurs couleurs automnales. La photo spoiler me guide jusqu’à la cache. Merci Opmb40 pour la cache.\end{cacheText}

\cacheNumber{2344}\needspace{5\baselineskip}\cacheName{\href{http://coord.info/GC7PC3R}{Capharnaüm N°101} — \href{http://coord.info/GC7PC3R\Number{}819209470}{2344}}\cacheData{{2018/12/04 Opmb40, Traditional Cache (1.5/1.5)}}\begin{cacheText}Pas de passage sur ce chemin de campagne: l’indice me mène tout droit au trésor. Merci pour la cache.\end{cacheText}

\cacheNumber{2345}\needspace{5\baselineskip}\cacheName{\href{http://coord.info/GC7PC5Z}{Capharnaüm N°116} — \href{http://coord.info/GC7PC5Z\Number{}818731298}{2345}}\cacheData{{2018/12/04 Opmb40, Traditional Cache (1.5/2.5)}}\begin{cacheText}Le bel arbre est reconnaissable. Après avoir cherché sous les feuilles mortes je finis par déloger la belle.Alors que je lève la tête, j’aperçois la cache initiale!!!Je signe et j’enlève la seconde boîte. Merci pour la cache.\end{cacheText}

\cacheNumber{2346}\needspace{5\baselineskip}\cacheName{\href{http://coord.info/GC7PC6M}{Capharnaüm N°115} — \href{http://coord.info/GC7PC6M\Number{}818723720}{2346}}\cacheData{{2018/12/04 Opmb40, Traditional Cache (1.5/1.5)}}\begin{cacheText}La journée géocaching commence fort : arrivée au PZ, des milliers de dindes convergent vers moi. Impressionnant le bruit qu’elles font!!!Pour la discrétion c’est loupé! Je met la main sur la cache et j’ai le temps de signer le logbook lorsque le propriétaire arrive sur les lieux ,plutôt en colère car je stresse ses dindes!!!Après m’être excusée prétextant vouloir faire une photo , il devient plus sympa et je repars avec la cache dans les mains. Ce n’est qu’en fin d’après-midi que je repasse déposer la cache au pas de course pour ne pas tomber nez à nez avec le propriétaire. Oufff mission accomplie. Merci pour la cache.\end{cacheText}

\cacheNumber{2347}\needspace{5\baselineskip}\cacheName{\href{http://coord.info/GC7PC8H}{Capharnaüm N°126} — \href{http://coord.info/GC7PC8H\Number{}818724985}{2347}}\cacheData{{2018/12/04 Opmb40, Traditional Cache (1.5/1.5)}}\begin{cacheText}L’endroit est très calme et aucune Dinde à l’horizon ! L’indice ne laisse aucun doute et pourtant Rien!!! C’est en cherchant au pied que je finis par la trouver!!! Elle a du tomber... je la remet en place. Merci pour la cache.\end{cacheText}

\cacheNumber{2348}\needspace{5\baselineskip}\cacheName{\href{http://coord.info/GC7PC8Z}{Capharnaüm N°128} — \href{http://coord.info/GC7PC8Z\Number{}818727121}{2348}}\cacheData{{2018/12/04 Opmb40, Traditional Cache (1.5/1.5)}}\begin{cacheText}Oufff ….elle m’en a donné du mal celle-là!!! En fait, le spoiler m’a induit en erreur... mais mes yeux ont finit par la voir. Merci pour la cache\end{cacheText}

\cacheNumber{2349}\needspace{5\baselineskip}\cacheName{\href{http://coord.info/GC7PCBB}{Capharnaüm N°133} — \href{http://coord.info/GC7PCBB\Number{}818732027}{2349}}\cacheData{{2018/12/04 Opmb40, Traditional Cache (1.5/1.5)}}\begin{cacheText}Le PZ est désert . Aucun doute sur le lieu de la cache!!! Merci\end{cacheText}

\cacheNumber{2350}\needspace{5\baselineskip}\cacheName{\href{http://coord.info/GC4BBK6}{La place des platanes} — \href{http://coord.info/GC4BBK6\Number{}819234510}{2350}}\cacheData{{2018/12/07 pierredesgaves, Traditional Cache (1.5/1)}}\begin{cacheText}C'est au retour de la promenade dans les bois d'agan64 à Andoins que je m'arrête faire cette cache. La place est en travaux et le banc a été arraché par la pelleteuse!!! Je regarde à tout hasard….Bingo elle est toujours en place. Merci pour la cache.\end{cacheText}

\cacheNumber{2351}\needspace{5\baselineskip}\cacheName{\href{http://coord.info/GC4BYVJ}{Le pont des pêcheurs} — \href{http://coord.info/GC4BYVJ\Number{}819242619}{2351}}\cacheData{{2018/12/07 pierredesgaves, Traditional Cache (1.5/1.5)}}\begin{cacheText}La cache est très abimée, je la remplace donc pour sa survie.           Merci pour la cache.\end{cacheText}

\cacheNumber{2352}\needspace{5\baselineskip}\cacheName{\href{http://coord.info/GC7Y9D2}{Un petit tour dans le bois \Number{}1} — \href{http://coord.info/GC7Y9D2\Number{}819230373}{2352}}\cacheData{{2018/12/07 agan 64, Traditional Cache (1.5/2)}}\begin{cacheText}Après avoir fait une promenade matinale dans Pau pour résoudre Pau c’est royal !,nous optons avec Dune33, par ce bel après-midi, pour un petit tour dans les bois. Le parcours est très agréable sous ce beau soleil. Les caches, bien travaillées, nous ont fait bien chercher mais au final toutes seront jaunies. Les indices en poche, nous passons le checkeur au vert sans difficulté pour la bonus qui est adorable. Des PF sont distribués pour récompenser les belles caches et l' ensemble du travail. Merci agan64 pour cet excellent moment.\end{cacheText}

\cacheNumber{2353}\needspace{5\baselineskip}\cacheName{\href{http://coord.info/GC7Y9GZ}{Un petit tour dans le bois  \Number{}3} — \href{http://coord.info/GC7Y9GZ\Number{}819230792}{2353}}\cacheData{{2018/12/07 agan 64, Traditional Cache (1.5/2)}}\begin{cacheText}Après avoir fait une promenade matinale dans Pau pour résoudre Pau c’est royal !,nous optons avec Dune33, par ce bel après-midi, pour un petit tour dans les bois. Le parcours est très agréable sous ce beau soleil. Les caches, bien travaillées, nous ont fait bien chercher mais au final toutes seront jaunies. Les indices en poche, nous passons le checkeur au vert sans difficulté pour la bonus qui est adorable. Des PF sont distribués pour récompenser les belles caches et l' ensemble du travail. Merci agan64 pour cet excellent moment.\end{cacheText}

\cacheNumber{2354}\needspace{5\baselineskip}\cacheName{\href{http://coord.info/GC7YAR3}{Un petit tour dans le bois \Number{}6} — \href{http://coord.info/GC7YAR3\Number{}819231341}{2354}}\cacheData{{2018/12/07 agan 64, Traditional Cache (1.5/2)}}\begin{cacheText}Après avoir fait une promenade matinale dans Pau pour résoudre Pau c’est royal !,nous optons avec Dune33, par ce bel après-midi, pour un petit tour dans les bois. Le parcours est très agréable sous ce beau soleil. Les caches, bien travaillées, nous ont fait bien chercher mais au final toutes seront jaunies. Les indices en poche, nous passons le checkeur au vert sans difficulté pour la bonus qui est adorable. Des PF sont distribués pour récompenser les belles caches et l' ensemble du travail. Merci agan64 pour cet excellent moment.\end{cacheText}

\cacheNumber{2355}\needspace{5\baselineskip}\cacheName{\href{http://coord.info/GC7YARA}{Un petit tour dans le bois \Number{}7} — \href{http://coord.info/GC7YARA\Number{}819231506}{2355}}\cacheData{{2018/12/07 agan 64, Traditional Cache (1.5/2)}}\begin{cacheText}Après avoir fait une promenade matinale dans Pau pour résoudre Pau c’est royal !,nous optons avec Dune33, par ce bel après-midi, pour un petit tour dans les bois. Le parcours est très agréable sous ce beau soleil. Les caches, bien travaillées, nous ont fait bien chercher mais au final toutes seront jaunies. Les indices en poche, nous passons le checkeur au vert sans difficulté pour la bonus qui est adorable. Des PF sont distribués pour récompenser les belles caches et l' ensemble du travail. Merci agan64 pour cet excellent moment.\end{cacheText}

\cacheNumber{2356}\needspace{5\baselineskip}\cacheName{\href{http://coord.info/GC7YARG}{Un petit tour dans le bois \Number{}8} — \href{http://coord.info/GC7YARG\Number{}819231749}{2356}}\cacheData{{2018/12/07 agan 64, Traditional Cache (1.5/2)}}\begin{cacheText}Après avoir fait une promenade matinale dans Pau pour résoudre Pau c’est royal !,nous optons avec Dune33, par ce bel après-midi, pour un petit tour dans les bois. Le parcours est très agréable sous ce beau soleil. Les caches, bien travaillées, nous ont fait bien chercher mais au final toutes seront jaunies. Les indices en poche, nous passons le checkeur au vert sans difficulté pour la bonus qui est adorable. Des PF sont distribués pour récompenser les belles caches et l' ensemble du travail. Merci agan64 pour cet excellent moment.\end{cacheText}

\cacheNumber{2357}\needspace{5\baselineskip}\cacheName{\href{http://coord.info/GC7YARR}{Un petit tour dans le bois \Number{} 9} — \href{http://coord.info/GC7YARR\Number{}819231994}{2357}}\cacheData{{2018/12/07 agan 64, Traditional Cache (1.5/2)}}\begin{cacheText}Après avoir fait une promenade matinale dans Pau pour résoudre Pau c’est royal !,nous optons avec Dune33, par ce bel après-midi, pour un petit tour dans les bois. Le parcours est très agréable sous ce beau soleil. Les caches, bien travaillées, nous ont fait bien chercher mais au final toutes seront jaunies. Les indices en poche, nous passons le checkeur au vert sans difficulté pour la bonus qui est adorable. Des PF sont distribués pour récompenser les belles caches et l' ensemble du travail. Merci agan64 pour cet excellent moment.\end{cacheText}

\cacheNumber{2358}\needspace{5\baselineskip}\cacheName{\href{http://coord.info/GC7YAT9}{Un petit tour dans le bois \Number{} 10} — \href{http://coord.info/GC7YAT9\Number{}819232224}{2358}}\cacheData{{2018/12/07 agan 64, Traditional Cache (1.5/2)}}\begin{cacheText}Après avoir fait une promenade matinale dans Pau pour résoudre Pau c’est royal !,nous optons avec Dune33, par ce bel après-midi, pour un petit tour dans les bois. Le parcours est très agréable sous ce beau soleil. Les caches, bien travaillées, nous ont fait bien chercher mais au final toutes seront jaunies. Les indices en poche, nous passons le checkeur au vert sans difficulté pour la bonus qui est adorable. Des PF sont distribués pour récompenser les belles caches et l' ensemble du travail. Merci agan64 pour cet excellent moment.\end{cacheText}

\cacheNumber{2359}\needspace{5\baselineskip}\cacheName{\href{http://coord.info/GC7YATN}{Un petit tour dans les bois \Number{}11} — \href{http://coord.info/GC7YATN\Number{}819232465}{2359}}\cacheData{{2018/12/07 agan 64, Traditional Cache (1.5/2)}}\begin{cacheText}Après avoir fait une promenade matinale dans Pau pour résoudre Pau c’est royal !,nous optons avec Dune33, par ce bel après-midi, pour un petit tour dans les bois. Le parcours est très agréable sous ce beau soleil. Les caches, bien travaillées, nous ont fait bien chercher mais au final toutes seront jaunies. Les indices en poche, nous passons le checkeur au vert sans difficulté pour la bonus qui est adorable. Des PF sont distribués pour récompenser les belles caches et l' ensemble du travail. Merci agan64 pour cet excellent moment.\end{cacheText}

\cacheNumber{2360}\needspace{5\baselineskip}\cacheName{\href{http://coord.info/GC7YATT}{Un petit tour dans le bois \Number{} 12} — \href{http://coord.info/GC7YATT\Number{}819232685}{2360}}\cacheData{{2018/12/07 agan 64, Traditional Cache (1.5/2)}}\begin{cacheText}Après avoir fait une promenade matinale dans Pau pour résoudre Pau c’est royal !,nous optons avec Dune33, par ce bel après-midi, pour un petit tour dans les bois. Le parcours est très agréable sous ce beau soleil. Les caches, bien travaillées, nous ont fait bien chercher mais au final toutes seront jaunies. Les indices en poche, nous passons le checkeur au vert sans difficulté pour la bonus qui est adorable. Des PF sont distribués pour récompenser les belles caches et l' ensemble du travail. Merci agan64 pour cet excellent moment.\end{cacheText}

\cacheNumber{2361}\needspace{5\baselineskip}\cacheName{\href{http://coord.info/GC7YAV9}{BONUS de la série \Quoted{Un petit tour dans le bois}} — \href{http://coord.info/GC7YAV9\Number{}819233141}{2361}}\cacheData{{2018/12/07 agan 64, Unknown Cache (1.5/2)}}\begin{cacheText}Après avoir fait une promenade matinale dans Pau pour résoudre Pau c’est royal !,nous optons avec Dune33, par ce bel après-midi, pour un petit tour dans les bois. Le parcours est très agréable sous ce beau soleil. Les caches, bien travaillées, nous ont fait bien chercher mais au final toutes seront jaunies. Les indices en poche, nous passons le checkeur au vert sans difficulté pour la bonus qui est adorable. Des PF sont distribués pour récompenser les belles caches et l' ensemble du travail. Merci agan64 pour cet excellent moment.\end{cacheText}

\cacheNumber{2362}\needspace{5\baselineskip}\cacheName{\href{http://coord.info/GC7YRF7}{Un petit tour dans le bois \Number{}2} — \href{http://coord.info/GC7YRF7\Number{}819230609}{2362}}\cacheData{{2018/12/07 agan 64, Traditional Cache (1.5/2)}}\begin{cacheText}Après avoir fait une promenade matinale dans Pau pour résoudre Pau c’est royal !,nous optons avec Dune33, par ce bel après-midi, pour un petit tour dans les bois. Le parcours est très agréable sous ce beau soleil. Les caches, bien travaillées, nous ont fait bien chercher mais au final toutes seront jaunies. Les indices en poche, nous passons le checkeur au vert sans difficulté pour la bonus qui est adorable. Des PF sont distribués pour récompenser les belles caches et l' ensemble du travail. Merci agan64 pour cet excellent moment.\end{cacheText}

\cacheNumber{2363}\needspace{5\baselineskip}\cacheName{\href{http://coord.info/GC7YRFK}{Un petit tour dans le bois  \Number{}4} — \href{http://coord.info/GC7YRFK\Number{}819231016}{2363}}\cacheData{{2018/12/07 agan 64, Traditional Cache (1.5/2)}}\begin{cacheText}Après avoir fait une promenade matinale dans Pau pour résoudre Pau c’est royal !,nous optons avec Dune33, par ce bel après-midi, pour un petit tour dans les bois. Le parcours est très agréable sous ce beau soleil. Les caches, bien travaillées, nous ont fait bien chercher mais au final toutes seront jaunies. Les indices en poche, nous passons le checkeur au vert sans difficulté pour la bonus qui est adorable. Des PF sont distribués pour récompenser les belles caches et l' ensemble du travail. Merci agan64 pour cet excellent moment.\end{cacheText}

\cacheNumber{2364}\needspace{5\baselineskip}\cacheName{\href{http://coord.info/GC7YRFY}{Un petit tour dans le bois \Number{}5} — \href{http://coord.info/GC7YRFY\Number{}819231188}{2364}}\cacheData{{2018/12/07 agan 64, Traditional Cache (1.5/2)}}\begin{cacheText}Après avoir fait une promenade matinale dans Pau pour résoudre Pau c’est royal !,nous optons avec Dune33, par ce bel après-midi, pour un petit tour dans les bois. Le parcours est très agréable sous ce beau soleil. Les caches, bien travaillées, nous ont fait bien chercher mais au final toutes seront jaunies. Les indices en poche, nous passons le checkeur au vert sans difficulté pour la bonus qui est adorable. Des PF sont distribués pour récompenser les belles caches et l' ensemble du travail. Merci agan64 pour cet excellent moment.\end{cacheText}

\cacheNumber{2365}\needspace{5\baselineskip}\cacheName{\href{http://coord.info/GC7YRHD}{Un petit tour dans le bois \Number{} 13} — \href{http://coord.info/GC7YRHD\Number{}819232859}{2365}}\cacheData{{2018/12/07 agan 64, Traditional Cache (1.5/2)}}\begin{cacheText}Après avoir fait une promenade matinale dans Pau pour résoudre Pau c’est royal !,nous optons avec Dune33, par ce bel après-midi, pour un petit tour dans les bois. Le parcours est très agréable sous ce beau soleil. Les caches, bien travaillées, nous ont fait bien chercher mais au final toutes seront jaunies. Les indices en poche, nous passons le checkeur au vert sans difficulté pour la bonus qui est adorable. Des PF sont distribués pour récompenser les belles caches et l' ensemble du travail. Merci agan64 pour cet excellent moment.\end{cacheText}

\cacheNumber{2366}\needspace{5\baselineskip}\cacheName{\href{http://coord.info/GC813A7}{PAU, C’est royal !} — \href{http://coord.info/GC813A7\Number{}819216586}{2366}}\cacheData{{2018/12/07 Terraaventura, Multi-cache (1.5/2)}}\begin{cacheText}En compagnie de Dune33, nous découvrons cette très jolie ville de Pau. Les indices sont récoltés plus ou moins facilement!!!Les portes cochères nous ont donné du mal mais nous finissons par verdir!!!!Merci les sages pour cette cache.\end{cacheText}

\cacheNumber{2367}\needspace{5\baselineskip}\cacheName{\href{http://coord.info/GC817BT}{Elu roi des menteurs !} — \href{http://coord.info/GC817BT\Number{}819685599}{2367}}\cacheData{{2018/12/13 Terraaventura, Multi-cache (3/3)}}\begin{cacheText}Cache trouvée le 08/08/2018 avec l'application Terra.Nous avons passé un excellent moment avec tous les menteurs et avons bien rigolé.Un grand merci les sages.\end{cacheText}

\cacheNumber{2368}\needspace{5\baselineskip}\cacheName{\href{http://coord.info/GC817FH}{Un loup dans la bastide} — \href{http://coord.info/GC817FH\Number{}819685139}{2368}}\cacheData{{2018/12/13 Terraaventura, Multi-cache (2/1.5)}}\begin{cacheText}Cache trouvée le 08/08/18.

La ville est magnifique et le parcours vraiment très agréable. Merci les sages .\end{cacheText}

\cacheNumber{2369}\needspace{5\baselineskip}\cacheName{\href{http://coord.info/GC81B17}{Saint Vincent à toutes berzingues} — \href{http://coord.info/GC81B17\Number{}819760293}{2369}}\cacheData{{2018/12/14 Terraaventura, Multi-cache (2/1.5)}}\begin{cacheText}Trouvée le 15/07/18 avec l'application Terraaventura. Super promenade dans Saint Vincent de Paul. Que de belles découvertes!!! La boite est bien camouflée.... osez!!! Merci les Sages\end{cacheText}

